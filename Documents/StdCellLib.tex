%%  ************    LibreSilicon's StdCellLibrary   *******************
%%
%%  Organisation:   Chipforge
%%                  Germany / European Union
%%
%%  Profile:        Chipforge focus on fine System-on-Chip Cores in
%%                  Verilog HDL Code which are easy understandable and
%%                  adjustable. For further information see
%%                          www.chipforge.org
%%                  there are projects from small cores up to PCBs, too.
%%
%%  File:           StdCellLib/Documents/LaTeX/StdCellLib.tex
%%
%%  Purpose:        Top Level File for Standard Cell Library Documentation
%%
%%  ************    LaTeX with circdia.sty package      ***************
%%
%%  ///////////////////////////////////////////////////////////////////
%%
%%  Copyright (c) 2018 - 2022 by
%%                chipforge <stdcelllib@nospam.chipforge.org>
%%  All rights reserved.
%%
%%      This Standard Cell Library is licensed under the Libre Silicon
%%      public license; you can redistribute it and/or modify it under
%%      the terms of the Libre Silicon public license as published by
%%      the Libre Silicon alliance, either version 1 of the License, or
%%      (at your option) any later version.
%%
%%      This design is distributed in the hope that it will be useful,
%%      but WITHOUT ANY WARRANTY; without even the implied warranty of
%%      MERCHANTABILITY or FITNESS FOR A PARTICULAR PURPOSE.
%%      See the Libre Silicon Public License for more details.
%%
%%  ///////////////////////////////////////////////////////////////////
\documentclass[a4paper, twoside,
               openany,
               headsepline, footsepline]{scrbook}
\usepackage[utf8]{inputenc}
\usepackage[english]{babel}
\usepackage[style=authoryear]{biblatex}
\usepackage{amsmath}
\usepackage{amsfonts}
\usepackage{amssymb}
%\usepackage{gensymb}
\usepackage{booktabs}
\usepackage{caption} \captionsetup{labelformat=empty}
\usepackage{graphicx}
\usepackage[digital,srcmeas,semicon]{circdia}
% \usepackage[dvipsnames]{xcolor}
\usepackage[left=2cm,right=2cm,top=2cm,bottom=2cm]{geometry}
\usepackage{pdflscape}
\usepackage[inkscape=png]{svg}
\usepackage{nameref}
\usepackage[colorlinks=true,linkcolor=gray,citecolor=orange,filecolor=green,urlcolor=cyan]{hyperref}
\pagestyle{myheadings}
\markboth{LibreSilicon's Standard Cell Library - Reference Manual -\hfill}{\hfill - Reference Manual - LibreSilicon's Standard Cell Library}

%%  ************    LibreSilicon's StdCellLibrary   *******************
%%
%%  Organisation:   Chipforge
%%                  Germany / European Union
%%
%%  Profile:        Chipforge focus on fine System-on-Chip Cores in
%%                  Verilog HDL Code which are easy understandable and
%%                  adjustable. For further information see
%%                          www.chipforge.org
%%                  there are projects from small cores up to PCBs, too.
%%
%%  File:           StdCellLib/Documents/glossary.tex
%%
%%  Purpose:        Glossary List
%%
%%  ************    LaTeX with circdia.sty package      ***************
%%
%%  ///////////////////////////////////////////////////////////////////
%%
%%  Copyright (c) 2018 - 2022 by
%%                  chipforge <stdcelllib@nospam.chipforge.org>
%%  All rights reserved.
%%
%%      This Standard Cell Library is licensed under the Libre Silicon
%%      public license; you can redistribute it and/or modify it under
%%      the terms of the Libre Silicon public license as published by
%%      the Libre Silicon alliance, either version 1 of the License, or
%%      (at your option) any later version.
%%
%%      This design is distributed in the hope that it will be useful,
%%      but WITHOUT ANY WARRANTY; without even the implied warranty of
%%      MERCHANTABILITY or FITNESS FOR A PARTICULAR PURPOSE.
%%      See the Libre Silicon Public License for more details.
%%
%%  ///////////////////////////////////////////////////////////////////
\usepackage[acronym,toc,numberedsection=autolabel]{glossaries}
%\setglossarystyle{treegroup}
\setglossarystyle{tree}
\makeglossaries
%%  -------------------------------------------------------------------
%%                                  C
%%  -------------------------------------------------------------------

\newglossaryentry{CMOS}{name={CMOS}, description={Complementary metal-oxide-semiconductor}}

%%  -------------------------------------------------------------------
%%                                  M
%%  -------------------------------------------------------------------

\newglossaryentry{MOSFET}{name={MOSFET}, description={metal–oxide–semiconductor field-effect transistor}}

%%  -------------------------------------------------------------------
%%                                  N
%%  -------------------------------------------------------------------

\newglossaryentry{NMOS}{name={NMOS}, description={N-channel metal-oxide-semiconductor}}

%%  -------------------------------------------------------------------
%%                                  P
%%  -------------------------------------------------------------------

\newglossaryentry{PMOS}{name={PMOS}, description={P-channel metal-oxide-semiconductor}}




\addbibresource{readings.bib}

\title{LibreSilicon's Standard Cell Library}
\subtitle{- Reference Manual -}
\author{chipforge \texttt{<stdcelllib@nospam.chipforge.org>}}
\date{\today}

\begin{document}

\maketitle
%\setlength{\parindent}{0pt} % get rid of annoying indents

\begin{quote}
Copyright \textcopyright  2018 - 2022 CHIPFORGE. All rights reserved.

This process is licensed under the Libre Silicon public license; you can redistribute it and/or modify it under the terms of the Libre Silicon public license as published by the Libre Silicon alliance either version 2 of the License, or (at your option) any later version.

This design is distributed in the hope that it will be useful, but WITHOUT ANY WARRANTY; without even the implied warranty of MERCHANTABILITY or FITNESS FOR A PARTICULAR PURPOSE. See the Libre Silicon Public License for more details.

For further clarification consult the complete documentation of the process.
\end{quote}
\uppertitleback

This work was supported by the Stichting NLnet\footnotemark NGI Zero Program under MoU 2019-12-075.
\begin{figure}[htp]
    \begin{minipage}{.5\linewidth}
        \includesvg[width=6cm]{NLnet_banner}
    \end{minipage}
    \hfill
    \begin{minipage}{.25\linewidth}
        \includesvg[width=2cm]{NGIZero-white}
    \end{minipage}
\end{figure}
\footnotetext[1]{https://nlnet.nl}

\vfill
%%  ************    LibreSilicon's StdCellLibrary   *******************
%%
%%  Organisation:   Chipforge
%%                  Germany / European Union
%%
%%  Profile:        Chipforge focus on fine System-on-Chip Cores in
%%                  Verilog HDL Code which are easy understandable and
%%                  adjustable. For further information see
%%                          www.chipforge.org
%%                  there are projects from small cores up to PCBs, too.
%%
%%  File:           StdCellLib/Documents/table-revision-history.tex
%%
%%  Purpose:        Revision History File
%%
%%  ************    LaTeX with circdia.sty package      ***************
%%
%%  ///////////////////////////////////////////////////////////////////
%%
%%  Copyright (c) 2018 - 2022 by
%%                  chipforge <stdcelllib@nospam.chipforge.org>
%%  All rights reserved.
%%
%%      This Standard Cell Library is licensed under the Libre Silicon
%%      public license; you can redistribute it and/or modify it under
%%      the terms of the Libre Silicon public license as published by
%%      the Libre Silicon alliance, either version 1 of the License, or
%%      (at your option) any later version.
%%
%%      This design is distributed in the hope that it will be useful,
%%      but WITHOUT ANY WARRANTY; without even the implied warranty of
%%      MERCHANTABILITY or FITNESS FOR A PARTICULAR PURPOSE.
%%      See the Libre Silicon Public License for more details.
%%
%%  ///////////////////////////////////////////////////////////////////
\begin{table}[h]
    \centering
    \begin{tabular}{lcl}
        \toprule
        Revision & Date & Description \\
        \midrule
        Draft 0.0 & 2018-02-01 & START with empty document, ADD many cells \\
        Release Candidate 0.9 & 2022-07-04 & FIRST almost complete version \\
        \bottomrule
    \end{tabular}
\end{table}




\tableofcontents
%%  -------------------------------------------------------------------
%%                  PART I
%%  -------------------------------------------------------------------

%%  ************    LibreSilicon's StdCellLibrary   *******************
%%
%%  Organisation:   Chipforge
%%                  Germany / European Union
%%
%%  Profile:        Chipforge focus on fine System-on-Chip Cores in
%%                  Verilog HDL Code which are easy understandable and
%%                  adjustable. For further information see
%%                          www.chipforge.org
%%                  there are projects from small cores up to PCBs, too.
%%
%%  File:           StdCellLib/Documents/Book/part-informal.tex
%%
%%  Purpose:        Part Level File for Standard Cell Library Documentation
%%
%%  ************    LaTeX with circdia.sty package      ***************
%%
%%  ///////////////////////////////////////////////////////////////////
%%
%%  Copyright (c) 2018 - 2022 by
%%                  chipforge <stdcelllib@nospam.chipforge.org>
%%  All rights reserved.
%%
%%      This Standard Cell Library is licensed under the Libre Silicon
%%      public license; you can redistribute it and/or modify it under
%%      the terms of the Libre Silicon public license as published by
%%      the Libre Silicon alliance, either version 1 of the License, or
%%      (at your option) any later version.
%%
%%      This design is distributed in the hope that it will be useful,
%%      but WITHOUT ANY WARRANTY; without even the implied warranty of
%%      MERCHANTABILITY or FITNESS FOR A PARTICULAR PURPOSE.
%%      See the Libre Silicon Public License for more details.
%%
%%  ///////////////////////////////////////////////////////////////////
\part{Informal}

%%  -------------------------------------------------------------------
%%                  CHAPTER 1
%%  -------------------------------------------------------------------

\glsaddall
\printglossary[style=altlist,nonumberlist]
\noist

%%  -------------------------------------------------------------------
%%                  CHAPTER 2
%%  -------------------------------------------------------------------

\chapter{CMOS in a nutshell}

This basic initial project is dedicated to the CMOS Technology only and for this reason two types of metal-\-oxide-\-semiconductor field-\-effect transistors (MOSFET) are required.

Historicaly, the first chips with MOSFETs on the mass market were p-channel MOSFETs in enhancement-mode.

\begin{center}
    enhancement-mode PMOS transistor use-case
    \begin{figure}[h] %\caption{enhancement-mode PMOS transistor use-case}
        \centering
        \begin{circuitdiagram}{20}{20}
        \power{15}{18.5}{U}{}  % power above pmos
        \wire{15}{18}{15}{16}   % wire above pmos
        \trans{penh}{13}{14}{R}{}{} % pmos -> right
        \Voltarrow{14}{18}{10}{16}{u}{$-V_{GS}$}
        \wire{15}{12}{15}{8}   % wire below pmos
        \resis{15}{5}{V}{$R_D$}{}  % resistor on drain
        \wire{15}{1}{15}{2}   % wire below pmos
        \ground{15}{0.5}{D}  % ground below resistor
        \othersrc[\sigsym{-rec}]{o}{5.5}{15.5}{H}{}{signal}
        \pin{9.5}{15.5}{R}{}    % pin in
        \ground{2.5}{0.5}{D}  % ground below signal source
        \wire{2.5}{1}{2.5}{15.5}   % wire below signal
        \junct{15}{10}   % dot
        \wire{15}{10}{16}{10}   % wire before out
        \pin{17}{10}{R}{out}    % pin out
        \end{circuitdiagram}
    \end{figure}
\end{center}

The sectional view of a PMOS transistor in silicon is being shown below
%\begin{center}
%\end{center}

Historicaly later, faster chips with MOSFETs on the mass market were marked as n-channel MOSFETs in enhancement mode also.

\begin{center}
    enhancement-mode NMOS transistor use-case
    \begin{figure}[h] %\caption{enhancement-mode NMOS transistor use-case}
        \centering
        \begin{circuitdiagram}{20}{20}
        \power{15}{18.5}{U}{}  % power above resistor
        \wire{15}{17}{15}{18}   % wire above resistor
        \resis{15}{14}{V}{$R_D$}{}  % resistor on drain
        \wire{15}{11}{15}{8}   % wire between resistor and nmos
        \trans{nenh}{13}{6}{R}{}{} % nmos -> right
        \Voltarrow{10}{4}{14}{1}{d}{$+V_{GS}$}
        \wire{15}{1}{15}{4}   % wire below nmos
        \ground{15}{0.5}{D}  % ground below nmos
        \othersrc[\sigsym{rec}]{o}{5.5}{4.5}{H}{}{signal}
        \pin{9.5}{4.5}{R}{}    % pin in
        \ground{2.5}{0.5}{D}  % ground below signal source
        \wire{2.5}{1}{2.5}{4.5}   % wire below signal
        \junct{15}{10}   % dot
        \wire{15}{10}{16}{10}   % wire before out
        \pin{17}{10}{R}{out}    % pin out
        \end{circuitdiagram}
    \end{figure}
\end{center}

The sectional view of a NMOS transistor in silicon is being shown here also.
%\begin{center}
%\end{center}

Both technologies, the older NMOS as the newer PMOS, have the same disadvantage. Every time, the transistor is switched on, the current between Drain and Source of the transistor is limited by the Resistor on Drain only. Higher currents here meaning higher power consumption for the chip where the transistors are integrated also. If the transistors are switched off, no currents flows between Drain and Source anymore, the power consumption of the chip also goes low.

Et violà, the US-Patent with Number 3356858\footnotemark changed the world and combines both technologies to the new complementary metal-oxide-semiconductor (CMOS) technology. Instead of every transistor is working against a weak resistor, the transistor works against a complementary switched-off transistor. With the Eyes of our antecessor CMOS doubles the transistor count, but contemporary chips all are build in CMOS.
\footnotetext[1]{https://www.google.com/patents/US3356858}

\begin{center}
    complementary PMOS and NMOS transistor couple use-case
    \begin{figure}[h] %\caption{complementary PMOS and NMOS transistor couple use-case}
        \centering
        \begin{circuitdiagram}{20}{20}
        \power{15}{18.5}{U}{}  % power above pmos 
        \wire{15}{16}{15}{18}   % wire above pmos
        \trans{penh}{13}{14}{R}{}{} % pmos -> right
        \Voltarrow{14}{18}{10}{16}{u}{$-V_{GS}$}
        \wire{15}{8}{15}{12}   % wire between pmos and nmos
        \trans{nenh}{13}{6}{R}{}{} % nmos -> right
        \Voltarrow{10}{4}{14}{1}{d}{$+V_{GS}$}
        \wire{15}{1}{15}{4}   % wire below nmos
        \ground{15}{0.5}{D}  % ground below nmos
        \othersrc[\sigsym{rec}]{o}{5}{10}{H}{}{signal}
        \pin{9}{10}{R}{}    % pin in
        \wire{9.5}{10}{10}{10}   % wire before gates
        \wire{10}{15.5}{10}{4.5}   % wire between gates
        \junct{10}{10}   % dot
        \ground{2}{0.5}{D}  % ground below signal source
        \wire{2}{1}{2}{10}   % wire below signal
        \junct{15}{10}   % dot
        \wire{15}{10}{16}{10}   % wire before out
        \pin{17}{10}{R}{out}    % pin out
        \end{circuitdiagram}
    \end{figure}
\end{center}

The sectional view of a NMOS and PMOS transistors couple in silicon - building the CMOS technology - are being shown here also.

\clearpage


%%  -------------------------------------------------------------------
%%                  CHAPTER 3
%%  -------------------------------------------------------------------

%%  ************    LibreSilicon's StdCellLibrary   *******************
%%
%%  Organisation:   Chipforge
%%                  Germany / European Union
%%
%%  Profile:        Chipforge focus on fine System-on-Chip Cores in
%%                  Verilog HDL Code which are easy understandable and
%%                  adjustable. For further information see
%%                          www.chipforge.org
%%                  there are projects from small cores up to PCBs, too.
%%
%%  File:           StdCellLib/Documents/Book/chapter-design-considerations.tex
%%
%%  Purpose:        Chapter Level File for Standard Cell Library Documentation
%%
%%  ************    LaTeX with circdia.sty package      ***************
%%
%%  ///////////////////////////////////////////////////////////////////
%%
%%  Copyright (c) 2018 - 2022 by
%%                  chipforge <stdcelllib@nospam.chipforge.org>
%%  All rights reserved.
%%
%%      This Standard Cell Library is licensed under the Libre Silicon
%%      public license; you can redistribute it and/or modify it under
%%      the terms of the Libre Silicon public license as published by
%%      the Libre Silicon alliance, either version 1 of the License, or
%%      (at your option) any later version.
%%
%%      This design is distributed in the hope that it will be useful,
%%      but WITHOUT ANY WARRANTY; without even the implied warranty of
%%      MERCHANTABILITY or FITNESS FOR A PARTICULAR PURPOSE.
%%      See the Libre Silicon Public License for more details.
%%
\chapter{Design Considerations}

\section{Number of stacked transistors}
\section{Output level restoration}
\section{Port naming rules}
\section{Gate naming rules}

\clearpage


% !! add much, much more




%%  -------------------------------------------------------------------
%%                  PART II
%%  -------------------------------------------------------------------

%%  ************    LibreSilicon's StdCellLibrary   *******************
%%
%%  Organisation:   Chipforge
%%                  Germany / European Union
%%
%%  Profile:        Chipforge focus on fine System-on-Chip Cores in
%%                  Verilog HDL Code which are easy understandable and
%%                  adjustable. For further information see
%%                          www.chipforge.org
%%                  there are projects from small cores up to PCBs, too.
%%
%%  File:           StdCellLib/Documents/LaTeX/part-catalog.tex
%%
%%  Purpose:        Part Level File for Standard Cell Library Documentation
%%
%%  ************    LaTeX with circdia.sty package      ***************
%%
%%  ///////////////////////////////////////////////////////////////////
%%
%%  Copyright (c) 2018 - 2021 by chipforge <stdcelllib@nospam.chipforge.org>
%%  All rights reserved.
%%
%%      This Standard Cell Library is licensed under the Libre Silicon
%%      public license; you can redistribute it and/or modify it under
%%      the terms of the Libre Silicon public license as published by
%%      the Libre Silicon alliance, either version 1 of the License, or
%%      (at your option) any later version.
%%
%%      This design is distributed in the hope that it will be useful,
%%      but WITHOUT ANY WARRANTY; without even the implied warranty of
%%      MERCHANTABILITY or FITNESS FOR A PARTICULAR PURPOSE.
%%      See the Libre Silicon Public License for more details.
%%
%%  ///////////////////////////////////////////////////////////////////
\part{Cell Catalog}
\pagestyle{headings}

%%  -------------------------------------------------------------------
%%                  CHAPTER 1
%%  -------------------------------------------------------------------

\chapter{Combinatorial Cells}
\clearpage

%%  ------------    one phase   ---------------------------------------

%%  ************    LibreSilicon's StdCellLibrary   *******************
%%
%%  Organisation:   Chipforge
%%                  Germany / European Union
%%
%%  Profile:        Chipforge focus on fine System-on-Chip Cores in
%%                  Verilog HDL Code which are easy understandable and
%%                  adjustable. For further information see
%%                          www.chipforge.org
%%                  there are projects from small cores up to PCBs, too.
%%
%%  File:           StdCellLib/Documents/Book/section-INV_buffers.tex
%%
%%  Purpose:        Section Level File for Standard Cell Library Documentation
%%
%%  ************    LaTeX with circdia.sty package      ***************
%%
%%  ///////////////////////////////////////////////////////////////////
%%
%%  Copyright (c) 2018 - 2022 by
%%                  chipforge <stdcelllib@nospam.chipforge.org>
%%  All rights reserved.
%%
%%      This Standard Cell Library is licensed under the Libre Silicon
%%      public license; you can redistribute it and/or modify it under
%%      the terms of the Libre Silicon public license as published by
%%      the Libre Silicon alliance, either version 1 of the License, or
%%      (at your option) any later version.
%%
%%      This design is distributed in the hope that it will be useful,
%%      but WITHOUT ANY WARRANTY; without even the implied warranty of
%%      MERCHANTABILITY or FITNESS FOR A PARTICULAR PURPOSE.
%%      See the Libre Silicon Public License for more details.
%%
%%  ///////////////////////////////////////////////////////////////////
\section{Inverting Buffers}

%%  ************    LibreSilicon's StdCellLibrary   *******************
%%
%%  Organisation:   Chipforge
%%                  Germany / European Union
%%
%%  Profile:        Chipforge focus on fine System-on-Chip Cores in
%%                  Verilog HDL Code which are easy understandable and
%%                  adjustable. For further information see
%%                          www.chipforge.org
%%                  there are projects from small cores up to PCBs, too.
%%
%%  File:           StdCellLib/Documents/Datasheets/Circuitry/INV.tex
%%
%%  Purpose:        Circuit File for INV
%%
%%  ************    LaTeX with circdia.sty package      ***************
%%
%%  ///////////////////////////////////////////////////////////////////
%%
%%  Copyright (c) 2018 - 2022 by
%%                  chipforge <stdcelllib@nospam.chipforge.org>
%%  All rights reserved.
%%
%%      This Standard Cell Library is licensed under the Libre Silicon
%%      public license; you can redistribute it and/or modify it under
%%      the terms of the Libre Silicon public license as published by
%%      the Libre Silicon alliance, either version 1 of the License, or
%%      (at your option) any later version.
%%
%%      This design is distributed in the hope that it will be useful,
%%      but WITHOUT ANY WARRANTY; without even the implied warranty of
%%      MERCHANTABILITY or FITNESS FOR A PARTICULAR PURPOSE.
%%      See the Libre Silicon Public License for more details.
%%
%%  ///////////////////////////////////////////////////////////////////
\begin{circuitdiagram}[draft]{10}{6}

    \usgate
    % ----  1st column  ----
    \pin{1}{3}{L}{A}
    \gate{not}{5}{3}{R}{}{}

    % ----  result ----
    \pin{9}{3}{R}{Y}

\end{circuitdiagram}



%%  ************    LibreSilicon's StdCellLibrary   *******************
%%
%%  Organisation:   Chipforge
%%                  Germany / European Union
%%
%%  Profile:        Chipforge focus on fine System-on-Chip Cores in
%%                  Verilog HDL Code which are easy understandable and
%%                  adjustable. For further information see
%%                          www.chipforge.org
%%                  there are projects from small cores up to PCBs, too.
%%
%%  File:           StdCellLib/Documents/section-NAND_gates.tex
%%
%%  Purpose:        Section Level File for Standard Cell Library Documentation
%%
%%  ************    LaTeX with circdia.sty package      ***************
%%
%%  ///////////////////////////////////////////////////////////////////
%%
%%  Copyright (c) 2018 - 2022 by
%%                  chipforge <stdcelllib@nospam.chipforge.org>
%%  All rights reserved.
%%
%%      This Standard Cell Library is licensed under the Libre Silicon
%%      public license; you can redistribute it and/or modify it under
%%      the terms of the Libre Silicon public license as published by
%%      the Libre Silicon alliance, either version 1 of the License, or
%%      (at your option) any later version.
%%
%%      This design is distributed in the hope that it will be useful,
%%      but WITHOUT ANY WARRANTY; without even the implied warranty of
%%      MERCHANTABILITY or FITNESS FOR A PARTICULAR PURPOSE.
%%      See the Libre Silicon Public License for more details.
%%
%%  ///////////////////////////////////////////////////////////////////
\section{NAND and AND Gates}

%%  ************    LibreSilicon's StdCellLibrary   *******************
%%
%%  Organisation:   Chipforge
%%                  Germany / European Union
%%
%%  Profile:        Chipforge focus on fine System-on-Chip Cores in
%%                  Verilog HDL Code which are easy understandable and
%%                  adjustable. For further information see
%%                          www.chipforge.org
%%                  there are projects from small cores up to PCBs, too.
%%
%%  File:           StdCellLib/Documents/Circuits/NAND2.tex
%%
%%  Purpose:        Circuit File for NAND2
%%
%%  ************    LaTeX with circdia.sty package      ***************
%%
%%  ///////////////////////////////////////////////////////////////////
%%
%%  Copyright (c) 2019 - 2021 by
%%                chipforge <stdcelllib@nospam.chipforge.org>
%%  All rights reserved.
%%
%%      This Standard Cell Library is licensed under the Libre Silicon
%%      public license; you can redistribute it and/or modify it under
%%      the terms of the Libre Silicon public license as published by
%%      the Libre Silicon alliance, either version 1 of the License, or
%%      (at your option) any later version.
%%
%%      This design is distributed in the hope that it will be useful,
%%      but WITHOUT ANY WARRANTY; without even the implied warranty of
%%      MERCHANTABILITY or FITNESS FOR A PARTICULAR PURPOSE.
%%      See the Libre Silicon Public License for more details.
%%
%%  ///////////////////////////////////////////////////////////////////
\begin{circuitdiagram}{11}{6}

    \usgate
    \gate[\inputs{2}]{nand}{5}{3}{R}{}{} % NAND
    \pin{1}{5}{L}{A}    % pin A
    \pin{1}{1}{L}{A1}   % pin A1
    \pin{10}{3}{R}{Y}   % pin Y

\end{circuitdiagram}
 %%  ************    LibreSilicon's StdCellLibrary   *******************
%%
%%  Organisation:   Chipforge
%%                  Germany / European Union
%%
%%  Profile:        Chipforge focus on fine System-on-Chip Cores in
%%                  Verilog HDL Code which are easy understandable and
%%                  adjustable. For further information see
%%                          www.chipforge.org
%%                  there are projects from small cores up to PCBs, too.
%%
%%  File:           StdCellLib/Documents/Datasheets/Circuitry/AND2.tex
%%
%%  Purpose:        Circuit File for AND2
%%
%%  ************    LaTeX with circdia.sty package      ***************
%%
%%  ///////////////////////////////////////////////////////////////////
%%
%%  Copyright (c) 2018 - 2022 by
%%                  chipforge <stdcelllib@nospam.chipforge.org>
%%  All rights reserved.
%%
%%      This Standard Cell Library is licensed under the Libre Silicon
%%      public license; you can redistribute it and/or modify it under
%%      the terms of the Libre Silicon public license as published by
%%      the Libre Silicon alliance, either version 1 of the License, or
%%      (at your option) any later version.
%%
%%      This design is distributed in the hope that it will be useful,
%%      but WITHOUT ANY WARRANTY; without even the implied warranty of
%%      MERCHANTABILITY or FITNESS FOR A PARTICULAR PURPOSE.
%%      See the Libre Silicon Public License for more details.
%%
%%  ///////////////////////////////////////////////////////////////////
\begin{circuitdiagram}[draft]{17}{6}

    \usgate
    % ----  1st column  ----
    \pin{1}{1}{L}{A}
    \pin{1}{5}{L}{A1}
    \gate[\inputs{2}]{nand}{5}{3}{R}{}{}

    % ----  2nd column  ----
    \gate{not}{12}{3}{R}{}{}

    % ----  result ----
    \pin{16}{3}{R}{Z}

\end{circuitdiagram}

%%  ************    LibreSilicon's StdCellLibrary   *******************
%%
%%  Organisation:   Chipforge
%%                  Germany / European Union
%%
%%  Profile:        Chipforge focus on fine System-on-Chip Cores in
%%                  Verilog HDL Code which are easy understandable and
%%                  adjustable. For further information see
%%                          www.chipforge.org
%%                  there are projects from small cores up to PCBs, too.
%%
%%  File:           StdCellLib/Documents/Circuits/NAND3.tex
%%
%%  Purpose:        Circuit File for NAND3
%%
%%  ************    LaTeX with circdia.sty package      ***************
%%
%%  ///////////////////////////////////////////////////////////////////
%%
%%  Copyright (c) 2019 by chipforge <stdcelllib@nospam.chipforge.org>
%%  All rights reserved.
%%
%%      This Standard Cell Library is licensed under the Libre Silicon
%%      public license; you can redistribute it and/or modify it under
%%      the terms of the Libre Silicon public license as published by
%%      the Libre Silicon alliance, either version 1 of the License, or
%%      (at your option) any later version.
%%
%%      This design is distributed in the hope that it will be useful,
%%      but WITHOUT ANY WARRANTY; without even the implied warranty of
%%      MERCHANTABILITY or FITNESS FOR A PARTICULAR PURPOSE.
%%      See the Libre Silicon Public License for more details.
%%
%%  ///////////////////////////////////////////////////////////////////
\begin{circuitdiagram}{11}{6}

    \usgate
    \gate[\inputs{3}]{nand}{5}{3}{R}{}{}  % NAND
    \pin{1}{1}{L}{A}    % pin A
    \pin{1}{3}{L}{A1}   % pin A1
    \pin{1}{5}{L}{A2}   % pin A2
    \pin{10}{3}{R}{Y}   % pin Y

\end{circuitdiagram}
 %%  ************    LibreSilicon's StdCellLibrary   *******************
%%
%%  Organisation:   Chipforge
%%                  Germany / European Union
%%
%%  Profile:        Chipforge focus on fine System-on-Chip Cores in
%%                  Verilog HDL Code which are easy understandable and
%%                  adjustable. For further information see
%%                          www.chipforge.org
%%                  there are projects from small cores up to PCBs, too.
%%
%%  File:           StdCellLib/Documents/Circuits/AND3.tex
%%
%%  Purpose:        Circuit File for AND3
%%
%%  ************    LaTeX with circdia.sty package      ***************
%%
%%  ///////////////////////////////////////////////////////////////////
%%
%%  Copyright (c) 2019 by chipforge <stdcelllib@nospam.chipforge.org>
%%  All rights reserved.
%%
%%      This Standard Cell Library is licensed under the Libre Silicon
%%      public license; you can redistribute it and/or modify it under
%%      the terms of the Libre Silicon public license as published by
%%      the Libre Silicon alliance, either version 1 of the License, or
%%      (at your option) any later version.
%%
%%      This design is distributed in the hope that it will be useful,
%%      but WITHOUT ANY WARRANTY; without even the implied warranty of
%%      MERCHANTABILITY or FITNESS FOR A PARTICULAR PURPOSE.
%%      See the Libre Silicon Public License for more details.
%%
%%  ///////////////////////////////////////////////////////////////////
\begin{circuitdiagram}{17}{6}

    \usgate
    \gate[\inputs{3}]{nand}{5}{3}{R}{}{}  % NAND
    \gate{not}{12}{3}{R}{}{}    % NOT
    \pin{1}{1}{L}{A}    % pin A
    \pin{1}{3}{L}{A1}   % pin A1
    \pin{1}{5}{L}{A2}   % pin A2
    \pin{16}{3}{R}{Z}   % pin Z

\end{circuitdiagram}

%%  ************    LibreSilicon's StdCellLibrary   *******************
%%
%%  Organisation:   Chipforge
%%                  Germany / European Union
%%
%%  Profile:        Chipforge focus on fine System-on-Chip Cores in
%%                  Verilog HDL Code which are easy understandable and
%%                  adjustable. For further information see
%%                          www.chipforge.org
%%                  there are projects from small cores up to PCBs, too.
%%
%%  File:           StdCellLib/Documents/Datasheets/Circuitry/NAND4.tex
%%
%%  Purpose:        Circuit File for NAND4
%%
%%  ************    LaTeX with circdia.sty package      ***************
%%
%%  ///////////////////////////////////////////////////////////////////
%%
%%  Copyright (c) 2018 - 2022 by
%%                  chipforge <stdcelllib@nospam.chipforge.org>
%%  All rights reserved.
%%
%%      This Standard Cell Library is licensed under the Libre Silicon
%%      public license; you can redistribute it and/or modify it under
%%      the terms of the Libre Silicon public license as published by
%%      the Libre Silicon alliance, either version 1 of the License, or
%%      (at your option) any later version.
%%
%%      This design is distributed in the hope that it will be useful,
%%      but WITHOUT ANY WARRANTY; without even the implied warranty of
%%      MERCHANTABILITY or FITNESS FOR A PARTICULAR PURPOSE.
%%      See the Libre Silicon Public License for more details.
%%
%%  ///////////////////////////////////////////////////////////////////
\begin{circuitdiagram}[draft]{11}{8}

    \usgate
    % ----  1st column  ----
    \pin{1}{1}{L}{A}
    \pin{1}{3}{L}{A1}
    \pin{1}{5}{L}{A2}
    \pin{1}{7}{L}{A3}
    \gate[\inputs{4}]{nand}{5}{4}{R}{}{} % NAND

    % ----  result ----
    \pin{10}{4}{R}{Y}

\end{circuitdiagram}
 %%  ************    LibreSilicon's StdCellLibrary   *******************
%%
%%  Organisation:   Chipforge
%%                  Germany / European Union
%%
%%  Profile:        Chipforge focus on fine System-on-Chip Cores in
%%                  Verilog HDL Code which are easy understandable and
%%                  adjustable. For further information see
%%                          www.chipforge.org
%%                  there are projects from small cores up to PCBs, too.
%%
%%  File:           StdCellLib/Documents/Datasheets/Circuitry/AND4.tex
%%
%%  Purpose:        Circuit File for AND4
%%
%%  ************    LaTeX with circdia.sty package      ***************
%%
%%  ///////////////////////////////////////////////////////////////////
%%
%%  Copyright (c) 2018 - 2022 by
%%                  chipforge <stdcelllib@nospam.chipforge.org>
%%  All rights reserved.
%%
%%      This Standard Cell Library is licensed under the Libre Silicon
%%      public license; you can redistribute it and/or modify it under
%%      the terms of the Libre Silicon public license as published by
%%      the Libre Silicon alliance, either version 1 of the License, or
%%      (at your option) any later version.
%%
%%      This design is distributed in the hope that it will be useful,
%%      but WITHOUT ANY WARRANTY; without even the implied warranty of
%%      MERCHANTABILITY or FITNESS FOR A PARTICULAR PURPOSE.
%%      See the Libre Silicon Public License for more details.
%%
%%  ///////////////////////////////////////////////////////////////////
\begin{circuitdiagram}[draft]{17}{8}

    \usgate
    % ----  1st column  ----
    \pin{1}{1}{L}{A}
    \pin{1}{3}{L}{A1}
    \pin{1}{5}{L}{A2}
    \pin{1}{7}{L}{A3}
    \gate[\inputs{4}]{nand}{5}{4}{R}{}{}

    % ----  2nd column  ----
    \gate{not}{12}{4}{R}{}{}

    % ----  result ----
    \pin{16}{4}{R}{Z}

\end{circuitdiagram}

%%  ************    LibreSilicon's StdCellLibrary   *******************
%%
%%  Organisation:   Chipforge
%%                  Germany / European Union
%%
%%  Profile:        Chipforge focus on fine System-on-Chip Cores in
%%                  Verilog HDL Code which are easy understandable and
%%                  adjustable. For further information see
%%                          www.chipforge.org
%%                  there are projects from small cores up to PCBs, too.
%%
%%  File:           StdCellLib/Documents/Datasheets/Circuitry/NAND5.tex
%%
%%  Purpose:        Circuit File for NAND5
%%
%%  ************    LaTeX with circdia.sty package      ***************
%%
%%  ///////////////////////////////////////////////////////////////////
%%
%%  Copyright (c) 2018 - 2022 by
%%                  chipforge <stdcelllib@nospam.chipforge.org>
%%  All rights reserved.
%%
%%      This Standard Cell Library is licensed under the Libre Silicon
%%      public license; you can redistribute it and/or modify it under
%%      the terms of the Libre Silicon public license as published by
%%      the Libre Silicon alliance, either version 1 of the License, or
%%      (at your option) any later version.
%%
%%      This design is distributed in the hope that it will be useful,
%%      but WITHOUT ANY WARRANTY; without even the implied warranty of
%%      MERCHANTABILITY or FITNESS FOR A PARTICULAR PURPOSE.
%%      See the Libre Silicon Public License for more details.
%%
%%  ///////////////////////////////////////////////////////////////////
\begin{circuitdiagram}[draft]{11}{10}

    \usgate
    % ----  1st column  ----
    \pin{1}{1}{L}{A}
    \pin{1}{3}{L}{A1}
    \pin{1}{5}{L}{A2}
    \pin{1}{7}{L}{A3}
    \pin{1}{9}{L}{A4}
    \gate[\inputs{5}]{nand}{5}{5}{R}{}{}

    % ----  result ----
    \pin{10}{5}{R}{Y}

\end{circuitdiagram}
 %%  ************    LibreSilicon's StdCellLibrary   *******************
%%
%%  Organisation:   Chipforge
%%                  Germany / European Union
%%
%%  Profile:        Chipforge focus on fine System-on-Chip Cores in
%%                  Verilog HDL Code which are easy understandable and
%%                  adjustable. For further information see
%%                          www.chipforge.org
%%                  there are projects from small cores up to PCBs, too.
%%
%%  File:           StdCellLib/Documents/Datasheets/Circuitry/AND4.tex
%%
%%  Purpose:        Circuit File for AND5
%%
%%  ************    LaTeX with circdia.sty package      ***************
%%
%%  ///////////////////////////////////////////////////////////////////
%%
%%  Copyright (c) 2018 - 2022 by
%%                  chipforge <stdcelllib@nospam.chipforge.org>
%%  All rights reserved.
%%
%%      This Standard Cell Library is licensed under the Libre Silicon
%%      public license; you can redistribute it and/or modify it under
%%      the terms of the Libre Silicon public license as published by
%%      the Libre Silicon alliance, either version 1 of the License, or
%%      (at your option) any later version.
%%
%%      This design is distributed in the hope that it will be useful,
%%      but WITHOUT ANY WARRANTY; without even the implied warranty of
%%      MERCHANTABILITY or FITNESS FOR A PARTICULAR PURPOSE.
%%      See the Libre Silicon Public License for more details.
%%
%%  ///////////////////////////////////////////////////////////////////
\begin{circuitdiagram}[draft]{17}{10}

    \usgate
    % ----  1st column  ----
    \pin{1}{1}{L}{A}
    \pin{1}{3}{L}{A1}
    \pin{1}{5}{L}{A2}
    \pin{1}{7}{L}{A3}
    \pin{1}{9}{L}{A4}
    \gate[\inputs{5}]{nand}{5}{5}{R}{}{}

    % ----  2nd column  ----
    \gate{not}{12}{5}{R}{}{}

    % ----  result ----
    \pin{16}{5}{R}{Z}

\end{circuitdiagram}


%%  ************    LibreSilicon's StdCellLibrary   *******************
%%
%%  Organisation:   Chipforge
%%                  Germany / European Union
%%
%%  Profile:        Chipforge focus on fine System-on-Chip Cores in
%%                  Verilog HDL Code which are easy understandable and
%%                  adjustable. For further information see
%%                          www.chipforge.org
%%                  there are projects from small cores up to PCBs, too.
%%
%%  File:           StdCellLib/Documents/LaTeX/section-NOR_gates.tex
%%
%%  Purpose:        Section Level File for Standard Cell Library Documentation
%%
%%  ************    LaTeX with circdia.sty package      ***************
%%
%%  ///////////////////////////////////////////////////////////////////
%%
%%  Copyright (c) 2018 - 2021 by
%%                  chipforge <stdcelllib@nospam.chipforge.org>
%%  All rights reserved.
%%
%%      This Standard Cell Library is licensed under the Libre Silicon
%%      public license; you can redistribute it and/or modify it under
%%      the terms of the Libre Silicon public license as published by
%%      the Libre Silicon alliance, either version 1 of the License, or
%%      (at your option) any later version.
%%
%%      This design is distributed in the hope that it will be useful,
%%      but WITHOUT ANY WARRANTY; without even the implied warranty of
%%      MERCHANTABILITY or FITNESS FOR A PARTICULAR PURPOSE.
%%      See the Libre Silicon Public License for more details.
%%
\section{NOR and OR Gates}

\include{NOR2_datasheet} \include{OR2_datasheet}
\include{NOR3_datasheet} \include{OR3_datasheet}
\include{NOR4_datasheet} \include{OR4_datasheet}
\include{NOR5_datasheet} \include{OR5_datasheet}

\section{(Inverting) Multiplexer}

%%  ************    LibreSilicon's StdCellLibrary   *******************
%%
%%  Organisation:   Chipforge
%%                  Germany / European Union
%%
%%  Profile:        Chipforge focus on fine System-on-Chip Cores in
%%                  Verilog HDL Code which are easy understandable and
%%                  adjustable. For further information see
%%                          www.chipforge.org
%%                  there are projects from small cores up to PCBs, too.
%%
%%  File:           StdCellLib/Documents/LaTeX/manpage_MUXI21.tex
%%
%%  Purpose:        Manual Page File for MUXI21
%%
%%  ************    LaTeX with circdia.sty package      ***************
%%
%%  ///////////////////////////////////////////////////////////////////
%%
%%  Copyright (c) 2018 by chipforge <hsank@nospam.chipforge.org>
%%  All rights reserved.
%%
%%      This Standard Cell Library is licensed under the Libre Silicon
%%      public license; you can redistribute it and/or modify it under
%%      the terms of the Libre Silicon public license as published by
%%      the Libre Silicon alliance, either version 1 of the License, or
%%      (at your option) any later version.
%%
%%      This design is distributed in the hope that it will be useful,
%%      but WITHOUT ANY WARRANTY; without even the implied warranty of
%%      MERCHANTABILITY or FITNESS FOR A PARTICULAR PURPOSE.
%%      See the Libre Silicon Public License for more details.
%%
%%  ///////////////////////////////////////////////////////////////////
\label{MUXI21}
\paragraph{Cell}
\begin{quote}
    \textbf{MUXI21} - an inverting 2-to-1 Multiplexor
\end{quote}

\paragraph{Synopsys}
\begin{quote}
    MUXI21(Y, A1, A, S)
\end{quote}

\paragraph{Description}
\input{MUXI21_circuit.tex}
\input{MUXI21_schematic.tex}

\paragraph{Truth Table}
\input{MUXI21_truthtable.tex}

\paragraph{Usage}

\paragraph{Fan-in / Fan-out}

\paragraph{Layout}

\paragraph{Files}
%\input{files_MUXI21.tex}

%\input{MUXIE21_manpage.tex}
%\input{MUXIEN21_manpage.tex}
%\input{MUXI41_manpage.tex}
%\input{MUXIE41_manpage.tex}
%\input{MUXIEN41_manpage.tex}
%\input{MUXI81_manpage.tex}
%\input{MUXIE81_manpage.tex}
%\input{MUXIEN81_manpage.tex}


%%  ------------    two phases  ---------------------------------------

%%  ************    LibreSilicon's StdCellLibrary   *******************
%%
%%  Organisation:   Chipforge
%%                  Germany / European Union
%%
%%  Profile:        Chipforge focus on fine System-on-Chip Cores in
%%                  Verilog HDL Code which are easy understandable and
%%                  adjustable. For further information see
%%                          www.chipforge.org
%%                  there are projects from small cores up to PCBs, too.
%%
%%  File:           StdCellLib/Documents/Book/section-AOI_complex.tex
%%
%%  Purpose:        Section Level File for Standard Cell Library Documentation
%%
%%  ************    LaTeX with circdia.sty package      ***************
%%
%%  ///////////////////////////////////////////////////////////////////
%%
%%  Copyright (c) 2018 - 2022 by
%%                  chipforge <stdcelllib@nospam.chipforge.org>
%%  All rights reserved.
%%
%%      This Standard Cell Library is licensed under the Libre Silicon
%%      public license; you can redistribute it and/or modify it under
%%      the terms of the Libre Silicon public license as published by
%%      the Libre Silicon alliance, either version 1 of the License, or
%%      (at your option) any later version.
%%
%%      This design is distributed in the hope that it will be useful,
%%      but WITHOUT ANY WARRANTY; without even the implied warranty of
%%      MERCHANTABILITY or FITNESS FOR A PARTICULAR PURPOSE.
%%      See the Libre Silicon Public License for more details.
%%
%%  ///////////////////////////////////////////////////////////////////
\section{AND-OR(-Invert) Complex Gates}

%%  ************    LibreSilicon's StdCellLibrary   *******************
%%
%%  Organisation:   Chipforge
%%                  Germany / European Union
%%
%%  Profile:        Chipforge focus on fine System-on-Chip Cores in
%%                  Verilog HDL Code which are easy understandable and
%%                  adjustable. For further information see
%%                          www.chipforge.org
%%                  there are projects from small cores up to PCBs, too.
%%
%%  File:           StdCellLib/Documents/Circuits/AOI21.tex
%%
%%  Purpose:        Circuit File for AOI21
%%
%%  ************    LaTeX with circdia.sty package      ***************
%%
%%  ///////////////////////////////////////////////////////////////////
%%
%%  Copyright (c) 2019 - 2021 by
%%                chipforge <stdcelllib@nospam.chipforge.org>
%%  All rights reserved.
%%
%%      This Standard Cell Library is licensed under the Libre Silicon
%%      public license; you can redistribute it and/or modify it under
%%      the terms of the Libre Silicon public license as published by
%%      the Libre Silicon alliance, either version 1 of the License, or
%%      (at your option) any later version.
%%
%%      This design is distributed in the hope that it will be useful,
%%      but WITHOUT ANY WARRANTY; without even the implied warranty of
%%      MERCHANTABILITY or FITNESS FOR A PARTICULAR PURPOSE.
%%      See the Libre Silicon Public License for more details.
%%
%%  ///////////////////////////////////////////////////////////////////
\begin{circuitdiagram}{18}{8}

    \usgate
    \gate[\inputs{2}]{and}{5}{5}{R}{}{}  % AND
    \gate[\inputs{2}]{nor}{12}{3}{R}{}{} % NOR
    \pin{1}{7}{L}{A}    % pin A
    \pin{1}{3}{L}{A1}   % pin A1
    \pin{1}{1}{L}{B}    % pin B
    \wire{2}{1}{9}{1}   % wire from pin A
    \pin{17}{3}{R}{Y}   % pin Y

\end{circuitdiagram}
 %%  ************    LibreSilicon's StdCellLibrary   *******************
%%
%%  Organisation:   Chipforge
%%                  Germany / European Union
%%
%%  Profile:        Chipforge focus on fine System-on-Chip Cores in
%%                  Verilog HDL Code which are easy understandable and
%%                  adjustable. For further information see
%%                          www.chipforge.org
%%                  there are projects from small cores up to PCBs, too.
%%
%%  File:           StdCellLib/Documents/Circuits/AO21.tex
%%
%%  Purpose:        Circuit File for AO21
%%
%%  ************    LaTeX with circdia.sty package      ***************
%%
%%  ///////////////////////////////////////////////////////////////////
%%
%%  Copyright (c) 2019 by chipforge <stdcelllib@nospam.chipforge.org>
%%  All rights reserved.
%%
%%      This Standard Cell Library is licensed under the Libre Silicon
%%      public license; you can redistribute it and/or modify it under
%%      the terms of the Libre Silicon public license as published by
%%      the Libre Silicon alliance, either version 1 of the License, or
%%      (at your option) any later version.
%%
%%      This design is distributed in the hope that it will be useful,
%%      but WITHOUT ANY WARRANTY; without even the implied warranty of
%%      MERCHANTABILITY or FITNESS FOR A PARTICULAR PURPOSE.
%%      See the Libre Silicon Public License for more details.
%%
%%  ///////////////////////////////////////////////////////////////////
\begin{circuitdiagram}{24}{8}

    \usgate
    \gate[\inputs{2}]{and}{5}{5}{R}{}{}    % AND
    \gate[\inputs{2}]{nor}{12}{3}{R}{}{}   % NOR
    \gate{not}{19}{3}{R}{}{}    % NOT
    \pin{1}{1}{L}{A}    % pin A
    \pin{1}{3}{L}{B}    % pin B
    \pin{1}{7}{L}{B1}   % pin B1
    \wire{2}{1}{9}{1}   % wire from pin A
    \pin{23}{3}{R}{Z}   % pin Z

\end{circuitdiagram}

%%  ************    LibreSilicon's StdCellLibrary   *******************
%%
%%  Organisation:   Chipforge
%%                  Germany / European Union
%%
%%  Profile:        Chipforge focus on fine System-on-Chip Cores in
%%                  Verilog HDL Code which are easy understandable and
%%                  adjustable. For further information see
%%                          www.chipforge.org
%%                  there are projects from small cores up to PCBs, too.
%%
%%  File:           StdCellLib/Documents/Datasheets/Circuitry/AOI22.tex
%%
%%  Purpose:        Circuit File for AOI22
%%
%%  ************    LaTeX with circdia.sty package      ***************
%%
%%  ///////////////////////////////////////////////////////////////////
%%
%%  Copyright (c) 2018 - 2022 by
%%                  chipforge <stdcelllib@nospam.chipforge.org>
%%  All rights reserved.
%%
%%      This Standard Cell Library is licensed under the Libre Silicon
%%      public license; you can redistribute it and/or modify it under
%%      the terms of the Libre Silicon public license as published by
%%      the Libre Silicon alliance, either version 1 of the License, or
%%      (at your option) any later version.
%%
%%      This design is distributed in the hope that it will be useful,
%%      but WITHOUT ANY WARRANTY; without even the implied warranty of
%%      MERCHANTABILITY or FITNESS FOR A PARTICULAR PURPOSE.
%%      See the Libre Silicon Public License for more details.
%%
%%  ///////////////////////////////////////////////////////////////////
\begin{circuitdiagram}[draft]{18}{10}

    \usgate
    % ----  1st column  ----
    \pin{1}{1}{L}{A}
    \pin{1}{5}{L}{A1}
    \gate[\inputs{2}]{and}{5}{3}{R}{}{}

    % ----  2nd column  ----
    \pin{8}{7}{L}{B}
    \pin{8}{9}{L}{B1}
    \wire{9}{3}{9}{5}
    \gate[\inputs{3}]{nor}{12}{7}{R}{}{}

    % ----  result ----
    \pin{17}{7}{R}{Y}

\end{circuitdiagram}
 %%  ************    LibreSilicon's StdCellLibrary   *******************
%%
%%  Organisation:   Chipforge
%%                  Germany / European Union
%%
%%  Profile:        Chipforge focus on fine System-on-Chip Cores in
%%                  Verilog HDL Code which are easy understandable and
%%                  adjustable. For further information see
%%                          www.chipforge.org
%%                  there are projects from small cores up to PCBs, too.
%%
%%  File:           StdCellLib/Documents/Datasheets/Circuitry/AO22.tex
%%
%%  Purpose:        Circuit File for AO22
%%
%%  ************    LaTeX with circdia.sty package      ***************
%%
%%  ///////////////////////////////////////////////////////////////////
%%
%%  Copyright (c) 2018 - 2022 by
%%                  chipforge <stdcelllib@nospam.chipforge.org>
%%  All rights reserved.
%%
%%      This Standard Cell Library is licensed under the Libre Silicon
%%      public license; you can redistribute it and/or modify it under
%%      the terms of the Libre Silicon public license as published by
%%      the Libre Silicon alliance, either version 1 of the License, or
%%      (at your option) any later version.
%%
%%      This design is distributed in the hope that it will be useful,
%%      but WITHOUT ANY WARRANTY; without even the implied warranty of
%%      MERCHANTABILITY or FITNESS FOR A PARTICULAR PURPOSE.
%%      See the Libre Silicon Public License for more details.
%%
%%  ///////////////////////////////////////////////////////////////////
\begin{circuitdiagram}[draft]{24}{10}

    \usgate
    % ----  1st column  ----
    \pin{1}{1}{L}{A}
    \pin{1}{5}{L}{A1}
    \gate[\inputs{2}]{and}{5}{3}{R}{}{}

    % ----  2nd column  ----
    \pin{8}{7}{L}{B}
    \pin{8}{9}{L}{B1}
    \wire{9}{3}{9}{5}
    \gate[\inputs{3}]{nor}{12}{7}{R}{}{}

    % ----  3rd column  ----
    \gate{not}{19}{7}{R}{}{}

    % ----  result ----
    \pin{23}{7}{R}{Z}

\end{circuitdiagram}

%%  ************    LibreSilicon's StdCellLibrary   *******************
%%
%%  Organisation:   Chipforge
%%                  Germany / European Union
%%
%%  Profile:        Chipforge focus on fine System-on-Chip Cores in
%%                  Verilog HDL Code which are easy understandable and
%%                  adjustable. For further information see
%%                          www.chipforge.org
%%                  there are projects from small cores up to PCBs, too.
%%
%%  File:           StdCellLib/Documents/Circuits/AOI23.tex
%%
%%  Purpose:        Circuit File for AOI23
%%
%%  ************    LaTeX with circdia.sty package      ***************
%%
%%  ///////////////////////////////////////////////////////////////////
%%
%%  Copyright (c) 2019 - 2021 by
%%                chipforge <stdcelllib@nospam.chipforge.org>
%%  All rights reserved.
%%
%%      This Standard Cell Library is licensed under the Libre Silicon
%%      public license; you can redistribute it and/or modify it under
%%      the terms of the Libre Silicon public license as published by
%%      the Libre Silicon alliance, either version 1 of the License, or
%%      (at your option) any later version.
%%
%%      This design is distributed in the hope that it will be useful,
%%      but WITHOUT ANY WARRANTY; without even the implied warranty of
%%      MERCHANTABILITY or FITNESS FOR A PARTICULAR PURPOSE.
%%      See the Libre Silicon Public License for more details.
%%
%%  ///////////////////////////////////////////////////////////////////
\begin{circuitdiagram}{18}{12}

    \usgate
    \gate[\inputs{2}]{and}{5}{9}{R}{}{}  % AND
    \gate[\inputs{4}]{nor}{12}{4}{R}{}{} % NOR
    \pin{1}{1}{L}{A}    % pin A
    \wire{2}{1}{9}{1}   % wire pin A
    \pin{1}{3}{L}{A1}   % pin A1
    \wire{2}{3}{9}{3}   % wire pin A1
    \pin{1}{5}{L}{A2}   % pin A2
    \wire{2}{5}{9}{5}   % wire pin A2
    \pin{1}{7}{L}{B}    % pin B
    \pin{1}{11}{L}{B1}  % pin B1
    \wire{9}{7}{9}{9}   % wire between OR and NAND
    \pin{17}{4}{R}{Y}   % pin Y

\end{circuitdiagram}
 %%  ************    LibreSilicon's StdCellLibrary   *******************
%%
%%  Organisation:   Chipforge
%%                  Germany / European Union
%%
%%  Profile:        Chipforge focus on fine System-on-Chip Cores in
%%                  Verilog HDL Code which are easy understandable and
%%                  adjustable. For further information see
%%                          www.chipforge.org
%%                  there are projects from small cores up to PCBs, too.
%%
%%  File:           StdCellLib/Documents/Datasheets/Circuitry/AO23.tex
%%
%%  Purpose:        Circuit File for AO23
%%
%%  ************    LaTeX with circdia.sty package      ***************
%%
%%  ///////////////////////////////////////////////////////////////////
%%
%%  Copyright (c) 2018 - 2022 by
%%                  chipforge <stdcelllib@nospam.chipforge.org>
%%  All rights reserved.
%%
%%      This Standard Cell Library is licensed under the Libre Silicon
%%      public license; you can redistribute it and/or modify it under
%%      the terms of the Libre Silicon public license as published by
%%      the Libre Silicon alliance, either version 1 of the License, or
%%      (at your option) any later version.
%%
%%      This design is distributed in the hope that it will be useful,
%%      but WITHOUT ANY WARRANTY; without even the implied warranty of
%%      MERCHANTABILITY or FITNESS FOR A PARTICULAR PURPOSE.
%%      See the Libre Silicon Public License for more details.
%%
%%  ///////////////////////////////////////////////////////////////////
\begin{circuitdiagram}[draft]{24}{12}

    \usgate
    % ----  1st column  ----
    \pin{1}{1}{L}{A}
    \pin{1}{5}{L}{A1}
    \gate[\inputs{2}]{and}{5}{3}{R}{}{}

    % ----  2nd column  ----
    \pin{8}{7}{L}{B}
    \pin{8}{9}{L}{B1}
    \pin{8}{11}{L}{B2}
    \wire{9}{3}{9}{5}
    \gate[\inputs{4}]{nor}{12}{8}{R}{}{}

    % ----  3rd column  ----
    \gate{not}{19}{8}{R}{}{}

    % ----  result ----
    \pin{23}{8}{R}{Z}

\end{circuitdiagram}

%%  ************    LibreSilicon's StdCellLibrary   *******************
%%
%%  Organisation:   Chipforge
%%                  Germany / European Union
%%
%%  Profile:        Chipforge focus on fine System-on-Chip Cores in
%%                  Verilog HDL Code which are easy understandable and
%%                  adjustable. For further information see
%%                          www.chipforge.org
%%                  there are projects from small cores up to PCBs, too.
%%
%%  File:           StdCellLib/Documents/Circuits/circuit_AOI31.tex
%%
%%  Purpose:        Circuit File for AOI31
%%
%%  ************    LaTeX with circdia.sty package      ***************
%%
%%  ///////////////////////////////////////////////////////////////////
%%
%%  Copyright (c) 2019 - 2021 by
%%                chipforge <stdcelllib@nospam.chipforge.org>
%%  All rights reserved.
%%
%%      This Standard Cell Library is licensed under the Libre Silicon
%%      public license; you can redistribute it and/or modify it under
%%      the terms of the Libre Silicon public license as published by
%%      the Libre Silicon alliance, either version 1 of the License, or
%%      (at your option) any later version.
%%
%%      This design is distributed in the hope that it will be useful,
%%      but WITHOUT ANY WARRANTY; without even the implied warranty of
%%      MERCHANTABILITY or FITNESS FOR A PARTICULAR PURPOSE.
%%      See the Libre Silicon Public License for more details.
%%
%%  ///////////////////////////////////////////////////////////////////
\begin{circuitdiagram}{18}{8}

    \usgate
    \gate[\inputs{3}]{and}{5}{5}{R}{}{}    % AND gate -> right
    \gate[\inputs{2}]{nor}{12}{3}{R}{}{}    % NOR gate -> right
    \pin{1}{1}{L}{A}    % pin A
    \pin{1}{3}{L}{B0}   % pin B0
    \pin{1}{5}{L}{B1}   % pin B1
    \pin{1}{7}{L}{B2}   % pin B2
    \wire{2}{1}{9}{1}     % wire from pin A
    \pin{17}{3}{R}{Y}   % pin Y

\end{circuitdiagram}
 %%  ************    LibreSilicon's StdCellLibrary   *******************
%%
%%  Organisation:   Chipforge
%%                  Germany / European Union
%%
%%  Profile:        Chipforge focus on fine System-on-Chip Cores in
%%                  Verilog HDL Code which are easy understandable and
%%                  adjustable. For further information see
%%                          www.chipforge.org
%%                  there are projects from small cores up to PCBs, too.
%%
%%  File:           StdCellLib/Documents/Circuits/AO31.tex
%%
%%  Purpose:        Circuit File for AO31
%%
%%  ************    LaTeX with circdia.sty package      ***************
%%
%%  ///////////////////////////////////////////////////////////////////
%%
%%  Copyright (c) 2019 by chipforge <stdcelllib@nospam.chipforge.org>
%%  All rights reserved.
%%
%%      This Standard Cell Library is licensed under the Libre Silicon
%%      public license; you can redistribute it and/or modify it under
%%      the terms of the Libre Silicon public license as published by
%%      the Libre Silicon alliance, either version 1 of the License, or
%%      (at your option) any later version.
%%
%%      This design is distributed in the hope that it will be useful,
%%      but WITHOUT ANY WARRANTY; without even the implied warranty of
%%      MERCHANTABILITY or FITNESS FOR A PARTICULAR PURPOSE.
%%      See the Libre Silicon Public License for more details.
%%
%%  ///////////////////////////////////////////////////////////////////
\begin{circuitdiagram}{24}{8}

    \usgate
    \gate[\inputs{3}]{and}{5}{5}{R}{}{}   % AND
    \gate[\inputs{2}]{nor}{12}{3}{R}{}{}  % NOR
    \gate{not}{19}{3}{R}{}{}    % NOT
    \pin{1}{1}{L}{A}    % pin A
    \pin{1}{3}{L}{B}    % pin B
    \pin{1}{5}{L}{B1}   % pin B1
    \pin{1}{7}{L}{B2}   % pin B2
    \wire{2}{1}{9}{1}   % wire from pin A
    \pin{23}{3}{R}{Z}   % pin Z

\end{circuitdiagram}

%%  ************    LibreSilicon's StdCellLibrary   *******************
%%
%%  Organisation:   Chipforge
%%                  Germany / European Union
%%
%%  Profile:        Chipforge focus on fine System-on-Chip Cores in
%%                  Verilog HDL Code which are easy understandable and
%%                  adjustable. For further information see
%%                          www.chipforge.org
%%                  there are projects from small cores up to PCBs, too.
%%
%%  File:           StdCellLib/Documents/Datasheets/Circuitry/AOI32.tex
%%
%%  Purpose:        Circuit File for AOI32
%%
%%  ************    LaTeX with circdia.sty package      ***************
%%
%%  ///////////////////////////////////////////////////////////////////
%%
%%  Copyright (c) 2018 - 2022 by
%%                  chipforge <stdcelllib@nospam.chipforge.org>
%%  All rights reserved.
%%
%%      This Standard Cell Library is licensed under the Libre Silicon
%%      public license; you can redistribute it and/or modify it under
%%      the terms of the Libre Silicon public license as published by
%%      the Libre Silicon alliance, either version 1 of the License, or
%%      (at your option) any later version.
%%
%%      This design is distributed in the hope that it will be useful,
%%      but WITHOUT ANY WARRANTY; without even the implied warranty of
%%      MERCHANTABILITY or FITNESS FOR A PARTICULAR PURPOSE.
%%      See the Libre Silicon Public License for more details.
%%
%%  ///////////////////////////////////////////////////////////////////
\begin{circuitdiagram}[draft]{18}{10}

    \usgate
    % ----  1st column  ----
    \pin{1}{1}{L}{A}
    \pin{1}{3}{L}{A1}
    \pin{1}{5}{L}{A2}
    \gate[\inputs{3}]{and}{5}{3}{R}{}{}

    % ----  2nd column  ----
    \pin{8}{7}{L}{B}
    \pin{8}{9}{L}{B1}
    \wire{9}{3}{9}{5}
    \gate[\inputs{3}]{nor}{12}{7}{R}{}{}

    % ----  result ----
    \pin{17}{7}{R}{Y}

\end{circuitdiagram}
 %%  ************    LibreSilicon's StdCellLibrary   *******************
%%
%%  Organisation:   Chipforge
%%                  Germany / European Union
%%
%%  Profile:        Chipforge focus on fine System-on-Chip Cores in
%%                  Verilog HDL Code which are easy understandable and
%%                  adjustable. For further information see
%%                          www.chipforge.org
%%                  there are projects from small cores up to PCBs, too.
%%
%%  File:           StdCellLib/Documents/Circuits/AO32.tex
%%
%%  Purpose:        Circuit File for AOI32
%%
%%  ************    LaTeX with circdia.sty package      ***************
%%
%%  ///////////////////////////////////////////////////////////////////
%%
%%  Copyright (c) 2019 by chipforge <stdcelllib@nospam.chipforge.org>
%%  All rights reserved.
%%
%%      This Standard Cell Library is licensed under the Libre Silicon
%%      public license; you can redistribute it and/or modify it under
%%      the terms of the Libre Silicon public license as published by
%%      the Libre Silicon alliance, either version 1 of the License, or
%%      (at your option) any later version.
%%
%%      This design is distributed in the hope that it will be useful,
%%      but WITHOUT ANY WARRANTY; without even the implied warranty of
%%      MERCHANTABILITY or FITNESS FOR A PARTICULAR PURPOSE.
%%      See the Libre Silicon Public License for more details.
%%
%%  ///////////////////////////////////////////////////////////////////
\begin{circuitdiagram}{24}{10}

    \usgate
    \gate[\inputs{3}]{and}{5}{7}{R}{}{}    % AND
    \gate[\inputs{3}]{nor}{12}{3}{R}{}{}    % NOR
    \gate{not}{19}{3}{R}{}{}    % NOT
    \pin{1}{1}{L}{A}    % pin A
    \pin{1}{3}{L}{A1}   % pin A1
    \pin{1}{5}{L}{B}    % pin B
    \pin{1}{7}{L}{B1}   % pin B1
    \pin{1}{9}{L}{B2}   % pin B2
    \wire{2}{1}{9}{1}     % wire from pin A
    \wire{2}{3}{9}{3}     % wire from pin A1
    \wire{9}{7}{9}{5}     % wire between AND and NOR
    \pin{23}{3}{R}{Z}   % pin Z

\end{circuitdiagram}

%%  ************    LibreSilicon's StdCellLibrary   *******************
%%
%%  Organisation:   Chipforge
%%                  Germany / European Union
%%
%%  Profile:        Chipforge focus on fine System-on-Chip Cores in
%%                  Verilog HDL Code which are easy understandable and
%%                  adjustable. For further information see
%%                          www.chipforge.org
%%                  there are projects from small cores up to PCBs, too.
%%
%%  File:           StdCellLib/Documents/Datasheets/Circuitry/AOI33.tex
%%
%%  Purpose:        Circuit File for AOI33
%%
%%  ************    LaTeX with circdia.sty package      ***************
%%
%%  ///////////////////////////////////////////////////////////////////
%%
%%  Copyright (c) 2018 - 2022 by
%%                  chipforge <stdcelllib@nospam.chipforge.org>
%%  All rights reserved.
%%
%%      This Standard Cell Library is licensed under the Libre Silicon
%%      public license; you can redistribute it and/or modify it under
%%      the terms of the Libre Silicon public license as published by
%%      the Libre Silicon alliance, either version 1 of the License, or
%%      (at your option) any later version.
%%
%%      This design is distributed in the hope that it will be useful,
%%      but WITHOUT ANY WARRANTY; without even the implied warranty of
%%      MERCHANTABILITY or FITNESS FOR A PARTICULAR PURPOSE.
%%      See the Libre Silicon Public License for more details.
%%
%%  ///////////////////////////////////////////////////////////////////
\begin{circuitdiagram}[draft]{18}{12}

    \usgate
    % ----  1st column  ----
    \pin{1}{1}{L}{A}
    \pin{1}{3}{L}{A1}
    \pin{1}{5}{L}{A2}
    \gate[\inputs{3}]{and}{5}{3}{R}{}{}

    % ----  2nd column  ----
    \pin{8}{7}{L}{B}
    \pin{8}{9}{L}{B1}
    \pin{8}{11}{L}{B2}
    \wire{9}{3}{9}{5}
    \gate[\inputs{4}]{nor}{12}{8}{R}{}{}

    % ----  result ----
    \pin{17}{8}{R}{Y}

\end{circuitdiagram}
 %%  ************    LibreSilicon's StdCellLibrary   *******************
%%
%%  Organisation:   Chipforge
%%                  Germany / European Union
%%
%%  Profile:        Chipforge focus on fine System-on-Chip Cores in
%%                  Verilog HDL Code which are easy understandable and
%%                  adjustable. For further information see
%%                          www.chipforge.org
%%                  there are projects from small cores up to PCBs, too.
%%
%%  File:           StdCellLib/Documents/Datasheets/Circuitry/AO33.tex
%%
%%  Purpose:        Circuit File for AO33
%%
%%  ************    LaTeX with circdia.sty package      ***************
%%
%%  ///////////////////////////////////////////////////////////////////
%%
%%  Copyright (c) 2018 - 2022 by
%%                  chipforge <stdcelllib@nospam.chipforge.org>
%%  All rights reserved.
%%
%%      This Standard Cell Library is licensed under the Libre Silicon
%%      public license; you can redistribute it and/or modify it under
%%      the terms of the Libre Silicon public license as published by
%%      the Libre Silicon alliance, either version 1 of the License, or
%%      (at your option) any later version.
%%
%%      This design is distributed in the hope that it will be useful,
%%      but WITHOUT ANY WARRANTY; without even the implied warranty of
%%      MERCHANTABILITY or FITNESS FOR A PARTICULAR PURPOSE.
%%      See the Libre Silicon Public License for more details.
%%
%%  ///////////////////////////////////////////////////////////////////
\begin{circuitdiagram}[draft]{24}{12}

    \usgate
    % ----  1st column  ----
    \pin{1}{1}{L}{A}
    \pin{1}{3}{L}{A1}
    \pin{1}{5}{L}{A2}
    \gate[\inputs{3}]{and}{5}{3}{R}{}{}

    % ----  2nd column  ----
    \pin{8}{7}{L}{B}
    \pin{8}{9}{L}{B1}
    \pin{8}{11}{L}{B2}
    \wire{9}{3}{9}{5}
    \gate[\inputs{4}]{nor}{12}{8}{R}{}{}

    % ----  3rd column  ----
    \gate{not}{19}{8}{R}{}{}

    % ----  result ----
    \pin{23}{8}{R}{Z}

\end{circuitdiagram}

%%  ************    LibreSilicon's StdCellLibrary   *******************
%%
%%  Organisation:   Chipforge
%%                  Germany / European Union
%%
%%  Profile:        Chipforge focus on fine System-on-Chip Cores in
%%                  Verilog HDL Code which are easy understandable and
%%                  adjustable. For further information see
%%                          www.chipforge.org
%%                  there are projects from small cores up to PCBs, too.
%%
%%  File:           StdCellLib/Documents/Circuits/AOI41.tex
%%
%%  Purpose:        Circuit File for AOI41
%%
%%  ************    LaTeX with circdia.sty package      ***************
%%
%%  ///////////////////////////////////////////////////////////////////
%%
%%  Copyright (c) 2019 - 2021 by
%%                chipforge <stdcelllib@nospam.chipforge.org>
%%  All rights reserved.
%%
%%      This Standard Cell Library is licensed under the Libre Silicon
%%      public license; you can redistribute it and/or modify it under
%%      the terms of the Libre Silicon public license as published by
%%      the Libre Silicon alliance, either version 1 of the License, or
%%      (at your option) any later version.
%%
%%      This design is distributed in the hope that it will be useful,
%%      but WITHOUT ANY WARRANTY; without even the implied warranty of
%%      MERCHANTABILITY or FITNESS FOR A PARTICULAR PURPOSE.
%%      See the Libre Silicon Public License for more details.
%%
%%  ///////////////////////////////////////////////////////////////////
\begin{circuitdiagram}{18}{10}

    \usgate
    \gate[\inputs{4}]{and}{5}{6}{R}{}{}  % AND
    \gate[\inputs{2}]{nor}{12}{3}{R}{}{} % NOR
    \pin{1}{1}{L}{A}    % pin A
    \pin{1}{3}{L}{B}    % pin B
    \pin{1}{5}{L}{B1}   % pin B1
    \pin{1}{7}{L}{B2}   % pin B2
    \pin{1}{9}{L}{B3}   % pin B3
    \wire{9}{5}{9}{6}   % wire between OR and NAND
    \wire{2}{1}{9}{1}   % wire from pin A
    \pin{17}{3}{R}{Y}   % pin Y

\end{circuitdiagram}
 %%  ************    LibreSilicon's StdCellLibrary   *******************
%%
%%  Organisation:   Chipforge
%%                  Germany / European Union
%%
%%  Profile:        Chipforge focus on fine System-on-Chip Cores in
%%                  Verilog HDL Code which are easy understandable and
%%                  adjustable. For further information see
%%                          www.chipforge.org
%%                  there are projects from small cores up to PCBs, too.
%%
%%  File:           StdCellLib/Documents/Circuits/AO41.tex
%%
%%  Purpose:        Circuit File for AO41
%%
%%  ************    LaTeX with circdia.sty package      ***************
%%
%%  ///////////////////////////////////////////////////////////////////
%%
%%  Copyright (c) 2019 by chipforge <stdcelllib@nospam.chipforge.org>
%%  All rights reserved.
%%
%%      This Standard Cell Library is licensed under the Libre Silicon
%%      public license; you can redistribute it and/or modify it under
%%      the terms of the Libre Silicon public license as published by
%%      the Libre Silicon alliance, either version 1 of the License, or
%%      (at your option) any later version.
%%
%%      This design is distributed in the hope that it will be useful,
%%      but WITHOUT ANY WARRANTY; without even the implied warranty of
%%      MERCHANTABILITY or FITNESS FOR A PARTICULAR PURPOSE.
%%      See the Libre Silicon Public License for more details.
%%
%%  ///////////////////////////////////////////////////////////////////
\begin{circuitdiagram}{24}{10}

    \usgate
    \gate[\inputs{4}]{and}{5}{6}{R}{}{}  % AND
    \gate[\inputs{2}]{nor}{12}{3}{R}{}{} % NOR
    \gate{not}{19}{3}{R}{}{} % NOT
    \pin{1}{1}{L}{A}    % pin A
    \pin{1}{3}{L}{B}    % pin B
    \pin{1}{5}{L}{B1}   % pin B1
    \pin{1}{7}{L}{B2}   % pin B2
    \pin{1}{9}{L}{B3}   % pin B3
    \wire{9}{5}{9}{6}   % wire between OR and NAND
    \wire{2}{1}{9}{1}   % wire from pin A
    \pin{23}{3}{R}{Z}   % pin Z

\end{circuitdiagram}

%%  ************    LibreSilicon's StdCellLibrary   *******************
%%
%%  Organisation:   Chipforge
%%                  Germany / European Union
%%
%%  Profile:        Chipforge focus on fine System-on-Chip Cores in
%%                  Verilog HDL Code which are easy understandable and
%%                  adjustable. For further information see
%%                          www.chipforge.org
%%                  there are projects from small cores up to PCBs, too.
%%
%%  File:           StdCellLib/Documents/Circuits/AOI42.tex
%%
%%  Purpose:        Circuit File for AOI42
%%
%%  ************    LaTeX with circdia.sty package      ***************
%%
%%  ///////////////////////////////////////////////////////////////////
%%
%%  Copyright (c) 2019 - 2021 by
%%                chipforge <stdcelllib@nospam.chipforge.org>
%%  All rights reserved.
%%
%%      This Standard Cell Library is licensed under the Libre Silicon
%%      public license; you can redistribute it and/or modify it under
%%      the terms of the Libre Silicon public license as published by
%%      the Libre Silicon alliance, either version 1 of the License, or
%%      (at your option) any later version.
%%
%%      This design is distributed in the hope that it will be useful,
%%      but WITHOUT ANY WARRANTY; without even the implied warranty of
%%      MERCHANTABILITY or FITNESS FOR A PARTICULAR PURPOSE.
%%      See the Libre Silicon Public License for more details.
%%
%%  ///////////////////////////////////////////////////////////////////
\begin{circuitdiagram}{18}{10}

    \usgate
    \gate[\inputs{4}]{and}{5}{8}{R}{}{}   % AND
    \gate[\inputs{3}]{nor}{12}{3}{R}{}{}  % NOR
    \pin{1}{1}{L}{A}    % pin A
    \wire{2}{1}{9}{1}   % wire pin A
    \pin{1}{3}{L}{A1}   % pin A1
    \wire{2}{3}{9}{3}   % wire pin A1
    \pin{1}{5}{L}{B}    % pin B
    \pin{1}{7}{L}{B1}   % pin B1
    \pin{1}{9}{L}{B2}   % pin B2
    \pin{1}{11}{L}{B3}  % pin B3
    \wire{9}{8}{9}{5}   % wire between AND and NOR
    \pin{17}{3}{R}{Y}   % pin Y

\end{circuitdiagram}
 %%  ************    LibreSilicon's StdCellLibrary   *******************
%%
%%  Organisation:   Chipforge
%%                  Germany / European Union
%%
%%  Profile:        Chipforge focus on fine System-on-Chip Cores in
%%                  Verilog HDL Code which are easy understandable and
%%                  adjustable. For further information see
%%                          www.chipforge.org
%%                  there are projects from small cores up to PCBs, too.
%%
%%  File:           StdCellLib/Documents/Datasheets/Circuitry/AO42.tex
%%
%%  Purpose:        Circuit File for AO42
%%
%%  ************    LaTeX with circdia.sty package      ***************
%%
%%  ///////////////////////////////////////////////////////////////////
%%
%%  Copyright (c) 2018 - 2022 by
%%                  chipforge <stdcelllib@nospam.chipforge.org>
%%  All rights reserved.
%%
%%      This Standard Cell Library is licensed under the Libre Silicon
%%      public license; you can redistribute it and/or modify it under
%%      the terms of the Libre Silicon public license as published by
%%      the Libre Silicon alliance, either version 1 of the License, or
%%      (at your option) any later version.
%%
%%      This design is distributed in the hope that it will be useful,
%%      but WITHOUT ANY WARRANTY; without even the implied warranty of
%%      MERCHANTABILITY or FITNESS FOR A PARTICULAR PURPOSE.
%%      See the Libre Silicon Public License for more details.
%%
%%  ///////////////////////////////////////////////////////////////////
\begin{circuitdiagram}[draft]{24}{11}

    \usgate
    % ----  1st column  ----
    \pin{1}{1}{L}{A}
    \pin{1}{3}{L}{A1}
    \pin{1}{5}{L}{A2}
    \pin{1}{7}{L}{A3}
    \gate[\inputs{4}]{and}{5}{4}{R}{}{}

    % ----  2nd column  ----
    \pin{8}{8}{L}{B}
    \pin{8}{10}{L}{B1}
    \wire{9}{4}{9}{6}
    \gate[\inputs{3}]{nor}{12}{8}{R}{}{}

    % ----  3rd column  ----
    \gate{not}{19}{8}{R}{}{}

    % ----  result ----
    \pin{23}{8}{R}{Z}

\end{circuitdiagram}

%%  ************    LibreSilicon's StdCellLibrary   *******************
%%
%%  Organisation:   Chipforge
%%                  Germany / European Union
%%
%%  Profile:        Chipforge focus on fine System-on-Chip Cores in
%%                  Verilog HDL Code which are easy understandable and
%%                  adjustable. For further information see
%%                          www.chipforge.org
%%                  there are projects from small cores up to PCBs, too.
%%
%%  File:           StdCellLib/Documents/Circuits/AOI43.tex
%%
%%  Purpose:        Circuit File for AOI43
%%
%%  ************    LaTeX with circdia.sty package      ***************
%%
%%  ///////////////////////////////////////////////////////////////////
%%
%%  Copyright (c) 2019 - 2021 by
%%                chipforge <stdcelllib@nospam.chipforge.org>
%%  All rights reserved.
%%
%%      This Standard Cell Library is licensed under the Libre Silicon
%%      public license; you can redistribute it and/or modify it under
%%      the terms of the Libre Silicon public license as published by
%%      the Libre Silicon alliance, either version 1 of the License, or
%%      (at your option) any later version.
%%
%%      This design is distributed in the hope that it will be useful,
%%      but WITHOUT ANY WARRANTY; without even the implied warranty of
%%      MERCHANTABILITY or FITNESS FOR A PARTICULAR PURPOSE.
%%      See the Libre Silicon Public License for more details.
%%
%%  ///////////////////////////////////////////////////////////////////
\begin{circuitdiagram}{18}{12}

    \usgate
    \gate[\inputs{4}]{and}{5}{10}{R}{}{} % AND
    \gate[\inputs{4}]{nor}{12}{4}{R}{}{} % NOR
    \pin{1}{1}{L}{A}    % pin A
    \wire{2}{1}{9}{1}   % wire pin A
    \pin{1}{3}{L}{A1}   % pin A1
    \wire{2}{3}{9}{3}   % wire pin A1
    \pin{1}{5}{L}{A2}   % pin A2
    \wire{2}{5}{9}{5}   % wire pin 42
    \pin{1}{7}{L}{B}    % pin B
    \pin{1}{9}{L}{B1}   % pin B1
    \pin{1}{11}{L}{B2}  % pin B2
    \pin{1}{13}{L}{B3}  % pin B3
    \wire{9}{7}{9}{10}  % wire between OR and NAND
    \pin{17}{4}{R}{Y}   % pin Y

\end{circuitdiagram}
 %%  ************    LibreSilicon's StdCellLibrary   *******************
%%
%%  Organisation:   Chipforge
%%                  Germany / European Union
%%
%%  Profile:        Chipforge focus on fine System-on-Chip Cores in
%%                  Verilog HDL Code which are easy understandable and
%%                  adjustable. For further information see
%%                          www.chipforge.org
%%                  there are projects from small cores up to PCBs, too.
%%
%%  File:           StdCellLib/Documents/Circuits/AO43.tex
%%
%%  Purpose:        Circuit File for AO43
%%
%%  ************    LaTeX with circdia.sty package      ***************
%%
%%  ///////////////////////////////////////////////////////////////////
%%
%%  Copyright (c) 2019 by chipforge <stdcelllib@nospam.chipforge.org>
%%  All rights reserved.
%%
%%      This Standard Cell Library is licensed under the Libre Silicon
%%      public license; you can redistribute it and/or modify it under
%%      the terms of the Libre Silicon public license as published by
%%      the Libre Silicon alliance, either version 1 of the License, or
%%      (at your option) any later version.
%%
%%      This design is distributed in the hope that it will be useful,
%%      but WITHOUT ANY WARRANTY; without even the implied warranty of
%%      MERCHANTABILITY or FITNESS FOR A PARTICULAR PURPOSE.
%%      See the Libre Silicon Public License for more details.
%%
%%  ///////////////////////////////////////////////////////////////////
\begin{circuitdiagram}{24}{12}

    \usgate
    \gate[\inputs{4}]{and}{5}{10}{R}{}{} % AND
    \gate[\inputs{4}]{nor}{12}{4}{R}{}{} % NOR
    \gate{not}{19}{4}{R}{}{} % NOT
    \pin{1}{1}{L}{A}    % pin A
    \wire{2}{1}{9}{1}   % wire pin A
    \pin{1}{3}{L}{A1}   % pin A1
    \wire{2}{3}{9}{3}   % wire pin A1
    \pin{1}{5}{L}{A2}   % pin A2
    \wire{2}{5}{9}{5}   % wire pin 42
    \pin{1}{7}{L}{B}    % pin B
    \pin{1}{9}{L}{B1}   % pin B1
    \pin{1}{11}{L}{B2}  % pin B2
    \pin{1}{13}{L}{B3}  % pin B3
    \wire{9}{7}{9}{10}  % wire between OR and NAND
    \pin{23}{4}{R}{Z}   % pin Z

\end{circuitdiagram}


%%  ************    LibreSilicon's StdCellLibrary   *******************
%%
%%  Organisation:   Chipforge
%%                  Germany / European Union
%%
%%  Profile:        Chipforge focus on fine System-on-Chip Cores in
%%                  Verilog HDL Code which are easy understandable and
%%                  adjustable. For further information see
%%                          www.chipforge.org
%%                  there are projects from small cores up to PCBs, too.
%%
%%  File:           StdCellLib/Documents/section-OAI_complex.tex
%%
%%  Purpose:        Section Level File for Standard Cell Library Documentation
%%
%%  ************    LaTeX with circdia.sty package      ***************
%%
%%  ///////////////////////////////////////////////////////////////////
%%
%%  Copyright (c) 2018 - 2022 by
%%                  chipforge <stdcelllib@nospam.chipforge.org>
%%  All rights reserved.
%%
%%      This Standard Cell Library is licensed under the Libre Silicon
%%      public license; you can redistribute it and/or modify it under
%%      the terms of the Libre Silicon public license as published by
%%      the Libre Silicon alliance, either version 1 of the License, or
%%      (at your option) any later version.
%%
%%      This design is distributed in the hope that it will be useful,
%%      but WITHOUT ANY WARRANTY; without even the implied warranty of
%%      MERCHANTABILITY or FITNESS FOR A PARTICULAR PURPOSE.
%%      See the Libre Silicon Public License for more details.
%%
%%  ///////////////////////////////////////////////////////////////////
\section{OR-AND(-Invert) Complex Gates}

%%  ************    LibreSilicon's StdCellLibrary   *******************
%%
%%  Organisation:   Chipforge
%%                  Germany / European Union
%%
%%  Profile:        Chipforge focus on fine System-on-Chip Cores in
%%                  Verilog HDL Code which are easy understandable and
%%                  adjustable. For further information see
%%                          www.chipforge.org
%%                  there are projects from small cores up to PCBs, too.
%%
%%  File:           StdCellLib/Documents/Datasheets/Circuitry/OAI21.tex
%%
%%  Purpose:        Circuit File for OAI21
%%
%%  ************    LaTeX with circdia.sty package      ***************
%%
%%  ///////////////////////////////////////////////////////////////////
%%
%%  Copyright (c) 2018 - 2022 by
%%                  chipforge <stdcelllib@nospam.chipforge.org>
%%  All rights reserved.
%%
%%      This Standard Cell Library is licensed under the Libre Silicon
%%      public license; you can redistribute it and/or modify it under
%%      the terms of the Libre Silicon public license as published by
%%      the Libre Silicon alliance, either version 1 of the License, or
%%      (at your option) any later version.
%%
%%      This design is distributed in the hope that it will be useful,
%%      but WITHOUT ANY WARRANTY; without even the implied warranty of
%%      MERCHANTABILITY or FITNESS FOR A PARTICULAR PURPOSE.
%%      See the Libre Silicon Public License for more details.
%%
%%  ///////////////////////////////////////////////////////////////////
\begin{circuitdiagram}[draft]{18}{8}

    \usgate
    % ----  1st column  ----
    \pin{1}{1}{L}{A}
    \pin{1}{5}{L}{A1}
    \gate[\inputs{2}]{or}{5}{3}{R}{}{}

    % ----  2nd column  ----
    \pin{8}{7}{L}{B}
    \gate[\inputs{2}]{nand}{12}{5}{R}{}{}

    % ----  result ----
    \pin{17}{5}{R}{Y}

\end{circuitdiagram}
 %%  ************    LibreSilicon's StdCellLibrary   *******************
%%
%%  Organisation:   Chipforge
%%                  Germany / European Union
%%
%%  Profile:        Chipforge focus on fine System-on-Chip Cores in
%%                  Verilog HDL Code which are easy understandable and
%%                  adjustable. For further information see
%%                          www.chipforge.org
%%                  there are projects from small cores up to PCBs, too.
%%
%%  File:           StdCellLib/Documents/Datasheets/Circuitry/OA21.tex
%%
%%  Purpose:        Circuit File for OA21
%%
%%  ************    LaTeX with circdia.sty package      ***************
%%
%%  ///////////////////////////////////////////////////////////////////
%%
%%  Copyright (c) 2018 - 2022 by
%%                  chipforge <stdcelllib@nospam.chipforge.org>
%%  All rights reserved.
%%
%%      This Standard Cell Library is licensed under the Libre Silicon
%%      public license; you can redistribute it and/or modify it under
%%      the terms of the Libre Silicon public license as published by
%%      the Libre Silicon alliance, either version 1 of the License, or
%%      (at your option) any later version.
%%
%%      This design is distributed in the hope that it will be useful,
%%      but WITHOUT ANY WARRANTY; without even the implied warranty of
%%      MERCHANTABILITY or FITNESS FOR A PARTICULAR PURPOSE.
%%      See the Libre Silicon Public License for more details.
%%
%%  ///////////////////////////////////////////////////////////////////
\begin{circuitdiagram}[draft]{24}{8}

    \usgate
    % ----  1st column  ----
    \pin{1}{1}{L}{A}
    \pin{1}{5}{L}{A1}
    \gate[\inputs{2}]{or}{5}{3}{R}{}{}

    % ----  2nd column  ----
    \pin{8}{7}{L}{B}
    \gate[\inputs{2}]{nand}{12}{5}{R}{}{}

    % ----  3rd column  ----
    \gate{not}{19}{5}{R}{}{}

    % ----  result ----
    \pin{23}{5}{R}{Z}

\end{circuitdiagram}

%%  ************    LibreSilicon's StdCellLibrary   *******************
%%
%%  Organisation:   Chipforge
%%                  Germany / European Union
%%
%%  Profile:        Chipforge focus on fine System-on-Chip Cores in
%%                  Verilog HDL Code which are easy understandable and
%%                  adjustable. For further information see
%%                          www.chipforge.org
%%                  there are projects from small cores up to PCBs, too.
%%
%%  File:           StdCellLib/Documents/Circuits/OAI22.tex
%%
%%  Purpose:        Circuit File for OAI22
%%
%%  ************    LaTeX with circdia.sty package      ***************
%%
%%  ///////////////////////////////////////////////////////////////////
%%
%%  Copyright (c) 2019 by chipforge <stdcelllib@nospam.chipforge.org>
%%  All rights reserved.
%%
%%      This Standard Cell Library is licensed under the Libre Silicon
%%      public license; you can redistribute it and/or modify it under
%%      the terms of the Libre Silicon public license as published by
%%      the Libre Silicon alliance, either version 1 of the License, or
%%      (at your option) any later version.
%%
%%      This design is distributed in the hope that it will be useful,
%%      but WITHOUT ANY WARRANTY; without even the implied warranty of
%%      MERCHANTABILITY or FITNESS FOR A PARTICULAR PURPOSE.
%%      See the Libre Silicon Public License for more details.
%%
%%  ///////////////////////////////////////////////////////////////////
\begin{circuitdiagram}{18}{10}

    \usgate
    \gate[\inputs{2}]{or}{5}{7}{R}{}{}    % OR
    \gate[\inputs{3}]{nand}{12}{3}{R}{}{} % NAND
    \pin{1}{1}{L}{A}    % pin A
    \pin{1}{3}{L}{A1}   % pin A1
    \pin{1}{5}{L}{B}    % pin B
    \pin{1}{9}{L}{B1}   % pin B1
    \wire{9}{5}{9}{7}   % wire between OR and NAND
    \wire{2}{1}{9}{1}   % wire from pin A
    \wire{2}{3}{9}{3}   % wire from pin A1
    \pin{17}{3}{R}{Y}   % pin Y

\end{circuitdiagram}
 %%  ************    LibreSilicon's StdCellLibrary   *******************
%%
%%  Organisation:   Chipforge
%%                  Germany / European Union
%%
%%  Profile:        Chipforge focus on fine System-on-Chip Cores in
%%                  Verilog HDL Code which are easy understandable and
%%                  adjustable. For further information see
%%                          www.chipforge.org
%%                  there are projects from small cores up to PCBs, too.
%%
%%  File:           StdCellLib/Documents/Datasheets/Circuitry/OA22.tex
%%
%%  Purpose:        Circuit File for OA22
%%
%%  ************    LaTeX with circdia.sty package      ***************
%%
%%  ///////////////////////////////////////////////////////////////////
%%
%%  Copyright (c) 2018 - 2022 by
%%                  chipforge <stdcelllib@nospam.chipforge.org>
%%  All rights reserved.
%%
%%      This Standard Cell Library is licensed under the Libre Silicon
%%      public license; you can redistribute it and/or modify it under
%%      the terms of the Libre Silicon public license as published by
%%      the Libre Silicon alliance, either version 1 of the License, or
%%      (at your option) any later version.
%%
%%      This design is distributed in the hope that it will be useful,
%%      but WITHOUT ANY WARRANTY; without even the implied warranty of
%%      MERCHANTABILITY or FITNESS FOR A PARTICULAR PURPOSE.
%%      See the Libre Silicon Public License for more details.
%%
%%  ///////////////////////////////////////////////////////////////////
\begin{circuitdiagram}[draft]{24}{10}

    \usgate
    % ----  1st column  ----
    \pin{1}{1}{L}{A}
    \pin{1}{5}{L}{A1}
    \gate[\inputs{2}]{or}{5}{3}{R}{}{}

    % ----  2nd column  ----
    \pin{8}{7}{L}{B}
    \pin{8}{9}{L}{B1}
    \wire{9}{3}{9}{5}
    \gate[\inputs{3}]{nand}{12}{7}{R}{}{}

    % ----  3rd column  ----
    \gate{not}{19}{7}{R}{}{}

    % ----  result ----
    \pin{23}{7}{R}{Z}

\end{circuitdiagram}

%%  ************    LibreSilicon's StdCellLibrary   *******************
%%
%%  Organisation:   Chipforge
%%                  Germany / European Union
%%
%%  Profile:        Chipforge focus on fine System-on-Chip Cores in
%%                  Verilog HDL Code which are easy understandable and
%%                  adjustable. For further information see
%%                          www.chipforge.org
%%                  there are projects from small cores up to PCBs, too.
%%
%%  File:           StdCellLib/Documents/Circuits/OAI23.tex
%%
%%  Purpose:        Circuit File for OAI23
%%
%%  ************    LaTeX with circdia.sty package      ***************
%%
%%  ///////////////////////////////////////////////////////////////////
%%
%%  Copyright (c) 2019 by chipforge <stdcelllib@nospam.chipforge.org>
%%  All rights reserved.
%%
%%      This Standard Cell Library is licensed under the Libre Silicon
%%      public license; you can redistribute it and/or modify it under
%%      the terms of the Libre Silicon public license as published by
%%      the Libre Silicon alliance, either version 1 of the License, or
%%      (at your option) any later version.
%%
%%      This design is distributed in the hope that it will be useful,
%%      but WITHOUT ANY WARRANTY; without even the implied warranty of
%%      MERCHANTABILITY or FITNESS FOR A PARTICULAR PURPOSE.
%%      See the Libre Silicon Public License for more details.
%%
%%  ///////////////////////////////////////////////////////////////////
\begin{circuitdiagram}{18}{12}

    \usgate
    \gate[\inputs{2}]{or}{5}{9}{R}{}{}    % OR
    \gate[\inputs{4}]{nand}{12}{4}{R}{}{} % NAND
    \pin{1}{1}{L}{A}    % pin A
    \pin{1}{3}{L}{A1}   % pin A1
    \pin{1}{5}{L}{A2}   % pin A2
    \pin{1}{7}{L}{B}    % pin B
    \pin{1}{11}{L}{B1}  % pin B1
    \wire{2}{1}{9}{1}   % wire from pin A
    \wire{2}{3}{9}{3}   % wire from pin A1
    \wire{2}{5}{9}{5}   % wire from pin A2
    \wire{9}{9}{9}{7}   % wire between OR and NAND
    \pin{17}{4}{R}{Y}   % pin Y

\end{circuitdiagram}
 %%  ************    LibreSilicon's StdCellLibrary   *******************
%%
%%  Organisation:   Chipforge
%%                  Germany / European Union
%%
%%  Profile:        Chipforge focus on fine System-on-Chip Cores in
%%                  Verilog HDL Code which are easy understandable and
%%                  adjustable. For further information see
%%                          www.chipforge.org
%%                  there are projects from small cores up to PCBs, too.
%%
%%  File:           StdCellLib/Documents/Datasheets/Circuitry/OA23.tex
%%
%%  Purpose:        Circuit File for OA23
%%
%%  ************    LaTeX with circdia.sty package      ***************
%%
%%  ///////////////////////////////////////////////////////////////////
%%
%%  Copyright (c) 2018 - 2022 by
%%                  chipforge <stdcelllib@nospam.chipforge.org>
%%  All rights reserved.
%%
%%      This Standard Cell Library is licensed under the Libre Silicon
%%      public license; you can redistribute it and/or modify it under
%%      the terms of the Libre Silicon public license as published by
%%      the Libre Silicon alliance, either version 1 of the License, or
%%      (at your option) any later version.
%%
%%      This design is distributed in the hope that it will be useful,
%%      but WITHOUT ANY WARRANTY; without even the implied warranty of
%%      MERCHANTABILITY or FITNESS FOR A PARTICULAR PURPOSE.
%%      See the Libre Silicon Public License for more details.
%%
%%  ///////////////////////////////////////////////////////////////////
\begin{circuitdiagram}[draft]{24}{12}

    \usgate
    % ----  1st column  ----
    \pin{1}{1}{L}{A}
    \pin{1}{5}{L}{A1}
    \gate[\inputs{2}]{or}{5}{3}{R}{}{}

    % ----  2nd column  ----
    \pin{8}{7}{L}{B}
    \pin{8}{9}{L}{B1}
    \pin{8}{11}{L}{B2}
    \wire{9}{3}{9}{5}
    \gate[\inputs{4}]{nand}{12}{8}{R}{}{}

    % ----  3rd column  ----
    \gate{not}{19}{8}{R}{}{}

    % ----  result ----
    \pin{23}{8}{R}{Z}

\end{circuitdiagram}

%%  ************    LibreSilicon's StdCellLibrary   *******************
%%
%%  Organisation:   Chipforge
%%                  Germany / European Union
%%
%%  Profile:        Chipforge focus on fine System-on-Chip Cores in
%%                  Verilog HDL Code which are easy understandable and
%%                  adjustable. For further information see
%%                          www.chipforge.org
%%                  there are projects from small cores up to PCBs, too.
%%
%%  File:           StdCellLib/Documents/Circuits/OAI31.tex
%%
%%  Purpose:        Circuit File for OAI31
%%
%%  ************    LaTeX with circdia.sty package      ***************
%%
%%  ///////////////////////////////////////////////////////////////////
%%
%%  Copyright (c) 2019 by chipforge <stdcelllib@nospam.chipforge.org>
%%  All rights reserved.
%%
%%      This Standard Cell Library is licensed under the Libre Silicon
%%      public license; you can redistribute it and/or modify it under
%%      the terms of the Libre Silicon public license as published by
%%      the Libre Silicon alliance, either version 1 of the License, or
%%      (at your option) any later version.
%%
%%      This design is distributed in the hope that it will be useful,
%%      but WITHOUT ANY WARRANTY; without even the implied warranty of
%%      MERCHANTABILITY or FITNESS FOR A PARTICULAR PURPOSE.
%%      See the Libre Silicon Public License for more details.
%%
%%  ///////////////////////////////////////////////////////////////////
\begin{circuitdiagram}{18}{8}

    \usgate
    \gate[\inputs{3}]{or}{5}{5}{R}{}{}    % OR
    \gate[\inputs{2}]{nand}{12}{3}{R}{}{} % NAND
    \pin{1}{1}{L}{A}    % pin A
    \pin{1}{3}{L}{B}    % pin B
    \pin{1}{5}{L}{B1}   % pin B1
    \pin{1}{7}{L}{B2}   % pin B2
    \wire{2}{1}{9}{1}   % wire from pin A
    \pin{17}{3}{R}{Y}   % pin Y

\end{circuitdiagram}
 %%  ************    LibreSilicon's StdCellLibrary   *******************
%%
%%  Organisation:   Chipforge
%%                  Germany / European Union
%%
%%  Profile:        Chipforge focus on fine System-on-Chip Cores in
%%                  Verilog HDL Code which are easy understandable and
%%                  adjustable. For further information see
%%                          www.chipforge.org
%%                  there are projects from small cores up to PCBs, too.
%%
%%  File:           StdCellLib/Documents/Datasheets/Circuitry/OA31.tex
%%
%%  Purpose:        Circuit File for OA31
%%
%%  ************    LaTeX with circdia.sty package      ***************
%%
%%  ///////////////////////////////////////////////////////////////////
%%
%%  Copyright (c) 2018 - 2022 by
%%                  chipforge <stdcelllib@nospam.chipforge.org>
%%  All rights reserved.
%%
%%      This Standard Cell Library is licensed under the Libre Silicon
%%      public license; you can redistribute it and/or modify it under
%%      the terms of the Libre Silicon public license as published by
%%      the Libre Silicon alliance, either version 1 of the License, or
%%      (at your option) any later version.
%%
%%      This design is distributed in the hope that it will be useful,
%%      but WITHOUT ANY WARRANTY; without even the implied warranty of
%%      MERCHANTABILITY or FITNESS FOR A PARTICULAR PURPOSE.
%%      See the Libre Silicon Public License for more details.
%%
%%  ///////////////////////////////////////////////////////////////////
\begin{circuitdiagram}[draft]{24}{8}

    \usgate
    % ----  1st column  ----
    \pin{1}{1}{L}{A}
    \pin{1}{3}{L}{A1}
    \pin{1}{5}{L}{A2}
    \gate[\inputs{3}]{or}{5}{3}{R}{}{}

    % ----  2nd column  ----
    \pin{8}{7}{L}{B}
    \gate[\inputs{2}]{nand}{12}{5}{R}{}{}

    % ----  3rd column  ----
    \gate{not}{19}{5}{R}{}{}

    % ----  result ----
    \pin{23}{5}{R}{Z}

\end{circuitdiagram}

%%  ************    LibreSilicon's StdCellLibrary   *******************
%%
%%  Organisation:   Chipforge
%%                  Germany / European Union
%%
%%  Profile:        Chipforge focus on fine System-on-Chip Cores in
%%                  Verilog HDL Code which are easy understandable and
%%                  adjustable. For further information see
%%                          www.chipforge.org
%%                  there are projects from small cores up to PCBs, too.
%%
%%  File:           StdCellLib/Documents/Circuits/OAI32.tex
%%
%%  Purpose:        Circuit File for OAI32
%%
%%  ************    LaTeX with circdia.sty package      ***************
%%
%%  ///////////////////////////////////////////////////////////////////
%%
%%  Copyright (c) 2019 by chipforge <stdcelllib@nospam.chipforge.org>
%%  All rights reserved.
%%
%%      This Standard Cell Library is licensed under the Libre Silicon
%%      public license; you can redistribute it and/or modify it under
%%      the terms of the Libre Silicon public license as published by
%%      the Libre Silicon alliance, either version 1 of the License, or
%%      (at your option) any later version.
%%
%%      This design is distributed in the hope that it will be useful,
%%      but WITHOUT ANY WARRANTY; without even the implied warranty of
%%      MERCHANTABILITY or FITNESS FOR A PARTICULAR PURPOSE.
%%      See the Libre Silicon Public License for more details.
%%
%%  ///////////////////////////////////////////////////////////////////
\begin{circuitdiagram}{18}{10}

    \usgate
    \gate[\inputs{3}]{or}{5}{7}{R}{}{}    % OR
    \gate[\inputs{3}]{nand}{12}{3}{R}{}{} % NAND
    \pin{1}{1}{L}{A}    % pin A
    \pin{1}{3}{L}{A1}   % pin A1
    \pin{1}{5}{L}{B}    % pin B
    \pin{1}{7}{L}{B1}   % pin B1
    \pin{1}{9}{L}{B2}   % pin B2
    \wire{9}{5}{9}{7}   % wire between OR and NAND
    \wire{2}{1}{9}{1}   % wire from pin A
    \wire{2}{3}{9}{3}   % wire from pin A1
    \pin{17}{3}{R}{Y}   % pin Y

\end{circuitdiagram}
 %%  ************    LibreSilicon's StdCellLibrary   *******************
%%
%%  Organisation:   Chipforge
%%                  Germany / European Union
%%
%%  Profile:        Chipforge focus on fine System-on-Chip Cores in
%%                  Verilog HDL Code which are easy understandable and
%%                  adjustable. For further information see
%%                          www.chipforge.org
%%                  there are projects from small cores up to PCBs, too.
%%
%%  File:           StdCellLib/Documents/Datasheets/Circuitry/OA32.tex
%%
%%  Purpose:        Circuit File for OA32
%%
%%  ************    LaTeX with circdia.sty package      ***************
%%
%%  ///////////////////////////////////////////////////////////////////
%%
%%  Copyright (c) 2018 - 2022 by
%%                  chipforge <stdcelllib@nospam.chipforge.org>
%%  All rights reserved.
%%
%%      This Standard Cell Library is licensed under the Libre Silicon
%%      public license; you can redistribute it and/or modify it under
%%      the terms of the Libre Silicon public license as published by
%%      the Libre Silicon alliance, either version 1 of the License, or
%%      (at your option) any later version.
%%
%%      This design is distributed in the hope that it will be useful,
%%      but WITHOUT ANY WARRANTY; without even the implied warranty of
%%      MERCHANTABILITY or FITNESS FOR A PARTICULAR PURPOSE.
%%      See the Libre Silicon Public License for more details.
%%
%%  ///////////////////////////////////////////////////////////////////
\begin{circuitdiagram}[draft]{24}{10}

    \usgate
    % ----  1st column  ----
    \pin{1}{1}{L}{A}
    \pin{1}{3}{L}{A1}
    \pin{1}{5}{L}{A2}
    \gate[\inputs{3}]{or}{5}{3}{R}{}{}

    % ----  2nd column  ----
    \pin{8}{7}{L}{B}
    \pin{8}{9}{L}{B1}
    \wire{9}{3}{9}{5}
    \gate[\inputs{3}]{nand}{12}{7}{R}{}{}

    % ----  3rd column  ----
    \gate{not}{19}{7}{R}{}{}

    % ----  result ----
    \pin{23}{7}{R}{Z}

\end{circuitdiagram}

%%  ************    LibreSilicon's StdCellLibrary   *******************
%%
%%  Organisation:   Chipforge
%%                  Germany / European Union
%%
%%  Profile:        Chipforge focus on fine System-on-Chip Cores in
%%                  Verilog HDL Code which are easy understandable and
%%                  adjustable. For further information see
%%                          www.chipforge.org
%%                  there are projects from small cores up to PCBs, too.
%%
%%  File:           StdCellLib/Documents/Circuits/OAI23.tex
%%
%%  Purpose:        Circuit File for OAI23
%%
%%  ************    LaTeX with circdia.sty package      ***************
%%
%%  ///////////////////////////////////////////////////////////////////
%%
%%  Copyright (c) 2019 by chipforge <stdcelllib@nospam.chipforge.org>
%%  All rights reserved.
%%
%%      This Standard Cell Library is licensed under the Libre Silicon
%%      public license; you can redistribute it and/or modify it under
%%      the terms of the Libre Silicon public license as published by
%%      the Libre Silicon alliance, either version 1 of the License, or
%%      (at your option) any later version.
%%
%%      This design is distributed in the hope that it will be useful,
%%      but WITHOUT ANY WARRANTY; without even the implied warranty of
%%      MERCHANTABILITY or FITNESS FOR A PARTICULAR PURPOSE.
%%      See the Libre Silicon Public License for more details.
%%
%%  ///////////////////////////////////////////////////////////////////
\begin{circuitdiagram}{18}{12}

    \usgate
    \gate[\inputs{3}]{or}{5}{9}{R}{}{}    % OR
    \gate[\inputs{4}]{nand}{12}{4}{R}{}{} % NAND
    \pin{1}{1}{L}{A}    % pin A
    \pin{1}{3}{L}{A1}   % pin A1
    \pin{1}{5}{L}{A2}   % pin A2
    \pin{1}{7}{L}{B}    % pin B
    \pin{1}{9}{L}{B1}   % pin B1
    \pin{1}{11}{L}{B2}  % pin B2
    \wire{2}{1}{9}{1}   % wire from pin A
    \wire{2}{3}{9}{3}   % wire from pin A1
    \wire{2}{5}{9}{5}   % wire from pin A2
    \wire{9}{9}{9}{7}   % wire between OR and NAND
    \pin{17}{4}{R}{Y}   % pin Y

\end{circuitdiagram}
 %%  ************    LibreSilicon's StdCellLibrary   *******************
%%
%%  Organisation:   Chipforge
%%                  Germany / European Union
%%
%%  Profile:        Chipforge focus on fine System-on-Chip Cores in
%%                  Verilog HDL Code which are easy understandable and
%%                  adjustable. For further information see
%%                          www.chipforge.org
%%                  there are projects from small cores up to PCBs, too.
%%
%%  File:           StdCellLib/Documents/Datasheets/Circuitry/OA33.tex
%%
%%  Purpose:        Circuit File for OA33
%%
%%  ************    LaTeX with circdia.sty package      ***************
%%
%%  ///////////////////////////////////////////////////////////////////
%%
%%  Copyright (c) 2018 - 2022 by
%%                  chipforge <stdcelllib@nospam.chipforge.org>
%%  All rights reserved.
%%
%%      This Standard Cell Library is licensed under the Libre Silicon
%%      public license; you can redistribute it and/or modify it under
%%      the terms of the Libre Silicon public license as published by
%%      the Libre Silicon alliance, either version 1 of the License, or
%%      (at your option) any later version.
%%
%%      This design is distributed in the hope that it will be useful,
%%      but WITHOUT ANY WARRANTY; without even the implied warranty of
%%      MERCHANTABILITY or FITNESS FOR A PARTICULAR PURPOSE.
%%      See the Libre Silicon Public License for more details.
%%
%%  ///////////////////////////////////////////////////////////////////
\begin{circuitdiagram}[draft]{24}{12}

    \usgate
    % ----  1st column  ----
    \pin{1}{1}{L}{A}
    \pin{1}{3}{L}{A1}
    \pin{1}{5}{L}{A2}
    \gate[\inputs{3}]{or}{5}{3}{R}{}{}

    % ----  2nd column  ----
    \pin{8}{7}{L}{B}
    \pin{8}{9}{L}{B1}
    \pin{8}{11}{L}{B2}
    \wire{9}{3}{9}{5}
    \gate[\inputs{4}]{nand}{12}{8}{R}{}{}

    % ----  3rd column  ----
    \gate{not}{19}{8}{R}{}{}

    % ----  result ----
    \pin{23}{8}{R}{Z}

\end{circuitdiagram}

%%  ************    LibreSilicon's StdCellLibrary   *******************
%%
%%  Organisation:   Chipforge
%%                  Germany / European Union
%%
%%  Profile:        Chipforge focus on fine System-on-Chip Cores in
%%                  Verilog HDL Code which are easy understandable and
%%                  adjustable. For further information see
%%                          www.chipforge.org
%%                  there are projects from small cores up to PCBs, too.
%%
%%  File:           StdCellLib/Documents/Datasheets/Circuitry/OAI41.tex
%%
%%  Purpose:        Circuit File for OAI41
%%
%%  ************    LaTeX with circdia.sty package      ***************
%%
%%  ///////////////////////////////////////////////////////////////////
%%
%%  Copyright (c) 2018 - 2022 by
%%                  chipforge <stdcelllib@nospam.chipforge.org>
%%  All rights reserved.
%%
%%      This Standard Cell Library is licensed under the Libre Silicon
%%      public license; you can redistribute it and/or modify it under
%%      the terms of the Libre Silicon public license as published by
%%      the Libre Silicon alliance, either version 1 of the License, or
%%      (at your option) any later version.
%%
%%      This design is distributed in the hope that it will be useful,
%%      but WITHOUT ANY WARRANTY; without even the implied warranty of
%%      MERCHANTABILITY or FITNESS FOR A PARTICULAR PURPOSE.
%%      See the Libre Silicon Public License for more details.
%%
%%  ///////////////////////////////////////////////////////////////////
\begin{circuitdiagram}[draft]{18}{9}

    \usgate
    % ----  1st column  ----
    \pin{1}{1}{L}{A}
    \pin{1}{3}{L}{A1}
    \pin{1}{5}{L}{A2}
    \pin{1}{7}{L}{A3}
    \gate[\inputs{4}]{or}{5}{4}{R}{}{}

    % ----  2nd column  ----
    \pin{8}{8}{L}{B}
    \gate[\inputs{2}]{nand}{12}{6}{R}{}{}

    % ----  result ----
    \pin{17}{6}{R}{Y}

\end{circuitdiagram}
 %%  ************    LibreSilicon's StdCellLibrary   *******************
%%
%%  Organisation:   Chipforge
%%                  Germany / European Union
%%
%%  Profile:        Chipforge focus on fine System-on-Chip Cores in
%%                  Verilog HDL Code which are easy understandable and
%%                  adjustable. For further information see
%%                          www.chipforge.org
%%                  there are projects from small cores up to PCBs, too.
%%
%%  File:           StdCellLib/Documents/Circuits/OA41.tex
%%
%%  Purpose:        Circuit File for OA41
%%
%%  ************    LaTeX with circdia.sty package      ***************
%%
%%  ///////////////////////////////////////////////////////////////////
%%
%%  Copyright (c) 2019 by chipforge <stdcelllib@nospam.chipforge.org>
%%  All rights reserved.
%%
%%      This Standard Cell Library is licensed under the Libre Silicon
%%      public license; you can redistribute it and/or modify it under
%%      the terms of the Libre Silicon public license as published by
%%      the Libre Silicon alliance, either version 1 of the License, or
%%      (at your option) any later version.
%%
%%      This design is distributed in the hope that it will be useful,
%%      but WITHOUT ANY WARRANTY; without even the implied warranty of
%%      MERCHANTABILITY or FITNESS FOR A PARTICULAR PURPOSE.
%%      See the Libre Silicon Public License for more details.
%%
%%  ///////////////////////////////////////////////////////////////////
\begin{center}
    Circuit
    \begin{figure}[h]
        \begin{center}
            \begin{circuitdiagram}{24}{10}
            \usgate
            \gate[\inputs{4}]{or}{5}{6}{R}{}{}    % OR
            \gate[\inputs{2}]{nand}{12}{3}{R}{}{} % NAND
            \gate{not}{19}{3}{R}{}{} % NOT
            \pin{1}{1}{L}{A}    % pin A
            \pin{1}{3}{L}{B}    % pin B
            \pin{1}{5}{L}{B1}   % pin B1
            \pin{1}{7}{L}{B2}   % pin B2
            \pin{1}{9}{L}{B3}   % pin B3
            \wire{9}{5}{9}{6}   % wire between OR and NAND
            \wire{2}{1}{9}{1}   % wire from pin A
            \pin{23}{3}{R}{Z}   % pin Z
            \end{circuitdiagram}
        \end{center}
    \end{figure}
\end{center}

%%  ************    LibreSilicon's StdCellLibrary   *******************
%%
%%  Organisation:   Chipforge
%%                  Germany / European Union
%%
%%  Profile:        Chipforge focus on fine System-on-Chip Cores in
%%                  Verilog HDL Code which are easy understandable and
%%                  adjustable. For further information see
%%                          www.chipforge.org
%%                  there are projects from small cores up to PCBs, too.
%%
%%  File:           StdCellLib/Documents/Datasheets/Circuitry/OAI42.tex
%%
%%  Purpose:        Circuit File for OAI42
%%
%%  ************    LaTeX with circdia.sty package      ***************
%%
%%  ///////////////////////////////////////////////////////////////////
%%
%%  Copyright (c) 2018 - 2022 by
%%                  chipforge <stdcelllib@nospam.chipforge.org>
%%  All rights reserved.
%%
%%      This Standard Cell Library is licensed under the Libre Silicon
%%      public license; you can redistribute it and/or modify it under
%%      the terms of the Libre Silicon public license as published by
%%      the Libre Silicon alliance, either version 1 of the License, or
%%      (at your option) any later version.
%%
%%      This design is distributed in the hope that it will be useful,
%%      but WITHOUT ANY WARRANTY; without even the implied warranty of
%%      MERCHANTABILITY or FITNESS FOR A PARTICULAR PURPOSE.
%%      See the Libre Silicon Public License for more details.
%%
%%  ///////////////////////////////////////////////////////////////////
\begin{circuitdiagram}[draft]{18}{11}

    \usgate
    % ----  1st column  ----
    \pin{1}{1}{L}{A}
    \pin{1}{3}{L}{A1}
    \pin{1}{5}{L}{A2}
    \pin{1}{7}{L}{A3}
    \gate[\inputs{4}]{or}{5}{4}{R}{}{}

    % ----  2nd column  ----
    \pin{8}{8}{L}{B}
    \pin{8}{10}{L}{B1}
    \wire{9}{4}{9}{6}
    \gate[\inputs{3}]{nand}{12}{8}{R}{}{}

    % ----  result ----
    \pin{17}{8}{R}{Y}

\end{circuitdiagram}
 %%  ************    LibreSilicon's StdCellLibrary   *******************
%%
%%  Organisation:   Chipforge
%%                  Germany / European Union
%%
%%  Profile:        Chipforge focus on fine System-on-Chip Cores in
%%                  Verilog HDL Code which are easy understandable and
%%                  adjustable. For further information see
%%                          www.chipforge.org
%%                  there are projects from small cores up to PCBs, too.
%%
%%  File:           StdCellLib/Documents/Datasheets/Circuitry/OA42.tex
%%
%%  Purpose:        Circuit File for OA42
%%
%%  ************    LaTeX with circdia.sty package      ***************
%%
%%  ///////////////////////////////////////////////////////////////////
%%
%%  Copyright (c) 2018 - 2022 by
%%                  chipforge <stdcelllib@nospam.chipforge.org>
%%  All rights reserved.
%%
%%      This Standard Cell Library is licensed under the Libre Silicon
%%      public license; you can redistribute it and/or modify it under
%%      the terms of the Libre Silicon public license as published by
%%      the Libre Silicon alliance, either version 1 of the License, or
%%      (at your option) any later version.
%%
%%      This design is distributed in the hope that it will be useful,
%%      but WITHOUT ANY WARRANTY; without even the implied warranty of
%%      MERCHANTABILITY or FITNESS FOR A PARTICULAR PURPOSE.
%%      See the Libre Silicon Public License for more details.
%%
%%  ///////////////////////////////////////////////////////////////////
\begin{circuitdiagram}[draft]{24}{11}

    \usgate
    % ----  1st column  ----
    \pin{1}{1}{L}{A}
    \pin{1}{3}{L}{A1}
    \pin{1}{5}{L}{A2}
    \pin{1}{7}{L}{A3}
    \gate[\inputs{4}]{or}{5}{4}{R}{}{}

    % ----  2nd column  ----
    \pin{8}{8}{L}{B}
    \pin{8}{10}{L}{B1}
    \wire{9}{4}{9}{6}
    \gate[\inputs{3}]{nand}{12}{8}{R}{}{}

    % ----  3rd column  ----
    \gate{not}{19}{8}{R}{}{}

    % ----  result ----
    \pin{23}{8}{R}{Z}

\end{circuitdiagram}

%%  ************    LibreSilicon's StdCellLibrary   *******************
%%
%%  Organisation:   Chipforge
%%                  Germany / European Union
%%
%%  Profile:        Chipforge focus on fine System-on-Chip Cores in
%%                  Verilog HDL Code which are easy understandable and
%%                  adjustable. For further information see
%%                          www.chipforge.org
%%                  there are projects from small cores up to PCBs, too.
%%
%%  File:           StdCellLib/Documents/Circuits/OAI23.tex
%%
%%  Purpose:        Circuit File for OAI23
%%
%%  ************    LaTeX with circdia.sty package      ***************
%%
%%  ///////////////////////////////////////////////////////////////////
%%
%%  Copyright (c) 2019 by chipforge <stdcelllib@nospam.chipforge.org>
%%  All rights reserved.
%%
%%      This Standard Cell Library is licensed under the Libre Silicon
%%      public license; you can redistribute it and/or modify it under
%%      the terms of the Libre Silicon public license as published by
%%      the Libre Silicon alliance, either version 1 of the License, or
%%      (at your option) any later version.
%%
%%      This design is distributed in the hope that it will be useful,
%%      but WITHOUT ANY WARRANTY; without even the implied warranty of
%%      MERCHANTABILITY or FITNESS FOR A PARTICULAR PURPOSE.
%%      See the Libre Silicon Public License for more details.
%%
%%  ///////////////////////////////////////////////////////////////////
\begin{circuitdiagram}{18}{14}

    \usgate
    \gate[\inputs{4}]{or}{5}{10}{R}{}{}   % OR
    \gate[\inputs{4}]{nand}{12}{4}{R}{}{} % NAND
    \pin{1}{1}{L}{A}    % pin A
    \pin{1}{3}{L}{A1}   % pin A1
    \pin{1}{5}{L}{A2}   % pin A2
    \pin{1}{7}{L}{B}    % pin B
    \pin{1}{9}{L}{B1}   % pin B1
    \pin{1}{11}{L}{B2}  % pin B2
    \pin{1}{13}{L}{B3}  % pin B3
    \wire{2}{1}{9}{1}   % wire from pin A
    \wire{2}{3}{9}{3}   % wire from pin A1
    \wire{2}{5}{9}{5}   % wire from pin A2
    \wire{9}{7}{9}{10}   % wire between OR and NAND
    \pin{17}{4}{R}{Y}   % pin Y

\end{circuitdiagram}
 %%  ************    LibreSilicon's StdCellLibrary   *******************
%%
%%  Organisation:   Chipforge
%%                  Germany / European Union
%%
%%  Profile:        Chipforge focus on fine System-on-Chip Cores in
%%                  Verilog HDL Code which are easy understandable and
%%                  adjustable. For further information see
%%                          www.chipforge.org
%%                  there are projects from small cores up to PCBs, too.
%%
%%  File:           StdCellLib/Documents/Datasheets/Circuitry/OA43.tex
%%
%%  Purpose:        Circuit File for OA43
%%
%%  ************    LaTeX with circdia.sty package      ***************
%%
%%  ///////////////////////////////////////////////////////////////////
%%
%%  Copyright (c) 2018 - 2022 by
%%                  chipforge <stdcelllib@nospam.chipforge.org>
%%  All rights reserved.
%%
%%      This Standard Cell Library is licensed under the Libre Silicon
%%      public license; you can redistribute it and/or modify it under
%%      the terms of the Libre Silicon public license as published by
%%      the Libre Silicon alliance, either version 1 of the License, or
%%      (at your option) any later version.
%%
%%      This design is distributed in the hope that it will be useful,
%%      but WITHOUT ANY WARRANTY; without even the implied warranty of
%%      MERCHANTABILITY or FITNESS FOR A PARTICULAR PURPOSE.
%%      See the Libre Silicon Public License for more details.
%%
%%  ///////////////////////////////////////////////////////////////////
\begin{circuitdiagram}[draft]{24}{13}

    \usgate
    % ----  1st column  ----
    \pin{1}{1}{L}{A}
    \pin{1}{3}{L}{A1}
    \pin{1}{5}{L}{A2}
    \pin{1}{7}{L}{A3}
    \gate[\inputs{4}]{or}{5}{4}{R}{}{}

    % ----  2nd column  ----
    \pin{8}{8}{L}{B}
    \pin{8}{10}{L}{B1}
    \pin{8}{12}{L}{B2}
    \wire{9}{4}{9}{6}
    \gate[\inputs{4}]{nand}{12}{9}{R}{}{}

    % ----  3rd column  ----
    \gate{not}{19}{9}{R}{}{}

    % ----  result ----
    \pin{23}{9}{R}{Z}

\end{circuitdiagram}


\section{AND-AND-OR(-Invert) Complex Gates}

\include{AAOI22_datasheet} \include{AAO22_datasheet}
\include{AAOI32_datasheet} \include{AAO32_datasheet}
\include{AAOI33_datasheet} \include{AAO33_datasheet}
\include{AAOI42_datasheet} \include{AAO42_datasheet}
\include{AAOI43_datasheet} \include{AAO43_datasheet}
\include{AAOI44_datasheet} \include{AAO44_datasheet}

\include{AAOI221_datasheet} \include{AAO221_datasheet}
\include{AAOI321_datasheet} \include{AAO321_datasheet}
\include{AAOI331_datasheet} \include{AAO331_datasheet}
\include{AAOI421_datasheet} \include{AAO421_datasheet}
\include{AAOI431_datasheet} \include{AAO431_datasheet}

\section{OR-OR-AND(-Invert) Complex Gates}

\include{OOAI22_datasheet} \include{OOA22_datasheet}
\include{OOAI32_datasheet} \include{OOA32_datasheet}
\include{OOAI33_datasheet} \include{OOA33_datasheet}
\include{OOAI42_datasheet} \include{OOA42_datasheet}
\include{OOAI43_datasheet} \include{OOA43_datasheet}
\include{OOAI44_datasheet} \include{OOA44_datasheet}

\include{OOAI221_datasheet} \include{OOA221_datasheet}
\include{OOAI321_datasheet} \include{OOA321_datasheet}
\include{OOAI331_datasheet} \include{OOA331_datasheet}
\include{OOAI421_datasheet} \include{OOA421_datasheet}
\include{OOAI431_datasheet} \include{OOA431_datasheet}

%%  ************    LibreSilicon's StdCellLibrary   *******************
%%
%%  Organisation:   Chipforge
%%                  Germany / European Union
%%
%%  Profile:        Chipforge focus on fine System-on-Chip Cores in
%%                  Verilog HDL Code which are easy understandable and
%%                  adjustable. For further information see
%%                          www.chipforge.org
%%                  there are projects from small cores up to PCBs, too.
%%
%%  File:           StdCellLib/Documents/section-AAAOI_complex.tex
%%
%%  Purpose:        Section Level File for Standard Cell Library Documentation
%%
%%  ************    LaTeX with circdia.sty package      ***************
%%
%%  ///////////////////////////////////////////////////////////////////
%%
%%  Copyright (c) 2018 - 2022 by
%%                  chipforge <stdcelllib@nospam.chipforge.org>
%%  All rights reserved.
%%
%%      This Standard Cell Library is licensed under the Libre Silicon
%%      public license; you can redistribute it and/or modify it under
%%      the terms of the Libre Silicon public license as published by
%%      the Libre Silicon alliance, either version 1 of the License, or
%%      (at your option) any later version.
%%
%%      This design is distributed in the hope that it will be useful,
%%      but WITHOUT ANY WARRANTY; without even the implied warranty of
%%      MERCHANTABILITY or FITNESS FOR A PARTICULAR PURPOSE.
%%      See the Libre Silicon Public License for more details.
%%
%%  ///////////////////////////////////////////////////////////////////
\section{AND-AND-AND-OR(-Invert) Complex Gates}

%%  ************    LibreSilicon's StdCellLibrary   *******************
%%
%%  Organisation:   Chipforge
%%                  Germany / European Union
%%
%%  Profile:        Chipforge focus on fine System-on-Chip Cores in
%%                  Verilog HDL Code which are easy understandable and
%%                  adjustable. For further information see
%%                          www.chipforge.org
%%                  there are projects from small cores up to PCBs, too.
%%
%%  File:           StdCellLib/Documents/Circuits/AAAOI222.tex
%%
%%  Purpose:        Circuit File for AAAOI222
%%
%%  ************    LaTeX with circdia.sty package      ***************
%%
%%  ///////////////////////////////////////////////////////////////////
%%
%%  Copyright (c) 2019 by chipforge <stdcelllib@nospam.chipforge.org>
%%  All rights reserved.
%%
%%      This Standard Cell Library is licensed under the Libre Silicon
%%      public license; you can redistribute it and/or modify it under
%%      the terms of the Libre Silicon public license as published by
%%      the Libre Silicon alliance, either version 1 of the License, or
%%      (at your option) any later version.
%%
%%      This design is distributed in the hope that it will be useful,
%%      but WITHOUT ANY WARRANTY; without even the implied warranty of
%%      MERCHANTABILITY or FITNESS FOR A PARTICULAR PURPOSE.
%%      See the Libre Silicon Public License for more details.
%%
%%  ///////////////////////////////////////////////////////////////////
\begin{circuitdiagram}{18}{18}

    \usgate
    \gate[\inputs{2}]{and}{5}{3}{R}{}{}    % AND
    \gate[\inputs{2}]{and}{5}{9}{R}{}{}    % AND
    \gate[\inputs{2}]{and}{5}{15}{R}{}{}   % AND
    \gate[\inputs{3}]{nor}{12}{9}{R}{}{}   % NOR
    \pin{1}{1}{L}{A}     % pin A
    \pin{1}{5}{L}{A1}    % pin A1
    \pin{1}{7}{L}{B}     % pin B
    \pin{1}{11}{L}{B1}   % pin B1
    \pin{1}{13}{L}{C}    % pin C
    \pin{1}{17}{L}{C1}   % pin C1
    \wire{9}{3}{9}{7}    % wire between AND and NOR
    \wire{9}{11}{9}{15}  % wire between AND and NOR
    \pin{17}{9}{R}{Y}    % pin Y

\end{circuitdiagram}
 %%  ************    LibreSilicon's StdCellLibrary   *******************
%%
%%  Organisation:   Chipforge
%%                  Germany / European Union
%%
%%  Profile:        Chipforge focus on fine System-on-Chip Cores in
%%                  Verilog HDL Code which are easy understandable and
%%                  adjustable. For further information see
%%                          www.chipforge.org
%%                  there are projects from small cores up to PCBs, too.
%%
%%  File:           StdCellLib/Documents/Datasheets/Circuitry/AAAO222.tex
%%
%%  Purpose:        Circuit File for AAAO222
%%
%%  ************    LaTeX with circdia.sty package      ***************
%%
%%  ///////////////////////////////////////////////////////////////////
%%
%%  Copyright (c) 2018 - 2022 by
%%                  chipforge <stdcelllib@nospam.chipforge.org>
%%  All rights reserved.
%%
%%      This Standard Cell Library is licensed under the Libre Silicon
%%      public license; you can redistribute it and/or modify it under
%%      the terms of the Libre Silicon public license as published by
%%      the Libre Silicon alliance, either version 1 of the License, or
%%      (at your option) any later version.
%%
%%      This design is distributed in the hope that it will be useful,
%%      but WITHOUT ANY WARRANTY; without even the implied warranty of
%%      MERCHANTABILITY or FITNESS FOR A PARTICULAR PURPOSE.
%%      See the Libre Silicon Public License for more details.
%%
%%  ///////////////////////////////////////////////////////////////////
\begin{circuitdiagram}[draft]{24}{18}

    \usgate
    % ----  1st column  ----
    \pin{1}{1}{L}{A}
    \pin{1}{5}{L}{A1}
    \gate[\inputs{2}]{and}{5}{3}{R}{}{}

    \pin{1}{7}{L}{B}
    \pin{1}{11}{L}{B1}
    \gate[\inputs{2}]{and}{5}{9}{R}{}{}

    \pin{1}{13}{L}{C}
    \pin{1}{17}{L}{C1}
    \gate[\inputs{2}]{and}{5}{15}{R}{}{}

    % ----  2nd column  ----
    \wire{9}{3}{9}{7}
    \wire{9}{11}{9}{15}
    \gate[\inputs{3}]{nor}{12}{9}{R}{}{}

    % ----  3rd column  ----
    \gate{not}{19}{9}{R}{}{}

    % ----  result ----
    \pin{23}{9}{R}{Z}

\end{circuitdiagram}

%%  ************    LibreSilicon's StdCellLibrary   *******************
%%
%%  Organisation:   Chipforge
%%                  Germany / European Union
%%
%%  Profile:        Chipforge focus on fine System-on-Chip Cores in
%%                  Verilog HDL Code which are easy understandable and
%%                  adjustable. For further information see
%%                          www.chipforge.org
%%                  there are projects from small cores up to PCBs, too.
%%
%%  File:           StdCellLib/Documents/Circuits/AAAOI322.tex
%%
%%  Purpose:        Circuit File for AAAOI322
%%
%%  ************    LaTeX with circdia.sty package      ***************
%%
%%  ///////////////////////////////////////////////////////////////////
%%
%%  Copyright (c) 2019 by chipforge <stdcelllib@nospam.chipforge.org>
%%  All rights reserved.
%%
%%      This Standard Cell Library is licensed under the Libre Silicon
%%      public license; you can redistribute it and/or modify it under
%%      the terms of the Libre Silicon public license as published by
%%      the Libre Silicon alliance, either version 1 of the License, or
%%      (at your option) any later version.
%%
%%      This design is distributed in the hope that it will be useful,
%%      but WITHOUT ANY WARRANTY; without even the implied warranty of
%%      MERCHANTABILITY or FITNESS FOR A PARTICULAR PURPOSE.
%%      See the Libre Silicon Public License for more details.
%%
%%  ///////////////////////////////////////////////////////////////////
\begin{circuitdiagram}{18}{18}

    \usgate
    \gate[\inputs{2}]{and}{5}{3}{R}{}{}    % AND
    \gate[\inputs{2}]{and}{5}{9}{R}{}{}    % AND
    \gate[\inputs{3}]{and}{5}{15}{R}{}{}   % AND
    \gate[\inputs{3}]{nor}{12}{9}{R}{}{}   % NOR
    \pin{1}{1}{L}{A}     % pin A
    \pin{1}{5}{L}{A1}    % pin A1
    \pin{1}{7}{L}{B}     % pin B
    \pin{1}{11}{L}{B1}   % pin B1
    \pin{1}{13}{L}{C}    % pin C
    \pin{1}{15}{L}{C1}   % pin C1
    \pin{1}{17}{L}{C2}   % pin C2
    \wire{9}{3}{9}{7}     % wire between AND and NOR
    \wire{9}{11}{9}{15}   % wire between AND and NOR
    \pin{17}{9}{R}{Y}   % pin Y

\end{circuitdiagram}
 %%  ************    LibreSilicon's StdCellLibrary   *******************
%%
%%  Organisation:   Chipforge
%%                  Germany / European Union
%%
%%  Profile:        Chipforge focus on fine System-on-Chip Cores in
%%                  Verilog HDL Code which are easy understandable and
%%                  adjustable. For further information see
%%                          www.chipforge.org
%%                  there are projects from small cores up to PCBs, too.
%%
%%  File:           StdCellLib/Documents/Datasheets/Circuitry/AAAO322.tex
%%
%%  Purpose:        Circuit File for AAAO322
%%
%%  ************    LaTeX with circdia.sty package      ***************
%%
%%  ///////////////////////////////////////////////////////////////////
%%
%%  Copyright (c) 2018 - 2022 by
%%                  chipforge <stdcelllib@nospam.chipforge.org>
%%  All rights reserved.
%%
%%      This Standard Cell Library is licensed under the Libre Silicon
%%      public license; you can redistribute it and/or modify it under
%%      the terms of the Libre Silicon public license as published by
%%      the Libre Silicon alliance, either version 1 of the License, or
%%      (at your option) any later version.
%%
%%      This design is distributed in the hope that it will be useful,
%%      but WITHOUT ANY WARRANTY; without even the implied warranty of
%%      MERCHANTABILITY or FITNESS FOR A PARTICULAR PURPOSE.
%%      See the Libre Silicon Public License for more details.
%%
%%  ///////////////////////////////////////////////////////////////////
\begin{circuitdiagram}[draft]{24}{18}

    \usgate
    % ----  1st column  ----
    \pin{1}{1}{L}{A}
    \pin{1}{3}{L}{A1}
    \pin{1}{5}{L}{A2}
    \gate[\inputs{3}]{and}{5}{3}{R}{}{}

    \pin{1}{7}{L}{B}
    \pin{1}{11}{L}{B1}
    \gate[\inputs{2}]{and}{5}{9}{R}{}{}

    \pin{1}{13}{L}{C}
    \pin{1}{17}{L}{C1}
    \gate[\inputs{2}]{and}{5}{15}{R}{}{}

    % ----  2nd column  ----
    \wire{9}{3}{9}{7}
    \wire{9}{11}{9}{15}
    \gate[\inputs{3}]{nor}{12}{9}{R}{}{}

    % ----  3rd column  ----
    \gate{not}{19}{9}{R}{}{}

    % ----  result ----
    \pin{23}{9}{R}{Z}

\end{circuitdiagram}

%%  ************    LibreSilicon's StdCellLibrary   *******************
%%
%%  Organisation:   Chipforge
%%                  Germany / European Union
%%
%%  Profile:        Chipforge focus on fine System-on-Chip Cores in
%%                  Verilog HDL Code which are easy understandable and
%%                  adjustable. For further information see
%%                          www.chipforge.org
%%                  there are projects from small cores up to PCBs, too.
%%
%%  File:           StdCellLib/Documents/Datasheets/Circuitry/AAAOI332.tex
%%
%%  Purpose:        Circuit File for AAAOI322
%%
%%  ************    LaTeX with circdia.sty package      ***************
%%
%%  ///////////////////////////////////////////////////////////////////
%%
%%  Copyright (c) 2018 - 2022 by
%%                  chipforge <stdcelllib@nospam.chipforge.org>
%%  All rights reserved.
%%
%%      This Standard Cell Library is licensed under the Libre Silicon
%%      public license; you can redistribute it and/or modify it under
%%      the terms of the Libre Silicon public license as published by
%%      the Libre Silicon alliance, either version 1 of the License, or
%%      (at your option) any later version.
%%
%%      This design is distributed in the hope that it will be useful,
%%      but WITHOUT ANY WARRANTY; without even the implied warranty of
%%      MERCHANTABILITY or FITNESS FOR A PARTICULAR PURPOSE.
%%      See the Libre Silicon Public License for more details.
%%
%%  ///////////////////////////////////////////////////////////////////
\begin{circuitdiagram}[draft]{18}{18}

    \usgate
    % ----  1st column  ----
    \pin{1}{1}{L}{A}
    \pin{1}{3}{L}{A1}
    \pin{1}{5}{L}{A2}
    \gate[\inputs{3}]{and}{5}{3}{R}{}{}

    \pin{1}{7}{L}{B}
    \pin{1}{9}{L}{B1}
    \pin{1}{11}{L}{B2}
    \gate[\inputs{3}]{and}{5}{9}{R}{}{}

    \pin{1}{13}{L}{C}
    \pin{1}{17}{L}{C1}
    \gate[\inputs{2}]{and}{5}{15}{R}{}{}

    % ----  2nd column  ----
    \wire{9}{3}{9}{7}
    \wire{9}{11}{9}{15}
    \gate[\inputs{3}]{nor}{12}{9}{R}{}{}

    % ----  result ----
    \pin{17}{9}{R}{Y}

\end{circuitdiagram}
 %%  ************    LibreSilicon's StdCellLibrary   *******************
%%
%%  Organisation:   Chipforge
%%                  Germany / European Union
%%
%%  Profile:        Chipforge focus on fine System-on-Chip Cores in
%%                  Verilog HDL Code which are easy understandable and
%%                  adjustable. For further information see
%%                          www.chipforge.org
%%                  there are projects from small cores up to PCBs, too.
%%
%%  File:           StdCellLib/Documents/Datasheets/Circuitry/AAAO332.tex
%%
%%  Purpose:        Circuit File for AAAO332
%%
%%  ************    LaTeX with circdia.sty package      ***************
%%
%%  ///////////////////////////////////////////////////////////////////
%%
%%  Copyright (c) 2018 - 2022 by
%%                  chipforge <stdcelllib@nospam.chipforge.org>
%%  All rights reserved.
%%
%%      This Standard Cell Library is licensed under the Libre Silicon
%%      public license; you can redistribute it and/or modify it under
%%      the terms of the Libre Silicon public license as published by
%%      the Libre Silicon alliance, either version 1 of the License, or
%%      (at your option) any later version.
%%
%%      This design is distributed in the hope that it will be useful,
%%      but WITHOUT ANY WARRANTY; without even the implied warranty of
%%      MERCHANTABILITY or FITNESS FOR A PARTICULAR PURPOSE.
%%      See the Libre Silicon Public License for more details.
%%
%%  ///////////////////////////////////////////////////////////////////
\begin{circuitdiagram}[draft]{24}{18}

    \usgate
    % ----  1st column  ----
    \pin{1}{1}{L}{A}
    \pin{1}{3}{L}{A1}
    \pin{1}{5}{L}{A2}
    \gate[\inputs{3}]{and}{5}{3}{R}{}{}

    \pin{1}{7}{L}{B}
    \pin{1}{9}{L}{B1}
    \pin{1}{11}{L}{B2}
    \gate[\inputs{3}]{and}{5}{9}{R}{}{}

    \pin{1}{13}{L}{C}
    \pin{1}{17}{L}{C1}
    \gate[\inputs{2}]{and}{5}{15}{R}{}{}

    % ----  2nd column  ----
    \wire{9}{3}{9}{7}
    \wire{9}{11}{9}{15}
    \gate[\inputs{3}]{nor}{12}{9}{R}{}{}

    % ----  3rd column  ----
    \gate{not}{19}{9}{R}{}{}

    % ----  result ----
    \pin{23}{9}{R}{Z}

\end{circuitdiagram}

%%  ************    LibreSilicon's StdCellLibrary   *******************
%%
%%  Organisation:   Chipforge
%%                  Germany / European Union
%%
%%  Profile:        Chipforge focus on fine System-on-Chip Cores in
%%                  Verilog HDL Code which are easy understandable and
%%                  adjustable. For further information see
%%                          www.chipforge.org
%%                  there are projects from small cores up to PCBs, too.
%%
%%  File:           StdCellLib/Documents/Circuits/AAAOI333.tex
%%
%%  Purpose:        Circuit File for AAAOI333
%%
%%  ************    LaTeX with circdia.sty package      ***************
%%
%%  ///////////////////////////////////////////////////////////////////
%%
%%  Copyright (c) 2019 by chipforge <stdcelllib@nospam.chipforge.org>
%%  All rights reserved.
%%
%%      This Standard Cell Library is licensed under the Libre Silicon
%%      public license; you can redistribute it and/or modify it under
%%      the terms of the Libre Silicon public license as published by
%%      the Libre Silicon alliance, either version 1 of the License, or
%%      (at your option) any later version.
%%
%%      This design is distributed in the hope that it will be useful,
%%      but WITHOUT ANY WARRANTY; without even the implied warranty of
%%      MERCHANTABILITY or FITNESS FOR A PARTICULAR PURPOSE.
%%      See the Libre Silicon Public License for more details.
%%
%%  ///////////////////////////////////////////////////////////////////
\begin{circuitdiagram}{18}{18}

    \usgate
    \gate[\inputs{3}]{and}{5}{3}{R}{}{}    % AND
    \gate[\inputs{3}]{and}{5}{9}{R}{}{}    % AND
    \gate[\inputs{3}]{and}{5}{15}{R}{}{}   % AND
    \gate[\inputs{3}]{nor}{12}{9}{R}{}{}   % NOR
    \pin{1}{1}{L}{A}    % pin A
    \pin{1}{3}{L}{A1}   % pin A1
    \pin{1}{5}{L}{A2}   % pin A2
    \pin{1}{7}{L}{B}    % pin B
    \pin{1}{9}{L}{B1}   % pin B1
    \pin{1}{11}{L}{B2}  % pin B2
    \pin{1}{13}{L}{C}   % pin C
    \pin{1}{15}{L}{C1}  % pin C1
    \pin{1}{17}{L}{C2}  % pin C2
    \wire{9}{3}{9}{7}   % wire between AND and NOR
    \wire{9}{11}{9}{15} % wire between AND and NOR
    \pin{17}{9}{R}{Y}   % pin Y

\end{circuitdiagram}
 %%  ************    LibreSilicon's StdCellLibrary   *******************
%%
%%  Organisation:   Chipforge
%%                  Germany / European Union
%%
%%  Profile:        Chipforge focus on fine System-on-Chip Cores in
%%                  Verilog HDL Code which are easy understandable and
%%                  adjustable. For further information see
%%                          www.chipforge.org
%%                  there are projects from small cores up to PCBs, too.
%%
%%  File:           StdCellLib/Documents/Circuits/AAAO333.tex
%%
%%  Purpose:        Circuit File for AAAO333
%%
%%  ************    LaTeX with circdia.sty package      ***************
%%
%%  ///////////////////////////////////////////////////////////////////
%%
%%  Copyright (c) 2019 by chipforge <stdcelllib@nospam.chipforge.org>
%%  All rights reserved.
%%
%%      This Standard Cell Library is licensed under the Libre Silicon
%%      public license; you can redistribute it and/or modify it under
%%      the terms of the Libre Silicon public license as published by
%%      the Libre Silicon alliance, either version 1 of the License, or
%%      (at your option) any later version.
%%
%%      This design is distributed in the hope that it will be useful,
%%      but WITHOUT ANY WARRANTY; without even the implied warranty of
%%      MERCHANTABILITY or FITNESS FOR A PARTICULAR PURPOSE.
%%      See the Libre Silicon Public License for more details.
%%
%%  ///////////////////////////////////////////////////////////////////
\begin{circuitdiagram}{24}{18}

    \usgate
    \gate[\inputs{3}]{and}{5}{3}{R}{}{}    % AND
    \gate[\inputs{3}]{and}{5}{9}{R}{}{}    % AND
    \gate[\inputs{3}]{and}{5}{15}{R}{}{}   % AND
    \gate[\inputs{3}]{nor}{12}{9}{R}{}{}   % NOR
    \gate{not}{19}{9}{R}{}{}   % NOT
    \pin{1}{1}{L}{A}    % pin A
    \pin{1}{3}{L}{A1}   % pin A1
    \pin{1}{5}{L}{A2}   % pin A2
    \pin{1}{7}{L}{B}    % pin B
    \pin{1}{9}{L}{B1}   % pin B1
    \pin{1}{11}{L}{B2}  % pin B2
    \pin{1}{13}{L}{C}   % pin C
    \pin{1}{15}{L}{C1}  % pin C1
    \pin{1}{17}{L}{C2}  % pin C2
    \wire{9}{3}{9}{7}   % wire between AND and NOR
    \wire{9}{11}{9}{15} % wire between AND and NOR
    \pin{23}{9}{R}{Z}   % pin Z

\end{circuitdiagram}

\include{Datasheets/AAAOI422} \include{Datasheets/AAAO422}
%%  ************    LibreSilicon's StdCellLibrary   *******************
%%
%%  Organisation:   Chipforge
%%                  Germany / European Union
%%
%%  Profile:        Chipforge focus on fine System-on-Chip Cores in
%%                  Verilog HDL Code which are easy understandable and
%%                  adjustable. For further information see
%%                          www.chipforge.org
%%                  there are projects from small cores up to PCBs, too.
%%
%%  File:           StdCellLib/Documents/Circuits/AAAOI432.tex
%%
%%  Purpose:        Circuit File for AAAOI432
%%
%%  ************    LaTeX with circdia.sty package      ***************
%%
%%  ///////////////////////////////////////////////////////////////////
%%
%%  Copyright (c) 2019 by chipforge <stdcelllib@nospam.chipforge.org>
%%  All rights reserved.
%%
%%      This Standard Cell Library is licensed under the Libre Silicon
%%      public license; you can redistribute it and/or modify it under
%%      the terms of the Libre Silicon public license as published by
%%      the Libre Silicon alliance, either version 1 of the License, or
%%      (at your option) any later version.
%%
%%      This design is distributed in the hope that it will be useful,
%%      but WITHOUT ANY WARRANTY; without even the implied warranty of
%%      MERCHANTABILITY or FITNESS FOR A PARTICULAR PURPOSE.
%%      See the Libre Silicon Public License for more details.
%%
%%  ///////////////////////////////////////////////////////////////////
\begin{circuitdiagram}{18}{18}

    \usgate
    \gate[\inputs{2}]{and}{5}{3}{R}{}{}    % AND
    \gate[\inputs{3}]{and}{5}{9}{R}{}{}    % AND
    \gate[\inputs{4}]{and}{5}{16}{R}{}{}   % AND
    \gate[\inputs{3}]{nor}{12}{9}{R}{}{}   % NOR
    \pin{1}{1}{L}{A}     % pin A
    \pin{1}{5}{L}{A1}    % pin A1
    \pin{1}{7}{L}{B}     % pin B
    \pin{1}{9}{L}{B1}    % pin B1
    \pin{1}{11}{L}{B2}   % pin B2
    \pin{1}{13}{L}{C}    % pin C
    \pin{1}{15}{L}{C1}   % pin C1
    \pin{1}{17}{L}{C2}   % pin C2
    \pin{1}{19}{L}{C3}   % pin C3
    \wire{9}{3}{9}{7}    % wire between AND and NOR
    \wire{9}{11}{9}{16}  % wire between AND and NOR
    \pin{17}{9}{R}{Y}   % pin Y

\end{circuitdiagram}
 %%  ************    LibreSilicon's StdCellLibrary   *******************
%%
%%  Organisation:   Chipforge
%%                  Germany / European Union
%%
%%  Profile:        Chipforge focus on fine System-on-Chip Cores in
%%                  Verilog HDL Code which are easy understandable and
%%                  adjustable. For further information see
%%                          www.chipforge.org
%%                  there are projects from small cores up to PCBs, too.
%%
%%  File:           StdCellLib/Documents/Circuits/AAAO432.tex
%%
%%  Purpose:        Circuit File for AAAO432
%%
%%  ************    LaTeX with circdia.sty package      ***************
%%
%%  ///////////////////////////////////////////////////////////////////
%%
%%  Copyright (c) 2019 by chipforge <stdcelllib@nospam.chipforge.org>
%%  All rights reserved.
%%
%%      This Standard Cell Library is licensed under the Libre Silicon
%%      public license; you can redistribute it and/or modify it under
%%      the terms of the Libre Silicon public license as published by
%%      the Libre Silicon alliance, either version 1 of the License, or
%%      (at your option) any later version.
%%
%%      This design is distributed in the hope that it will be useful,
%%      but WITHOUT ANY WARRANTY; without even the implied warranty of
%%      MERCHANTABILITY or FITNESS FOR A PARTICULAR PURPOSE.
%%      See the Libre Silicon Public License for more details.
%%
%%  ///////////////////////////////////////////////////////////////////
\begin{circuitdiagram}{24}{18}

    \usgate
    \gate[\inputs{2}]{and}{5}{3}{R}{}{}    % AND
    \gate[\inputs{3}]{and}{5}{9}{R}{}{}    % AND
    \gate[\inputs{4}]{and}{5}{16}{R}{}{}   % AND
    \gate[\inputs{3}]{nor}{12}{9}{R}{}{}   % NOR
    \gate{not}{19}{9}{R}{}{}   % NOT
    \pin{1}{1}{L}{A}     % pin A
    \pin{1}{5}{L}{A1}    % pin A1
    \pin{1}{7}{L}{B}     % pin B
    \pin{1}{9}{L}{B1}    % pin B1
    \pin{1}{11}{L}{B2}   % pin B2
    \pin{1}{13}{L}{C}    % pin C
    \pin{1}{15}{L}{C1}   % pin C1
    \pin{1}{17}{L}{C2}   % pin C2
    \pin{1}{19}{L}{C3}   % pin C3
    \wire{9}{3}{9}{7}    % wire between AND and NOR
    \wire{9}{11}{9}{16}  % wire between AND and NOR
    \pin{23}{9}{R}{Z}    % pin Z

\end{circuitdiagram}


%%  ************    LibreSilicon's StdCellLibrary   *******************
%%
%%  Organisation:   Chipforge
%%                  Germany / European Union
%%
%%  Profile:        Chipforge focus on fine System-on-Chip Cores in
%%                  Verilog HDL Code which are easy understandable and
%%                  adjustable. For further information see
%%                          www.chipforge.org
%%                  there are projects from small cores up to PCBs, too.
%%
%%  File:           StdCellLib/Documents/Datasheets/Circuitry/AAAOI2221.tex
%%
%%  Purpose:        Circuit File for AAAOI2221
%%
%%  ************    LaTeX with circdia.sty package      ***************
%%
%%  ///////////////////////////////////////////////////////////////////
%%
%%  Copyright (c) 2018 - 2022 by
%%                  chipforge <stdcelllib@nospam.chipforge.org>
%%  All rights reserved.
%%
%%      This Standard Cell Library is licensed under the Libre Silicon
%%      public license; you can redistribute it and/or modify it under
%%      the terms of the Libre Silicon public license as published by
%%      the Libre Silicon alliance, either version 1 of the License, or
%%      (at your option) any later version.
%%
%%      This design is distributed in the hope that it will be useful,
%%      but WITHOUT ANY WARRANTY; without even the implied warranty of
%%      MERCHANTABILITY or FITNESS FOR A PARTICULAR PURPOSE.
%%      See the Libre Silicon Public License for more details.
%%
%%  ///////////////////////////////////////////////////////////////////
\begin{circuitdiagram}[draft]{20}{18}

    \usgate
    % ----  1st column  ----
    \pin{1}{1}{L}{A}
    \pin{1}{5}{L}{A1}
    \gate[\inputs{2}]{and}{5}{3}{R}{}{}

    \pin{1}{7}{L}{B}
    \pin{1}{11}{L}{B1}
    \gate[\inputs{2}]{and}{5}{9}{R}{}{}

    \pin{1}{13}{L}{C}
    \pin{1}{17}{L}{C1}
    \gate[\inputs{2}]{and}{5}{15}{R}{}{}

    % ----  2nd column  ----
    \wire{9}{3}{11}{3}
    \wire{11}{3}{11}{11}
    \wire{9}{9}{9}{13}
    \wire{9}{13}{11}{13}
    \wire{9}{15}{11}{15}
    \pin{10}{17}{L}{D}
    \gate[\inputs{4}]{nor}{14}{14}{R}{}{}

    % ----  result ----
    \pin{19}{14}{R}{Y}

\end{circuitdiagram}
 %%  ************    LibreSilicon's StdCellLibrary   *******************
%%
%%  Organisation:   Chipforge
%%                  Germany / European Union
%%
%%  Profile:        Chipforge focus on fine System-on-Chip Cores in
%%                  Verilog HDL Code which are easy understandable and
%%                  adjustable. For further information see
%%                          www.chipforge.org
%%                  there are projects from small cores up to PCBs, too.
%%
%%  File:           StdCellLib/Documents/Datasheets/Circuitry/AAAO2221.tex
%%
%%  Purpose:        Circuit File for AAAO2221
%%
%%  ************    LaTeX with circdia.sty package      ***************
%%
%%  ///////////////////////////////////////////////////////////////////
%%
%%  Copyright (c) 2018 - 2022 by
%%                  chipforge <stdcelllib@nospam.chipforge.org>
%%  All rights reserved.
%%
%%      This Standard Cell Library is licensed under the Libre Silicon
%%      public license; you can redistribute it and/or modify it under
%%      the terms of the Libre Silicon public license as published by
%%      the Libre Silicon alliance, either version 1 of the License, or
%%      (at your option) any later version.
%%
%%      This design is distributed in the hope that it will be useful,
%%      but WITHOUT ANY WARRANTY; without even the implied warranty of
%%      MERCHANTABILITY or FITNESS FOR A PARTICULAR PURPOSE.
%%      See the Libre Silicon Public License for more details.
%%
%%  ///////////////////////////////////////////////////////////////////
\begin{circuitdiagram}[draft]{26}{18}

    \usgate
    % ----  1st column  ----
    \pin{1}{1}{L}{A}
    \pin{1}{5}{L}{A1}
    \gate[\inputs{2}]{and}{5}{3}{R}{}{}

    \pin{1}{7}{L}{B}
    \pin{1}{11}{L}{B1}
    \gate[\inputs{2}]{and}{5}{9}{R}{}{}

    \pin{1}{13}{L}{C}
    \pin{1}{17}{L}{C1}
    \gate[\inputs{2}]{and}{5}{15}{R}{}{}

    % ----  2nd column  ----
    \wire{9}{3}{11}{3}
    \wire{11}{3}{11}{11}
    \wire{9}{9}{9}{13}
    \wire{9}{13}{11}{13}
    \wire{9}{15}{11}{15}
    \pin{10}{17}{L}{D}
    \gate[\inputs{4}]{nor}{14}{14}{R}{}{}

    % ----  3rd column  ----
    \gate{not}{21}{14}{R}{}{}

    % ----  result ----
    \pin{25}{14}{R}{Z}

\end{circuitdiagram}

%%  ************    LibreSilicon's StdCellLibrary   *******************
%%
%%  Organisation:   Chipforge
%%                  Germany / European Union
%%
%%  Profile:        Chipforge focus on fine System-on-Chip Cores in
%%                  Verilog HDL Code which are easy understandable and
%%                  adjustable. For further information see
%%                          www.chipforge.org
%%                  there are projects from small cores up to PCBs, too.
%%
%%  File:           StdCellLib/Documents/Datasheets/Circuitry/AAAOI3221.tex
%%
%%  Purpose:        Circuit File for AAAOI3221
%%
%%  ************    LaTeX with circdia.sty package      ***************
%%
%%  ///////////////////////////////////////////////////////////////////
%%
%%  Copyright (c) 2018 - 2022 by
%%                  chipforge <stdcelllib@nospam.chipforge.org>
%%  All rights reserved.
%%
%%      This Standard Cell Library is licensed under the Libre Silicon
%%      public license; you can redistribute it and/or modify it under
%%      the terms of the Libre Silicon public license as published by
%%      the Libre Silicon alliance, either version 1 of the License, or
%%      (at your option) any later version.
%%
%%      This design is distributed in the hope that it will be useful,
%%      but WITHOUT ANY WARRANTY; without even the implied warranty of
%%      MERCHANTABILITY or FITNESS FOR A PARTICULAR PURPOSE.
%%      See the Libre Silicon Public License for more details.
%%
%%  ///////////////////////////////////////////////////////////////////
\begin{circuitdiagram}[draft]{20}{18}

    \usgate
    % ----  1st column  ----
    \pin{1}{1}{L}{A}
    \pin{1}{3}{L}{A1}
    \pin{1}{5}{L}{A2}
    \gate[\inputs{3}]{and}{5}{3}{R}{}{}

    \pin{1}{7}{L}{B}
    \pin{1}{11}{L}{B1}
    \gate[\inputs{2}]{and}{5}{9}{R}{}{}

    \pin{1}{13}{L}{C}
    \pin{1}{17}{L}{C1}
    \gate[\inputs{2}]{and}{5}{15}{R}{}{}

    % ----  2nd column  ----
    \wire{9}{3}{11}{3}
    \wire{11}{3}{11}{11}
    \wire{9}{9}{9}{13}
    \wire{9}{13}{11}{13}
    \wire{9}{15}{11}{15}
    \pin{10}{17}{L}{D}
    \gate[\inputs{4}]{nor}{14}{14}{R}{}{}

    % ----  result ----
    \pin{19}{14}{R}{Y}

\end{circuitdiagram}
 %%  ************    LibreSilicon's StdCellLibrary   *******************
%%
%%  Organisation:   Chipforge
%%                  Germany / European Union
%%
%%  Profile:        Chipforge focus on fine System-on-Chip Cores in
%%                  Verilog HDL Code which are easy understandable and
%%                  adjustable. For further information see
%%                          www.chipforge.org
%%                  there are projects from small cores up to PCBs, too.
%%
%%  File:           StdCellLib/Documents/Datasheets/Circuitry/AAAO3221.tex
%%
%%  Purpose:        Circuit File for AAAO3221
%%
%%  ************    LaTeX with circdia.sty package      ***************
%%
%%  ///////////////////////////////////////////////////////////////////
%%
%%  Copyright (c) 2018 - 2022 by
%%                  chipforge <stdcelllib@nospam.chipforge.org>
%%  All rights reserved.
%%
%%      This Standard Cell Library is licensed under the Libre Silicon
%%      public license; you can redistribute it and/or modify it under
%%      the terms of the Libre Silicon public license as published by
%%      the Libre Silicon alliance, either version 1 of the License, or
%%      (at your option) any later version.
%%
%%      This design is distributed in the hope that it will be useful,
%%      but WITHOUT ANY WARRANTY; without even the implied warranty of
%%      MERCHANTABILITY or FITNESS FOR A PARTICULAR PURPOSE.
%%      See the Libre Silicon Public License for more details.
%%
%%  ///////////////////////////////////////////////////////////////////
\begin{circuitdiagram}[draft]{26}{18}

    \usgate
    % ----  1st column  ----
    \pin{1}{1}{L}{A}
    \pin{1}{3}{L}{A1}
    \pin{1}{5}{L}{A2}
    \gate[\inputs{3}]{and}{5}{3}{R}{}{}

    \pin{1}{7}{L}{B}
    \pin{1}{11}{L}{B1}
    \gate[\inputs{2}]{and}{5}{9}{R}{}{}

    \pin{1}{13}{L}{C}
    \pin{1}{17}{L}{C1}
    \gate[\inputs{2}]{and}{5}{15}{R}{}{}

    % ----  2nd column  ----
    \wire{9}{3}{11}{3}
    \wire{11}{3}{11}{11}
    \wire{9}{9}{9}{13}
    \wire{9}{13}{11}{13}
    \wire{9}{15}{11}{15}
    \pin{10}{17}{L}{D}
    \gate[\inputs{4}]{nor}{14}{14}{R}{}{}

    % ----  3rd column  ----
    \gate{not}{21}{14}{R}{}{}

    % ----  result ----
    \pin{25}{14}{R}{Z}

\end{circuitdiagram}

%%  ************    LibreSilicon's StdCellLibrary   *******************
%%
%%  Organisation:   Chipforge
%%                  Germany / European Union
%%
%%  Profile:        Chipforge focus on fine System-on-Chip Cores in
%%                  Verilog HDL Code which are easy understandable and
%%                  adjustable. For further information see
%%                          www.chipforge.org
%%                  there are projects from small cores up to PCBs, too.
%%
%%  File:           StdCellLib/Documents/Datasheets/Circuitry/AAAOI3321.tex
%%
%%  Purpose:        Circuit File for AAAOI3321
%%
%%  ************    LaTeX with circdia.sty package      ***************
%%
%%  ///////////////////////////////////////////////////////////////////
%%
%%  Copyright (c) 2018 - 2022 by
%%                  chipforge <stdcelllib@nospam.chipforge.org>
%%  All rights reserved.
%%
%%      This Standard Cell Library is licensed under the Libre Silicon
%%      public license; you can redistribute it and/or modify it under
%%      the terms of the Libre Silicon public license as published by
%%      the Libre Silicon alliance, either version 1 of the License, or
%%      (at your option) any later version.
%%
%%      This design is distributed in the hope that it will be useful,
%%      but WITHOUT ANY WARRANTY; without even the implied warranty of
%%      MERCHANTABILITY or FITNESS FOR A PARTICULAR PURPOSE.
%%      See the Libre Silicon Public License for more details.
%%
%%  ///////////////////////////////////////////////////////////////////
\begin{circuitdiagram}[draft]{20}{18}

    \usgate
    % ----  1st column  ----
    \pin{1}{1}{L}{A}
    \pin{1}{3}{L}{A1}
    \pin{1}{5}{L}{A2}
    \gate[\inputs{3}]{and}{5}{3}{R}{}{}

    \pin{1}{7}{L}{B}
    \pin{1}{9}{L}{B1}
    \pin{1}{11}{L}{B2}
    \gate[\inputs{3}]{and}{5}{9}{R}{}{}

    \pin{1}{13}{L}{C}
    \pin{1}{17}{L}{C1}
    \gate[\inputs{2}]{and}{5}{15}{R}{}{}

    % ----  2nd column  ----
    \wire{9}{3}{11}{3}
    \wire{11}{3}{11}{11}
    \wire{9}{9}{9}{13}
    \wire{9}{13}{11}{13}
    \wire{9}{15}{11}{15}
    \pin{10}{17}{L}{D}
    \gate[\inputs{4}]{nor}{14}{14}{R}{}{}

    % ----  result ----
    \pin{19}{14}{R}{Y}

\end{circuitdiagram}
 %%  ************    LibreSilicon's StdCellLibrary   *******************
%%
%%  Organisation:   Chipforge
%%                  Germany / European Union
%%
%%  Profile:        Chipforge focus on fine System-on-Chip Cores in
%%                  Verilog HDL Code which are easy understandable and
%%                  adjustable. For further information see
%%                          www.chipforge.org
%%                  there are projects from small cores up to PCBs, too.
%%
%%  File:           StdCellLib/Documents/Datasheets/Circuitry/AAAO3321.tex
%%
%%  Purpose:        Circuit File for AAAO3321
%%
%%  ************    LaTeX with circdia.sty package      ***************
%%
%%  ///////////////////////////////////////////////////////////////////
%%
%%  Copyright (c) 2018 - 2022 by
%%                  chipforge <stdcelllib@nospam.chipforge.org>
%%  All rights reserved.
%%
%%      This Standard Cell Library is licensed under the Libre Silicon
%%      public license; you can redistribute it and/or modify it under
%%      the terms of the Libre Silicon public license as published by
%%      the Libre Silicon alliance, either version 1 of the License, or
%%      (at your option) any later version.
%%
%%      This design is distributed in the hope that it will be useful,
%%      but WITHOUT ANY WARRANTY; without even the implied warranty of
%%      MERCHANTABILITY or FITNESS FOR A PARTICULAR PURPOSE.
%%      See the Libre Silicon Public License for more details.
%%
%%  ///////////////////////////////////////////////////////////////////
\begin{circuitdiagram}[draft]{26}{18}

    \usgate
    % ----  1st column  ----
    \pin{1}{1}{L}{A}
    \pin{1}{3}{L}{A1}
    \pin{1}{5}{L}{A2}
    \gate[\inputs{3}]{and}{5}{3}{R}{}{}

    \pin{1}{7}{L}{B}
    \pin{1}{9}{L}{B1}
    \pin{1}{11}{L}{B2}
    \gate[\inputs{3}]{and}{5}{9}{R}{}{}

    \pin{1}{13}{L}{C}
    \pin{1}{17}{L}{C1}
    \gate[\inputs{2}]{and}{5}{15}{R}{}{}

    % ----  2nd column  ----
    \wire{9}{3}{11}{3}
    \wire{11}{3}{11}{11}
    \wire{9}{9}{9}{13}
    \wire{9}{13}{11}{13}
    \wire{9}{15}{11}{15}
    \pin{10}{17}{L}{D}
    \gate[\inputs{4}]{nor}{14}{14}{R}{}{}

    % ----  3rd column  ----
    \gate{not}{21}{14}{R}{}{}

    % ----  result ----
    \pin{25}{14}{R}{Z}

\end{circuitdiagram}

%%  ************    LibreSilicon's StdCellLibrary   *******************
%%
%%  Organisation:   Chipforge
%%                  Germany / European Union
%%
%%  Profile:        Chipforge focus on fine System-on-Chip Cores in
%%                  Verilog HDL Code which are easy understandable and
%%                  adjustable. For further information see
%%                          www.chipforge.org
%%                  there are projects from small cores up to PCBs, too.
%%
%%  File:           StdCellLib/Documents/Datasheets/Circuitry/AAAOI3331.tex
%%
%%  Purpose:        Circuit File for AAAOI3331
%%
%%  ************    LaTeX with circdia.sty package      ***************
%%
%%  ///////////////////////////////////////////////////////////////////
%%
%%  Copyright (c) 2018 - 2022 by
%%                  chipforge <stdcelllib@nospam.chipforge.org>
%%  All rights reserved.
%%
%%      This Standard Cell Library is licensed under the Libre Silicon
%%      public license; you can redistribute it and/or modify it under
%%      the terms of the Libre Silicon public license as published by
%%      the Libre Silicon alliance, either version 1 of the License, or
%%      (at your option) any later version.
%%
%%      This design is distributed in the hope that it will be useful,
%%      but WITHOUT ANY WARRANTY; without even the implied warranty of
%%      MERCHANTABILITY or FITNESS FOR A PARTICULAR PURPOSE.
%%      See the Libre Silicon Public License for more details.
%%
%%  ///////////////////////////////////////////////////////////////////
\begin{circuitdiagram}[draft]{20}{18}

    \usgate
    % ----  1st column  ----
    \pin{1}{1}{L}{A}
    \pin{1}{3}{L}{A1}
    \pin{1}{5}{L}{A2}
    \gate[\inputs{3}]{and}{5}{3}{R}{}{}

    \pin{1}{7}{L}{B}
    \pin{1}{9}{L}{B1}
    \pin{1}{11}{L}{B2}
    \gate[\inputs{3}]{and}{5}{9}{R}{}{}

    \pin{1}{13}{L}{C}
    \pin{1}{15}{L}{C1}
    \pin{1}{17}{L}{C2}
    \gate[\inputs{3}]{and}{5}{15}{R}{}{}

    % ----  2nd column  ----
    \wire{9}{3}{11}{3}
    \wire{11}{3}{11}{11}
    \wire{9}{9}{9}{13}
    \wire{9}{13}{11}{13}
    \wire{9}{15}{11}{15}
    \pin{10}{17}{L}{D}
    \gate[\inputs{4}]{nor}{14}{14}{R}{}{}

    % ----  result ----
    \pin{19}{14}{R}{Y}

\end{circuitdiagram}
 %%  ************    LibreSilicon's StdCellLibrary   *******************
%%
%%  Organisation:   Chipforge
%%                  Germany / European Union
%%
%%  Profile:        Chipforge focus on fine System-on-Chip Cores in
%%                  Verilog HDL Code which are easy understandable and
%%                  adjustable. For further information see
%%                          www.chipforge.org
%%                  there are projects from small cores up to PCBs, too.
%%
%%  File:           StdCellLib/Documents/Datasheets/Circuitry/AAAO3331.tex
%%
%%  Purpose:        Circuit File for AAAO3331
%%
%%  ************    LaTeX with circdia.sty package      ***************
%%
%%  ///////////////////////////////////////////////////////////////////
%%
%%  Copyright (c) 2018 - 2022 by
%%                  chipforge <stdcelllib@nospam.chipforge.org>
%%  All rights reserved.
%%
%%      This Standard Cell Library is licensed under the Libre Silicon
%%      public license; you can redistribute it and/or modify it under
%%      the terms of the Libre Silicon public license as published by
%%      the Libre Silicon alliance, either version 1 of the License, or
%%      (at your option) any later version.
%%
%%      This design is distributed in the hope that it will be useful,
%%      but WITHOUT ANY WARRANTY; without even the implied warranty of
%%      MERCHANTABILITY or FITNESS FOR A PARTICULAR PURPOSE.
%%      See the Libre Silicon Public License for more details.
%%
%%  ///////////////////////////////////////////////////////////////////
\begin{circuitdiagram}[draft]{26}{18}

    \usgate
    % ----  1st column  ----
    \pin{1}{1}{L}{A}
    \pin{1}{3}{L}{A1}
    \pin{1}{5}{L}{A2}
    \gate[\inputs{3}]{and}{5}{3}{R}{}{}

    \pin{1}{7}{L}{B}
    \pin{1}{9}{L}{B1}
    \pin{1}{11}{L}{B2}
    \gate[\inputs{3}]{and}{5}{9}{R}{}{}

    \pin{1}{13}{L}{C}
    \pin{1}{15}{L}{C1}
    \pin{1}{17}{L}{C2}
    \gate[\inputs{3}]{and}{5}{15}{R}{}{}

    % ----  2nd column  ----
    \wire{9}{3}{11}{3}
    \wire{11}{3}{11}{11}
    \wire{9}{9}{9}{13}
    \wire{9}{13}{11}{13}
    \wire{9}{15}{11}{15}
    \pin{10}{17}{L}{D}
    \gate[\inputs{4}]{nor}{14}{14}{R}{}{}

    % ----  3rd column  ----
    \gate{not}{21}{14}{R}{}{}

    % ----  result ----
    \pin{25}{14}{R}{Z}

\end{circuitdiagram}


\section{OR-OR-OR-AND(-Invert) Complex Gates}

\include{OOOAI222_datasheet} \include{OOOA222_datasheet}
\include{OOOAI322_datasheet} \include{OOOA322_datasheet}
\include{OOOAI332_datasheet} \include{OOOA332_datasheet}
\include{OOOAI333_datasheet} \include{OOOA333_datasheet}
\include{OOOAI432_datasheet} \include{OOOA432_datasheet}


%%  ------------    three phases    -----------------------------------

%%  ************    LibreSilicon's StdCellLibrary   *******************
%%
%%  Organisation:   Chipforge
%%                  Germany / European Union
%%
%%  Profile:        Chipforge focus on fine System-on-Chip Cores in
%%                  Verilog HDL Code which are easy understandable and
%%                  adjustable. For further information see
%%                          www.chipforge.org
%%                  there are projects from small cores up to PCBs, too.
%%
%%  File:           StdCellLib/Documents/section-AOAI_complex.tex
%%
%%  Purpose:        Section Level File for Standard Cell Library Documentation
%%
%%  ************    LaTeX with circdia.sty package      ***************
%%
%%  ///////////////////////////////////////////////////////////////////
%%
%%  Copyright (c) 2018 - 2022 by
%%                  chipforge <stdcelllib@nospam.chipforge.org>
%%  All rights reserved.
%%
%%      This Standard Cell Library is licensed under the Libre Silicon
%%      public license; you can redistribute it and/or modify it under
%%      the terms of the Libre Silicon public license as published by
%%      the Libre Silicon alliance, either version 1 of the License, or
%%      (at your option) any later version.
%%
%%      This design is distributed in the hope that it will be useful,
%%      but WITHOUT ANY WARRANTY; without even the implied warranty of
%%      MERCHANTABILITY or FITNESS FOR A PARTICULAR PURPOSE.
%%      See the Libre Silicon Public License for more details.
%%
%%  ///////////////////////////////////////////////////////////////////
\section{AND-OR-AND(-Invert) Complex Gates}

%%  ************    LibreSilicon's StdCellLibrary   *******************
%%
%%  Organisation:   Chipforge
%%                  Germany / European Union
%%
%%  Profile:        Chipforge focus on fine System-on-Chip Cores in
%%                  Verilog HDL Code which are easy understandable and
%%                  adjustable. For further information see
%%                          www.chipforge.org
%%                  there are projects from small cores up to PCBs, too.
%%
%%  File:           StdCellLib/Documents/Circuits/AOAI211.tex
%%
%%  Purpose:        Circuit File for AOAI211
%%
%%  ************    LaTeX with circdia.sty package      ***************
%%
%%  ///////////////////////////////////////////////////////////////////
%%
%%  Copyright (c) 2019 by chipforge <stdcelllib@nospam.chipforge.org>
%%  All rights reserved.
%%
%%      This Standard Cell Library is licensed under the Libre Silicon
%%      public license; you can redistribute it and/or modify it under
%%      the terms of the Libre Silicon public license as published by
%%      the Libre Silicon alliance, either version 1 of the License, or
%%      (at your option) any later version.
%%
%%      This design is distributed in the hope that it will be useful,
%%      but WITHOUT ANY WARRANTY; without even the implied warranty of
%%      MERCHANTABILITY or FITNESS FOR A PARTICULAR PURPOSE.
%%      See the Libre Silicon Public License for more details.
%%
%%  ///////////////////////////////////////////////////////////////////
\begin{circuitdiagram}{25}{10}

    \usgate
    \gate[\inputs{2}]{and}{5}{7}{R}{}{}    % AND
    \gate[\inputs{2}]{or}{12}{5}{R}{}{}    % OR
    \gate[\inputs{2}]{nand}{19}{3}{R}{}{}  % NAND
    \pin{1}{1}{L}{A}    % pin A
    \pin{1}{3}{L}{B}    % pin B
    \pin{1}{5}{L}{C}    % pin C
    \pin{1}{9}{L}{C1}   % pin C1
    \wire{2}{1}{16}{1}  % wire from pin A
    \wire{2}{3}{9}{3}   % wire from pin C
    \pin{24}{3}{R}{Y}   % pin Y

\end{circuitdiagram}
 %%  ************    LibreSilicon's StdCellLibrary   *******************
%%
%%  Organisation:   Chipforge
%%                  Germany / European Union
%%
%%  Profile:        Chipforge focus on fine System-on-Chip Cores in
%%                  Verilog HDL Code which are easy understandable and
%%                  adjustable. For further information see
%%                          www.chipforge.org
%%                  there are projects from small cores up to PCBs, too.
%%
%%  File:           StdCellLib/Documents/Datasheets/Circuitry/AOA211.tex
%%
%%  Purpose:        Circuit File for AOA211
%%
%%  ************    LaTeX with circdia.sty package      ***************
%%
%%  ///////////////////////////////////////////////////////////////////
%%
%%  Copyright (c) 2018 - 2022 by
%%                  chipforge <stdcelllib@nospam.chipforge.org>
%%  All rights reserved.
%%
%%      This Standard Cell Library is licensed under the Libre Silicon
%%      public license; you can redistribute it and/or modify it under
%%      the terms of the Libre Silicon public license as published by
%%      the Libre Silicon alliance, either version 1 of the License, or
%%      (at your option) any later version.
%%
%%      This design is distributed in the hope that it will be useful,
%%      but WITHOUT ANY WARRANTY; without even the implied warranty of
%%      MERCHANTABILITY or FITNESS FOR A PARTICULAR PURPOSE.
%%      See the Libre Silicon Public License for more details.
%%
%%  ///////////////////////////////////////////////////////////////////
\begin{circuitdiagram}[draft]{31}{10}

    \usgate
    % ----  1st column  ----
    \pin{1}{1}{L}{A}
    \pin{1}{5}{L}{A1}
    \gate[\inputs{2}]{and}{5}{3}{R}{}{}

    % ----  2nd column  ----
    \pin{8}{7}{L}{B}
    \gate[\inputs{2}]{or}{12}{5}{R}{}{}

    % ----  3rd column  ----
    \pin{15}{9}{L}{C}
    \gate[\inputs{2}]{nand}{19}{7}{R}{}{}

    % ----  4th column  ----
    \gate{not}{26}{7}{R}{}{}

    % ----  result ----
    \pin{30}{7}{R}{Z}

\end{circuitdiagram}

%%  ************    LibreSilicon's StdCellLibrary   *******************
%%
%%  Organisation:   Chipforge
%%                  Germany / European Union
%%
%%  Profile:        Chipforge focus on fine System-on-Chip Cores in
%%                  Verilog HDL Code which are easy understandable and
%%                  adjustable. For further information see
%%                          www.chipforge.org
%%                  there are projects from small cores up to PCBs, too.
%%
%%  File:           StdCellLib/Documents/Circuits/AOAI212.tex
%%
%%  Purpose:        Circuit File for AOAI212
%%
%%  ************    LaTeX with circdia.sty package      ***************
%%
%%  ///////////////////////////////////////////////////////////////////
%%
%%  Copyright (c) 2019 by chipforge <stdcelllib@nospam.chipforge.org>
%%  All rights reserved.
%%
%%      This Standard Cell Library is licensed under the Libre Silicon
%%      public license; you can redistribute it and/or modify it under
%%      the terms of the Libre Silicon public license as published by
%%      the Libre Silicon alliance, either version 1 of the License, or
%%      (at your option) any later version.
%%
%%      This design is distributed in the hope that it will be useful,
%%      but WITHOUT ANY WARRANTY; without even the implied warranty of
%%      MERCHANTABILITY or FITNESS FOR A PARTICULAR PURPOSE.
%%      See the Libre Silicon Public License for more details.
%%
%%  ///////////////////////////////////////////////////////////////////
\begin{circuitdiagram}{25}{12}

    \usgate
    \gate[\inputs{2}]{and}{5}{9}{R}{}{}    % AND
    \gate[\inputs{2}]{or}{12}{7}{R}{}{}    % OR
    \gate[\inputs{3}]{nand}{19}{3}{R}{}{}  % NAND
    \pin{1}{1}{L}{A}    % pin A
    \pin{1}{3}{L}{A1}   % pin A1
    \pin{1}{5}{L}{B}    % pin B
    \pin{1}{7}{L}{C}    % pin C
    \pin{1}{11}{L}{C1}  % pin C1
    \wire{16}{5}{16}{7} % wire between AND and OR
    \wire{2}{1}{16}{1}  % wire from pin A
    \wire{2}{3}{16}{3}  % wire from pin A1
    \wire{2}{5}{9}{5}   % wire from pin B
    \pin{24}{3}{R}{Y}   % pin Y

\end{circuitdiagram}
 %%  ************    LibreSilicon's StdCellLibrary   *******************
%%
%%  Organisation:   Chipforge
%%                  Germany / European Union
%%
%%  Profile:        Chipforge focus on fine System-on-Chip Cores in
%%                  Verilog HDL Code which are easy understandable and
%%                  adjustable. For further information see
%%                          www.chipforge.org
%%                  there are projects from small cores up to PCBs, too.
%%
%%  File:           StdCellLib/Documents/Circuits/AOA212.tex
%%
%%  Purpose:        Circuit File for AOA212
%%
%%  ************    LaTeX with circdia.sty package      ***************
%%
%%  ///////////////////////////////////////////////////////////////////
%%
%%  Copyright (c) 2019 by chipforge <stdcelllib@nospam.chipforge.org>
%%  All rights reserved.
%%
%%      This Standard Cell Library is licensed under the Libre Silicon
%%      public license; you can redistribute it and/or modify it under
%%      the terms of the Libre Silicon public license as published by
%%      the Libre Silicon alliance, either version 1 of the License, or
%%      (at your option) any later version.
%%
%%      This design is distributed in the hope that it will be useful,
%%      but WITHOUT ANY WARRANTY; without even the implied warranty of
%%      MERCHANTABILITY or FITNESS FOR A PARTICULAR PURPOSE.
%%      See the Libre Silicon Public License for more details.
%%
%%  ///////////////////////////////////////////////////////////////////
\begin{circuitdiagram}{31}{12}

    \usgate
    \gate[\inputs{2}]{and}{5}{9}{R}{}{}    % AND
    \gate[\inputs{2}]{or}{12}{7}{R}{}{}    % OR
    \gate[\inputs{3}]{nand}{19}{3}{R}{}{}  % NAND
    \gate{not}{26}{3}{R}{}{}  % NOT
    \pin{1}{1}{L}{A}    % pin A
    \pin{1}{3}{L}{A1}   % pin A1
    \pin{1}{5}{L}{B}    % pin B
    \pin{1}{7}{L}{C}    % pin C
    \pin{1}{11}{L}{C1}  % pin C1
    \wire{16}{5}{16}{7} % wire between AND and OR
    \wire{2}{1}{16}{1}  % wire from pin A
    \wire{2}{3}{16}{3}  % wire from pin A1
    \wire{2}{5}{9}{5}   % wire from pin B
    \pin{30}{3}{R}{Z}   % pin Z

\end{circuitdiagram}

%%  ************    LibreSilicon's StdCellLibrary   *******************
%%
%%  Organisation:   Chipforge
%%                  Germany / European Union
%%
%%  Profile:        Chipforge focus on fine System-on-Chip Cores in
%%                  Verilog HDL Code which are easy understandable and
%%                  adjustable. For further information see
%%                          www.chipforge.org
%%                  there are projects from small cores up to PCBs, too.
%%
%%  File:           StdCellLib/Documents/Datasheets/Circuitry/AOAI221.tex
%%
%%  Purpose:        Circuit File for AOAI221
%%
%%  ************    LaTeX with circdia.sty package      ***************
%%
%%  ///////////////////////////////////////////////////////////////////
%%
%%  Copyright (c) 2018 - 2022 by
%%                  chipforge <stdcelllib@nospam.chipforge.org>
%%  All rights reserved.
%%
%%      This Standard Cell Library is licensed under the Libre Silicon
%%      public license; you can redistribute it and/or modify it under
%%      the terms of the Libre Silicon public license as published by
%%      the Libre Silicon alliance, either version 1 of the License, or
%%      (at your option) any later version.
%%
%%      This design is distributed in the hope that it will be useful,
%%      but WITHOUT ANY WARRANTY; without even the implied warranty of
%%      MERCHANTABILITY or FITNESS FOR A PARTICULAR PURPOSE.
%%      See the Libre Silicon Public License for more details.
%%
%%  ///////////////////////////////////////////////////////////////////
\begin{circuitdiagram}[draft]{25}{12}

    \usgate
    % ----  1st column  ----
    \pin{1}{1}{L}{A}
    \pin{1}{5}{L}{A1}
    \gate[\inputs{2}]{and}{5}{3}{R}{}{}

    % ----  2nd column  ----
    \pin{8}{7}{L}{B}
    \pin{8}{9}{L}{B1}
    \wire{9}{3}{9}{5}
    \gate[\inputs{3}]{or}{12}{7}{R}{}{}

    % ----  3rd column  ----
    \pin{15}{11}{L}{C}
    \gate[\inputs{2}]{nand}{19}{9}{R}{}{}

    % ----  result ----
    \pin{24}{9}{R}{Y}

\end{circuitdiagram}
 %%  ************    LibreSilicon's StdCellLibrary   *******************
%%
%%  Organisation:   Chipforge
%%                  Germany / European Union
%%
%%  Profile:        Chipforge focus on fine System-on-Chip Cores in
%%                  Verilog HDL Code which are easy understandable and
%%                  adjustable. For further information see
%%                          www.chipforge.org
%%                  there are projects from small cores up to PCBs, too.
%%
%%  File:           StdCellLib/Documents/Datasheets/Circuitry/AOA221.tex
%%
%%  Purpose:        Circuit File for AOA221
%%
%%  ************    LaTeX with circdia.sty package      ***************
%%
%%  ///////////////////////////////////////////////////////////////////
%%
%%  Copyright (c) 2018 - 2022 by
%%                  chipforge <stdcelllib@nospam.chipforge.org>
%%  All rights reserved.
%%
%%      This Standard Cell Library is licensed under the Libre Silicon
%%      public license; you can redistribute it and/or modify it under
%%      the terms of the Libre Silicon public license as published by
%%      the Libre Silicon alliance, either version 1 of the License, or
%%      (at your option) any later version.
%%
%%      This design is distributed in the hope that it will be useful,
%%      but WITHOUT ANY WARRANTY; without even the implied warranty of
%%      MERCHANTABILITY or FITNESS FOR A PARTICULAR PURPOSE.
%%      See the Libre Silicon Public License for more details.
%%
%%  ///////////////////////////////////////////////////////////////////
\begin{circuitdiagram}[draft]{31}{12}

    \usgate
    % ----  1st column  ----
    \pin{1}{1}{L}{A}
    \pin{1}{5}{L}{A1}
    \gate[\inputs{2}]{and}{5}{3}{R}{}{}

    % ----  2nd column  ----
    \pin{8}{7}{L}{B}
    \pin{8}{9}{L}{B1}
    \wire{9}{3}{9}{5}
    \gate[\inputs{3}]{or}{12}{7}{R}{}{}

    % ----  3rd column  ----
    \pin{15}{11}{L}{C}
    \gate[\inputs{2}]{nand}{19}{9}{R}{}{}

    % ----  4th column  ----
    \gate{not}{26}{9}{R}{}{}

    % ----  result ----
    \pin{30}{9}{R}{Z}

\end{circuitdiagram}

%%  ************    LibreSilicon's StdCellLibrary   *******************
%%
%%  Organisation:   Chipforge
%%                  Germany / European Union
%%
%%  Profile:        Chipforge focus on fine System-on-Chip Cores in
%%                  Verilog HDL Code which are easy understandable and
%%                  adjustable. For further information see
%%                          www.chipforge.org
%%                  there are projects from small cores up to PCBs, too.
%%
%%  File:           StdCellLib/Documents/Datasheets/Circuitry/AOAI222.tex
%%
%%  Purpose:        Circuit File for AOAI222
%%
%%  ************    LaTeX with circdia.sty package      ***************
%%
%%  ///////////////////////////////////////////////////////////////////
%%
%%  Copyright (c) 2018 - 2022 by
%%                  chipforge <stdcelllib@nospam.chipforge.org>
%%  All rights reserved.
%%
%%      This Standard Cell Library is licensed under the Libre Silicon
%%      public license; you can redistribute it and/or modify it under
%%      the terms of the Libre Silicon public license as published by
%%      the Libre Silicon alliance, either version 1 of the License, or
%%      (at your option) any later version.
%%
%%      This design is distributed in the hope that it will be useful,
%%      but WITHOUT ANY WARRANTY; without even the implied warranty of
%%      MERCHANTABILITY or FITNESS FOR A PARTICULAR PURPOSE.
%%      See the Libre Silicon Public License for more details.
%%
%%  ///////////////////////////////////////////////////////////////////
\begin{circuitdiagram}[draft]{25}{14}

    \usgate
    % ----  1st column  ----
    \pin{1}{1}{L}{A}
    \pin{1}{5}{L}{A1}
    \gate[\inputs{2}]{and}{5}{3}{R}{}{}

    % ----  2nd column  ----
    \pin{8}{7}{L}{B}
    \pin{8}{9}{L}{B1}
    \wire{9}{3}{9}{5}
    \gate[\inputs{3}]{or}{12}{7}{R}{}{}

    % ----  3rd column  ----
    \pin{15}{11}{L}{C}
    \pin{15}{13}{L}{C1}
    \wire{16}{7}{16}{9}
    \gate[\inputs{3}]{nand}{19}{11}{R}{}{}

    % ----  result ----
    \pin{24}{11}{R}{Y}

\end{circuitdiagram}
 %%  ************    LibreSilicon's StdCellLibrary   *******************
%%
%%  Organisation:   Chipforge
%%                  Germany / European Union
%%
%%  Profile:        Chipforge focus on fine System-on-Chip Cores in
%%                  Verilog HDL Code which are easy understandable and
%%                  adjustable. For further information see
%%                          www.chipforge.org
%%                  there are projects from small cores up to PCBs, too.
%%
%%  File:           StdCellLib/Documents/Datasheets/Circuitry/AOA222.tex
%%
%%  Purpose:        Circuit File for AOA222
%%
%%  ************    LaTeX with circdia.sty package      ***************
%%
%%  ///////////////////////////////////////////////////////////////////
%%
%%  Copyright (c) 2018 - 2022 by
%%                  chipforge <stdcelllib@nospam.chipforge.org>
%%  All rights reserved.
%%
%%      This Standard Cell Library is licensed under the Libre Silicon
%%      public license; you can redistribute it and/or modify it under
%%      the terms of the Libre Silicon public license as published by
%%      the Libre Silicon alliance, either version 1 of the License, or
%%      (at your option) any later version.
%%
%%      This design is distributed in the hope that it will be useful,
%%      but WITHOUT ANY WARRANTY; without even the implied warranty of
%%      MERCHANTABILITY or FITNESS FOR A PARTICULAR PURPOSE.
%%      See the Libre Silicon Public License for more details.
%%
%%  ///////////////////////////////////////////////////////////////////
\begin{circuitdiagram}[draft]{31}{14}

    \usgate
    % ----  1st column  ----
    \pin{1}{1}{L}{A}
    \pin{1}{5}{L}{A1}
    \gate[\inputs{2}]{and}{5}{3}{R}{}{}

    % ----  2nd column  ----
    \pin{8}{7}{L}{B}
    \pin{8}{9}{L}{B1}
    \wire{9}{3}{9}{5}
    \gate[\inputs{3}]{or}{12}{7}{R}{}{}

    % ----  3rd column  ----
    \pin{15}{11}{L}{C}
    \pin{15}{13}{L}{C1}
    \wire{16}{7}{16}{9}
    \gate[\inputs{3}]{nand}{19}{11}{R}{}{}

    % ----  4th column  ----
    \gate{not}{26}{11}{R}{}{}

    % ----  result ----
    \pin{30}{11}{R}{Z}

\end{circuitdiagram}

%%  ************    LibreSilicon's StdCellLibrary   *******************
%%
%%  Organisation:   Chipforge
%%                  Germany / European Union
%%
%%  Profile:        Chipforge focus on fine System-on-Chip Cores in
%%                  Verilog HDL Code which are easy understandable and
%%                  adjustable. For further information see
%%                          www.chipforge.org
%%                  there are projects from small cores up to PCBs, too.
%%
%%  File:           StdCellLib/Documents/Datasheets/Circuitry/AOAI222.tex
%%
%%  Purpose:        Circuit File for AOAI222
%%
%%  ************    LaTeX with circdia.sty package      ***************
%%
%%  ///////////////////////////////////////////////////////////////////
%%
%%  Copyright (c) 2018 - 2022 by
%%                  chipforge <stdcelllib@nospam.chipforge.org>
%%  All rights reserved.
%%
%%      This Standard Cell Library is licensed under the Libre Silicon
%%      public license; you can redistribute it and/or modify it under
%%      the terms of the Libre Silicon public license as published by
%%      the Libre Silicon alliance, either version 1 of the License, or
%%      (at your option) any later version.
%%
%%      This design is distributed in the hope that it will be useful,
%%      but WITHOUT ANY WARRANTY; without even the implied warranty of
%%      MERCHANTABILITY or FITNESS FOR A PARTICULAR PURPOSE.
%%      See the Libre Silicon Public License for more details.
%%
%%  ///////////////////////////////////////////////////////////////////
\begin{circuitdiagram}[draft]{25}{13}

    \usgate
    % ----  1st column  ----
    \pin{1}{1}{L}{A}
    \pin{1}{5}{L}{A1}
    \gate[\inputs{2}]{and}{5}{3}{R}{}{}

    % ----  2nd column  ----
    \pin{8}{7}{L}{B}
    \pin{8}{9}{L}{B1}
    \pin{8}{11}{L}{B2}
    \wire{9}{3}{9}{5}
    \gate[\inputs{4}]{or}{12}{8}{R}{}{}

    % ----  3rd column  ----
    \pin{15}{12}{L}{C}
    \gate[\inputs{2}]{nand}{19}{10}{R}{}{}

    % ----  result ----
    \pin{24}{10}{R}{Y}

\end{circuitdiagram}
 %%  ************    LibreSilicon's StdCellLibrary   *******************
%%
%%  Organisation:   Chipforge
%%                  Germany / European Union
%%
%%  Profile:        Chipforge focus on fine System-on-Chip Cores in
%%                  Verilog HDL Code which are easy understandable and
%%                  adjustable. For further information see
%%                          www.chipforge.org
%%                  there are projects from small cores up to PCBs, too.
%%
%%  File:           StdCellLib/Documents/Datasheets/Circuitry/AOA222.tex
%%
%%  Purpose:        Circuit File for AOA222
%%
%%  ************    LaTeX with circdia.sty package      ***************
%%
%%  ///////////////////////////////////////////////////////////////////
%%
%%  Copyright (c) 2018 - 2022 by
%%                  chipforge <stdcelllib@nospam.chipforge.org>
%%  All rights reserved.
%%
%%      This Standard Cell Library is licensed under the Libre Silicon
%%      public license; you can redistribute it and/or modify it under
%%      the terms of the Libre Silicon public license as published by
%%      the Libre Silicon alliance, either version 1 of the License, or
%%      (at your option) any later version.
%%
%%      This design is distributed in the hope that it will be useful,
%%      but WITHOUT ANY WARRANTY; without even the implied warranty of
%%      MERCHANTABILITY or FITNESS FOR A PARTICULAR PURPOSE.
%%      See the Libre Silicon Public License for more details.
%%
%%  ///////////////////////////////////////////////////////////////////
\begin{circuitdiagram}[draft]{31}{13}

    \usgate
    % ----  1st column  ----
    \pin{1}{1}{L}{A}
    \pin{1}{5}{L}{A1}
    \gate[\inputs{2}]{and}{5}{3}{R}{}{}

    % ----  2nd column  ----
    \pin{8}{7}{L}{B}
    \pin{8}{9}{L}{B1}
    \pin{8}{11}{L}{B2}
    \wire{9}{3}{9}{5}
    \gate[\inputs{4}]{or}{12}{8}{R}{}{}

    % ----  3rd column  ----
    \pin{15}{12}{L}{C}
    \gate[\inputs{2}]{nand}{19}{10}{R}{}{}

    % ----  4th column  ----
    \gate{not}{26}{10}{R}{}{}

    % ----  result ----
    \pin{30}{10}{R}{Z}

\end{circuitdiagram}

%%  ************    LibreSilicon's StdCellLibrary   *******************
%%
%%  Organisation:   Chipforge
%%                  Germany / European Union
%%
%%  Profile:        Chipforge focus on fine System-on-Chip Cores in
%%                  Verilog HDL Code which are easy understandable and
%%                  adjustable. For further information see
%%                          www.chipforge.org
%%                  there are projects from small cores up to PCBs, too.
%%
%%  File:           StdCellLib/Documents/Datasheets/Circuitry/AOAI232.tex
%%
%%  Purpose:        Circuit File for AOAI232
%%
%%  ************    LaTeX with circdia.sty package      ***************
%%
%%  ///////////////////////////////////////////////////////////////////
%%
%%  Copyright (c) 2018 - 2022 by
%%                  chipforge <stdcelllib@nospam.chipforge.org>
%%  All rights reserved.
%%
%%      This Standard Cell Library is licensed under the Libre Silicon
%%      public license; you can redistribute it and/or modify it under
%%      the terms of the Libre Silicon public license as published by
%%      the Libre Silicon alliance, either version 1 of the License, or
%%      (at your option) any later version.
%%
%%      This design is distributed in the hope that it will be useful,
%%      but WITHOUT ANY WARRANTY; without even the implied warranty of
%%      MERCHANTABILITY or FITNESS FOR A PARTICULAR PURPOSE.
%%      See the Libre Silicon Public License for more details.
%%
%%  ///////////////////////////////////////////////////////////////////
\begin{circuitdiagram}[draft]{25}{15}

    \usgate
    % ----  1st column  ----
    \pin{1}{1}{L}{A}
    \pin{1}{5}{L}{A1}
    \gate[\inputs{2}]{and}{5}{3}{R}{}{}

    % ----  2nd column  ----
    \pin{8}{7}{L}{B}
    \pin{8}{9}{L}{B1}
    \pin{8}{11}{L}{B2}
    \wire{9}{3}{9}{5}
    \gate[\inputs{4}]{or}{12}{8}{R}{}{}

    % ----  3rd column  ----
    \wire{16}{8}{16}{10}
    \pin{15}{12}{L}{C}
    \pin{15}{14}{L}{C1}
    \gate[\inputs{3}]{nand}{19}{12}{R}{}{}

    % ----  result ----
    \pin{24}{12}{R}{Y}

\end{circuitdiagram}
 %%  ************    LibreSilicon's StdCellLibrary   *******************
%%
%%  Organisation:   Chipforge
%%                  Germany / European Union
%%
%%  Profile:        Chipforge focus on fine System-on-Chip Cores in
%%                  Verilog HDL Code which are easy understandable and
%%                  adjustable. For further information see
%%                          www.chipforge.org
%%                  there are projects from small cores up to PCBs, too.
%%
%%  File:           StdCellLib/Documents/Datasheets/Circuitry/AOA232.tex
%%
%%  Purpose:        Circuit File for AOA232
%%
%%  ************    LaTeX with circdia.sty package      ***************
%%
%%  ///////////////////////////////////////////////////////////////////
%%
%%  Copyright (c) 2018 - 2022 by
%%                  chipforge <stdcelllib@nospam.chipforge.org>
%%  All rights reserved.
%%
%%      This Standard Cell Library is licensed under the Libre Silicon
%%      public license; you can redistribute it and/or modify it under
%%      the terms of the Libre Silicon public license as published by
%%      the Libre Silicon alliance, either version 1 of the License, or
%%      (at your option) any later version.
%%
%%      This design is distributed in the hope that it will be useful,
%%      but WITHOUT ANY WARRANTY; without even the implied warranty of
%%      MERCHANTABILITY or FITNESS FOR A PARTICULAR PURPOSE.
%%      See the Libre Silicon Public License for more details.
%%
%%  ///////////////////////////////////////////////////////////////////
\begin{circuitdiagram}[draft]{31}{15}

    \usgate
    % ----  1st column  ----
    \pin{1}{1}{L}{A}
    \pin{1}{5}{L}{A1}
    \gate[\inputs{2}]{and}{5}{3}{R}{}{}

    % ----  2nd column  ----
    \pin{8}{7}{L}{B}
    \pin{8}{9}{L}{B1}
    \pin{8}{11}{L}{B2}
    \wire{9}{3}{9}{5}
    \gate[\inputs{4}]{or}{12}{8}{R}{}{}

    % ----  3rd column  ----
    \wire{16}{8}{16}{10}
    \pin{15}{12}{L}{C}
    \pin{15}{14}{L}{C1}
    \gate[\inputs{3}]{nand}{19}{12}{R}{}{}

    % ----  4th column  ----
    \gate{not}{26}{12}{R}{}{}

    % ----  result ----
    \pin{30}{12}{R}{Z}

\end{circuitdiagram}

%%  ************    LibreSilicon's StdCellLibrary   *******************
%%
%%  Organisation:   Chipforge
%%                  Germany / European Union
%%
%%  Profile:        Chipforge focus on fine System-on-Chip Cores in
%%                  Verilog HDL Code which are easy understandable and
%%                  adjustable. For further information see
%%                          www.chipforge.org
%%                  there are projects from small cores up to PCBs, too.
%%
%%  File:           StdCellLib/Documents/Circuits/AOAI311.tex
%%
%%  Purpose:        Circuit File for AOAI311
%%
%%  ************    LaTeX with circdia.sty package      ***************
%%
%%  ///////////////////////////////////////////////////////////////////
%%
%%  Copyright (c) 2019 by chipforge <stdcelllib@nospam.chipforge.org>
%%  All rights reserved.
%%
%%      This Standard Cell Library is licensed under the Libre Silicon
%%      public license; you can redistribute it and/or modify it under
%%      the terms of the Libre Silicon public license as published by
%%      the Libre Silicon alliance, either version 1 of the License, or
%%      (at your option) any later version.
%%
%%      This design is distributed in the hope that it will be useful,
%%      but WITHOUT ANY WARRANTY; without even the implied warranty of
%%      MERCHANTABILITY or FITNESS FOR A PARTICULAR PURPOSE.
%%      See the Libre Silicon Public License for more details.
%%
%%  ///////////////////////////////////////////////////////////////////
\begin{circuitdiagram}{25}{10}

    \usgate
    \gate[\inputs{3}]{and}{5}{7}{R}{}{}    % AND
    \gate[\inputs{2}]{or}{12}{5}{R}{}{}    % OR
    \gate[\inputs{2}]{nand}{19}{3}{R}{}{}  % NAND
    \pin{1}{1}{L}{A}    % pin A
    \pin{1}{3}{L}{B}    % pin B
    \pin{1}{5}{L}{C}    % pin C
    \pin{1}{7}{L}{C1}   % pin C1
    \pin{1}{9}{L}{C2}   % pin C2
    \wire{2}{1}{16}{1}  % wire from pin A
    \wire{2}{3}{9}{3}   % wire from pin B
    \pin{24}{3}{R}{Y}   % pin Y

\end{circuitdiagram}
 %%  ************    LibreSilicon's StdCellLibrary   *******************
%%
%%  Organisation:   Chipforge
%%                  Germany / European Union
%%
%%  Profile:        Chipforge focus on fine System-on-Chip Cores in
%%                  Verilog HDL Code which are easy understandable and
%%                  adjustable. For further information see
%%                          www.chipforge.org
%%                  there are projects from small cores up to PCBs, too.
%%
%%  File:           StdCellLib/Documents/Circuits/AOA311.tex
%%
%%  Purpose:        Circuit File for AOA311
%%
%%  ************    LaTeX with circdia.sty package      ***************
%%
%%  ///////////////////////////////////////////////////////////////////
%%
%%  Copyright (c) 2019 by chipforge <stdcelllib@nospam.chipforge.org>
%%  All rights reserved.
%%
%%      This Standard Cell Library is licensed under the Libre Silicon
%%      public license; you can redistribute it and/or modify it under
%%      the terms of the Libre Silicon public license as published by
%%      the Libre Silicon alliance, either version 1 of the License, or
%%      (at your option) any later version.
%%
%%      This design is distributed in the hope that it will be useful,
%%      but WITHOUT ANY WARRANTY; without even the implied warranty of
%%      MERCHANTABILITY or FITNESS FOR A PARTICULAR PURPOSE.
%%      See the Libre Silicon Public License for more details.
%%
%%  ///////////////////////////////////////////////////////////////////
\begin{circuitdiagram}{31}{10}

    \usgate
    \gate[\inputs{3}]{and}{5}{7}{R}{}{}    % AND
    \gate[\inputs{2}]{or}{12}{5}{R}{}{}    % OR
    \gate[\inputs{2}]{nand}{19}{3}{R}{}{}  % NAND
    \gate{not}{26}{3}{R}{}{}  % NOT
    \pin{1}{1}{L}{A}    % pin A
    \pin{1}{3}{L}{B}    % pin B
    \pin{1}{5}{L}{C}    % pin C
    \pin{1}{7}{L}{C1}   % pin C1
    \pin{1}{9}{L}{C2}   % pin C2
    \wire{2}{1}{16}{1}  % wire from pin A
    \wire{2}{3}{9}{3}   % wire from pin C
    \pin{30}{3}{R}{Z}   % pin Z

\end{circuitdiagram}

%%  ************    LibreSilicon's StdCellLibrary   *******************
%%
%%  Organisation:   Chipforge
%%                  Germany / European Union
%%
%%  Profile:        Chipforge focus on fine System-on-Chip Cores in
%%                  Verilog HDL Code which are easy understandable and
%%                  adjustable. For further information see
%%                          www.chipforge.org
%%                  there are projects from small cores up to PCBs, too.
%%
%%  File:           StdCellLib/Documents/Datasheets/Circuitry/AOAI321.tex
%%
%%  Purpose:        Circuit File for AOAI321
%%
%%  ************    LaTeX with circdia.sty package      ***************
%%
%%  ///////////////////////////////////////////////////////////////////
%%
%%  Copyright (c) 2018 - 2022 by
%%                  chipforge <stdcelllib@nospam.chipforge.org>
%%  All rights reserved.
%%
%%      This Standard Cell Library is licensed under the Libre Silicon
%%      public license; you can redistribute it and/or modify it under
%%      the terms of the Libre Silicon public license as published by
%%      the Libre Silicon alliance, either version 1 of the License, or
%%      (at your option) any later version.
%%
%%      This design is distributed in the hope that it will be useful,
%%      but WITHOUT ANY WARRANTY; without even the implied warranty of
%%      MERCHANTABILITY or FITNESS FOR A PARTICULAR PURPOSE.
%%      See the Libre Silicon Public License for more details.
%%
%%  ///////////////////////////////////////////////////////////////////
\begin{circuitdiagram}[draft]{25}{12}

    \usgate
    % ----  1st column  ----
    \pin{1}{1}{L}{A}
    \pin{1}{3}{L}{A1}
    \pin{1}{5}{L}{A2}
    \gate[\inputs{3}]{and}{5}{3}{R}{}{}

    % ----  2nd column  ----
    \pin{8}{7}{L}{B}
    \pin{8}{9}{L}{B1}
    \wire{9}{3}{9}{5}
    \gate[\inputs{3}]{or}{12}{7}{R}{}{}

    % ----  3rd column  ----
    \pin{15}{11}{L}{C}
    \gate[\inputs{2}]{nand}{19}{9}{R}{}{}

    % ----  result ----
    \pin{24}{9}{R}{Y}

\end{circuitdiagram}
 %%  ************    LibreSilicon's StdCellLibrary   *******************
%%
%%  Organisation:   Chipforge
%%                  Germany / European Union
%%
%%  Profile:        Chipforge focus on fine System-on-Chip Cores in
%%                  Verilog HDL Code which are easy understandable and
%%                  adjustable. For further information see
%%                          www.chipforge.org
%%                  there are projects from small cores up to PCBs, too.
%%
%%  File:           StdCellLib/Documents/Datasheets/Circuitry/AOA321.tex
%%
%%  Purpose:        Circuit File for AOA321
%%
%%  ************    LaTeX with circdia.sty package      ***************
%%
%%  ///////////////////////////////////////////////////////////////////
%%
%%  Copyright (c) 2018 - 2022 by
%%                  chipforge <stdcelllib@nospam.chipforge.org>
%%  All rights reserved.
%%
%%      This Standard Cell Library is licensed under the Libre Silicon
%%      public license; you can redistribute it and/or modify it under
%%      the terms of the Libre Silicon public license as published by
%%      the Libre Silicon alliance, either version 1 of the License, or
%%      (at your option) any later version.
%%
%%      This design is distributed in the hope that it will be useful,
%%      but WITHOUT ANY WARRANTY; without even the implied warranty of
%%      MERCHANTABILITY or FITNESS FOR A PARTICULAR PURPOSE.
%%      See the Libre Silicon Public License for more details.
%%
%%  ///////////////////////////////////////////////////////////////////
\begin{circuitdiagram}[draft]{31}{12}

    \usgate
    % ----  1st column  ----
    \pin{1}{1}{L}{A}
    \pin{1}{3}{L}{A1}
    \pin{1}{5}{L}{A2}
    \gate[\inputs{3}]{and}{5}{3}{R}{}{}

    % ----  2nd column  ----
    \pin{8}{7}{L}{B}
    \pin{8}{9}{L}{B1}
    \wire{9}{3}{9}{5}
    \gate[\inputs{3}]{or}{12}{7}{R}{}{}

    % ----  3rd column  ----
    \pin{15}{11}{L}{C}
    \gate[\inputs{2}]{nand}{19}{9}{R}{}{}

    % ----  4th column  ----
    \gate{not}{26}{9}{R}{}{}

    % ----  result ----
    \pin{30}{9}{R}{Z}

\end{circuitdiagram}

%%  ************    LibreSilicon's StdCellLibrary   *******************
%%
%%  Organisation:   Chipforge
%%                  Germany / European Union
%%
%%  Profile:        Chipforge focus on fine System-on-Chip Cores in
%%                  Verilog HDL Code which are easy understandable and
%%                  adjustable. For further information see
%%                          www.chipforge.org
%%                  there are projects from small cores up to PCBs, too.
%%
%%  File:           StdCellLib/Documents/Datasheets/Circuitry/AOAI331.tex
%%
%%  Purpose:        Circuit File for AOAI331
%%
%%  ************    LaTeX with circdia.sty package      ***************
%%
%%  ///////////////////////////////////////////////////////////////////
%%
%%  Copyright (c) 2018 - 2022 by
%%                  chipforge <stdcelllib@nospam.chipforge.org>
%%  All rights reserved.
%%
%%      This Standard Cell Library is licensed under the Libre Silicon
%%      public license; you can redistribute it and/or modify it under
%%      the terms of the Libre Silicon public license as published by
%%      the Libre Silicon alliance, either version 1 of the License, or
%%      (at your option) any later version.
%%
%%      This design is distributed in the hope that it will be useful,
%%      but WITHOUT ANY WARRANTY; without even the implied warranty of
%%      MERCHANTABILITY or FITNESS FOR A PARTICULAR PURPOSE.
%%      See the Libre Silicon Public License for more details.
%%
%%  ///////////////////////////////////////////////////////////////////
\begin{circuitdiagram}[draft]{25}{13}

    \usgate
    % ----  1st column  ----
    \pin{1}{1}{L}{A}
    \pin{1}{3}{L}{A1}
    \pin{1}{5}{L}{A2}
    \gate[\inputs{3}]{and}{5}{3}{R}{}{}

    % ----  2nd column  ----
    \pin{8}{7}{L}{B}
    \pin{8}{9}{L}{B1}
    \pin{8}{11}{L}{B2}
    \wire{9}{3}{9}{5}
    \gate[\inputs{4}]{or}{12}{8}{R}{}{}

    % ----  3rd column  ----
    \pin{15}{12}{L}{C}
    \gate[\inputs{2}]{nand}{19}{10}{R}{}{}

    % ----  result ----
    \pin{24}{10}{R}{Y}

\end{circuitdiagram}
 %%  ************    LibreSilicon's StdCellLibrary   *******************
%%
%%  Organisation:   Chipforge
%%                  Germany / European Union
%%
%%  Profile:        Chipforge focus on fine System-on-Chip Cores in
%%                  Verilog HDL Code which are easy understandable and
%%                  adjustable. For further information see
%%                          www.chipforge.org
%%                  there are projects from small cores up to PCBs, too.
%%
%%  File:           StdCellLib/Documents/Datasheets/Circuitry/AOA331.tex
%%
%%  Purpose:        Circuit File for AOA331
%%
%%  ************    LaTeX with circdia.sty package      ***************
%%
%%  ///////////////////////////////////////////////////////////////////
%%
%%  Copyright (c) 2018 - 2022 by
%%                  chipforge <stdcelllib@nospam.chipforge.org>
%%  All rights reserved.
%%
%%      This Standard Cell Library is licensed under the Libre Silicon
%%      public license; you can redistribute it and/or modify it under
%%      the terms of the Libre Silicon public license as published by
%%      the Libre Silicon alliance, either version 1 of the License, or
%%      (at your option) any later version.
%%
%%      This design is distributed in the hope that it will be useful,
%%      but WITHOUT ANY WARRANTY; without even the implied warranty of
%%      MERCHANTABILITY or FITNESS FOR A PARTICULAR PURPOSE.
%%      See the Libre Silicon Public License for more details.
%%
%%  ///////////////////////////////////////////////////////////////////
\begin{circuitdiagram}[draft]{31}{13}

    \usgate
    % ----  1st column  ----
    \pin{1}{1}{L}{A}
    \pin{1}{3}{L}{A1}
    \pin{1}{5}{L}{A2}
    \gate[\inputs{3}]{and}{5}{3}{R}{}{}

    % ----  2nd column  ----
    \pin{8}{7}{L}{B}
    \pin{8}{9}{L}{B1}
    \pin{8}{11}{L}{B2}
    \wire{9}{3}{9}{5}
    \gate[\inputs{4}]{or}{12}{8}{R}{}{}

    % ----  3rd column  ----
    \pin{15}{12}{L}{C}
    \gate[\inputs{2}]{nand}{19}{10}{R}{}{}

    % ----  4th column  ----
    \gate{not}{26}{10}{R}{}{}

    % ----  result ----
    \pin{30}{10}{R}{Z}

\end{circuitdiagram}


%%  ************    LibreSilicon's StdCellLibrary   *******************
%%
%%  Organisation:   Chipforge
%%                  Germany / European Union
%%
%%  Profile:        Chipforge focus on fine System-on-Chip Cores in
%%                  Verilog HDL Code which are easy understandable and
%%                  adjustable. For further information see
%%                          www.chipforge.org
%%                  there are projects from small cores up to PCBs, too.
%%
%%  File:           StdCellLib/Documents/Book/section-OAOI_complex.tex
%%
%%  Purpose:        Section Level File for Standard Cell Library Documentation
%%
%%  ************    LaTeX with circdia.sty package      ***************
%%
%%  ///////////////////////////////////////////////////////////////////
%%
%%  Copyright (c) 2018 - 2022 by
%%                  chipforge <stdcelllib@nospam.chipforge.org>
%%  All rights reserved.
%%
%%      This Standard Cell Library is licensed under the Libre Silicon
%%      public license; you can redistribute it and/or modify it under
%%      the terms of the Libre Silicon public license as published by
%%      the Libre Silicon alliance, either version 1 of the License, or
%%      (at your option) any later version.
%%
%%      This design is distributed in the hope that it will be useful,
%%      but WITHOUT ANY WARRANTY; without even the implied warranty of
%%      MERCHANTABILITY or FITNESS FOR A PARTICULAR PURPOSE.
%%      See the Libre Silicon Public License for more details.
%%
%%  ///////////////////////////////////////////////////////////////////
\section{OR-AND-OR(-Invert) Complex Gates}

%%  ************    LibreSilicon's StdCellLibrary   *******************
%%
%%  Organisation:   Chipforge
%%                  Germany / European Union
%%
%%  Profile:        Chipforge focus on fine System-on-Chip Cores in
%%                  Verilog HDL Code which are easy understandable and
%%                  adjustable. For further information see
%%                          www.chipforge.org
%%                  there are projects from small cores up to PCBs, too.
%%
%%  File:           StdCellLib/Documents/Circuits/OAOI211.tex
%%
%%  Purpose:        Circuit File for OAOI211
%%
%%  ************    LaTeX with circdia.sty package      ***************
%%
%%  ///////////////////////////////////////////////////////////////////
%%
%%  Copyright (c) 2019 by chipforge <stdcelllib@nospam.chipforge.org>
%%  All rights reserved.
%%
%%      This Standard Cell Library is licensed under the Libre Silicon
%%      public license; you can redistribute it and/or modify it under
%%      the terms of the Libre Silicon public license as published by
%%      the Libre Silicon alliance, either version 1 of the License, or
%%      (at your option) any later version.
%%
%%      This design is distributed in the hope that it will be useful,
%%      but WITHOUT ANY WARRANTY; without even the implied warranty of
%%      MERCHANTABILITY or FITNESS FOR A PARTICULAR PURPOSE.
%%      See the Libre Silicon Public License for more details.
%%
%%  ///////////////////////////////////////////////////////////////////
\begin{circuitdiagram}{25}{10}

    \usgate
    \gate[\inputs{2}]{or}{5}{7}{R}{}{}    % OR
    \gate[\inputs{2}]{and}{12}{5}{R}{}{}  % AND
    \gate[\inputs{2}]{nor}{19}{3}{R}{}{}  % NOR
    \pin{1}{1}{L}{A}    % pin A
    \pin{1}{3}{L}{B}    % pin B
    \pin{1}{5}{L}{C}    % pin C
    \pin{1}{9}{L}{C1}   % pin C1
    \wire{2}{1}{16}{1}  % wire from pin A
    \wire{2}{3}{9}{3}   % wire from pin C
    \pin{24}{3}{R}{Y}   % pin Y

\end{circuitdiagram}
 %%  ************    LibreSilicon's StdCellLibrary   *******************
%%
%%  Organisation:   Chipforge
%%                  Germany / European Union
%%
%%  Profile:        Chipforge focus on fine System-on-Chip Cores in
%%                  Verilog HDL Code which are easy understandable and
%%                  adjustable. For further information see
%%                          www.chipforge.org
%%                  there are projects from small cores up to PCBs, too.
%%
%%  File:           StdCellLib/Documents/Datasheets/Circuitry/OAO211.tex
%%
%%  Purpose:        Circuit File for OAO211
%%
%%  ************    LaTeX with circdia.sty package      ***************
%%
%%  ///////////////////////////////////////////////////////////////////
%%
%%  Copyright (c) 2018 - 2022 by
%%                  chipforge <stdcelllib@nospam.chipforge.org>
%%  All rights reserved.
%%
%%      This Standard Cell Library is licensed under the Libre Silicon
%%      public license; you can redistribute it and/or modify it under
%%      the terms of the Libre Silicon public license as published by
%%      the Libre Silicon alliance, either version 1 of the License, or
%%      (at your option) any later version.
%%
%%      This design is distributed in the hope that it will be useful,
%%      but WITHOUT ANY WARRANTY; without even the implied warranty of
%%      MERCHANTABILITY or FITNESS FOR A PARTICULAR PURPOSE.
%%      See the Libre Silicon Public License for more details.
%%
%%  ///////////////////////////////////////////////////////////////////
\begin{circuitdiagram}[draft]{31}{10}

    \usgate
    % ----  1st column  ----
    \pin{1}{1}{L}{A}
    \pin{1}{5}{L}{A1}
    \gate[\inputs{2}]{or}{5}{3}{R}{}{}

    % ----  2nd column  ----
    \pin{8}{7}{L}{B}
    \gate[\inputs{2}]{and}{12}{5}{R}{}{}

    % ----  3rd column  ----
    \pin{15}{9}{L}{C}
    \gate[\inputs{2}]{nor}{19}{7}{R}{}{}

    % ----  4th column  ----
    \gate{not}{26}{7}{R}{}{}

    % ----  result ----
    \pin{30}{7}{R}{Z}

\end{circuitdiagram}

%%  ************    LibreSilicon's StdCellLibrary   *******************
%%
%%  Organisation:   Chipforge
%%                  Germany / European Union
%%
%%  Profile:        Chipforge focus on fine System-on-Chip Cores in
%%                  Verilog HDL Code which are easy understandable and
%%                  adjustable. For further information see
%%                          www.chipforge.org
%%                  there are projects from small cores up to PCBs, too.
%%
%%  File:           StdCellLib/Documents/Datasheets/Circuitry/OAOI212.tex
%%
%%  Purpose:        Circuit File for OAOI212
%%
%%  ************    LaTeX with circdia.sty package      ***************
%%
%%  ///////////////////////////////////////////////////////////////////
%%
%%  Copyright (c) 2018 - 2022 by
%%                  chipforge <stdcelllib@nospam.chipforge.org>
%%  All rights reserved.
%%
%%      This Standard Cell Library is licensed under the Libre Silicon
%%      public license; you can redistribute it and/or modify it under
%%      the terms of the Libre Silicon public license as published by
%%      the Libre Silicon alliance, either version 1 of the License, or
%%      (at your option) any later version.
%%
%%      This design is distributed in the hope that it will be useful,
%%      but WITHOUT ANY WARRANTY; without even the implied warranty of
%%      MERCHANTABILITY or FITNESS FOR A PARTICULAR PURPOSE.
%%      See the Libre Silicon Public License for more details.
%%
%%  ///////////////////////////////////////////////////////////////////
\begin{circuitdiagram}[draft]{25}{12}

    \usgate
    % ----  1st column  ----
    \pin{1}{1}{L}{A}
    \pin{1}{5}{L}{A1}
    \gate[\inputs{2}]{or}{5}{3}{R}{}{}

    % ----  2nd column  ----
    \pin{8}{7}{L}{B}
    \gate[\inputs{2}]{and}{12}{5}{R}{}{}

    % ----  3rd column  ----
    \pin{15}{9}{L}{C}
    \pin{15}{11}{L}{C1}
    \wire{16}{5}{16}{7}
    \gate[\inputs{3}]{nor}{19}{9}{R}{}{}

    % ----  result ----
    \pin{24}{9}{R}{Y}

\end{circuitdiagram}
 %%  ************    LibreSilicon's StdCellLibrary   *******************
%%
%%  Organisation:   Chipforge
%%                  Germany / European Union
%%
%%  Profile:        Chipforge focus on fine System-on-Chip Cores in
%%                  Verilog HDL Code which are easy understandable and
%%                  adjustable. For further information see
%%                          www.chipforge.org
%%                  there are projects from small cores up to PCBs, too.
%%
%%  File:           StdCellLib/Documents/Circuits/OAO212.tex
%%
%%  Purpose:        Circuit File for OAO212
%%
%%  ************    LaTeX with circdia.sty package      ***************
%%
%%  ///////////////////////////////////////////////////////////////////
%%
%%  Copyright (c) 2019 by chipforge <stdcelllib@nospam.chipforge.org>
%%  All rights reserved.
%%
%%      This Standard Cell Library is licensed under the Libre Silicon
%%      public license; you can redistribute it and/or modify it under
%%      the terms of the Libre Silicon public license as published by
%%      the Libre Silicon alliance, either version 1 of the License, or
%%      (at your option) any later version.
%%
%%      This design is distributed in the hope that it will be useful,
%%      but WITHOUT ANY WARRANTY; without even the implied warranty of
%%      MERCHANTABILITY or FITNESS FOR A PARTICULAR PURPOSE.
%%      See the Libre Silicon Public License for more details.
%%
%%  ///////////////////////////////////////////////////////////////////
\begin{center}
    Circuit
    \begin{figure}[h]
        \begin{center}
            \begin{circuitdiagram}{31}{12}
            \usgate
            \gate[\inputs{2}]{or}{5}{9}{R}{}{}    % OR
            \gate[\inputs{2}]{and}{12}{7}{R}{}{}  % AND
            \gate[\inputs{3}]{nor}{19}{3}{R}{}{}  % NOR
            \gate{not}{26}{3}{R}{}{}  % NOR
            \pin{1}{1}{L}{A}    % pin A
            \pin{1}{3}{L}{B}    % pin B
            \pin{1}{5}{L}{C}    % pin C
            \pin{1}{7}{L}{D}    % pin D
            \pin{1}{11}{L}{D1}  % pin D1
            \wire{2}{1}{16}{1}  % wire from pin A
            \wire{2}{3}{16}{3}  % wire from pin B
            \wire{2}{5}{9}{5}   % wire from pin C
            \wire{16}{7}{16}{5} % wire between AND and NOR
            \pin{30}{3}{R}{Z}   % pin Z
            \end{circuitdiagram}
        \end{center}
    \end{figure}
\end{center}

%%  ************    LibreSilicon's StdCellLibrary   *******************
%%
%%  Organisation:   Chipforge
%%                  Germany / European Union
%%
%%  Profile:        Chipforge focus on fine System-on-Chip Cores in
%%                  Verilog HDL Code which are easy understandable and
%%                  adjustable. For further information see
%%                          www.chipforge.org
%%                  there are projects from small cores up to PCBs, too.
%%
%%  File:           StdCellLib/Documents/Datasheets/Circuitry/OAOI221.tex
%%
%%  Purpose:        Circuit File for OAOI221
%%
%%  ************    LaTeX with circdia.sty package      ***************
%%
%%  ///////////////////////////////////////////////////////////////////
%%
%%  Copyright (c) 2018 - 2022 by
%%                  chipforge <stdcelllib@nospam.chipforge.org>
%%  All rights reserved.
%%
%%      This Standard Cell Library is licensed under the Libre Silicon
%%      public license; you can redistribute it and/or modify it under
%%      the terms of the Libre Silicon public license as published by
%%      the Libre Silicon alliance, either version 1 of the License, or
%%      (at your option) any later version.
%%
%%      This design is distributed in the hope that it will be useful,
%%      but WITHOUT ANY WARRANTY; without even the implied warranty of
%%      MERCHANTABILITY or FITNESS FOR A PARTICULAR PURPOSE.
%%      See the Libre Silicon Public License for more details.
%%
%%  ///////////////////////////////////////////////////////////////////
\begin{circuitdiagram}[draft]{25}{12}

    \usgate
    % ----  1st column  ----
    \pin{1}{1}{L}{A}
    \pin{1}{5}{L}{A1}
    \gate[\inputs{2}]{or}{5}{3}{R}{}{}

    % ----  2nd column  ----
    \pin{8}{7}{L}{B}
    \pin{8}{9}{L}{B1}
    \wire{9}{3}{9}{5}
    \gate[\inputs{3}]{and}{12}{7}{R}{}{}

    % ----  3rd column  ----
    \pin{15}{11}{L}{C}
    \gate[\inputs{2}]{nor}{19}{9}{R}{}{}

    % ----  result ----
    \pin{24}{9}{R}{Y}

\end{circuitdiagram}
 %%  ************    LibreSilicon's StdCellLibrary   *******************
%%
%%  Organisation:   Chipforge
%%                  Germany / European Union
%%
%%  Profile:        Chipforge focus on fine System-on-Chip Cores in
%%                  Verilog HDL Code which are easy understandable and
%%                  adjustable. For further information see
%%                          www.chipforge.org
%%                  there are projects from small cores up to PCBs, too.
%%
%%  File:           StdCellLib/Documents/Datasheets/Circuitry/OAO221.tex
%%
%%  Purpose:        Circuit File for OAO221
%%
%%  ************    LaTeX with circdia.sty package      ***************
%%
%%  ///////////////////////////////////////////////////////////////////
%%
%%  Copyright (c) 2018 - 2022 by
%%                  chipforge <stdcelllib@nospam.chipforge.org>
%%  All rights reserved.
%%
%%      This Standard Cell Library is licensed under the Libre Silicon
%%      public license; you can redistribute it and/or modify it under
%%      the terms of the Libre Silicon public license as published by
%%      the Libre Silicon alliance, either version 1 of the License, or
%%      (at your option) any later version.
%%
%%      This design is distributed in the hope that it will be useful,
%%      but WITHOUT ANY WARRANTY; without even the implied warranty of
%%      MERCHANTABILITY or FITNESS FOR A PARTICULAR PURPOSE.
%%      See the Libre Silicon Public License for more details.
%%
%%  ///////////////////////////////////////////////////////////////////
\begin{circuitdiagram}[draft]{31}{12}

    \usgate
    % ----  1st column  ----
    \pin{1}{1}{L}{A}
    \pin{1}{5}{L}{A1}
    \gate[\inputs{2}]{or}{5}{3}{R}{}{}

    % ----  2nd column  ----
    \pin{8}{7}{L}{B}
    \pin{8}{9}{L}{B1}
    \wire{9}{3}{9}{5}
    \gate[\inputs{3}]{and}{12}{7}{R}{}{}

    % ----  3rd column  ----
    \pin{15}{11}{L}{C}
    \gate[\inputs{2}]{nor}{19}{9}{R}{}{}

    % ----  4th column  ----
    \gate{not}{26}{9}{R}{}{}

    % ----  result ----
    \pin{30}{9}{R}{Z}

\end{circuitdiagram}

%%  ************    LibreSilicon's StdCellLibrary   *******************
%%
%%  Organisation:   Chipforge
%%                  Germany / European Union
%%
%%  Profile:        Chipforge focus on fine System-on-Chip Cores in
%%                  Verilog HDL Code which are easy understandable and
%%                  adjustable. For further information see
%%                          www.chipforge.org
%%                  there are projects from small cores up to PCBs, too.
%%
%%  File:           StdCellLib/Documents/Circuits/OAOI222.tex
%%
%%  Purpose:        Circuit File for OAOI222
%%
%%  ************    LaTeX with circdia.sty package      ***************
%%
%%  ///////////////////////////////////////////////////////////////////
%%
%%  Copyright (c) 2019 by chipforge <stdcelllib@nospam.chipforge.org>
%%  All rights reserved.
%%
%%      This Standard Cell Library is licensed under the Libre Silicon
%%      public license; you can redistribute it and/or modify it under
%%      the terms of the Libre Silicon public license as published by
%%      the Libre Silicon alliance, either version 1 of the License, or
%%      (at your option) any later version.
%%
%%      This design is distributed in the hope that it will be useful,
%%      but WITHOUT ANY WARRANTY; without even the implied warranty of
%%      MERCHANTABILITY or FITNESS FOR A PARTICULAR PURPOSE.
%%      See the Libre Silicon Public License for more details.
%%
%%  ///////////////////////////////////////////////////////////////////
\begin{circuitdiagram}{25}{12}

    \usgate
    \gate[\inputs{2}]{or}{5}{11}{R}{}{}   % OR
    \gate[\inputs{3}]{and}{12}{7}{R}{}{}  % AND
    \gate[\inputs{3}]{nor}{19}{3}{R}{}{}  % NOR
    \pin{1}{1}{L}{A}    % pin A
    \pin{1}{3}{L}{A1}   % pin A1
    \pin{1}{5}{L}{B}    % pin B
    \pin{1}{7}{L}{B1}   % pin B1
    \pin{1}{9}{L}{C}    % pin C
    \pin{1}{13}{L}{C1}  % pin C1
    \wire{9}{9}{9}{11}  % wire between OR and AND
    \wire{16}{5}{16}{7} % wire between AND and NOR
    \wire{2}{1}{16}{1}  % wire from pin A
    \wire{2}{3}{16}{3}  % wire from pin A1
    \wire{2}{5}{9}{5}   % wire from pin B
    \wire{2}{7}{9}{7}   % wire from pin B1
    \pin{24}{3}{R}{Y}   % pin Y

\end{circuitdiagram}
 %%  ************    LibreSilicon's StdCellLibrary   *******************
%%
%%  Organisation:   Chipforge
%%                  Germany / European Union
%%
%%  Profile:        Chipforge focus on fine System-on-Chip Cores in
%%                  Verilog HDL Code which are easy understandable and
%%                  adjustable. For further information see
%%                          www.chipforge.org
%%                  there are projects from small cores up to PCBs, too.
%%
%%  File:           StdCellLib/Documents/Datasheets/Circuitry/OAO222.tex
%%
%%  Purpose:        Circuit File for OAO222
%%
%%  ************    LaTeX with circdia.sty package      ***************
%%
%%  ///////////////////////////////////////////////////////////////////
%%
%%  Copyright (c) 2018 - 2022 by
%%                  chipforge <stdcelllib@nospam.chipforge.org>
%%  All rights reserved.
%%
%%      This Standard Cell Library is licensed under the Libre Silicon
%%      public license; you can redistribute it and/or modify it under
%%      the terms of the Libre Silicon public license as published by
%%      the Libre Silicon alliance, either version 1 of the License, or
%%      (at your option) any later version.
%%
%%      This design is distributed in the hope that it will be useful,
%%      but WITHOUT ANY WARRANTY; without even the implied warranty of
%%      MERCHANTABILITY or FITNESS FOR A PARTICULAR PURPOSE.
%%      See the Libre Silicon Public License for more details.
%%
%%  ///////////////////////////////////////////////////////////////////
\begin{circuitdiagram}[draft]{31}{14}

    \usgate
    % ----  1st column  ----
    \pin{1}{1}{L}{A}
    \pin{1}{5}{L}{A1}
    \gate[\inputs{2}]{or}{5}{3}{R}{}{}

    % ----  2nd column  ----
    \pin{8}{7}{L}{B}
    \pin{8}{9}{L}{B1}
    \wire{9}{3}{9}{5}
    \gate[\inputs{3}]{and}{12}{7}{R}{}{}

    % ----  3rd column  ----
    \pin{15}{11}{L}{C}
    \pin{15}{13}{L}{C1}
    \wire{16}{7}{16}{9}
    \gate[\inputs{3}]{nor}{19}{11}{R}{}{}

    % ----  4th column  ----
    \gate{not}{26}{11}{R}{}{}

    % ----  result ----
    \pin{30}{11}{R}{Z}

\end{circuitdiagram}

%%  ************    LibreSilicon's StdCellLibrary   *******************
%%
%%  Organisation:   Chipforge
%%                  Germany / European Union
%%
%%  Profile:        Chipforge focus on fine System-on-Chip Cores in
%%                  Verilog HDL Code which are easy understandable and
%%                  adjustable. For further information see
%%                          www.chipforge.org
%%                  there are projects from small cores up to PCBs, too.
%%
%%  File:           StdCellLib/Documents/Datasheets/Circuitry/OAOI232.tex
%%
%%  Purpose:        Circuit File for OAOI232
%%
%%  ************    LaTeX with circdia.sty package      ***************
%%
%%  ///////////////////////////////////////////////////////////////////
%%
%%  Copyright (c) 2018 - 2022 by
%%                  chipforge <stdcelllib@nospam.chipforge.org>
%%  All rights reserved.
%%
%%      This Standard Cell Library is licensed under the Libre Silicon
%%      public license; you can redistribute it and/or modify it under
%%      the terms of the Libre Silicon public license as published by
%%      the Libre Silicon alliance, either version 1 of the License, or
%%      (at your option) any later version.
%%
%%      This design is distributed in the hope that it will be useful,
%%      but WITHOUT ANY WARRANTY; without even the implied warranty of
%%      MERCHANTABILITY or FITNESS FOR A PARTICULAR PURPOSE.
%%      See the Libre Silicon Public License for more details.
%%
%%  ///////////////////////////////////////////////////////////////////
\begin{circuitdiagram}[draft]{25}{15}

    \usgate
    % ----  1st column  ----
    \pin{1}{1}{L}{A}
    \pin{1}{5}{L}{A1}
    \gate[\inputs{2}]{or}{5}{3}{R}{}{}

    % ----  2nd column  ----
    \pin{8}{7}{L}{B}
    \pin{8}{9}{L}{B1}
    \pin{8}{11}{L}{B2}
    \wire{9}{3}{9}{5}
    \gate[\inputs{4}]{and}{12}{8}{R}{}{}

    % ----  3rd column  ----
    \wire{16}{8}{16}{10}
    \pin{15}{12}{L}{C}
    \pin{15}{14}{L}{C1}
    \gate[\inputs{3}]{nor}{19}{12}{R}{}{}

    % ----  result ----
    \pin{24}{12}{R}{Y}

\end{circuitdiagram}
 %%  ************    LibreSilicon's StdCellLibrary   *******************
%%
%%  Organisation:   Chipforge
%%                  Germany / European Union
%%
%%  Profile:        Chipforge focus on fine System-on-Chip Cores in
%%                  Verilog HDL Code which are easy understandable and
%%                  adjustable. For further information see
%%                          www.chipforge.org
%%                  there are projects from small cores up to PCBs, too.
%%
%%  File:           StdCellLib/Documents/Datasheets/Circuitry/OAO232.tex
%%
%%  Purpose:        Circuit File for OAO232
%%
%%  ************    LaTeX with circdia.sty package      ***************
%%
%%  ///////////////////////////////////////////////////////////////////
%%
%%  Copyright (c) 2018 - 2022 by
%%                  chipforge <stdcelllib@nospam.chipforge.org>
%%  All rights reserved.
%%
%%      This Standard Cell Library is licensed under the Libre Silicon
%%      public license; you can redistribute it and/or modify it under
%%      the terms of the Libre Silicon public license as published by
%%      the Libre Silicon alliance, either version 1 of the License, or
%%      (at your option) any later version.
%%
%%      This design is distributed in the hope that it will be useful,
%%      but WITHOUT ANY WARRANTY; without even the implied warranty of
%%      MERCHANTABILITY or FITNESS FOR A PARTICULAR PURPOSE.
%%      See the Libre Silicon Public License for more details.
%%
%%  ///////////////////////////////////////////////////////////////////
\begin{circuitdiagram}[draft]{31}{15}

    \usgate
    % ----  1st column  ----
    \pin{1}{1}{L}{A}
    \pin{1}{5}{L}{A1}
    \gate[\inputs{2}]{or}{5}{3}{R}{}{}

    % ----  2nd column  ----
    \pin{8}{7}{L}{B}
    \pin{8}{9}{L}{B1}
    \pin{8}{11}{L}{B2}
    \wire{9}{3}{9}{5}
    \gate[\inputs{4}]{and}{12}{8}{R}{}{}

    % ----  3rd column  ----
    \wire{16}{8}{16}{10}
    \pin{15}{12}{L}{C}
    \pin{15}{14}{L}{C1}
    \gate[\inputs{3}]{nor}{19}{12}{R}{}{}

    % ----  4th column  ----
    \gate{not}{26}{12}{R}{}{}

    % ----  result ----
    \pin{30}{12}{R}{Z}

\end{circuitdiagram}

%%  ************    LibreSilicon's StdCellLibrary   *******************
%%
%%  Organisation:   Chipforge
%%                  Germany / European Union
%%
%%  Profile:        Chipforge focus on fine System-on-Chip Cores in
%%                  Verilog HDL Code which are easy understandable and
%%                  adjustable. For further information see
%%                          www.chipforge.org
%%                  there are projects from small cores up to PCBs, too.
%%
%%  File:           StdCellLib/Documents/Datasheets/Circuitry/OAOI311.tex
%%
%%  Purpose:        Circuit File for OAOI311
%%
%%  ************    LaTeX with circdia.sty package      ***************
%%
%%  ///////////////////////////////////////////////////////////////////
%%
%%  Copyright (c) 2018 - 2022 by
%%                  chipforge <stdcelllib@nospam.chipforge.org>
%%  All rights reserved.
%%
%%      This Standard Cell Library is licensed under the Libre Silicon
%%      public license; you can redistribute it and/or modify it under
%%      the terms of the Libre Silicon public license as published by
%%      the Libre Silicon alliance, either version 1 of the License, or
%%      (at your option) any later version.
%%
%%      This design is distributed in the hope that it will be useful,
%%      but WITHOUT ANY WARRANTY; without even the implied warranty of
%%      MERCHANTABILITY or FITNESS FOR A PARTICULAR PURPOSE.
%%      See the Libre Silicon Public License for more details.
%%
%%  ///////////////////////////////////////////////////////////////////
\begin{circuitdiagram}[draft]{25}{10}

    \usgate
    % ----  1st column  ----
    \pin{1}{1}{L}{A}
    \pin{1}{3}{L}{A1}
    \pin{1}{5}{L}{A2}
    \gate[\inputs{3}]{or}{5}{3}{R}{}{}

    % ----  2nd column  ----
    \pin{8}{7}{L}{B}
    \gate[\inputs{2}]{and}{12}{5}{R}{}{}

    % ----  3rd column  ----
    \pin{15}{9}{L}{C}
    \gate[\inputs{2}]{nor}{19}{7}{R}{}{}

    % ----  result ----
    \pin{24}{7}{R}{Y}

\end{circuitdiagram}
 %%  ************    LibreSilicon's StdCellLibrary   *******************
%%
%%  Organisation:   Chipforge
%%                  Germany / European Union
%%
%%  Profile:        Chipforge focus on fine System-on-Chip Cores in
%%                  Verilog HDL Code which are easy understandable and
%%                  adjustable. For further information see
%%                          www.chipforge.org
%%                  there are projects from small cores up to PCBs, too.
%%
%%  File:           StdCellLib/Documents/Circuits/OAO311.tex
%%
%%  Purpose:        Circuit File for OAO311
%%
%%  ************    LaTeX with circdia.sty package      ***************
%%
%%  ///////////////////////////////////////////////////////////////////
%%
%%  Copyright (c) 2019 by chipforge <stdcelllib@nospam.chipforge.org>
%%  All rights reserved.
%%
%%      This Standard Cell Library is licensed under the Libre Silicon
%%      public license; you can redistribute it and/or modify it under
%%      the terms of the Libre Silicon public license as published by
%%      the Libre Silicon alliance, either version 1 of the License, or
%%      (at your option) any later version.
%%
%%      This design is distributed in the hope that it will be useful,
%%      but WITHOUT ANY WARRANTY; without even the implied warranty of
%%      MERCHANTABILITY or FITNESS FOR A PARTICULAR PURPOSE.
%%      See the Libre Silicon Public License for more details.
%%
%%  ///////////////////////////////////////////////////////////////////
\begin{center}
    Circuit
    \begin{figure}[h]
        \begin{center}
            \begin{circuitdiagram}{31}{10}
            \usgate
            \gate[\inputs{3}]{or}{5}{7}{R}{}{}   % OR
            \gate[\inputs{2}]{and}{12}{5}{R}{}{} % AND
            \gate[\inputs{2}]{nor}{19}{3}{R}{}{} % NOR
            \gate{not}{26}{3}{R}{}{}  % NOT
            \pin{1}{1}{L}{A}    % pin A
            \pin{1}{3}{L}{B}    % pin B
            \pin{1}{5}{L}{C}    % pin C
            \pin{1}{7}{L}{C1}   % pin C1
            \pin{1}{9}{L}{C2}   % pin C2
            \wire{2}{1}{16}{1}  % wire from pin A
            \wire{2}{3}{9}{3}   % wire from pin B
            \pin{30}{3}{R}{Z}   % pin Z
            \end{circuitdiagram}
        \end{center}
    \end{figure}
\end{center}


%%  ************    LibreSilicon's StdCellLibrary   *******************
%%
%%  Organisation:   Chipforge
%%                  Germany / European Union
%%
%%  Profile:        Chipforge focus on fine System-on-Chip Cores in
%%                  Verilog HDL Code which are easy understandable and
%%                  adjustable. For further information see
%%                          www.chipforge.org
%%                  there are projects from small cores up to PCBs, too.
%%
%%  File:           StdCellLib/Documents/Book/section-AOOAI_complex.tex
%%
%%  Purpose:        Section Level File for Standard Cell Library Documentation
%%
%%  ************    LaTeX with circdia.sty package      ***************
%%
%%  ///////////////////////////////////////////////////////////////////
%%
%%  Copyright (c) 2018 - 2022 by
%%                  chipforge <stdcelllib@nospam.chipforge.org>
%%  All rights reserved.
%%
%%      This Standard Cell Library is licensed under the Libre Silicon
%%      public license; you can redistribute it and/or modify it under
%%      the terms of the Libre Silicon public license as published by
%%      the Libre Silicon alliance, either version 1 of the License, or
%%      (at your option) any later version.
%%
%%      This design is distributed in the hope that it will be useful,
%%      but WITHOUT ANY WARRANTY; without even the implied warranty of
%%      MERCHANTABILITY or FITNESS FOR A PARTICULAR PURPOSE.
%%      See the Libre Silicon Public License for more details.
%%
%%  ///////////////////////////////////////////////////////////////////
\section{AND-OR-OR-AND(-Invert) Complex Gates}

%%  ************    LibreSilicon's StdCellLibrary   *******************
%%
%%  Organisation:   Chipforge
%%                  Germany / European Union
%%
%%  Profile:        Chipforge focus on fine System-on-Chip Cores in
%%                  Verilog HDL Code which are easy understandable and
%%                  adjustable. For further information see
%%                          www.chipforge.org
%%                  there are projects from small cores up to PCBs, too.
%%
%%  File:           StdCellLib/Documents/Datasheets/Circuitry/AOOAI212.tex
%%
%%  Purpose:        Circuit File for AOOAI212
%%
%%  ************    LaTeX with circdia.sty package      ***************
%%
%%  ///////////////////////////////////////////////////////////////////
%%
%%  Copyright (c) 2018 - 2022 by
%%                  chipforge <stdcelllib@nospam.chipforge.org>
%%  All rights reserved.
%%
%%      This Standard Cell Library is licensed under the Libre Silicon
%%      public license; you can redistribute it and/or modify it under
%%      the terms of the Libre Silicon public license as published by
%%      the Libre Silicon alliance, either version 1 of the License, or
%%      (at your option) any later version.
%%
%%      This design is distributed in the hope that it will be useful,
%%      but WITHOUT ANY WARRANTY; without even the implied warranty of
%%      MERCHANTABILITY or FITNESS FOR A PARTICULAR PURPOSE.
%%      See the Libre Silicon Public License for more details.
%%
%%  ///////////////////////////////////////////////////////////////////
\begin{circuitdiagram}[draft]{25}{14}

    \usgate
    % ----  1st column  ----
    \pin{1}{1}{L}{A}
    \pin{1}{5}{L}{A1}
    \gate[\inputs{2}]{and}{5}{3}{R}{}{}

    % ----  2nd column  ----
    \pin{8}{7}{L}{B}
    \gate[\inputs{2}]{or}{12}{5}{R}{}{}

    \pin{8}{9}{L}{C}
    \pin{8}{13}{L}{C1}
    \gate[\inputs{2}]{or}{12}{11}{R}{}{}

    % ----  3rd column  ----
    \wire{16}{5}{16}{6}
    \wire{16}{10}{16}{11}
    \gate[\inputs{2}]{nand}{19}{8}{R}{}{}

    % ----  result ----
    \pin{24}{8}{R}{Y}

\end{circuitdiagram}
 %%  ************    LibreSilicon's StdCellLibrary   *******************
%%
%%  Organisation:   Chipforge
%%                  Germany / European Union
%%
%%  Profile:        Chipforge focus on fine System-on-Chip Cores in
%%                  Verilog HDL Code which are easy understandable and
%%                  adjustable. For further information see
%%                          www.chipforge.org
%%                  there are projects from small cores up to PCBs, too.
%%
%%  File:           StdCellLib/Documents/Circuits/AOOA212.tex
%%
%%  Purpose:        Circuit File for AOOA212
%%
%%  ************    LaTeX with circdia.sty package      ***************
%%
%%  ///////////////////////////////////////////////////////////////////
%%
%%  Copyright (c) 2019 by chipforge <stdcelllib@nospam.chipforge.org>
%%  All rights reserved.
%%
%%      This Standard Cell Library is licensed under the Libre Silicon
%%      public license; you can redistribute it and/or modify it under
%%      the terms of the Libre Silicon public license as published by
%%      the Libre Silicon alliance, either version 1 of the License, or
%%      (at your option) any later version.
%%
%%      This design is distributed in the hope that it will be useful,
%%      but WITHOUT ANY WARRANTY; without even the implied warranty of
%%      MERCHANTABILITY or FITNESS FOR A PARTICULAR PURPOSE.
%%      See the Libre Silicon Public License for more details.
%%
%%  ///////////////////////////////////////////////////////////////////
\begin{circuitdiagram}{31}{14}

    \usgate
    \gate[\inputs{2}]{and}{5}{11}{R}{}{}  % AND
    \gate[\inputs{2}]{or}{12}{9}{R}{}{}   % OR
    \gate[\inputs{2}]{or}{12}{3}{R}{}{}   % OR
    \gate[\inputs{2}]{nand}{19}{6}{R}{}{} % NAND
    \gate{not}{26}{6}{R}{}{} % NOT
    \pin{1}{1}{L}{A}     % pin A
    \pin{1}{5}{L}{A1}    % pin A1
    \pin{1}{7}{L}{B}     % pin B
    \pin{1}{9}{L}{C}     % pin C
    \pin{1}{13}{L}{C1}   % pin C1
    \wire{2}{1}{9}{1}    % wire pin A
    \wire{2}{5}{9}{5}    % wire pin A1
    \wire{2}{7}{9}{7}    % wire pin B
    \wire{16}{3}{16}{4}  % wire between OR and NAND 
    \wire{16}{9}{16}{8}  % wire between OR and NAND 
    \pin{30}{6}{R}{Z}    % pin Z

\end{circuitdiagram}

%%  ************    LibreSilicon's StdCellLibrary   *******************
%%
%%  Organisation:   Chipforge
%%                  Germany / European Union
%%
%%  Profile:        Chipforge focus on fine System-on-Chip Cores in
%%                  Verilog HDL Code which are easy understandable and
%%                  adjustable. For further information see
%%                          www.chipforge.org
%%                  there are projects from small cores up to PCBs, too.
%%
%%  File:           StdCellLib/Documents/Datasheets/Circuitry/AOOAI222.tex
%%
%%  Purpose:        Circuit File for AOOAI222
%%
%%  ************    LaTeX with circdia.sty package      ***************
%%
%%  ///////////////////////////////////////////////////////////////////
%%
%%  Copyright (c) 2018 - 2022 by
%%                  chipforge <stdcelllib@nospam.chipforge.org>
%%  All rights reserved.
%%
%%      This Standard Cell Library is licensed under the Libre Silicon
%%      public license; you can redistribute it and/or modify it under
%%      the terms of the Libre Silicon public license as published by
%%      the Libre Silicon alliance, either version 1 of the License, or
%%      (at your option) any later version.
%%
%%      This design is distributed in the hope that it will be useful,
%%      but WITHOUT ANY WARRANTY; without even the implied warranty of
%%      MERCHANTABILITY or FITNESS FOR A PARTICULAR PURPOSE.
%%      See the Libre Silicon Public License for more details.
%%
%%  ///////////////////////////////////////////////////////////////////
\begin{circuitdiagram}[draft]{25}{16}

    \usgate
    % ----  1st column  ----
    \pin{1}{1}{L}{A}
    \pin{1}{5}{L}{A1}
    \gate[\inputs{2}]{and}{5}{3}{R}{}{}

    % ----  2nd column  ----
    \wire{9}{3}{9}{5}
    \pin{8}{7}{L}{B}
    \pin{8}{9}{L}{B1}
    \gate[\inputs{3}]{or}{12}{7}{R}{}{}

    \pin{8}{11}{L}{C}
    \pin{8}{15}{L}{C1}
    \gate[\inputs{2}]{or}{12}{13}{R}{}{}

    % ----  3rd column  ----
    \wire{16}{7}{16}{9}
    \gate[\inputs{2}]{nand}{19}{11}{R}{}{}

    % ----  result ----
    \pin{24}{11}{R}{Y}

\end{circuitdiagram}
 %%  ************    LibreSilicon's StdCellLibrary   *******************
%%
%%  Organisation:   Chipforge
%%                  Germany / European Union
%%
%%  Profile:        Chipforge focus on fine System-on-Chip Cores in
%%                  Verilog HDL Code which are easy understandable and
%%                  adjustable. For further information see
%%                          www.chipforge.org
%%                  there are projects from small cores up to PCBs, too.
%%
%%  File:           StdCellLib/Documents/Datasheets/Circuitry/AOOA222.tex
%%
%%  Purpose:        Circuit File for AOOA222
%%
%%  ************    LaTeX with circdia.sty package      ***************
%%
%%  ///////////////////////////////////////////////////////////////////
%%
%%  Copyright (c) 2018 - 2022 by
%%                  chipforge <stdcelllib@nospam.chipforge.org>
%%  All rights reserved.
%%
%%      This Standard Cell Library is licensed under the Libre Silicon
%%      public license; you can redistribute it and/or modify it under
%%      the terms of the Libre Silicon public license as published by
%%      the Libre Silicon alliance, either version 1 of the License, or
%%      (at your option) any later version.
%%
%%      This design is distributed in the hope that it will be useful,
%%      but WITHOUT ANY WARRANTY; without even the implied warranty of
%%      MERCHANTABILITY or FITNESS FOR A PARTICULAR PURPOSE.
%%      See the Libre Silicon Public License for more details.
%%
%%  ///////////////////////////////////////////////////////////////////
\begin{circuitdiagram}[draft]{31}{16}

    \usgate
    % ----  1st column  ----
    \pin{1}{1}{L}{A}
    \pin{1}{5}{L}{A1}
    \gate[\inputs{2}]{and}{5}{3}{R}{}{}

    % ----  2nd column  ----
    \wire{9}{3}{9}{5}
    \pin{8}{7}{L}{B}
    \pin{8}{9}{L}{B1}
    \gate[\inputs{3}]{or}{12}{7}{R}{}{}

    \pin{8}{11}{L}{C}
    \pin{8}{15}{L}{C1}
    \gate[\inputs{2}]{or}{12}{13}{R}{}{}

    % ----  3rd column  ----
    \wire{16}{7}{16}{9}
    \gate[\inputs{2}]{nand}{19}{11}{R}{}{}

    % ----  4th column  ----
    \gate{not}{26}{11}{R}{}{}

    % ----  result ----
    \pin{30}{11}{R}{Z}

\end{circuitdiagram}

%%  ************    LibreSilicon's StdCellLibrary   *******************
%%
%%  Organisation:   Chipforge
%%                  Germany / European Union
%%
%%  Profile:        Chipforge focus on fine System-on-Chip Cores in
%%                  Verilog HDL Code which are easy understandable and
%%                  adjustable. For further information see
%%                          www.chipforge.org
%%                  there are projects from small cores up to PCBs, too.
%%
%%  File:           StdCellLib/Documents/Datasheets/Circuitry/AOOAI223.tex
%%
%%  Purpose:        Circuit File for AOOAI223
%%
%%  ************    LaTeX with circdia.sty package      ***************
%%
%%  ///////////////////////////////////////////////////////////////////
%%
%%  Copyright (c) 2018 - 2022 by
%%                  chipforge <stdcelllib@nospam.chipforge.org>
%%  All rights reserved.
%%
%%      This Standard Cell Library is licensed under the Libre Silicon
%%      public license; you can redistribute it and/or modify it under
%%      the terms of the Libre Silicon public license as published by
%%      the Libre Silicon alliance, either version 1 of the License, or
%%      (at your option) any later version.
%%
%%      This design is distributed in the hope that it will be useful,
%%      but WITHOUT ANY WARRANTY; without even the implied warranty of
%%      MERCHANTABILITY or FITNESS FOR A PARTICULAR PURPOSE.
%%      See the Libre Silicon Public License for more details.
%%
%%  ///////////////////////////////////////////////////////////////////
\begin{circuitdiagram}[draft]{25}{16}

    \usgate
    % ----  1st column  ----
    \pin{1}{1}{L}{A}
    \pin{1}{5}{L}{A1}
    \gate[\inputs{2}]{and}{5}{3}{R}{}{}

    % ----  2nd column  ----
    \wire{9}{3}{9}{5}
    \pin{8}{7}{L}{B}
    \pin{8}{9}{L}{B1}
    \gate[\inputs{3}]{or}{12}{7}{R}{}{}

    \pin{8}{11}{L}{C}
    \pin{8}{13}{L}{C1}
    \pin{8}{15}{L}{C1}
    \gate[\inputs{3}]{or}{12}{13}{R}{}{}

    % ----  3rd column  ----
    \wire{16}{7}{16}{9}
    \gate[\inputs{2}]{nand}{19}{11}{R}{}{}

    % ----  result ----
    \pin{24}{11}{R}{Y}

\end{circuitdiagram}
 %%  ************    LibreSilicon's StdCellLibrary   *******************
%%
%%  Organisation:   Chipforge
%%                  Germany / European Union
%%
%%  Profile:        Chipforge focus on fine System-on-Chip Cores in
%%                  Verilog HDL Code which are easy understandable and
%%                  adjustable. For further information see
%%                          www.chipforge.org
%%                  there are projects from small cores up to PCBs, too.
%%
%%  File:           StdCellLib/Documents/Datasheets/Circuitry/AOOA223.tex
%%
%%  Purpose:        Circuit File for AOOA223
%%
%%  ************    LaTeX with circdia.sty package      ***************
%%
%%  ///////////////////////////////////////////////////////////////////
%%
%%  Copyright (c) 2018 - 2022 by
%%                  chipforge <stdcelllib@nospam.chipforge.org>
%%  All rights reserved.
%%
%%      This Standard Cell Library is licensed under the Libre Silicon
%%      public license; you can redistribute it and/or modify it under
%%      the terms of the Libre Silicon public license as published by
%%      the Libre Silicon alliance, either version 1 of the License, or
%%      (at your option) any later version.
%%
%%      This design is distributed in the hope that it will be useful,
%%      but WITHOUT ANY WARRANTY; without even the implied warranty of
%%      MERCHANTABILITY or FITNESS FOR A PARTICULAR PURPOSE.
%%      See the Libre Silicon Public License for more details.
%%
%%  ///////////////////////////////////////////////////////////////////
\begin{circuitdiagram}[draft]{31}{16}

    \usgate
    % ----  1st column  ----
    \pin{1}{1}{L}{A}
    \pin{1}{5}{L}{A1}
    \gate[\inputs{2}]{and}{5}{3}{R}{}{}

    % ----  2nd column  ----
    \wire{9}{3}{9}{5}
    \pin{8}{7}{L}{B}
    \pin{8}{9}{L}{B1}
    \gate[\inputs{3}]{or}{12}{7}{R}{}{}

    \pin{8}{11}{L}{C}
    \pin{8}{13}{L}{C1}
    \pin{8}{15}{L}{C1}
    \gate[\inputs{3}]{or}{12}{13}{R}{}{}

    % ----  3rd column  ----
    \wire{16}{7}{16}{9}
    \gate[\inputs{2}]{nand}{19}{11}{R}{}{}

    % ----  4th column  ----
    \gate{not}{26}{11}{R}{}{}

    % ----  result ----
    \pin{30}{11}{R}{Z}

\end{circuitdiagram}

%%  ************    LibreSilicon's StdCellLibrary   *******************
%%
%%  Organisation:   Chipforge
%%                  Germany / European Union
%%
%%  Profile:        Chipforge focus on fine System-on-Chip Cores in
%%                  Verilog HDL Code which are easy understandable and
%%                  adjustable. For further information see
%%                          www.chipforge.org
%%                  there are projects from small cores up to PCBs, too.
%%
%%  File:           StdCellLib/Documents/Datasheets/Circuitry/AOOAI232.tex
%%
%%  Purpose:        Circuit File for AOOAI232
%%
%%  ************    LaTeX with circdia.sty package      ***************
%%
%%  ///////////////////////////////////////////////////////////////////
%%
%%  Copyright (c) 2018 - 2022 by
%%                  chipforge <stdcelllib@nospam.chipforge.org>
%%  All rights reserved.
%%
%%      This Standard Cell Library is licensed under the Libre Silicon
%%      public license; you can redistribute it and/or modify it under
%%      the terms of the Libre Silicon public license as published by
%%      the Libre Silicon alliance, either version 1 of the License, or
%%      (at your option) any later version.
%%
%%      This design is distributed in the hope that it will be useful,
%%      but WITHOUT ANY WARRANTY; without even the implied warranty of
%%      MERCHANTABILITY or FITNESS FOR A PARTICULAR PURPOSE.
%%      See the Libre Silicon Public License for more details.
%%
%%  ///////////////////////////////////////////////////////////////////
\begin{circuitdiagram}[draft]{25}{18}

    \usgate
    % ----  1st column  ----
    \pin{1}{1}{L}{A}
    \pin{1}{5}{L}{A1}
    \gate[\inputs{2}]{and}{5}{3}{R}{}{}

    % ----  2nd column  ----
    \wire{9}{3}{9}{5}
    \pin{8}{7}{L}{B}
    \pin{8}{9}{L}{B1}
    \pin{8}{11}{L}{B2}
    \gate[\inputs{4}]{or}{12}{8}{R}{}{}

    \pin{8}{13}{L}{C}
    \pin{8}{17}{L}{C1}
    \gate[\inputs{2}]{or}{12}{15}{R}{}{}

    % ----  3rd column  ----
    \wire{16}{8}{16}{11}
    \gate[\inputs{2}]{nand}{19}{13}{R}{}{}

    % ----  result ----
    \pin{24}{13}{R}{Y}

\end{circuitdiagram}
 %%  ************    LibreSilicon's StdCellLibrary   *******************
%%
%%  Organisation:   Chipforge
%%                  Germany / European Union
%%
%%  Profile:        Chipforge focus on fine System-on-Chip Cores in
%%                  Verilog HDL Code which are easy understandable and
%%                  adjustable. For further information see
%%                          www.chipforge.org
%%                  there are projects from small cores up to PCBs, too.
%%
%%  File:           StdCellLib/Documents/Datasheets/Circuitry/AOOA232.tex
%%
%%  Purpose:        Circuit File for AOOA232
%%
%%  ************    LaTeX with circdia.sty package      ***************
%%
%%  ///////////////////////////////////////////////////////////////////
%%
%%  Copyright (c) 2018 - 2022 by
%%                  chipforge <stdcelllib@nospam.chipforge.org>
%%  All rights reserved.
%%
%%      This Standard Cell Library is licensed under the Libre Silicon
%%      public license; you can redistribute it and/or modify it under
%%      the terms of the Libre Silicon public license as published by
%%      the Libre Silicon alliance, either version 1 of the License, or
%%      (at your option) any later version.
%%
%%      This design is distributed in the hope that it will be useful,
%%      but WITHOUT ANY WARRANTY; without even the implied warranty of
%%      MERCHANTABILITY or FITNESS FOR A PARTICULAR PURPOSE.
%%      See the Libre Silicon Public License for more details.
%%
%%  ///////////////////////////////////////////////////////////////////
\begin{circuitdiagram}[draft]{31}{18}

    \usgate
    % ----  1st column  ----
    \pin{1}{1}{L}{A}
    \pin{1}{5}{L}{A1}
    \gate[\inputs{2}]{and}{5}{3}{R}{}{}

    % ----  2nd column  ----
    \wire{9}{3}{9}{5}
    \pin{8}{7}{L}{B}
    \pin{8}{9}{L}{B1}
    \pin{8}{11}{L}{B2}
    \gate[\inputs{4}]{or}{12}{8}{R}{}{}

    \pin{8}{13}{L}{C}
    \pin{8}{17}{L}{C1}
    \gate[\inputs{2}]{or}{12}{15}{R}{}{}

    % ----  3rd column  ----
    \wire{16}{8}{16}{11}
    \gate[\inputs{2}]{nand}{19}{13}{R}{}{}

    % ----  4th column  ----
    \gate{not}{26}{13}{R}{}{}

    % ----  result ----
    \pin{30}{13}{R}{Z}

\end{circuitdiagram}

%%  ************    LibreSilicon's StdCellLibrary   *******************
%%
%%  Organisation:   Chipforge
%%                  Germany / European Union
%%
%%  Profile:        Chipforge focus on fine System-on-Chip Cores in
%%                  Verilog HDL Code which are easy understandable and
%%                  adjustable. For further information see
%%                          www.chipforge.org
%%                  there are projects from small cores up to PCBs, too.
%%
%%  File:           StdCellLib/Documents/Datasheets/Circuitry/AOOAI233.tex
%%
%%  Purpose:        Circuit File for AOOAI233
%%
%%  ************    LaTeX with circdia.sty package      ***************
%%
%%  ///////////////////////////////////////////////////////////////////
%%
%%  Copyright (c) 2018 - 2022 by
%%                  chipforge <stdcelllib@nospam.chipforge.org>
%%  All rights reserved.
%%
%%      This Standard Cell Library is licensed under the Libre Silicon
%%      public license; you can redistribute it and/or modify it under
%%      the terms of the Libre Silicon public license as published by
%%      the Libre Silicon alliance, either version 1 of the License, or
%%      (at your option) any later version.
%%
%%      This design is distributed in the hope that it will be useful,
%%      but WITHOUT ANY WARRANTY; without even the implied warranty of
%%      MERCHANTABILITY or FITNESS FOR A PARTICULAR PURPOSE.
%%      See the Libre Silicon Public License for more details.
%%
%%  ///////////////////////////////////////////////////////////////////
\begin{circuitdiagram}[draft]{25}{18}

    \usgate
    % ----  1st column  ----
    \pin{1}{1}{L}{A}
    \pin{1}{5}{L}{A1}
    \gate[\inputs{2}]{and}{5}{3}{R}{}{}

    % ----  2nd column  ----
    \wire{9}{3}{9}{5}
    \pin{8}{7}{L}{B}
    \pin{8}{9}{L}{B1}
    \pin{8}{11}{L}{B2}
    \gate[\inputs{4}]{or}{12}{8}{R}{}{}

    \pin{8}{13}{L}{C}
    \pin{8}{15}{L}{C1}
    \pin{8}{17}{L}{C2}
    \gate[\inputs{3}]{or}{12}{15}{R}{}{}

    % ----  3rd column  ----
    \wire{16}{8}{16}{11}
    \gate[\inputs{2}]{nand}{19}{13}{R}{}{}

    % ----  result ----
    \pin{24}{13}{R}{Y}

\end{circuitdiagram}
 %%  ************    LibreSilicon's StdCellLibrary   *******************
%%
%%  Organisation:   Chipforge
%%                  Germany / European Union
%%
%%  Profile:        Chipforge focus on fine System-on-Chip Cores in
%%                  Verilog HDL Code which are easy understandable and
%%                  adjustable. For further information see
%%                          www.chipforge.org
%%                  there are projects from small cores up to PCBs, too.
%%
%%  File:           StdCellLib/Documents/Datasheets/Circuitry/AOOA233.tex
%%
%%  Purpose:        Circuit File for AOOA233
%%
%%  ************    LaTeX with circdia.sty package      ***************
%%
%%  ///////////////////////////////////////////////////////////////////
%%
%%  Copyright (c) 2018 - 2022 by
%%                  chipforge <stdcelllib@nospam.chipforge.org>
%%  All rights reserved.
%%
%%      This Standard Cell Library is licensed under the Libre Silicon
%%      public license; you can redistribute it and/or modify it under
%%      the terms of the Libre Silicon public license as published by
%%      the Libre Silicon alliance, either version 1 of the License, or
%%      (at your option) any later version.
%%
%%      This design is distributed in the hope that it will be useful,
%%      but WITHOUT ANY WARRANTY; without even the implied warranty of
%%      MERCHANTABILITY or FITNESS FOR A PARTICULAR PURPOSE.
%%      See the Libre Silicon Public License for more details.
%%
%%  ///////////////////////////////////////////////////////////////////
\begin{circuitdiagram}[draft]{31}{18}

    \usgate
    % ----  1st column  ----
    \pin{1}{1}{L}{A}
    \pin{1}{5}{L}{A1}
    \gate[\inputs{2}]{and}{5}{3}{R}{}{}

    % ----  2nd column  ----
    \wire{9}{3}{9}{5}
    \pin{8}{7}{L}{B}
    \pin{8}{9}{L}{B1}
    \pin{8}{11}{L}{B2}
    \gate[\inputs{4}]{or}{12}{8}{R}{}{}

    \pin{8}{13}{L}{C}
    \pin{8}{15}{L}{C1}
    \pin{8}{17}{L}{C2}
    \gate[\inputs{3}]{or}{12}{15}{R}{}{}

    % ----  3rd column  ----
    \wire{16}{8}{16}{11}
    \gate[\inputs{2}]{nand}{19}{13}{R}{}{}

    % ----  last column ----
    \gate{not}{26}{13}{R}{}{}

    % ----  result ----
    \pin{30}{13}{R}{Z}

\end{circuitdiagram}

%%  ************    LibreSilicon's StdCellLibrary   *******************
%%
%%  Organisation:   Chipforge
%%                  Germany / European Union
%%
%%  Profile:        Chipforge focus on fine System-on-Chip Cores in
%%                  Verilog HDL Code which are easy understandable and
%%                  adjustable. For further information see
%%                          www.chipforge.org
%%                  there are projects from small cores up to PCBs, too.
%%
%%  File:           StdCellLib/Documents/Datasheets/Circuitry/AOOAI234.tex
%%
%%  Purpose:        Circuit File for AOOAI234
%%
%%  ************    LaTeX with circdia.sty package      ***************
%%
%%  ///////////////////////////////////////////////////////////////////
%%
%%  Copyright (c) 2018 - 2022 by
%%                  chipforge <stdcelllib@nospam.chipforge.org>
%%  All rights reserved.
%%
%%      This Standard Cell Library is licensed under the Libre Silicon
%%      public license; you can redistribute it and/or modify it under
%%      the terms of the Libre Silicon public license as published by
%%      the Libre Silicon alliance, either version 1 of the License, or
%%      (at your option) any later version.
%%
%%      This design is distributed in the hope that it will be useful,
%%      but WITHOUT ANY WARRANTY; without even the implied warranty of
%%      MERCHANTABILITY or FITNESS FOR A PARTICULAR PURPOSE.
%%      See the Libre Silicon Public License for more details.
%%
%%  ///////////////////////////////////////////////////////////////////
\begin{circuitdiagram}[draft]{25}{20}

    \usgate
    % ----  1st column  ----
    \pin{1}{1}{L}{A}
    \pin{1}{5}{L}{A1}
    \gate[\inputs{2}]{and}{5}{3}{R}{}{}

    % ----  2nd column  ----
    \wire{9}{3}{9}{5}
    \pin{8}{7}{L}{B}
    \pin{8}{9}{L}{B1}
    \pin{8}{11}{L}{B2}
    \gate[\inputs{4}]{or}{12}{8}{R}{}{}

    \pin{8}{13}{L}{C}
    \pin{8}{15}{L}{C1}
    \pin{8}{17}{L}{C2}
    \pin{8}{19}{L}{C3}
    \gate[\inputs{4}]{or}{12}{16}{R}{}{}

    % ----  3rd column  ----
    \wire{16}{8}{16}{12}
    \gate[\inputs{2}]{nand}{19}{14}{R}{}{}

    % ----  result ----
    \pin{24}{14}{R}{Y}

\end{circuitdiagram}
 %%  ************    LibreSilicon's StdCellLibrary   *******************
%%
%%  Organisation:   Chipforge
%%                  Germany / European Union
%%
%%  Profile:        Chipforge focus on fine System-on-Chip Cores in
%%                  Verilog HDL Code which are easy understandable and
%%                  adjustable. For further information see
%%                          www.chipforge.org
%%                  there are projects from small cores up to PCBs, too.
%%
%%  File:           StdCellLib/Documents/Datasheets/Circuitry/AOOA234.tex
%%
%%  Purpose:        Circuit File for AOOA234
%%
%%  ************    LaTeX with circdia.sty package      ***************
%%
%%  ///////////////////////////////////////////////////////////////////
%%
%%  Copyright (c) 2018 - 2022 by
%%                  chipforge <stdcelllib@nospam.chipforge.org>
%%  All rights reserved.
%%
%%      This Standard Cell Library is licensed under the Libre Silicon
%%      public license; you can redistribute it and/or modify it under
%%      the terms of the Libre Silicon public license as published by
%%      the Libre Silicon alliance, either version 1 of the License, or
%%      (at your option) any later version.
%%
%%      This design is distributed in the hope that it will be useful,
%%      but WITHOUT ANY WARRANTY; without even the implied warranty of
%%      MERCHANTABILITY or FITNESS FOR A PARTICULAR PURPOSE.
%%      See the Libre Silicon Public License for more details.
%%
%%  ///////////////////////////////////////////////////////////////////
\begin{circuitdiagram}[draft]{31}{20}

    \usgate
    % ----  1st column  ----
    \pin{1}{1}{L}{A}
    \pin{1}{5}{L}{A1}
    \gate[\inputs{2}]{and}{5}{3}{R}{}{}

    % ----  2nd column  ----
    \wire{9}{3}{9}{5}
    \pin{8}{7}{L}{B}
    \pin{8}{9}{L}{B1}
    \pin{8}{11}{L}{B2}
    \gate[\inputs{4}]{or}{12}{8}{R}{}{}

    \pin{8}{13}{L}{C}
    \pin{8}{15}{L}{C1}
    \pin{8}{17}{L}{C2}
    \pin{8}{19}{L}{C3}
    \gate[\inputs{4}]{or}{12}{16}{R}{}{}

    % ----  3rd column  ----
    \wire{16}{8}{16}{12}
    \gate[\inputs{2}]{nand}{19}{14}{R}{}{}

    % ----  last column ----
    \gate{not}{26}{14}{R}{}{}

    % ----  result ----
    \pin{30}{14}{R}{Z}

\end{circuitdiagram}

%%  ************    LibreSilicon's StdCellLibrary   *******************
%%
%%  Organisation:   Chipforge
%%                  Germany / European Union
%%
%%  Profile:        Chipforge focus on fine System-on-Chip Cores in
%%                  Verilog HDL Code which are easy understandable and
%%                  adjustable. For further information see
%%                          www.chipforge.org
%%                  there are projects from small cores up to PCBs, too.
%%
%%  File:           StdCellLib/Documents/Circuits/AOOAI312.tex
%%
%%  Purpose:        Circuit File for AOOAI312
%%
%%  ************    LaTeX with circdia.sty package      ***************
%%
%%  ///////////////////////////////////////////////////////////////////
%%
%%  Copyright (c) 2019 by chipforge <stdcelllib@nospam.chipforge.org>
%%  All rights reserved.
%%
%%      This Standard Cell Library is licensed under the Libre Silicon
%%      public license; you can redistribute it and/or modify it under
%%      the terms of the Libre Silicon public license as published by
%%      the Libre Silicon alliance, either version 1 of the License, or
%%      (at your option) any later version.
%%
%%      This design is distributed in the hope that it will be useful,
%%      but WITHOUT ANY WARRANTY; without even the implied warranty of
%%      MERCHANTABILITY or FITNESS FOR A PARTICULAR PURPOSE.
%%      See the Libre Silicon Public License for more details.
%%
%%  ///////////////////////////////////////////////////////////////////
\begin{circuitdiagram}{25}{14}

    \usgate
    \gate[\inputs{3}]{and}{5}{11}{R}{}{}  % AND
    \gate[\inputs{2}]{or}{12}{9}{R}{}{}   % OR
    \gate[\inputs{2}]{or}{12}{3}{R}{}{}   % OR
    \gate[\inputs{2}]{nand}{19}{6}{R}{}{} % NAND
    \pin{1}{1}{L}{A}     % pin A
    \pin{1}{5}{L}{A1}    % pin A1
    \pin{1}{7}{L}{B}     % pin B
    \pin{1}{9}{L}{C}     % pin C
    \pin{1}{11}{L}{C1}   % pin C1
    \pin{1}{13}{L}{C2}   % pin C2
    \wire{2}{1}{9}{1}    % wire pin A
    \wire{2}{5}{9}{5}    % wire pin A1
    \wire{2}{7}{9}{7}    % wire pin B
    \wire{16}{3}{16}{4}  % wire between OR and NAND
    \wire{16}{9}{16}{8}  % wire between OR and NAND
    \pin{24}{6}{R}{Y}    % pin Y

\end{circuitdiagram}
 %%  ************    LibreSilicon's StdCellLibrary   *******************
%%
%%  Organisation:   Chipforge
%%                  Germany / European Union
%%
%%  Profile:        Chipforge focus on fine System-on-Chip Cores in
%%                  Verilog HDL Code which are easy understandable and
%%                  adjustable. For further information see
%%                          www.chipforge.org
%%                  there are projects from small cores up to PCBs, too.
%%
%%  File:           StdCellLib/Documents/Circuits/AOOA312.tex
%%
%%  Purpose:        Circuit File for AOOA312
%%
%%  ************    LaTeX with circdia.sty package      ***************
%%
%%  ///////////////////////////////////////////////////////////////////
%%
%%  Copyright (c) 2019 by chipforge <stdcelllib@nospam.chipforge.org>
%%  All rights reserved.
%%
%%      This Standard Cell Library is licensed under the Libre Silicon
%%      public license; you can redistribute it and/or modify it under
%%      the terms of the Libre Silicon public license as published by
%%      the Libre Silicon alliance, either version 1 of the License, or
%%      (at your option) any later version.
%%
%%      This design is distributed in the hope that it will be useful,
%%      but WITHOUT ANY WARRANTY; without even the implied warranty of
%%      MERCHANTABILITY or FITNESS FOR A PARTICULAR PURPOSE.
%%      See the Libre Silicon Public License for more details.
%%
%%  ///////////////////////////////////////////////////////////////////
\begin{circuitdiagram}{31}{14}

    \usgate
    \gate[\inputs{3}]{and}{5}{11}{R}{}{}  % AND
    \gate[\inputs{2}]{or}{12}{9}{R}{}{}   % OR
    \gate[\inputs{2}]{or}{12}{3}{R}{}{}   % OR
    \gate[\inputs{2}]{nand}{19}{6}{R}{}{} % NAND
    \gate{not}{26}{6}{R}{}{} % NOT
    \pin{1}{1}{L}{A}     % pin A
    \pin{1}{5}{L}{A1}    % pin A1
    \pin{1}{7}{L}{B}     % pin B
    \pin{1}{9}{L}{C}     % pin C
    \pin{1}{11}{L}{C1}   % pin C1
    \pin{1}{13}{L}{C2}   % pin C2
    \wire{2}{1}{9}{1}    % wire pin A
    \wire{2}{5}{9}{5}    % wire pin A1
    \wire{2}{7}{9}{7}    % wire pin B
    \wire{16}{3}{16}{4}  % wire between OR and NAND
    \wire{16}{9}{16}{8}  % wire between OR and NAND
    \pin{30}{6}{R}{Z}    % pin Z

\end{circuitdiagram}


%%  ************    LibreSilicon's StdCellLibrary   *******************
%%
%%  Organisation:   Chipforge
%%                  Germany / European Union
%%
%%  Profile:        Chipforge focus on fine System-on-Chip Cores in
%%                  Verilog HDL Code which are easy understandable and
%%                  adjustable. For further information see
%%                          www.chipforge.org
%%                  there are projects from small cores up to PCBs, too.
%%
%%  File:           StdCellLib/Documents/Datasheets/Circuitry/AOOAI2121.tex
%%
%%  Purpose:        Circuit File for AOOAI2121
%%
%%  ************    LaTeX with circdia.sty package      ***************
%%
%%  ///////////////////////////////////////////////////////////////////
%%
%%  Copyright (c) 2018 - 2022 by
%%                  chipforge <stdcelllib@nospam.chipforge.org>
%%  All rights reserved.
%%
%%      This Standard Cell Library is licensed under the Libre Silicon
%%      public license; you can redistribute it and/or modify it under
%%      the terms of the Libre Silicon public license as published by
%%      the Libre Silicon alliance, either version 1 of the License, or
%%      (at your option) any later version.
%%
%%      This design is distributed in the hope that it will be useful,
%%      but WITHOUT ANY WARRANTY; without even the implied warranty of
%%      MERCHANTABILITY or FITNESS FOR A PARTICULAR PURPOSE.
%%      See the Libre Silicon Public License for more details.
%%
%%  ///////////////////////////////////////////////////////////////////
\begin{circuitdiagram}[draft]{25}{16}

    \usgate
    % ----  1st column  ----
    \pin{1}{1}{L}{A}
    \pin{1}{5}{L}{A1}
    \gate[\inputs{2}]{and}{5}{3}{R}{}{}

    % ----  2nd column  ----
    \pin{8}{7}{L}{B}
    \gate[\inputs{2}]{or}{12}{5}{R}{}{}

    \pin{8}{9}{L}{C}
    \pin{8}{13}{L}{C1}
    \gate[\inputs{2}]{or}{12}{11}{R}{}{}

    % ----  3rd column  ----
    \pin{15}{15}{L}{D}
    \wire{16}{5}{16}{9}
    \wire{16}{13}{16}{15}
    \gate[\inputs{3}]{nand}{19}{11}{R}{}{}

    % ----  result ----
    \pin{24}{11}{R}{Y}

\end{circuitdiagram}
 %%  ************    LibreSilicon's StdCellLibrary   *******************
%%
%%  Organisation:   Chipforge
%%                  Germany / European Union
%%
%%  Profile:        Chipforge focus on fine System-on-Chip Cores in
%%                  Verilog HDL Code which are easy understandable and
%%                  adjustable. For further information see
%%                          www.chipforge.org
%%                  there are projects from small cores up to PCBs, too.
%%
%%  File:           StdCellLib/Documents/Datasheets/Circuitry/AOOA2121.tex
%%
%%  Purpose:        Circuit File for AOOA2121
%%
%%  ************    LaTeX with circdia.sty package      ***************
%%
%%  ///////////////////////////////////////////////////////////////////
%%
%%  Copyright (c) 2018 - 2022 by
%%                  chipforge <stdcelllib@nospam.chipforge.org>
%%  All rights reserved.
%%
%%      This Standard Cell Library is licensed under the Libre Silicon
%%      public license; you can redistribute it and/or modify it under
%%      the terms of the Libre Silicon public license as published by
%%      the Libre Silicon alliance, either version 1 of the License, or
%%      (at your option) any later version.
%%
%%      This design is distributed in the hope that it will be useful,
%%      but WITHOUT ANY WARRANTY; without even the implied warranty of
%%      MERCHANTABILITY or FITNESS FOR A PARTICULAR PURPOSE.
%%      See the Libre Silicon Public License for more details.
%%
%%  ///////////////////////////////////////////////////////////////////
\begin{circuitdiagram}[draft]{31}{16}

    \usgate
    % ----  1st column  ----
    \pin{1}{1}{L}{A}
    \pin{1}{5}{L}{A1}
    \gate[\inputs{2}]{and}{5}{3}{R}{}{}

    % ----  2nd column  ----
    \pin{8}{7}{L}{B}
    \gate[\inputs{2}]{or}{12}{5}{R}{}{}

    \pin{8}{9}{L}{C}
    \pin{8}{13}{L}{C1}
    \gate[\inputs{2}]{or}{12}{11}{R}{}{}

    % ----  3rd column  ----
    \pin{15}{15}{L}{D}
    \wire{16}{5}{16}{9}
    \wire{16}{13}{16}{15}
    \gate[\inputs{3}]{nand}{19}{11}{R}{}{}

    % ----  4th column  ----
    \gate{not}{26}{11}{R}{}{}

    % ----  result ----
    \pin{30}{11}{R}{Z}

\end{circuitdiagram}

%%  ************    LibreSilicon's StdCellLibrary   *******************
%%
%%  Organisation:   Chipforge
%%                  Germany / European Union
%%
%%  Profile:        Chipforge focus on fine System-on-Chip Cores in
%%                  Verilog HDL Code which are easy understandable and
%%                  adjustable. For further information see
%%                          www.chipforge.org
%%                  there are projects from small cores up to PCBs, too.
%%
%%  File:           StdCellLib/Documents/Datasheets/Circuitry/AOOAI2221.tex
%%
%%  Purpose:        Circuit File for AOOAI2221
%%
%%  ************    LaTeX with circdia.sty package      ***************
%%
%%  ///////////////////////////////////////////////////////////////////
%%
%%  Copyright (c) 2018 - 2022 by
%%                  chipforge <stdcelllib@nospam.chipforge.org>
%%  All rights reserved.
%%
%%      This Standard Cell Library is licensed under the Libre Silicon
%%      public license; you can redistribute it and/or modify it under
%%      the terms of the Libre Silicon public license as published by
%%      the Libre Silicon alliance, either version 1 of the License, or
%%      (at your option) any later version.
%%
%%      This design is distributed in the hope that it will be useful,
%%      but WITHOUT ANY WARRANTY; without even the implied warranty of
%%      MERCHANTABILITY or FITNESS FOR A PARTICULAR PURPOSE.
%%      See the Libre Silicon Public License for more details.
%%
%%  ///////////////////////////////////////////////////////////////////
\begin{circuitdiagram}[draft]{25}{18}

    \usgate
    % ----  1st column  ----
    \pin{1}{1}{L}{A}
    \pin{1}{5}{L}{A1}
    \gate[\inputs{2}]{and}{5}{3}{R}{}{}

    % ----  2nd column  ----
    \wire{9}{3}{9}{5}
    \pin{8}{7}{L}{B}
    \pin{8}{9}{L}{B1}
    \gate[\inputs{3}]{or}{12}{7}{R}{}{}

    \pin{8}{11}{L}{C}
    \pin{8}{15}{L}{C1}
    \gate[\inputs{2}]{or}{12}{13}{R}{}{}

    % ----  3rd column  ----
    \wire{16}{7}{16}{11}
    \pin{15}{17}{L}{D}
    \wire{16}{15}{16}{17}
    \gate[\inputs{3}]{nand}{19}{13}{R}{}{}

    % ----  result ----
    \pin{24}{13}{R}{Y}

\end{circuitdiagram}
 %%  ************    LibreSilicon's StdCellLibrary   *******************
%%
%%  Organisation:   Chipforge
%%                  Germany / European Union
%%
%%  Profile:        Chipforge focus on fine System-on-Chip Cores in
%%                  Verilog HDL Code which are easy understandable and
%%                  adjustable. For further information see
%%                          www.chipforge.org
%%                  there are projects from small cores up to PCBs, too.
%%
%%  File:           StdCellLib/Documents/Datasheets/Circuitry/AOOA2221.tex
%%
%%  Purpose:        Circuit File for AOOA2221
%%
%%  ************    LaTeX with circdia.sty package      ***************
%%
%%  ///////////////////////////////////////////////////////////////////
%%
%%  Copyright (c) 2018 - 2022 by
%%                  chipforge <stdcelllib@nospam.chipforge.org>
%%  All rights reserved.
%%
%%      This Standard Cell Library is licensed under the Libre Silicon
%%      public license; you can redistribute it and/or modify it under
%%      the terms of the Libre Silicon public license as published by
%%      the Libre Silicon alliance, either version 1 of the License, or
%%      (at your option) any later version.
%%
%%      This design is distributed in the hope that it will be useful,
%%      but WITHOUT ANY WARRANTY; without even the implied warranty of
%%      MERCHANTABILITY or FITNESS FOR A PARTICULAR PURPOSE.
%%      See the Libre Silicon Public License for more details.
%%
%%  ///////////////////////////////////////////////////////////////////
\begin{circuitdiagram}[draft]{31}{18}

    \usgate
    % ----  1st column  ----
    \pin{1}{1}{L}{A}
    \pin{1}{5}{L}{A1}
    \gate[\inputs{2}]{and}{5}{3}{R}{}{}

    % ----  2nd column  ----
    \wire{9}{3}{9}{5}
    \pin{8}{7}{L}{B}
    \pin{8}{9}{L}{B1}
    \gate[\inputs{3}]{or}{12}{7}{R}{}{}

    \pin{8}{11}{L}{C}
    \pin{8}{15}{L}{C1}
    \gate[\inputs{2}]{or}{12}{13}{R}{}{}

    % ----  3rd column  ----
    \wire{16}{7}{16}{11}
    \pin{15}{17}{L}{D}
    \wire{16}{15}{16}{17}
    \gate[\inputs{3}]{nand}{19}{13}{R}{}{}

    % ----  4th column  ----
    \gate{not}{26}{13}{R}{}{}

    % ----  result ----
    \pin{30}{13}{R}{Z}

\end{circuitdiagram}

\include{Datasheets/AOOAI2231} \include{Datasheets/AOOA2231}

%%  ************    LibreSilicon's StdCellLibrary   *******************
%%
%%  Organisation:   Chipforge
%%                  Germany / European Union
%%
%%  Profile:        Chipforge focus on fine System-on-Chip Cores in
%%                  Verilog HDL Code which are easy understandable and
%%                  adjustable. For further information see
%%                          www.chipforge.org
%%                  there are projects from small cores up to PCBs, too.
%%
%%  File:           StdCellLib/Documents/section-OAAOI_complex.tex
%%
%%  Purpose:        Section Level File for Standard Cell Library Documentation
%%
%%  ************    LaTeX with circdia.sty package      ***************
%%
%%  ///////////////////////////////////////////////////////////////////
%%
%%  Copyright (c) 2018 - 2022 by
%%                  chipforge <stdcelllib@nospam.chipforge.org>
%%  All rights reserved.
%%
%%      This Standard Cell Library is licensed under the Libre Silicon
%%      public license; you can redistribute it and/or modify it under
%%      the terms of the Libre Silicon public license as published by
%%      the Libre Silicon alliance, either version 1 of the License, or
%%      (at your option) any later version.
%%
%%      This design is distributed in the hope that it will be useful,
%%      but WITHOUT ANY WARRANTY; without even the implied warranty of
%%      MERCHANTABILITY or FITNESS FOR A PARTICULAR PURPOSE.
%%      See the Libre Silicon Public License for more details.
%%
%%  ///////////////////////////////////////////////////////////////////
\section{OR-AND-AND-OR(-Invert) Complex Gates}

%%  ************    LibreSilicon's StdCellLibrary   *******************
%%
%%  Organisation:   Chipforge
%%                  Germany / European Union
%%
%%  Profile:        Chipforge focus on fine System-on-Chip Cores in
%%                  Verilog HDL Code which are easy understandable and
%%                  adjustable. For further information see
%%                          www.chipforge.org
%%                  there are projects from small cores up to PCBs, too.
%%
%%  File:           StdCellLib/Documents/Circuits/OAAOI212.tex
%%
%%  Purpose:        Circuit File for OAAOI212
%%
%%  ************    LaTeX with circdia.sty package      ***************
%%
%%  ///////////////////////////////////////////////////////////////////
%%
%%  Copyright (c) 2019 by chipforge <stdcelllib@nospam.chipforge.org>
%%  All rights reserved.
%%
%%      This Standard Cell Library is licensed under the Libre Silicon
%%      public license; you can redistribute it and/or modify it under
%%      the terms of the Libre Silicon public license as published by
%%      the Libre Silicon alliance, either version 1 of the License, or
%%      (at your option) any later version.
%%
%%      This design is distributed in the hope that it will be useful,
%%      but WITHOUT ANY WARRANTY; without even the implied warranty of
%%      MERCHANTABILITY or FITNESS FOR A PARTICULAR PURPOSE.
%%      See the Libre Silicon Public License for more details.
%%
%%  ///////////////////////////////////////////////////////////////////
\begin{center}
    Circuit
    \begin{figure}[h]
        \begin{center}
            \begin{circuitdiagram}{25}{14}
            \usgate
            \gate[\inputs{2}]{or}{5}{11}{R}{}{}  % OR
            \gate[\inputs{2}]{and}{12}{9}{R}{}{} % AND
            \gate[\inputs{2}]{and}{12}{3}{R}{}{} % AND
            \gate[\inputs{2}]{nor}{19}{6}{R}{}{} % NOR
            \pin{1}{1}{L}{A}     % pin A
            \pin{1}{5}{L}{A1}    % pin A1
            \pin{1}{7}{L}{B}     % pin B
            \pin{1}{9}{L}{C}     % pin C
            \pin{1}{13}{L}{C1}   % pin C1
            \wire{2}{1}{9}{1}    % wire pin A
            \wire{2}{5}{9}{5}    % wire pin A1
            \wire{2}{7}{9}{7}    % wire pin B
            \wire{16}{3}{16}{4}  % wire between AND and NOR
            \wire{16}{9}{16}{8}  % wire between AND and NOR
            \pin{24}{6}{R}{Y}    % pin Y
            \end{circuitdiagram}
        \end{center}
    \end{figure}
\end{center}
 %%  ************    LibreSilicon's StdCellLibrary   *******************
%%
%%  Organisation:   Chipforge
%%                  Germany / European Union
%%
%%  Profile:        Chipforge focus on fine System-on-Chip Cores in
%%                  Verilog HDL Code which are easy understandable and
%%                  adjustable. For further information see
%%                          www.chipforge.org
%%                  there are projects from small cores up to PCBs, too.
%%
%%  File:           StdCellLib/Documents/Datasheets/Circuitry/OAAO212.tex
%%
%%  Purpose:        Circuit File for OAAO212
%%
%%  ************    LaTeX with circdia.sty package      ***************
%%
%%  ///////////////////////////////////////////////////////////////////
%%
%%  Copyright (c) 2018 - 2022 by
%%                  chipforge <stdcelllib@nospam.chipforge.org>
%%  All rights reserved.
%%
%%      This Standard Cell Library is licensed under the Libre Silicon
%%      public license; you can redistribute it and/or modify it under
%%      the terms of the Libre Silicon public license as published by
%%      the Libre Silicon alliance, either version 1 of the License, or
%%      (at your option) any later version.
%%
%%      This design is distributed in the hope that it will be useful,
%%      but WITHOUT ANY WARRANTY; without even the implied warranty of
%%      MERCHANTABILITY or FITNESS FOR A PARTICULAR PURPOSE.
%%      See the Libre Silicon Public License for more details.
%%
%%  ///////////////////////////////////////////////////////////////////
\begin{circuitdiagram}[draft]{31}{14}

    \usgate
    % ----  1st column  ----
    \pin{1}{1}{L}{A}
    \pin{1}{5}{L}{A1}
    \gate[\inputs{2}]{or}{5}{3}{R}{}{}

    % ----  2nd column  ----
    \pin{8}{7}{L}{B}
    \gate[\inputs{2}]{and}{12}{5}{R}{}{}

    \pin{8}{9}{L}{C}
    \pin{8}{13}{L}{C1}
    \gate[\inputs{2}]{and}{12}{11}{R}{}{}

    % ----  3rd column  ----
    \wire{16}{5}{16}{7}
    \gate[\inputs{2}]{nor}{19}{9}{R}{}{}

    % ----  4th column  ----
    \gate{not}{26}{9}{R}{}{}

    % ----  result ----
    \pin{30}{9}{R}{Z}

\end{circuitdiagram}

%%  ************    LibreSilicon's StdCellLibrary   *******************
%%
%%  Organisation:   Chipforge
%%                  Germany / European Union
%%
%%  Profile:        Chipforge focus on fine System-on-Chip Cores in
%%                  Verilog HDL Code which are easy understandable and
%%                  adjustable. For further information see
%%                          www.chipforge.org
%%                  there are projects from small cores up to PCBs, too.
%%
%%  File:           StdCellLib/Documents/Datasheets/Circuitry/OAAOI213.tex
%%
%%  Purpose:        Circuit File for OAAOI213 
%%
%%  ************    LaTeX with circdia.sty package      ***************
%%
%%  ///////////////////////////////////////////////////////////////////
%%
%%  Copyright (c) 2018 - 2022 by
%%                  chipforge <stdcelllib@nospam.chipforge.org>
%%  All rights reserved.
%%
%%      This Standard Cell Library is licensed under the Libre Silicon
%%      public license; you can redistribute it and/or modify it under
%%      the terms of the Libre Silicon public license as published by
%%      the Libre Silicon alliance, either version 1 of the License, or
%%      (at your option) any later version.
%%
%%      This design is distributed in the hope that it will be useful,
%%      but WITHOUT ANY WARRANTY; without even the implied warranty of
%%      MERCHANTABILITY or FITNESS FOR A PARTICULAR PURPOSE.
%%      See the Libre Silicon Public License for more details.
%%
%%  ///////////////////////////////////////////////////////////////////
\begin{circuitdiagram}[draft]{25}{14}

    \usgate
    % ----  1st column  ----
    \pin{1}{1}{L}{A}
    \pin{1}{5}{L}{A1}
    \gate[\inputs{2}]{or}{5}{3}{R}{}{}

    % ----  2nd column  ----
    \pin{8}{7}{L}{B}
    \gate[\inputs{2}]{and}{12}{5}{R}{}{}

    \pin{8}{9}{L}{C}
    \pin{8}{11}{L}{C1}
    \pin{8}{13}{L}{C2}
    \gate[\inputs{3}]{and}{12}{11}{R}{}{}

    % ----  3rd column  ----
    \wire{16}{5}{16}{7}
    \gate[\inputs{2}]{nor}{19}{9}{R}{}{}

    % ----  result ----
    \pin{24}{9}{R}{Y}

\end{circuitdiagram}
 %%  ************    LibreSilicon's StdCellLibrary   *******************
%%
%%  Organisation:   Chipforge
%%                  Germany / European Union
%%
%%  Profile:        Chipforge focus on fine System-on-Chip Cores in
%%                  Verilog HDL Code which are easy understandable and
%%                  adjustable. For further information see
%%                          www.chipforge.org
%%                  there are projects from small cores up to PCBs, too.
%%
%%  File:           StdCellLib/Documents/Datasheets/Circuitry/OAAO213.tex
%%
%%  Purpose:        Circuit File for OAAO213
%%
%%  ************    LaTeX with circdia.sty package      ***************
%%
%%  ///////////////////////////////////////////////////////////////////
%%
%%  Copyright (c) 2018 - 2022 by
%%                  chipforge <stdcelllib@nospam.chipforge.org>
%%  All rights reserved.
%%
%%      This Standard Cell Library is licensed under the Libre Silicon
%%      public license; you can redistribute it and/or modify it under
%%      the terms of the Libre Silicon public license as published by
%%      the Libre Silicon alliance, either version 1 of the License, or
%%      (at your option) any later version.
%%
%%      This design is distributed in the hope that it will be useful,
%%      but WITHOUT ANY WARRANTY; without even the implied warranty of
%%      MERCHANTABILITY or FITNESS FOR A PARTICULAR PURPOSE.
%%      See the Libre Silicon Public License for more details.
%%
%%  ///////////////////////////////////////////////////////////////////
\begin{circuitdiagram}[draft]{31}{14}

    \usgate
    % ----  1st column  ----
    \pin{1}{1}{L}{A}
    \pin{1}{5}{L}{A1}
    \gate[\inputs{2}]{or}{5}{3}{R}{}{}

    % ----  2nd column  ----
    \pin{8}{7}{L}{B}
    \gate[\inputs{2}]{and}{12}{5}{R}{}{}

    \pin{8}{9}{L}{C}
    \pin{8}{11}{L}{C1}
    \pin{8}{13}{L}{C2}
    \gate[\inputs{3}]{and}{12}{11}{R}{}{}

    % ----  3rd column  ----
    \wire{16}{5}{16}{7}
    \gate[\inputs{2}]{nor}{19}{9}{R}{}{}

    % ----  4th column  ----
    \gate{not}{26}{9}{R}{}{}

    % ----  result ----
    \pin{30}{9}{R}{Z}

\end{circuitdiagram}

%%  ************    LibreSilicon's StdCellLibrary   *******************
%%
%%  Organisation:   Chipforge
%%                  Germany / European Union
%%
%%  Profile:        Chipforge focus on fine System-on-Chip Cores in
%%                  Verilog HDL Code which are easy understandable and
%%                  adjustable. For further information see
%%                          www.chipforge.org
%%                  there are projects from small cores up to PCBs, too.
%%
%%  File:           StdCellLib/Documents/Datasheets/Circuitry/OAAOI214.tex
%%
%%  Purpose:        Circuit File for OAAOI214
%%
%%  ************    LaTeX with circdia.sty package      ***************
%%
%%  ///////////////////////////////////////////////////////////////////
%%
%%  Copyright (c) 2018 - 2022 by
%%                  chipforge <stdcelllib@nospam.chipforge.org>
%%  All rights reserved.
%%
%%      This Standard Cell Library is licensed under the Libre Silicon
%%      public license; you can redistribute it and/or modify it under
%%      the terms of the Libre Silicon public license as published by
%%      the Libre Silicon alliance, either version 1 of the License, or
%%      (at your option) any later version.
%%
%%      This design is distributed in the hope that it will be useful,
%%      but WITHOUT ANY WARRANTY; without even the implied warranty of
%%      MERCHANTABILITY or FITNESS FOR A PARTICULAR PURPOSE.
%%      See the Libre Silicon Public License for more details.
%%
%%  ///////////////////////////////////////////////////////////////////
\begin{circuitdiagram}[draft]{25}{16}

    \usgate
    % ----  1st column  ----
    \pin{1}{1}{L}{A}
    \pin{1}{5}{L}{A1}
    \gate[\inputs{2}]{or}{5}{3}{R}{}{}

    % ----  2nd column  ----
    \pin{8}{7}{L}{B}
    \gate[\inputs{2}]{and}{12}{5}{R}{}{}

    \pin{8}{9}{L}{C}
    \pin{8}{11}{L}{C1}
    \pin{8}{13}{L}{C2}
    \pin{8}{15}{L}{C3}
    \gate[\inputs{4}]{and}{12}{12}{R}{}{}

    % ----  3rd column  ----
    \wire{16}{5}{16}{8}
    \gate[\inputs{2}]{nor}{19}{10}{R}{}{}

    % ----  result ----
    \pin{24}{10}{R}{Y}

\end{circuitdiagram}
 %%  ************    LibreSilicon's StdCellLibrary   *******************
%%
%%  Organisation:   Chipforge
%%                  Germany / European Union
%%
%%  Profile:        Chipforge focus on fine System-on-Chip Cores in
%%                  Verilog HDL Code which are easy understandable and
%%                  adjustable. For further information see
%%                          www.chipforge.org
%%                  there are projects from small cores up to PCBs, too.
%%
%%  File:           StdCellLib/Documents/Datasheets/Circuitry/OAAO214.tex
%%
%%  Purpose:        Circuit File for OAAO214
%%
%%  ************    LaTeX with circdia.sty package      ***************
%%
%%  ///////////////////////////////////////////////////////////////////
%%
%%  Copyright (c) 2018 - 2022 by
%%                  chipforge <stdcelllib@nospam.chipforge.org>
%%  All rights reserved.
%%
%%      This Standard Cell Library is licensed under the Libre Silicon
%%      public license; you can redistribute it and/or modify it under
%%      the terms of the Libre Silicon public license as published by
%%      the Libre Silicon alliance, either version 1 of the License, or
%%      (at your option) any later version.
%%
%%      This design is distributed in the hope that it will be useful,
%%      but WITHOUT ANY WARRANTY; without even the implied warranty of
%%      MERCHANTABILITY or FITNESS FOR A PARTICULAR PURPOSE.
%%      See the Libre Silicon Public License for more details.
%%
%%  ///////////////////////////////////////////////////////////////////
\begin{circuitdiagram}[draft]{31}{16}

    \usgate
    % ----  1st column  ----
    \pin{1}{1}{L}{A}
    \pin{1}{5}{L}{A1}
    \gate[\inputs{2}]{or}{5}{3}{R}{}{}

    % ----  2nd column  ----
    \pin{8}{7}{L}{B}
    \gate[\inputs{2}]{and}{12}{5}{R}{}{}

    \pin{8}{9}{L}{C}
    \pin{8}{11}{L}{C1}
    \pin{8}{13}{L}{C2}
    \pin{8}{15}{L}{C3}
    \gate[\inputs{4}]{and}{12}{12}{R}{}{}

    % ----  3rd column  ----
    \wire{16}{5}{16}{8}
    \gate[\inputs{2}]{nor}{19}{10}{R}{}{}

    % ----  4th column  ----
    \gate{not}{26}{10}{R}{}{}

    % ----  result ----
    \pin{30}{10}{R}{Z}

\end{circuitdiagram}

%%  ************    LibreSilicon's StdCellLibrary   *******************
%%
%%  Organisation:   Chipforge
%%                  Germany / European Union
%%
%%  Profile:        Chipforge focus on fine System-on-Chip Cores in
%%                  Verilog HDL Code which are easy understandable and
%%                  adjustable. For further information see
%%                          www.chipforge.org
%%                  there are projects from small cores up to PCBs, too.
%%
%%  File:           StdCellLib/Documents/Circuits/OAAOI222.tex
%%
%%  Purpose:        Circuit File for OAAOI222
%%
%%  ************    LaTeX with circdia.sty package      ***************
%%
%%  ///////////////////////////////////////////////////////////////////
%%
%%  Copyright (c) 2019 by chipforge <stdcelllib@nospam.chipforge.org>
%%  All rights reserved.
%%
%%      This Standard Cell Library is licensed under the Libre Silicon
%%      public license; you can redistribute it and/or modify it under
%%      the terms of the Libre Silicon public license as published by
%%      the Libre Silicon alliance, either version 1 of the License, or
%%      (at your option) any later version.
%%
%%      This design is distributed in the hope that it will be useful,
%%      but WITHOUT ANY WARRANTY; without even the implied warranty of
%%      MERCHANTABILITY or FITNESS FOR A PARTICULAR PURPOSE.
%%      See the Libre Silicon Public License for more details.
%%
%%  ///////////////////////////////////////////////////////////////////
\begin{center}
    Circuit
    \begin{figure}[h]
        \begin{center}
            \begin{circuitdiagram}{25}{16}
            \usgate
            \gate[\inputs{2}]{or}{5}{13}{R}{}{}  % OR
            \gate[\inputs{3}]{and}{12}{9}{R}{}{} % AND
            \gate[\inputs{2}]{and}{12}{3}{R}{}{} % AND
            \gate[\inputs{2}]{nor}{19}{6}{R}{}{} % NOR
            \pin{1}{1}{L}{A}     % pin A
            \wire{2}{1}{9}{1}    % wire pin A
            \pin{1}{5}{L}{A1}    % pin A1
            \wire{2}{5}{9}{5}    % wire pin A1
            \pin{1}{7}{L}{B}     % pin B
            \wire{2}{7}{9}{7}    % wire pin B
            \pin{1}{9}{L}{B1}    % pin B1
            \wire{2}{9}{9}{9}    % wire pin B
            \pin{1}{11}{L}{C}    % pin C
            \pin{1}{15}{L}{C1}   % pin C1
            \wire{9}{11}{9}{13}  % wire between OR and AND
            \wire{16}{3}{16}{4}  % wire between AND and NOR
            \wire{16}{9}{16}{8}  % wire between AND and NOR
            \pin{24}{6}{R}{Y}    % pin Y
            \end{circuitdiagram}
        \end{center}
    \end{figure}
\end{center}
 %%  ************    LibreSilicon's StdCellLibrary   *******************
%%
%%  Organisation:   Chipforge
%%                  Germany / European Union
%%
%%  Profile:        Chipforge focus on fine System-on-Chip Cores in
%%                  Verilog HDL Code which are easy understandable and
%%                  adjustable. For further information see
%%                          www.chipforge.org
%%                  there are projects from small cores up to PCBs, too.
%%
%%  File:           StdCellLib/Documents/Datasheets/Circuitry/OAAO213.tex
%%
%%  Purpose:        Circuit File for OAAO213
%%
%%  ************    LaTeX with circdia.sty package      ***************
%%
%%  ///////////////////////////////////////////////////////////////////
%%
%%  Copyright (c) 2018 - 2022 by
%%                  chipforge <stdcelllib@nospam.chipforge.org>
%%  All rights reserved.
%%
%%      This Standard Cell Library is licensed under the Libre Silicon
%%      public license; you can redistribute it and/or modify it under
%%      the terms of the Libre Silicon public license as published by
%%      the Libre Silicon alliance, either version 1 of the License, or
%%      (at your option) any later version.
%%
%%      This design is distributed in the hope that it will be useful,
%%      but WITHOUT ANY WARRANTY; without even the implied warranty of
%%      MERCHANTABILITY or FITNESS FOR A PARTICULAR PURPOSE.
%%      See the Libre Silicon Public License for more details.
%%
%%  ///////////////////////////////////////////////////////////////////
\begin{circuitdiagram}[draft]{31}{16}

    \usgate
    % ----  1st column  ----
    \pin{1}{1}{L}{A}
    \pin{1}{5}{L}{A1}
    \gate[\inputs{2}]{or}{5}{3}{R}{}{}

    % ----  2nd column  ----
    \wire{9}{3}{9}{5}
    \pin{8}{7}{L}{B}
    \pin{8}{9}{L}{B1}
    \gate[\inputs{3}]{and}{12}{7}{R}{}{}

    \pin{8}{11}{L}{C}
    \pin{8}{15}{L}{C1}
    \gate[\inputs{2}]{and}{12}{13}{R}{}{}

    % ----  3rd column  ----
    \wire{16}{7}{16}{9}
    \gate[\inputs{2}]{nor}{19}{11}{R}{}{}

    % ----  4th column  ----
    \gate{not}{26}{11}{R}{}{}

    % ----  result ----
    \pin{30}{11}{R}{Z}

\end{circuitdiagram}

%%  ************    LibreSilicon's StdCellLibrary   *******************
%%
%%  Organisation:   Chipforge
%%                  Germany / European Union
%%
%%  Profile:        Chipforge focus on fine System-on-Chip Cores in
%%                  Verilog HDL Code which are easy understandable and
%%                  adjustable. For further information see
%%                          www.chipforge.org
%%                  there are projects from small cores up to PCBs, too.
%%
%%  File:           StdCellLib/Documents/Circuits/OAAOI223.tex
%%
%%  Purpose:        Circuit File for OAAOI223
%%
%%  ************    LaTeX with circdia.sty package      ***************
%%
%%  ///////////////////////////////////////////////////////////////////
%%
%%  Copyright (c) 2019 by chipforge <stdcelllib@nospam.chipforge.org>
%%  All rights reserved.
%%
%%      This Standard Cell Library is licensed under the Libre Silicon
%%      public license; you can redistribute it and/or modify it under
%%      the terms of the Libre Silicon public license as published by
%%      the Libre Silicon alliance, either version 1 of the License, or
%%      (at your option) any later version.
%%
%%      This design is distributed in the hope that it will be useful,
%%      but WITHOUT ANY WARRANTY; without even the implied warranty of
%%      MERCHANTABILITY or FITNESS FOR A PARTICULAR PURPOSE.
%%      See the Libre Silicon Public License for more details.
%%
%%  ///////////////////////////////////////////////////////////////////
\begin{center}
    Circuit
    \begin{figure}[h]
        \begin{center}
            \begin{circuitdiagram}{25}{16}
            \usgate
            \gate[\inputs{2}]{or}{5}{13}{R}{}{}  % OR
            \gate[\inputs{3}]{and}{12}{9}{R}{}{} % AND
            \gate[\inputs{3}]{and}{12}{3}{R}{}{} % AND
            \gate[\inputs{2}]{nor}{19}{6}{R}{}{} % NOR
            \pin{1}{1}{L}{A}     % pin A
            \wire{2}{1}{9}{1}    % wire pin A
            \pin{1}{3}{L}{A1}    % pin A1
            \wire{2}{3}{9}{3}    % wire pin A1
            \pin{1}{5}{L}{A2}    % pin A2
            \wire{2}{5}{9}{5}    % wire pin A2
            \pin{1}{7}{L}{B}     % pin B
            \wire{2}{7}{9}{7}    % wire pin B
            \pin{1}{9}{L}{B1}    % pin B1
            \wire{2}{9}{9}{9}    % wire pin B
            \pin{1}{11}{L}{C}    % pin C
            \pin{1}{15}{L}{C1}   % pin C1
            \wire{9}{11}{9}{13}  % wire between OR and AND
            \wire{16}{3}{16}{4}  % wire between AND and NOR
            \wire{16}{9}{16}{8}  % wire between AND and NOR
            \pin{24}{6}{R}{Y}    % pin Y
            \end{circuitdiagram}
        \end{center}
    \end{figure}
\end{center}
 %%  ************    LibreSilicon's StdCellLibrary   *******************
%%
%%  Organisation:   Chipforge
%%                  Germany / European Union
%%
%%  Profile:        Chipforge focus on fine System-on-Chip Cores in
%%                  Verilog HDL Code which are easy understandable and
%%                  adjustable. For further information see
%%                          www.chipforge.org
%%                  there are projects from small cores up to PCBs, too.
%%
%%  File:           StdCellLib/Documents/Datasheets/Circuitry/OAAO223.tex
%%
%%  Purpose:        Circuit File for OAAO223
%%
%%  ************    LaTeX with circdia.sty package      ***************
%%
%%  ///////////////////////////////////////////////////////////////////
%%
%%  Copyright (c) 2018 - 2022 by
%%                  chipforge <stdcelllib@nospam.chipforge.org>
%%  All rights reserved.
%%
%%      This Standard Cell Library is licensed under the Libre Silicon
%%      public license; you can redistribute it and/or modify it under
%%      the terms of the Libre Silicon public license as published by
%%      the Libre Silicon alliance, either version 1 of the License, or
%%      (at your option) any later version.
%%
%%      This design is distributed in the hope that it will be useful,
%%      but WITHOUT ANY WARRANTY; without even the implied warranty of
%%      MERCHANTABILITY or FITNESS FOR A PARTICULAR PURPOSE.
%%      See the Libre Silicon Public License for more details.
%%
%%  ///////////////////////////////////////////////////////////////////
\begin{circuitdiagram}[draft]{31}{16}

    \usgate
    % ----  1st column  ----
    \pin{1}{1}{L}{A}
    \pin{1}{5}{L}{A1}
    \gate[\inputs{2}]{or}{5}{3}{R}{}{}

    % ----  2nd column  ----
    \wire{9}{3}{9}{5}
    \pin{8}{7}{L}{B}
    \pin{8}{9}{L}{B1}
    \gate[\inputs{3}]{and}{12}{7}{R}{}{}

    \pin{8}{11}{L}{C}
    \pin{8}{13}{L}{C1}
    \pin{8}{15}{L}{C2}
    \gate[\inputs{3}]{and}{12}{13}{R}{}{}

    % ----  3rd column  ----
    \wire{16}{7}{16}{9}
    \gate[\inputs{2}]{nor}{19}{11}{R}{}{}

    % ----  4th column  ----
    \gate{not}{26}{11}{R}{}{}

    % ----  result ----
    \pin{30}{11}{R}{Z}

\end{circuitdiagram}

%%  ************    LibreSilicon's StdCellLibrary   *******************
%%
%%  Organisation:   Chipforge
%%                  Germany / European Union
%%
%%  Profile:        Chipforge focus on fine System-on-Chip Cores in
%%                  Verilog HDL Code which are easy understandable and
%%                  adjustable. For further information see
%%                          www.chipforge.org
%%                  there are projects from small cores up to PCBs, too.
%%
%%  File:           StdCellLib/Documents/Datasheets/Circuitry/OAAOI224.tex
%%
%%  Purpose:        Circuit File for OAAOI224
%%
%%  ************    LaTeX with circdia.sty package      ***************
%%
%%  ///////////////////////////////////////////////////////////////////
%%
%%  Copyright (c) 2018 - 2022 by
%%                  chipforge <stdcelllib@nospam.chipforge.org>
%%  All rights reserved.
%%
%%      This Standard Cell Library is licensed under the Libre Silicon
%%      public license; you can redistribute it and/or modify it under
%%      the terms of the Libre Silicon public license as published by
%%      the Libre Silicon alliance, either version 1 of the License, or
%%      (at your option) any later version.
%%
%%      This design is distributed in the hope that it will be useful,
%%      but WITHOUT ANY WARRANTY; without even the implied warranty of
%%      MERCHANTABILITY or FITNESS FOR A PARTICULAR PURPOSE.
%%      See the Libre Silicon Public License for more details.
%%
%%  ///////////////////////////////////////////////////////////////////
\begin{circuitdiagram}[draft]{25}{18}

    \usgate
    % ----  1st column  ----
    \pin{1}{1}{L}{A}
    \pin{1}{5}{L}{A1}
    \gate[\inputs{2}]{or}{5}{3}{R}{}{}

    % ----  2nd column  ----
    \wire{9}{3}{9}{5}
    \pin{8}{7}{L}{B}
    \pin{8}{9}{L}{B1}
    \gate[\inputs{3}]{and}{12}{7}{R}{}{}

    \pin{8}{11}{L}{C}
    \pin{8}{13}{L}{C1}
    \pin{8}{15}{L}{C2}
    \pin{8}{17}{L}{C3}
    \gate[\inputs{4}]{and}{12}{14}{R}{}{}

    % ----  3rd column  ----
    \wire{16}{7}{16}{10}
    \gate[\inputs{2}]{nor}{19}{12}{R}{}{}

    % ----  result ----
    \pin{24}{12}{R}{Y}

\end{circuitdiagram}
 %%  ************    LibreSilicon's StdCellLibrary   *******************
%%
%%  Organisation:   Chipforge
%%                  Germany / European Union
%%
%%  Profile:        Chipforge focus on fine System-on-Chip Cores in
%%                  Verilog HDL Code which are easy understandable and
%%                  adjustable. For further information see
%%                          www.chipforge.org
%%                  there are projects from small cores up to PCBs, too.
%%
%%  File:           StdCellLib/Documents/Datasheets/Circuitry/OAAO224.tex
%%
%%  Purpose:        Circuit File for OAAO224
%%
%%  ************    LaTeX with circdia.sty package      ***************
%%
%%  ///////////////////////////////////////////////////////////////////
%%
%%  Copyright (c) 2018 - 2022 by
%%                  chipforge <stdcelllib@nospam.chipforge.org>
%%  All rights reserved.
%%
%%      This Standard Cell Library is licensed under the Libre Silicon
%%      public license; you can redistribute it and/or modify it under
%%      the terms of the Libre Silicon public license as published by
%%      the Libre Silicon alliance, either version 1 of the License, or
%%      (at your option) any later version.
%%
%%      This design is distributed in the hope that it will be useful,
%%      but WITHOUT ANY WARRANTY; without even the implied warranty of
%%      MERCHANTABILITY or FITNESS FOR A PARTICULAR PURPOSE.
%%      See the Libre Silicon Public License for more details.
%%
%%  ///////////////////////////////////////////////////////////////////
\begin{circuitdiagram}[draft]{31}{18}

    \usgate
    % ----  1st column  ----
    \pin{1}{1}{L}{A}
    \pin{1}{5}{L}{A1}
    \gate[\inputs{2}]{or}{5}{3}{R}{}{}

    % ----  2nd column  ----
    \wire{9}{3}{9}{5}
    \pin{8}{7}{L}{B}
    \pin{8}{9}{L}{B1}
    \gate[\inputs{3}]{and}{12}{7}{R}{}{}

    \pin{8}{11}{L}{C}
    \pin{8}{13}{L}{C1}
    \pin{8}{15}{L}{C2}
    \pin{8}{17}{L}{C3}
    \gate[\inputs{4}]{and}{12}{14}{R}{}{}

    % ----  3rd column  ----
    \wire{16}{7}{16}{10}
    \gate[\inputs{2}]{nor}{19}{12}{R}{}{}

    % ----  4th column  ----
    \gate{not}{26}{12}{R}{}{}

    % ----  result ----
    \pin{30}{12}{R}{Z}

\end{circuitdiagram}

%%  ************    LibreSilicon's StdCellLibrary   *******************
%%
%%  Organisation:   Chipforge
%%                  Germany / European Union
%%
%%  Profile:        Chipforge focus on fine System-on-Chip Cores in
%%                  Verilog HDL Code which are easy understandable and
%%                  adjustable. For further information see
%%                          www.chipforge.org
%%                  there are projects from small cores up to PCBs, too.
%%
%%  File:           StdCellLib/Documents/Datasheets/Circuitry/OAAOI232.tex
%%
%%  Purpose:        Circuit File for OAAOI232
%%
%%  ************    LaTeX with circdia.sty package      ***************
%%
%%  ///////////////////////////////////////////////////////////////////
%%
%%  Copyright (c) 2018 - 2022 by
%%                  chipforge <stdcelllib@nospam.chipforge.org>
%%  All rights reserved.
%%
%%      This Standard Cell Library is licensed under the Libre Silicon
%%      public license; you can redistribute it and/or modify it under
%%      the terms of the Libre Silicon public license as published by
%%      the Libre Silicon alliance, either version 1 of the License, or
%%      (at your option) any later version.
%%
%%      This design is distributed in the hope that it will be useful,
%%      but WITHOUT ANY WARRANTY; without even the implied warranty of
%%      MERCHANTABILITY or FITNESS FOR A PARTICULAR PURPOSE.
%%      See the Libre Silicon Public License for more details.
%%
%%  ///////////////////////////////////////////////////////////////////
\begin{circuitdiagram}[draft]{25}{18}

    \usgate
    % ----  1st column  ----
    \pin{1}{1}{L}{A}
    \pin{1}{5}{L}{A1}
    \gate[\inputs{2}]{or}{5}{3}{R}{}{}

    % ----  2nd column  ----
    \wire{9}{3}{9}{5}
    \pin{8}{7}{L}{B}
    \pin{8}{9}{L}{B1}
    \pin{8}{11}{L}{B2}
    \gate[\inputs{4}]{and}{12}{8}{R}{}{}

    \pin{8}{13}{L}{C}
    \pin{8}{17}{L}{C1}
    \gate[\inputs{2}]{and}{12}{15}{R}{}{}

    % ----  3rd column  ----
    \wire{16}{8}{16}{11}
    \gate[\inputs{2}]{nor}{19}{13}{R}{}{}

    % ----  result ----
    \pin{24}{13}{R}{Y}

\end{circuitdiagram}
 %%  ************    LibreSilicon's StdCellLibrary   *******************
%%
%%  Organisation:   Chipforge
%%                  Germany / European Union
%%
%%  Profile:        Chipforge focus on fine System-on-Chip Cores in
%%                  Verilog HDL Code which are easy understandable and
%%                  adjustable. For further information see
%%                          www.chipforge.org
%%                  there are projects from small cores up to PCBs, too.
%%
%%  File:           StdCellLib/Documents/Circuits/OAAO232.tex
%%
%%  Purpose:        Circuit File for OAAO232
%%
%%  ************    LaTeX with circdia.sty package      ***************
%%
%%  ///////////////////////////////////////////////////////////////////
%%
%%  Copyright (c) 2019 by chipforge <stdcelllib@nospam.chipforge.org>
%%  All rights reserved.
%%
%%      This Standard Cell Library is licensed under the Libre Silicon
%%      public license; you can redistribute it and/or modify it under
%%      the terms of the Libre Silicon public license as published by
%%      the Libre Silicon alliance, either version 1 of the License, or
%%      (at your option) any later version.
%%
%%      This design is distributed in the hope that it will be useful,
%%      but WITHOUT ANY WARRANTY; without even the implied warranty of
%%      MERCHANTABILITY or FITNESS FOR A PARTICULAR PURPOSE.
%%      See the Libre Silicon Public License for more details.
%%
%%  ///////////////////////////////////////////////////////////////////
\begin{center}
    Circuit
    \begin{figure}[h]
        \begin{center}
            \begin{circuitdiagram}{31}{18}
            \usgate
            \gate[\inputs{2}]{or}{5}{15}{R}{}{}  % OR
            \gate[\inputs{4}]{and}{12}{10}{R}{}{}% AND
            \gate[\inputs{2}]{and}{12}{3}{R}{}{} % AND
            \gate[\inputs{2}]{nor}{19}{7}{R}{}{} % NOR
            \gate{not}{26}{7}{R}{}{} % NOT
            \pin{1}{1}{L}{A}     % pin A
            \wire{2}{1}{9}{1}    % wire pin A
            \pin{1}{5}{L}{A1}    % pin A1
            \wire{2}{5}{9}{5}    % wire pin A2
            \pin{1}{7}{L}{B}     % pin B
            \wire{2}{7}{9}{7}    % wire pin B
            \pin{1}{9}{L}{B1}    % pin B1
            \wire{2}{9}{9}{9}    % wire pin B1
            \pin{1}{11}{L}{B2}   % pin B2
            \wire{2}{11}{9}{11}  % wire pin B2
            \pin{1}{13}{L}{C}    % pin C
            \pin{1}{17}{L}{C1}   % pin C1
            \wire{9}{13}{9}{15}  % wire between OR and AND
            \wire{16}{3}{16}{5}  % wire between AND and NOR
            \wire{16}{9}{16}{10} % wire between AND and NOR
            \pin{30}{7}{R}{Y}    % pin Y
            \end{circuitdiagram}
        \end{center}
    \end{figure}
\end{center}

%%  ************    LibreSilicon's StdCellLibrary   *******************
%%
%%  Organisation:   Chipforge
%%                  Germany / European Union
%%
%%  Profile:        Chipforge focus on fine System-on-Chip Cores in
%%                  Verilog HDL Code which are easy understandable and
%%                  adjustable. For further information see
%%                          www.chipforge.org
%%                  there are projects from small cores up to PCBs, too.
%%
%%  File:           StdCellLib/Documents/Circuits/OAAOI233.tex
%%
%%  Purpose:        Circuit File for OAAOI233
%%
%%  ************    LaTeX with circdia.sty package      ***************
%%
%%  ///////////////////////////////////////////////////////////////////
%%
%%  Copyright (c) 2019 by chipforge <stdcelllib@nospam.chipforge.org>
%%  All rights reserved.
%%
%%      This Standard Cell Library is licensed under the Libre Silicon
%%      public license; you can redistribute it and/or modify it under
%%      the terms of the Libre Silicon public license as published by
%%      the Libre Silicon alliance, either version 1 of the License, or
%%      (at your option) any later version.
%%
%%      This design is distributed in the hope that it will be useful,
%%      but WITHOUT ANY WARRANTY; without even the implied warranty of
%%      MERCHANTABILITY or FITNESS FOR A PARTICULAR PURPOSE.
%%      See the Libre Silicon Public License for more details.
%%
%%  ///////////////////////////////////////////////////////////////////
\begin{center}
    Circuit
    \begin{figure}[h]
        \begin{center}
            \begin{circuitdiagram}{25}{18}
            \usgate
            \gate[\inputs{2}]{or}{5}{15}{R}{}{}  % OR
            \gate[\inputs{4}]{and}{12}{10}{R}{}{}% AND
            \gate[\inputs{3}]{and}{12}{3}{R}{}{} % AND
            \gate[\inputs{2}]{nor}{19}{7}{R}{}{} % NOR
            \pin{1}{1}{L}{A}     % pin A
            \wire{2}{1}{9}{1}    % wire pin A
            \pin{1}{3}{L}{A1}    % pin A1
            \wire{2}{3}{9}{3}    % wire pin A2
            \pin{1}{5}{L}{A2}    % pin A2
            \wire{2}{5}{9}{5}    % wire pin A2
            \pin{1}{7}{L}{B}     % pin B
            \wire{2}{7}{9}{7}    % wire pin B
            \pin{1}{9}{L}{B1}    % pin B1
            \wire{2}{9}{9}{9}    % wire pin B1
            \pin{1}{11}{L}{B2}   % pin B2
            \wire{2}{11}{9}{11}  % wire pin B2
            \pin{1}{13}{L}{C}    % pin C
            \pin{1}{17}{L}{C1}   % pin C1
            \wire{9}{13}{9}{15}  % wire between OR and AND
            \wire{16}{3}{16}{5}  % wire between AND and NOR
            \wire{16}{9}{16}{10} % wire between AND and NOR
            \pin{24}{7}{R}{Y}    % pin Y
            \end{circuitdiagram}
        \end{center}
    \end{figure}
\end{center}
 %%  ************    LibreSilicon's StdCellLibrary   *******************
%%
%%  Organisation:   Chipforge
%%                  Germany / European Union
%%
%%  Profile:        Chipforge focus on fine System-on-Chip Cores in
%%                  Verilog HDL Code which are easy understandable and
%%                  adjustable. For further information see
%%                          www.chipforge.org
%%                  there are projects from small cores up to PCBs, too.
%%
%%  File:           StdCellLib/Documents/Circuits/OAAO233.tex
%%
%%  Purpose:        Circuit File for OAAO233
%%
%%  ************    LaTeX with circdia.sty package      ***************
%%
%%  ///////////////////////////////////////////////////////////////////
%%
%%  Copyright (c) 2019 by chipforge <stdcelllib@nospam.chipforge.org>
%%  All rights reserved.
%%
%%      This Standard Cell Library is licensed under the Libre Silicon
%%      public license; you can redistribute it and/or modify it under
%%      the terms of the Libre Silicon public license as published by
%%      the Libre Silicon alliance, either version 1 of the License, or
%%      (at your option) any later version.
%%
%%      This design is distributed in the hope that it will be useful,
%%      but WITHOUT ANY WARRANTY; without even the implied warranty of
%%      MERCHANTABILITY or FITNESS FOR A PARTICULAR PURPOSE.
%%      See the Libre Silicon Public License for more details.
%%
%%  ///////////////////////////////////////////////////////////////////
\begin{center}
    Circuit
    \begin{figure}[h]
        \begin{center}
            \begin{circuitdiagram}{31}{18}
            \usgate
            \gate[\inputs{2}]{or}{5}{15}{R}{}{}  % OR
            \gate[\inputs{4}]{and}{12}{10}{R}{}{}% AND
            \gate[\inputs{3}]{and}{12}{3}{R}{}{} % AND
            \gate[\inputs{2}]{nor}{19}{7}{R}{}{} % NOR
            \gate{not}{26}{7}{R}{}{} % NOT
            \pin{1}{1}{L}{A}     % pin A
            \wire{2}{1}{9}{1}    % wire pin A
            \pin{1}{3}{L}{A1}    % pin A1
            \wire{2}{3}{9}{3}    % wire pin A1
            \pin{1}{5}{L}{A2}    % pin A2
            \wire{2}{5}{9}{5}    % wire pin A2
            \pin{1}{7}{L}{B}     % pin B
            \wire{2}{7}{9}{7}    % wire pin B
            \pin{1}{9}{L}{B1}    % pin B1
            \wire{2}{9}{9}{9}    % wire pin B1
            \pin{1}{11}{L}{B2}   % pin B2
            \wire{2}{11}{9}{11}  % wire pin B2
            \pin{1}{13}{L}{C}    % pin C
            \pin{1}{17}{L}{C1}   % pin C1
            \wire{9}{13}{9}{15}  % wire between OR and AND
            \wire{16}{3}{16}{5}  % wire between AND and NOR
            \wire{16}{9}{16}{10} % wire between AND and NOR
            \pin{30}{7}{R}{Y}    % pin Y
            \end{circuitdiagram}
        \end{center}
    \end{figure}
\end{center}

%%  ************    LibreSilicon's StdCellLibrary   *******************
%%
%%  Organisation:   Chipforge
%%                  Germany / European Union
%%
%%  Profile:        Chipforge focus on fine System-on-Chip Cores in
%%                  Verilog HDL Code which are easy understandable and
%%                  adjustable. For further information see
%%                          www.chipforge.org
%%                  there are projects from small cores up to PCBs, too.
%%
%%  File:           StdCellLib/Documents/Circuits/OAAOI233.tex
%%
%%  Purpose:        Circuit File for OAAOI233
%%
%%  ************    LaTeX with circdia.sty package      ***************
%%
%%  ///////////////////////////////////////////////////////////////////
%%
%%  Copyright (c) 2019 by chipforge <stdcelllib@nospam.chipforge.org>
%%  All rights reserved.
%%
%%      This Standard Cell Library is licensed under the Libre Silicon
%%      public license; you can redistribute it and/or modify it under
%%      the terms of the Libre Silicon public license as published by
%%      the Libre Silicon alliance, either version 1 of the License, or
%%      (at your option) any later version.
%%
%%      This design is distributed in the hope that it will be useful,
%%      but WITHOUT ANY WARRANTY; without even the implied warranty of
%%      MERCHANTABILITY or FITNESS FOR A PARTICULAR PURPOSE.
%%      See the Libre Silicon Public License for more details.
%%
%%  ///////////////////////////////////////////////////////////////////
\begin{center}
    Circuit
    \begin{figure}[h]
        \begin{center}
            \begin{circuitdiagram}{25}{20}
            \usgate
            \gate[\inputs{2}]{or}{5}{17}{R}{}{}  % OR
            \gate[\inputs{4}]{and}{12}{12}{R}{}{}% AND
            \gate[\inputs{4}]{and}{12}{4}{R}{}{} % AND
            \gate[\inputs{2}]{nor}{19}{8}{R}{}{} % NOR
            \pin{1}{1}{L}{A}     % pin A
            \wire{2}{1}{9}{1}    % wire pin A
            \pin{1}{3}{L}{A1}    % pin A1
            \wire{2}{3}{9}{3}    % wire pin A2
            \pin{1}{5}{L}{A2}    % pin A2
            \wire{2}{5}{9}{5}    % wire pin A2
            \pin{1}{7}{L}{A3}    % pin A3
            \wire{2}{7}{9}{7}    % wire pin A3
            \pin{1}{9}{L}{B}     % pin B
            \wire{2}{9}{9}{9}    % wire pin B
            \pin{1}{11}{L}{B1}   % pin B1
            \wire{2}{11}{9}{11}  % wire pin B1
            \pin{1}{13}{L}{B2}   % pin B2
            \wire{2}{13}{9}{13}  % wire pin B2
            \pin{1}{15}{L}{C}    % pin C
            \pin{1}{19}{L}{C1}   % pin C1
            \wire{9}{15}{9}{17}  % wire between OR and AND
            \wire{16}{4}{16}{6}  % wire between AND and NOR
            \wire{16}{10}{16}{12}% wire between AND and NOR
            \pin{24}{8}{R}{Y}    % pin Y
            \end{circuitdiagram}
        \end{center}
    \end{figure}
\end{center}
 %%  ************    LibreSilicon's StdCellLibrary   *******************
%%
%%  Organisation:   Chipforge
%%                  Germany / European Union
%%
%%  Profile:        Chipforge focus on fine System-on-Chip Cores in
%%                  Verilog HDL Code which are easy understandable and
%%                  adjustable. For further information see
%%                          www.chipforge.org
%%                  there are projects from small cores up to PCBs, too.
%%
%%  File:           StdCellLib/Documents/Circuits/OAAO233.tex
%%
%%  Purpose:        Circuit File for OAAO233
%%
%%  ************    LaTeX with circdia.sty package      ***************
%%
%%  ///////////////////////////////////////////////////////////////////
%%
%%  Copyright (c) 2019 by chipforge <stdcelllib@nospam.chipforge.org>
%%  All rights reserved.
%%
%%      This Standard Cell Library is licensed under the Libre Silicon
%%      public license; you can redistribute it and/or modify it under
%%      the terms of the Libre Silicon public license as published by
%%      the Libre Silicon alliance, either version 1 of the License, or
%%      (at your option) any later version.
%%
%%      This design is distributed in the hope that it will be useful,
%%      but WITHOUT ANY WARRANTY; without even the implied warranty of
%%      MERCHANTABILITY or FITNESS FOR A PARTICULAR PURPOSE.
%%      See the Libre Silicon Public License for more details.
%%
%%  ///////////////////////////////////////////////////////////////////
\begin{center}
    Circuit
    \begin{figure}[h]
        \begin{center}
            \begin{circuitdiagram}{31}{20}
            \usgate
            \gate[\inputs{2}]{or}{5}{17}{R}{}{}  % OR
            \gate[\inputs{4}]{and}{12}{12}{R}{}{}% AND
            \gate[\inputs{4}]{and}{12}{4}{R}{}{} % AND
            \gate[\inputs{2}]{nor}{19}{8}{R}{}{} % NOR
            \gate{not}{26}{8}{R}{}{} % NOT
            \pin{1}{1}{L}{A}     % pin A
            \wire{2}{1}{9}{1}    % wire pin A
            \pin{1}{3}{L}{A1}    % pin A1
            \wire{2}{3}{9}{3}    % wire pin A2
            \pin{1}{5}{L}{A2}    % pin A2
            \wire{2}{5}{9}{5}    % wire pin A2
            \pin{1}{7}{L}{A3}    % pin A3
            \wire{2}{7}{9}{7}    % wire pin A3
            \pin{1}{9}{L}{B}     % pin B
            \wire{2}{9}{9}{9}    % wire pin B
            \pin{1}{11}{L}{B1}   % pin B1
            \wire{2}{11}{9}{11}  % wire pin B1
            \pin{1}{13}{L}{B2}   % pin B2
            \wire{2}{13}{9}{13}  % wire pin B2
            \pin{1}{15}{L}{C}    % pin C
            \pin{1}{19}{L}{C1}   % pin C1
            \wire{9}{15}{9}{17}  % wire between OR and AND
            \wire{16}{4}{16}{6}  % wire between AND and NOR
            \wire{16}{10}{16}{12}% wire between AND and NOR
            \pin{30}{8}{R}{Z}    % pin Z
            \end{circuitdiagram}
        \end{center}
    \end{figure}
\end{center}

%%  ************    LibreSilicon's StdCellLibrary   *******************
%%
%%  Organisation:   Chipforge
%%                  Germany / European Union
%%
%%  Profile:        Chipforge focus on fine System-on-Chip Cores in
%%                  Verilog HDL Code which are easy understandable and
%%                  adjustable. For further information see
%%                          www.chipforge.org
%%                  there are projects from small cores up to PCBs, too.
%%
%%  File:           StdCellLib/Documents/Datasheets/Circuitry/OAAOI312.tex
%%
%%  Purpose:        Circuit File for OAAOI312
%%
%%  ************    LaTeX with circdia.sty package      ***************
%%
%%  ///////////////////////////////////////////////////////////////////
%%
%%  Copyright (c) 2018 - 2022 by
%%                  chipforge <stdcelllib@nospam.chipforge.org>
%%  All rights reserved.
%%
%%      This Standard Cell Library is licensed under the Libre Silicon
%%      public license; you can redistribute it and/or modify it under
%%      the terms of the Libre Silicon public license as published by
%%      the Libre Silicon alliance, either version 1 of the License, or
%%      (at your option) any later version.
%%
%%      This design is distributed in the hope that it will be useful,
%%      but WITHOUT ANY WARRANTY; without even the implied warranty of
%%      MERCHANTABILITY or FITNESS FOR A PARTICULAR PURPOSE.
%%      See the Libre Silicon Public License for more details.
%%
%%  ///////////////////////////////////////////////////////////////////
\begin{circuitdiagram}[draft]{25}{14}

    \usgate
    % ----  1st column  ----
    \pin{1}{1}{L}{A}
    \pin{1}{3}{L}{A1}
    \pin{1}{5}{L}{A2}
    \gate[\inputs{3}]{or}{5}{3}{R}{}{}

    % ----  2nd column  ----
    \pin{8}{7}{L}{B}
    \gate[\inputs{2}]{and}{12}{5}{R}{}{}

    \pin{8}{9}{L}{C}
    \pin{8}{13}{L}{C1}
    \gate[\inputs{2}]{and}{12}{11}{R}{}{}

    % ----  3rd column  ----
    \wire{16}{5}{16}{7}
    \gate[\inputs{2}]{nor}{19}{9}{R}{}{}

    % ----  result ----
    \pin{24}{9}{R}{Y}

\end{circuitdiagram}
 %%  ************    LibreSilicon's StdCellLibrary   *******************
%%
%%  Organisation:   Chipforge
%%                  Germany / European Union
%%
%%  Profile:        Chipforge focus on fine System-on-Chip Cores in
%%                  Verilog HDL Code which are easy understandable and
%%                  adjustable. For further information see
%%                          www.chipforge.org
%%                  there are projects from small cores up to PCBs, too.
%%
%%  File:           StdCellLib/Documents/Datasheets/Circuitry/OAAO312.tex
%%
%%  Purpose:        Circuit File for OAAO312
%%
%%  ************    LaTeX with circdia.sty package      ***************
%%
%%  ///////////////////////////////////////////////////////////////////
%%
%%  Copyright (c) 2018 - 2022 by
%%                  chipforge <stdcelllib@nospam.chipforge.org>
%%  All rights reserved.
%%
%%      This Standard Cell Library is licensed under the Libre Silicon
%%      public license; you can redistribute it and/or modify it under
%%      the terms of the Libre Silicon public license as published by
%%      the Libre Silicon alliance, either version 1 of the License, or
%%      (at your option) any later version.
%%
%%      This design is distributed in the hope that it will be useful,
%%      but WITHOUT ANY WARRANTY; without even the implied warranty of
%%      MERCHANTABILITY or FITNESS FOR A PARTICULAR PURPOSE.
%%      See the Libre Silicon Public License for more details.
%%
%%  ///////////////////////////////////////////////////////////////////
\begin{circuitdiagram}[draft]{31}{14}

    \usgate
    % ----  1st column  ----
    \pin{1}{1}{L}{A}
    \pin{1}{3}{L}{A1}
    \pin{1}{5}{L}{A2}
    \gate[\inputs{3}]{or}{5}{3}{R}{}{}

    % ----  2nd column  ----
    \pin{8}{7}{L}{B}
    \gate[\inputs{2}]{and}{12}{5}{R}{}{}

    \pin{8}{9}{L}{C}
    \pin{8}{13}{L}{C1}
    \gate[\inputs{2}]{and}{12}{11}{R}{}{}

    % ----  3rd column  ----
    \wire{16}{5}{16}{7}
    \gate[\inputs{2}]{nor}{19}{9}{R}{}{}

    % ----  4th column  ----
    \gate{not}{26}{9}{R}{}{}

    % ----  result ----
    \pin{30}{9}{R}{Z}

\end{circuitdiagram}

%%  ************    LibreSilicon's StdCellLibrary   *******************
%%
%%  Organisation:   Chipforge
%%                  Germany / European Union
%%
%%  Profile:        Chipforge focus on fine System-on-Chip Cores in
%%                  Verilog HDL Code which are easy understandable and
%%                  adjustable. For further information see
%%                          www.chipforge.org
%%                  there are projects from small cores up to PCBs, too.
%%
%%  File:           StdCellLib/Documents/Datasheets/Circuitry/OAAOI313.tex
%%
%%  Purpose:        Circuit File for OAAOI313
%%
%%  ************    LaTeX with circdia.sty package      ***************
%%
%%  ///////////////////////////////////////////////////////////////////
%%
%%  Copyright (c) 2018 - 2022 by
%%                  chipforge <stdcelllib@nospam.chipforge.org>
%%  All rights reserved.
%%
%%      This Standard Cell Library is licensed under the Libre Silicon
%%      public license; you can redistribute it and/or modify it under
%%      the terms of the Libre Silicon public license as published by
%%      the Libre Silicon alliance, either version 1 of the License, or
%%      (at your option) any later version.
%%
%%      This design is distributed in the hope that it will be useful,
%%      but WITHOUT ANY WARRANTY; without even the implied warranty of
%%      MERCHANTABILITY or FITNESS FOR A PARTICULAR PURPOSE.
%%      See the Libre Silicon Public License for more details.
%%
%%  ///////////////////////////////////////////////////////////////////
\begin{circuitdiagram}[draft]{25}{14}

    \usgate
    % ----  1st column  ----
    \pin{1}{1}{L}{A}
    \pin{1}{3}{L}{A1}
    \pin{1}{5}{L}{A2}
    \gate[\inputs{3}]{or}{5}{3}{R}{}{}

    % ----  2nd column  ----
    \pin{8}{7}{L}{B}
    \gate[\inputs{2}]{and}{12}{5}{R}{}{}

    \pin{8}{9}{L}{C}
    \pin{8}{11}{L}{C1}
    \pin{8}{13}{L}{C2}
    \gate[\inputs{3}]{and}{12}{11}{R}{}{}

    % ----  3rd column  ----
    \wire{16}{5}{16}{7}
    \gate[\inputs{2}]{nor}{19}{9}{R}{}{}

    % ----  result ----
    \pin{24}{9}{R}{Y}

\end{circuitdiagram}
 %%  ************    LibreSilicon's StdCellLibrary   *******************
%%
%%  Organisation:   Chipforge
%%                  Germany / European Union
%%
%%  Profile:        Chipforge focus on fine System-on-Chip Cores in
%%                  Verilog HDL Code which are easy understandable and
%%                  adjustable. For further information see
%%                          www.chipforge.org
%%                  there are projects from small cores up to PCBs, too.
%%
%%  File:           StdCellLib/Documents/Datasheets/Circuitry/OAAO313.tex
%%
%%  Purpose:        Circuit File for OAAO313
%%
%%  ************    LaTeX with circdia.sty package      ***************
%%
%%  ///////////////////////////////////////////////////////////////////
%%
%%  Copyright (c) 2018 - 2022 by
%%                  chipforge <stdcelllib@nospam.chipforge.org>
%%  All rights reserved.
%%
%%      This Standard Cell Library is licensed under the Libre Silicon
%%      public license; you can redistribute it and/or modify it under
%%      the terms of the Libre Silicon public license as published by
%%      the Libre Silicon alliance, either version 1 of the License, or
%%      (at your option) any later version.
%%
%%      This design is distributed in the hope that it will be useful,
%%      but WITHOUT ANY WARRANTY; without even the implied warranty of
%%      MERCHANTABILITY or FITNESS FOR A PARTICULAR PURPOSE.
%%      See the Libre Silicon Public License for more details.
%%
%%  ///////////////////////////////////////////////////////////////////
\begin{circuitdiagram}[draft]{31}{14}

    \usgate
    % ----  1st column  ----
    \pin{1}{1}{L}{A}
    \pin{1}{3}{L}{A1}
    \pin{1}{5}{L}{A2}
    \gate[\inputs{3}]{or}{5}{3}{R}{}{}

    % ----  2nd column  ----
    \pin{8}{7}{L}{B}
    \gate[\inputs{2}]{and}{12}{5}{R}{}{}

    \pin{8}{9}{L}{C}
    \pin{8}{11}{L}{C1}
    \pin{8}{13}{L}{C2}
    \gate[\inputs{3}]{and}{12}{11}{R}{}{}

    % ----  3rd column  ----
    \wire{16}{5}{16}{7}
    \gate[\inputs{2}]{nor}{19}{9}{R}{}{}

    % ----  4th column  ----
    \gate{not}{26}{9}{R}{}{}

    % ----  result ----
    \pin{30}{9}{R}{Z}

\end{circuitdiagram}

%%  ************    LibreSilicon's StdCellLibrary   *******************
%%
%%  Organisation:   Chipforge
%%                  Germany / European Union
%%
%%  Profile:        Chipforge focus on fine System-on-Chip Cores in
%%                  Verilog HDL Code which are easy understandable and
%%                  adjustable. For further information see
%%                          www.chipforge.org
%%                  there are projects from small cores up to PCBs, too.
%%
%%  File:           StdCellLib/Documents/Datasheets/Circuitry/OAAOI314.tex
%%
%%  Purpose:        Circuit File for OAAOI314
%%
%%  ************    LaTeX with circdia.sty package      ***************
%%
%%  ///////////////////////////////////////////////////////////////////
%%
%%  Copyright (c) 2018 - 2022 by
%%                  chipforge <stdcelllib@nospam.chipforge.org>
%%  All rights reserved.
%%
%%      This Standard Cell Library is licensed under the Libre Silicon
%%      public license; you can redistribute it and/or modify it under
%%      the terms of the Libre Silicon public license as published by
%%      the Libre Silicon alliance, either version 1 of the License, or
%%      (at your option) any later version.
%%
%%      This design is distributed in the hope that it will be useful,
%%      but WITHOUT ANY WARRANTY; without even the implied warranty of
%%      MERCHANTABILITY or FITNESS FOR A PARTICULAR PURPOSE.
%%      See the Libre Silicon Public License for more details.
%%
%%  ///////////////////////////////////////////////////////////////////
\begin{circuitdiagram}[draft]{25}{16}

    \usgate
    % ----  1st column  ----
    \pin{1}{1}{L}{A}
    \pin{1}{3}{L}{A1}
    \pin{1}{5}{L}{A2}
    \gate[\inputs{3}]{or}{5}{3}{R}{}{}

    % ----  2nd column  ----
    \pin{8}{7}{L}{B}
    \gate[\inputs{2}]{and}{12}{5}{R}{}{}

    \pin{8}{9}{L}{C}
    \pin{8}{11}{L}{C1}
    \pin{8}{13}{L}{C2}
    \pin{8}{15}{L}{C3}
    \gate[\inputs{4}]{and}{12}{12}{R}{}{}

    % ----  3rd column  ----
    \wire{16}{5}{16}{8}
    \gate[\inputs{2}]{nor}{19}{10}{R}{}{}

    % ----  result ----
    \pin{24}{10}{R}{Y}

\end{circuitdiagram}
 %%  ************    LibreSilicon's StdCellLibrary   *******************
%%
%%  Organisation:   Chipforge
%%                  Germany / European Union
%%
%%  Profile:        Chipforge focus on fine System-on-Chip Cores in
%%                  Verilog HDL Code which are easy understandable and
%%                  adjustable. For further information see
%%                          www.chipforge.org
%%                  there are projects from small cores up to PCBs, too.
%%
%%  File:           StdCellLib/Documents/Datasheets/Circuitry/OAAO314.tex
%%
%%  Purpose:        Circuit File for OAAO314
%%
%%  ************    LaTeX with circdia.sty package      ***************
%%
%%  ///////////////////////////////////////////////////////////////////
%%
%%  Copyright (c) 2018 - 2022 by
%%                  chipforge <stdcelllib@nospam.chipforge.org>
%%  All rights reserved.
%%
%%      This Standard Cell Library is licensed under the Libre Silicon
%%      public license; you can redistribute it and/or modify it under
%%      the terms of the Libre Silicon public license as published by
%%      the Libre Silicon alliance, either version 1 of the License, or
%%      (at your option) any later version.
%%
%%      This design is distributed in the hope that it will be useful,
%%      but WITHOUT ANY WARRANTY; without even the implied warranty of
%%      MERCHANTABILITY or FITNESS FOR A PARTICULAR PURPOSE.
%%      See the Libre Silicon Public License for more details.
%%
%%  ///////////////////////////////////////////////////////////////////
\begin{circuitdiagram}[draft]{31}{16}

    \usgate
    % ----  1st column  ----
    \pin{1}{1}{L}{A}
    \pin{1}{3}{L}{A1}
    \pin{1}{5}{L}{A2}
    \gate[\inputs{3}]{or}{5}{3}{R}{}{}

    % ----  2nd column  ----
    \pin{8}{7}{L}{B}
    \gate[\inputs{2}]{and}{12}{5}{R}{}{}

    \pin{8}{9}{L}{C}
    \pin{8}{11}{L}{C1}
    \pin{8}{13}{L}{C2}
    \pin{8}{15}{L}{C3}
    \gate[\inputs{4}]{and}{12}{12}{R}{}{}

    % ----  3rd column  ----
    \wire{16}{5}{16}{8}
    \gate[\inputs{2}]{nor}{19}{10}{R}{}{}

    % ----  last column ----
    \gate{not}{26}{10}{R}{}{}

    % ----  result ----
    \pin{30}{10}{R}{Z}

\end{circuitdiagram}

%%  ************    LibreSilicon's StdCellLibrary   *******************
%%
%%  Organisation:   Chipforge
%%                  Germany / European Union
%%
%%  Profile:        Chipforge focus on fine System-on-Chip Cores in
%%                  Verilog HDL Code which are easy understandable and
%%                  adjustable. For further information see
%%                          www.chipforge.org
%%                  there are projects from small cores up to PCBs, too.
%%
%%  File:           StdCellLib/Documents/Datasheets/Circuitry/OAAOI322.tex
%%
%%  Purpose:        Circuit File for OAAOI322
%%
%%  ************    LaTeX with circdia.sty package      ***************
%%
%%  ///////////////////////////////////////////////////////////////////
%%
%%  Copyright (c) 2018 - 2022 by
%%                  chipforge <stdcelllib@nospam.chipforge.org>
%%  All rights reserved.
%%
%%      This Standard Cell Library is licensed under the Libre Silicon
%%      public license; you can redistribute it and/or modify it under
%%      the terms of the Libre Silicon public license as published by
%%      the Libre Silicon alliance, either version 1 of the License, or
%%      (at your option) any later version.
%%
%%      This design is distributed in the hope that it will be useful,
%%      but WITHOUT ANY WARRANTY; without even the implied warranty of
%%      MERCHANTABILITY or FITNESS FOR A PARTICULAR PURPOSE.
%%      See the Libre Silicon Public License for more details.
%%
%%  ///////////////////////////////////////////////////////////////////
\begin{circuitdiagram}[draft]{25}{16}

    \usgate
    % ----  1st column  ----
    \pin{1}{1}{L}{A}
    \pin{1}{3}{L}{A1}
    \pin{1}{5}{L}{A2}
    \gate[\inputs{3}]{or}{5}{3}{R}{}{}

    % ----  2nd column  ----
    \wire{9}{3}{9}{5}
    \pin{8}{7}{L}{B}
    \pin{8}{9}{L}{B1}
    \gate[\inputs{3}]{and}{12}{7}{R}{}{}

    \pin{8}{11}{L}{C}
    \pin{8}{15}{L}{C1}
    \gate[\inputs{2}]{and}{12}{13}{R}{}{}

    % ----  3rd column  ----
    \wire{16}{7}{16}{9}
    \gate[\inputs{2}]{nor}{19}{11}{R}{}{}

    % ----  result ----
    \pin{24}{11}{R}{Y}

\end{circuitdiagram}
 %%  ************    LibreSilicon's StdCellLibrary   *******************
%%
%%  Organisation:   Chipforge
%%                  Germany / European Union
%%
%%  Profile:        Chipforge focus on fine System-on-Chip Cores in
%%                  Verilog HDL Code which are easy understandable and
%%                  adjustable. For further information see
%%                          www.chipforge.org
%%                  there are projects from small cores up to PCBs, too.
%%
%%  File:           StdCellLib/Documents/Datasheets/Circuitry/OAAO322.tex
%%
%%  Purpose:        Circuit File for OAAO322
%%
%%  ************    LaTeX with circdia.sty package      ***************
%%
%%  ///////////////////////////////////////////////////////////////////
%%
%%  Copyright (c) 2018 - 2022 by
%%                  chipforge <stdcelllib@nospam.chipforge.org>
%%  All rights reserved.
%%
%%      This Standard Cell Library is licensed under the Libre Silicon
%%      public license; you can redistribute it and/or modify it under
%%      the terms of the Libre Silicon public license as published by
%%      the Libre Silicon alliance, either version 1 of the License, or
%%      (at your option) any later version.
%%
%%      This design is distributed in the hope that it will be useful,
%%      but WITHOUT ANY WARRANTY; without even the implied warranty of
%%      MERCHANTABILITY or FITNESS FOR A PARTICULAR PURPOSE.
%%      See the Libre Silicon Public License for more details.
%%
%%  ///////////////////////////////////////////////////////////////////
\begin{circuitdiagram}[draft]{31}{16}

    \usgate
    % ----  1st column  ----
    \pin{1}{1}{L}{A}
    \pin{1}{3}{L}{A1}
    \pin{1}{5}{L}{A2}
    \gate[\inputs{3}]{or}{5}{3}{R}{}{}

    % ----  2nd column  ----
    \wire{9}{3}{9}{5}
    \pin{8}{7}{L}{B}
    \pin{8}{9}{L}{B1}
    \gate[\inputs{3}]{and}{12}{7}{R}{}{}

    \pin{8}{11}{L}{C}
    \pin{8}{15}{L}{C1}
    \gate[\inputs{2}]{and}{12}{13}{R}{}{}

    % ----  3rd column  ----
    \wire{16}{7}{16}{9}
    \gate[\inputs{2}]{nor}{19}{11}{R}{}{}

    % ----  4th column  ----
    \gate{not}{26}{11}{R}{}{}

    % ----  result ----
    \pin{30}{11}{R}{Z}

\end{circuitdiagram}

%%  ************    LibreSilicon's StdCellLibrary   *******************
%%
%%  Organisation:   Chipforge
%%                  Germany / European Union
%%
%%  Profile:        Chipforge focus on fine System-on-Chip Cores in
%%                  Verilog HDL Code which are easy understandable and
%%                  adjustable. For further information see
%%                          www.chipforge.org
%%                  there are projects from small cores up to PCBs, too.
%%
%%  File:           StdCellLib/Documents/Datasheets/Circuitry/OAAOI323.tex
%%
%%  Purpose:        Circuit File for OAAOI323
%%
%%  ************    LaTeX with circdia.sty package      ***************
%%
%%  ///////////////////////////////////////////////////////////////////
%%
%%  Copyright (c) 2018 - 2022 by
%%                  chipforge <stdcelllib@nospam.chipforge.org>
%%  All rights reserved.
%%
%%      This Standard Cell Library is licensed under the Libre Silicon
%%      public license; you can redistribute it and/or modify it under
%%      the terms of the Libre Silicon public license as published by
%%      the Libre Silicon alliance, either version 1 of the License, or
%%      (at your option) any later version.
%%
%%      This design is distributed in the hope that it will be useful,
%%      but WITHOUT ANY WARRANTY; without even the implied warranty of
%%      MERCHANTABILITY or FITNESS FOR A PARTICULAR PURPOSE.
%%      See the Libre Silicon Public License for more details.
%%
%%  ///////////////////////////////////////////////////////////////////
\begin{circuitdiagram}[draft]{25}{16}

    \usgate
    % ----  1st column  ----
    \pin{1}{1}{L}{A}
    \pin{1}{3}{L}{A1}
    \pin{1}{5}{L}{A2}
    \gate[\inputs{3}]{or}{5}{3}{R}{}{}

    % ----  2nd column  ----
    \wire{9}{3}{9}{5}
    \pin{8}{7}{L}{B}
    \pin{8}{9}{L}{B1}
    \gate[\inputs{3}]{and}{12}{7}{R}{}{}

    \pin{8}{11}{L}{C}
    \pin{8}{13}{L}{C1}
    \pin{8}{15}{L}{C1}
    \gate[\inputs{3}]{and}{12}{13}{R}{}{}

    % ----  3rd column  ----
    \wire{16}{7}{16}{9}
    \gate[\inputs{2}]{nor}{19}{11}{R}{}{}

    % ----  result ----
    \pin{24}{11}{R}{Y}

\end{circuitdiagram}
 %%  ************    LibreSilicon's StdCellLibrary   *******************
%%
%%  Organisation:   Chipforge
%%                  Germany / European Union
%%
%%  Profile:        Chipforge focus on fine System-on-Chip Cores in
%%                  Verilog HDL Code which are easy understandable and
%%                  adjustable. For further information see
%%                          www.chipforge.org
%%                  there are projects from small cores up to PCBs, too.
%%
%%  File:           StdCellLib/Documents/Datasheets/Circuitry/OAAO323.tex
%%
%%  Purpose:        Circuit File for OAAO323
%%
%%  ************    LaTeX with circdia.sty package      ***************
%%
%%  ///////////////////////////////////////////////////////////////////
%%
%%  Copyright (c) 2018 - 2022 by
%%                  chipforge <stdcelllib@nospam.chipforge.org>
%%  All rights reserved.
%%
%%      This Standard Cell Library is licensed under the Libre Silicon
%%      public license; you can redistribute it and/or modify it under
%%      the terms of the Libre Silicon public license as published by
%%      the Libre Silicon alliance, either version 1 of the License, or
%%      (at your option) any later version.
%%
%%      This design is distributed in the hope that it will be useful,
%%      but WITHOUT ANY WARRANTY; without even the implied warranty of
%%      MERCHANTABILITY or FITNESS FOR A PARTICULAR PURPOSE.
%%      See the Libre Silicon Public License for more details.
%%
%%  ///////////////////////////////////////////////////////////////////
\begin{circuitdiagram}[draft]{31}{16}

    \usgate
    % ----  1st column  ----
    \pin{1}{1}{L}{A}
    \pin{1}{3}{L}{A1}
    \pin{1}{5}{L}{A2}
    \gate[\inputs{3}]{or}{5}{3}{R}{}{}

    % ----  2nd column  ----
    \wire{9}{3}{9}{5}
    \pin{8}{7}{L}{B}
    \pin{8}{9}{L}{B1}
    \gate[\inputs{3}]{and}{12}{7}{R}{}{}

    \pin{8}{11}{L}{C}
    \pin{8}{13}{L}{C1}
    \pin{8}{15}{L}{C1}
    \gate[\inputs{3}]{and}{12}{13}{R}{}{}

    % ----  3rd column  ----
    \wire{16}{7}{16}{9}
    \gate[\inputs{2}]{nor}{19}{11}{R}{}{}

    % ----  last column ----
    \gate{not}{26}{11}{R}{}{}

    % ----  result ----
    \pin{30}{11}{R}{Z}

\end{circuitdiagram}


%%  ************    LibreSilicon's StdCellLibrary   *******************
%%
%%  Organisation:   Chipforge
%%                  Germany / European Union
%%
%%  Profile:        Chipforge focus on fine System-on-Chip Cores in
%%                  Verilog HDL Code which are easy understandable and
%%                  adjustable. For further information see
%%                          www.chipforge.org
%%                  there are projects from small cores up to PCBs, too.
%%
%%  File:           StdCellLib/Documents/Datasheets/Circuitry/OAAOI2121.tex
%%
%%  Purpose:        Circuit File for OAAOI2121
%%
%%  ************    LaTeX with circdia.sty package      ***************
%%
%%  ///////////////////////////////////////////////////////////////////
%%
%%  Copyright (c) 2018 - 2022 by
%%                  chipforge <stdcelllib@nospam.chipforge.org>
%%  All rights reserved.
%%
%%      This Standard Cell Library is licensed under the Libre Silicon
%%      public license; you can redistribute it and/or modify it under
%%      the terms of the Libre Silicon public license as published by
%%      the Libre Silicon alliance, either version 1 of the License, or
%%      (at your option) any later version.
%%
%%      This design is distributed in the hope that it will be useful,
%%      but WITHOUT ANY WARRANTY; without even the implied warranty of
%%      MERCHANTABILITY or FITNESS FOR A PARTICULAR PURPOSE.
%%      See the Libre Silicon Public License for more details.
%%
%%  ///////////////////////////////////////////////////////////////////
\begin{circuitdiagram}[draft]{25}{16}

    \usgate
    % ----  1st column  ----
    \pin{1}{1}{L}{A}
    \pin{1}{5}{L}{A1}
    \gate[\inputs{2}]{or}{5}{3}{R}{}{}

    % ----  2nd column  ----
    \pin{8}{7}{L}{B}
    \gate[\inputs{2}]{and}{12}{5}{R}{}{}

    \pin{8}{9}{L}{C}
    \pin{8}{13}{L}{C1}
    \gate[\inputs{2}]{and}{12}{11}{R}{}{}

    % ----  3rd column  ----
    \pin{15}{15}{L}{D}
    \wire{16}{5}{16}{9}
    \wire{16}{13}{16}{15}
    \gate[\inputs{3}]{nor}{19}{11}{R}{}{}

    % ----  result ----
    \pin{24}{11}{R}{Y}

\end{circuitdiagram}
 %%  ************    LibreSilicon's StdCellLibrary   *******************
%%
%%  Organisation:   Chipforge
%%                  Germany / European Union
%%
%%  Profile:        Chipforge focus on fine System-on-Chip Cores in
%%                  Verilog HDL Code which are easy understandable and
%%                  adjustable. For further information see
%%                          www.chipforge.org
%%                  there are projects from small cores up to PCBs, too.
%%
%%  File:           StdCellLib/Documents/Circuits/OAAO2121.tex
%%
%%  Purpose:        Circuit File for OAAO2121
%%
%%  ************    LaTeX with circdia.sty package      ***************
%%
%%  ///////////////////////////////////////////////////////////////////
%%
%%  Copyright (c) 2019 by chipforge <stdcelllib@nospam.chipforge.org>
%%  All rights reserved.
%%
%%      This Standard Cell Library is licensed under the Libre Silicon
%%      public license; you can redistribute it and/or modify it under
%%      the terms of the Libre Silicon public license as published by
%%      the Libre Silicon alliance, either version 1 of the License, or
%%      (at your option) any later version.
%%
%%      This design is distributed in the hope that it will be useful,
%%      but WITHOUT ANY WARRANTY; without even the implied warranty of
%%      MERCHANTABILITY or FITNESS FOR A PARTICULAR PURPOSE.
%%      See the Libre Silicon Public License for more details.
%%
%%  ///////////////////////////////////////////////////////////////////
\begin{center}
    Circuit
    \begin{figure}[h]
        \begin{center}
            \begin{circuitdiagram}{31}{16}
            \usgate
            \gate[\inputs{2}]{or}{5}{13}{R}{}{}  % OR
            \gate[\inputs{2}]{and}{12}{11}{R}{}{} % AND
            \gate[\inputs{2}]{and}{12}{5}{R}{}{} % AND
            \gate[\inputs{3}]{nor}{19}{5}{R}{}{} % NOR
            \gate{not}{26}{5}{R}{}{} % NOT
            \pin{1}{1}{L}{A}     % pin A
            \wire{2}{1}{16}{1}   % wire pin A
            \pin{1}{3}{L}{B}     % pin B
            \wire{2}{3}{9}{3}    % wire pin B
            \pin{1}{7}{L}{B1}    % pin B1
            \wire{2}{7}{9}{7}    % wire pin B1
            \pin{1}{9}{L}{C}     % pin C
            \wire{2}{9}{9}{9}    % wire pin C
            \pin{1}{11}{L}{D}    % pin D
            \pin{1}{15}{L}{D1}   % pin D1
            \wire{16}{1}{16}{3}  % wire between AND and NOR
            \wire{16}{7}{16}{11} % wire between AND and NOR
            \pin{30}{5}{R}{Z}    % pin Z
            \end{circuitdiagram}
        \end{center}
    \end{figure}
\end{center}

%%  ************    LibreSilicon's StdCellLibrary   *******************
%%
%%  Organisation:   Chipforge
%%                  Germany / European Union
%%
%%  Profile:        Chipforge focus on fine System-on-Chip Cores in
%%                  Verilog HDL Code which are easy understandable and
%%                  adjustable. For further information see
%%                          www.chipforge.org
%%                  there are projects from small cores up to PCBs, too.
%%
%%  File:           StdCellLib/Documents/Datasheets/Circuitry/OAAOI2131.tex
%%
%%  Purpose:        Circuit File for OAAOI2321
%%
%%  ************    LaTeX with circdia.sty package      ***************
%%
%%  ///////////////////////////////////////////////////////////////////
%%
%%  Copyright (c) 2018 - 2022 by
%%                  chipforge <stdcelllib@nospam.chipforge.org>
%%  All rights reserved.
%%
%%      This Standard Cell Library is licensed under the Libre Silicon
%%      public license; you can redistribute it and/or modify it under
%%      the terms of the Libre Silicon public license as published by
%%      the Libre Silicon alliance, either version 1 of the License, or
%%      (at your option) any later version.
%%
%%      This design is distributed in the hope that it will be useful,
%%      but WITHOUT ANY WARRANTY; without even the implied warranty of
%%      MERCHANTABILITY or FITNESS FOR A PARTICULAR PURPOSE.
%%      See the Libre Silicon Public License for more details.
%%
%%  ///////////////////////////////////////////////////////////////////
\begin{circuitdiagram}[draft]{25}{16}

    \usgate
    % ----  1st column  ----
    \pin{1}{1}{L}{A}
    \pin{1}{5}{L}{A1}
    \gate[\inputs{2}]{or}{5}{3}{R}{}{}

    % ----  2nd column  ----
    \pin{8}{7}{L}{B}
    \gate[\inputs{2}]{and}{12}{5}{R}{}{}

    \pin{8}{9}{L}{C}
    \pin{8}{11}{L}{C1}
    \pin{8}{13}{L}{C2}
    \gate[\inputs{3}]{and}{12}{11}{R}{}{}

    % ----  3rd column  ----
    \pin{15}{15}{L}{D}
    \wire{16}{5}{16}{9}
    \wire{16}{13}{16}{15}
    \gate[\inputs{3}]{nor}{19}{11}{R}{}{}

    % ----  result ----
    \pin{24}{11}{R}{Y}

\end{circuitdiagram}
 %%  ************    LibreSilicon's StdCellLibrary   *******************
%%
%%  Organisation:   Chipforge
%%                  Germany / European Union
%%
%%  Profile:        Chipforge focus on fine System-on-Chip Cores in
%%                  Verilog HDL Code which are easy understandable and
%%                  adjustable. For further information see
%%                          www.chipforge.org
%%                  there are projects from small cores up to PCBs, too.
%%
%%  File:           StdCellLib/Documents/Datasheets/Circuitry/OAAO2131.tex
%%
%%  Purpose:        Circuit File for OAAO2131
%%
%%  ************    LaTeX with circdia.sty package      ***************
%%
%%  ///////////////////////////////////////////////////////////////////
%%
%%  Copyright (c) 2018 - 2022 by
%%                  chipforge <stdcelllib@nospam.chipforge.org>
%%  All rights reserved.
%%
%%      This Standard Cell Library is licensed under the Libre Silicon
%%      public license; you can redistribute it and/or modify it under
%%      the terms of the Libre Silicon public license as published by
%%      the Libre Silicon alliance, either version 1 of the License, or
%%      (at your option) any later version.
%%
%%      This design is distributed in the hope that it will be useful,
%%      but WITHOUT ANY WARRANTY; without even the implied warranty of
%%      MERCHANTABILITY or FITNESS FOR A PARTICULAR PURPOSE.
%%      See the Libre Silicon Public License for more details.
%%
%%  ///////////////////////////////////////////////////////////////////
\begin{circuitdiagram}[draft]{31}{16}

    \usgate
    % ----  1st column  ----
    \pin{1}{1}{L}{A}
    \pin{1}{5}{L}{A1}
    \gate[\inputs{2}]{or}{5}{3}{R}{}{}

    % ----  2nd column  ----
    \pin{8}{7}{L}{B}
    \gate[\inputs{2}]{and}{12}{5}{R}{}{}

    \pin{8}{9}{L}{C}
    \pin{8}{11}{L}{C1}
    \pin{8}{13}{L}{C2}
    \gate[\inputs{3}]{and}{12}{11}{R}{}{}

    % ----  3rd column  ----
    \pin{15}{15}{L}{D}
    \wire{16}{5}{16}{9}
    \wire{16}{13}{16}{15}
    \gate[\inputs{3}]{nor}{19}{11}{R}{}{}

    % ----  4th column  ----
    \gate{not}{26}{11}{R}{}{}

    % ----  result ----
    \pin{30}{11}{R}{Z}

\end{circuitdiagram}

%%  ************    LibreSilicon's StdCellLibrary   *******************
%%
%%  Organisation:   Chipforge
%%                  Germany / European Union
%%
%%  Profile:        Chipforge focus on fine System-on-Chip Cores in
%%                  Verilog HDL Code which are easy understandable and
%%                  adjustable. For further information see
%%                          www.chipforge.org
%%                  there are projects from small cores up to PCBs, too.
%%
%%  File:           StdCellLib/Documents/Datasheets/Circuitry/OAAOI2141.tex
%%
%%  Purpose:        Circuit File for OAAOI2141
%%
%%  ************    LaTeX with circdia.sty package      ***************
%%
%%  ///////////////////////////////////////////////////////////////////
%%
%%  Copyright (c) 2018 - 2022 by
%%                  chipforge <stdcelllib@nospam.chipforge.org>
%%  All rights reserved.
%%
%%      This Standard Cell Library is licensed under the Libre Silicon
%%      public license; you can redistribute it and/or modify it under
%%      the terms of the Libre Silicon public license as published by
%%      the Libre Silicon alliance, either version 1 of the License, or
%%      (at your option) any later version.
%%
%%      This design is distributed in the hope that it will be useful,
%%      but WITHOUT ANY WARRANTY; without even the implied warranty of
%%      MERCHANTABILITY or FITNESS FOR A PARTICULAR PURPOSE.
%%      See the Libre Silicon Public License for more details.
%%
%%  ///////////////////////////////////////////////////////////////////
\begin{circuitdiagram}[draft]{25}{17}

    \usgate
    % ----  1st column  ----
    \pin{1}{1}{L}{A}
    \pin{1}{5}{L}{A1}
    \gate[\inputs{2}]{or}{5}{3}{R}{}{}

    % ----  2nd column  ----
    \pin{8}{7}{L}{B}
    \gate[\inputs{2}]{and}{12}{5}{R}{}{}

    \pin{8}{9}{L}{C}
    \pin{8}{11}{L}{C1}
    \pin{8}{13}{L}{C2}
    \pin{8}{15}{L}{C3}
    \gate[\inputs{4}]{and}{12}{12}{R}{}{}

    % ----  3rd column  ----
    \pin{15}{16}{L}{D}
    \wire{16}{5}{16}{10}
    \wire{16}{14}{16}{16}
    \gate[\inputs{3}]{nor}{19}{12}{R}{}{}

    % ----  result ----
    \pin{24}{12}{R}{Y}

\end{circuitdiagram}
 %%  ************    LibreSilicon's StdCellLibrary   *******************
%%
%%  Organisation:   Chipforge
%%                  Germany / European Union
%%
%%  Profile:        Chipforge focus on fine System-on-Chip Cores in
%%                  Verilog HDL Code which are easy understandable and
%%                  adjustable. For further information see
%%                          www.chipforge.org
%%                  there are projects from small cores up to PCBs, too.
%%
%%  File:           StdCellLib/Documents/Datasheets/Circuitry/OAAO2141.tex
%%
%%  Purpose:        Circuit File for OAAO2141
%%
%%  ************    LaTeX with circdia.sty package      ***************
%%
%%  ///////////////////////////////////////////////////////////////////
%%
%%  Copyright (c) 2018 - 2022 by
%%                  chipforge <stdcelllib@nospam.chipforge.org>
%%  All rights reserved.
%%
%%      This Standard Cell Library is licensed under the Libre Silicon
%%      public license; you can redistribute it and/or modify it under
%%      the terms of the Libre Silicon public license as published by
%%      the Libre Silicon alliance, either version 1 of the License, or
%%      (at your option) any later version.
%%
%%      This design is distributed in the hope that it will be useful,
%%      but WITHOUT ANY WARRANTY; without even the implied warranty of
%%      MERCHANTABILITY or FITNESS FOR A PARTICULAR PURPOSE.
%%      See the Libre Silicon Public License for more details.
%%
%%  ///////////////////////////////////////////////////////////////////
\begin{circuitdiagram}[draft]{31}{17}

    \usgate
    % ----  1st column  ----
    \pin{1}{1}{L}{A}
    \pin{1}{5}{L}{A1}
    \gate[\inputs{2}]{or}{5}{3}{R}{}{}

    % ----  2nd column  ----
    \pin{8}{7}{L}{B}
    \gate[\inputs{2}]{and}{12}{5}{R}{}{}

    \pin{8}{9}{L}{C}
    \pin{8}{11}{L}{C1}
    \pin{8}{13}{L}{C2}
    \pin{8}{15}{L}{C3}
    \gate[\inputs{4}]{and}{12}{12}{R}{}{}

    % ----  3rd column  ----
    \pin{15}{16}{L}{D}
    \wire{16}{5}{16}{10}
    \wire{16}{14}{16}{16}
    \gate[\inputs{3}]{nor}{19}{12}{R}{}{}

    % ----  4th column  ----
    \gate{not}{26}{12}{R}{}{}

    % ----  result ----
    \pin{30}{12}{R}{Z}

\end{circuitdiagram}

%%  ************    LibreSilicon's StdCellLibrary   *******************
%%
%%  Organisation:   Chipforge
%%                  Germany / European Union
%%
%%  Profile:        Chipforge focus on fine System-on-Chip Cores in
%%                  Verilog HDL Code which are easy understandable and
%%                  adjustable. For further information see
%%                          www.chipforge.org
%%                  there are projects from small cores up to PCBs, too.
%%
%%  File:           StdCellLib/Documents/Circuits/OAAOI2221.tex
%%
%%  Purpose:        Circuit File for OAAOI2221
%%
%%  ************    LaTeX with circdia.sty package      ***************
%%
%%  ///////////////////////////////////////////////////////////////////
%%
%%  Copyright (c) 2019 by chipforge <stdcelllib@nospam.chipforge.org>
%%  All rights reserved.
%%
%%      This Standard Cell Library is licensed under the Libre Silicon
%%      public license; you can redistribute it and/or modify it under
%%      the terms of the Libre Silicon public license as published by
%%      the Libre Silicon alliance, either version 1 of the License, or
%%      (at your option) any later version.
%%
%%      This design is distributed in the hope that it will be useful,
%%      but WITHOUT ANY WARRANTY; without even the implied warranty of
%%      MERCHANTABILITY or FITNESS FOR A PARTICULAR PURPOSE.
%%      See the Libre Silicon Public License for more details.
%%
%%  ///////////////////////////////////////////////////////////////////
\begin{center}
    Circuit
    \begin{figure}[h]
        \begin{center}
            \begin{circuitdiagram}{25}{18}
            \usgate
            \gate[\inputs{2}]{or}{5}{15}{R}{}{}  % OR
            \gate[\inputs{3}]{and}{12}{11}{R}{}{} % AND
            \gate[\inputs{2}]{and}{12}{5}{R}{}{} % AND
            \gate[\inputs{3}]{nor}{19}{5}{R}{}{} % NOR
            \pin{1}{1}{L}{A}     % pin A
            \wire{2}{1}{16}{1}   % wire pin A
            \pin{1}{3}{L}{B}     % pin B
            \wire{2}{3}{9}{3}    % wire pin B
            \pin{1}{7}{L}{B1}    % pin B1
            \wire{2}{7}{9}{7}    % wire pin B1
            \pin{1}{9}{L}{C}     % pin C
            \wire{2}{9}{9}{9}    % wire pin C
            \pin{1}{11}{L}{C1}   % pin C1
            \wire{2}{11}{9}{11}  % wire pin C1
            \pin{1}{13}{L}{D}    % pin D
            \pin{1}{17}{L}{D1}   % pin D1
            \wire{9}{13}{9}{15}  % wire between OR and AND
            \wire{16}{1}{16}{3}  % wire between AND and NOR
            \wire{16}{7}{16}{11} % wire between AND and NOR
            \pin{24}{5}{R}{Y}    % pin Y
            \end{circuitdiagram}
        \end{center}
    \end{figure}
\end{center}
 %%  ************    LibreSilicon's StdCellLibrary   *******************
%%
%%  Organisation:   Chipforge
%%                  Germany / European Union
%%
%%  Profile:        Chipforge focus on fine System-on-Chip Cores in
%%                  Verilog HDL Code which are easy understandable and
%%                  adjustable. For further information see
%%                          www.chipforge.org
%%                  there are projects from small cores up to PCBs, too.
%%
%%  File:           StdCellLib/Documents/Datasheets/Circuitry/OAAO2221.tex
%%
%%  Purpose:        Circuit File for OAAO2221
%%
%%  ************    LaTeX with circdia.sty package      ***************
%%
%%  ///////////////////////////////////////////////////////////////////
%%
%%  Copyright (c) 2018 - 2022 by
%%                  chipforge <stdcelllib@nospam.chipforge.org>
%%  All rights reserved.
%%
%%      This Standard Cell Library is licensed under the Libre Silicon
%%      public license; you can redistribute it and/or modify it under
%%      the terms of the Libre Silicon public license as published by
%%      the Libre Silicon alliance, either version 1 of the License, or
%%      (at your option) any later version.
%%
%%      This design is distributed in the hope that it will be useful,
%%      but WITHOUT ANY WARRANTY; without even the implied warranty of
%%      MERCHANTABILITY or FITNESS FOR A PARTICULAR PURPOSE.
%%      See the Libre Silicon Public License for more details.
%%
%%  ///////////////////////////////////////////////////////////////////
\begin{circuitdiagram}[draft]{31}{18}

    \usgate
    % ----  1st column  ----
    \pin{1}{1}{L}{A}
    \pin{1}{5}{L}{A1}
    \gate[\inputs{2}]{or}{5}{3}{R}{}{}

    % ----  2nd column  ----
    \wire{9}{3}{9}{5}
    \pin{8}{7}{L}{B}
    \pin{8}{9}{L}{B1}
    \gate[\inputs{3}]{and}{12}{7}{R}{}{}

    \pin{8}{11}{L}{C}
    \pin{8}{15}{L}{C1}
    \gate[\inputs{2}]{and}{12}{13}{R}{}{}

    % ----  3rd column  ----
    \wire{16}{7}{16}{11}
    \pin{15}{17}{L}{D}
    \wire{16}{15}{16}{17}
    \gate[\inputs{3}]{nor}{19}{13}{R}{}{}

    % ----  4th column  ----
    \gate{not}{26}{13}{R}{}{}

    % ----  result ----
    \pin{30}{13}{R}{Z}

\end{circuitdiagram}

%%  ************    LibreSilicon's StdCellLibrary   *******************
%%
%%  Organisation:   Chipforge
%%                  Germany / European Union
%%
%%  Profile:        Chipforge focus on fine System-on-Chip Cores in
%%                  Verilog HDL Code which are easy understandable and
%%                  adjustable. For further information see
%%                          www.chipforge.org
%%                  there are projects from small cores up to PCBs, too.
%%
%%  File:           StdCellLib/Documents/Circuits/OAAOI2231.tex
%%
%%  Purpose:        Circuit File for OAAOI2231
%%
%%  ************    LaTeX with circdia.sty package      ***************
%%
%%  ///////////////////////////////////////////////////////////////////
%%
%%  Copyright (c) 2019 by chipforge <stdcelllib@nospam.chipforge.org>
%%  All rights reserved.
%%
%%      This Standard Cell Library is licensed under the Libre Silicon
%%      public license; you can redistribute it and/or modify it under
%%      the terms of the Libre Silicon public license as published by
%%      the Libre Silicon alliance, either version 1 of the License, or
%%      (at your option) any later version.
%%
%%      This design is distributed in the hope that it will be useful,
%%      but WITHOUT ANY WARRANTY; without even the implied warranty of
%%      MERCHANTABILITY or FITNESS FOR A PARTICULAR PURPOSE.
%%      See the Libre Silicon Public License for more details.
%%
%%  ///////////////////////////////////////////////////////////////////
\begin{center}
    Circuit
    \begin{figure}[h]
        \begin{center}
            \begin{circuitdiagram}{25}{18}
            \usgate
            \gate[\inputs{2}]{or}{5}{15}{R}{}{}  % OR
            \gate[\inputs{3}]{and}{12}{11}{R}{}{} % AND
            \gate[\inputs{3}]{and}{12}{5}{R}{}{} % AND
            \gate[\inputs{3}]{nor}{19}{5}{R}{}{} % NOR
            \pin{1}{1}{L}{A}     % pin A
            \wire{2}{1}{16}{1}   % wire pin A
            \pin{1}{3}{L}{B}     % pin B
            \wire{2}{3}{9}{3}    % wire pin B
            \pin{1}{5}{L}{B1}    % pin B1
            \wire{2}{5}{9}{5}    % wire pin B1
            \pin{1}{7}{L}{B2}    % pin B2
            \wire{2}{7}{9}{7}    % wire pin B2
            \pin{1}{9}{L}{C}     % pin C
            \wire{2}{9}{9}{9}    % wire pin C
            \pin{1}{11}{L}{C1}   % pin C1
            \wire{2}{11}{9}{11}  % wire pin C1
            \pin{1}{13}{L}{D}    % pin D
            \pin{1}{17}{L}{D1}   % pin D1
            \wire{9}{13}{9}{15}  % wire between OR and AND
            \wire{16}{1}{16}{3}  % wire between AND and NOR
            \wire{16}{7}{16}{11} % wire between AND and NOR
            \pin{24}{5}{R}{Y}    % pin Y
            \end{circuitdiagram}
        \end{center}
    \end{figure}
\end{center}
 %%  ************    LibreSilicon's StdCellLibrary   *******************
%%
%%  Organisation:   Chipforge
%%                  Germany / European Union
%%
%%  Profile:        Chipforge focus on fine System-on-Chip Cores in
%%                  Verilog HDL Code which are easy understandable and
%%                  adjustable. For further information see
%%                          www.chipforge.org
%%                  there are projects from small cores up to PCBs, too.
%%
%%  File:           StdCellLib/Documents/Datasheets/Circuitry/OAAO2231.tex
%%
%%  Purpose:        Circuit File for OAAO2231
%%
%%  ************    LaTeX with circdia.sty package      ***************
%%
%%  ///////////////////////////////////////////////////////////////////
%%
%%  Copyright (c) 2018 - 2022 by
%%                  chipforge <stdcelllib@nospam.chipforge.org>
%%  All rights reserved.
%%
%%      This Standard Cell Library is licensed under the Libre Silicon
%%      public license; you can redistribute it and/or modify it under
%%      the terms of the Libre Silicon public license as published by
%%      the Libre Silicon alliance, either version 1 of the License, or
%%      (at your option) any later version.
%%
%%      This design is distributed in the hope that it will be useful,
%%      but WITHOUT ANY WARRANTY; without even the implied warranty of
%%      MERCHANTABILITY or FITNESS FOR A PARTICULAR PURPOSE.
%%      See the Libre Silicon Public License for more details.
%%
%%  ///////////////////////////////////////////////////////////////////
\begin{circuitdiagram}[draft]{31}{18}

    \usgate
    % ----  1st column  ----
    \pin{1}{1}{L}{A}
    \pin{1}{5}{L}{A1}
    \gate[\inputs{2}]{or}{5}{3}{R}{}{}

    % ----  2nd column  ----
    \pin{8}{7}{L}{B}
    \pin{8}{9}{L}{B1}
    \wire{9}{3}{9}{5}
    \gate[\inputs{3}]{and}{12}{7}{R}{}{}

    \pin{8}{11}{L}{C}
    \pin{8}{13}{L}{C1}
    \pin{8}{15}{L}{C2}
    \gate[\inputs{3}]{and}{12}{13}{R}{}{}

    % ----  3rd column  ----
    \pin{15}{17}{L}{D}
    \wire{16}{7}{16}{11}
    \wire{16}{15}{16}{17}
    \gate[\inputs{3}]{nor}{19}{13}{R}{}{}

    % ----  4th column  ----
    \gate{not}{26}{13}{R}{}{}

    % ----  result ----
    \pin{30}{13}{R}{Z}

\end{circuitdiagram}

%%  ************    LibreSilicon's StdCellLibrary   *******************
%%
%%  Organisation:   Chipforge
%%                  Germany / European Union
%%
%%  Profile:        Chipforge focus on fine System-on-Chip Cores in
%%                  Verilog HDL Code which are easy understandable and
%%                  adjustable. For further information see
%%                          www.chipforge.org
%%                  there are projects from small cores up to PCBs, too.
%%
%%  File:           StdCellLib/Documents/Datasheets/Circuitry/OAAOI2241.tex
%%
%%  Purpose:        Circuit File for OAAOI2241
%%
%%  ************    LaTeX with circdia.sty package      ***************
%%
%%  ///////////////////////////////////////////////////////////////////
%%
%%  Copyright (c) 2018 - 2022 by
%%                  chipforge <stdcelllib@nospam.chipforge.org>
%%  All rights reserved.
%%
%%      This Standard Cell Library is licensed under the Libre Silicon
%%      public license; you can redistribute it and/or modify it under
%%      the terms of the Libre Silicon public license as published by
%%      the Libre Silicon alliance, either version 1 of the License, or
%%      (at your option) any later version.
%%
%%      This design is distributed in the hope that it will be useful,
%%      but WITHOUT ANY WARRANTY; without even the implied warranty of
%%      MERCHANTABILITY or FITNESS FOR A PARTICULAR PURPOSE.
%%      See the Libre Silicon Public License for more details.
%%
%%  ///////////////////////////////////////////////////////////////////
\begin{circuitdiagram}[draft]{25}{19}

    \usgate
    % ----  1st column  ----
    \pin{1}{1}{L}{A}
    \pin{1}{5}{L}{A1}
    \gate[\inputs{2}]{or}{5}{3}{R}{}{}

    % ----  2nd column  ----
    \pin{8}{7}{L}{B}
    \pin{8}{9}{L}{B1}
    \wire{9}{3}{9}{5}
    \gate[\inputs{3}]{and}{12}{7}{R}{}{}

    \pin{8}{11}{L}{C}
    \pin{8}{13}{L}{C1}
    \pin{8}{15}{L}{C2}
    \pin{8}{17}{L}{C3}
    \gate[\inputs{4}]{and}{12}{14}{R}{}{}

    % ----  3rd column  ----
    \pin{15}{18}{L}{D}
    \wire{16}{7}{16}{12}
    \wire{16}{16}{16}{18}
    \gate[\inputs{3}]{nor}{19}{14}{R}{}{}

    % ----  result ----
    \pin{24}{14}{R}{Y}

\end{circuitdiagram}
 %%  ************    LibreSilicon's StdCellLibrary   *******************
%%
%%  Organisation:   Chipforge
%%                  Germany / European Union
%%
%%  Profile:        Chipforge focus on fine System-on-Chip Cores in
%%                  Verilog HDL Code which are easy understandable and
%%                  adjustable. For further information see
%%                          www.chipforge.org
%%                  there are projects from small cores up to PCBs, too.
%%
%%  File:           StdCellLib/Documents/Datasheets/Circuitry/OAAO2241.tex
%%
%%  Purpose:        Circuit File for OAAO2241
%%
%%  ************    LaTeX with circdia.sty package      ***************
%%
%%  ///////////////////////////////////////////////////////////////////
%%
%%  Copyright (c) 2018 - 2022 by
%%                  chipforge <stdcelllib@nospam.chipforge.org>
%%  All rights reserved.
%%
%%      This Standard Cell Library is licensed under the Libre Silicon
%%      public license; you can redistribute it and/or modify it under
%%      the terms of the Libre Silicon public license as published by
%%      the Libre Silicon alliance, either version 1 of the License, or
%%      (at your option) any later version.
%%
%%      This design is distributed in the hope that it will be useful,
%%      but WITHOUT ANY WARRANTY; without even the implied warranty of
%%      MERCHANTABILITY or FITNESS FOR A PARTICULAR PURPOSE.
%%      See the Libre Silicon Public License for more details.
%%
%%  ///////////////////////////////////////////////////////////////////
\begin{circuitdiagram}[draft]{31}{19}

    \usgate
    % ----  1st column  ----
    \pin{1}{1}{L}{A}
    \pin{1}{5}{L}{A1}
    \gate[\inputs{2}]{or}{5}{3}{R}{}{}

    % ----  2nd column  ----
    \pin{8}{7}{L}{B}
    \pin{8}{9}{L}{B1}
    \wire{9}{3}{9}{5}
    \gate[\inputs{3}]{and}{12}{7}{R}{}{}

    \pin{8}{11}{L}{C}
    \pin{8}{13}{L}{C1}
    \pin{8}{15}{L}{C2}
    \pin{8}{17}{L}{C3}
    \gate[\inputs{4}]{and}{12}{14}{R}{}{}

    % ----  3rd column  ----
    \pin{15}{18}{L}{D}
    \wire{16}{7}{16}{12}
    \wire{16}{16}{16}{18}
    \gate[\inputs{3}]{nor}{19}{14}{R}{}{}

    % ----  last column ----
    \gate{not}{26}{14}{R}{}{}

    % ----  result ----
    \pin{30}{14}{R}{Z}

\end{circuitdiagram}


%%  ************    LibreSilicon's StdCellLibrary   *******************
%%
%%  Organisation:   Chipforge
%%                  Germany / European Union
%%
%%  Profile:        Chipforge focus on fine System-on-Chip Cores in
%%                  Verilog HDL Code which are easy understandable and
%%                  adjustable. For further information see
%%                          www.chipforge.org
%%                  there are projects from small cores up to PCBs, too.
%%
%%  File:           StdCellLib/Documents/section-AAOOAI_complex.tex
%%
%%  Purpose:        Section Level File for Standard Cell Library Documentation
%%
%%  ************    LaTeX with circdia.sty package      ***************
%%
%%  ///////////////////////////////////////////////////////////////////
%%
%%  Copyright (c) 2018 - 2022 by
%%                  chipforge <stdcelllib@nospam.chipforge.org>
%%  All rights reserved.
%%
%%      This Standard Cell Library is licensed under the Libre Silicon
%%      public license; you can redistribute it and/or modify it under
%%      the terms of the Libre Silicon public license as published by
%%      the Libre Silicon alliance, either version 1 of the License, or
%%      (at your option) any later version.
%%
%%      This design is distributed in the hope that it will be useful,
%%      but WITHOUT ANY WARRANTY; without even the implied warranty of
%%      MERCHANTABILITY or FITNESS FOR A PARTICULAR PURPOSE.
%%      See the Libre Silicon Public License for more details.
%%
%%  ///////////////////////////////////////////////////////////////////
\section{AND-AND-OR-OR-AND(-Invert) Complex Gates}

%%  ************    LibreSilicon's StdCellLibrary   *******************
%%
%%  Organisation:   Chipforge
%%                  Germany / European Union
%%
%%  Profile:        Chipforge focus on fine System-on-Chip Cores in
%%                  Verilog HDL Code which are easy understandable and
%%                  adjustable. For further information see
%%                          www.chipforge.org
%%                  there are projects from small cores up to PCBs, too.
%%
%%  File:           StdCellLib/Documents/Datasheets/Circuitry/AAOOAI222.tex
%%
%%  Purpose:        Circuit File for AAOOAI222
%%
%%  ************    LaTeX with circdia.sty package      ***************
%%
%%  ///////////////////////////////////////////////////////////////////
%%
%%  Copyright (c) 2018 - 2022 by
%%                  chipforge <stdcelllib@nospam.chipforge.org>
%%  All rights reserved.
%%
%%      This Standard Cell Library is licensed under the Libre Silicon
%%      public license; you can redistribute it and/or modify it under
%%      the terms of the Libre Silicon public license as published by
%%      the Libre Silicon alliance, either version 1 of the License, or
%%      (at your option) any later version.
%%
%%      This design is distributed in the hope that it will be useful,
%%      but WITHOUT ANY WARRANTY; without even the implied warranty of
%%      MERCHANTABILITY or FITNESS FOR A PARTICULAR PURPOSE.
%%      See the Libre Silicon Public License for more details.
%%
%%  ///////////////////////////////////////////////////////////////////
\begin{circuitdiagram}[draft]{25}{18}

    \usgate
    % ----  1st column  ----
    \pin{1}{1}{L}{A}
    \pin{1}{5}{L}{A1}
    \gate[\inputs{2}]{and}{5}{3}{R}{}{}

    \pin{1}{7}{L}{B}
    \pin{1}{11}{L}{B1}
    \gate[\inputs{2}]{and}{5}{9}{R}{}{}

    % ----  2nd column  ----
    \wire{9}{3}{9}{5}
    \gate[\inputs{2}]{or}{12}{7}{R}{}{}

    \pin{8}{13}{L}{C}
    \pin{8}{17}{L}{C1}
    \gate[\inputs{2}]{or}{12}{15}{R}{}{}

    % ----  3rd column  ----
    \wire{16}{7}{16}{11}
    \gate[\inputs{2}]{nand}{19}{13}{R}{}{}

    % ----  result ----
    \pin{24}{13}{R}{Y}

\end{circuitdiagram}
 %%  ************    LibreSilicon's StdCellLibrary   *******************
%%
%%  Organisation:   Chipforge
%%                  Germany / European Union
%%
%%  Profile:        Chipforge focus on fine System-on-Chip Cores in
%%                  Verilog HDL Code which are easy understandable and
%%                  adjustable. For further information see
%%                          www.chipforge.org
%%                  there are projects from small cores up to PCBs, too.
%%
%%  File:           StdCellLib/Documents/Circuits/AAOOA222.tex
%%
%%  Purpose:        Circuit File for AAOOA222
%%
%%  ************    LaTeX with circdia.sty package      ***************
%%
%%  ///////////////////////////////////////////////////////////////////
%%
%%  Copyright (c) 2019 by chipforge <stdcelllib@nospam.chipforge.org>
%%  All rights reserved.
%%
%%      This Standard Cell Library is licensed under the Libre Silicon
%%      public license; you can redistribute it and/or modify it under
%%      the terms of the Libre Silicon public license as published by
%%      the Libre Silicon alliance, either version 1 of the License, or
%%      (at your option) any later version.
%%
%%      This design is distributed in the hope that it will be useful,
%%      but WITHOUT ANY WARRANTY; without even the implied warranty of
%%      MERCHANTABILITY or FITNESS FOR A PARTICULAR PURPOSE.
%%      See the Libre Silicon Public License for more details.
%%
%%  ///////////////////////////////////////////////////////////////////
\begin{center}
    Circuit
    \begin{figure}[h]
        \begin{center}
            \begin{circuitdiagram}{31}{18}
            \usgate
            \gate[\inputs{2}]{and}{5}{15}{R}{}{}  % OR
            \gate[\inputs{2}]{and}{5}{9}{R}{}{}   % OR
            \gate[\inputs{2}]{or}{12}{12}{R}{}{}  % AND
            \gate[\inputs{2}]{or}{12}{3}{R}{}{}   % AND
            \gate[\inputs{2}]{nand}{19}{8}{R}{}{} % NOR
            \gate{not}{26}{8}{R}{}{} % NOT
            \pin{1}{1}{L}{A}     % pin A
            \pin{1}{5}{L}{A1}    % pin A1
            \pin{1}{7}{L}{B}     % pin B
            \pin{1}{11}{L}{B1}   % pin B1
            \pin{1}{13}{L}{C}    % pin C
            \pin{1}{17}{L}{C1}   % pin C1
            \wire{2}{1}{9}{1}    % wire pin A
            \wire{2}{5}{9}{5}    % wire pin A1
            \wire{9}{14}{9}{15}  % wire between AND and OR
            \wire{9}{9}{9}{10}   % wire between AND and OR
            \wire{16}{3}{16}{6}  % wire between OR and NAND
            \wire{16}{10}{16}{12}  % wire between OR and NAND
            \pin{30}{8}{R}{Z}    % pin Z
            \end{circuitdiagram}
        \end{center}
    \end{figure}
\end{center}

%%  ************    LibreSilicon's StdCellLibrary   *******************
%%
%%  Organisation:   Chipforge
%%                  Germany / European Union
%%
%%  Profile:        Chipforge focus on fine System-on-Chip Cores in
%%                  Verilog HDL Code which are easy understandable and
%%                  adjustable. For further information see
%%                          www.chipforge.org
%%                  there are projects from small cores up to PCBs, too.
%%
%%  File:           StdCellLib/Documents/Datasheets/Circuitry/AAOOAI222.tex
%%
%%  Purpose:        Circuit File for AAOOAI222
%%
%%  ************    LaTeX with circdia.sty package      ***************
%%
%%  ///////////////////////////////////////////////////////////////////
%%
%%  Copyright (c) 2018 - 2022 by
%%                  chipforge <stdcelllib@nospam.chipforge.org>
%%  All rights reserved.
%%
%%      This Standard Cell Library is licensed under the Libre Silicon
%%      public license; you can redistribute it and/or modify it under
%%      the terms of the Libre Silicon public license as published by
%%      the Libre Silicon alliance, either version 1 of the License, or
%%      (at your option) any later version.
%%
%%      This design is distributed in the hope that it will be useful,
%%      but WITHOUT ANY WARRANTY; without even the implied warranty of
%%      MERCHANTABILITY or FITNESS FOR A PARTICULAR PURPOSE.
%%      See the Libre Silicon Public License for more details.
%%
%%  ///////////////////////////////////////////////////////////////////
\begin{circuitdiagram}[draft]{25}{18}

    \usgate
    % ----  1st column  ----
    \pin{1}{1}{L}{A}
    \pin{1}{5}{L}{A1}
    \gate[\inputs{2}]{and}{5}{3}{R}{}{}

    \pin{1}{7}{L}{B}
    \pin{1}{11}{L}{B1}
    \gate[\inputs{2}]{and}{5}{9}{R}{}{}

    % ----  2nd column  ----
    \wire{9}{3}{9}{5}
    \gate[\inputs{2}]{or}{12}{7}{R}{}{}

    \pin{8}{13}{L}{C}
    \pin{8}{15}{L}{C1}
    \pin{8}{17}{L}{C2}
    \gate[\inputs{3}]{or}{12}{15}{R}{}{}

    % ----  3rd column  ----
    \wire{16}{7}{16}{11}
    \gate[\inputs{2}]{nand}{19}{13}{R}{}{}

    % ----  result ----
    \pin{24}{13}{R}{Y}

\end{circuitdiagram}
 %%  ************    LibreSilicon's StdCellLibrary   *******************
%%
%%  Organisation:   Chipforge
%%                  Germany / European Union
%%
%%  Profile:        Chipforge focus on fine System-on-Chip Cores in
%%                  Verilog HDL Code which are easy understandable and
%%                  adjustable. For further information see
%%                          www.chipforge.org
%%                  there are projects from small cores up to PCBs, too.
%%
%%  File:           StdCellLib/Documents/Datasheets/Circuitry/AAOOA222.tex
%%
%%  Purpose:        Circuit File for AAOOA222 
%%
%%  ************    LaTeX with circdia.sty package      ***************
%%
%%  ///////////////////////////////////////////////////////////////////
%%
%%  Copyright (c) 2018 - 2022 by
%%                  chipforge <stdcelllib@nospam.chipforge.org>
%%  All rights reserved.
%%
%%      This Standard Cell Library is licensed under the Libre Silicon
%%      public license; you can redistribute it and/or modify it under
%%      the terms of the Libre Silicon public license as published by
%%      the Libre Silicon alliance, either version 1 of the License, or
%%      (at your option) any later version.
%%
%%      This design is distributed in the hope that it will be useful,
%%      but WITHOUT ANY WARRANTY; without even the implied warranty of
%%      MERCHANTABILITY or FITNESS FOR A PARTICULAR PURPOSE.
%%      See the Libre Silicon Public License for more details.
%%
%%  ///////////////////////////////////////////////////////////////////
\begin{circuitdiagram}[draft]{31}{18}

    \usgate
    % ----  1st column  ----
    \pin{1}{1}{L}{A}
    \pin{1}{5}{L}{A1}
    \gate[\inputs{2}]{and}{5}{3}{R}{}{}

    \pin{1}{7}{L}{B}
    \pin{1}{11}{L}{B1}
    \gate[\inputs{2}]{and}{5}{9}{R}{}{}

    % ----  2nd column  ----
    \wire{9}{3}{9}{5}
    \gate[\inputs{2}]{or}{12}{7}{R}{}{}

    \pin{8}{13}{L}{C}
    \pin{8}{15}{L}{C1}
    \pin{8}{17}{L}{C2}
    \gate[\inputs{3}]{or}{12}{15}{R}{}{}

    % ----  3rd column  ----
    \wire{16}{7}{16}{11}
    \gate[\inputs{2}]{nand}{19}{13}{R}{}{}

    % ----  4th column  ----
    \gate{not}{26}{13}{R}{}{}

    % ----  result ----
    \pin{30}{13}{R}{Z}

\end{circuitdiagram}

%%  ************    LibreSilicon's StdCellLibrary   *******************
%%
%%  Organisation:   Chipforge
%%                  Germany / European Union
%%
%%  Profile:        Chipforge focus on fine System-on-Chip Cores in
%%                  Verilog HDL Code which are easy understandable and
%%                  adjustable. For further information see
%%                          www.chipforge.org
%%                  there are projects from small cores up to PCBs, too.
%%
%%  File:           StdCellLib/Documents/Datasheets/Circuitry/AAOOAI224.tex
%%
%%  Purpose:        Circuit File for AAOOAI224
%%
%%  ************    LaTeX with circdia.sty package      ***************
%%
%%  ///////////////////////////////////////////////////////////////////
%%
%%  Copyright (c) 2018 - 2022 by
%%                  chipforge <stdcelllib@nospam.chipforge.org>
%%  All rights reserved.
%%
%%      This Standard Cell Library is licensed under the Libre Silicon
%%      public license; you can redistribute it and/or modify it under
%%      the terms of the Libre Silicon public license as published by
%%      the Libre Silicon alliance, either version 1 of the License, or
%%      (at your option) any later version.
%%
%%      This design is distributed in the hope that it will be useful,
%%      but WITHOUT ANY WARRANTY; without even the implied warranty of
%%      MERCHANTABILITY or FITNESS FOR A PARTICULAR PURPOSE.
%%      See the Libre Silicon Public License for more details.
%%
%%  ///////////////////////////////////////////////////////////////////
\begin{circuitdiagram}[draft]{25}{20}

    \usgate
    % ----  1st column  ----
    \pin{1}{1}{L}{A}
    \pin{1}{5}{L}{A1}
    \gate[\inputs{2}]{and}{5}{3}{R}{}{}

    \pin{1}{7}{L}{B}
    \pin{1}{11}{L}{B1}
    \gate[\inputs{2}]{and}{5}{9}{R}{}{}

    % ----  2nd column  ----
    \wire{9}{3}{9}{5}
    \gate[\inputs{2}]{or}{12}{7}{R}{}{}

    \pin{8}{13}{L}{C}
    \pin{8}{15}{L}{C1}
    \pin{8}{17}{L}{C2}
    \pin{8}{19}{L}{C3}
    \gate[\inputs{4}]{or}{12}{16}{R}{}{}

    % ----  3rd column  ----
    \wire{16}{7}{16}{12}
    \gate[\inputs{2}]{nand}{19}{14}{R}{}{}

    % ----  result ----
    \pin{24}{14}{R}{Y}

\end{circuitdiagram}
 %%  ************    LibreSilicon's StdCellLibrary   *******************
%%
%%  Organisation:   Chipforge
%%                  Germany / European Union
%%
%%  Profile:        Chipforge focus on fine System-on-Chip Cores in
%%                  Verilog HDL Code which are easy understandable and
%%                  adjustable. For further information see
%%                          www.chipforge.org
%%                  there are projects from small cores up to PCBs, too.
%%
%%  File:           StdCellLib/Documents/Datasheets/Circuitry/AAOOA224.tex
%%
%%  Purpose:        Circuit File for AAOOA224
%%
%%  ************    LaTeX with circdia.sty package      ***************
%%
%%  ///////////////////////////////////////////////////////////////////
%%
%%  Copyright (c) 2018 - 2022 by
%%                  chipforge <stdcelllib@nospam.chipforge.org>
%%  All rights reserved.
%%
%%      This Standard Cell Library is licensed under the Libre Silicon
%%      public license; you can redistribute it and/or modify it under
%%      the terms of the Libre Silicon public license as published by
%%      the Libre Silicon alliance, either version 1 of the License, or
%%      (at your option) any later version.
%%
%%      This design is distributed in the hope that it will be useful,
%%      but WITHOUT ANY WARRANTY; without even the implied warranty of
%%      MERCHANTABILITY or FITNESS FOR A PARTICULAR PURPOSE.
%%      See the Libre Silicon Public License for more details.
%%
%%  ///////////////////////////////////////////////////////////////////
\begin{circuitdiagram}[draft]{31}{20}

    \usgate
    % ----  1st column  ----
    \pin{1}{1}{L}{A}
    \pin{1}{5}{L}{A1}
    \gate[\inputs{2}]{and}{5}{3}{R}{}{}

    \pin{1}{7}{L}{B}
    \pin{1}{11}{L}{B1}
    \gate[\inputs{2}]{and}{5}{9}{R}{}{}

    % ----  2nd column  ----
    \wire{9}{3}{9}{5}
    \gate[\inputs{2}]{or}{12}{7}{R}{}{}

    \pin{8}{13}{L}{C}
    \pin{8}{15}{L}{C1}
    \pin{8}{17}{L}{C2}
    \pin{8}{19}{L}{C3}
    \gate[\inputs{4}]{or}{12}{16}{R}{}{}

    % ----  3rd column  ----
    \wire{16}{7}{16}{12}
    \gate[\inputs{2}]{nand}{19}{14}{R}{}{}

    % ----  4th column  ----
    \gate{not}{26}{14}{R}{}{}

    % ----  result ----
    \pin{30}{14}{R}{Z}

\end{circuitdiagram}

%%  ************    LibreSilicon's StdCellLibrary   *******************
%%
%%  Organisation:   Chipforge
%%                  Germany / European Union
%%
%%  Profile:        Chipforge focus on fine System-on-Chip Cores in
%%                  Verilog HDL Code which are easy understandable and
%%                  adjustable. For further information see
%%                          www.chipforge.org
%%                  there are projects from small cores up to PCBs, too.
%%
%%  File:           StdCellLib/Documents/Datasheets/Circuitry/AAOOAI322.tex
%%
%%  Purpose:        Circuit File for AAOOAI322
%%
%%  ************    LaTeX with circdia.sty package      ***************
%%
%%  ///////////////////////////////////////////////////////////////////
%%
%%  Copyright (c) 2018 - 2022 by
%%                  chipforge <stdcelllib@nospam.chipforge.org>
%%  All rights reserved.
%%
%%      This Standard Cell Library is licensed under the Libre Silicon
%%      public license; you can redistribute it and/or modify it under
%%      the terms of the Libre Silicon public license as published by
%%      the Libre Silicon alliance, either version 1 of the License, or
%%      (at your option) any later version.
%%
%%      This design is distributed in the hope that it will be useful,
%%      but WITHOUT ANY WARRANTY; without even the implied warranty of
%%      MERCHANTABILITY or FITNESS FOR A PARTICULAR PURPOSE.
%%      See the Libre Silicon Public License for more details.
%%
%%  ///////////////////////////////////////////////////////////////////
\begin{circuitdiagram}[draft]{25}{18}

    \usgate
    % ----  1st column  ----
    \pin{1}{1}{L}{A}
    \pin{1}{3}{L}{A1}
    \pin{1}{5}{L}{A2}
    \gate[\inputs{3}]{and}{5}{3}{R}{}{}

    \pin{1}{7}{L}{B}
    \pin{1}{11}{L}{B1}
    \gate[\inputs{2}]{and}{5}{9}{R}{}{}

    % ----  2nd column  ----
    \wire{9}{3}{9}{5}
    \gate[\inputs{2}]{or}{12}{7}{R}{}{}

    \pin{8}{13}{L}{C}
    \pin{8}{17}{L}{C1}
    \gate[\inputs{2}]{or}{12}{15}{R}{}{}

    % ----  3rd column  ----
    \wire{16}{7}{16}{11}
    \gate[\inputs{2}]{nand}{19}{13}{R}{}{}

    % ----  result ----
    \pin{24}{13}{R}{Y}

\end{circuitdiagram}
 %%  ************    LibreSilicon's StdCellLibrary   *******************
%%
%%  Organisation:   Chipforge
%%                  Germany / European Union
%%
%%  Profile:        Chipforge focus on fine System-on-Chip Cores in
%%                  Verilog HDL Code which are easy understandable and
%%                  adjustable. For further information see
%%                          www.chipforge.org
%%                  there are projects from small cores up to PCBs, too.
%%
%%  File:           StdCellLib/Documents/Datasheets/Circuitry/AAOOA322.tex
%%
%%  Purpose:        Circuit File for AAOOA322
%%
%%  ************    LaTeX with circdia.sty package      ***************
%%
%%  ///////////////////////////////////////////////////////////////////
%%
%%  Copyright (c) 2018 - 2022 by
%%                  chipforge <stdcelllib@nospam.chipforge.org>
%%  All rights reserved.
%%
%%      This Standard Cell Library is licensed under the Libre Silicon
%%      public license; you can redistribute it and/or modify it under
%%      the terms of the Libre Silicon public license as published by
%%      the Libre Silicon alliance, either version 1 of the License, or
%%      (at your option) any later version.
%%
%%      This design is distributed in the hope that it will be useful,
%%      but WITHOUT ANY WARRANTY; without even the implied warranty of
%%      MERCHANTABILITY or FITNESS FOR A PARTICULAR PURPOSE.
%%      See the Libre Silicon Public License for more details.
%%
%%  ///////////////////////////////////////////////////////////////////
\begin{circuitdiagram}[draft]{31}{18}

    \usgate
    % ----  1st column  ----
    \pin{1}{1}{L}{A}
    \pin{1}{3}{L}{A1}
    \pin{1}{5}{L}{A2}
    \gate[\inputs{3}]{and}{5}{3}{R}{}{}

    \pin{1}{7}{L}{B}
    \pin{1}{11}{L}{B1}
    \gate[\inputs{2}]{and}{5}{9}{R}{}{}

    % ----  2nd column  ----
    \wire{9}{3}{9}{5}
    \gate[\inputs{2}]{or}{12}{7}{R}{}{}

    \pin{8}{13}{L}{C}
    \pin{8}{17}{L}{C1}
    \gate[\inputs{2}]{or}{12}{15}{R}{}{}

    % ----  3rd column  ----
    \wire{16}{7}{16}{11}
    \gate[\inputs{2}]{nand}{19}{13}{R}{}{}

    % ----  4th column  ----
    \gate{not}{26}{13}{R}{}{}

    % ----  result ----
    \pin{30}{13}{R}{Z}

\end{circuitdiagram}

%%  ************    LibreSilicon's StdCellLibrary   *******************
%%
%%  Organisation:   Chipforge
%%                  Germany / European Union
%%
%%  Profile:        Chipforge focus on fine System-on-Chip Cores in
%%                  Verilog HDL Code which are easy understandable and
%%                  adjustable. For further information see
%%                          www.chipforge.org
%%                  there are projects from small cores up to PCBs, too.
%%
%%  File:           StdCellLib/Documents/Datasheets/Circuitry/AAOOAI323.tex
%%
%%  Purpose:        Circuit File for AAOOAI323
%%
%%  ************    LaTeX with circdia.sty package      ***************
%%
%%  ///////////////////////////////////////////////////////////////////
%%
%%  Copyright (c) 2018 - 2022 by
%%                  chipforge <stdcelllib@nospam.chipforge.org>
%%  All rights reserved.
%%
%%      This Standard Cell Library is licensed under the Libre Silicon
%%      public license; you can redistribute it and/or modify it under
%%      the terms of the Libre Silicon public license as published by
%%      the Libre Silicon alliance, either version 1 of the License, or
%%      (at your option) any later version.
%%
%%      This design is distributed in the hope that it will be useful,
%%      but WITHOUT ANY WARRANTY; without even the implied warranty of
%%      MERCHANTABILITY or FITNESS FOR A PARTICULAR PURPOSE.
%%      See the Libre Silicon Public License for more details.
%%
%%  ///////////////////////////////////////////////////////////////////
\begin{circuitdiagram}[draft]{25}{18}

    \usgate
    % ----  1st column  ----
    \pin{1}{1}{L}{A}
    \pin{1}{3}{L}{A1}
    \pin{1}{5}{L}{A2}
    \gate[\inputs{3}]{and}{5}{3}{R}{}{}

    \pin{1}{7}{L}{B}
    \pin{1}{11}{L}{B1}
    \gate[\inputs{2}]{and}{5}{9}{R}{}{}

    % ----  2nd column  ----
    \wire{9}{3}{9}{5}
    \gate[\inputs{2}]{or}{12}{7}{R}{}{}

    \pin{8}{13}{L}{C}
    \pin{8}{15}{L}{C1}
    \pin{8}{17}{L}{C2}
    \gate[\inputs{3}]{or}{12}{15}{R}{}{}

    % ----  3rd column  ----
    \wire{16}{7}{16}{11}
    \gate[\inputs{2}]{nand}{19}{13}{R}{}{}

    % ----  result ----
    \pin{24}{13}{R}{Y}

\end{circuitdiagram}
 %%  ************    LibreSilicon's StdCellLibrary   *******************
%%
%%  Organisation:   Chipforge
%%                  Germany / European Union
%%
%%  Profile:        Chipforge focus on fine System-on-Chip Cores in
%%                  Verilog HDL Code which are easy understandable and
%%                  adjustable. For further information see
%%                          www.chipforge.org
%%                  there are projects from small cores up to PCBs, too.
%%
%%  File:           StdCellLib/Documents/Datasheets/Circuitry/AAOOA323.tex
%%
%%  Purpose:        Circuit File for AAOOA323
%%
%%  ************    LaTeX with circdia.sty package      ***************
%%
%%  ///////////////////////////////////////////////////////////////////
%%
%%  Copyright (c) 2018 - 2022 by
%%                  chipforge <stdcelllib@nospam.chipforge.org>
%%  All rights reserved.
%%
%%      This Standard Cell Library is licensed under the Libre Silicon
%%      public license; you can redistribute it and/or modify it under
%%      the terms of the Libre Silicon public license as published by
%%      the Libre Silicon alliance, either version 1 of the License, or
%%      (at your option) any later version.
%%
%%      This design is distributed in the hope that it will be useful,
%%      but WITHOUT ANY WARRANTY; without even the implied warranty of
%%      MERCHANTABILITY or FITNESS FOR A PARTICULAR PURPOSE.
%%      See the Libre Silicon Public License for more details.
%%
%%  ///////////////////////////////////////////////////////////////////
\begin{circuitdiagram}[draft]{31}{18}

    \usgate
    % ----  1st column  ----
    \pin{1}{1}{L}{A}
    \pin{1}{3}{L}{A1}
    \pin{1}{5}{L}{A2}
    \gate[\inputs{3}]{and}{5}{3}{R}{}{}

    \pin{1}{7}{L}{B}
    \pin{1}{11}{L}{B1}
    \gate[\inputs{2}]{and}{5}{9}{R}{}{}

    % ----  2nd column  ----
    \wire{9}{3}{9}{5}
    \gate[\inputs{2}]{or}{12}{7}{R}{}{}

    \pin{8}{13}{L}{C}
    \pin{8}{15}{L}{C1}
    \pin{8}{17}{L}{C2}
    \gate[\inputs{3}]{or}{12}{15}{R}{}{}

    % ----  3rd column  ----
    \wire{16}{7}{16}{11}
    \gate[\inputs{2}]{nand}{19}{13}{R}{}{}

    % ----  last column ----
    \gate{not}{26}{13}{R}{}{}

    % ----  result ----
    \pin{30}{13}{R}{Z}

\end{circuitdiagram}

%%  ************    LibreSilicon's StdCellLibrary   *******************
%%
%%  Organisation:   Chipforge
%%                  Germany / European Union
%%
%%  Profile:        Chipforge focus on fine System-on-Chip Cores in
%%                  Verilog HDL Code which are easy understandable and
%%                  adjustable. For further information see
%%                          www.chipforge.org
%%                  there are projects from small cores up to PCBs, too.
%%
%%  File:           StdCellLib/Documents/Datasheets/Circuitry/AAOOAI332.tex
%%
%%  Purpose:        Circuit File for AAOOAI332
%%
%%  ************    LaTeX with circdia.sty package      ***************
%%
%%  ///////////////////////////////////////////////////////////////////
%%
%%  Copyright (c) 2018 - 2022 by
%%                  chipforge <stdcelllib@nospam.chipforge.org>
%%  All rights reserved.
%%
%%      This Standard Cell Library is licensed under the Libre Silicon
%%      public license; you can redistribute it and/or modify it under
%%      the terms of the Libre Silicon public license as published by
%%      the Libre Silicon alliance, either version 1 of the License, or
%%      (at your option) any later version.
%%
%%      This design is distributed in the hope that it will be useful,
%%      but WITHOUT ANY WARRANTY; without even the implied warranty of
%%      MERCHANTABILITY or FITNESS FOR A PARTICULAR PURPOSE.
%%      See the Libre Silicon Public License for more details.
%%
%%  ///////////////////////////////////////////////////////////////////
\begin{circuitdiagram}[draft]{25}{18}

    \usgate
    % ----  1st column  ----
    \pin{1}{1}{L}{A}
    \pin{1}{3}{L}{A1}
    \pin{1}{5}{L}{A2}
    \gate[\inputs{3}]{and}{5}{3}{R}{}{}

    \pin{1}{7}{L}{B}
    \pin{1}{9}{L}{B1}
    \pin{1}{11}{L}{B2}
    \gate[\inputs{3}]{and}{5}{9}{R}{}{}

    % ----  2nd column  ----
    \wire{9}{3}{9}{5}
    \gate[\inputs{2}]{or}{12}{7}{R}{}{}

    \pin{8}{13}{L}{C}
    \pin{8}{17}{L}{C1}
    \gate[\inputs{2}]{or}{12}{15}{R}{}{}

    % ----  3rd column  ----
    \wire{16}{7}{16}{11}
    \gate[\inputs{2}]{nand}{19}{13}{R}{}{}

    % ----  result ----
    \pin{24}{13}{R}{Y}

\end{circuitdiagram}
 %%  ************    LibreSilicon's StdCellLibrary   *******************
%%
%%  Organisation:   Chipforge
%%                  Germany / European Union
%%
%%  Profile:        Chipforge focus on fine System-on-Chip Cores in
%%                  Verilog HDL Code which are easy understandable and
%%                  adjustable. For further information see
%%                          www.chipforge.org
%%                  there are projects from small cores up to PCBs, too.
%%
%%  File:           StdCellLib/Documents/Datasheets/Circuitry/AAOOA332.tex
%%
%%  Purpose:        Circuit File for AAOOA332
%%
%%  ************    LaTeX with circdia.sty package      ***************
%%
%%  ///////////////////////////////////////////////////////////////////
%%
%%  Copyright (c) 2018 - 2022 by
%%                  chipforge <stdcelllib@nospam.chipforge.org>
%%  All rights reserved.
%%
%%      This Standard Cell Library is licensed under the Libre Silicon
%%      public license; you can redistribute it and/or modify it under
%%      the terms of the Libre Silicon public license as published by
%%      the Libre Silicon alliance, either version 1 of the License, or
%%      (at your option) any later version.
%%
%%      This design is distributed in the hope that it will be useful,
%%      but WITHOUT ANY WARRANTY; without even the implied warranty of
%%      MERCHANTABILITY or FITNESS FOR A PARTICULAR PURPOSE.
%%      See the Libre Silicon Public License for more details.
%%
%%  ///////////////////////////////////////////////////////////////////
\begin{circuitdiagram}[draft]{31}{18}

    \usgate
    % ----  1st column  ----
    \pin{1}{1}{L}{A}
    \pin{1}{3}{L}{A1}
    \pin{1}{5}{L}{A2}
    \gate[\inputs{3}]{and}{5}{3}{R}{}{}

    \pin{1}{7}{L}{B}
    \pin{1}{9}{L}{B1}
    \pin{1}{11}{L}{B2}
    \gate[\inputs{3}]{and}{5}{9}{R}{}{}

    % ----  2nd column  ----
    \wire{9}{3}{9}{5}
    \gate[\inputs{2}]{or}{12}{7}{R}{}{}

    \pin{8}{13}{L}{C}
    \pin{8}{17}{L}{C1}
    \gate[\inputs{2}]{or}{12}{15}{R}{}{}

    % ----  3rd column  ----
    \wire{16}{7}{16}{11}
    \gate[\inputs{2}]{nand}{19}{13}{R}{}{}

    % ----  last column ----
    \gate{not}{26}{13}{R}{}{}

    % ----  result ----
    \pin{30}{13}{R}{Z}

\end{circuitdiagram}


%%  ************    LibreSilicon's StdCellLibrary   *******************
%%
%%  Organisation:   Chipforge
%%                  Germany / European Union
%%
%%  Profile:        Chipforge focus on fine System-on-Chip Cores in
%%                  Verilog HDL Code which are easy understandable and
%%                  adjustable. For further information see
%%                          www.chipforge.org
%%                  there are projects from small cores up to PCBs, too.
%%
%%  File:           StdCellLib/Documents/Datasheets/Circuitry/AAOOAI2212.tex
%%
%%  Purpose:        Circuit File for AAOOAI2212
%%
%%  ************    LaTeX with circdia.sty package      ***************
%%
%%  ///////////////////////////////////////////////////////////////////
%%
%%  Copyright (c) 2018 - 2022 by
%%                  chipforge <stdcelllib@nospam.chipforge.org>
%%  All rights reserved.
%%
%%      This Standard Cell Library is licensed under the Libre Silicon
%%      public license; you can redistribute it and/or modify it under
%%      the terms of the Libre Silicon public license as published by
%%      the Libre Silicon alliance, either version 1 of the License, or
%%      (at your option) any later version.
%%
%%      This design is distributed in the hope that it will be useful,
%%      but WITHOUT ANY WARRANTY; without even the implied warranty of
%%      MERCHANTABILITY or FITNESS FOR A PARTICULAR PURPOSE.
%%      See the Libre Silicon Public License for more details.
%%
%%  ///////////////////////////////////////////////////////////////////
\begin{circuitdiagram}[draft]{25}{20}

    \usgate
    % ----  1st column  ----
    \pin{1}{1}{L}{A}
    \pin{1}{5}{L}{A1}
    \gate[\inputs{2}]{and}{5}{3}{R}{}{}

    \pin{1}{7}{L}{B}
    \pin{1}{11}{L}{B1}
    \gate[\inputs{2}]{and}{5}{9}{R}{}{}

    % ----  2nd column  ----
    \pin{8}{13}{L}{C}
    \wire{9}{3}{9}{7}
    \wire{9}{11}{9}{13}
    \gate[\inputs{3}]{or}{12}{9}{R}{}{}

    \pin{8}{15}{L}{D}
    \pin{8}{19}{L}{D1}
    \gate[\inputs{2}]{or}{12}{17}{R}{}{}

    % ----  3rd column  ----
    \wire{16}{9}{16}{13}
    \gate[\inputs{2}]{nand}{19}{15}{R}{}{}

    % ----  result ----
    \pin{24}{15}{R}{Y}

\end{circuitdiagram}
 %%  ************    LibreSilicon's StdCellLibrary   *******************
%%
%%  Organisation:   Chipforge
%%                  Germany / European Union
%%
%%  Profile:        Chipforge focus on fine System-on-Chip Cores in
%%                  Verilog HDL Code which are easy understandable and
%%                  adjustable. For further information see
%%                          www.chipforge.org
%%                  there are projects from small cores up to PCBs, too.
%%
%%  File:           StdCellLib/Documents/Datasheets/Circuitry/AAOOA2212.tex
%%
%%  Purpose:        Circuit File for AAOOA2212 
%%
%%  ************    LaTeX with circdia.sty package      ***************
%%
%%  ///////////////////////////////////////////////////////////////////
%%
%%  Copyright (c) 2018 - 2022 by
%%                  chipforge <stdcelllib@nospam.chipforge.org>
%%  All rights reserved.
%%
%%      This Standard Cell Library is licensed under the Libre Silicon
%%      public license; you can redistribute it and/or modify it under
%%      the terms of the Libre Silicon public license as published by
%%      the Libre Silicon alliance, either version 1 of the License, or
%%      (at your option) any later version.
%%
%%      This design is distributed in the hope that it will be useful,
%%      but WITHOUT ANY WARRANTY; without even the implied warranty of
%%      MERCHANTABILITY or FITNESS FOR A PARTICULAR PURPOSE.
%%      See the Libre Silicon Public License for more details.
%%
%%  ///////////////////////////////////////////////////////////////////
\begin{circuitdiagram}[draft]{31}{20}

    \usgate
    % ----  1st column  ----
    \pin{1}{1}{L}{A}
    \pin{1}{5}{L}{A1}
    \gate[\inputs{2}]{and}{5}{3}{R}{}{}

    \pin{1}{7}{L}{B}
    \pin{1}{11}{L}{B1}
    \gate[\inputs{2}]{and}{5}{9}{R}{}{}

    % ----  2nd column  ----
    \pin{8}{13}{L}{C}
    \wire{9}{3}{9}{7}
    \wire{9}{11}{9}{13}
    \gate[\inputs{3}]{or}{12}{9}{R}{}{}

    \pin{8}{15}{L}{D}
    \pin{8}{19}{L}{D1}
    \gate[\inputs{2}]{or}{12}{17}{R}{}{}

    % ----  3rd column  ----
    \wire{16}{9}{16}{13}
    \gate[\inputs{2}]{nand}{19}{15}{R}{}{}

    % ----  4th column  ----
    \gate{not}{26}{15}{R}{}{}

    % ----  result ----
    \pin{30}{15}{R}{Z}

\end{circuitdiagram}

%%  ************    LibreSilicon's StdCellLibrary   *******************
%%
%%  Organisation:   Chipforge
%%                  Germany / European Union
%%
%%  Profile:        Chipforge focus on fine System-on-Chip Cores in
%%                  Verilog HDL Code which are easy understandable and
%%                  adjustable. For further information see
%%                          www.chipforge.org
%%                  there are projects from small cores up to PCBs, too.
%%
%%  File:           StdCellLib/Documents/Datasheets/Circuitry/AAOOAI2213.tex
%%
%%  Purpose:        Circuit File for AAOOAI2213
%%
%%  ************    LaTeX with circdia.sty package      ***************
%%
%%  ///////////////////////////////////////////////////////////////////
%%
%%  Copyright (c) 2018 - 2022 by
%%                  chipforge <stdcelllib@nospam.chipforge.org>
%%  All rights reserved.
%%
%%      This Standard Cell Library is licensed under the Libre Silicon
%%      public license; you can redistribute it and/or modify it under
%%      the terms of the Libre Silicon public license as published by
%%      the Libre Silicon alliance, either version 1 of the License, or
%%      (at your option) any later version.
%%
%%      This design is distributed in the hope that it will be useful,
%%      but WITHOUT ANY WARRANTY; without even the implied warranty of
%%      MERCHANTABILITY or FITNESS FOR A PARTICULAR PURPOSE.
%%      See the Libre Silicon Public License for more details.
%%
%%  ///////////////////////////////////////////////////////////////////
\begin{circuitdiagram}[draft]{25}{20}

    \usgate
    % ----  1st column  ----
    \pin{1}{1}{L}{A}
    \pin{1}{5}{L}{A1}
    \gate[\inputs{2}]{and}{5}{3}{R}{}{}

    \pin{1}{7}{L}{B}
    \pin{1}{11}{L}{B1}
    \gate[\inputs{2}]{and}{5}{9}{R}{}{}

    % ----  2nd column  ----
    \pin{8}{13}{L}{C}
    \wire{9}{3}{9}{7}
    \wire{9}{11}{9}{13}
    \gate[\inputs{3}]{or}{12}{9}{R}{}{}

    \pin{8}{15}{L}{D}
    \pin{8}{17}{L}{D1}
    \pin{8}{19}{L}{D2}
    \gate[\inputs{3}]{or}{12}{17}{R}{}{}

    % ----  3rd column  ----
    \wire{16}{9}{16}{13}
    \gate[\inputs{2}]{nand}{19}{15}{R}{}{}

    % ----  result ----
    \pin{24}{15}{R}{Y}

\end{circuitdiagram}
 %%  ************    LibreSilicon's StdCellLibrary   *******************
%%
%%  Organisation:   Chipforge
%%                  Germany / European Union
%%
%%  Profile:        Chipforge focus on fine System-on-Chip Cores in
%%                  Verilog HDL Code which are easy understandable and
%%                  adjustable. For further information see
%%                          www.chipforge.org
%%                  there are projects from small cores up to PCBs, too.
%%
%%  File:           StdCellLib/Documents/Datasheets/Circuitry/AAOOA2213.tex
%%
%%  Purpose:        Circuit File for AAOOA2213
%%
%%  ************    LaTeX with circdia.sty package      ***************
%%
%%  ///////////////////////////////////////////////////////////////////
%%
%%  Copyright (c) 2018 - 2022 by
%%                  chipforge <stdcelllib@nospam.chipforge.org>
%%  All rights reserved.
%%
%%      This Standard Cell Library is licensed under the Libre Silicon
%%      public license; you can redistribute it and/or modify it under
%%      the terms of the Libre Silicon public license as published by
%%      the Libre Silicon alliance, either version 1 of the License, or
%%      (at your option) any later version.
%%
%%      This design is distributed in the hope that it will be useful,
%%      but WITHOUT ANY WARRANTY; without even the implied warranty of
%%      MERCHANTABILITY or FITNESS FOR A PARTICULAR PURPOSE.
%%      See the Libre Silicon Public License for more details.
%%
%%  ///////////////////////////////////////////////////////////////////
\begin{circuitdiagram}[draft]{31}{20}

    \usgate
    % ----  1st column  ----
    \pin{1}{1}{L}{A}
    \pin{1}{5}{L}{A1}
    \gate[\inputs{2}]{and}{5}{3}{R}{}{}

    \pin{1}{7}{L}{B}
    \pin{1}{11}{L}{B1}
    \gate[\inputs{2}]{and}{5}{9}{R}{}{}

    % ----  2nd column  ----
    \pin{8}{13}{L}{C}
    \wire{9}{3}{9}{7}
    \wire{9}{11}{9}{13}
    \gate[\inputs{3}]{or}{12}{9}{R}{}{}

    \pin{8}{15}{L}{D}
    \pin{8}{17}{L}{D1}
    \pin{8}{19}{L}{D2}
    \gate[\inputs{3}]{or}{12}{17}{R}{}{}

    % ----  3rd column  ----
    \wire{16}{9}{16}{13}
    \gate[\inputs{2}]{nand}{19}{15}{R}{}{}

    % ----  4th column  ----
    \gate{not}{26}{15}{R}{}{}

    % ----  result ----
    \pin{30}{15}{R}{Z}

\end{circuitdiagram}

%%  ************    LibreSilicon's StdCellLibrary   *******************
%%
%%  Organisation:   Chipforge
%%                  Germany / European Union
%%
%%  Profile:        Chipforge focus on fine System-on-Chip Cores in
%%                  Verilog HDL Code which are easy understandable and
%%                  adjustable. For further information see
%%                          www.chipforge.org
%%                  there are projects from small cores up to PCBs, too.
%%
%%  File:           StdCellLib/Documents/Datasheets/Circuitry/AAOOAI2214.tex
%%
%%  Purpose:        Circuit File for AAOOAI2214
%%
%%  ************    LaTeX with circdia.sty package      ***************
%%
%%  ///////////////////////////////////////////////////////////////////
%%
%%  Copyright (c) 2018 - 2022 by
%%                  chipforge <stdcelllib@nospam.chipforge.org>
%%  All rights reserved.
%%
%%      This Standard Cell Library is licensed under the Libre Silicon
%%      public license; you can redistribute it and/or modify it under
%%      the terms of the Libre Silicon public license as published by
%%      the Libre Silicon alliance, either version 1 of the License, or
%%      (at your option) any later version.
%%
%%      This design is distributed in the hope that it will be useful,
%%      but WITHOUT ANY WARRANTY; without even the implied warranty of
%%      MERCHANTABILITY or FITNESS FOR A PARTICULAR PURPOSE.
%%      See the Libre Silicon Public License for more details.
%%
%%  ///////////////////////////////////////////////////////////////////
\begin{circuitdiagram}[draft]{25}{22}

    \usgate
    % ----  1st column  ----
    \pin{1}{1}{L}{A}
    \pin{1}{5}{L}{A1}
    \gate[\inputs{2}]{and}{5}{3}{R}{}{}

    \pin{1}{7}{L}{B}
    \pin{1}{11}{L}{B1}
    \gate[\inputs{2}]{and}{5}{9}{R}{}{}

    % ----  2nd column  ----
    \pin{8}{13}{L}{C}
    \wire{9}{3}{9}{7}
    \wire{9}{11}{9}{13}
    \gate[\inputs{3}]{or}{12}{9}{R}{}{}

    \pin{8}{15}{L}{D}
    \pin{8}{17}{L}{D1}
    \pin{8}{19}{L}{D2}
    \pin{8}{21}{L}{D3}
    \gate[\inputs{4}]{or}{12}{18}{R}{}{}

    % ----  3rd column  ----
    \wire{16}{9}{16}{14}
    \gate[\inputs{2}]{nand}{19}{16}{R}{}{}

    % ----  result ----
    \pin{24}{16}{R}{Y}

\end{circuitdiagram}
 %%  ************    LibreSilicon's StdCellLibrary   *******************
%%
%%  Organisation:   Chipforge
%%                  Germany / European Union
%%
%%  Profile:        Chipforge focus on fine System-on-Chip Cores in
%%                  Verilog HDL Code which are easy understandable and
%%                  adjustable. For further information see
%%                          www.chipforge.org
%%                  there are projects from small cores up to PCBs, too.
%%
%%  File:           StdCellLib/Documents/Datasheets/Circuitry/AAOOA2214.tex
%%
%%  Purpose:        Circuit File for AAOOA2214
%%
%%  ************    LaTeX with circdia.sty package      ***************
%%
%%  ///////////////////////////////////////////////////////////////////
%%
%%  Copyright (c) 2018 - 2022 by
%%                  chipforge <stdcelllib@nospam.chipforge.org>
%%  All rights reserved.
%%
%%      This Standard Cell Library is licensed under the Libre Silicon
%%      public license; you can redistribute it and/or modify it under
%%      the terms of the Libre Silicon public license as published by
%%      the Libre Silicon alliance, either version 1 of the License, or
%%      (at your option) any later version.
%%
%%      This design is distributed in the hope that it will be useful,
%%      but WITHOUT ANY WARRANTY; without even the implied warranty of
%%      MERCHANTABILITY or FITNESS FOR A PARTICULAR PURPOSE.
%%      See the Libre Silicon Public License for more details.
%%
%%  ///////////////////////////////////////////////////////////////////
\begin{circuitdiagram}[draft]{31}{22}

    \usgate
    % ----  1st column  ----
    \pin{1}{1}{L}{A}
    \pin{1}{5}{L}{A1}
    \gate[\inputs{2}]{and}{5}{3}{R}{}{}

    \pin{1}{7}{L}{B}
    \pin{1}{11}{L}{B1}
    \gate[\inputs{2}]{and}{5}{9}{R}{}{}

    % ----  2nd column  ----
    \pin{8}{13}{L}{C}
    \wire{9}{3}{9}{7}
    \wire{9}{11}{9}{13}
    \gate[\inputs{3}]{or}{12}{9}{R}{}{}

    \pin{8}{15}{L}{D}
    \pin{8}{17}{L}{D1}
    \pin{8}{19}{L}{D2}
    \pin{8}{21}{L}{D3}
    \gate[\inputs{4}]{or}{12}{18}{R}{}{}

    % ----  3rd column  ----
    \wire{16}{9}{16}{14}
    \gate[\inputs{2}]{nand}{19}{16}{R}{}{}

    % ----  4th column  ----
    \gate{not}{26}{16}{R}{}{}

    % ----  result ----
    \pin{30}{16}{R}{Z}

\end{circuitdiagram}

%%  ************    LibreSilicon's StdCellLibrary   *******************
%%
%%  Organisation:   Chipforge
%%                  Germany / European Union
%%
%%  Profile:        Chipforge focus on fine System-on-Chip Cores in
%%                  Verilog HDL Code which are easy understandable and
%%                  adjustable. For further information see
%%                          www.chipforge.org
%%                  there are projects from small cores up to PCBs, too.
%%
%%  File:           StdCellLib/Documents/Datasheets/Circuitry/AAOOAI2221.tex
%%
%%  Purpose:        Circuit File for AAOOAI2221
%%
%%  ************    LaTeX with circdia.sty package      ***************
%%
%%  ///////////////////////////////////////////////////////////////////
%%
%%  Copyright (c) 2018 - 2022 by
%%                  chipforge <stdcelllib@nospam.chipforge.org>
%%  All rights reserved.
%%
%%      This Standard Cell Library is licensed under the Libre Silicon
%%      public license; you can redistribute it and/or modify it under
%%      the terms of the Libre Silicon public license as published by
%%      the Libre Silicon alliance, either version 1 of the License, or
%%      (at your option) any later version.
%%
%%      This design is distributed in the hope that it will be useful,
%%      but WITHOUT ANY WARRANTY; without even the implied warranty of
%%      MERCHANTABILITY or FITNESS FOR A PARTICULAR PURPOSE.
%%      See the Libre Silicon Public License for more details.
%%
%%  ///////////////////////////////////////////////////////////////////
\begin{circuitdiagram}[draft]{25}{20}

    \usgate
    % ----  1st column  ----
    \pin{1}{1}{L}{A}
    \pin{1}{5}{L}{A1}
    \gate[\inputs{2}]{and}{5}{3}{R}{}{}

    \pin{1}{7}{L}{B}
    \pin{1}{11}{L}{B1}
    \gate[\inputs{2}]{and}{5}{9}{R}{}{}

    % ----  2nd column  ----
    \wire{9}{3}{9}{5}
    \gate[\inputs{2}]{or}{12}{7}{R}{}{}

    \pin{8}{13}{L}{C}
    \pin{8}{17}{L}{C1}
    \gate[\inputs{2}]{or}{12}{15}{R}{}{}

    % ----  3rd column  ----
    \wire{16}{7}{16}{13}
    \pin{15}{19}{L}{D}
    \wire{16}{17}{16}{19}
    \gate[\inputs{3}]{nand}{19}{15}{R}{}{}

    % ----  result ----
    \pin{24}{15}{R}{Y}

\end{circuitdiagram}
 %%  ************    LibreSilicon's StdCellLibrary   *******************
%%
%%  Organisation:   Chipforge
%%                  Germany / European Union
%%
%%  Profile:        Chipforge focus on fine System-on-Chip Cores in
%%                  Verilog HDL Code which are easy understandable and
%%                  adjustable. For further information see
%%                          www.chipforge.org
%%                  there are projects from small cores up to PCBs, too.
%%
%%  File:           StdCellLib/Documents/Datasheets/Circuitry/AAOOA2221.tex
%%
%%  Purpose:        Circuit File for AAOOA2221
%%
%%  ************    LaTeX with circdia.sty package      ***************
%%
%%  ///////////////////////////////////////////////////////////////////
%%
%%  Copyright (c) 2018 - 2022 by
%%                  chipforge <stdcelllib@nospam.chipforge.org>
%%  All rights reserved.
%%
%%      This Standard Cell Library is licensed under the Libre Silicon
%%      public license; you can redistribute it and/or modify it under
%%      the terms of the Libre Silicon public license as published by
%%      the Libre Silicon alliance, either version 1 of the License, or
%%      (at your option) any later version.
%%
%%      This design is distributed in the hope that it will be useful,
%%      but WITHOUT ANY WARRANTY; without even the implied warranty of
%%      MERCHANTABILITY or FITNESS FOR A PARTICULAR PURPOSE.
%%      See the Libre Silicon Public License for more details.
%%
%%  ///////////////////////////////////////////////////////////////////
\begin{circuitdiagram}[draft]{31}{20}

    \usgate
    % ----  1st column  ----
    \pin{1}{1}{L}{A}
    \pin{1}{5}{L}{A1}
    \gate[\inputs{2}]{and}{5}{3}{R}{}{}

    \pin{1}{7}{L}{B}
    \pin{1}{11}{L}{B1}
    \gate[\inputs{2}]{and}{5}{9}{R}{}{}

    % ----  2nd column  ----
    \wire{9}{3}{9}{5}
    \gate[\inputs{2}]{or}{12}{7}{R}{}{}

    \pin{8}{13}{L}{C}
    \pin{8}{17}{L}{C1}
    \gate[\inputs{2}]{or}{12}{15}{R}{}{}

    % ----  3rd column  ----
    \wire{16}{7}{16}{13}
    \pin{15}{19}{L}{D}
    \wire{16}{17}{16}{19}
    \gate[\inputs{3}]{nand}{19}{15}{R}{}{}

    % ----  4th column  ----
    \gate{not}{26}{15}{R}{}{}

    % ----  result ----
    \pin{30}{15}{R}{Z}

\end{circuitdiagram}

%%  ************    LibreSilicon's StdCellLibrary   *******************
%%
%%  Organisation:   Chipforge
%%                  Germany / European Union
%%
%%  Profile:        Chipforge focus on fine System-on-Chip Cores in
%%                  Verilog HDL Code which are easy understandable and
%%                  adjustable. For further information see
%%                          www.chipforge.org
%%                  there are projects from small cores up to PCBs, too.
%%
%%  File:           StdCellLib/Documents/Datasheets/Circuitry/AAOOAI2222.tex
%%
%%  Purpose:        Circuit File for AAOOAI2222
%%
%%  ************    LaTeX with circdia.sty package      ***************
%%
%%  ///////////////////////////////////////////////////////////////////
%%
%%  Copyright (c) 2018 - 2022 by
%%                  chipforge <stdcelllib@nospam.chipforge.org>
%%  All rights reserved.
%%
%%      This Standard Cell Library is licensed under the Libre Silicon
%%      public license; you can redistribute it and/or modify it under
%%      the terms of the Libre Silicon public license as published by
%%      the Libre Silicon alliance, either version 1 of the License, or
%%      (at your option) any later version.
%%
%%      This design is distributed in the hope that it will be useful,
%%      but WITHOUT ANY WARRANTY; without even the implied warranty of
%%      MERCHANTABILITY or FITNESS FOR A PARTICULAR PURPOSE.
%%      See the Libre Silicon Public License for more details.
%%
%%  ///////////////////////////////////////////////////////////////////
\begin{circuitdiagram}[draft]{25}{24}

    \usgate
    % ----  1st column  ----
    \pin{1}{1}{L}{A}
    \pin{1}{5}{L}{A1}
    \gate[\inputs{2}]{and}{5}{3}{R}{}{}

    \pin{1}{7}{L}{B}
    \pin{1}{11}{L}{B1}
    \gate[\inputs{2}]{and}{5}{9}{R}{}{}

    % ----  2nd column  ----
    \wire{9}{3}{9}{5}
    \gate[\inputs{2}]{or}{12}{7}{R}{}{}

    \pin{8}{13}{L}{C}
    \pin{8}{17}{L}{C1}
    \gate[\inputs{2}]{or}{12}{15}{R}{}{}

    \pin{8}{19}{L}{D}
    \pin{8}{23}{L}{D1}
    \gate[\inputs{2}]{or}{12}{21}{R}{}{}

    % ----  3rd column  ----
    \wire{16}{7}{16}{13}
    \wire{16}{17}{16}{21}
    \gate[\inputs{3}]{nand}{19}{15}{R}{}{}

    % ----  result ----
    \pin{24}{15}{R}{Y}

\end{circuitdiagram}
 %%  ************    LibreSilicon's StdCellLibrary   *******************
%%
%%  Organisation:   Chipforge
%%                  Germany / European Union
%%
%%  Profile:        Chipforge focus on fine System-on-Chip Cores in
%%                  Verilog HDL Code which are easy understandable and
%%                  adjustable. For further information see
%%                          www.chipforge.org
%%                  there are projects from small cores up to PCBs, too.
%%
%%  File:           StdCellLib/Documents/Datasheets/Circuitry/AAOOA2222.tex
%%
%%  Purpose:        Circuit File for AAOOA2222
%%
%%  ************    LaTeX with circdia.sty package      ***************
%%
%%  ///////////////////////////////////////////////////////////////////
%%
%%  Copyright (c) 2018 - 2022 by
%%                  chipforge <stdcelllib@nospam.chipforge.org>
%%  All rights reserved.
%%
%%      This Standard Cell Library is licensed under the Libre Silicon
%%      public license; you can redistribute it and/or modify it under
%%      the terms of the Libre Silicon public license as published by
%%      the Libre Silicon alliance, either version 1 of the License, or
%%      (at your option) any later version.
%%
%%      This design is distributed in the hope that it will be useful,
%%      but WITHOUT ANY WARRANTY; without even the implied warranty of
%%      MERCHANTABILITY or FITNESS FOR A PARTICULAR PURPOSE.
%%      See the Libre Silicon Public License for more details.
%%
%%  ///////////////////////////////////////////////////////////////////
\begin{circuitdiagram}[draft]{31}{24}

    \usgate
    % ----  1st column  ----
    \pin{1}{1}{L}{A}
    \pin{1}{5}{L}{A1}
    \gate[\inputs{2}]{and}{5}{3}{R}{}{}

    \pin{1}{7}{L}{B}
    \pin{1}{11}{L}{B1}
    \gate[\inputs{2}]{and}{5}{9}{R}{}{}

    % ----  2nd column  ----
    \wire{9}{3}{9}{5}
    \gate[\inputs{2}]{or}{12}{7}{R}{}{}

    \pin{8}{13}{L}{C}
    \pin{8}{17}{L}{C1}
    \gate[\inputs{2}]{or}{12}{15}{R}{}{}

    \pin{8}{19}{L}{D}
    \pin{8}{23}{L}{D1}
    \gate[\inputs{2}]{or}{12}{21}{R}{}{}

    % ----  3rd column  ----
    \wire{16}{7}{16}{13}
    \wire{16}{17}{16}{21}
    \gate[\inputs{3}]{nand}{19}{15}{R}{}{}

    % ----  last column ----
    \gate{not}{26}{15}{R}{}{}

    % ----  result ----
    \pin{30}{15}{R}{Z}

\end{circuitdiagram}

%%  ************    LibreSilicon's StdCellLibrary   *******************
%%
%%  Organisation:   Chipforge
%%                  Germany / European Union
%%
%%  Profile:        Chipforge focus on fine System-on-Chip Cores in
%%                  Verilog HDL Code which are easy understandable and
%%                  adjustable. For further information see
%%                          www.chipforge.org
%%                  there are projects from small cores up to PCBs, too.
%%
%%  File:           StdCellLib/Documents/Datasheets/Circuitry/AAOOAI2223.tex
%%
%%  Purpose:        Circuit File for AAOOAI2223
%%
%%  ************    LaTeX with circdia.sty package      ***************
%%
%%  ///////////////////////////////////////////////////////////////////
%%
%%  Copyright (c) 2018 - 2022 by
%%                  chipforge <stdcelllib@nospam.chipforge.org>
%%  All rights reserved.
%%
%%      This Standard Cell Library is licensed under the Libre Silicon
%%      public license; you can redistribute it and/or modify it under
%%      the terms of the Libre Silicon public license as published by
%%      the Libre Silicon alliance, either version 1 of the License, or
%%      (at your option) any later version.
%%
%%      This design is distributed in the hope that it will be useful,
%%      but WITHOUT ANY WARRANTY; without even the implied warranty of
%%      MERCHANTABILITY or FITNESS FOR A PARTICULAR PURPOSE.
%%      See the Libre Silicon Public License for more details.
%%
%%  ///////////////////////////////////////////////////////////////////
\begin{circuitdiagram}[draft]{27}{20}

    \usgate
    % ----  1st column  ----
    \pin{1}{1}{L}{A}
    \pin{1}{5}{L}{A1}
    \gate[\inputs{2}]{and}{5}{3}{R}{}{}

    \pin{1}{7}{L}{B}
    \pin{1}{11}{L}{B1}
    \gate[\inputs{2}]{and}{5}{9}{R}{}{}

    % ----  2nd column  ----
    \wire{9}{3}{9}{7}
    \wire{9}{7}{11}{7}
    \wire{9}{9}{11}{9}
    \pin{10}{11}{L}{C}
    \pin{10}{13}{L}{C1}
    \gate[\inputs{4}]{or}{14}{10}{R}{}{}

    \pin{10}{15}{L}{D}
    \pin{10}{17}{L}{D1}
    \pin{10}{19}{L}{D2}
    \gate[\inputs{3}]{or}{14}{17}{R}{}{}

    % ----  3rd column  ----
    \wire{18}{10}{18}{13}
    \gate[\inputs{2}]{nand}{21}{15}{R}{}{}

    % ----  result ----
    \pin{26}{15}{R}{Y}

\end{circuitdiagram}
 %%  ************    LibreSilicon's StdCellLibrary   *******************
%%
%%  Organisation:   Chipforge
%%                  Germany / European Union
%%
%%  Profile:        Chipforge focus on fine System-on-Chip Cores in
%%                  Verilog HDL Code which are easy understandable and
%%                  adjustable. For further information see
%%                          www.chipforge.org
%%                  there are projects from small cores up to PCBs, too.
%%
%%  File:           StdCellLib/Documents/Datasheets/Circuitry/AAOOA2223.tex
%%
%%  Purpose:        Circuit File for AAOOA2223
%%
%%  ************    LaTeX with circdia.sty package      ***************
%%
%%  ///////////////////////////////////////////////////////////////////
%%
%%  Copyright (c) 2018 - 2022 by
%%                  chipforge <stdcelllib@nospam.chipforge.org>
%%  All rights reserved.
%%
%%      This Standard Cell Library is licensed under the Libre Silicon
%%      public license; you can redistribute it and/or modify it under
%%      the terms of the Libre Silicon public license as published by
%%      the Libre Silicon alliance, either version 1 of the License, or
%%      (at your option) any later version.
%%
%%      This design is distributed in the hope that it will be useful,
%%      but WITHOUT ANY WARRANTY; without even the implied warranty of
%%      MERCHANTABILITY or FITNESS FOR A PARTICULAR PURPOSE.
%%      See the Libre Silicon Public License for more details.
%%
%%  ///////////////////////////////////////////////////////////////////
\begin{circuitdiagram}[draft]{33}{20}

    \usgate
    % ----  1st column  ----
    \pin{1}{1}{L}{A}
    \pin{1}{5}{L}{A1}
    \gate[\inputs{2}]{and}{5}{3}{R}{}{}

    \pin{1}{7}{L}{B}
    \pin{1}{11}{L}{B1}
    \gate[\inputs{2}]{and}{5}{9}{R}{}{}

    % ----  2nd column  ----
    \wire{9}{3}{9}{7}
    \wire{9}{7}{11}{7}
    \wire{9}{9}{11}{9}
    \pin{10}{11}{L}{C}
    \pin{10}{13}{L}{C1}
    \gate[\inputs{4}]{or}{14}{10}{R}{}{}

    \pin{10}{15}{L}{D}
    \pin{10}{17}{L}{D1}
    \pin{10}{19}{L}{D2}
    \gate[\inputs{3}]{or}{14}{17}{R}{}{}

    % ----  3rd column  ----
    \wire{18}{10}{18}{13}
    \gate[\inputs{2}]{nand}{21}{15}{R}{}{}

    % ----  last column ----
    \gate{not}{28}{15}{R}{}{}

    % ----  result ----
    \pin{32}{15}{R}{Z}

\end{circuitdiagram}

%%  ************    LibreSilicon's StdCellLibrary   *******************
%%
%%  Organisation:   Chipforge
%%                  Germany / European Union
%%
%%  Profile:        Chipforge focus on fine System-on-Chip Cores in
%%                  Verilog HDL Code which are easy understandable and
%%                  adjustable. For further information see
%%                          www.chipforge.org
%%                  there are projects from small cores up to PCBs, too.
%%
%%  File:           StdCellLib/Documents/Datasheets/Circuitry/AAOOAI2231.tex
%%
%%  Purpose:        Circuit File for AAOOAI2231
%%
%%  ************    LaTeX with circdia.sty package      ***************
%%
%%  ///////////////////////////////////////////////////////////////////
%%
%%  Copyright (c) 2018 - 2022 by
%%                  chipforge <stdcelllib@nospam.chipforge.org>
%%  All rights reserved.
%%
%%      This Standard Cell Library is licensed under the Libre Silicon
%%      public license; you can redistribute it and/or modify it under
%%      the terms of the Libre Silicon public license as published by
%%      the Libre Silicon alliance, either version 1 of the License, or
%%      (at your option) any later version.
%%
%%      This design is distributed in the hope that it will be useful,
%%      but WITHOUT ANY WARRANTY; without even the implied warranty of
%%      MERCHANTABILITY or FITNESS FOR A PARTICULAR PURPOSE.
%%      See the Libre Silicon Public License for more details.
%%
%%  ///////////////////////////////////////////////////////////////////
\begin{circuitdiagram}[draft]{25}{20}

    \usgate
    % ----  1st column  ----
    \pin{1}{1}{L}{A}
    \pin{1}{5}{L}{A1}
    \gate[\inputs{2}]{and}{5}{3}{R}{}{}

    \pin{1}{7}{L}{B}
    \pin{1}{11}{L}{B1}
    \gate[\inputs{2}]{and}{5}{9}{R}{}{}

    % ----  2nd column  ----
    \wire{9}{3}{9}{5}
    \gate[\inputs{2}]{or}{12}{7}{R}{}{}

    \pin{8}{13}{L}{C}
    \pin{8}{15}{L}{C1}
    \pin{8}{17}{L}{C2}
    \gate[\inputs{3}]{or}{12}{15}{R}{}{}

    % ----  3rd column  ----
    \wire{16}{7}{16}{13}
    \pin{15}{19}{L}{D}
    \wire{16}{17}{16}{19}
    \gate[\inputs{3}]{nand}{19}{15}{R}{}{}

    % ----  result ----
    \pin{24}{15}{R}{Y}

\end{circuitdiagram}
 %%  ************    LibreSilicon's StdCellLibrary   *******************
%%
%%  Organisation:   Chipforge
%%                  Germany / European Union
%%
%%  Profile:        Chipforge focus on fine System-on-Chip Cores in
%%                  Verilog HDL Code which are easy understandable and
%%                  adjustable. For further information see
%%                          www.chipforge.org
%%                  there are projects from small cores up to PCBs, too.
%%
%%  File:           StdCellLib/Documents/Datasheets/Circuitry/AAOOA2231.tex
%%
%%  Purpose:        Circuit File for AAOOA2231
%%
%%  ************    LaTeX with circdia.sty package      ***************
%%
%%  ///////////////////////////////////////////////////////////////////
%%
%%  Copyright (c) 2018 - 2022 by
%%                  chipforge <stdcelllib@nospam.chipforge.org>
%%  All rights reserved.
%%
%%      This Standard Cell Library is licensed under the Libre Silicon
%%      public license; you can redistribute it and/or modify it under
%%      the terms of the Libre Silicon public license as published by
%%      the Libre Silicon alliance, either version 1 of the License, or
%%      (at your option) any later version.
%%
%%      This design is distributed in the hope that it will be useful,
%%      but WITHOUT ANY WARRANTY; without even the implied warranty of
%%      MERCHANTABILITY or FITNESS FOR A PARTICULAR PURPOSE.
%%      See the Libre Silicon Public License for more details.
%%
%%  ///////////////////////////////////////////////////////////////////
\begin{circuitdiagram}[draft]{31}{20}

    \usgate
    % ----  1st column  ----
    \pin{1}{1}{L}{A}
    \pin{1}{5}{L}{A1}
    \gate[\inputs{2}]{and}{5}{3}{R}{}{}

    \pin{1}{7}{L}{B}
    \pin{1}{11}{L}{B1}
    \gate[\inputs{2}]{and}{5}{9}{R}{}{}

    % ----  2nd column  ----
    \wire{9}{3}{9}{5}
    \gate[\inputs{2}]{or}{12}{7}{R}{}{}

    \pin{8}{13}{L}{C}
    \pin{8}{15}{L}{C1}
    \pin{8}{17}{L}{C2}
    \gate[\inputs{3}]{or}{12}{15}{R}{}{}

    % ----  3rd column  ----
    \wire{16}{7}{16}{13}
    \pin{15}{19}{L}{D}
    \wire{16}{17}{16}{19}
    \gate[\inputs{3}]{nand}{19}{15}{R}{}{}

    % ----  4th column  ----
    \gate{not}{26}{15}{R}{}{}

    % ----  result ----
    \pin{30}{15}{R}{Z}

\end{circuitdiagram}

%%  ************    LibreSilicon's StdCellLibrary   *******************
%%
%%  Organisation:   Chipforge
%%                  Germany / European Union
%%
%%  Profile:        Chipforge focus on fine System-on-Chip Cores in
%%                  Verilog HDL Code which are easy understandable and
%%                  adjustable. For further information see
%%                          www.chipforge.org
%%                  there are projects from small cores up to PCBs, too.
%%
%%  File:           StdCellLib/Documents/Datasheets/Circuitry/AAOOAI2241.tex
%%
%%  Purpose:        Circuit File for AAOOAI2241
%%
%%  ************    LaTeX with circdia.sty package      ***************
%%
%%  ///////////////////////////////////////////////////////////////////
%%
%%  Copyright (c) 2018 - 2022 by
%%                  chipforge <stdcelllib@nospam.chipforge.org>
%%  All rights reserved.
%%
%%      This Standard Cell Library is licensed under the Libre Silicon
%%      public license; you can redistribute it and/or modify it under
%%      the terms of the Libre Silicon public license as published by
%%      the Libre Silicon alliance, either version 1 of the License, or
%%      (at your option) any later version.
%%
%%      This design is distributed in the hope that it will be useful,
%%      but WITHOUT ANY WARRANTY; without even the implied warranty of
%%      MERCHANTABILITY or FITNESS FOR A PARTICULAR PURPOSE.
%%      See the Libre Silicon Public License for more details.
%%
%%  ///////////////////////////////////////////////////////////////////
\begin{circuitdiagram}[draft]{25}{21}

    \usgate
    % ----  1st column  ----
    \pin{1}{1}{L}{A}
    \pin{1}{5}{L}{A1}
    \gate[\inputs{2}]{and}{5}{3}{R}{}{}

    \pin{1}{7}{L}{B}
    \pin{1}{11}{L}{B1}
    \gate[\inputs{2}]{and}{5}{9}{R}{}{}

    % ----  2nd column  ----
    \wire{9}{3}{9}{5}
    \gate[\inputs{2}]{or}{12}{7}{R}{}{}

    \pin{8}{13}{L}{C}
    \pin{8}{15}{L}{C1}
    \pin{8}{17}{L}{C2}
    \pin{8}{19}{L}{C3}
    \gate[\inputs{4}]{or}{12}{16}{R}{}{}

    % ----  3rd column  ----
    \wire{16}{7}{16}{14}
    \pin{15}{20}{L}{D}
    \wire{16}{18}{16}{20}
    \gate[\inputs{3}]{nand}{19}{16}{R}{}{}

    % ----  result ----
    \pin{24}{16}{R}{Y}

\end{circuitdiagram}
 %%  ************    LibreSilicon's StdCellLibrary   *******************
%%
%%  Organisation:   Chipforge
%%                  Germany / European Union
%%
%%  Profile:        Chipforge focus on fine System-on-Chip Cores in
%%                  Verilog HDL Code which are easy understandable and
%%                  adjustable. For further information see
%%                          www.chipforge.org
%%                  there are projects from small cores up to PCBs, too.
%%
%%  File:           StdCellLib/Documents/Datasheets/Circuitry/AAOOA2241.tex
%%
%%  Purpose:        Circuit File for AAOOA2241
%%
%%  ************    LaTeX with circdia.sty package      ***************
%%
%%  ///////////////////////////////////////////////////////////////////
%%
%%  Copyright (c) 2018 - 2022 by
%%                  chipforge <stdcelllib@nospam.chipforge.org>
%%  All rights reserved.
%%
%%      This Standard Cell Library is licensed under the Libre Silicon
%%      public license; you can redistribute it and/or modify it under
%%      the terms of the Libre Silicon public license as published by
%%      the Libre Silicon alliance, either version 1 of the License, or
%%      (at your option) any later version.
%%
%%      This design is distributed in the hope that it will be useful,
%%      but WITHOUT ANY WARRANTY; without even the implied warranty of
%%      MERCHANTABILITY or FITNESS FOR A PARTICULAR PURPOSE.
%%      See the Libre Silicon Public License for more details.
%%
%%  ///////////////////////////////////////////////////////////////////
\begin{circuitdiagram}[draft]{31}{21}

    \usgate
    % ----  1st column  ----
    \pin{1}{1}{L}{A}
    \pin{1}{5}{L}{A1}
    \gate[\inputs{2}]{and}{5}{3}{R}{}{}

    \pin{1}{7}{L}{B}
    \pin{1}{11}{L}{B1}
    \gate[\inputs{2}]{and}{5}{9}{R}{}{}

    % ----  2nd column  ----
    \wire{9}{3}{9}{5}
    \gate[\inputs{2}]{or}{12}{7}{R}{}{}

    \pin{8}{13}{L}{C}
    \pin{8}{15}{L}{C1}
    \pin{8}{17}{L}{C2}
    \pin{8}{19}{L}{C3}
    \gate[\inputs{4}]{or}{12}{16}{R}{}{}

    % ----  3rd column  ----
    \wire{16}{7}{16}{14}
    \pin{15}{20}{L}{D}
    \wire{16}{18}{16}{20}
    \gate[\inputs{3}]{nand}{19}{16}{R}{}{}

    % ----  last column ----
    \gate{not}{26}{16}{R}{}{}

    % ----  result ----
    \pin{30}{16}{R}{Z}

\end{circuitdiagram}

%%  ************    LibreSilicon's StdCellLibrary   *******************
%%
%%  Organisation:   Chipforge
%%                  Germany / European Union
%%
%%  Profile:        Chipforge focus on fine System-on-Chip Cores in
%%                  Verilog HDL Code which are easy understandable and
%%                  adjustable. For further information see
%%                          www.chipforge.org
%%                  there are projects from small cores up to PCBs, too.
%%
%%  File:           StdCellLib/Documents/Datasheets/Circuitry/AAOOAI3212.tex
%%
%%  Purpose:        Circuit File for AAOOAI3212
%%
%%  ************    LaTeX with circdia.sty package      ***************
%%
%%  ///////////////////////////////////////////////////////////////////
%%
%%  Copyright (c) 2018 - 2022 by
%%                  chipforge <stdcelllib@nospam.chipforge.org>
%%  All rights reserved.
%%
%%      This Standard Cell Library is licensed under the Libre Silicon
%%      public license; you can redistribute it and/or modify it under
%%      the terms of the Libre Silicon public license as published by
%%      the Libre Silicon alliance, either version 1 of the License, or
%%      (at your option) any later version.
%%
%%      This design is distributed in the hope that it will be useful,
%%      but WITHOUT ANY WARRANTY; without even the implied warranty of
%%      MERCHANTABILITY or FITNESS FOR A PARTICULAR PURPOSE.
%%      See the Libre Silicon Public License for more details.
%%
%%  ///////////////////////////////////////////////////////////////////
\begin{circuitdiagram}[draft]{25}{20}

    \usgate
    % ----  1st column  ----
    \pin{1}{1}{L}{A}
    \pin{1}{3}{L}{A1}
    \pin{1}{5}{L}{A2}
    \gate[\inputs{3}]{and}{5}{3}{R}{}{}

    \pin{1}{7}{L}{B}
    \pin{1}{11}{L}{B1}
    \gate[\inputs{2}]{and}{5}{9}{R}{}{}

    % ----  2nd column  ----
    \pin{8}{13}{L}{C}
    \wire{9}{3}{9}{7}
    \wire{9}{11}{9}{13}
    \gate[\inputs{3}]{or}{12}{9}{R}{}{}

    \pin{8}{15}{L}{D}
    \pin{8}{19}{L}{D1}
    \gate[\inputs{2}]{or}{12}{17}{R}{}{}

    % ----  3rd column  ----
    \wire{16}{9}{16}{13}
    \gate[\inputs{2}]{nand}{19}{15}{R}{}{}

    % ----  result ----
    \pin{24}{15}{R}{Y}

\end{circuitdiagram}
 %%  ************    LibreSilicon's StdCellLibrary   *******************
%%
%%  Organisation:   Chipforge
%%                  Germany / European Union
%%
%%  Profile:        Chipforge focus on fine System-on-Chip Cores in
%%                  Verilog HDL Code which are easy understandable and
%%                  adjustable. For further information see
%%                          www.chipforge.org
%%                  there are projects from small cores up to PCBs, too.
%%
%%  File:           StdCellLib/Documents/Datasheets/Circuitry/AAOOA3212.tex
%%
%%  Purpose:        Circuit File for AAOOA3212
%%
%%  ************    LaTeX with circdia.sty package      ***************
%%
%%  ///////////////////////////////////////////////////////////////////
%%
%%  Copyright (c) 2018 - 2022 by
%%                  chipforge <stdcelllib@nospam.chipforge.org>
%%  All rights reserved.
%%
%%      This Standard Cell Library is licensed under the Libre Silicon
%%      public license; you can redistribute it and/or modify it under
%%      the terms of the Libre Silicon public license as published by
%%      the Libre Silicon alliance, either version 1 of the License, or
%%      (at your option) any later version.
%%
%%      This design is distributed in the hope that it will be useful,
%%      but WITHOUT ANY WARRANTY; without even the implied warranty of
%%      MERCHANTABILITY or FITNESS FOR A PARTICULAR PURPOSE.
%%      See the Libre Silicon Public License for more details.
%%
%%  ///////////////////////////////////////////////////////////////////
\begin{circuitdiagram}[draft]{31}{20}

    \usgate
    % ----  1st column  ----
    \pin{1}{1}{L}{A}
    \pin{1}{3}{L}{A1}
    \pin{1}{5}{L}{A2}
    \gate[\inputs{3}]{and}{5}{3}{R}{}{}

    \pin{1}{7}{L}{B}
    \pin{1}{11}{L}{B1}
    \gate[\inputs{2}]{and}{5}{9}{R}{}{}

    % ----  2nd column  ----
    \pin{8}{13}{L}{C}
    \wire{9}{3}{9}{7}
    \wire{9}{11}{9}{13}
    \gate[\inputs{3}]{or}{12}{9}{R}{}{}

    \pin{8}{15}{L}{D}
    \pin{8}{19}{L}{D1}
    \gate[\inputs{2}]{or}{12}{17}{R}{}{}

    % ----  3rd column  ----
    \wire{16}{9}{16}{13}
    \gate[\inputs{2}]{nand}{19}{15}{R}{}{}

    % ----  last column ----
    \gate{not}{26}{15}{R}{}{}

    % ----  result ----
    \pin{30}{15}{R}{Z}

\end{circuitdiagram}

%%  ************    LibreSilicon's StdCellLibrary   *******************
%%
%%  Organisation:   Chipforge
%%                  Germany / European Union
%%
%%  Profile:        Chipforge focus on fine System-on-Chip Cores in
%%                  Verilog HDL Code which are easy understandable and
%%                  adjustable. For further information see
%%                          www.chipforge.org
%%                  there are projects from small cores up to PCBs, too.
%%
%%  File:           StdCellLib/Documents/Datasheets/Circuitry/AAOOAI3312.tex
%%
%%  Purpose:        Circuit File for AAOOAI3312
%%
%%  ************    LaTeX with circdia.sty package      ***************
%%
%%  ///////////////////////////////////////////////////////////////////
%%
%%  Copyright (c) 2018 - 2022 by
%%                  chipforge <stdcelllib@nospam.chipforge.org>
%%  All rights reserved.
%%
%%      This Standard Cell Library is licensed under the Libre Silicon
%%      public license; you can redistribute it and/or modify it under
%%      the terms of the Libre Silicon public license as published by
%%      the Libre Silicon alliance, either version 1 of the License, or
%%      (at your option) any later version.
%%
%%      This design is distributed in the hope that it will be useful,
%%      but WITHOUT ANY WARRANTY; without even the implied warranty of
%%      MERCHANTABILITY or FITNESS FOR A PARTICULAR PURPOSE.
%%      See the Libre Silicon Public License for more details.
%%
%%  ///////////////////////////////////////////////////////////////////
\begin{circuitdiagram}[draft]{25}{20}

    \usgate
    % ----  1st column  ----
    \pin{1}{1}{L}{A}
    \pin{1}{3}{L}{A1}
    \pin{1}{5}{L}{A2}
    \gate[\inputs{3}]{and}{5}{3}{R}{}{}

    \pin{1}{7}{L}{B}
    \pin{1}{9}{L}{B1}
    \pin{1}{11}{L}{B2}
    \gate[\inputs{3}]{and}{5}{9}{R}{}{}

    % ----  2nd column  ----
    \pin{8}{13}{L}{C}
    \wire{9}{3}{9}{7}
    \wire{9}{11}{9}{13}
    \gate[\inputs{3}]{or}{12}{9}{R}{}{}

    \pin{8}{15}{L}{D}
    \pin{8}{19}{L}{D1}
    \gate[\inputs{2}]{or}{12}{17}{R}{}{}

    % ----  3rd column  ----
    \wire{16}{9}{16}{13}
    \gate[\inputs{2}]{nand}{19}{15}{R}{}{}

    % ----  result ----
    \pin{24}{15}{R}{Y}

\end{circuitdiagram}
 %%  ************    LibreSilicon's StdCellLibrary   *******************
%%
%%  Organisation:   Chipforge
%%                  Germany / European Union
%%
%%  Profile:        Chipforge focus on fine System-on-Chip Cores in
%%                  Verilog HDL Code which are easy understandable and
%%                  adjustable. For further information see
%%                          www.chipforge.org
%%                  there are projects from small cores up to PCBs, too.
%%
%%  File:           StdCellLib/Documents/Datasheets/Circuitry/AAOOA3312.tex
%%
%%  Purpose:        Circuit File for AAOOA3312
%%
%%  ************    LaTeX with circdia.sty package      ***************
%%
%%  ///////////////////////////////////////////////////////////////////
%%
%%  Copyright (c) 2018 - 2022 by
%%                  chipforge <stdcelllib@nospam.chipforge.org>
%%  All rights reserved.
%%
%%      This Standard Cell Library is licensed under the Libre Silicon
%%      public license; you can redistribute it and/or modify it under
%%      the terms of the Libre Silicon public license as published by
%%      the Libre Silicon alliance, either version 1 of the License, or
%%      (at your option) any later version.
%%
%%      This design is distributed in the hope that it will be useful,
%%      but WITHOUT ANY WARRANTY; without even the implied warranty of
%%      MERCHANTABILITY or FITNESS FOR A PARTICULAR PURPOSE.
%%      See the Libre Silicon Public License for more details.
%%
%%  ///////////////////////////////////////////////////////////////////
\begin{circuitdiagram}[draft]{31}{20}

    \usgate
    % ----  1st column  ----
    \pin{1}{1}{L}{A}
    \pin{1}{3}{L}{A1}
    \pin{1}{5}{L}{A2}
    \gate[\inputs{3}]{and}{5}{3}{R}{}{}

    \pin{1}{7}{L}{B}
    \pin{1}{9}{L}{B1}
    \pin{1}{11}{L}{B2}
    \gate[\inputs{3}]{and}{5}{9}{R}{}{}

    % ----  2nd column  ----
    \pin{8}{13}{L}{C}
    \wire{9}{3}{9}{7}
    \wire{9}{11}{9}{13}
    \gate[\inputs{3}]{or}{12}{9}{R}{}{}

    \pin{8}{15}{L}{D}
    \pin{8}{19}{L}{D1}
    \gate[\inputs{2}]{or}{12}{17}{R}{}{}

    % ----  3rd column  ----
    \wire{16}{9}{16}{13}
    \gate[\inputs{2}]{nand}{19}{15}{R}{}{}

    % ----  last column ----
    \gate{not}{26}{15}{R}{}{}

    % ----  result ----
    \pin{30}{15}{R}{Z}

\end{circuitdiagram}


%%  ************    LibreSilicon's StdCellLibrary   *******************
%%
%%  Organisation:   Chipforge
%%                  Germany / European Union
%%
%%  Profile:        Chipforge focus on fine System-on-Chip Cores in
%%                  Verilog HDL Code which are easy understandable and
%%                  adjustable. For further information see
%%                          www.chipforge.org
%%                  there are projects from small cores up to PCBs, too.
%%
%%  File:           StdCellLib/Documents/Datasheets/Circuitry/AAOOAI22121.tex
%%
%%  Purpose:        Circuit File for AAOOAI22121
%%
%%  ************    LaTeX with circdia.sty package      ***************
%%
%%  ///////////////////////////////////////////////////////////////////
%%
%%  Copyright (c) 2018 - 2022 by
%%                  chipforge <stdcelllib@nospam.chipforge.org>
%%  All rights reserved.
%%
%%      This Standard Cell Library is licensed under the Libre Silicon
%%      public license; you can redistribute it and/or modify it under
%%      the terms of the Libre Silicon public license as published by
%%      the Libre Silicon alliance, either version 1 of the License, or
%%      (at your option) any later version.
%%
%%      This design is distributed in the hope that it will be useful,
%%      but WITHOUT ANY WARRANTY; without even the implied warranty of
%%      MERCHANTABILITY or FITNESS FOR A PARTICULAR PURPOSE.
%%      See the Libre Silicon Public License for more details.
%%
%%  ///////////////////////////////////////////////////////////////////
\begin{circuitdiagram}[draft]{25}{22}

    \usgate
    % ----  1st column  ----
    \pin{1}{1}{L}{A}
    \pin{1}{5}{L}{A1}
    \gate[\inputs{2}]{and}{5}{3}{R}{}{}

    \pin{1}{7}{L}{B}
    \pin{1}{11}{L}{B1}
    \gate[\inputs{2}]{and}{5}{9}{R}{}{}

    % ----  2nd column  ----
    \pin{8}{13}{L}{C}
    \wire{9}{3}{9}{7}
    \wire{9}{11}{9}{13}
    \gate[\inputs{3}]{or}{12}{9}{R}{}{}

    \pin{8}{15}{L}{D}
    \pin{8}{19}{L}{D1}
    \gate[\inputs{2}]{or}{12}{17}{R}{}{}

    % ----  3rd column  ----
    \wire{16}{9}{16}{15}
    \pin{15}{21}{L}{E}
    \wire{16}{19}{16}{21}
    \gate[\inputs{3}]{nand}{19}{17}{R}{}{}

    % ----  result ----
    \pin{24}{17}{R}{Y}

\end{circuitdiagram}
 %%  ************    LibreSilicon's StdCellLibrary   *******************
%%
%%  Organisation:   Chipforge
%%                  Germany / European Union
%%
%%  Profile:        Chipforge focus on fine System-on-Chip Cores in
%%                  Verilog HDL Code which are easy understandable and
%%                  adjustable. For further information see
%%                          www.chipforge.org
%%                  there are projects from small cores up to PCBs, too.
%%
%%  File:           StdCellLib/Documents/Datasheets/Circuitry/AAOOA22121.tex
%%
%%  Purpose:        Circuit File for AAOOA22121
%%
%%  ************    LaTeX with circdia.sty package      ***************
%%
%%  ///////////////////////////////////////////////////////////////////
%%
%%  Copyright (c) 2018 - 2022 by
%%                  chipforge <stdcelllib@nospam.chipforge.org>
%%  All rights reserved.
%%
%%      This Standard Cell Library is licensed under the Libre Silicon
%%      public license; you can redistribute it and/or modify it under
%%      the terms of the Libre Silicon public license as published by
%%      the Libre Silicon alliance, either version 1 of the License, or
%%      (at your option) any later version.
%%
%%      This design is distributed in the hope that it will be useful,
%%      but WITHOUT ANY WARRANTY; without even the implied warranty of
%%      MERCHANTABILITY or FITNESS FOR A PARTICULAR PURPOSE.
%%      See the Libre Silicon Public License for more details.
%%
%%  ///////////////////////////////////////////////////////////////////
\begin{circuitdiagram}[draft]{31}{22}

    \usgate
    % ----  1st column  ----
    \pin{1}{1}{L}{A}
    \pin{1}{5}{L}{A1}
    \gate[\inputs{2}]{and}{5}{3}{R}{}{}

    \pin{1}{7}{L}{B}
    \pin{1}{11}{L}{B1}
    \gate[\inputs{2}]{and}{5}{9}{R}{}{}

    % ----  2nd column  ----
    \pin{8}{13}{L}{C}
    \wire{9}{3}{9}{7}
    \wire{9}{11}{9}{13}
    \gate[\inputs{3}]{or}{12}{9}{R}{}{}

    \pin{8}{15}{L}{D}
    \pin{8}{19}{L}{D1}
    \gate[\inputs{2}]{or}{12}{17}{R}{}{}

    % ----  3rd column  ----
    \wire{16}{9}{16}{15}
    \pin{15}{21}{L}{E}
    \wire{16}{19}{16}{21}
    \gate[\inputs{3}]{nand}{19}{17}{R}{}{}

    % ----  4th column  ----
    \gate{not}{26}{17}{R}{}{}

    % ----  result ----
    \pin{30}{17}{R}{Z}

\end{circuitdiagram}

%%  ************    LibreSilicon's StdCellLibrary   *******************
%%
%%  Organisation:   Chipforge
%%                  Germany / European Union
%%
%%  Profile:        Chipforge focus on fine System-on-Chip Cores in
%%                  Verilog HDL Code which are easy understandable and
%%                  adjustable. For further information see
%%                          www.chipforge.org
%%                  there are projects from small cores up to PCBs, too.
%%
%%  File:           StdCellLib/Documents/Datasheets/Circuitry/AAOOAI22131.tex
%%
%%  Purpose:        Circuit File for AAOOAI22131
%%
%%  ************    LaTeX with circdia.sty package      ***************
%%
%%  ///////////////////////////////////////////////////////////////////
%%
%%  Copyright (c) 2018 - 2022 by
%%                  chipforge <stdcelllib@nospam.chipforge.org>
%%  All rights reserved.
%%
%%      This Standard Cell Library is licensed under the Libre Silicon
%%      public license; you can redistribute it and/or modify it under
%%      the terms of the Libre Silicon public license as published by
%%      the Libre Silicon alliance, either version 1 of the License, or
%%      (at your option) any later version.
%%
%%      This design is distributed in the hope that it will be useful,
%%      but WITHOUT ANY WARRANTY; without even the implied warranty of
%%      MERCHANTABILITY or FITNESS FOR A PARTICULAR PURPOSE.
%%      See the Libre Silicon Public License for more details.
%%
%%  ///////////////////////////////////////////////////////////////////
\begin{circuitdiagram}[draft]{25}{22}

    \usgate
    % ----  1st column  ----
    \pin{1}{1}{L}{A}
    \pin{1}{5}{L}{A1}
    \gate[\inputs{2}]{and}{5}{3}{R}{}{}

    \pin{1}{7}{L}{B}
    \pin{1}{11}{L}{B1}
    \gate[\inputs{2}]{and}{5}{9}{R}{}{}

    % ----  2nd column  ----
    \pin{8}{13}{L}{C}
    \wire{9}{3}{9}{7}
    \wire{9}{11}{9}{13}
    \gate[\inputs{3}]{or}{12}{9}{R}{}{}

    \pin{8}{15}{L}{D}
    \pin{8}{17}{L}{D1}
    \pin{8}{19}{L}{D2}
    \gate[\inputs{3}]{or}{12}{17}{R}{}{}

    % ----  3rd column  ----
    \wire{16}{9}{16}{15}
    \pin{15}{21}{L}{E}
    \wire{16}{19}{16}{21}
    \gate[\inputs{3}]{nand}{19}{17}{R}{}{}

    % ----  result ----
    \pin{24}{17}{R}{Y}

\end{circuitdiagram}
 %%  ************    LibreSilicon's StdCellLibrary   *******************
%%
%%  Organisation:   Chipforge
%%                  Germany / European Union
%%
%%  Profile:        Chipforge focus on fine System-on-Chip Cores in
%%                  Verilog HDL Code which are easy understandable and
%%                  adjustable. For further information see
%%                          www.chipforge.org
%%                  there are projects from small cores up to PCBs, too.
%%
%%  File:           StdCellLib/Documents/Datasheets/Circuitry/AAOOA22131.tex
%%
%%  Purpose:        Circuit File for AAOOA22131
%%
%%  ************    LaTeX with circdia.sty package      ***************
%%
%%  ///////////////////////////////////////////////////////////////////
%%
%%  Copyright (c) 2018 - 2022 by
%%                  chipforge <stdcelllib@nospam.chipforge.org>
%%  All rights reserved.
%%
%%      This Standard Cell Library is licensed under the Libre Silicon
%%      public license; you can redistribute it and/or modify it under
%%      the terms of the Libre Silicon public license as published by
%%      the Libre Silicon alliance, either version 1 of the License, or
%%      (at your option) any later version.
%%
%%      This design is distributed in the hope that it will be useful,
%%      but WITHOUT ANY WARRANTY; without even the implied warranty of
%%      MERCHANTABILITY or FITNESS FOR A PARTICULAR PURPOSE.
%%      See the Libre Silicon Public License for more details.
%%
%%  ///////////////////////////////////////////////////////////////////
\begin{circuitdiagram}[draft]{31}{22}

    \usgate
    % ----  1st column  ----
    \pin{1}{1}{L}{A}
    \pin{1}{5}{L}{A1}
    \gate[\inputs{2}]{and}{5}{3}{R}{}{}

    \pin{1}{7}{L}{B}
    \pin{1}{11}{L}{B1}
    \gate[\inputs{2}]{and}{5}{9}{R}{}{}

    % ----  2nd column  ----
    \pin{8}{13}{L}{C}
    \wire{9}{3}{9}{7}
    \wire{9}{11}{9}{13}
    \gate[\inputs{3}]{or}{12}{9}{R}{}{}

    \pin{8}{15}{L}{D}
    \pin{8}{17}{L}{D1}
    \pin{8}{19}{L}{D2}
    \gate[\inputs{3}]{or}{12}{17}{R}{}{}

    % ----  3rd column  ----
    \wire{16}{9}{16}{15}
    \pin{15}{21}{L}{E}
    \wire{16}{19}{16}{21}
    \gate[\inputs{3}]{nand}{19}{17}{R}{}{}

    % ----  4th column  ----
    \gate{not}{26}{17}{R}{}{}

    % ----  result ----
    \pin{30}{17}{R}{Z}

\end{circuitdiagram}

%%  ************    LibreSilicon's StdCellLibrary   *******************
%%
%%  Organisation:   Chipforge
%%                  Germany / European Union
%%
%%  Profile:        Chipforge focus on fine System-on-Chip Cores in
%%                  Verilog HDL Code which are easy understandable and
%%                  adjustable. For further information see
%%                          www.chipforge.org
%%                  there are projects from small cores up to PCBs, too.
%%
%%  File:           StdCellLib/Documents/Datasheets/Circuitry/AAOOAI22141.tex
%%
%%  Purpose:        Circuit File for AAOOAI22141
%%
%%  ************    LaTeX with circdia.sty package      ***************
%%
%%  ///////////////////////////////////////////////////////////////////
%%
%%  Copyright (c) 2018 - 2022 by
%%                  chipforge <stdcelllib@nospam.chipforge.org>
%%  All rights reserved.
%%
%%      This Standard Cell Library is licensed under the Libre Silicon
%%      public license; you can redistribute it and/or modify it under
%%      the terms of the Libre Silicon public license as published by
%%      the Libre Silicon alliance, either version 1 of the License, or
%%      (at your option) any later version.
%%
%%      This design is distributed in the hope that it will be useful,
%%      but WITHOUT ANY WARRANTY; without even the implied warranty of
%%      MERCHANTABILITY or FITNESS FOR A PARTICULAR PURPOSE.
%%      See the Libre Silicon Public License for more details.
%%
%%  ///////////////////////////////////////////////////////////////////
\begin{circuitdiagram}[draft]{25}{23}

    \usgate
    % ----  1st column  ----
    \pin{1}{1}{L}{A}
    \pin{1}{5}{L}{A1}
    \gate[\inputs{2}]{and}{5}{3}{R}{}{}

    \pin{1}{7}{L}{B}
    \pin{1}{11}{L}{B1}
    \gate[\inputs{2}]{and}{5}{9}{R}{}{}

    % ----  2nd column  ----
    \pin{8}{13}{L}{C}
    \wire{9}{3}{9}{7}
    \wire{9}{11}{9}{13}
    \gate[\inputs{3}]{or}{12}{9}{R}{}{}

    \pin{8}{15}{L}{D}
    \pin{8}{17}{L}{D1}
    \pin{8}{19}{L}{D2}
    \pin{8}{21}{L}{D3}
    \gate[\inputs{4}]{or}{12}{18}{R}{}{}

    % ----  3rd column  ----
    \wire{16}{9}{16}{16}
    \pin{15}{22}{L}{E}
    \wire{16}{20}{16}{22}
    \gate[\inputs{3}]{nand}{19}{18}{R}{}{}

    % ----  result ----
    \pin{24}{18}{R}{Y}

\end{circuitdiagram}
 %%  ************    LibreSilicon's StdCellLibrary   *******************
%%
%%  Organisation:   Chipforge
%%                  Germany / European Union
%%
%%  Profile:        Chipforge focus on fine System-on-Chip Cores in
%%                  Verilog HDL Code which are easy understandable and
%%                  adjustable. For further information see
%%                          www.chipforge.org
%%                  there are projects from small cores up to PCBs, too.
%%
%%  File:           StdCellLib/Documents/Datasheets/Circuitry/AAOOA22141.tex
%%
%%  Purpose:        Circuit File for AAOOA22141
%%
%%  ************    LaTeX with circdia.sty package      ***************
%%
%%  ///////////////////////////////////////////////////////////////////
%%
%%  Copyright (c) 2018 - 2022 by
%%                  chipforge <stdcelllib@nospam.chipforge.org>
%%  All rights reserved.
%%
%%      This Standard Cell Library is licensed under the Libre Silicon
%%      public license; you can redistribute it and/or modify it under
%%      the terms of the Libre Silicon public license as published by
%%      the Libre Silicon alliance, either version 1 of the License, or
%%      (at your option) any later version.
%%
%%      This design is distributed in the hope that it will be useful,
%%      but WITHOUT ANY WARRANTY; without even the implied warranty of
%%      MERCHANTABILITY or FITNESS FOR A PARTICULAR PURPOSE.
%%      See the Libre Silicon Public License for more details.
%%
%%  ///////////////////////////////////////////////////////////////////
\begin{circuitdiagram}[draft]{31}{23}

    \usgate
    % ----  1st column  ----
    \pin{1}{1}{L}{A}
    \pin{1}{5}{L}{A1}
    \gate[\inputs{2}]{and}{5}{3}{R}{}{}

    \pin{1}{7}{L}{B}
    \pin{1}{11}{L}{B1}
    \gate[\inputs{2}]{and}{5}{9}{R}{}{}

    % ----  2nd column  ----
    \pin{8}{13}{L}{C}
    \wire{9}{3}{9}{7}
    \wire{9}{11}{9}{13}
    \gate[\inputs{3}]{or}{12}{9}{R}{}{}

    \pin{8}{15}{L}{D}
    \pin{8}{17}{L}{D1}
    \pin{8}{19}{L}{D2}
    \pin{8}{21}{L}{D3}
    \gate[\inputs{4}]{or}{12}{18}{R}{}{}

    % ----  3rd column  ----
    \wire{16}{9}{16}{16}
    \pin{15}{22}{L}{E}
    \wire{16}{20}{16}{22}
    \gate[\inputs{3}]{nand}{19}{18}{R}{}{}

    % ----  last column ----
    \gate{not}{26}{18}{R}{}{}

    % ----  result ----
    \pin{30}{18}{R}{Z}

\end{circuitdiagram}

%%  ************    LibreSilicon's StdCellLibrary   *******************
%%
%%  Organisation:   Chipforge
%%                  Germany / European Union
%%
%%  Profile:        Chipforge focus on fine System-on-Chip Cores in
%%                  Verilog HDL Code which are easy understandable and
%%                  adjustable. For further information see
%%                          www.chipforge.org
%%                  there are projects from small cores up to PCBs, too.
%%
%%  File:           StdCellLib/Documents/Datasheets/Circuitry/AAOOAI22221.tex
%%
%%  Purpose:        Circuit File for AAOOAI22221
%%
%%  ************    LaTeX with circdia.sty package      ***************
%%
%%  ///////////////////////////////////////////////////////////////////
%%
%%  Copyright (c) 2018 - 2022 by
%%                  chipforge <stdcelllib@nospam.chipforge.org>
%%  All rights reserved.
%%
%%      This Standard Cell Library is licensed under the Libre Silicon
%%      public license; you can redistribute it and/or modify it under
%%      the terms of the Libre Silicon public license as published by
%%      the Libre Silicon alliance, either version 1 of the License, or
%%      (at your option) any later version.
%%
%%      This design is distributed in the hope that it will be useful,
%%      but WITHOUT ANY WARRANTY; without even the implied warranty of
%%      MERCHANTABILITY or FITNESS FOR A PARTICULAR PURPOSE.
%%      See the Libre Silicon Public License for more details.
%%
%%  ///////////////////////////////////////////////////////////////////
\begin{circuitdiagram}[draft]{27}{22}

    \usgate
    % ----  1st column  ----
    \pin{1}{1}{L}{A}
    \pin{1}{5}{L}{A1}
    \gate[\inputs{2}]{and}{5}{3}{R}{}{}

    \pin{1}{7}{L}{B}
    \pin{1}{11}{L}{B1}
    \gate[\inputs{2}]{and}{5}{9}{R}{}{}

    % ----  2nd column  ----
    \wire{9}{3}{9}{7}
    \wire{9}{7}{11}{7}
    \wire{9}{9}{11}{9}
    \pin{10}{11}{L}{C}
    \pin{10}{13}{L}{C1}
    \gate[\inputs{4}]{or}{14}{10}{R}{}{}

    \pin{10}{15}{L}{D}
    \pin{10}{19}{L}{D1}
    \gate[\inputs{2}]{or}{14}{17}{R}{}{}

    % ----  3rd column  ----
    \wire{18}{10}{18}{15}
    \pin{17}{21}{L}{E}
    \wire{18}{19}{18}{21}
    \gate[\inputs{3}]{nand}{21}{17}{R}{}{}

    % ----  result ----
    \pin{26}{17}{R}{Y}

\end{circuitdiagram}
 %%  ************    LibreSilicon's StdCellLibrary   *******************
%%
%%  Organisation:   Chipforge
%%                  Germany / European Union
%%
%%  Profile:        Chipforge focus on fine System-on-Chip Cores in
%%                  Verilog HDL Code which are easy understandable and
%%                  adjustable. For further information see
%%                          www.chipforge.org
%%                  there are projects from small cores up to PCBs, too.
%%
%%  File:           StdCellLib/Documents/Datasheets/Circuitry/AAOOA22221.tex
%%
%%  Purpose:        Circuit File for AAOOA2221
%%
%%  ************    LaTeX with circdia.sty package      ***************
%%
%%  ///////////////////////////////////////////////////////////////////
%%
%%  Copyright (c) 2018 - 2022 by
%%                  chipforge <stdcelllib@nospam.chipforge.org>
%%  All rights reserved.
%%
%%      This Standard Cell Library is licensed under the Libre Silicon
%%      public license; you can redistribute it and/or modify it under
%%      the terms of the Libre Silicon public license as published by
%%      the Libre Silicon alliance, either version 1 of the License, or
%%      (at your option) any later version.
%%
%%      This design is distributed in the hope that it will be useful,
%%      but WITHOUT ANY WARRANTY; without even the implied warranty of
%%      MERCHANTABILITY or FITNESS FOR A PARTICULAR PURPOSE.
%%      See the Libre Silicon Public License for more details.
%%
%%  ///////////////////////////////////////////////////////////////////
\begin{circuitdiagram}[draft]{33}{22}

    \usgate
    % ----  1st column  ----
    \pin{1}{1}{L}{A}
    \pin{1}{5}{L}{A1}
    \gate[\inputs{2}]{and}{5}{3}{R}{}{}

    \pin{1}{7}{L}{B}
    \pin{1}{11}{L}{B1}
    \gate[\inputs{2}]{and}{5}{9}{R}{}{}

    % ----  2nd column  ----
    \wire{9}{3}{9}{7}
    \wire{9}{7}{11}{7}
    \wire{9}{9}{11}{9}
    \pin{10}{11}{L}{C}
    \pin{10}{13}{L}{C1}
    \gate[\inputs{4}]{or}{14}{10}{R}{}{}

    \pin{10}{15}{L}{D}
    \pin{10}{19}{L}{D1}
    \gate[\inputs{2}]{or}{14}{17}{R}{}{}

    % ----  3rd column  ----
    \wire{18}{10}{18}{15}
    \pin{17}{21}{L}{E}
    \wire{18}{19}{18}{21}
    \gate[\inputs{3}]{nand}{21}{17}{R}{}{}

    % ----  last column ----
    \gate{not}{28}{17}{R}{}{}

    % ----  result ----
    \pin{32}{17}{R}{Z}

\end{circuitdiagram}


%%  ************    LibreSilicon's StdCellLibrary   *******************
%%
%%  Organisation:   Chipforge
%%                  Germany / European Union
%%
%%  Profile:        Chipforge focus on fine System-on-Chip Cores in
%%                  Verilog HDL Code which are easy understandable and
%%                  adjustable. For further information see
%%                          www.chipforge.org
%%                  there are projects from small cores up to PCBs, too.
%%
%%  File:           StdCellLib/Documents/Book/section-OOAAOI_complex.tex
%%
%%  Purpose:        Section Level File for Standard Cell Library Documentation
%%
%%  ************    LaTeX with circdia.sty package      ***************
%%
%%  ///////////////////////////////////////////////////////////////////
%%
%%  Copyright (c) 2018 - 2022 by
%%                  chipforge <stdcelllib@nospam.chipforge.org>
%%  All rights reserved.
%%
%%      This Standard Cell Library is licensed under the Libre Silicon
%%      public license; you can redistribute it and/or modify it under
%%      the terms of the Libre Silicon public license as published by
%%      the Libre Silicon alliance, either version 1 of the License, or
%%      (at your option) any later version.
%%
%%      This design is distributed in the hope that it will be useful,
%%      but WITHOUT ANY WARRANTY; without even the implied warranty of
%%      MERCHANTABILITY or FITNESS FOR A PARTICULAR PURPOSE.
%%      See the Libre Silicon Public License for more details.
%%
%%  ///////////////////////////////////////////////////////////////////
\section{OR-OR-AND-AND-OR(-Invert) Complex Gates}

%%  ************    LibreSilicon's StdCellLibrary   *******************
%%
%%  Organisation:   Chipforge
%%                  Germany / European Union
%%
%%  Profile:        Chipforge focus on fine System-on-Chip Cores in
%%                  Verilog HDL Code which are easy understandable and
%%                  adjustable. For further information see
%%                          www.chipforge.org
%%                  there are projects from small cores up to PCBs, too.
%%
%%  File:           StdCellLib/Documents/Circuits/OOAAOI222.tex
%%
%%  Purpose:        Circuit File for OOAAOI222
%%
%%  ************    LaTeX with circdia.sty package      ***************
%%
%%  ///////////////////////////////////////////////////////////////////
%%
%%  Copyright (c) 2019 by chipforge <stdcelllib@nospam.chipforge.org>
%%  All rights reserved.
%%
%%      This Standard Cell Library is licensed under the Libre Silicon
%%      public license; you can redistribute it and/or modify it under
%%      the terms of the Libre Silicon public license as published by
%%      the Libre Silicon alliance, either version 1 of the License, or
%%      (at your option) any later version.
%%
%%      This design is distributed in the hope that it will be useful,
%%      but WITHOUT ANY WARRANTY; without even the implied warranty of
%%      MERCHANTABILITY or FITNESS FOR A PARTICULAR PURPOSE.
%%      See the Libre Silicon Public License for more details.
%%
%%  ///////////////////////////////////////////////////////////////////
\begin{center}
    Circuit
    \begin{figure}[h]
        \begin{center}
            \begin{circuitdiagram}{25}{18}
            \usgate
            \gate[\inputs{2}]{or}{5}{15}{R}{}{}  % OR
            \gate[\inputs{2}]{or}{5}{9}{R}{}{}   % OR
            \gate[\inputs{2}]{and}{12}{12}{R}{}{} % AND
            \gate[\inputs{2}]{and}{12}{3}{R}{}{} % AND
            \gate[\inputs{2}]{nor}{19}{8}{R}{}{} % NOR
            \pin{1}{1}{L}{A}     % pin A
            \pin{1}{5}{L}{A1}    % pin A1
            \pin{1}{7}{L}{B}     % pin B
            \pin{1}{11}{L}{B1}    % pin B1
            \pin{1}{13}{L}{C}    % pin C
            \pin{1}{17}{L}{C1}   % pin C1
            \wire{2}{1}{9}{1}    % wire pin A
            \wire{2}{5}{9}{5}    % wire pin A1
            \wire{9}{14}{9}{15}  % wire between OR and AND
            \wire{9}{9}{9}{10}   % wire between OR and AND
            \wire{16}{3}{16}{6}  % wire between AND and NOR
            \wire{16}{10}{16}{12}  % wire between AND and NOR
            \pin{24}{8}{R}{Z}    % pin Z
            \end{circuitdiagram}
        \end{center}
    \end{figure}
\end{center}
 %%  ************    LibreSilicon's StdCellLibrary   *******************
%%
%%  Organisation:   Chipforge
%%                  Germany / European Union
%%
%%  Profile:        Chipforge focus on fine System-on-Chip Cores in
%%                  Verilog HDL Code which are easy understandable and
%%                  adjustable. For further information see
%%                          www.chipforge.org
%%                  there are projects from small cores up to PCBs, too.
%%
%%  File:           StdCellLib/Documents/Circuits/OOAAO222.tex
%%
%%  Purpose:        Circuit File for OOAAO222
%%
%%  ************    LaTeX with circdia.sty package      ***************
%%
%%  ///////////////////////////////////////////////////////////////////
%%
%%  Copyright (c) 2019 by chipforge <stdcelllib@nospam.chipforge.org>
%%  All rights reserved.
%%
%%      This Standard Cell Library is licensed under the Libre Silicon
%%      public license; you can redistribute it and/or modify it under
%%      the terms of the Libre Silicon public license as published by
%%      the Libre Silicon alliance, either version 1 of the License, or
%%      (at your option) any later version.
%%
%%      This design is distributed in the hope that it will be useful,
%%      but WITHOUT ANY WARRANTY; without even the implied warranty of
%%      MERCHANTABILITY or FITNESS FOR A PARTICULAR PURPOSE.
%%      See the Libre Silicon Public License for more details.
%%
%%  ///////////////////////////////////////////////////////////////////
\begin{center}
    Circuit
    \begin{figure}[h]
        \begin{center}
            \begin{circuitdiagram}{31}{18}
            \usgate
            \gate[\inputs{2}]{or}{5}{15}{R}{}{}  % OR
            \gate[\inputs{2}]{or}{5}{9}{R}{}{}   % OR
            \gate[\inputs{2}]{and}{12}{12}{R}{}{} % AND
            \gate[\inputs{2}]{and}{12}{3}{R}{}{} % AND
            \gate[\inputs{2}]{nor}{19}{8}{R}{}{} % NOR
            \gate{not}{26}{8}{R}{}{} % NOT
            \pin{1}{1}{L}{A}     % pin A
            \pin{1}{5}{L}{A1}    % pin A1
            \pin{1}{7}{L}{B}     % pin B
            \pin{1}{11}{L}{B1}    % pin B1
            \pin{1}{13}{L}{C}    % pin C
            \pin{1}{17}{L}{C1}   % pin C1
            \wire{2}{1}{9}{1}    % wire pin A
            \wire{2}{5}{9}{5}    % wire pin A1
            \wire{9}{14}{9}{15}  % wire between OR and AND
            \wire{9}{9}{9}{10}   % wire between OR and AND
            \wire{16}{3}{16}{6}  % wire between AND and NOR
            \wire{16}{10}{16}{12}  % wire between AND and NOR
            \pin{30}{8}{R}{Z}    % pin Z
            \end{circuitdiagram}
        \end{center}
    \end{figure}
\end{center}


%%  ************    LibreSilicon's StdCellLibrary   *******************
%%
%%  Organisation:   Chipforge
%%                  Germany / European Union
%%
%%  Profile:        Chipforge focus on fine System-on-Chip Cores in
%%                  Verilog HDL Code which are easy understandable and
%%                  adjustable. For further information see
%%                          www.chipforge.org
%%                  there are projects from small cores up to PCBs, too.
%%
%%  File:           StdCellLib/Documents/Book/section-AOOOAI_complex.tex
%%
%%  Purpose:        Section Level File for Standard Cell Library Documentation
%%
%%  ************    LaTeX with circdia.sty package      ***************
%%
%%  ///////////////////////////////////////////////////////////////////
%%
%%  Copyright (c) 2018 - 2022 by
%%                  chipforge <stdcelllib@nospam.chipforge.org>
%%  All rights reserved.
%%
%%      This Standard Cell Library is licensed under the Libre Silicon
%%      public license; you can redistribute it and/or modify it under
%%      the terms of the Libre Silicon public license as published by
%%      the Libre Silicon alliance, either version 1 of the License, or
%%      (at your option) any later version.
%%
%%      This design is distributed in the hope that it will be useful,
%%      but WITHOUT ANY WARRANTY; without even the implied warranty of
%%      MERCHANTABILITY or FITNESS FOR A PARTICULAR PURPOSE.
%%      See the Libre Silicon Public License for more details.
%%
%%  ///////////////////////////////////////////////////////////////////
\section{AND-OR-OR-OR-AND(-Invert) Complex Gates}


%%  ************    LibreSilicon's StdCellLibrary   *******************
%%
%%  Organisation:   Chipforge
%%                  Germany / European Union
%%
%%  Profile:        Chipforge focus on fine System-on-Chip Cores in
%%                  Verilog HDL Code which are easy understandable and
%%                  adjustable. For further information see
%%                          www.chipforge.org
%%                  there are projects from small cores up to PCBs, too.
%%
%%  File:           StdCellLib/Documents/section-OAAAOI_complex.tex
%%
%%  Purpose:        Section Level File for Standard Cell Library Documentation
%%
%%  ************    LaTeX with circdia.sty package      ***************
%%
%%  ///////////////////////////////////////////////////////////////////
%%
%%  Copyright (c) 2018 - 2022 by
%%                  chipforge <stdcelllib@nospam.chipforge.org>
%%  All rights reserved.
%%
%%      This Standard Cell Library is licensed under the Libre Silicon
%%      public license; you can redistribute it and/or modify it under
%%      the terms of the Libre Silicon public license as published by
%%      the Libre Silicon alliance, either version 1 of the License, or
%%      (at your option) any later version.
%%
%%      This design is distributed in the hope that it will be useful,
%%      but WITHOUT ANY WARRANTY; without even the implied warranty of
%%      MERCHANTABILITY or FITNESS FOR A PARTICULAR PURPOSE.
%%      See the Libre Silicon Public License for more details.
%%
%%  ///////////////////////////////////////////////////////////////////
\section{OR-AND-AND-AND-OR(-Invert) Complex Gates}

%%  ************    LibreSilicon's StdCellLibrary   *******************
%%
%%  Organisation:   Chipforge
%%                  Germany / European Union
%%
%%  Profile:        Chipforge focus on fine System-on-Chip Cores in
%%                  Verilog HDL Code which are easy understandable and
%%                  adjustable. For further information see
%%                          www.chipforge.org
%%                  there are projects from small cores up to PCBs, too.
%%
%%  File:           StdCellLib/Documents/Datasheets/Circuitry/OAAAOI2122.tex
%%
%%  Purpose:        Circuit File for OAAAOI2122
%%
%%  ************    LaTeX with circdia.sty package      ***************
%%
%%  ///////////////////////////////////////////////////////////////////
%%
%%  Copyright (c) 2018 - 2022 by
%%                  chipforge <stdcelllib@nospam.chipforge.org>
%%  All rights reserved.
%%
%%      This Standard Cell Library is licensed under the Libre Silicon
%%      public license; you can redistribute it and/or modify it under
%%      the terms of the Libre Silicon public license as published by
%%      the Libre Silicon alliance, either version 1 of the License, or
%%      (at your option) any later version.
%%
%%      This design is distributed in the hope that it will be useful,
%%      but WITHOUT ANY WARRANTY; without even the implied warranty of
%%      MERCHANTABILITY or FITNESS FOR A PARTICULAR PURPOSE.
%%      See the Libre Silicon Public License for more details.
%%
%%  ///////////////////////////////////////////////////////////////////
\begin{circuitdiagram}[draft]{25}{20}

    \usgate
    % ----  1st column  ----
    \pin{1}{1}{L}{A}
    \pin{1}{5}{L}{A1}
    \gate[\inputs{2}]{or}{5}{3}{R}{}{}

    % ----  2nd column  ----
    \pin{8}{7}{L}{B}
    \gate[\inputs{2}]{and}{12}{5}{R}{}{}

    \pin{8}{9}{L}{C}
    \pin{8}{13}{L}{C1}
    \gate[\inputs{2}]{and}{12}{11}{R}{}{}

    \pin{8}{15}{L}{D}
    \pin{8}{19}{L}{D1}
    \gate[\inputs{2}]{and}{12}{17}{R}{}{}

    % ----  3rd column  ----
    \wire{16}{5}{16}{9}
    \wire{16}{13}{16}{17}
    \gate[\inputs{3}]{nor}{19}{11}{R}{}{}

    % ----  result ----
    \pin{24}{11}{R}{Y}

\end{circuitdiagram}
 %%  ************    LibreSilicon's StdCellLibrary   *******************
%%
%%  Organisation:   Chipforge
%%                  Germany / European Union
%%
%%  Profile:        Chipforge focus on fine System-on-Chip Cores in
%%                  Verilog HDL Code which are easy understandable and
%%                  adjustable. For further information see
%%                          www.chipforge.org
%%                  there are projects from small cores up to PCBs, too.
%%
%%  File:           StdCellLib/Documents/Datasheets/Circuitry/OAAAO2122.tex
%%
%%  Purpose:        Circuit File for OAAAO2122
%%
%%  ************    LaTeX with circdia.sty package      ***************
%%
%%  ///////////////////////////////////////////////////////////////////
%%
%%  Copyright (c) 2018 - 2022 by
%%                  chipforge <stdcelllib@nospam.chipforge.org>
%%  All rights reserved.
%%
%%      This Standard Cell Library is licensed under the Libre Silicon
%%      public license; you can redistribute it and/or modify it under
%%      the terms of the Libre Silicon public license as published by
%%      the Libre Silicon alliance, either version 1 of the License, or
%%      (at your option) any later version.
%%
%%      This design is distributed in the hope that it will be useful,
%%      but WITHOUT ANY WARRANTY; without even the implied warranty of
%%      MERCHANTABILITY or FITNESS FOR A PARTICULAR PURPOSE.
%%      See the Libre Silicon Public License for more details.
%%
%%  ///////////////////////////////////////////////////////////////////
\begin{circuitdiagram}[draft]{31}{20}

    \usgate
    % ----  1st column  ----
    \pin{1}{1}{L}{A}
    \pin{1}{5}{L}{A1}
    \gate[\inputs{2}]{or}{5}{3}{R}{}{}

    % ----  2nd column  ----
    \pin{8}{7}{L}{B}
    \gate[\inputs{2}]{and}{12}{5}{R}{}{}

    \pin{8}{9}{L}{C}
    \pin{8}{13}{L}{C1}
    \gate[\inputs{2}]{and}{12}{11}{R}{}{}

    \pin{8}{15}{L}{D}
    \pin{8}{19}{L}{D1}
    \gate[\inputs{2}]{and}{12}{17}{R}{}{}

    % ----  3rd column  ----
    \wire{16}{5}{16}{9}
    \wire{16}{13}{16}{17}
    \gate[\inputs{3}]{nor}{19}{11}{R}{}{}

    % ----  4th column  ----
    \gate{not}{26}{11}{R}{}{}

    % ----  result ----
    \pin{30}{11}{R}{Z}

\end{circuitdiagram}

%%  ************    LibreSilicon's StdCellLibrary   *******************
%%
%%  Organisation:   Chipforge
%%                  Germany / European Union
%%
%%  Profile:        Chipforge focus on fine System-on-Chip Cores in
%%                  Verilog HDL Code which are easy understandable and
%%                  adjustable. For further information see
%%                          www.chipforge.org
%%                  there are projects from small cores up to PCBs, too.
%%
%%  File:           StdCellLib/Documents/Datasheets/Circuitry/OAAAOI2132.tex
%%
%%  Purpose:        Circuit File for OAAAOI2132
%%
%%  ************    LaTeX with circdia.sty package      ***************
%%
%%  ///////////////////////////////////////////////////////////////////
%%
%%  Copyright (c) 2018 - 2022 by
%%                  chipforge <stdcelllib@nospam.chipforge.org>
%%  All rights reserved.
%%
%%      This Standard Cell Library is licensed under the Libre Silicon
%%      public license; you can redistribute it and/or modify it under
%%      the terms of the Libre Silicon public license as published by
%%      the Libre Silicon alliance, either version 1 of the License, or
%%      (at your option) any later version.
%%
%%      This design is distributed in the hope that it will be useful,
%%      but WITHOUT ANY WARRANTY; without even the implied warranty of
%%      MERCHANTABILITY or FITNESS FOR A PARTICULAR PURPOSE.
%%      See the Libre Silicon Public License for more details.
%%
%%  ///////////////////////////////////////////////////////////////////
\begin{circuitdiagram}[draft]{25}{20}

    \usgate
    % ----  1st column  ----
    \pin{1}{1}{L}{A}
    \pin{1}{5}{L}{A1}
    \gate[\inputs{2}]{or}{5}{3}{R}{}{}

    % ----  2nd column  ----
    \pin{8}{7}{L}{B}
    \gate[\inputs{2}]{and}{12}{5}{R}{}{}

    \pin{8}{9}{L}{C}
    \pin{8}{11}{L}{C1}
    \pin{8}{13}{L}{C2}
    \gate[\inputs{3}]{and}{12}{11}{R}{}{}

    \pin{8}{15}{L}{D}
    \pin{8}{19}{L}{D1}
    \gate[\inputs{2}]{and}{12}{17}{R}{}{}

    % ----  3rd column  ----
    \wire{16}{5}{16}{9}
    \wire{16}{13}{16}{17}
    \gate[\inputs{3}]{nor}{19}{11}{R}{}{}

    % ----  result ----
    \pin{24}{11}{R}{Y}

\end{circuitdiagram}
 %%  ************    LibreSilicon's StdCellLibrary   *******************
%%
%%  Organisation:   Chipforge
%%                  Germany / European Union
%%
%%  Profile:        Chipforge focus on fine System-on-Chip Cores in
%%                  Verilog HDL Code which are easy understandable and
%%                  adjustable. For further information see
%%                          www.chipforge.org
%%                  there are projects from small cores up to PCBs, too.
%%
%%  File:           StdCellLib/Documents/Datasheets/Circuitry/OAAAO2132.tex
%%
%%  Purpose:        Circuit File for OAAAO2132
%%
%%  ************    LaTeX with circdia.sty package      ***************
%%
%%  ///////////////////////////////////////////////////////////////////
%%
%%  Copyright (c) 2018 - 2022 by
%%                  chipforge <stdcelllib@nospam.chipforge.org>
%%  All rights reserved.
%%
%%      This Standard Cell Library is licensed under the Libre Silicon
%%      public license; you can redistribute it and/or modify it under
%%      the terms of the Libre Silicon public license as published by
%%      the Libre Silicon alliance, either version 1 of the License, or
%%      (at your option) any later version.
%%
%%      This design is distributed in the hope that it will be useful,
%%      but WITHOUT ANY WARRANTY; without even the implied warranty of
%%      MERCHANTABILITY or FITNESS FOR A PARTICULAR PURPOSE.
%%      See the Libre Silicon Public License for more details.
%%
%%  ///////////////////////////////////////////////////////////////////
\begin{circuitdiagram}[draft]{31}{20}

    \usgate
    % ----  1st column  ----
    \pin{1}{1}{L}{A}
    \pin{1}{5}{L}{A1}
    \gate[\inputs{2}]{or}{5}{3}{R}{}{}

    % ----  2nd column  ----
    \pin{8}{7}{L}{B}
    \gate[\inputs{2}]{and}{12}{5}{R}{}{}

    \pin{8}{9}{L}{C}
    \pin{8}{11}{L}{C1}
    \pin{8}{13}{L}{C2}
    \gate[\inputs{3}]{and}{12}{11}{R}{}{}

    \pin{8}{15}{L}{D}
    \pin{8}{19}{L}{D1}
    \gate[\inputs{2}]{and}{12}{17}{R}{}{}

    % ----  3rd column  ----
    \wire{16}{5}{16}{9}
    \wire{16}{13}{16}{17}
    \gate[\inputs{3}]{nor}{19}{11}{R}{}{}

    % ----  4th column  ----
    \gate{not}{26}{11}{R}{}{}

    % ----  result ----
    \pin{30}{11}{R}{Z}

\end{circuitdiagram}

\include{Datasheets/OAAAOI2133} \include{Datasheets/OAAAO2133}
\include{Datasheets/OAAAOI2142} \include{Datasheets/OAAAO2142}
%%  ************    LibreSilicon's StdCellLibrary   *******************
%%
%%  Organisation:   Chipforge
%%                  Germany / European Union
%%
%%  Profile:        Chipforge focus on fine System-on-Chip Cores in
%%                  Verilog HDL Code which are easy understandable and
%%                  adjustable. For further information see
%%                          www.chipforge.org
%%                  there are projects from small cores up to PCBs, too.
%%
%%  File:           StdCellLib/Documents/Datasheets/Circuitry/OAAAOI2222.tex
%%
%%  Purpose:        Circuit File for OAAAOI2222
%%
%%  ************    LaTeX with circdia.sty package      ***************
%%
%%  ///////////////////////////////////////////////////////////////////
%%
%%  Copyright (c) 2018 - 2022 by
%%                  chipforge <stdcelllib@nospam.chipforge.org>
%%  All rights reserved.
%%
%%      This Standard Cell Library is licensed under the Libre Silicon
%%      public license; you can redistribute it and/or modify it under
%%      the terms of the Libre Silicon public license as published by
%%      the Libre Silicon alliance, either version 1 of the License, or
%%      (at your option) any later version.
%%
%%      This design is distributed in the hope that it will be useful,
%%      but WITHOUT ANY WARRANTY; without even the implied warranty of
%%      MERCHANTABILITY or FITNESS FOR A PARTICULAR PURPOSE.
%%      See the Libre Silicon Public License for more details.
%%
%%  ///////////////////////////////////////////////////////////////////
\begin{circuitdiagram}[draft]{25}{22}

    \usgate
    % ----  1st column  ----
    \pin{1}{1}{L}{A}
    \pin{1}{5}{L}{A1}
    \gate[\inputs{2}]{or}{5}{3}{R}{}{}

    % ----  2nd column  ----
    \wire{9}{3}{9}{5}
    \pin{8}{7}{L}{B}
    \pin{8}{9}{L}{B1}
    \gate[\inputs{3}]{and}{12}{7}{R}{}{}

    \pin{8}{11}{L}{C}
    \pin{8}{15}{L}{C1}
    \gate[\inputs{2}]{and}{12}{13}{R}{}{}

    \pin{8}{17}{L}{D}
    \pin{8}{21}{L}{D1}
    \gate[\inputs{2}]{and}{12}{19}{R}{}{}

    % ----  3rd column  ----
    \wire{16}{7}{16}{11}
    \wire{16}{15}{16}{19}
    \gate[\inputs{3}]{nor}{19}{13}{R}{}{}

    % ----  result ----
    \pin{24}{13}{R}{Y}

\end{circuitdiagram}
 %%  ************    LibreSilicon's StdCellLibrary   *******************
%%
%%  Organisation:   Chipforge
%%                  Germany / European Union
%%
%%  Profile:        Chipforge focus on fine System-on-Chip Cores in
%%                  Verilog HDL Code which are easy understandable and
%%                  adjustable. For further information see
%%                          www.chipforge.org
%%                  there are projects from small cores up to PCBs, too.
%%
%%  File:           StdCellLib/Documents/Datasheets/Circuitry/OAAAO2222.tex
%%
%%  Purpose:        Circuit File for OAAAO2222
%%
%%  ************    LaTeX with circdia.sty package      ***************
%%
%%  ///////////////////////////////////////////////////////////////////
%%
%%  Copyright (c) 2018 - 2022 by
%%                  chipforge <stdcelllib@nospam.chipforge.org>
%%  All rights reserved.
%%
%%      This Standard Cell Library is licensed under the Libre Silicon
%%      public license; you can redistribute it and/or modify it under
%%      the terms of the Libre Silicon public license as published by
%%      the Libre Silicon alliance, either version 1 of the License, or
%%      (at your option) any later version.
%%
%%      This design is distributed in the hope that it will be useful,
%%      but WITHOUT ANY WARRANTY; without even the implied warranty of
%%      MERCHANTABILITY or FITNESS FOR A PARTICULAR PURPOSE.
%%      See the Libre Silicon Public License for more details.
%%
%%  ///////////////////////////////////////////////////////////////////
\begin{circuitdiagram}[draft]{31}{22}

    \usgate
    % ----  1st column  ----
    \pin{1}{1}{L}{A}
    \pin{1}{5}{L}{A1}
    \gate[\inputs{2}]{or}{5}{3}{R}{}{}

    % ----  2nd column  ----
    \wire{9}{3}{9}{5}
    \pin{8}{7}{L}{B}
    \pin{8}{9}{L}{B1}
    \gate[\inputs{3}]{and}{12}{7}{R}{}{}

    \pin{8}{11}{L}{C}
    \pin{8}{15}{L}{C1}
    \gate[\inputs{2}]{and}{12}{13}{R}{}{}

    \pin{8}{17}{L}{D}
    \pin{8}{21}{L}{D1}
    \gate[\inputs{2}]{and}{12}{19}{R}{}{}

    % ----  3rd column  ----
    \wire{16}{7}{16}{11}
    \wire{16}{15}{16}{19}
    \gate[\inputs{3}]{nor}{19}{13}{R}{}{}

    % ----  last column ----
    \gate{not}{26}{13}{R}{}{}

    % ----  result ----
    \pin{30}{13}{R}{Z}

\end{circuitdiagram}

\include{Datasheets/OAAAOI2223} \include{Datasheets/OAAAO2223}
%%  ************    LibreSilicon's StdCellLibrary   *******************
%%
%%  Organisation:   Chipforge
%%                  Germany / European Union
%%
%%  Profile:        Chipforge focus on fine System-on-Chip Cores in
%%                  Verilog HDL Code which are easy understandable and
%%                  adjustable. For further information see
%%                          www.chipforge.org
%%                  there are projects from small cores up to PCBs, too.
%%
%%  File:           StdCellLib/Documents/Datasheets/Circuitry/OAAAOI2232.tex
%%
%%  Purpose:        Circuit File for OAAAOI2232
%%
%%  ************    LaTeX with circdia.sty package      ***************
%%
%%  ///////////////////////////////////////////////////////////////////
%%
%%  Copyright (c) 2018 - 2022 by
%%                  chipforge <stdcelllib@nospam.chipforge.org>
%%  All rights reserved.
%%
%%      This Standard Cell Library is licensed under the Libre Silicon
%%      public license; you can redistribute it and/or modify it under
%%      the terms of the Libre Silicon public license as published by
%%      the Libre Silicon alliance, either version 1 of the License, or
%%      (at your option) any later version.
%%
%%      This design is distributed in the hope that it will be useful,
%%      but WITHOUT ANY WARRANTY; without even the implied warranty of
%%      MERCHANTABILITY or FITNESS FOR A PARTICULAR PURPOSE.
%%      See the Libre Silicon Public License for more details.
%%
%%  ///////////////////////////////////////////////////////////////////
\begin{circuitdiagram}[draft]{25}{22}

    \usgate
    % ----  1st column  ----
    \pin{1}{1}{L}{A}
    \pin{1}{5}{L}{A1}
    \gate[\inputs{2}]{or}{5}{3}{R}{}{}

    % ----  2nd column  ----
    \wire{9}{3}{9}{5}
    \pin{8}{7}{L}{B}
    \pin{8}{9}{L}{B1}
    \gate[\inputs{3}]{and}{12}{7}{R}{}{}

    \pin{8}{11}{L}{C}
    \pin{8}{13}{L}{C1}
    \pin{8}{15}{L}{C2}
    \gate[\inputs{3}]{and}{12}{13}{R}{}{}

    \pin{8}{17}{L}{D}
    \pin{8}{21}{L}{D1}
    \gate[\inputs{2}]{and}{12}{19}{R}{}{}

    % ----  3rd column  ----
    \wire{16}{7}{16}{11}
    \wire{16}{15}{16}{19}
    \gate[\inputs{3}]{nor}{19}{13}{R}{}{}

    % ----  result ----
    \pin{24}{13}{R}{Y}

\end{circuitdiagram}
 %%  ************    LibreSilicon's StdCellLibrary   *******************
%%
%%  Organisation:   Chipforge
%%                  Germany / European Union
%%
%%  Profile:        Chipforge focus on fine System-on-Chip Cores in
%%                  Verilog HDL Code which are easy understandable and
%%                  adjustable. For further information see
%%                          www.chipforge.org
%%                  there are projects from small cores up to PCBs, too.
%%
%%  File:           StdCellLib/Documents/Circuits/OAAAO2232.tex
%%
%%  Purpose:        Circuit File for OAAAO2232
%%
%%  ************    LaTeX with circdia.sty package      ***************
%%
%%  ///////////////////////////////////////////////////////////////////
%%
%%  Copyright (c) 2019 by chipforge <stdcelllib@nospam.chipforge.org>
%%  All rights reserved.
%%
%%      This Standard Cell Library is licensed under the Libre Silicon
%%      public license; you can redistribute it and/or modify it under
%%      the terms of the Libre Silicon public license as published by
%%      the Libre Silicon alliance, either version 1 of the License, or
%%      (at your option) any later version.
%%
%%      This design is distributed in the hope that it will be useful,
%%      but WITHOUT ANY WARRANTY; without even the implied warranty of
%%      MERCHANTABILITY or FITNESS FOR A PARTICULAR PURPOSE.
%%      See the Libre Silicon Public License for more details.
%%
%%  ///////////////////////////////////////////////////////////////////
\begin{center}
    Circuit
    \begin{figure}[h]
        \begin{center}
            \begin{circuitdiagram}{31}{22}
            \usgate
            \gate[\inputs{2}]{or}{5}{19}{R}{}{}    % OR
            \gate[\inputs{2}]{and}{12}{3}{R}{}{}   % AND
            \gate[\inputs{3}]{and}{12}{9}{R}{}{}   % AND
            \gate[\inputs{3}]{and}{12}{15}{R}{}{}  % AND
            \gate[\inputs{3}]{nor}{19}{9}{R}{}{}   % NOR
            \gate{not}{26}{9}{R}{}{}   % NOT
            \pin{1}{1}{L}{A}     % pin A
            \wire{2}{1}{9}{1}    % wire A
            \pin{1}{5}{L}{A1}    % pin A1
            \wire{2}{5}{9}{5}    % wire A1
            \pin{1}{7}{L}{B}     % pin B
            \wire{2}{7}{9}{7}    % wire B
            \pin{1}{9}{L}{B1}    % pin B1
            \wire{2}{9}{9}{9}    % wire B1
            \pin{1}{11}{L}{B2}   % pin B2
            \wire{2}{11}{9}{11}  % wire B2
            \pin{1}{13}{L}{C}    % pin C
            \wire{2}{13}{9}{13}  % wire C
            \pin{1}{15}{L}{C1}   % pin C1
            \wire{2}{15}{9}{15}  % wire C1
            \pin{1}{17}{L}{D}    % pin D
            \pin{1}{21}{L}{D1}   % pin D1
            \wire{9}{17}{9}{19}  % wire between OR and AND
            \wire{16}{3}{16}{7}  % wire between AND and NOR
            \wire{16}{11}{16}{15}% wire between AND and NOR
            \pin{30}{9}{R}{Z}    % pin Z
            \end{circuitdiagram}
        \end{center}
    \end{figure}
\end{center}

\include{Datasheets/OAAAOI2322} \include{Datasheets/OAAAO2322}

%%  ************    LibreSilicon's StdCellLibrary   *******************
%%
%%  Organisation:   Chipforge
%%                  Germany / European Union
%%
%%  Profile:        Chipforge focus on fine System-on-Chip Cores in
%%                  Verilog HDL Code which are easy understandable and
%%                  adjustable. For further information see
%%                          www.chipforge.org
%%                  there are projects from small cores up to PCBs, too.
%%
%%  File:           StdCellLib/Documents/section-AAOAI_complex.tex
%%
%%  Purpose:        Section Level File for Standard Cell Library Documentation
%%
%%  ************    LaTeX with circdia.sty package      ***************
%%
%%  ///////////////////////////////////////////////////////////////////
%%
%%  Copyright (c) 2018 - 2022 by
%%                  chipforge <stdcelllib@nospam.chipforge.org>
%%  All rights reserved.
%%
%%      This Standard Cell Library is licensed under the Libre Silicon
%%      public license; you can redistribute it and/or modify it under
%%      the terms of the Libre Silicon public license as published by
%%      the Libre Silicon alliance, either version 1 of the License, or
%%      (at your option) any later version.
%%
%%      This design is distributed in the hope that it will be useful,
%%      but WITHOUT ANY WARRANTY; without even the implied warranty of
%%      MERCHANTABILITY or FITNESS FOR A PARTICULAR PURPOSE.
%%      See the Libre Silicon Public License for more details.
%%
%%  ///////////////////////////////////////////////////////////////////
\section{AND-AND-OR-AND(-Invert) Complex Gates}

%%  ************    LibreSilicon's StdCellLibrary   *******************
%%
%%  Organisation:   Chipforge
%%                  Germany / European Union
%%
%%  Profile:        Chipforge focus on fine System-on-Chip Cores in
%%                  Verilog HDL Code which are easy understandable and
%%                  adjustable. For further information see
%%                          www.chipforge.org
%%                  there are projects from small cores up to PCBs, too.
%%
%%  File:           StdCellLib/Documents/Circuits/AAOAI221.tex
%%
%%  Purpose:        Circuit File for AAOAI221
%%
%%  ************    LaTeX with circdia.sty package      ***************
%%
%%  ///////////////////////////////////////////////////////////////////
%%
%%  Copyright (c) 2019 by chipforge <stdcelllib@nospam.chipforge.org>
%%  All rights reserved.
%%
%%      This Standard Cell Library is licensed under the Libre Silicon
%%      public license; you can redistribute it and/or modify it under
%%      the terms of the Libre Silicon public license as published by
%%      the Libre Silicon alliance, either version 1 of the License, or
%%      (at your option) any later version.
%%
%%      This design is distributed in the hope that it will be useful,
%%      but WITHOUT ANY WARRANTY; without even the implied warranty of
%%      MERCHANTABILITY or FITNESS FOR A PARTICULAR PURPOSE.
%%      See the Libre Silicon Public License for more details.
%%
%%  ///////////////////////////////////////////////////////////////////
\begin{circuitdiagram}{25}{14}

    \usgate
    \gate[\inputs{2}]{and}{5}{11}{R}{}{}  % AND
    \gate[\inputs{2}]{and}{5}{5}{R}{}{}   % AND
    \gate[\inputs{2}]{or}{12}{8}{R}{}{}   % OR
    \gate[\inputs{2}]{nand}{19}{4}{R}{}{} % NAND
    \pin{1}{1}{L}{A}     % pin A
    \pin{1}{3}{L}{B}     % pin B
    \pin{1}{7}{L}{B1}    % pin B1
    \pin{1}{9}{L}{C}     % pin C
    \pin{1}{13}{L}{C1}   % pin C1
    \wire{2}{1}{16}{1}   % wire pin A
    \wire{9}{5}{9}{6}    % wire between AND and OR
    \wire{9}{10}{9}{11}  % wire between AND and OR
    \wire{16}{1}{16}{2}  % wire between OR and NAND
    \wire{16}{6}{16}{8}  % wire between OR and NAND
    \pin{24}{4}{R}{Y}    % pin Y

\end{circuitdiagram}
 %%  ************    LibreSilicon's StdCellLibrary   *******************
%%
%%  Organisation:   Chipforge
%%                  Germany / European Union
%%
%%  Profile:        Chipforge focus on fine System-on-Chip Cores in
%%                  Verilog HDL Code which are easy understandable and
%%                  adjustable. For further information see
%%                          www.chipforge.org
%%                  there are projects from small cores up to PCBs, too.
%%
%%  File:           StdCellLib/Documents/Circuits/AAOA221.tex
%%
%%  Purpose:        Circuit File for AAOA221
%%
%%  ************    LaTeX with circdia.sty package      ***************
%%
%%  ///////////////////////////////////////////////////////////////////
%%
%%  Copyright (c) 2019 by chipforge <stdcelllib@nospam.chipforge.org>
%%  All rights reserved.
%%
%%      This Standard Cell Library is licensed under the Libre Silicon
%%      public license; you can redistribute it and/or modify it under
%%      the terms of the Libre Silicon public license as published by
%%      the Libre Silicon alliance, either version 1 of the License, or
%%      (at your option) any later version.
%%
%%      This design is distributed in the hope that it will be useful,
%%      but WITHOUT ANY WARRANTY; without even the implied warranty of
%%      MERCHANTABILITY or FITNESS FOR A PARTICULAR PURPOSE.
%%      See the Libre Silicon Public License for more details.
%%
%%  ///////////////////////////////////////////////////////////////////
\begin{circuitdiagram}{31}{14}

    \usgate
    \gate[\inputs{2}]{and}{5}{11}{R}{}{}  % AND
    \gate[\inputs{2}]{and}{5}{5}{R}{}{}   % AND
    \gate[\inputs{2}]{or}{12}{8}{R}{}{}   % OR
    \gate[\inputs{2}]{nand}{19}{4}{R}{}{} % NAND
    \gate{not}{26}{4}{R}{}{} % NOT
    \pin{1}{1}{L}{A}     % pin A
    \pin{1}{3}{L}{B}     % pin B
    \pin{1}{7}{L}{B1}    % pin B1
    \pin{1}{9}{L}{C}     % pin C
    \pin{1}{13}{L}{C1}   % pin C1
    \wire{2}{1}{16}{1}   % wire pin A
    \wire{9}{5}{9}{6}    % wire between AND and OR
    \wire{9}{10}{9}{11}  % wire between AND and OR
    \wire{16}{1}{16}{2}  % wire between OR and NAND
    \wire{16}{6}{16}{8}  % wire between OR and NAND
    \pin{30}{4}{R}{Z}    % pin Z

\end{circuitdiagram}

%%  ************    LibreSilicon's StdCellLibrary   *******************
%%
%%  Organisation:   Chipforge
%%                  Germany / European Union
%%
%%  Profile:        Chipforge focus on fine System-on-Chip Cores in
%%                  Verilog HDL Code which are easy understandable and
%%                  adjustable. For further information see
%%                          www.chipforge.org
%%                  there are projects from small cores up to PCBs, too.
%%
%%  File:           StdCellLib/Documents/Datasheets/Circuitry/AAOAI222.tex
%%
%%  Purpose:        Circuit File for AAOAI222
%%
%%  ************    LaTeX with circdia.sty package      ***************
%%
%%  ///////////////////////////////////////////////////////////////////
%%
%%  Copyright (c) 2018 - 2022 by
%%                  chipforge <stdcelllib@nospam.chipforge.org>
%%  All rights reserved.
%%
%%      This Standard Cell Library is licensed under the Libre Silicon
%%      public license; you can redistribute it and/or modify it under
%%      the terms of the Libre Silicon public license as published by
%%      the Libre Silicon alliance, either version 1 of the License, or
%%      (at your option) any later version.
%%
%%      This design is distributed in the hope that it will be useful,
%%      but WITHOUT ANY WARRANTY; without even the implied warranty of
%%      MERCHANTABILITY or FITNESS FOR A PARTICULAR PURPOSE.
%%      See the Libre Silicon Public License for more details.
%%
%%  ///////////////////////////////////////////////////////////////////
\begin{circuitdiagram}[draft]{25}{14}

    \usgate
    % ----  1st column  ----
    \pin{1}{1}{L}{A}
    \pin{1}{5}{L}{A1}
    \gate[\inputs{2}]{and}{5}{3}{R}{}{}

    \pin{1}{7}{L}{B}
    \pin{1}{11}{L}{B1}
    \gate[\inputs{2}]{and}{5}{9}{R}{}{}

    % ----  2nd column  ----
    \wire{9}{3}{9}{5}
    \gate[\inputs{2}]{or}{12}{7}{R}{}{}

    % ----  3rd column  ----
    \pin{15}{11}{L}{C}
    \pin{15}{13}{L}{C1}
    \wire{16}{7}{16}{9}
    \gate[\inputs{3}]{nand}{19}{11}{R}{}{}

    % ----  result ----
    \pin{24}{11}{R}{Y}

\end{circuitdiagram}
 %%  ************    LibreSilicon's StdCellLibrary   *******************
%%
%%  Organisation:   Chipforge
%%                  Germany / European Union
%%
%%  Profile:        Chipforge focus on fine System-on-Chip Cores in
%%                  Verilog HDL Code which are easy understandable and
%%                  adjustable. For further information see
%%                          www.chipforge.org
%%                  there are projects from small cores up to PCBs, too.
%%
%%  File:           StdCellLib/Documents/Datasheets/Circuitry/AAOA222.tex
%%
%%  Purpose:        Circuit File for AAOA222
%%
%%  ************    LaTeX with circdia.sty package      ***************
%%
%%  ///////////////////////////////////////////////////////////////////
%%
%%  Copyright (c) 2018 - 2022 by
%%                  chipforge <stdcelllib@nospam.chipforge.org>
%%  All rights reserved.
%%
%%      This Standard Cell Library is licensed under the Libre Silicon
%%      public license; you can redistribute it and/or modify it under
%%      the terms of the Libre Silicon public license as published by
%%      the Libre Silicon alliance, either version 1 of the License, or
%%      (at your option) any later version.
%%
%%      This design is distributed in the hope that it will be useful,
%%      but WITHOUT ANY WARRANTY; without even the implied warranty of
%%      MERCHANTABILITY or FITNESS FOR A PARTICULAR PURPOSE.
%%      See the Libre Silicon Public License for more details.
%%
%%  ///////////////////////////////////////////////////////////////////
\begin{circuitdiagram}[draft]{31}{14}

    \usgate
    % ----  1st column  ----
    \pin{1}{1}{L}{A}
    \pin{1}{5}{L}{A1}
    \gate[\inputs{2}]{and}{5}{3}{R}{}{}

    \pin{1}{7}{L}{B}
    \pin{1}{11}{L}{B1}
    \gate[\inputs{2}]{and}{5}{9}{R}{}{}

    % ----  2nd column  ----
    \wire{9}{3}{9}{5}
    \gate[\inputs{2}]{or}{12}{7}{R}{}{}

    % ----  3rd column  ----
    \pin{15}{11}{L}{C}
    \pin{15}{13}{L}{C1}
    \wire{16}{7}{16}{9}
    \gate[\inputs{3}]{nand}{19}{11}{R}{}{}

    % ----  4th column  ----
    \gate{not}{26}{11}{R}{}{}

    % ----  result ----
    \pin{30}{11}{R}{Z}

\end{circuitdiagram}

%%  ************    LibreSilicon's StdCellLibrary   *******************
%%
%%  Organisation:   Chipforge
%%                  Germany / European Union
%%
%%  Profile:        Chipforge focus on fine System-on-Chip Cores in
%%                  Verilog HDL Code which are easy understandable and
%%                  adjustable. For further information see
%%                          www.chipforge.org
%%                  there are projects from small cores up to PCBs, too.
%%
%%  File:           StdCellLib/Documents/Datasheets/Circuitry/AAOAI321.tex
%%
%%  Purpose:        Circuit File for AAOAI321
%%
%%  ************    LaTeX with circdia.sty package      ***************
%%
%%  ///////////////////////////////////////////////////////////////////
%%
%%  Copyright (c) 2018 - 2022 by
%%                  chipforge <stdcelllib@nospam.chipforge.org>
%%  All rights reserved.
%%
%%      This Standard Cell Library is licensed under the Libre Silicon
%%      public license; you can redistribute it and/or modify it under
%%      the terms of the Libre Silicon public license as published by
%%      the Libre Silicon alliance, either version 1 of the License, or
%%      (at your option) any later version.
%%
%%      This design is distributed in the hope that it will be useful,
%%      but WITHOUT ANY WARRANTY; without even the implied warranty of
%%      MERCHANTABILITY or FITNESS FOR A PARTICULAR PURPOSE.
%%      See the Libre Silicon Public License for more details.
%%
%%  ///////////////////////////////////////////////////////////////////
\begin{circuitdiagram}[draft]{25}{12}

    \usgate
    % ----  1st column  ----
    \pin{1}{1}{L}{A}
    \pin{1}{3}{L}{A1}
    \pin{1}{5}{L}{A2}
    \gate[\inputs{3}]{and}{5}{3}{R}{}{}

    \pin{1}{7}{L}{B}
    \pin{1}{11}{L}{B1}
    \gate[\inputs{2}]{and}{5}{9}{R}{}{}

    % ----  2nd column  ----
    \wire{9}{3}{9}{5}
    \gate[\inputs{2}]{or}{12}{7}{R}{}{}

    % ----  3rd column  ----
    \pin{15}{11}{L}{C}
    \gate[\inputs{2}]{nand}{19}{9}{R}{}{}

    % ----  result ----
    \pin{24}{9}{R}{Y}

\end{circuitdiagram}
 %%  ************    LibreSilicon's StdCellLibrary   *******************
%%
%%  Organisation:   Chipforge
%%                  Germany / European Union
%%
%%  Profile:        Chipforge focus on fine System-on-Chip Cores in
%%                  Verilog HDL Code which are easy understandable and
%%                  adjustable. For further information see
%%                          www.chipforge.org
%%                  there are projects from small cores up to PCBs, too.
%%
%%  File:           StdCellLib/Documents/Circuits/AAOA321.tex
%%
%%  Purpose:        Circuit File for AAOA321
%%
%%  ************    LaTeX with circdia.sty package      ***************
%%
%%  ///////////////////////////////////////////////////////////////////
%%
%%  Copyright (c) 2019 by chipforge <stdcelllib@nospam.chipforge.org>
%%  All rights reserved.
%%
%%      This Standard Cell Library is licensed under the Libre Silicon
%%      public license; you can redistribute it and/or modify it under
%%      the terms of the Libre Silicon public license as published by
%%      the Libre Silicon alliance, either version 1 of the License, or
%%      (at your option) any later version.
%%
%%      This design is distributed in the hope that it will be useful,
%%      but WITHOUT ANY WARRANTY; without even the implied warranty of
%%      MERCHANTABILITY or FITNESS FOR A PARTICULAR PURPOSE.
%%      See the Libre Silicon Public License for more details.
%%
%%  ///////////////////////////////////////////////////////////////////
\begin{circuitdiagram}{31}{14}

    \usgate
    \gate[\inputs{3}]{and}{5}{11}{R}{}{}  % AND
    \gate[\inputs{2}]{and}{5}{5}{R}{}{}   % AND
    \gate[\inputs{2}]{or}{12}{8}{R}{}{}   % OR
    \gate[\inputs{2}]{nand}{19}{4}{R}{}{} % NAND
    \gate{not}{26}{4}{R}{}{} % NOT
    \pin{1}{1}{L}{A}     % pin A
    \pin{1}{3}{L}{B}     % pin B
    \pin{1}{7}{L}{B1}    % pin B1
    \pin{1}{9}{L}{C}     % pin C
    \pin{1}{11}{L}{C1}   % pin C1
    \pin{1}{13}{L}{C2}   % pin C2
    \wire{2}{1}{16}{1}   % wire pin A
    \wire{9}{5}{9}{6}    % wire between AND and OR
    \wire{9}{10}{9}{11}  % wire between AND and OR
    \wire{16}{1}{16}{2}  % wire between OR and NAND
    \wire{16}{6}{16}{8}  % wire between OR and NAND
    \pin{30}{4}{R}{Z}    % pin Z

\end{circuitdiagram}

%%  ************    LibreSilicon's StdCellLibrary   *******************
%%
%%  Organisation:   Chipforge
%%                  Germany / European Union
%%
%%  Profile:        Chipforge focus on fine System-on-Chip Cores in
%%                  Verilog HDL Code which are easy understandable and
%%                  adjustable. For further information see
%%                          www.chipforge.org
%%                  there are projects from small cores up to PCBs, too.
%%
%%  File:           StdCellLib/Documents/Circuits/AAOAI331.tex
%%
%%  Purpose:        Circuit File for AAOAI331
%%
%%  ************    LaTeX with circdia.sty package      ***************
%%
%%  ///////////////////////////////////////////////////////////////////
%%
%%  Copyright (c) 2019 by chipforge <stdcelllib@nospam.chipforge.org>
%%  All rights reserved.
%%
%%      This Standard Cell Library is licensed under the Libre Silicon
%%      public license; you can redistribute it and/or modify it under
%%      the terms of the Libre Silicon public license as published by
%%      the Libre Silicon alliance, either version 1 of the License, or
%%      (at your option) any later version.
%%
%%      This design is distributed in the hope that it will be useful,
%%      but WITHOUT ANY WARRANTY; without even the implied warranty of
%%      MERCHANTABILITY or FITNESS FOR A PARTICULAR PURPOSE.
%%      See the Libre Silicon Public License for more details.
%%
%%  ///////////////////////////////////////////////////////////////////
\begin{circuitdiagram}{25}{14}

    \usgate
    \gate[\inputs{3}]{and}{5}{11}{R}{}{}  % AND
    \gate[\inputs{3}]{and}{5}{5}{R}{}{}   % AND
    \gate[\inputs{2}]{or}{12}{8}{R}{}{}   % OR
    \gate[\inputs{2}]{nand}{19}{4}{R}{}{} % NAND
    \pin{1}{1}{L}{A}     % pin A
    \pin{1}{3}{L}{B}     % pin B
    \pin{1}{5}{L}{B1}    % pin B1
    \pin{1}{7}{L}{B2}    % pin B2
    \pin{1}{9}{L}{C}     % pin C
    \pin{1}{11}{L}{C1}   % pin C1
    \pin{1}{13}{L}{C2}   % pin C2
    \wire{2}{1}{16}{1}   % wire pin A
    \wire{9}{5}{9}{6}    % wire between AND and OR
    \wire{9}{10}{9}{11}  % wire between AND and OR
    \wire{16}{1}{16}{2}  % wire between OR and NAND
    \wire{16}{6}{16}{8}  % wire between OR and NAND
    \pin{24}{4}{R}{Y}    % pin Y

\end{circuitdiagram}
 %%  ************    LibreSilicon's StdCellLibrary   *******************
%%
%%  Organisation:   Chipforge
%%                  Germany / European Union
%%
%%  Profile:        Chipforge focus on fine System-on-Chip Cores in
%%                  Verilog HDL Code which are easy understandable and
%%                  adjustable. For further information see
%%                          www.chipforge.org
%%                  there are projects from small cores up to PCBs, too.
%%
%%  File:           StdCellLib/Documents/Circuits/AAOA331.tex
%%
%%  Purpose:        Circuit File for AAOA331
%%
%%  ************    LaTeX with circdia.sty package      ***************
%%
%%  ///////////////////////////////////////////////////////////////////
%%
%%  Copyright (c) 2019 by chipforge <stdcelllib@nospam.chipforge.org>
%%  All rights reserved.
%%
%%      This Standard Cell Library is licensed under the Libre Silicon
%%      public license; you can redistribute it and/or modify it under
%%      the terms of the Libre Silicon public license as published by
%%      the Libre Silicon alliance, either version 1 of the License, or
%%      (at your option) any later version.
%%
%%      This design is distributed in the hope that it will be useful,
%%      but WITHOUT ANY WARRANTY; without even the implied warranty of
%%      MERCHANTABILITY or FITNESS FOR A PARTICULAR PURPOSE.
%%      See the Libre Silicon Public License for more details.
%%
%%  ///////////////////////////////////////////////////////////////////
\begin{circuitdiagram}{31}{14}

    \usgate
    \gate[\inputs{3}]{and}{5}{11}{R}{}{}  % AND
    \gate[\inputs{3}]{and}{5}{5}{R}{}{}   % AND
    \gate[\inputs{2}]{or}{12}{8}{R}{}{}   % OR
    \gate[\inputs{2}]{nand}{19}{4}{R}{}{} % NAND
    \gate{not}{26}{4}{R}{}{} % NOT
    \pin{1}{1}{L}{A}     % pin A
    \pin{1}{3}{L}{B}     % pin B
    \pin{1}{5}{L}{B1}    % pin B1
    \pin{1}{7}{L}{B2}    % pin B2
    \pin{1}{9}{L}{C}     % pin C
    \pin{1}{11}{L}{C1}   % pin C1
    \pin{1}{13}{L}{C2}   % pin C2
    \wire{2}{1}{16}{1}   % wire pin A
    \wire{9}{5}{9}{6}    % wire between AND and OR
    \wire{9}{10}{9}{11}  % wire between AND and OR
    \wire{16}{1}{16}{2}  % wire between OR and NAND
    \wire{16}{6}{16}{8}  % wire between OR and NAND
    \pin{30}{4}{R}{Z}    % pin Z

\end{circuitdiagram}


%%  ************    LibreSilicon's StdCellLibrary   *******************
%%
%%  Organisation:   Chipforge
%%                  Germany / European Union
%%
%%  Profile:        Chipforge focus on fine System-on-Chip Cores in
%%                  Verilog HDL Code which are easy understandable and
%%                  adjustable. For further information see
%%                          www.chipforge.org
%%                  there are projects from small cores up to PCBs, too.
%%
%%  File:           StdCellLib/Documents/Datasheets/Circuitry/AAOAI2211.tex
%%
%%  Purpose:        Circuit File for AAOAI2211
%%
%%  ************    LaTeX with circdia.sty package      ***************
%%
%%  ///////////////////////////////////////////////////////////////////
%%
%%  Copyright (c) 2018 - 2022 by
%%                  chipforge <stdcelllib@nospam.chipforge.org>
%%  All rights reserved.
%%
%%      This Standard Cell Library is licensed under the Libre Silicon
%%      public license; you can redistribute it and/or modify it under
%%      the terms of the Libre Silicon public license as published by
%%      the Libre Silicon alliance, either version 1 of the License, or
%%      (at your option) any later version.
%%
%%      This design is distributed in the hope that it will be useful,
%%      but WITHOUT ANY WARRANTY; without even the implied warranty of
%%      MERCHANTABILITY or FITNESS FOR A PARTICULAR PURPOSE.
%%      See the Libre Silicon Public License for more details.
%%
%%  ///////////////////////////////////////////////////////////////////
\begin{circuitdiagram}[draft]{25}{14}

    \usgate
    % ----  1st column  ----
    \pin{1}{1}{L}{A}
    \pin{1}{5}{L}{A1}
    \gate[\inputs{2}]{and}{5}{3}{R}{}{}

    \pin{1}{7}{L}{B}
    \pin{1}{11}{L}{B1}
    \gate[\inputs{2}]{and}{5}{9}{R}{}{}

    % ----  2nd column  ----
    \wire{9}{3}{9}{7}
    \wire{9}{11}{9}{13}
    \pin{8}{13}{L}{C}
    \gate[\inputs{3}]{or}{12}{9}{R}{}{}

    % ----  3rd column  ----
    \pin{15}{13}{L}{D}
    \gate[\inputs{2}]{nand}{19}{11}{R}{}{}

    % ----  result ----
    \pin{24}{11}{R}{Y}

\end{circuitdiagram}
 %%  ************    LibreSilicon's StdCellLibrary   *******************
%%
%%  Organisation:   Chipforge
%%                  Germany / European Union
%%
%%  Profile:        Chipforge focus on fine System-on-Chip Cores in
%%                  Verilog HDL Code which are easy understandable and
%%                  adjustable. For further information see
%%                          www.chipforge.org
%%                  there are projects from small cores up to PCBs, too.
%%
%%  File:           StdCellLib/Documents/Datasheets/Circuitry/AAOA2211.tex
%%
%%  Purpose:        Circuit File for AAOA2211
%%
%%  ************    LaTeX with circdia.sty package      ***************
%%
%%  ///////////////////////////////////////////////////////////////////
%%
%%  Copyright (c) 2018 - 2022 by
%%                  chipforge <stdcelllib@nospam.chipforge.org>
%%  All rights reserved.
%%
%%      This Standard Cell Library is licensed under the Libre Silicon
%%      public license; you can redistribute it and/or modify it under
%%      the terms of the Libre Silicon public license as published by
%%      the Libre Silicon alliance, either version 1 of the License, or
%%      (at your option) any later version.
%%
%%      This design is distributed in the hope that it will be useful,
%%      but WITHOUT ANY WARRANTY; without even the implied warranty of
%%      MERCHANTABILITY or FITNESS FOR A PARTICULAR PURPOSE.
%%      See the Libre Silicon Public License for more details.
%%
%%  ///////////////////////////////////////////////////////////////////
\begin{circuitdiagram}[draft]{31}{14}

    \usgate
    % ----  1st column  ----
    \pin{1}{1}{L}{A}
    \pin{1}{5}{L}{A1}
    \gate[\inputs{2}]{and}{5}{3}{R}{}{}

    \pin{1}{7}{L}{B}
    \pin{1}{11}{L}{B1}
    \gate[\inputs{2}]{and}{5}{9}{R}{}{}

    % ----  2nd column  ----
    \wire{9}{3}{9}{7}
    \wire{9}{11}{9}{13}
    \pin{8}{13}{L}{C}
    \gate[\inputs{3}]{or}{12}{9}{R}{}{}

    % ----  3rd column  ----
    \pin{15}{13}{L}{D}
    \gate[\inputs{2}]{nand}{19}{11}{R}{}{}

    % ----  5th column  ----
    \gate{not}{26}{11}{R}{}{}

    % ----  result ----
    \pin{30}{11}{R}{Z}

\end{circuitdiagram}

%%  ************    LibreSilicon's StdCellLibrary   *******************
%%
%%  Organisation:   Chipforge
%%                  Germany / European Union
%%
%%  Profile:        Chipforge focus on fine System-on-Chip Cores in
%%                  Verilog HDL Code which are easy understandable and
%%                  adjustable. For further information see
%%                          www.chipforge.org
%%                  there are projects from small cores up to PCBs, too.
%%
%%  File:           StdCellLib/Documents/Datasheets/Circuitry/AAOAI2212.tex
%%
%%  Purpose:        Circuit File for AAOAI2212
%%
%%  ************    LaTeX with circdia.sty package      ***************
%%
%%  ///////////////////////////////////////////////////////////////////
%%
%%  Copyright (c) 2018 - 2022 by
%%                  chipforge <stdcelllib@nospam.chipforge.org>
%%  All rights reserved.
%%
%%      This Standard Cell Library is licensed under the Libre Silicon
%%      public license; you can redistribute it and/or modify it under
%%      the terms of the Libre Silicon public license as published by
%%      the Libre Silicon alliance, either version 1 of the License, or
%%      (at your option) any later version.
%%
%%      This design is distributed in the hope that it will be useful,
%%      but WITHOUT ANY WARRANTY; without even the implied warranty of
%%      MERCHANTABILITY or FITNESS FOR A PARTICULAR PURPOSE.
%%      See the Libre Silicon Public License for more details.
%%
%%  ///////////////////////////////////////////////////////////////////
\begin{circuitdiagram}[draft]{25}{16}

    \usgate
    % ----  1st column  ----
    \pin{1}{1}{L}{A}
    \pin{1}{5}{L}{A1}
    \gate[\inputs{2}]{and}{5}{3}{R}{}{}

    \pin{1}{7}{L}{B}
    \pin{1}{11}{L}{B1}
    \gate[\inputs{2}]{and}{5}{9}{R}{}{}

    % ----  2nd column  ----
    \wire{9}{3}{9}{7}
    \wire{9}{11}{9}{13}
    \pin{8}{13}{L}{C}
    \gate[\inputs{3}]{or}{12}{9}{R}{}{}

    % ----  3rd column  ----
    \wire{16}{9}{16}{11}
    \pin{15}{13}{L}{D}
    \pin{15}{15}{L}{D1}
    \gate[\inputs{3}]{nand}{19}{13}{R}{}{}

    % ----  result ----
    \pin{24}{13}{R}{Y}

\end{circuitdiagram}
 %%  ************    LibreSilicon's StdCellLibrary   *******************
%%
%%  Organisation:   Chipforge
%%                  Germany / European Union
%%
%%  Profile:        Chipforge focus on fine System-on-Chip Cores in
%%                  Verilog HDL Code which are easy understandable and
%%                  adjustable. For further information see
%%                          www.chipforge.org
%%                  there are projects from small cores up to PCBs, too.
%%
%%  File:           StdCellLib/Documents/Datasheets/Circuitry/AAOA2212.tex
%%
%%  Purpose:        Circuit File for AAOA2212
%%
%%  ************    LaTeX with circdia.sty package      ***************
%%
%%  ///////////////////////////////////////////////////////////////////
%%
%%  Copyright (c) 2018 - 2022 by
%%                  chipforge <stdcelllib@nospam.chipforge.org>
%%  All rights reserved.
%%
%%      This Standard Cell Library is licensed under the Libre Silicon
%%      public license; you can redistribute it and/or modify it under
%%      the terms of the Libre Silicon public license as published by
%%      the Libre Silicon alliance, either version 1 of the License, or
%%      (at your option) any later version.
%%
%%      This design is distributed in the hope that it will be useful,
%%      but WITHOUT ANY WARRANTY; without even the implied warranty of
%%      MERCHANTABILITY or FITNESS FOR A PARTICULAR PURPOSE.
%%      See the Libre Silicon Public License for more details.
%%
%%  ///////////////////////////////////////////////////////////////////
\begin{circuitdiagram}[draft]{31}{16}

    \usgate
    % ----  1st column  ----
    \pin{1}{1}{L}{A}
    \pin{1}{5}{L}{A1}
    \gate[\inputs{2}]{and}{5}{3}{R}{}{}

    \pin{1}{7}{L}{B}
    \pin{1}{11}{L}{B1}
    \gate[\inputs{2}]{and}{5}{9}{R}{}{}

    % ----  2nd column  ----
    \wire{9}{3}{9}{7}
    \wire{9}{11}{9}{13}
    \pin{8}{13}{L}{C}
    \gate[\inputs{3}]{or}{12}{9}{R}{}{}

    % ----  3rd column  ----
    \wire{16}{9}{16}{11}
    \pin{15}{13}{L}{D}
    \pin{15}{15}{L}{D1}
    \gate[\inputs{3}]{nand}{19}{13}{R}{}{}

    % ----  5th column  ----
    \gate{not}{26}{13}{R}{}{}

    % ----  result ----
    \pin{30}{13}{R}{Z}

\end{circuitdiagram}

%%  ************    LibreSilicon's StdCellLibrary   *******************
%%
%%  Organisation:   Chipforge
%%                  Germany / European Union
%%
%%  Profile:        Chipforge focus on fine System-on-Chip Cores in
%%                  Verilog HDL Code which are easy understandable and
%%                  adjustable. For further information see
%%                          www.chipforge.org
%%                  there are projects from small cores up to PCBs, too.
%%
%%  File:           StdCellLib/Documents/Datasheets/Circuitry/AAOAI2221.tex
%%
%%  Purpose:        Circuit File for AAOAI2221
%%
%%  ************    LaTeX with circdia.sty package      ***************
%%
%%  ///////////////////////////////////////////////////////////////////
%%
%%  Copyright (c) 2018 - 2022 by
%%                  chipforge <stdcelllib@nospam.chipforge.org>
%%  All rights reserved.
%%
%%      This Standard Cell Library is licensed under the Libre Silicon
%%      public license; you can redistribute it and/or modify it under
%%      the terms of the Libre Silicon public license as published by
%%      the Libre Silicon alliance, either version 1 of the License, or
%%      (at your option) any later version.
%%
%%      This design is distributed in the hope that it will be useful,
%%      but WITHOUT ANY WARRANTY; without even the implied warranty of
%%      MERCHANTABILITY or FITNESS FOR A PARTICULAR PURPOSE.
%%      See the Libre Silicon Public License for more details.
%%
%%  ///////////////////////////////////////////////////////////////////
\begin{circuitdiagram}[draft]{27}{15}

    \usgate
    % ----  1st column  ----
    \pin{1}{1}{L}{A}
    \pin{1}{5}{L}{A1}
    \gate[\inputs{2}]{and}{5}{3}{R}{}{}

    \pin{1}{7}{L}{B}
    \pin{1}{11}{L}{B1}
    \gate[\inputs{2}]{and}{5}{9}{R}{}{}

    % ----  2nd column  ----
    \wire{9}{3}{9}{7}
    \wire{9}{7}{11}{7}
    \wire{9}{9}{11}{9}
    \pin{10}{11}{L}{C}
    \pin{10}{13}{L}{C1}
    \gate[\inputs{4}]{or}{14}{10}{R}{}{}

    % ----  3rd column  ----
    \pin{17}{14}{L}{D}
    \gate[\inputs{2}]{nand}{21}{12}{R}{}{}

    % ----  result ----
    \pin{26}{12}{R}{Y}

\end{circuitdiagram}
 %%  ************    LibreSilicon's StdCellLibrary   *******************
%%
%%  Organisation:   Chipforge
%%                  Germany / European Union
%%
%%  Profile:        Chipforge focus on fine System-on-Chip Cores in
%%                  Verilog HDL Code which are easy understandable and
%%                  adjustable. For further information see
%%                          www.chipforge.org
%%                  there are projects from small cores up to PCBs, too.
%%
%%  File:           StdCellLib/Documents/Datasheets/Circuitry/AAOA2221.tex
%%
%%  Purpose:        Circuit File for AAOA2221
%%
%%  ************    LaTeX with circdia.sty package      ***************
%%
%%  ///////////////////////////////////////////////////////////////////
%%
%%  Copyright (c) 2018 - 2022 by
%%                  chipforge <stdcelllib@nospam.chipforge.org>
%%  All rights reserved.
%%
%%      This Standard Cell Library is licensed under the Libre Silicon
%%      public license; you can redistribute it and/or modify it under
%%      the terms of the Libre Silicon public license as published by
%%      the Libre Silicon alliance, either version 1 of the License, or
%%      (at your option) any later version.
%%
%%      This design is distributed in the hope that it will be useful,
%%      but WITHOUT ANY WARRANTY; without even the implied warranty of
%%      MERCHANTABILITY or FITNESS FOR A PARTICULAR PURPOSE.
%%      See the Libre Silicon Public License for more details.
%%
%%  ///////////////////////////////////////////////////////////////////
\begin{circuitdiagram}[draft]{33}{15}

    \usgate
    % ----  1st column  ----
    \pin{1}{1}{L}{A}
    \pin{1}{5}{L}{A1}
    \gate[\inputs{2}]{and}{5}{3}{R}{}{}

    \pin{1}{7}{L}{B}
    \pin{1}{11}{L}{B1}
    \gate[\inputs{2}]{and}{5}{9}{R}{}{}

    % ----  2nd column  ----
    \wire{9}{3}{9}{7}
    \wire{9}{7}{11}{7}
    \wire{9}{9}{11}{9}
    \pin{10}{11}{L}{C}
    \pin{10}{13}{L}{C1}
    \gate[\inputs{4}]{or}{14}{10}{R}{}{}

    % ----  3rd column  ----
    \pin{17}{14}{L}{D}
    \gate[\inputs{2}]{nand}{21}{12}{R}{}{}

    % ----  5th column  ----
    \gate{not}{28}{12}{R}{}{}

    % ----  result ----
    \pin{32}{12}{R}{Z}

\end{circuitdiagram}

%%  ************    LibreSilicon's StdCellLibrary   *******************
%%
%%  Organisation:   Chipforge
%%                  Germany / European Union
%%
%%  Profile:        Chipforge focus on fine System-on-Chip Cores in
%%                  Verilog HDL Code which are easy understandable and
%%                  adjustable. For further information see
%%                          www.chipforge.org
%%                  there are projects from small cores up to PCBs, too.
%%
%%  File:           StdCellLib/Documents/Datasheets/Circuitry/AAOAI3211.tex
%%
%%  Purpose:        Circuit File for AAOAI3211
%%
%%  ************    LaTeX with circdia.sty package      ***************
%%
%%  ///////////////////////////////////////////////////////////////////
%%
%%  Copyright (c) 2018 - 2022 by
%%                  chipforge <stdcelllib@nospam.chipforge.org>
%%  All rights reserved.
%%
%%      This Standard Cell Library is licensed under the Libre Silicon
%%      public license; you can redistribute it and/or modify it under
%%      the terms of the Libre Silicon public license as published by
%%      the Libre Silicon alliance, either version 1 of the License, or
%%      (at your option) any later version.
%%
%%      This design is distributed in the hope that it will be useful,
%%      but WITHOUT ANY WARRANTY; without even the implied warranty of
%%      MERCHANTABILITY or FITNESS FOR A PARTICULAR PURPOSE.
%%      See the Libre Silicon Public License for more details.
%%
%%  ///////////////////////////////////////////////////////////////////
\begin{circuitdiagram}[draft]{25}{14}

    \usgate
    % ----  1st column  ----
    \pin{1}{1}{L}{A}
    \pin{1}{3}{L}{A1}
    \pin{1}{5}{L}{A2}
    \gate[\inputs{3}]{and}{5}{3}{R}{}{}

    \pin{1}{7}{L}{B}
    \pin{1}{11}{L}{B1}
    \gate[\inputs{2}]{and}{5}{9}{R}{}{}

    % ----  2nd column  ----
    \wire{9}{3}{9}{7}
    \wire{9}{11}{9}{13}
    \pin{8}{13}{L}{C}
    \gate[\inputs{3}]{or}{12}{9}{R}{}{}

    % ----  3rd column  ----
    \pin{15}{13}{L}{D}
    \gate[\inputs{2}]{nand}{19}{11}{R}{}{}

    % ----  result ----
    \pin{24}{11}{R}{Y}

\end{circuitdiagram}
 %%  ************    LibreSilicon's StdCellLibrary   *******************
%%
%%  Organisation:   Chipforge
%%                  Germany / European Union
%%
%%  Profile:        Chipforge focus on fine System-on-Chip Cores in
%%                  Verilog HDL Code which are easy understandable and
%%                  adjustable. For further information see
%%                          www.chipforge.org
%%                  there are projects from small cores up to PCBs, too.
%%
%%  File:           StdCellLib/Documents/Datasheets/Circuitry/AAOA3211.tex
%%
%%  Purpose:        Circuit File for AAOA3211
%%
%%  ************    LaTeX with circdia.sty package      ***************
%%
%%  ///////////////////////////////////////////////////////////////////
%%
%%  Copyright (c) 2018 - 2022 by
%%                  chipforge <stdcelllib@nospam.chipforge.org>
%%  All rights reserved.
%%
%%      This Standard Cell Library is licensed under the Libre Silicon
%%      public license; you can redistribute it and/or modify it under
%%      the terms of the Libre Silicon public license as published by
%%      the Libre Silicon alliance, either version 1 of the License, or
%%      (at your option) any later version.
%%
%%      This design is distributed in the hope that it will be useful,
%%      but WITHOUT ANY WARRANTY; without even the implied warranty of
%%      MERCHANTABILITY or FITNESS FOR A PARTICULAR PURPOSE.
%%      See the Libre Silicon Public License for more details.
%%
%%  ///////////////////////////////////////////////////////////////////
\begin{circuitdiagram}[draft]{31}{14}

    \usgate
    % ----  1st column  ----
    \pin{1}{1}{L}{A}
    \pin{1}{3}{L}{A1}
    \pin{1}{5}{L}{A2}
    \gate[\inputs{3}]{and}{5}{3}{R}{}{}

    \pin{1}{7}{L}{B}
    \pin{1}{11}{L}{B1}
    \gate[\inputs{2}]{and}{5}{9}{R}{}{}

    % ----  2nd column  ----
    \wire{9}{3}{9}{7}
    \wire{9}{11}{9}{13}
    \pin{8}{13}{L}{C}
    \gate[\inputs{3}]{or}{12}{9}{R}{}{}

    % ----  3rd column  ----
    \pin{15}{13}{L}{D}
    \gate[\inputs{2}]{nand}{19}{11}{R}{}{}

    % ----  5th column  ----
    \gate{not}{26}{11}{R}{}{}

    % ----  result ----
    \pin{30}{11}{R}{Z}

\end{circuitdiagram}

%%  ************    LibreSilicon's StdCellLibrary   *******************
%%
%%  Organisation:   Chipforge
%%                  Germany / European Union
%%
%%  Profile:        Chipforge focus on fine System-on-Chip Cores in
%%                  Verilog HDL Code which are easy understandable and
%%                  adjustable. For further information see
%%                          www.chipforge.org
%%                  there are projects from small cores up to PCBs, too.
%%
%%  File:           StdCellLib/Documents/Datasheets/Circuitry/AAOAI3311.tex
%%
%%  Purpose:        Circuit File for AAOAI3311
%%
%%  ************    LaTeX with circdia.sty package      ***************
%%
%%  ///////////////////////////////////////////////////////////////////
%%
%%  Copyright (c) 2018 - 2022 by
%%                  chipforge <stdcelllib@nospam.chipforge.org>
%%  All rights reserved.
%%
%%      This Standard Cell Library is licensed under the Libre Silicon
%%      public license; you can redistribute it and/or modify it under
%%      the terms of the Libre Silicon public license as published by
%%      the Libre Silicon alliance, either version 1 of the License, or
%%      (at your option) any later version.
%%
%%      This design is distributed in the hope that it will be useful,
%%      but WITHOUT ANY WARRANTY; without even the implied warranty of
%%      MERCHANTABILITY or FITNESS FOR A PARTICULAR PURPOSE.
%%      See the Libre Silicon Public License for more details.
%%
%%  ///////////////////////////////////////////////////////////////////
\begin{circuitdiagram}[draft]{25}{14}

    \usgate
    % ----  1st column  ----
    \pin{1}{1}{L}{A}
    \pin{1}{3}{L}{A1}
    \pin{1}{5}{L}{A2}
    \gate[\inputs{3}]{and}{5}{3}{R}{}{}

    \pin{1}{7}{L}{B}
    \pin{1}{9}{L}{B1}
    \pin{1}{11}{L}{B2}
    \gate[\inputs{3}]{and}{5}{9}{R}{}{}

    % ----  2nd column  ----
    \wire{9}{3}{9}{7}
    \wire{9}{11}{9}{13}
    \pin{8}{13}{L}{C}
    \gate[\inputs{3}]{or}{12}{9}{R}{}{}

    % ----  3rd column  ----
    \pin{15}{13}{L}{D}
    \gate[\inputs{2}]{nand}{19}{11}{R}{}{}

    % ----  result ----
    \pin{24}{11}{R}{Y}

\end{circuitdiagram}
 %%  ************    LibreSilicon's StdCellLibrary   *******************
%%
%%  Organisation:   Chipforge
%%                  Germany / European Union
%%
%%  Profile:        Chipforge focus on fine System-on-Chip Cores in
%%                  Verilog HDL Code which are easy understandable and
%%                  adjustable. For further information see
%%                          www.chipforge.org
%%                  there are projects from small cores up to PCBs, too.
%%
%%  File:           StdCellLib/Documents/Datasheets/Circuitry/AAOA3311.tex
%%
%%  Purpose:        Circuit File for AAOA3311
%%
%%  ************    LaTeX with circdia.sty package      ***************
%%
%%  ///////////////////////////////////////////////////////////////////
%%
%%  Copyright (c) 2018 - 2022 by
%%                  chipforge <stdcelllib@nospam.chipforge.org>
%%  All rights reserved.
%%
%%      This Standard Cell Library is licensed under the Libre Silicon
%%      public license; you can redistribute it and/or modify it under
%%      the terms of the Libre Silicon public license as published by
%%      the Libre Silicon alliance, either version 1 of the License, or
%%      (at your option) any later version.
%%
%%      This design is distributed in the hope that it will be useful,
%%      but WITHOUT ANY WARRANTY; without even the implied warranty of
%%      MERCHANTABILITY or FITNESS FOR A PARTICULAR PURPOSE.
%%      See the Libre Silicon Public License for more details.
%%
%%  ///////////////////////////////////////////////////////////////////
\begin{circuitdiagram}[draft]{31}{14}

    \usgate
    % ----  1st column  ----
    \pin{1}{1}{L}{A}
    \pin{1}{3}{L}{A1}
    \pin{1}{5}{L}{A2}
    \gate[\inputs{3}]{and}{5}{3}{R}{}{}

    \pin{1}{7}{L}{B}
    \pin{1}{9}{L}{B1}
    \pin{1}{11}{L}{B2}
    \gate[\inputs{3}]{and}{5}{9}{R}{}{}

    % ----  2nd column  ----
    \wire{9}{3}{9}{7}
    \wire{9}{11}{9}{13}
    \pin{8}{13}{L}{C}
    \gate[\inputs{3}]{or}{12}{9}{R}{}{}

    % ----  3rd column  ----
    \pin{15}{13}{L}{D}
    \gate[\inputs{2}]{nand}{19}{11}{R}{}{}

    % ----  5th column  ----
    \gate{not}{26}{11}{R}{}{}

    % ----  result ----
    \pin{30}{11}{R}{Z}

\end{circuitdiagram}


%%  ************    LibreSilicon's StdCellLibrary   *******************
%%
%%  Organisation:   Chipforge
%%                  Germany / European Union
%%
%%  Profile:        Chipforge focus on fine System-on-Chip Cores in
%%                  Verilog HDL Code which are easy understandable and
%%                  adjustable. For further information see
%%                          www.chipforge.org
%%                  there are projects from small cores up to PCBs, too.
%%
%%  File:           StdCellLib/Documents/LaTeX/section-OOAOI_complex.tex
%%
%%  Purpose:        Section Level File for Standard Cell Library Documentation
%%
%%  ************    LaTeX with circdia.sty package      ***************
%%
%%  ///////////////////////////////////////////////////////////////////
%%
%%  Copyright (c) 2018 - 2021 by
%%                  chipforge <stdcelllib@nospam.chipforge.org>
%%  All rights reserved.
%%
%%      This Standard Cell Library is licensed under the Libre Silicon
%%      public license; you can redistribute it and/or modify it under
%%      the terms of the Libre Silicon public license as published by
%%      the Libre Silicon alliance, either version 1 of the License, or
%%      (at your option) any later version.
%%
%%      This design is distributed in the hope that it will be useful,
%%      but WITHOUT ANY WARRANTY; without even the implied warranty of
%%      MERCHANTABILITY or FITNESS FOR A PARTICULAR PURPOSE.
%%      See the Libre Silicon Public License for more details.
%%
\section{OR-OR-AND-OR(-Invert) Complex Gates}

\include{OOAOI221_datasheet} \include{OOAO221_datasheet}
\include{OOAOI321_datasheet} \include{OOAO321_datasheet}
\include{OOAOI331_datasheet} \include{OOAO331_datasheet}

%%  ************    LibreSilicon's StdCellLibrary   *******************
%%
%%  Organisation:   Chipforge
%%                  Germany / European Union
%%
%%  Profile:        Chipforge focus on fine System-on-Chip Cores in
%%                  Verilog HDL Code which are easy understandable and
%%                  adjustable. For further information see
%%                          www.chipforge.org
%%                  there are projects from small cores up to PCBs, too.
%%
%%  File:           StdCellLib/Documents/section-AAAOAI_complex.tex
%%
%%  Purpose:        Section Level File for Standard Cell Library Documentation
%%
%%  ************    LaTeX with circdia.sty package      ***************
%%
%%  ///////////////////////////////////////////////////////////////////
%%
%%  Copyright (c) 2018 - 2022 by
%%                  chipforge <stdcelllib@nospam.chipforge.org>
%%  All rights reserved.
%%
%%      This Standard Cell Library is licensed under the Libre Silicon
%%      public license; you can redistribute it and/or modify it under
%%      the terms of the Libre Silicon public license as published by
%%      the Libre Silicon alliance, either version 1 of the License, or
%%      (at your option) any later version.
%%
%%      This design is distributed in the hope that it will be useful,
%%      but WITHOUT ANY WARRANTY; without even the implied warranty of
%%      MERCHANTABILITY or FITNESS FOR A PARTICULAR PURPOSE.
%%      See the Libre Silicon Public License for more details.
%%
%%  ///////////////////////////////////////////////////////////////////
\section{AND-AND-AND-OR-AND(-Invert) Complex Gates}

%%  ************    LibreSilicon's StdCellLibrary   *******************
%%
%%  Organisation:   Chipforge
%%                  Germany / European Union
%%
%%  Profile:        Chipforge focus on fine System-on-Chip Cores in
%%                  Verilog HDL Code which are easy understandable and
%%                  adjustable. For further information see
%%                          www.chipforge.org
%%                  there are projects from small cores up to PCBs, too.
%%
%%  File:           StdCellLib/Documents/Datasheets/Circuitry/AAAOI2221.tex
%%
%%  Purpose:        Circuit File for AAAOI2221
%%
%%  ************    LaTeX with circdia.sty package      ***************
%%
%%  ///////////////////////////////////////////////////////////////////
%%
%%  Copyright (c) 2018 - 2022 by
%%                  chipforge <stdcelllib@nospam.chipforge.org>
%%  All rights reserved.
%%
%%      This Standard Cell Library is licensed under the Libre Silicon
%%      public license; you can redistribute it and/or modify it under
%%      the terms of the Libre Silicon public license as published by
%%      the Libre Silicon alliance, either version 1 of the License, or
%%      (at your option) any later version.
%%
%%      This design is distributed in the hope that it will be useful,
%%      but WITHOUT ANY WARRANTY; without even the implied warranty of
%%      MERCHANTABILITY or FITNESS FOR A PARTICULAR PURPOSE.
%%      See the Libre Silicon Public License for more details.
%%
%%  ///////////////////////////////////////////////////////////////////
\begin{circuitdiagram}[draft]{25}{18}

    \usgate
    % ----  1st column  ----
    \pin{1}{1}{L}{A}
    \pin{1}{5}{L}{A1}
    \gate[\inputs{2}]{and}{5}{3}{R}{}{}

    \pin{1}{7}{L}{B}
    \pin{1}{11}{L}{B1}
    \gate[\inputs{2}]{and}{5}{9}{R}{}{}

    \pin{1}{13}{L}{C}
    \pin{1}{17}{L}{C1}
    \gate[\inputs{2}]{and}{5}{15}{R}{}{}

    % ----  2nd column  ----
    \wire{9}{3}{9}{7}
    \wire{9}{11}{9}{15}
    \gate[\inputs{3}]{or}{12}{9}{R}{}{}

    % ----  3rd column  ----
    \pin{15}{13}{L}{D}
    \gate[\inputs{2}]{nand}{19}{11}{R}{}{}

    % ----  result ----
    \pin{24}{11}{R}{Y}

\end{circuitdiagram}
 %%  ************    LibreSilicon's StdCellLibrary   *******************
%%
%%  Organisation:   Chipforge
%%                  Germany / European Union
%%
%%  Profile:        Chipforge focus on fine System-on-Chip Cores in
%%                  Verilog HDL Code which are easy understandable and
%%                  adjustable. For further information see
%%                          www.chipforge.org
%%                  there are projects from small cores up to PCBs, too.
%%
%%  File:           StdCellLib/Documents/Datasheets/Circuitry/AAAO2221.tex
%%
%%  Purpose:        Circuit File for AAAO2221
%%
%%  ************    LaTeX with circdia.sty package      ***************
%%
%%  ///////////////////////////////////////////////////////////////////
%%
%%  Copyright (c) 2018 - 2022 by
%%                  chipforge <stdcelllib@nospam.chipforge.org>
%%  All rights reserved.
%%
%%      This Standard Cell Library is licensed under the Libre Silicon
%%      public license; you can redistribute it and/or modify it under
%%      the terms of the Libre Silicon public license as published by
%%      the Libre Silicon alliance, either version 1 of the License, or
%%      (at your option) any later version.
%%
%%      This design is distributed in the hope that it will be useful,
%%      but WITHOUT ANY WARRANTY; without even the implied warranty of
%%      MERCHANTABILITY or FITNESS FOR A PARTICULAR PURPOSE.
%%      See the Libre Silicon Public License for more details.
%%
%%  ///////////////////////////////////////////////////////////////////
\begin{circuitdiagram}[draft]{31}{18}

    \usgate
    % ----  1st column  ----
    \pin{1}{1}{L}{A}
    \pin{1}{5}{L}{A1}
    \gate[\inputs{2}]{and}{5}{3}{R}{}{}

    \pin{1}{7}{L}{B}
    \pin{1}{11}{L}{B1}
    \gate[\inputs{2}]{and}{5}{9}{R}{}{}

    \pin{1}{13}{L}{C}
    \pin{1}{17}{L}{C1}
    \gate[\inputs{2}]{and}{5}{15}{R}{}{}

    % ----  2nd column  ----
    \wire{9}{3}{9}{7}
    \wire{9}{11}{9}{15}
    \gate[\inputs{3}]{or}{12}{9}{R}{}{}

    % ----  3rd column  ----
    \pin{15}{13}{L}{D}
    \gate[\inputs{2}]{nand}{19}{11}{R}{}{}

    % ----  4th column  ----
    \gate{not}{26}{11}{R}{}{}

    % ----  result ----
    \pin{30}{11}{R}{Z}

\end{circuitdiagram}

%%  ************    LibreSilicon's StdCellLibrary   *******************
%%
%%  Organisation:   Chipforge
%%                  Germany / European Union
%%
%%  Profile:        Chipforge focus on fine System-on-Chip Cores in
%%                  Verilog HDL Code which are easy understandable and
%%                  adjustable. For further information see
%%                          www.chipforge.org
%%                  there are projects from small cores up to PCBs, too.
%%
%%  File:           StdCellLib/Documents/Datasheets/Circuitry/AAAOI2222.tex
%%
%%  Purpose:        Circuit File for AAAOI2222
%%
%%  ************    LaTeX with circdia.sty package      ***************
%%
%%  ///////////////////////////////////////////////////////////////////
%%
%%  Copyright (c) 2018 - 2022 by
%%                  chipforge <stdcelllib@nospam.chipforge.org>
%%  All rights reserved.
%%
%%      This Standard Cell Library is licensed under the Libre Silicon
%%      public license; you can redistribute it and/or modify it under
%%      the terms of the Libre Silicon public license as published by
%%      the Libre Silicon alliance, either version 1 of the License, or
%%      (at your option) any later version.
%%
%%      This design is distributed in the hope that it will be useful,
%%      but WITHOUT ANY WARRANTY; without even the implied warranty of
%%      MERCHANTABILITY or FITNESS FOR A PARTICULAR PURPOSE.
%%      See the Libre Silicon Public License for more details.
%%
%%  ///////////////////////////////////////////////////////////////////
\begin{circuitdiagram}[draft]{25}{18}

    \usgate
    % ----  1st column  ----
    \pin{1}{1}{L}{A}
    \pin{1}{5}{L}{A1}
    \gate[\inputs{2}]{and}{5}{3}{R}{}{}

    \pin{1}{7}{L}{B}
    \pin{1}{11}{L}{B1}
    \gate[\inputs{2}]{and}{5}{9}{R}{}{}

    \pin{1}{13}{L}{C}
    \pin{1}{17}{L}{C1}
    \gate[\inputs{2}]{and}{5}{15}{R}{}{}

    % ----  2nd column  ----
    \wire{9}{3}{9}{7}
    \wire{9}{11}{9}{15}
    \gate[\inputs{3}]{or}{12}{9}{R}{}{}

    % ----  3rd column  ----
    \pin{15}{15}{L}{D1}
    \pin{15}{13}{L}{D}
    \wire{16}{9}{16}{11}
    \gate[\inputs{3}]{nand}{19}{13}{R}{}{}

    % ----  result ----
    \pin{24}{13}{R}{Y}

\end{circuitdiagram}
 %%  ************    LibreSilicon's StdCellLibrary   *******************
%%
%%  Organisation:   Chipforge
%%                  Germany / European Union
%%
%%  Profile:        Chipforge focus on fine System-on-Chip Cores in
%%                  Verilog HDL Code which are easy understandable and
%%                  adjustable. For further information see
%%                          www.chipforge.org
%%                  there are projects from small cores up to PCBs, too.
%%
%%  File:           StdCellLib/Documents/Datasheets/Circuitry/AAAO2222.tex
%%
%%  Purpose:        Circuit File for AAAO2222
%%
%%  ************    LaTeX with circdia.sty package      ***************
%%
%%  ///////////////////////////////////////////////////////////////////
%%
%%  Copyright (c) 2018 - 2022 by
%%                  chipforge <stdcelllib@nospam.chipforge.org>
%%  All rights reserved.
%%
%%      This Standard Cell Library is licensed under the Libre Silicon
%%      public license; you can redistribute it and/or modify it under
%%      the terms of the Libre Silicon public license as published by
%%      the Libre Silicon alliance, either version 1 of the License, or
%%      (at your option) any later version.
%%
%%      This design is distributed in the hope that it will be useful,
%%      but WITHOUT ANY WARRANTY; without even the implied warranty of
%%      MERCHANTABILITY or FITNESS FOR A PARTICULAR PURPOSE.
%%      See the Libre Silicon Public License for more details.
%%
%%  ///////////////////////////////////////////////////////////////////
\begin{circuitdiagram}[draft]{31}{18}

    \usgate
    % ----  1st column  ----
    \pin{1}{1}{L}{A}
    \pin{1}{5}{L}{A1}
    \gate[\inputs{2}]{and}{5}{3}{R}{}{}

    \pin{1}{7}{L}{B}
    \pin{1}{11}{L}{B1}
    \gate[\inputs{2}]{and}{5}{9}{R}{}{}

    \pin{1}{13}{L}{C}
    \pin{1}{17}{L}{C1}
    \gate[\inputs{2}]{and}{5}{15}{R}{}{}

    % ----  2nd column  ----
    \wire{9}{3}{9}{7}
    \wire{9}{11}{9}{15}
    \gate[\inputs{3}]{or}{12}{9}{R}{}{}

    % ----  3rd column  ----
    \pin{15}{15}{L}{D1}
    \pin{15}{13}{L}{D}
    \wire{16}{9}{16}{11}
    \gate[\inputs{3}]{nand}{19}{13}{R}{}{}

    % ----  4th column  ----
    \gate{not}{26}{13}{R}{}{}

    % ----  result ----
    \pin{30}{13}{R}{Z}

\end{circuitdiagram}

%%  ************    LibreSilicon's StdCellLibrary   *******************
%%
%%  Organisation:   Chipforge
%%                  Germany / European Union
%%
%%  Profile:        Chipforge focus on fine System-on-Chip Cores in
%%                  Verilog HDL Code which are easy understandable and
%%                  adjustable. For further information see
%%                          www.chipforge.org
%%                  there are projects from small cores up to PCBs, too.
%%
%%  File:           StdCellLib/Documents/Datasheets/Circuitry/AAAOI3221.tex
%%
%%  Purpose:        Circuit File for AAAOI3221
%%
%%  ************    LaTeX with circdia.sty package      ***************
%%
%%  ///////////////////////////////////////////////////////////////////
%%
%%  Copyright (c) 2018 - 2022 by
%%                  chipforge <stdcelllib@nospam.chipforge.org>
%%  All rights reserved.
%%
%%      This Standard Cell Library is licensed under the Libre Silicon
%%      public license; you can redistribute it and/or modify it under
%%      the terms of the Libre Silicon public license as published by
%%      the Libre Silicon alliance, either version 1 of the License, or
%%      (at your option) any later version.
%%
%%      This design is distributed in the hope that it will be useful,
%%      but WITHOUT ANY WARRANTY; without even the implied warranty of
%%      MERCHANTABILITY or FITNESS FOR A PARTICULAR PURPOSE.
%%      See the Libre Silicon Public License for more details.
%%
%%  ///////////////////////////////////////////////////////////////////
\begin{circuitdiagram}[draft]{25}{18}

    \usgate
    % ----  1st column  ----
    \pin{1}{1}{L}{A}
    \pin{1}{3}{L}{A1}
    \pin{1}{5}{L}{A2}
    \gate[\inputs{3}]{and}{5}{3}{R}{}{}

    \pin{1}{7}{L}{B}
    \pin{1}{11}{L}{B1}
    \gate[\inputs{2}]{and}{5}{9}{R}{}{}

    \pin{1}{13}{L}{C}
    \pin{1}{17}{L}{C1}
    \gate[\inputs{2}]{and}{5}{15}{R}{}{}

    % ----  2nd column  ----
    \wire{9}{3}{9}{7}
    \wire{9}{11}{9}{15}
    \gate[\inputs{3}]{or}{12}{9}{R}{}{}

    % ----  3rd column  ----
    \pin{15}{13}{L}{D}
    \gate[\inputs{2}]{nand}{19}{11}{R}{}{}

    % ----  result ----
    \pin{24}{11}{R}{Y}

\end{circuitdiagram}
 %%  ************    LibreSilicon's StdCellLibrary   *******************
%%
%%  Organisation:   Chipforge
%%                  Germany / European Union
%%
%%  Profile:        Chipforge focus on fine System-on-Chip Cores in
%%                  Verilog HDL Code which are easy understandable and
%%                  adjustable. For further information see
%%                          www.chipforge.org
%%                  there are projects from small cores up to PCBs, too.
%%
%%  File:           StdCellLib/Documents/Datasheets/Circuitry/AAAO3221.tex
%%
%%  Purpose:        Circuit File for AAAO3221
%%
%%  ************    LaTeX with circdia.sty package      ***************
%%
%%  ///////////////////////////////////////////////////////////////////
%%
%%  Copyright (c) 2018 - 2022 by
%%                  chipforge <stdcelllib@nospam.chipforge.org>
%%  All rights reserved.
%%
%%      This Standard Cell Library is licensed under the Libre Silicon
%%      public license; you can redistribute it and/or modify it under
%%      the terms of the Libre Silicon public license as published by
%%      the Libre Silicon alliance, either version 1 of the License, or
%%      (at your option) any later version.
%%
%%      This design is distributed in the hope that it will be useful,
%%      but WITHOUT ANY WARRANTY; without even the implied warranty of
%%      MERCHANTABILITY or FITNESS FOR A PARTICULAR PURPOSE.
%%      See the Libre Silicon Public License for more details.
%%
%%  ///////////////////////////////////////////////////////////////////
\begin{circuitdiagram}[draft]{31}{18}

    \usgate
    % ----  1st column  ----
    \pin{1}{1}{L}{A}
    \pin{1}{3}{L}{A1}
    \pin{1}{5}{L}{A2}
    \gate[\inputs{3}]{and}{5}{3}{R}{}{}

    \pin{1}{7}{L}{B}
    \pin{1}{11}{L}{B1}
    \gate[\inputs{2}]{and}{5}{9}{R}{}{}

    \pin{1}{13}{L}{C}
    \pin{1}{17}{L}{C1}
    \gate[\inputs{2}]{and}{5}{15}{R}{}{}

    % ----  2nd column  ----
    \wire{9}{3}{9}{7}
    \wire{9}{11}{9}{15}
    \gate[\inputs{3}]{or}{12}{9}{R}{}{}

    % ----  3rd column  ----
    \pin{15}{13}{L}{D}
    \gate[\inputs{2}]{nand}{19}{11}{R}{}{}

    % ----  4th column  ----
    \gate{not}{26}{11}{R}{}{}

    % ----  result ----
    \pin{30}{11}{R}{Z}

\end{circuitdiagram}

%%  ************    LibreSilicon's StdCellLibrary   *******************
%%
%%  Organisation:   Chipforge
%%                  Germany / European Union
%%
%%  Profile:        Chipforge focus on fine System-on-Chip Cores in
%%                  Verilog HDL Code which are easy understandable and
%%                  adjustable. For further information see
%%                          www.chipforge.org
%%                  there are projects from small cores up to PCBs, too.
%%
%%  File:           StdCellLib/Documents/Datasheets/Circuitry/AAAOI3321.tex
%%
%%  Purpose:        Circuit File for AAAOI3321
%%
%%  ************    LaTeX with circdia.sty package      ***************
%%
%%  ///////////////////////////////////////////////////////////////////
%%
%%  Copyright (c) 2018 - 2022 by
%%                  chipforge <stdcelllib@nospam.chipforge.org>
%%  All rights reserved.
%%
%%      This Standard Cell Library is licensed under the Libre Silicon
%%      public license; you can redistribute it and/or modify it under
%%      the terms of the Libre Silicon public license as published by
%%      the Libre Silicon alliance, either version 1 of the License, or
%%      (at your option) any later version.
%%
%%      This design is distributed in the hope that it will be useful,
%%      but WITHOUT ANY WARRANTY; without even the implied warranty of
%%      MERCHANTABILITY or FITNESS FOR A PARTICULAR PURPOSE.
%%      See the Libre Silicon Public License for more details.
%%
%%  ///////////////////////////////////////////////////////////////////
\begin{circuitdiagram}[draft]{25}{18}

    \usgate
    % ----  1st column  ----
    \pin{1}{1}{L}{A}
    \pin{1}{3}{L}{A1}
    \pin{1}{5}{L}{A2}
    \gate[\inputs{3}]{and}{5}{3}{R}{}{}

    \pin{1}{7}{L}{B}
    \pin{1}{9}{L}{B1}
    \pin{1}{11}{L}{B2}
    \gate[\inputs{3}]{and}{5}{9}{R}{}{}

    \pin{1}{13}{L}{C}
    \pin{1}{17}{L}{C1}
    \gate[\inputs{2}]{and}{5}{15}{R}{}{}

    % ----  2nd column  ----
    \wire{9}{3}{9}{7}
    \wire{9}{11}{9}{15}
    \gate[\inputs{3}]{or}{12}{9}{R}{}{}

    % ----  3rd column  ----
    \pin{15}{13}{L}{D}
    \gate[\inputs{2}]{nand}{19}{11}{R}{}{}

    % ----  result ----
    \pin{24}{11}{R}{Y}

\end{circuitdiagram}
 %%  ************    LibreSilicon's StdCellLibrary   *******************
%%
%%  Organisation:   Chipforge
%%                  Germany / European Union
%%
%%  Profile:        Chipforge focus on fine System-on-Chip Cores in
%%                  Verilog HDL Code which are easy understandable and
%%                  adjustable. For further information see
%%                          www.chipforge.org
%%                  there are projects from small cores up to PCBs, too.
%%
%%  File:           StdCellLib/Documents/Datasheets/Circuitry/AAAO3321.tex
%%
%%  Purpose:        Circuit File for AAAO3321
%%
%%  ************    LaTeX with circdia.sty package      ***************
%%
%%  ///////////////////////////////////////////////////////////////////
%%
%%  Copyright (c) 2018 - 2022 by
%%                  chipforge <stdcelllib@nospam.chipforge.org>
%%  All rights reserved.
%%
%%      This Standard Cell Library is licensed under the Libre Silicon
%%      public license; you can redistribute it and/or modify it under
%%      the terms of the Libre Silicon public license as published by
%%      the Libre Silicon alliance, either version 1 of the License, or
%%      (at your option) any later version.
%%
%%      This design is distributed in the hope that it will be useful,
%%      but WITHOUT ANY WARRANTY; without even the implied warranty of
%%      MERCHANTABILITY or FITNESS FOR A PARTICULAR PURPOSE.
%%      See the Libre Silicon Public License for more details.
%%
%%  ///////////////////////////////////////////////////////////////////
\begin{circuitdiagram}[draft]{31}{18}

    \usgate
    % ----  1st column  ----
    \pin{1}{1}{L}{A}
    \pin{1}{3}{L}{A1}
    \pin{1}{5}{L}{A2}
    \gate[\inputs{3}]{and}{5}{3}{R}{}{}

    \pin{1}{7}{L}{B}
    \pin{1}{9}{L}{B1}
    \pin{1}{11}{L}{B2}
    \gate[\inputs{3}]{and}{5}{9}{R}{}{}

    \pin{1}{13}{L}{C}
    \pin{1}{17}{L}{C1}
    \gate[\inputs{2}]{and}{5}{15}{R}{}{}

    % ----  2nd column  ----
    \wire{9}{3}{9}{7}
    \wire{9}{11}{9}{15}
    \gate[\inputs{3}]{or}{12}{9}{R}{}{}

    % ----  3rd column  ----
    \pin{15}{13}{L}{D}
    \gate[\inputs{2}]{nand}{19}{11}{R}{}{}

    % ----  4th column  ----
    \gate{not}{26}{11}{R}{}{}

    % ----  result ----
    \pin{30}{11}{R}{Z}

\end{circuitdiagram}

%%  ************    LibreSilicon's StdCellLibrary   *******************
%%
%%  Organisation:   Chipforge
%%                  Germany / European Union
%%
%%  Profile:        Chipforge focus on fine System-on-Chip Cores in
%%                  Verilog HDL Code which are easy understandable and
%%                  adjustable. For further information see
%%                          www.chipforge.org
%%                  there are projects from small cores up to PCBs, too.
%%
%%  File:           StdCellLib/Documents/Datasheets/Circuitry/AAAOI3331.tex
%%
%%  Purpose:        Circuit File for AAAOI3331
%%
%%  ************    LaTeX with circdia.sty package      ***************
%%
%%  ///////////////////////////////////////////////////////////////////
%%
%%  Copyright (c) 2018 - 2022 by
%%                  chipforge <stdcelllib@nospam.chipforge.org>
%%  All rights reserved.
%%
%%      This Standard Cell Library is licensed under the Libre Silicon
%%      public license; you can redistribute it and/or modify it under
%%      the terms of the Libre Silicon public license as published by
%%      the Libre Silicon alliance, either version 1 of the License, or
%%      (at your option) any later version.
%%
%%      This design is distributed in the hope that it will be useful,
%%      but WITHOUT ANY WARRANTY; without even the implied warranty of
%%      MERCHANTABILITY or FITNESS FOR A PARTICULAR PURPOSE.
%%      See the Libre Silicon Public License for more details.
%%
%%  ///////////////////////////////////////////////////////////////////
\begin{circuitdiagram}[draft]{25}{18}

    \usgate
    % ----  1st column  ----
    \pin{1}{1}{L}{A}
    \pin{1}{3}{L}{A1}
    \pin{1}{5}{L}{A2}
    \gate[\inputs{3}]{and}{5}{3}{R}{}{}

    \pin{1}{7}{L}{B}
    \pin{1}{9}{L}{B1}
    \pin{1}{11}{L}{B2}
    \gate[\inputs{3}]{and}{5}{9}{R}{}{}

    \pin{1}{13}{L}{C}
    \pin{1}{15}{L}{C1}
    \pin{1}{17}{L}{C2}
    \gate[\inputs{3}]{and}{5}{15}{R}{}{}

    % ----  2nd column  ----
    \wire{9}{3}{9}{7}
    \wire{9}{11}{9}{15}
    \gate[\inputs{3}]{or}{12}{9}{R}{}{}

    % ----  3rd column  ----
    \pin{15}{13}{L}{D}
    \gate[\inputs{2}]{nand}{19}{11}{R}{}{}

    % ----  result ----
    \pin{24}{11}{R}{Y}

\end{circuitdiagram}
 %%  ************    LibreSilicon's StdCellLibrary   *******************
%%
%%  Organisation:   Chipforge
%%                  Germany / European Union
%%
%%  Profile:        Chipforge focus on fine System-on-Chip Cores in
%%                  Verilog HDL Code which are easy understandable and
%%                  adjustable. For further information see
%%                          www.chipforge.org
%%                  there are projects from small cores up to PCBs, too.
%%
%%  File:           StdCellLib/Documents/Datasheets/Circuitry/AAAO3331.tex
%%
%%  Purpose:        Circuit File for AAAO3331
%%
%%  ************    LaTeX with circdia.sty package      ***************
%%
%%  ///////////////////////////////////////////////////////////////////
%%
%%  Copyright (c) 2018 - 2022 by
%%                  chipforge <stdcelllib@nospam.chipforge.org>
%%  All rights reserved.
%%
%%      This Standard Cell Library is licensed under the Libre Silicon
%%      public license; you can redistribute it and/or modify it under
%%      the terms of the Libre Silicon public license as published by
%%      the Libre Silicon alliance, either version 1 of the License, or
%%      (at your option) any later version.
%%
%%      This design is distributed in the hope that it will be useful,
%%      but WITHOUT ANY WARRANTY; without even the implied warranty of
%%      MERCHANTABILITY or FITNESS FOR A PARTICULAR PURPOSE.
%%      See the Libre Silicon Public License for more details.
%%
%%  ///////////////////////////////////////////////////////////////////
\begin{circuitdiagram}[draft]{31}{18}

    \usgate
    % ----  1st column  ----
    \pin{1}{1}{L}{A}
    \pin{1}{3}{L}{A1}
    \pin{1}{5}{L}{A2}
    \gate[\inputs{3}]{and}{5}{3}{R}{}{}

    \pin{1}{7}{L}{B}
    \pin{1}{9}{L}{B1}
    \pin{1}{11}{L}{B2}
    \gate[\inputs{3}]{and}{5}{9}{R}{}{}

    \pin{1}{13}{L}{C}
    \pin{1}{15}{L}{C1}
    \pin{1}{17}{L}{C2}
    \gate[\inputs{3}]{and}{5}{15}{R}{}{}

    % ----  2nd column  ----
    \wire{9}{3}{9}{7}
    \wire{9}{11}{9}{15}
    \gate[\inputs{3}]{or}{12}{9}{R}{}{}

    % ----  3rd column  ----
    \pin{15}{13}{L}{D}
    \gate[\inputs{2}]{nand}{19}{11}{R}{}{}

    % ----  4th column  ----
    \gate{not}{26}{11}{R}{}{}

    % ----  result ----
    \pin{30}{11}{R}{Z}

\end{circuitdiagram}


%%  ************    LibreSilicon's StdCellLibrary   *******************
%%
%%  Organisation:   Chipforge
%%                  Germany / European Union
%%
%%  Profile:        Chipforge focus on fine System-on-Chip Cores in
%%                  Verilog HDL Code which are easy understandable and
%%                  adjustable. For further information see
%%                          www.chipforge.org
%%                  there are projects from small cores up to PCBs, too.
%%
%%  File:           StdCellLib/Documents/Datasheets/Circuitry/AAAOI22211.tex
%%
%%  Purpose:        Circuit File for AAAOI22211
%%
%%  ************    LaTeX with circdia.sty package      ***************
%%
%%  ///////////////////////////////////////////////////////////////////
%%
%%  Copyright (c) 2018 - 2022 by
%%                  chipforge <stdcelllib@nospam.chipforge.org>
%%  All rights reserved.
%%
%%      This Standard Cell Library is licensed under the Libre Silicon
%%      public license; you can redistribute it and/or modify it under
%%      the terms of the Libre Silicon public license as published by
%%      the Libre Silicon alliance, either version 1 of the License, or
%%      (at your option) any later version.
%%
%%      This design is distributed in the hope that it will be useful,
%%      but WITHOUT ANY WARRANTY; without even the implied warranty of
%%      MERCHANTABILITY or FITNESS FOR A PARTICULAR PURPOSE.
%%      See the Libre Silicon Public License for more details.
%%
%%  ///////////////////////////////////////////////////////////////////
\begin{circuitdiagram}[draft]{26}{19}

    \usgate
    % ----  1st column  ----
    \pin{1}{1}{L}{A}
    \pin{1}{5}{L}{A1}
    \gate[\inputs{2}]{and}{5}{3}{R}{}{}

    \pin{1}{7}{L}{B}
    \pin{1}{11}{L}{B1}
    \gate[\inputs{2}]{and}{5}{9}{R}{}{}

    \pin{1}{13}{L}{C}
    \pin{1}{17}{L}{C1}
    \gate[\inputs{2}]{and}{5}{15}{R}{}{}

    % ----  2nd column  ----
    \wire{9}{3}{10}{3} \wire{10}{3}{10}{9}
    \wire{9}{9}{9}{11} \wire{9}{11}{10}{11}
    \wire{9}{13}{9}{15} \wire{9}{13}{10}{13}
    \pin{9}{18}{L}{D} \wire{10}{15}{10}{18}
    \gate[\inputs{4}]{or}{13}{12}{R}{}{}

    % ----  3rd column  ----
    \pin{16}{16}{L}{E}
    \gate[\inputs{2}]{nand}{20}{14}{R}{}{}

    % ----  result ----
    \pin{25}{14}{R}{Y}

\end{circuitdiagram}
 %%  ************    LibreSilicon's StdCellLibrary   *******************
%%
%%  Organisation:   Chipforge
%%                  Germany / European Union
%%
%%  Profile:        Chipforge focus on fine System-on-Chip Cores in
%%                  Verilog HDL Code which are easy understandable and
%%                  adjustable. For further information see
%%                          www.chipforge.org
%%                  there are projects from small cores up to PCBs, too.
%%
%%  File:           StdCellLib/Documents/Datasheets/Circuitry/AAAO22211.tex
%%
%%  Purpose:        Circuit File for AAAO22211
%%
%%  ************    LaTeX with circdia.sty package      ***************
%%
%%  ///////////////////////////////////////////////////////////////////
%%
%%  Copyright (c) 2018 - 2022 by
%%                  chipforge <stdcelllib@nospam.chipforge.org>
%%  All rights reserved.
%%
%%      This Standard Cell Library is licensed under the Libre Silicon
%%      public license; you can redistribute it and/or modify it under
%%      the terms of the Libre Silicon public license as published by
%%      the Libre Silicon alliance, either version 1 of the License, or
%%      (at your option) any later version.
%%
%%      This design is distributed in the hope that it will be useful,
%%      but WITHOUT ANY WARRANTY; without even the implied warranty of
%%      MERCHANTABILITY or FITNESS FOR A PARTICULAR PURPOSE.
%%      See the Libre Silicon Public License for more details.
%%
%%  ///////////////////////////////////////////////////////////////////
\begin{circuitdiagram}[draft]{32}{19}

    \usgate
    % ----  1st column  ----
    \pin{1}{1}{L}{A}
    \pin{1}{5}{L}{A1}
    \gate[\inputs{2}]{and}{5}{3}{R}{}{}

    \pin{1}{7}{L}{B}
    \pin{1}{11}{L}{B1}
    \gate[\inputs{2}]{and}{5}{9}{R}{}{}

    \pin{1}{13}{L}{C}
    \pin{1}{17}{L}{C1}
    \gate[\inputs{2}]{and}{5}{15}{R}{}{}

    % ----  2nd column  ----
    \wire{9}{3}{10}{3} \wire{10}{3}{10}{9}
    \wire{9}{9}{9}{11} \wire{9}{11}{10}{11}
    \wire{9}{13}{9}{15} \wire{9}{13}{10}{13}
    \pin{9}{18}{L}{D} \wire{10}{15}{10}{18}
    \gate[\inputs{4}]{or}{13}{12}{R}{}{}

    % ----  3rd column  ----
    \pin{16}{16}{L}{E}
    \gate[\inputs{2}]{nand}{20}{14}{R}{}{}

    % ----  last column ----
    \gate{not}{27}{14}{R}{}{}

    % ----  result ----
    \pin{31}{14}{R}{Z}

\end{circuitdiagram}


%%  ************    LibreSilicon's StdCellLibrary   *******************
%%
%%  Organisation:   Chipforge
%%                  Germany / European Union
%%
%%  Profile:        Chipforge focus on fine System-on-Chip Cores in
%%                  Verilog HDL Code which are easy understandable and
%%                  adjustable. For further information see
%%                          www.chipforge.org
%%                  there are projects from small cores up to PCBs, too.
%%
%%  File:           StdCellLib/Documents/Book/section-OOOAOI_complex.tex
%%
%%  Purpose:        Section Level File for Standard Cell Library Documentation
%%
%%  ************    LaTeX with circdia.sty package      ***************
%%
%%  ///////////////////////////////////////////////////////////////////
%%
%%  Copyright (c) 2018 - 2022 by
%%                  chipforge <stdcelllib@nospam.chipforge.org>
%%  All rights reserved.
%%
%%      This Standard Cell Library is licensed under the Libre Silicon
%%      public license; you can redistribute it and/or modify it under
%%      the terms of the Libre Silicon public license as published by
%%      the Libre Silicon alliance, either version 1 of the License, or
%%      (at your option) any later version.
%%
%%      This design is distributed in the hope that it will be useful,
%%      but WITHOUT ANY WARRANTY; without even the implied warranty of
%%      MERCHANTABILITY or FITNESS FOR A PARTICULAR PURPOSE.
%%      See the Libre Silicon Public License for more details.
%%
%%  ///////////////////////////////////////////////////////////////////
\section{OR-OR-OR-AND-OR(-Invert) Complex Gates}



%%  ------------    four phases     -----------------------------------

%%  ************    LibreSilicon's StdCellLibrary   *******************
%%
%%  Organisation:   Chipforge
%%                  Germany / European Union
%%
%%  Profile:        Chipforge focus on fine System-on-Chip Cores in
%%                  Verilog HDL Code which are easy understandable and
%%                  adjustable. For further information see
%%                          www.chipforge.org
%%                  there are projects from small cores up to PCBs, too.
%%
%%  File:           StdCellLib/Documents/section-AOAOI_complex.tex
%%
%%  Purpose:        Section Level File for Standard Cell Library Documentation
%%
%%  ************    LaTeX with circdia.sty package      ***************
%%
%%  ///////////////////////////////////////////////////////////////////
%%
%%  Copyright (c) 2018 - 2022 by
%%                  chipforge <stdcelllib@nospam.chipforge.org>
%%  All rights reserved.
%%
%%      This Standard Cell Library is licensed under the Libre Silicon
%%      public license; you can redistribute it and/or modify it under
%%      the terms of the Libre Silicon public license as published by
%%      the Libre Silicon alliance, either version 1 of the License, or
%%      (at your option) any later version.
%%
%%      This design is distributed in the hope that it will be useful,
%%      but WITHOUT ANY WARRANTY; without even the implied warranty of
%%      MERCHANTABILITY or FITNESS FOR A PARTICULAR PURPOSE.
%%      See the Libre Silicon Public License for more details.
%%
%%  ///////////////////////////////////////////////////////////////////
\section{AND-OR-AND-OR(-Invert) Complex Gates}

%%  ************    LibreSilicon's StdCellLibrary   *******************
%%
%%  Organisation:   Chipforge
%%                  Germany / European Union
%%
%%  Profile:        Chipforge focus on fine System-on-Chip Cores in
%%                  Verilog HDL Code which are easy understandable and
%%                  adjustable. For further information see
%%                          www.chipforge.org
%%                  there are projects from small cores up to PCBs, too.
%%
%%  File:           StdCellLib/Documents/Circuits/AOAOI2111.tex
%%
%%  Purpose:        Circuit File for AOAOI2111
%%
%%  ************    LaTeX with circdia.sty package      ***************
%%
%%  ///////////////////////////////////////////////////////////////////
%%
%%  Copyright (c) 2019 by chipforge <stdcelllib@nospam.chipforge.org>
%%  All rights reserved.
%%
%%      This Standard Cell Library is licensed under the Libre Silicon
%%      public license; you can redistribute it and/or modify it under
%%      the terms of the Libre Silicon public license as published by
%%      the Libre Silicon alliance, either version 1 of the License, or
%%      (at your option) any later version.
%%
%%      This design is distributed in the hope that it will be useful,
%%      but WITHOUT ANY WARRANTY; without even the implied warranty of
%%      MERCHANTABILITY or FITNESS FOR A PARTICULAR PURPOSE.
%%      See the Libre Silicon Public License for more details.
%%
%%  ///////////////////////////////////////////////////////////////////
\begin{circuitdiagram}{32}{12}

    \usgate
    \gate[\inputs{2}]{and}{5}{9}{R}{}{}   % AND
    \gate[\inputs{2}]{or}{12}{7}{R}{}{}   % OR
    \gate[\inputs{2}]{and}{19}{5}{R}{}{}  % AND
    \gate[\inputs{2}]{nor}{26}{3}{R}{}{}  % NOR
    \pin{1}{1}{L}{A}    % pin A
    \pin{1}{3}{L}{B}    % pin B
    \pin{1}{5}{L}{C}    % pin C
    \pin{1}{7}{L}{D}    % pin D
    \pin{1}{11}{L}{D1}  % pin D1
    \wire{2}{1}{23}{1}  % wire from pin A
    \wire{2}{3}{16}{3}  % wire from pin B
    \wire{2}{5}{9}{5}   % wire from pin C
    \pin{31}{3}{R}{Y}   % pin Y

\end{circuitdiagram}
 %%  ************    LibreSilicon's StdCellLibrary   *******************
%%
%%  Organisation:   Chipforge
%%                  Germany / European Union
%%
%%  Profile:        Chipforge focus on fine System-on-Chip Cores in
%%                  Verilog HDL Code which are easy understandable and
%%                  adjustable. For further information see
%%                          www.chipforge.org
%%                  there are projects from small cores up to PCBs, too.
%%
%%  File:           StdCellLib/Documents/Datasheets/Circuitry/AOAO2111.tex
%%
%%  Purpose:        Circuit File for AOAO2111
%%
%%  ************    LaTeX with circdia.sty package      ***************
%%
%%  ///////////////////////////////////////////////////////////////////
%%
%%  Copyright (c) 2018 - 2022 by
%%                  chipforge <stdcelllib@nospam.chipforge.org>
%%  All rights reserved.
%%
%%      This Standard Cell Library is licensed under the Libre Silicon
%%      public license; you can redistribute it and/or modify it under
%%      the terms of the Libre Silicon public license as published by
%%      the Libre Silicon alliance, either version 1 of the License, or
%%      (at your option) any later version.
%%
%%      This design is distributed in the hope that it will be useful,
%%      but WITHOUT ANY WARRANTY; without even the implied warranty of
%%      MERCHANTABILITY or FITNESS FOR A PARTICULAR PURPOSE.
%%      See the Libre Silicon Public License for more details.
%%
%%  ///////////////////////////////////////////////////////////////////
\begin{circuitdiagram}[draft]{38}{12}

    \usgate
    % ----  1st column  ----
    \pin{1}{1}{L}{A}
    \pin{1}{5}{L}{A1}
    \gate[\inputs{2}]{and}{5}{3}{R}{}{}

    % ----  2nd column  ----
    \pin{8}{7}{L}{B}
    \gate[\inputs{2}]{or}{12}{5}{R}{}{}

    % ----  3rd column  ----
    \pin{15}{9}{L}{C}
    \gate[\inputs{2}]{and}{19}{7}{R}{}{}

    % ----  4th column  ----
    \pin{22}{11}{L}{D}
    \gate[\inputs{2}]{nor}{26}{9}{R}{}{}

    % ----  5th column  ----
    \gate{not}{33}{9}{R}{}{}

    % ----  result ----
    \pin{37}{9}{R}{Z}

\end{circuitdiagram}

%%  ************    LibreSilicon's StdCellLibrary   *******************
%%
%%  Organisation:   Chipforge
%%                  Germany / European Union
%%
%%  Profile:        Chipforge focus on fine System-on-Chip Cores in
%%                  Verilog HDL Code which are easy understandable and
%%                  adjustable. For further information see
%%                          www.chipforge.org
%%                  there are projects from small cores up to PCBs, too.
%%
%%  File:           StdCellLib/Documents/Datasheets/Circuitry/AOAOI2112.tex
%%
%%  Purpose:        Circuit File for AOAOI2112
%%
%%  ************    LaTeX with circdia.sty package      ***************
%%
%%  ///////////////////////////////////////////////////////////////////
%%
%%  Copyright (c) 2018 - 2022 by
%%                  chipforge <stdcelllib@nospam.chipforge.org>
%%  All rights reserved.
%%
%%      This Standard Cell Library is licensed under the Libre Silicon
%%      public license; you can redistribute it and/or modify it under
%%      the terms of the Libre Silicon public license as published by
%%      the Libre Silicon alliance, either version 1 of the License, or
%%      (at your option) any later version.
%%
%%      This design is distributed in the hope that it will be useful,
%%      but WITHOUT ANY WARRANTY; without even the implied warranty of
%%      MERCHANTABILITY or FITNESS FOR A PARTICULAR PURPOSE.
%%      See the Libre Silicon Public License for more details.
%%
%%  ///////////////////////////////////////////////////////////////////
\begin{circuitdiagram}[draft]{32}{14}

    \usgate
    % ----  1st column  ----
    \pin{1}{1}{L}{A}
    \pin{1}{5}{L}{A1}
    \gate[\inputs{2}]{and}{5}{3}{R}{}{}

    % ----  2nd column  ----
    \pin{8}{7}{L}{B}
    \gate[\inputs{2}]{or}{12}{5}{R}{}{}

    % ----  3rd column  ----
    \pin{15}{9}{L}{C}
    \gate[\inputs{2}]{and}{19}{7}{R}{}{}

    % ----  4th column  ----
    \pin{22}{11}{L}{D}
    \pin{22}{13}{L}{D1}
    \wire{23}{7}{23}{9}
    \gate[\inputs{3}]{nor}{26}{11}{R}{}{}


    % ----  result ----
    \pin{31}{11}{R}{Y}

\end{circuitdiagram}
 %%  ************    LibreSilicon's StdCellLibrary   *******************
%%
%%  Organisation:   Chipforge
%%                  Germany / European Union
%%
%%  Profile:        Chipforge focus on fine System-on-Chip Cores in
%%                  Verilog HDL Code which are easy understandable and
%%                  adjustable. For further information see
%%                          www.chipforge.org
%%                  there are projects from small cores up to PCBs, too.
%%
%%  File:           StdCellLib/Documents/Datasheets/Circuitry/AOAO2112.tex
%%
%%  Purpose:        Circuit File for AOAO2112
%%
%%  ************    LaTeX with circdia.sty package      ***************
%%
%%  ///////////////////////////////////////////////////////////////////
%%
%%  Copyright (c) 2018 - 2022 by
%%                  chipforge <stdcelllib@nospam.chipforge.org>
%%  All rights reserved.
%%
%%      This Standard Cell Library is licensed under the Libre Silicon
%%      public license; you can redistribute it and/or modify it under
%%      the terms of the Libre Silicon public license as published by
%%      the Libre Silicon alliance, either version 1 of the License, or
%%      (at your option) any later version.
%%
%%      This design is distributed in the hope that it will be useful,
%%      but WITHOUT ANY WARRANTY; without even the implied warranty of
%%      MERCHANTABILITY or FITNESS FOR A PARTICULAR PURPOSE.
%%      See the Libre Silicon Public License for more details.
%%
%%  ///////////////////////////////////////////////////////////////////
\begin{circuitdiagram}[draft]{38}{14}

    \usgate
    % ----  1st column  ----
    \pin{1}{1}{L}{A}
    \pin{1}{5}{L}{A1}
    \gate[\inputs{2}]{and}{5}{3}{R}{}{}

    % ----  2nd column  ----
    \pin{8}{7}{L}{B}
    \gate[\inputs{2}]{or}{12}{5}{R}{}{}

    % ----  3rd column  ----
    \pin{15}{9}{L}{C}
    \gate[\inputs{2}]{and}{19}{7}{R}{}{}

    % ----  4th column  ----
    \pin{22}{11}{L}{D}
    \pin{22}{13}{L}{D1}
    \wire{23}{7}{23}{9}
    \gate[\inputs{3}]{nor}{26}{11}{R}{}{}

    % ----  5th column  ----
    \gate{not}{33}{11}{R}{}{}

    % ----  result ----
    \pin{37}{11}{R}{Z}

\end{circuitdiagram}

%%  ************    LibreSilicon's StdCellLibrary   *******************
%%
%%  Organisation:   Chipforge
%%                  Germany / European Union
%%
%%  Profile:        Chipforge focus on fine System-on-Chip Cores in
%%                  Verilog HDL Code which are easy understandable and
%%                  adjustable. For further information see
%%                          www.chipforge.org
%%                  there are projects from small cores up to PCBs, too.
%%
%%  File:           StdCellLib/Documents/Datasheets/Circuitry/AOAOI2121.tex
%%
%%  Purpose:        Circuit File for AOAOI2121
%%
%%  ************    LaTeX with circdia.sty package      ***************
%%
%%  ///////////////////////////////////////////////////////////////////
%%
%%  Copyright (c) 2018 - 2022 by
%%                  chipforge <stdcelllib@nospam.chipforge.org>
%%  All rights reserved.
%%
%%      This Standard Cell Library is licensed under the Libre Silicon
%%      public license; you can redistribute it and/or modify it under
%%      the terms of the Libre Silicon public license as published by
%%      the Libre Silicon alliance, either version 1 of the License, or
%%      (at your option) any later version.
%%
%%      This design is distributed in the hope that it will be useful,
%%      but WITHOUT ANY WARRANTY; without even the implied warranty of
%%      MERCHANTABILITY or FITNESS FOR A PARTICULAR PURPOSE.
%%      See the Libre Silicon Public License for more details.
%%
%%  ///////////////////////////////////////////////////////////////////
\begin{circuitdiagram}[draft]{32}{14}

    \usgate
    % ----  1st column  ----
    \pin{1}{1}{L}{A}
    \pin{1}{5}{L}{A1}
    \gate[\inputs{2}]{and}{5}{3}{R}{}{}

    % ----  2nd column  ----
    \pin{8}{7}{L}{B}
    \gate[\inputs{2}]{or}{12}{5}{R}{}{}

    % ----  3rd column  ----
    \pin{15}{9}{L}{C}
    \pin{15}{11}{L}{C1}
    \wire{16}{5}{16}{7}
    \gate[\inputs{3}]{and}{19}{9}{R}{}{}

    % ----  4th column  ----
    \pin{22}{13}{L}{D}
    \gate[\inputs{2}]{nor}{26}{11}{R}{}{}


    % ----  result ----
    \pin{31}{11}{R}{Y}

\end{circuitdiagram}
 %%  ************    LibreSilicon's StdCellLibrary   *******************
%%
%%  Organisation:   Chipforge
%%                  Germany / European Union
%%
%%  Profile:        Chipforge focus on fine System-on-Chip Cores in
%%                  Verilog HDL Code which are easy understandable and
%%                  adjustable. For further information see
%%                          www.chipforge.org
%%                  there are projects from small cores up to PCBs, too.
%%
%%  File:           StdCellLib/Documents/Datasheets/Circuitry/AOAO2121.tex
%%
%%  Purpose:        Circuit File for AOAO2121
%%
%%  ************    LaTeX with circdia.sty package      ***************
%%
%%  ///////////////////////////////////////////////////////////////////
%%
%%  Copyright (c) 2018 - 2022 by
%%                  chipforge <stdcelllib@nospam.chipforge.org>
%%  All rights reserved.
%%
%%      This Standard Cell Library is licensed under the Libre Silicon
%%      public license; you can redistribute it and/or modify it under
%%      the terms of the Libre Silicon public license as published by
%%      the Libre Silicon alliance, either version 1 of the License, or
%%      (at your option) any later version.
%%
%%      This design is distributed in the hope that it will be useful,
%%      but WITHOUT ANY WARRANTY; without even the implied warranty of
%%      MERCHANTABILITY or FITNESS FOR A PARTICULAR PURPOSE.
%%      See the Libre Silicon Public License for more details.
%%
%%  ///////////////////////////////////////////////////////////////////
\begin{circuitdiagram}[draft]{38}{14}

    \usgate
    % ----  1st column  ----
    \pin{1}{1}{L}{A}
    \pin{1}{5}{L}{A1}
    \gate[\inputs{2}]{and}{5}{3}{R}{}{}

    % ----  2nd column  ----
    \pin{8}{7}{L}{B}
    \gate[\inputs{2}]{or}{12}{5}{R}{}{}

    % ----  3rd column  ----
    \pin{15}{9}{L}{C}
    \pin{15}{11}{L}{C1}
    \wire{16}{5}{16}{7}
    \gate[\inputs{3}]{and}{19}{9}{R}{}{}

    % ----  4th column  ----
    \pin{22}{13}{L}{D}
    \gate[\inputs{2}]{nor}{26}{11}{R}{}{}

    % ----  5th column  ----
    \gate{not}{33}{11}{R}{}{}

    % ----  result ----
    \pin{37}{11}{R}{Z}

\end{circuitdiagram}

%%  ************    LibreSilicon's StdCellLibrary   *******************
%%
%%  Organisation:   Chipforge
%%                  Germany / European Union
%%
%%  Profile:        Chipforge focus on fine System-on-Chip Cores in
%%                  Verilog HDL Code which are easy understandable and
%%                  adjustable. For further information see
%%                          www.chipforge.org
%%                  there are projects from small cores up to PCBs, too.
%%
%%  File:           StdCellLib/Documents/Datasheets/Circuitry/AOAOI2122.tex
%%
%%  Purpose:        Circuit File for AOAOI2122
%%
%%  ************    LaTeX with circdia.sty package      ***************
%%
%%  ///////////////////////////////////////////////////////////////////
%%
%%  Copyright (c) 2018 - 2022 by
%%                  chipforge <stdcelllib@nospam.chipforge.org>
%%  All rights reserved.
%%
%%      This Standard Cell Library is licensed under the Libre Silicon
%%      public license; you can redistribute it and/or modify it under
%%      the terms of the Libre Silicon public license as published by
%%      the Libre Silicon alliance, either version 1 of the License, or
%%      (at your option) any later version.
%%
%%      This design is distributed in the hope that it will be useful,
%%      but WITHOUT ANY WARRANTY; without even the implied warranty of
%%      MERCHANTABILITY or FITNESS FOR A PARTICULAR PURPOSE.
%%      See the Libre Silicon Public License for more details.
%%
%%  ///////////////////////////////////////////////////////////////////
\begin{circuitdiagram}[draft]{32}{16}

    \usgate
    % ----  1st column  ----
    \pin{1}{1}{L}{A}
    \pin{1}{5}{L}{A1}
    \gate[\inputs{2}]{and}{5}{3}{R}{}{}

    % ----  2nd column  ----
    \pin{8}{7}{L}{B}
    \gate[\inputs{2}]{or}{12}{5}{R}{}{}

    % ----  3rd column  ----
    \pin{15}{9}{L}{C}
    \pin{15}{11}{L}{C1}
    \wire{16}{5}{16}{7}
    \gate[\inputs{3}]{and}{19}{9}{R}{}{}

    % ----  4th column  ----
    \pin{22}{13}{L}{D}
    \pin{22}{15}{L}{D1}
    \wire{23}{9}{23}{11}
    \gate[\inputs{3}]{nor}{26}{13}{R}{}{}


    % ----  result ----
    \pin{31}{13}{R}{Y}

\end{circuitdiagram}
 %%  ************    LibreSilicon's StdCellLibrary   *******************
%%
%%  Organisation:   Chipforge
%%                  Germany / European Union
%%
%%  Profile:        Chipforge focus on fine System-on-Chip Cores in
%%                  Verilog HDL Code which are easy understandable and
%%                  adjustable. For further information see
%%                          www.chipforge.org
%%                  there are projects from small cores up to PCBs, too.
%%
%%  File:           StdCellLib/Documents/Datasheets/Circuitry/AOAO2122.tex
%%
%%  Purpose:        Circuit File for AOAO2122
%%
%%  ************    LaTeX with circdia.sty package      ***************
%%
%%  ///////////////////////////////////////////////////////////////////
%%
%%  Copyright (c) 2018 - 2022 by
%%                  chipforge <stdcelllib@nospam.chipforge.org>
%%  All rights reserved.
%%
%%      This Standard Cell Library is licensed under the Libre Silicon
%%      public license; you can redistribute it and/or modify it under
%%      the terms of the Libre Silicon public license as published by
%%      the Libre Silicon alliance, either version 1 of the License, or
%%      (at your option) any later version.
%%
%%      This design is distributed in the hope that it will be useful,
%%      but WITHOUT ANY WARRANTY; without even the implied warranty of
%%      MERCHANTABILITY or FITNESS FOR A PARTICULAR PURPOSE.
%%      See the Libre Silicon Public License for more details.
%%
%%  ///////////////////////////////////////////////////////////////////
\begin{circuitdiagram}[draft]{38}{16}

    \usgate
    % ----  1st column  ----
    \pin{1}{1}{L}{A}
    \pin{1}{5}{L}{A1}
    \gate[\inputs{2}]{and}{5}{3}{R}{}{}

    % ----  2nd column  ----
    \pin{8}{7}{L}{B}
    \gate[\inputs{2}]{or}{12}{5}{R}{}{}

    % ----  3rd column  ----
    \pin{15}{9}{L}{C}
    \pin{15}{11}{L}{C1}
    \wire{16}{5}{16}{7}
    \gate[\inputs{3}]{and}{19}{9}{R}{}{}

    % ----  4th column  ----
    \pin{22}{13}{L}{D}
    \pin{22}{15}{L}{D1}
    \wire{23}{9}{23}{11}
    \gate[\inputs{3}]{nor}{26}{13}{R}{}{}

    % ----  5th column  ----
    \gate{not}{33}{13}{R}{}{}

    % ----  result ----
    \pin{37}{13}{R}{Z}

\end{circuitdiagram}

%%  ************    LibreSilicon's StdCellLibrary   *******************
%%
%%  Organisation:   Chipforge
%%                  Germany / European Union
%%
%%  Profile:        Chipforge focus on fine System-on-Chip Cores in
%%                  Verilog HDL Code which are easy understandable and
%%                  adjustable. For further information see
%%                          www.chipforge.org
%%                  there are projects from small cores up to PCBs, too.
%%
%%  File:           StdCellLib/Documents/Circuits/AOAOI2211.tex
%%
%%  Purpose:        Circuit File for AOAOI2211
%%
%%  ************    LaTeX with circdia.sty package      ***************
%%
%%  ///////////////////////////////////////////////////////////////////
%%
%%  Copyright (c) 2019 by chipforge <stdcelllib@nospam.chipforge.org>
%%  All rights reserved.
%%
%%      This Standard Cell Library is licensed under the Libre Silicon
%%      public license; you can redistribute it and/or modify it under
%%      the terms of the Libre Silicon public license as published by
%%      the Libre Silicon alliance, either version 1 of the License, or
%%      (at your option) any later version.
%%
%%      This design is distributed in the hope that it will be useful,
%%      but WITHOUT ANY WARRANTY; without even the implied warranty of
%%      MERCHANTABILITY or FITNESS FOR A PARTICULAR PURPOSE.
%%      See the Libre Silicon Public License for more details.
%%
%%  ///////////////////////////////////////////////////////////////////
\begin{circuitdiagram}{32}{14}

    \usgate
    \gate[\inputs{2}]{and}{5}{11}{R}{}{}  % AND
    \gate[\inputs{3}]{or}{12}{7}{R}{}{}   % OR
    \gate[\inputs{2}]{and}{19}{5}{R}{}{}  % AND
    \gate[\inputs{2}]{nor}{26}{3}{R}{}{}  % NOR
    \pin{1}{1}{L}{A}    % pin A
    \pin{1}{3}{L}{B}    % pin B
    \pin{1}{5}{L}{C}    % pin C
    \pin{1}{7}{L}{C1}   % pin C1
    \pin{1}{9}{L}{D}    % pin D
    \pin{1}{13}{L}{D1}  % pin D1
    \wire{2}{1}{23}{1}  % wire from pin A
    \wire{2}{3}{16}{3}  % wire from pin B
    \wire{2}{5}{9}{5}   % wire from pin C
    \wire{2}{7}{9}{7}   % wire from pin C1
    \wire{9}{9}{9}{11}  % wire between AND and OR
    \pin{31}{3}{R}{Y}   % pin Y

\end{circuitdiagram}
 %%  ************    LibreSilicon's StdCellLibrary   *******************
%%
%%  Organisation:   Chipforge
%%                  Germany / European Union
%%
%%  Profile:        Chipforge focus on fine System-on-Chip Cores in
%%                  Verilog HDL Code which are easy understandable and
%%                  adjustable. For further information see
%%                          www.chipforge.org
%%                  there are projects from small cores up to PCBs, too.
%%
%%  File:           StdCellLib/Documents/Datasheets/Circuitry/AOAO2211.tex
%%
%%  Purpose:        Circuit File for AOAO2211
%%
%%  ************    LaTeX with circdia.sty package      ***************
%%
%%  ///////////////////////////////////////////////////////////////////
%%
%%  Copyright (c) 2018 - 2022 by
%%                  chipforge <stdcelllib@nospam.chipforge.org>
%%  All rights reserved.
%%
%%      This Standard Cell Library is licensed under the Libre Silicon
%%      public license; you can redistribute it and/or modify it under
%%      the terms of the Libre Silicon public license as published by
%%      the Libre Silicon alliance, either version 1 of the License, or
%%      (at your option) any later version.
%%
%%      This design is distributed in the hope that it will be useful,
%%      but WITHOUT ANY WARRANTY; without even the implied warranty of
%%      MERCHANTABILITY or FITNESS FOR A PARTICULAR PURPOSE.
%%      See the Libre Silicon Public License for more details.
%%
%%  ///////////////////////////////////////////////////////////////////
\begin{circuitdiagram}[draft]{38}{14}

    \usgate
    % ----  1st column  ----
    \pin{1}{1}{L}{A}
    \pin{1}{5}{L}{A1}
    \gate[\inputs{2}]{and}{5}{3}{R}{}{}

    % ----  2nd column  ----
    \pin{8}{7}{L}{B}
    \pin{8}{9}{L}{B1}
    \wire{9}{3}{9}{5}
    \gate[\inputs{3}]{or}{12}{7}{R}{}{}

    % ----  3rd column  ----
    \pin{15}{11}{L}{C}
    \gate[\inputs{2}]{and}{19}{9}{R}{}{}

    % ----  4th column  ----
    \pin{22}{13}{L}{D}
    \gate[\inputs{2}]{nor}{26}{11}{R}{}{}

    % ----  5th column  ----
    \gate{not}{33}{11}{R}{}{}

    % ----  result ----
    \pin{37}{11}{R}{Z}

\end{circuitdiagram}

%%  ************    LibreSilicon's StdCellLibrary   *******************
%%
%%  Organisation:   Chipforge
%%                  Germany / European Union
%%
%%  Profile:        Chipforge focus on fine System-on-Chip Cores in
%%                  Verilog HDL Code which are easy understandable and
%%                  adjustable. For further information see
%%                          www.chipforge.org
%%                  there are projects from small cores up to PCBs, too.
%%
%%  File:           StdCellLib/Documents/Datasheets/Circuitry/AOAOI2221.tex
%%
%%  Purpose:        Circuit File for AOAOI2221
%%
%%  ************    LaTeX with circdia.sty package      ***************
%%
%%  ///////////////////////////////////////////////////////////////////
%%
%%  Copyright (c) 2018 - 2022 by
%%                  chipforge <stdcelllib@nospam.chipforge.org>
%%  All rights reserved.
%%
%%      This Standard Cell Library is licensed under the Libre Silicon
%%      public license; you can redistribute it and/or modify it under
%%      the terms of the Libre Silicon public license as published by
%%      the Libre Silicon alliance, either version 1 of the License, or
%%      (at your option) any later version.
%%
%%      This design is distributed in the hope that it will be useful,
%%      but WITHOUT ANY WARRANTY; without even the implied warranty of
%%      MERCHANTABILITY or FITNESS FOR A PARTICULAR PURPOSE.
%%      See the Libre Silicon Public License for more details.
%%
%%  ///////////////////////////////////////////////////////////////////
\begin{circuitdiagram}[draft]{32}{16}

    \usgate
    % ----  1st column  ----
    \pin{1}{1}{L}{A}
    \pin{1}{5}{L}{A1}
    \gate[\inputs{2}]{and}{5}{3}{R}{}{}

    % ----  2nd column  ----
    \pin{8}{7}{L}{B}
    \pin{8}{9}{L}{B1}
    \wire{9}{3}{9}{5}
    \gate[\inputs{3}]{or}{12}{7}{R}{}{}

    % ----  3rd column  ----
    \wire{16}{7}{16}{9}
    \pin{15}{11}{L}{C}
    \pin{15}{13}{L}{C1}
    \gate[\inputs{3}]{and}{19}{11}{R}{}{}

    % ----  4th column  ----
    \pin{22}{15}{L}{D}
    \gate[\inputs{2}]{nor}{26}{13}{R}{}{}


    % ----  result ----
    \pin{31}{13}{R}{Y}

\end{circuitdiagram}
 %%  ************    LibreSilicon's StdCellLibrary   *******************
%%
%%  Organisation:   Chipforge
%%                  Germany / European Union
%%
%%  Profile:        Chipforge focus on fine System-on-Chip Cores in
%%                  Verilog HDL Code which are easy understandable and
%%                  adjustable. For further information see
%%                          www.chipforge.org
%%                  there are projects from small cores up to PCBs, too.
%%
%%  File:           StdCellLib/Documents/Datasheets/Circuitry/AOAO2221.tex
%%
%%  Purpose:        Circuit File for AOAO2221
%%
%%  ************    LaTeX with circdia.sty package      ***************
%%
%%  ///////////////////////////////////////////////////////////////////
%%
%%  Copyright (c) 2018 - 2022 by
%%                  chipforge <stdcelllib@nospam.chipforge.org>
%%  All rights reserved.
%%
%%      This Standard Cell Library is licensed under the Libre Silicon
%%      public license; you can redistribute it and/or modify it under
%%      the terms of the Libre Silicon public license as published by
%%      the Libre Silicon alliance, either version 1 of the License, or
%%      (at your option) any later version.
%%
%%      This design is distributed in the hope that it will be useful,
%%      but WITHOUT ANY WARRANTY; without even the implied warranty of
%%      MERCHANTABILITY or FITNESS FOR A PARTICULAR PURPOSE.
%%      See the Libre Silicon Public License for more details.
%%
%%  ///////////////////////////////////////////////////////////////////
\begin{circuitdiagram}[draft]{38}{16}

    \usgate
    % ----  1st column  ----
    \pin{1}{1}{L}{A}
    \pin{1}{5}{L}{A1}
    \gate[\inputs{2}]{and}{5}{3}{R}{}{}

    % ----  2nd column  ----
    \pin{8}{7}{L}{B}
    \pin{8}{9}{L}{B1}
    \wire{9}{3}{9}{5}
    \gate[\inputs{3}]{or}{12}{7}{R}{}{}

    % ----  3rd column  ----
    \wire{16}{7}{16}{9}
    \pin{15}{11}{L}{C}
    \pin{15}{13}{L}{C1}
    \gate[\inputs{3}]{and}{19}{11}{R}{}{}

    % ----  4th column  ----
    \pin{22}{15}{L}{D}
    \gate[\inputs{2}]{nor}{26}{13}{R}{}{}

    % ----  last column ----
    \gate{not}{33}{13}{R}{}{}

    % ----  result ----
    \pin{37}{13}{R}{Z}

\end{circuitdiagram}

%%  ************    LibreSilicon's StdCellLibrary   *******************
%%
%%  Organisation:   Chipforge
%%                  Germany / European Union
%%
%%  Profile:        Chipforge focus on fine System-on-Chip Cores in
%%                  Verilog HDL Code which are easy understandable and
%%                  adjustable. For further information see
%%                          www.chipforge.org
%%                  there are projects from small cores up to PCBs, too.
%%
%%  File:           StdCellLib/Documents/Datasheets/Circuitry/AOAOI3111.tex
%%
%%  Purpose:        Circuit File for AOAOI3111
%%
%%  ************    LaTeX with circdia.sty package      ***************
%%
%%  ///////////////////////////////////////////////////////////////////
%%
%%  Copyright (c) 2018 - 2022 by
%%                  chipforge <stdcelllib@nospam.chipforge.org>
%%  All rights reserved.
%%
%%      This Standard Cell Library is licensed under the Libre Silicon
%%      public license; you can redistribute it and/or modify it under
%%      the terms of the Libre Silicon public license as published by
%%      the Libre Silicon alliance, either version 1 of the License, or
%%      (at your option) any later version.
%%
%%      This design is distributed in the hope that it will be useful,
%%      but WITHOUT ANY WARRANTY; without even the implied warranty of
%%      MERCHANTABILITY or FITNESS FOR A PARTICULAR PURPOSE.
%%      See the Libre Silicon Public License for more details.
%%
%%  ///////////////////////////////////////////////////////////////////
\begin{circuitdiagram}[draft]{32}{12}

    \usgate
    % ----  1st column  ----
    \pin{1}{1}{L}{A}
    \pin{1}{3}{L}{A1}
    \pin{1}{5}{L}{A2}
    \gate[\inputs{3}]{and}{5}{3}{R}{}{}

    % ----  2nd column  ----
    \pin{8}{7}{L}{B}
    \gate[\inputs{2}]{or}{12}{5}{R}{}{}

    % ----  3rd column  ----
    \pin{15}{9}{L}{C}
    \gate[\inputs{2}]{and}{19}{7}{R}{}{}

    % ----  4th column  ----
    \pin{22}{11}{L}{D}
    \gate[\inputs{2}]{nor}{26}{9}{R}{}{}


    % ----  result ----
    \pin{31}{9}{R}{Y}

\end{circuitdiagram}
 %%  ************    LibreSilicon's StdCellLibrary   *******************
%%
%%  Organisation:   Chipforge
%%                  Germany / European Union
%%
%%  Profile:        Chipforge focus on fine System-on-Chip Cores in
%%                  Verilog HDL Code which are easy understandable and
%%                  adjustable. For further information see
%%                          www.chipforge.org
%%                  there are projects from small cores up to PCBs, too.
%%
%%  File:           StdCellLib/Documents/Datasheets/Circuitry/AOAO3111.tex
%%
%%  Purpose:        Circuit File for AOAO3111
%%
%%  ************    LaTeX with circdia.sty package      ***************
%%
%%  ///////////////////////////////////////////////////////////////////
%%
%%  Copyright (c) 2018 - 2022 by
%%                  chipforge <stdcelllib@nospam.chipforge.org>
%%  All rights reserved.
%%
%%      This Standard Cell Library is licensed under the Libre Silicon
%%      public license; you can redistribute it and/or modify it under
%%      the terms of the Libre Silicon public license as published by
%%      the Libre Silicon alliance, either version 1 of the License, or
%%      (at your option) any later version.
%%
%%      This design is distributed in the hope that it will be useful,
%%      but WITHOUT ANY WARRANTY; without even the implied warranty of
%%      MERCHANTABILITY or FITNESS FOR A PARTICULAR PURPOSE.
%%      See the Libre Silicon Public License for more details.
%%
%%  ///////////////////////////////////////////////////////////////////
\begin{circuitdiagram}[draft]{38}{12}

    \usgate
    % ----  1st column  ----
    \pin{1}{1}{L}{A}
    \pin{1}{3}{L}{A1}
    \pin{1}{5}{L}{A2}
    \gate[\inputs{3}]{and}{5}{3}{R}{}{}

    % ----  2nd column  ----
    \pin{8}{7}{L}{B}
    \gate[\inputs{2}]{or}{12}{5}{R}{}{}

    % ----  3rd column  ----
    \pin{15}{9}{L}{C}
    \gate[\inputs{2}]{and}{19}{7}{R}{}{}

    % ----  4th column  ----
    \pin{22}{11}{L}{D}
    \gate[\inputs{2}]{nor}{26}{9}{R}{}{}

    % ----  5th column  ----
    \gate{not}{33}{9}{R}{}{}

    % ----  result ----
    \pin{37}{9}{R}{Z}

\end{circuitdiagram}

%%  ************    LibreSilicon's StdCellLibrary   *******************
%%
%%  Organisation:   Chipforge
%%                  Germany / European Union
%%
%%  Profile:        Chipforge focus on fine System-on-Chip Cores in
%%                  Verilog HDL Code which are easy understandable and
%%                  adjustable. For further information see
%%                          www.chipforge.org
%%                  there are projects from small cores up to PCBs, too.
%%
%%  File:           StdCellLib/Documents/Datasheets/Circuitry/AOAOI3112.tex
%%
%%  Purpose:        Circuit File for AOAOI3112
%%
%%  ************    LaTeX with circdia.sty package      ***************
%%
%%  ///////////////////////////////////////////////////////////////////
%%
%%  Copyright (c) 2018 - 2022 by
%%                  chipforge <stdcelllib@nospam.chipforge.org>
%%  All rights reserved.
%%
%%      This Standard Cell Library is licensed under the Libre Silicon
%%      public license; you can redistribute it and/or modify it under
%%      the terms of the Libre Silicon public license as published by
%%      the Libre Silicon alliance, either version 1 of the License, or
%%      (at your option) any later version.
%%
%%      This design is distributed in the hope that it will be useful,
%%      but WITHOUT ANY WARRANTY; without even the implied warranty of
%%      MERCHANTABILITY or FITNESS FOR A PARTICULAR PURPOSE.
%%      See the Libre Silicon Public License for more details.
%%
%%  ///////////////////////////////////////////////////////////////////
\begin{circuitdiagram}[draft]{32}{14}

    \usgate
    % ----  1st column  ----
    \pin{1}{1}{L}{A}
    \pin{1}{3}{L}{A1}
    \pin{1}{5}{L}{A2}
    \gate[\inputs{3}]{and}{5}{3}{R}{}{}

    % ----  2nd column  ----
    \pin{8}{7}{L}{B}
    \gate[\inputs{2}]{or}{12}{5}{R}{}{}

    % ----  3rd column  ----
    \pin{15}{9}{L}{C}
    \gate[\inputs{2}]{and}{19}{7}{R}{}{}

    % ----  4th column  ----
    \pin{22}{11}{L}{D}
    \pin{22}{13}{L}{D1}
    \wire{23}{7}{23}{9}
    \gate[\inputs{3}]{nor}{26}{11}{R}{}{}


    % ----  result ----
    \pin{31}{11}{R}{Y}

\end{circuitdiagram}
 %%  ************    LibreSilicon's StdCellLibrary   *******************
%%
%%  Organisation:   Chipforge
%%                  Germany / European Union
%%
%%  Profile:        Chipforge focus on fine System-on-Chip Cores in
%%                  Verilog HDL Code which are easy understandable and
%%                  adjustable. For further information see
%%                          www.chipforge.org
%%                  there are projects from small cores up to PCBs, too.
%%
%%  File:           StdCellLib/Documents/Datasheets/Circuitry/AOAO3112.tex
%%
%%  Purpose:        Circuit File for AOAO3112
%%
%%  ************    LaTeX with circdia.sty package      ***************
%%
%%  ///////////////////////////////////////////////////////////////////
%%
%%  Copyright (c) 2018 - 2022 by
%%                  chipforge <stdcelllib@nospam.chipforge.org>
%%  All rights reserved.
%%
%%      This Standard Cell Library is licensed under the Libre Silicon
%%      public license; you can redistribute it and/or modify it under
%%      the terms of the Libre Silicon public license as published by
%%      the Libre Silicon alliance, either version 1 of the License, or
%%      (at your option) any later version.
%%
%%      This design is distributed in the hope that it will be useful,
%%      but WITHOUT ANY WARRANTY; without even the implied warranty of
%%      MERCHANTABILITY or FITNESS FOR A PARTICULAR PURPOSE.
%%      See the Libre Silicon Public License for more details.
%%
%%  ///////////////////////////////////////////////////////////////////
\begin{circuitdiagram}[draft]{38}{14}

    \usgate
    % ----  1st column  ----
    \pin{1}{1}{L}{A}
    \pin{1}{3}{L}{A1}
    \pin{1}{5}{L}{A2}
    \gate[\inputs{3}]{and}{5}{3}{R}{}{}

    % ----  2nd column  ----
    \pin{8}{7}{L}{B}
    \gate[\inputs{2}]{or}{12}{5}{R}{}{}

    % ----  3rd column  ----
    \pin{15}{9}{L}{C}
    \gate[\inputs{2}]{and}{19}{7}{R}{}{}

    % ----  4th column  ----
    \pin{22}{11}{L}{D}
    \pin{22}{13}{L}{D1}
    \wire{23}{7}{23}{9}
    \gate[\inputs{3}]{nor}{26}{11}{R}{}{}

    % ----  5th column  ----
    \gate{not}{33}{11}{R}{}{}

    % ----  result ----
    \pin{37}{11}{R}{Z}

\end{circuitdiagram}

%%  ************    LibreSilicon's StdCellLibrary   *******************
%%
%%  Organisation:   Chipforge
%%                  Germany / European Union
%%
%%  Profile:        Chipforge focus on fine System-on-Chip Cores in
%%                  Verilog HDL Code which are easy understandable and
%%                  adjustable. For further information see
%%                          www.chipforge.org
%%                  there are projects from small cores up to PCBs, too.
%%
%%  File:           StdCellLib/Documents/Circuits/AOAOI3211.tex
%%
%%  Purpose:        Circuit File for AOAOI3211
%%
%%  ************    LaTeX with circdia.sty package      ***************
%%
%%  ///////////////////////////////////////////////////////////////////
%%
%%  Copyright (c) 2019 by chipforge <stdcelllib@nospam.chipforge.org>
%%  All rights reserved.
%%
%%      This Standard Cell Library is licensed under the Libre Silicon
%%      public license; you can redistribute it and/or modify it under
%%      the terms of the Libre Silicon public license as published by
%%      the Libre Silicon alliance, either version 1 of the License, or
%%      (at your option) any later version.
%%
%%      This design is distributed in the hope that it will be useful,
%%      but WITHOUT ANY WARRANTY; without even the implied warranty of
%%      MERCHANTABILITY or FITNESS FOR A PARTICULAR PURPOSE.
%%      See the Libre Silicon Public License for more details.
%%
%%  ///////////////////////////////////////////////////////////////////
\begin{circuitdiagram}{32}{14}

    \usgate
    \gate[\inputs{3}]{and}{5}{11}{R}{}{} % AND
    \gate[\inputs{3}]{or}{12}{7}{R}{}{}  % OR
    \gate[\inputs{2}]{and}{19}{5}{R}{}{} % AND
    \gate[\inputs{2}]{nor}{26}{3}{R}{}{} % NOR
    \pin{1}{1}{L}{A}    % pin A
    \pin{1}{3}{L}{B}    % pin B
    \pin{1}{5}{L}{C}    % pin C
    \pin{1}{7}{L}{C1}   % pin C1
    \pin{1}{9}{L}{D}    % pin D
    \pin{1}{11}{L}{D1}  % pin D1
    \pin{1}{13}{L}{D2}  % pin D2
    \wire{2}{1}{23}{1}  % wire from pin A
    \wire{2}{3}{16}{3}  % wire from pin B
    \wire{2}{5}{9}{5}   % wire from pin C
    \wire{2}{7}{9}{7}   % wire from pin C1
    \wire{9}{9}{9}{11}  % wire between AND and OR
    \pin{31}{3}{R}{Y}   % pin Y

\end{circuitdiagram}
 %%  ************    LibreSilicon's StdCellLibrary   *******************
%%
%%  Organisation:   Chipforge
%%                  Germany / European Union
%%
%%  Profile:        Chipforge focus on fine System-on-Chip Cores in
%%                  Verilog HDL Code which are easy understandable and
%%                  adjustable. For further information see
%%                          www.chipforge.org
%%                  there are projects from small cores up to PCBs, too.
%%
%%  File:           StdCellLib/Documents/Circuits/AOAO3211.tex
%%
%%  Purpose:        Circuit File for AOAO3211
%%
%%  ************    LaTeX with circdia.sty package      ***************
%%
%%  ///////////////////////////////////////////////////////////////////
%%
%%  Copyright (c) 2019 by chipforge <stdcelllib@nospam.chipforge.org>
%%  All rights reserved.
%%
%%      This Standard Cell Library is licensed under the Libre Silicon
%%      public license; you can redistribute it and/or modify it under
%%      the terms of the Libre Silicon public license as published by
%%      the Libre Silicon alliance, either version 1 of the License, or
%%      (at your option) any later version.
%%
%%      This design is distributed in the hope that it will be useful,
%%      but WITHOUT ANY WARRANTY; without even the implied warranty of
%%      MERCHANTABILITY or FITNESS FOR A PARTICULAR PURPOSE.
%%      See the Libre Silicon Public License for more details.
%%
%%  ///////////////////////////////////////////////////////////////////
\begin{circuitdiagram}{38}{14}

    \usgate
    \gate[\inputs{3}]{and}{5}{11}{R}{}{} % AND
    \gate[\inputs{3}]{or}{12}{7}{R}{}{}  % OR
    \gate[\inputs{2}]{and}{19}{5}{R}{}{} % AND
    \gate[\inputs{2}]{nor}{26}{3}{R}{}{} % NOR
    \gate{not}{33}{3}{R}{}{}  % NOT
    \pin{1}{1}{L}{A}    % pin A
    \pin{1}{3}{L}{B}    % pin B
    \pin{1}{5}{L}{C}    % pin C
    \pin{1}{7}{L}{C1}   % pin C1
    \pin{1}{9}{L}{D}    % pin D
    \pin{1}{11}{L}{D1}  % pin D1
    \pin{1}{13}{L}{D2}  % pin D2
    \wire{2}{1}{23}{1}  % wire from pin A
    \wire{2}{3}{16}{3}  % wire from pin B
    \wire{2}{5}{9}{5}   % wire from pin C
    \wire{2}{7}{9}{7}   % wire from pin C1
    \wire{9}{9}{9}{11}  % wire between OR and AND
    \pin{37}{3}{R}{Z}   % pin Z

\end{circuitdiagram}


%%  ************    LibreSilicon's StdCellLibrary   *******************
%%
%%  Organisation:   Chipforge
%%                  Germany / European Union
%%
%%  Profile:        Chipforge focus on fine System-on-Chip Cores in
%%                  Verilog HDL Code which are easy understandable and
%%                  adjustable. For further information see
%%                          www.chipforge.org
%%                  there are projects from small cores up to PCBs, too.
%%
%%  File:           StdCellLib/Documents/Book/section-OAOAI_complex.tex
%%
%%  Purpose:        Section Level File for Standard Cell Library Documentation
%%
%%  ************    LaTeX with circdia.sty package      ***************
%%
%%  ///////////////////////////////////////////////////////////////////
%%
%%  Copyright (c) 2018 - 2022 by
%%                  chipforge <stdcelllib@nospam.chipforge.org>
%%  All rights reserved.
%%
%%      This Standard Cell Library is licensed under the Libre Silicon
%%      public license; you can redistribute it and/or modify it under
%%      the terms of the Libre Silicon public license as published by
%%      the Libre Silicon alliance, either version 1 of the License, or
%%      (at your option) any later version.
%%
%%      This design is distributed in the hope that it will be useful,
%%      but WITHOUT ANY WARRANTY; without even the implied warranty of
%%      MERCHANTABILITY or FITNESS FOR A PARTICULAR PURPOSE.
%%      See the Libre Silicon Public License for more details.
%%
%%  ///////////////////////////////////////////////////////////////////
\section{OR-AND-OR-AND(-Invert) Complex Gates}

%%  ************    LibreSilicon's StdCellLibrary   *******************
%%
%%  Organisation:   Chipforge
%%                  Germany / European Union
%%
%%  Profile:        Chipforge focus on fine System-on-Chip Cores in
%%                  Verilog HDL Code which are easy understandable and
%%                  adjustable. For further information see
%%                          www.chipforge.org
%%                  there are projects from small cores up to PCBs, too.
%%
%%  File:           StdCellLib/Documents/Circuits/OAOAI2111.tex
%%
%%  Purpose:        Circuit File for OAOAI2111
%%
%%  ************    LaTeX with circdia.sty package      ***************
%%
%%  ///////////////////////////////////////////////////////////////////
%%
%%  Copyright (c) 2019 by chipforge <stdcelllib@nospam.chipforge.org>
%%  All rights reserved.
%%
%%      This Standard Cell Library is licensed under the Libre Silicon
%%      public license; you can redistribute it and/or modify it under
%%      the terms of the Libre Silicon public license as published by
%%      the Libre Silicon alliance, either version 1 of the License, or
%%      (at your option) any later version.
%%
%%      This design is distributed in the hope that it will be useful,
%%      but WITHOUT ANY WARRANTY; without even the implied warranty of
%%      MERCHANTABILITY or FITNESS FOR A PARTICULAR PURPOSE.
%%      See the Libre Silicon Public License for more details.
%%
%%  ///////////////////////////////////////////////////////////////////
\begin{center}
    Circuit
    \begin{figure}[h]
        \begin{center}
            \begin{circuitdiagram}{32}{12}
            \usgate
            \gate[\inputs{2}]{or}{5}{9}{R}{}{}     % OR
            \gate[\inputs{2}]{and}{12}{7}{R}{}{}   % AND
            \gate[\inputs{2}]{or}{19}{5}{R}{}{}    % OR
            \gate[\inputs{2}]{nand}{26}{3}{R}{}{}  % NAND
            \pin{1}{1}{L}{A}    % pin A
            \pin{1}{3}{L}{B}    % pin B
            \pin{1}{5}{L}{C}    % pin C
            \pin{1}{7}{L}{D}    % pin D
            \pin{1}{11}{L}{D1}  % pin D1
            \wire{2}{1}{23}{1}  % wire from pin A
            \wire{2}{3}{16}{3}  % wire from pin B
            \wire{2}{5}{9}{5}   % wire from pin C
            \pin{31}{3}{R}{Y}   % pin Y
            \end{circuitdiagram}
        \end{center}
    \end{figure}
\end{center}
 %%  ************    LibreSilicon's StdCellLibrary   *******************
%%
%%  Organisation:   Chipforge
%%                  Germany / European Union
%%
%%  Profile:        Chipforge focus on fine System-on-Chip Cores in
%%                  Verilog HDL Code which are easy understandable and
%%                  adjustable. For further information see
%%                          www.chipforge.org
%%                  there are projects from small cores up to PCBs, too.
%%
%%  File:           StdCellLib/Documents/Circuits/OAOA2111.tex
%%
%%  Purpose:        Circuit File for OAOA2111
%%
%%  ************    LaTeX with circdia.sty package      ***************
%%
%%  ///////////////////////////////////////////////////////////////////
%%
%%  Copyright (c) 2019 by chipforge <stdcelllib@nospam.chipforge.org>
%%  All rights reserved.
%%
%%      This Standard Cell Library is licensed under the Libre Silicon
%%      public license; you can redistribute it and/or modify it under
%%      the terms of the Libre Silicon public license as published by
%%      the Libre Silicon alliance, either version 1 of the License, or
%%      (at your option) any later version.
%%
%%      This design is distributed in the hope that it will be useful,
%%      but WITHOUT ANY WARRANTY; without even the implied warranty of
%%      MERCHANTABILITY or FITNESS FOR A PARTICULAR PURPOSE.
%%      See the Libre Silicon Public License for more details.
%%
%%  ///////////////////////////////////////////////////////////////////
\begin{center}
    Circuit
    \begin{figure}[h]
        \begin{center}
            \begin{circuitdiagram}{38}{12}
            \usgate
            \gate[\inputs{2}]{or}{5}{9}{R}{}{}     % OR
            \gate[\inputs{2}]{and}{12}{7}{R}{}{}   % AND
            \gate[\inputs{2}]{or}{19}{5}{R}{}{}    % OR
            \gate[\inputs{2}]{nand}{26}{3}{R}{}{}  % NAND
            \gate{not}{33}{3}{R}{}{}  % NOT
            \pin{1}{1}{L}{A}    % pin A
            \pin{1}{3}{L}{B}    % pin B
            \pin{1}{5}{L}{C}    % pin C
            \pin{1}{7}{L}{D}    % pin D
            \pin{1}{11}{L}{D1}  % pin D1
            \wire{2}{1}{23}{1}  % wire from pin A
            \wire{2}{3}{16}{3}  % wire from pin B
            \wire{2}{5}{9}{5}   % wire from pin C
            \pin{37}{3}{R}{Z}   % pin Z
            \end{circuitdiagram}
        \end{center}
    \end{figure}
\end{center}

%%  ************    LibreSilicon's StdCellLibrary   *******************
%%
%%  Organisation:   Chipforge
%%                  Germany / European Union
%%
%%  Profile:        Chipforge focus on fine System-on-Chip Cores in
%%                  Verilog HDL Code which are easy understandable and
%%                  adjustable. For further information see
%%                          www.chipforge.org
%%                  there are projects from small cores up to PCBs, too.
%%
%%  File:           StdCellLib/Documents/Datasheets/Circuitry/OAOAI2211.tex
%%
%%  Purpose:        Circuit File for OAOAI2211
%%
%%  ************    LaTeX with circdia.sty package      ***************
%%
%%  ///////////////////////////////////////////////////////////////////
%%
%%  Copyright (c) 2018 - 2022 by
%%                  chipforge <stdcelllib@nospam.chipforge.org>
%%  All rights reserved.
%%
%%      This Standard Cell Library is licensed under the Libre Silicon
%%      public license; you can redistribute it and/or modify it under
%%      the terms of the Libre Silicon public license as published by
%%      the Libre Silicon alliance, either version 1 of the License, or
%%      (at your option) any later version.
%%
%%      This design is distributed in the hope that it will be useful,
%%      but WITHOUT ANY WARRANTY; without even the implied warranty of
%%      MERCHANTABILITY or FITNESS FOR A PARTICULAR PURPOSE.
%%      See the Libre Silicon Public License for more details.
%%
%%  ///////////////////////////////////////////////////////////////////
\begin{circuitdiagram}[draft]{32}{14}

    \usgate
    % ----  1st column  ----
    \pin{1}{1}{L}{A}
    \pin{1}{5}{L}{A1}
    \gate[\inputs{2}]{or}{5}{3}{R}{}{}

    % ----  2nd column  ----
    \pin{8}{7}{L}{B}
    \pin{8}{9}{L}{B1}
    \wire{9}{3}{9}{5}
    \gate[\inputs{3}]{and}{12}{7}{R}{}{}

    % ----  3rd column  ----
    \pin{15}{11}{L}{C}
    \gate[\inputs{2}]{or}{19}{9}{R}{}{}

    % ----  4th column  ----
    \pin{22}{13}{L}{D}
    \gate[\inputs{2}]{nand}{26}{11}{R}{}{}


    % ----  result ----
    \pin{31}{11}{R}{Y}

\end{circuitdiagram}
 %%  ************    LibreSilicon's StdCellLibrary   *******************
%%
%%  Organisation:   Chipforge
%%                  Germany / European Union
%%
%%  Profile:        Chipforge focus on fine System-on-Chip Cores in
%%                  Verilog HDL Code which are easy understandable and
%%                  adjustable. For further information see
%%                          www.chipforge.org
%%                  there are projects from small cores up to PCBs, too.
%%
%%  File:           StdCellLib/Documents/Circuits/OAOA2211.tex
%%
%%  Purpose:        Circuit File for OAOA2211
%%
%%  ************    LaTeX with circdia.sty package      ***************
%%
%%  ///////////////////////////////////////////////////////////////////
%%
%%  Copyright (c) 2019 by chipforge <stdcelllib@nospam.chipforge.org>
%%  All rights reserved.
%%
%%      This Standard Cell Library is licensed under the Libre Silicon
%%      public license; you can redistribute it and/or modify it under
%%      the terms of the Libre Silicon public license as published by
%%      the Libre Silicon alliance, either version 1 of the License, or
%%      (at your option) any later version.
%%
%%      This design is distributed in the hope that it will be useful,
%%      but WITHOUT ANY WARRANTY; without even the implied warranty of
%%      MERCHANTABILITY or FITNESS FOR A PARTICULAR PURPOSE.
%%      See the Libre Silicon Public License for more details.
%%
%%  ///////////////////////////////////////////////////////////////////
\begin{center}
    Circuit
    \begin{figure}[h]
        \begin{center}
            \begin{circuitdiagram}{38}{14}
            \usgate
            \gate[\inputs{2}]{or}{5}{11}{R}{}{}   % OR
            \gate[\inputs{3}]{and}{12}{7}{R}{}{}  % AND
            \gate[\inputs{2}]{or}{19}{5}{R}{}{}   % OR
            \gate[\inputs{2}]{nand}{26}{3}{R}{}{} % NAND
            \gate{not}{33}{3}{R}{}{}  % NOT
            \pin{1}{1}{L}{A}    % pin A
            \pin{1}{3}{L}{B}    % pin B
            \pin{1}{5}{L}{C}    % pin C
            \pin{1}{7}{L}{C1}   % pin C1
            \pin{1}{9}{L}{D}    % pin D
            \pin{1}{13}{L}{D1}  % pin D1
            \wire{2}{1}{23}{1}  % wire from pin A
            \wire{2}{3}{16}{3}  % wire from pin B
            \wire{2}{5}{9}{5}   % wire from pin C
            \wire{2}{7}{9}{7}   % wire from pin C1
            \wire{9}{9}{9}{11}  % wire between AND and OR
            \pin{37}{3}{R}{Z}   % pin Z
            \end{circuitdiagram}
        \end{center}
    \end{figure}
\end{center}

%%  ************    LibreSilicon's StdCellLibrary   *******************
%%
%%  Organisation:   Chipforge
%%                  Germany / European Union
%%
%%  Profile:        Chipforge focus on fine System-on-Chip Cores in
%%                  Verilog HDL Code which are easy understandable and
%%                  adjustable. For further information see
%%                          www.chipforge.org
%%                  there are projects from small cores up to PCBs, too.
%%
%%  File:           StdCellLib/Documents/Datasheets/Circuitry/OAOAI3211.tex
%%
%%  Purpose:        Circuit File for OAOAI3211
%%
%%  ************    LaTeX with circdia.sty package      ***************
%%
%%  ///////////////////////////////////////////////////////////////////
%%
%%  Copyright (c) 2018 - 2022 by
%%                  chipforge <stdcelllib@nospam.chipforge.org>
%%  All rights reserved.
%%
%%      This Standard Cell Library is licensed under the Libre Silicon
%%      public license; you can redistribute it and/or modify it under
%%      the terms of the Libre Silicon public license as published by
%%      the Libre Silicon alliance, either version 1 of the License, or
%%      (at your option) any later version.
%%
%%      This design is distributed in the hope that it will be useful,
%%      but WITHOUT ANY WARRANTY; without even the implied warranty of
%%      MERCHANTABILITY or FITNESS FOR A PARTICULAR PURPOSE.
%%      See the Libre Silicon Public License for more details.
%%
%%  ///////////////////////////////////////////////////////////////////
\begin{circuitdiagram}[draft]{32}{14}

    \usgate
    % ----  1st column  ----
    \pin{1}{1}{L}{A}
    \pin{1}{3}{L}{A1}
    \pin{1}{5}{L}{A2}
    \gate[\inputs{3}]{or}{5}{3}{R}{}{}

    % ----  2nd column  ----
    \wire{9}{3}{9}{5}
    \pin{8}{7}{L}{B}
    \pin{8}{9}{L}{B1}
    \gate[\inputs{3}]{and}{12}{7}{R}{}{}

    % ----  3rd column  ----
    \pin{15}{11}{L}{C}
    \gate[\inputs{2}]{or}{19}{9}{R}{}{}

    % ----  4th column  ----
    \pin{22}{13}{L}{D}
    \gate[\inputs{2}]{nand}{26}{11}{R}{}{}


    % ----  result ----
    \pin{31}{11}{R}{Y}

\end{circuitdiagram}
 %%  ************    LibreSilicon's StdCellLibrary   *******************
%%
%%  Organisation:   Chipforge
%%                  Germany / European Union
%%
%%  Profile:        Chipforge focus on fine System-on-Chip Cores in
%%                  Verilog HDL Code which are easy understandable and
%%                  adjustable. For further information see
%%                          www.chipforge.org
%%                  there are projects from small cores up to PCBs, too.
%%
%%  File:           StdCellLib/Documents/Datasheets/Circuitry/OAOA3211.tex
%%
%%  Purpose:        Circuit File for OAOA3211
%%
%%  ************    LaTeX with circdia.sty package      ***************
%%
%%  ///////////////////////////////////////////////////////////////////
%%
%%  Copyright (c) 2018 - 2022 by
%%                  chipforge <stdcelllib@nospam.chipforge.org>
%%  All rights reserved.
%%
%%      This Standard Cell Library is licensed under the Libre Silicon
%%      public license; you can redistribute it and/or modify it under
%%      the terms of the Libre Silicon public license as published by
%%      the Libre Silicon alliance, either version 1 of the License, or
%%      (at your option) any later version.
%%
%%      This design is distributed in the hope that it will be useful,
%%      but WITHOUT ANY WARRANTY; without even the implied warranty of
%%      MERCHANTABILITY or FITNESS FOR A PARTICULAR PURPOSE.
%%      See the Libre Silicon Public License for more details.
%%
%%  ///////////////////////////////////////////////////////////////////
\begin{circuitdiagram}[draft]{38}{14}

    \usgate
    % ----  1st column  ----
    \pin{1}{1}{L}{A}
    \pin{1}{3}{L}{A1}
    \pin{1}{5}{L}{A2}
    \gate[\inputs{3}]{or}{5}{3}{R}{}{}

    % ----  2nd column  ----
    \wire{9}{3}{9}{5}
    \pin{8}{7}{L}{B}
    \pin{8}{9}{L}{B1}
    \gate[\inputs{3}]{and}{12}{7}{R}{}{}

    % ----  3rd column  ----
    \pin{15}{11}{L}{C}
    \gate[\inputs{2}]{or}{19}{9}{R}{}{}

    % ----  4th column  ----
    \pin{22}{13}{L}{D}
    \gate[\inputs{2}]{nand}{26}{11}{R}{}{}

    % ----  5th column  ----
    \gate{not}{33}{11}{R}{}{}

    % ----  result ----
    \pin{37}{11}{R}{Z}

\end{circuitdiagram}


%%  ************    LibreSilicon's StdCellLibrary   *******************
%%
%%  Organisation:   Chipforge
%%                  Germany / European Union
%%
%%  Profile:        Chipforge focus on fine System-on-Chip Cores in
%%                  Verilog HDL Code which are easy understandable and
%%                  adjustable. For further information see
%%                          www.chipforge.org
%%                  there are projects from small cores up to PCBs, too.
%%
%%  File:           StdCellLib/Documents/section-AOAAOI_complex.tex
%%
%%  Purpose:        Section Level File for Standard Cell Library Documentation
%%
%%  ************    LaTeX with circdia.sty package      ***************
%%
%%  ///////////////////////////////////////////////////////////////////
%%
%%  Copyright (c) 2018 - 2022 by
%%                  chipforge <stdcelllib@nospam.chipforge.org>
%%  All rights reserved.
%%
%%      This Standard Cell Library is licensed under the Libre Silicon
%%      public license; you can redistribute it and/or modify it under
%%      the terms of the Libre Silicon public license as published by
%%      the Libre Silicon alliance, either version 1 of the License, or
%%      (at your option) any later version.
%%
%%      This design is distributed in the hope that it will be useful,
%%      but WITHOUT ANY WARRANTY; without even the implied warranty of
%%      MERCHANTABILITY or FITNESS FOR A PARTICULAR PURPOSE.
%%      See the Libre Silicon Public License for more details.
%%
%%  ///////////////////////////////////////////////////////////////////
\section{AND-OR-AND-AND-OR(-Invert) Complex Gates}

%%  ************    LibreSilicon's StdCellLibrary   *******************
%%
%%  Organisation:   Chipforge
%%                  Germany / European Union
%%
%%  Profile:        Chipforge focus on fine System-on-Chip Cores in
%%                  Verilog HDL Code which are easy understandable and
%%                  adjustable. For further information see
%%                          www.chipforge.org
%%                  there are projects from small cores up to PCBs, too.
%%
%%  File:           StdCellLib/Documents/Datasheets/Circuitry/AOAAOI2111.tex
%%
%%  Purpose:        Circuit File for AOAAOI2111
%%
%%  ************    LaTeX with circdia.sty package      ***************
%%
%%  ///////////////////////////////////////////////////////////////////
%%
%%  Copyright (c) 2018 - 2022 by
%%                  chipforge <stdcelllib@nospam.chipforge.org>
%%  All rights reserved.
%%
%%      This Standard Cell Library is licensed under the Libre Silicon
%%      public license; you can redistribute it and/or modify it under
%%      the terms of the Libre Silicon public license as published by
%%      the Libre Silicon alliance, either version 1 of the License, or
%%      (at your option) any later version.
%%
%%      This design is distributed in the hope that it will be useful,
%%      but WITHOUT ANY WARRANTY; without even the implied warranty of
%%      MERCHANTABILITY or FITNESS FOR A PARTICULAR PURPOSE.
%%      See the Libre Silicon Public License for more details.
%%
%%  ///////////////////////////////////////////////////////////////////
\begin{circuitdiagram}[draft]{32}{16}

    \usgate
    % ----  1st column  ----
    \pin{1}{1}{L}{A}
    \pin{1}{5}{L}{A1}
    \gate[\inputs{2}]{and}{5}{3}{R}{}{}

    % ----  2nd column  ----
    \pin{8}{7}{L}{B}
    \gate[\inputs{2}]{or}{12}{5}{R}{}{}

    % ----  3rd column  ----
    \pin{15}{9}{L}{C}
    \gate[\inputs{2}]{and}{19}{7}{R}{}{}

    \pin{15}{15}{L}{D1}
    \pin{15}{11}{L}{D}
    \gate[\inputs{2}]{and}{19}{13}{R}{}{}

    % ----  4th column  ----
    \wire{23}{7}{23}{9}
    \gate[\inputs{2}]{nor}{26}{11}{R}{}{}


    % ----  result ----
    \pin{31}{11}{R}{Y}

\end{circuitdiagram}
 %%  ************    LibreSilicon's StdCellLibrary   *******************
%%
%%  Organisation:   Chipforge
%%                  Germany / European Union
%%
%%  Profile:        Chipforge focus on fine System-on-Chip Cores in
%%                  Verilog HDL Code which are easy understandable and
%%                  adjustable. For further information see
%%                          www.chipforge.org
%%                  there are projects from small cores up to PCBs, too.
%%
%%  File:           StdCellLib/Documents/Datasheets/Circuitry/AOAAO2111.tex
%%
%%  Purpose:        Circuit File for AOAAO2111
%%
%%  ************    LaTeX with circdia.sty package      ***************
%%
%%  ///////////////////////////////////////////////////////////////////
%%
%%  Copyright (c) 2018 - 2022 by
%%                  chipforge <stdcelllib@nospam.chipforge.org>
%%  All rights reserved.
%%
%%      This Standard Cell Library is licensed under the Libre Silicon
%%      public license; you can redistribute it and/or modify it under
%%      the terms of the Libre Silicon public license as published by
%%      the Libre Silicon alliance, either version 1 of the License, or
%%      (at your option) any later version.
%%
%%      This design is distributed in the hope that it will be useful,
%%      but WITHOUT ANY WARRANTY; without even the implied warranty of
%%      MERCHANTABILITY or FITNESS FOR A PARTICULAR PURPOSE.
%%      See the Libre Silicon Public License for more details.
%%
%%  ///////////////////////////////////////////////////////////////////
\begin{circuitdiagram}[draft]{38}{16}

    \usgate
    % ----  1st column  ----
    \pin{1}{1}{L}{A}
    \pin{1}{5}{L}{A1}
    \gate[\inputs{2}]{and}{5}{3}{R}{}{}

    % ----  2nd column  ----
    \pin{8}{7}{L}{B}
    \gate[\inputs{2}]{or}{12}{5}{R}{}{}

    % ----  3rd column  ----
    \pin{15}{9}{L}{C}
    \gate[\inputs{2}]{and}{19}{7}{R}{}{}

    \pin{15}{15}{L}{D1}
    \pin{15}{11}{L}{D}
    \gate[\inputs{2}]{and}{19}{13}{R}{}{}

    % ----  4th column  ----
    \wire{23}{7}{23}{9}
    \gate[\inputs{2}]{nor}{26}{11}{R}{}{}

    % ----  5th column  ----
    \gate{not}{33}{11}{R}{}{}

    % ----  result ----
    \pin{37}{11}{R}{Z}

\end{circuitdiagram}

%%  ************    LibreSilicon's StdCellLibrary   *******************
%%
%%  Organisation:   Chipforge
%%                  Germany / European Union
%%
%%  Profile:        Chipforge focus on fine System-on-Chip Cores in
%%                  Verilog HDL Code which are easy understandable and
%%                  adjustable. For further information see
%%                          www.chipforge.org
%%                  there are projects from small cores up to PCBs, too.
%%
%%  File:           StdCellLib/Documents/Datasheets/Circuitry/AOAAOI2113.tex
%%
%%  Purpose:        Circuit File for AOAAOI2113
%%
%%  ************    LaTeX with circdia.sty package      ***************
%%
%%  ///////////////////////////////////////////////////////////////////
%%
%%  Copyright (c) 2018 - 2022 by
%%                  chipforge <stdcelllib@nospam.chipforge.org>
%%  All rights reserved.
%%
%%      This Standard Cell Library is licensed under the Libre Silicon
%%      public license; you can redistribute it and/or modify it under
%%      the terms of the Libre Silicon public license as published by
%%      the Libre Silicon alliance, either version 1 of the License, or
%%      (at your option) any later version.
%%
%%      This design is distributed in the hope that it will be useful,
%%      but WITHOUT ANY WARRANTY; without even the implied warranty of
%%      MERCHANTABILITY or FITNESS FOR A PARTICULAR PURPOSE.
%%      See the Libre Silicon Public License for more details.
%%
%%  ///////////////////////////////////////////////////////////////////
\begin{circuitdiagram}[draft]{32}{16}

    \usgate
    % ----  1st column  ----
    \pin{1}{1}{L}{A}
    \pin{1}{5}{L}{A1}
    \gate[\inputs{2}]{and}{5}{3}{R}{}{}

    % ----  2nd column  ----
    \pin{8}{7}{L}{B}
    \gate[\inputs{2}]{or}{12}{5}{R}{}{}

    % ----  3rd column  ----
    \pin{15}{9}{L}{C}
    \gate[\inputs{2}]{and}{19}{7}{R}{}{}

    \pin{15}{11}{L}{D}
    \pin{15}{13}{L}{D1}
    \pin{15}{15}{L}{D2}
    \gate[\inputs{3}]{and}{19}{13}{R}{}{}

    % ----  4th column  ----
    \wire{23}{7}{23}{9}
    \gate[\inputs{2}]{nor}{26}{11}{R}{}{}


    % ----  result ----
    \pin{31}{11}{R}{Y}

\end{circuitdiagram}
 %%  ************    LibreSilicon's StdCellLibrary   *******************
%%
%%  Organisation:   Chipforge
%%                  Germany / European Union
%%
%%  Profile:        Chipforge focus on fine System-on-Chip Cores in
%%                  Verilog HDL Code which are easy understandable and
%%                  adjustable. For further information see
%%                          www.chipforge.org
%%                  there are projects from small cores up to PCBs, too.
%%
%%  File:           StdCellLib/Documents/Datasheets/Circuitry/AOAAO2113.tex
%%
%%  Purpose:        Circuit File for AOAAO2113
%%
%%  ************    LaTeX with circdia.sty package      ***************
%%
%%  ///////////////////////////////////////////////////////////////////
%%
%%  Copyright (c) 2018 - 2022 by
%%                  chipforge <stdcelllib@nospam.chipforge.org>
%%  All rights reserved.
%%
%%      This Standard Cell Library is licensed under the Libre Silicon
%%      public license; you can redistribute it and/or modify it under
%%      the terms of the Libre Silicon public license as published by
%%      the Libre Silicon alliance, either version 1 of the License, or
%%      (at your option) any later version.
%%
%%      This design is distributed in the hope that it will be useful,
%%      but WITHOUT ANY WARRANTY; without even the implied warranty of
%%      MERCHANTABILITY or FITNESS FOR A PARTICULAR PURPOSE.
%%      See the Libre Silicon Public License for more details.
%%
%%  ///////////////////////////////////////////////////////////////////
\begin{circuitdiagram}[draft]{38}{16}

    \usgate
    % ----  1st column  ----
    \pin{1}{1}{L}{A}
    \pin{1}{5}{L}{A1}
    \gate[\inputs{2}]{and}{5}{3}{R}{}{}

    % ----  2nd column  ----
    \pin{8}{7}{L}{B}
    \gate[\inputs{2}]{or}{12}{5}{R}{}{}

    % ----  3rd column  ----
    \pin{15}{9}{L}{C}
    \gate[\inputs{2}]{and}{19}{7}{R}{}{}

    \pin{15}{11}{L}{D}
    \pin{15}{13}{L}{D1}
    \pin{15}{15}{L}{D2}
    \gate[\inputs{3}]{and}{19}{13}{R}{}{}

    % ----  4th column  ----
    \wire{23}{7}{23}{9}
    \gate[\inputs{2}]{nor}{26}{11}{R}{}{}

    % ----  5th column  ----
    \gate{not}{33}{11}{R}{}{}

    % ----  result ----
    \pin{37}{11}{R}{Z}

\end{circuitdiagram}

%%  ************    LibreSilicon's StdCellLibrary   *******************
%%
%%  Organisation:   Chipforge
%%                  Germany / European Union
%%
%%  Profile:        Chipforge focus on fine System-on-Chip Cores in
%%                  Verilog HDL Code which are easy understandable and
%%                  adjustable. For further information see
%%                          www.chipforge.org
%%                  there are projects from small cores up to PCBs, too.
%%
%%  File:           StdCellLib/Documents/Datasheets/Circuitry/AOAAOI2113.tex
%%
%%  Purpose:        Circuit File for AOAAOI2113
%%
%%  ************    LaTeX with circdia.sty package      ***************
%%
%%  ///////////////////////////////////////////////////////////////////
%%
%%  Copyright (c) 2018 - 2022 by
%%                  chipforge <stdcelllib@nospam.chipforge.org>
%%  All rights reserved.
%%
%%      This Standard Cell Library is licensed under the Libre Silicon
%%      public license; you can redistribute it and/or modify it under
%%      the terms of the Libre Silicon public license as published by
%%      the Libre Silicon alliance, either version 1 of the License, or
%%      (at your option) any later version.
%%
%%      This design is distributed in the hope that it will be useful,
%%      but WITHOUT ANY WARRANTY; without even the implied warranty of
%%      MERCHANTABILITY or FITNESS FOR A PARTICULAR PURPOSE.
%%      See the Libre Silicon Public License for more details.
%%
%%  ///////////////////////////////////////////////////////////////////
\begin{circuitdiagram}[draft]{32}{18}

    \usgate
    % ----  1st column  ----
    \pin{1}{1}{L}{A}
    \pin{1}{5}{L}{A1}
    \gate[\inputs{2}]{and}{5}{3}{R}{}{}

    % ----  2nd column  ----
    \pin{8}{7}{L}{B}
    \gate[\inputs{2}]{or}{12}{5}{R}{}{}

    % ----  3rd column  ----
    \pin{15}{9}{L}{C}
    \gate[\inputs{2}]{and}{19}{7}{R}{}{}

    \pin{15}{11}{L}{D}
    \pin{15}{13}{L}{D1}
    \pin{15}{15}{L}{D2}
    \pin{15}{17}{L}{D3}
    \gate[\inputs{4}]{and}{19}{14}{R}{}{}

    % ----  4th column  ----
    \wire{23}{7}{23}{10}
    \gate[\inputs{2}]{nor}{26}{12}{R}{}{}


    % ----  result ----
    \pin{31}{12}{R}{Y}

\end{circuitdiagram}
 %%  ************    LibreSilicon's StdCellLibrary   *******************
%%
%%  Organisation:   Chipforge
%%                  Germany / European Union
%%
%%  Profile:        Chipforge focus on fine System-on-Chip Cores in
%%                  Verilog HDL Code which are easy understandable and
%%                  adjustable. For further information see
%%                          www.chipforge.org
%%                  there are projects from small cores up to PCBs, too.
%%
%%  File:           StdCellLib/Documents/Datasheets/Circuitry/AOAAO2113.tex
%%
%%  Purpose:        Circuit File for AOAAO2113
%%
%%  ************    LaTeX with circdia.sty package      ***************
%%
%%  ///////////////////////////////////////////////////////////////////
%%
%%  Copyright (c) 2018 - 2022 by
%%                  chipforge <stdcelllib@nospam.chipforge.org>
%%  All rights reserved.
%%
%%      This Standard Cell Library is licensed under the Libre Silicon
%%      public license; you can redistribute it and/or modify it under
%%      the terms of the Libre Silicon public license as published by
%%      the Libre Silicon alliance, either version 1 of the License, or
%%      (at your option) any later version.
%%
%%      This design is distributed in the hope that it will be useful,
%%      but WITHOUT ANY WARRANTY; without even the implied warranty of
%%      MERCHANTABILITY or FITNESS FOR A PARTICULAR PURPOSE.
%%      See the Libre Silicon Public License for more details.
%%
%%  ///////////////////////////////////////////////////////////////////
\begin{circuitdiagram}[draft]{38}{18}

    \usgate
    % ----  1st column  ----
    \pin{1}{1}{L}{A}
    \pin{1}{5}{L}{A1}
    \gate[\inputs{2}]{and}{5}{3}{R}{}{}

    % ----  2nd column  ----
    \pin{8}{7}{L}{B}
    \gate[\inputs{2}]{or}{12}{5}{R}{}{}

    % ----  3rd column  ----
    \pin{15}{9}{L}{C}
    \gate[\inputs{2}]{and}{19}{7}{R}{}{}

    \pin{15}{11}{L}{D}
    \pin{15}{13}{L}{D1}
    \pin{15}{15}{L}{D2}
    \pin{15}{17}{L}{D3}
    \gate[\inputs{4}]{and}{19}{14}{R}{}{}

    % ----  4th column  ----
    \wire{23}{7}{23}{10}
    \gate[\inputs{2}]{nor}{26}{12}{R}{}{}

    % ----  5th column  ----
    \gate{not}{33}{12}{R}{}{}

    % ----  result ----
    \pin{37}{12}{R}{Z}

\end{circuitdiagram}

%%  ************    LibreSilicon's StdCellLibrary   *******************
%%
%%  Organisation:   Chipforge
%%                  Germany / European Union
%%
%%  Profile:        Chipforge focus on fine System-on-Chip Cores in
%%                  Verilog HDL Code which are easy understandable and
%%                  adjustable. For further information see
%%                          www.chipforge.org
%%                  there are projects from small cores up to PCBs, too.
%%
%%  File:           StdCellLib/Documents/Datasheets/Circuitry/AOAAOI2122.tex
%%
%%  Purpose:        Circuit File for AOAAOI2122
%%
%%  ************    LaTeX with circdia.sty package      ***************
%%
%%  ///////////////////////////////////////////////////////////////////
%%
%%  Copyright (c) 2018 - 2022 by
%%                  chipforge <stdcelllib@nospam.chipforge.org>
%%  All rights reserved.
%%
%%      This Standard Cell Library is licensed under the Libre Silicon
%%      public license; you can redistribute it and/or modify it under
%%      the terms of the Libre Silicon public license as published by
%%      the Libre Silicon alliance, either version 1 of the License, or
%%      (at your option) any later version.
%%
%%      This design is distributed in the hope that it will be useful,
%%      but WITHOUT ANY WARRANTY; without even the implied warranty of
%%      MERCHANTABILITY or FITNESS FOR A PARTICULAR PURPOSE.
%%      See the Libre Silicon Public License for more details.
%%
%%  ///////////////////////////////////////////////////////////////////
\begin{circuitdiagram}[draft]{32}{18}

    \usgate
    % ----  1st column  ----
    \pin{1}{1}{L}{A}
    \pin{1}{5}{L}{A1}
    \gate[\inputs{2}]{and}{5}{3}{R}{}{}

    % ----  2nd column  ----
    \pin{8}{7}{L}{B}
    \gate[\inputs{2}]{or}{12}{5}{R}{}{}

    % ----  3rd column  ----
    \wire{16}{5}{16}{7}
    \pin{15}{9}{L}{C}
    \pin{15}{11}{L}{C1}
    \gate[\inputs{3}]{and}{19}{9}{R}{}{}

    \pin{15}{13}{L}{D}
    \pin{15}{17}{L}{D1}
    \gate[\inputs{2}]{and}{19}{15}{R}{}{}

    % ----  4th column  ----
    \wire{23}{9}{23}{11}
    \gate[\inputs{2}]{nor}{26}{13}{R}{}{}


    % ----  result ----
    \pin{31}{13}{R}{Y}

\end{circuitdiagram}
 %%  ************    LibreSilicon's StdCellLibrary   *******************
%%
%%  Organisation:   Chipforge
%%                  Germany / European Union
%%
%%  Profile:        Chipforge focus on fine System-on-Chip Cores in
%%                  Verilog HDL Code which are easy understandable and
%%                  adjustable. For further information see
%%                          www.chipforge.org
%%                  there are projects from small cores up to PCBs, too.
%%
%%  File:           StdCellLib/Documents/Datasheets/Circuitry/AOAAO2122.tex
%%
%%  Purpose:        Circuit File for AOAAO2122
%%
%%  ************    LaTeX with circdia.sty package      ***************
%%
%%  ///////////////////////////////////////////////////////////////////
%%
%%  Copyright (c) 2018 - 2022 by
%%                  chipforge <stdcelllib@nospam.chipforge.org>
%%  All rights reserved.
%%
%%      This Standard Cell Library is licensed under the Libre Silicon
%%      public license; you can redistribute it and/or modify it under
%%      the terms of the Libre Silicon public license as published by
%%      the Libre Silicon alliance, either version 1 of the License, or
%%      (at your option) any later version.
%%
%%      This design is distributed in the hope that it will be useful,
%%      but WITHOUT ANY WARRANTY; without even the implied warranty of
%%      MERCHANTABILITY or FITNESS FOR A PARTICULAR PURPOSE.
%%      See the Libre Silicon Public License for more details.
%%
%%  ///////////////////////////////////////////////////////////////////
\begin{circuitdiagram}[draft]{38}{18}

    \usgate
    % ----  1st column  ----
    \pin{1}{1}{L}{A}
    \pin{1}{5}{L}{A1}
    \gate[\inputs{2}]{and}{5}{3}{R}{}{}

    % ----  2nd column  ----
    \pin{8}{7}{L}{B}
    \gate[\inputs{2}]{or}{12}{5}{R}{}{}

    % ----  3rd column  ----
    \wire{16}{5}{16}{7}
    \pin{15}{9}{L}{C}
    \pin{15}{11}{L}{C1}
    \gate[\inputs{3}]{and}{19}{9}{R}{}{}

    \pin{15}{13}{L}{D}
    \pin{15}{17}{L}{D1}
    \gate[\inputs{2}]{and}{19}{15}{R}{}{}

    % ----  4th column  ----
    \wire{23}{9}{23}{11}
    \gate[\inputs{2}]{nor}{26}{13}{R}{}{}

    % ----  5th column  ----
    \gate{not}{33}{13}{R}{}{}

    % ----  result ----
    \pin{37}{13}{R}{Z}

\end{circuitdiagram}

%%  ************    LibreSilicon's StdCellLibrary   *******************
%%
%%  Organisation:   Chipforge
%%                  Germany / European Union
%%
%%  Profile:        Chipforge focus on fine System-on-Chip Cores in
%%                  Verilog HDL Code which are easy understandable and
%%                  adjustable. For further information see
%%                          www.chipforge.org
%%                  there are projects from small cores up to PCBs, too.
%%
%%  File:           StdCellLib/Documents/Datasheets/Circuitry/AOAAOI2123.tex
%%
%%  Purpose:        Circuit File for AOAAOI2123
%%
%%  ************    LaTeX with circdia.sty package      ***************
%%
%%  ///////////////////////////////////////////////////////////////////
%%
%%  Copyright (c) 2018 - 2022 by
%%                  chipforge <stdcelllib@nospam.chipforge.org>
%%  All rights reserved.
%%
%%      This Standard Cell Library is licensed under the Libre Silicon
%%      public license; you can redistribute it and/or modify it under
%%      the terms of the Libre Silicon public license as published by
%%      the Libre Silicon alliance, either version 1 of the License, or
%%      (at your option) any later version.
%%
%%      This design is distributed in the hope that it will be useful,
%%      but WITHOUT ANY WARRANTY; without even the implied warranty of
%%      MERCHANTABILITY or FITNESS FOR A PARTICULAR PURPOSE.
%%      See the Libre Silicon Public License for more details.
%%
%%  ///////////////////////////////////////////////////////////////////
\begin{circuitdiagram}[draft]{32}{18}

    \usgate
    % ----  1st column  ----
    \pin{1}{1}{L}{A}
    \pin{1}{5}{L}{A1}
    \gate[\inputs{2}]{and}{5}{3}{R}{}{}

    % ----  2nd column  ----
    \pin{8}{7}{L}{B}
    \gate[\inputs{2}]{or}{12}{5}{R}{}{}

    % ----  3rd column  ----
    \wire{16}{5}{16}{7}
    \pin{15}{9}{L}{C}
    \pin{15}{11}{L}{C1}
    \gate[\inputs{3}]{and}{19}{9}{R}{}{}

    \pin{15}{13}{L}{D}
    \pin{15}{15}{L}{D1}
    \pin{15}{17}{L}{D2}
    \gate[\inputs{3}]{and}{19}{15}{R}{}{}

    % ----  4th column  ----
    \wire{23}{9}{23}{11}
    \gate[\inputs{2}]{nor}{26}{13}{R}{}{}


    % ----  result ----
    \pin{31}{13}{R}{Y}

\end{circuitdiagram}
 %%  ************    LibreSilicon's StdCellLibrary   *******************
%%
%%  Organisation:   Chipforge
%%                  Germany / European Union
%%
%%  Profile:        Chipforge focus on fine System-on-Chip Cores in
%%                  Verilog HDL Code which are easy understandable and
%%                  adjustable. For further information see
%%                          www.chipforge.org
%%                  there are projects from small cores up to PCBs, too.
%%
%%  File:           StdCellLib/Documents/Datasheets/Circuitry/AOAAO2123.tex
%%
%%  Purpose:        Circuit File for AOAAO2123
%%
%%  ************    LaTeX with circdia.sty package      ***************
%%
%%  ///////////////////////////////////////////////////////////////////
%%
%%  Copyright (c) 2018 - 2022 by
%%                  chipforge <stdcelllib@nospam.chipforge.org>
%%  All rights reserved.
%%
%%      This Standard Cell Library is licensed under the Libre Silicon
%%      public license; you can redistribute it and/or modify it under
%%      the terms of the Libre Silicon public license as published by
%%      the Libre Silicon alliance, either version 1 of the License, or
%%      (at your option) any later version.
%%
%%      This design is distributed in the hope that it will be useful,
%%      but WITHOUT ANY WARRANTY; without even the implied warranty of
%%      MERCHANTABILITY or FITNESS FOR A PARTICULAR PURPOSE.
%%      See the Libre Silicon Public License for more details.
%%
%%  ///////////////////////////////////////////////////////////////////
\begin{circuitdiagram}[draft]{38}{18}

    \usgate
    % ----  1st column  ----
    \pin{1}{1}{L}{A}
    \pin{1}{5}{L}{A1}
    \gate[\inputs{2}]{and}{5}{3}{R}{}{}

    % ----  2nd column  ----
    \pin{8}{7}{L}{B}
    \gate[\inputs{2}]{or}{12}{5}{R}{}{}

    % ----  3rd column  ----
    \wire{16}{5}{16}{7}
    \pin{15}{9}{L}{C}
    \pin{15}{11}{L}{C1}
    \gate[\inputs{3}]{and}{19}{9}{R}{}{}

    \pin{15}{13}{L}{D}
    \pin{15}{15}{L}{D1}
    \pin{15}{17}{L}{D2}
    \gate[\inputs{3}]{and}{19}{15}{R}{}{}

    % ----  4th column  ----
    \wire{23}{9}{23}{11}
    \gate[\inputs{2}]{nor}{26}{13}{R}{}{}

    % ----  last column ----
    \gate{not}{33}{13}{R}{}{}

    % ----  result ----
    \pin{37}{13}{R}{Z}

\end{circuitdiagram}

%%  ************    LibreSilicon's StdCellLibrary   *******************
%%
%%  Organisation:   Chipforge
%%                  Germany / European Union
%%
%%  Profile:        Chipforge focus on fine System-on-Chip Cores in
%%                  Verilog HDL Code which are easy understandable and
%%                  adjustable. For further information see
%%                          www.chipforge.org
%%                  there are projects from small cores up to PCBs, too.
%%
%%  File:           StdCellLib/Documents/Datasheets/Circuitry/AOAAOI2124.tex
%%
%%  Purpose:        Circuit File for AOAAOI2124
%%
%%  ************    LaTeX with circdia.sty package      ***************
%%
%%  ///////////////////////////////////////////////////////////////////
%%
%%  Copyright (c) 2018 - 2022 by
%%                  chipforge <stdcelllib@nospam.chipforge.org>
%%  All rights reserved.
%%
%%      This Standard Cell Library is licensed under the Libre Silicon
%%      public license; you can redistribute it and/or modify it under
%%      the terms of the Libre Silicon public license as published by
%%      the Libre Silicon alliance, either version 1 of the License, or
%%      (at your option) any later version.
%%
%%      This design is distributed in the hope that it will be useful,
%%      but WITHOUT ANY WARRANTY; without even the implied warranty of
%%      MERCHANTABILITY or FITNESS FOR A PARTICULAR PURPOSE.
%%      See the Libre Silicon Public License for more details.
%%
%%  ///////////////////////////////////////////////////////////////////
\begin{circuitdiagram}[draft]{32}{20}

    \usgate
    % ----  1st column  ----
    \pin{1}{1}{L}{A}
    \pin{1}{5}{L}{A1}
    \gate[\inputs{2}]{and}{5}{3}{R}{}{}

    % ----  2nd column  ----
    \pin{8}{7}{L}{B}
    \gate[\inputs{2}]{or}{12}{5}{R}{}{}

    % ----  3rd column  ----
    \wire{16}{5}{16}{7}
    \pin{15}{9}{L}{C}
    \pin{15}{11}{L}{C1}
    \gate[\inputs{3}]{and}{19}{9}{R}{}{}

    \pin{15}{13}{L}{D}
    \pin{15}{15}{L}{D1}
    \pin{15}{17}{L}{D2}
    \pin{15}{19}{L}{D3}
    \gate[\inputs{4}]{and}{19}{16}{R}{}{}

    % ----  4th column  ----
    \wire{23}{9}{23}{12}
    \gate[\inputs{2}]{nor}{26}{14}{R}{}{}


    % ----  result ----
    \pin{31}{14}{R}{Y}

\end{circuitdiagram}
 %%  ************    LibreSilicon's StdCellLibrary   *******************
%%
%%  Organisation:   Chipforge
%%                  Germany / European Union
%%
%%  Profile:        Chipforge focus on fine System-on-Chip Cores in
%%                  Verilog HDL Code which are easy understandable and
%%                  adjustable. For further information see
%%                          www.chipforge.org
%%                  there are projects from small cores up to PCBs, too.
%%
%%  File:           StdCellLib/Documents/Datasheets/Circuitry/AOAAO2124.tex
%%
%%  Purpose:        Circuit File for AOAAO2124
%%
%%  ************    LaTeX with circdia.sty package      ***************
%%
%%  ///////////////////////////////////////////////////////////////////
%%
%%  Copyright (c) 2018 - 2022 by
%%                  chipforge <stdcelllib@nospam.chipforge.org>
%%  All rights reserved.
%%
%%      This Standard Cell Library is licensed under the Libre Silicon
%%      public license; you can redistribute it and/or modify it under
%%      the terms of the Libre Silicon public license as published by
%%      the Libre Silicon alliance, either version 1 of the License, or
%%      (at your option) any later version.
%%
%%      This design is distributed in the hope that it will be useful,
%%      but WITHOUT ANY WARRANTY; without even the implied warranty of
%%      MERCHANTABILITY or FITNESS FOR A PARTICULAR PURPOSE.
%%      See the Libre Silicon Public License for more details.
%%
%%  ///////////////////////////////////////////////////////////////////
\begin{circuitdiagram}[draft]{38}{20}

    \usgate
    % ----  1st column  ----
    \pin{1}{1}{L}{A}
    \pin{1}{5}{L}{A1}
    \gate[\inputs{2}]{and}{5}{3}{R}{}{}

    % ----  2nd column  ----
    \pin{8}{7}{L}{B}
    \gate[\inputs{2}]{or}{12}{5}{R}{}{}

    % ----  3rd column  ----
    \wire{16}{5}{16}{7}
    \pin{15}{9}{L}{C}
    \pin{15}{11}{L}{C1}
    \gate[\inputs{3}]{and}{19}{9}{R}{}{}

    \pin{15}{13}{L}{D}
    \pin{15}{15}{L}{D1}
    \pin{15}{17}{L}{D2}
    \pin{15}{19}{L}{D3}
    \gate[\inputs{4}]{and}{19}{16}{R}{}{}

    % ----  4th column  ----
    \wire{23}{9}{23}{12}
    \gate[\inputs{2}]{nor}{26}{14}{R}{}{}

    % ----  last column ----
    \gate{not}{33}{14}{R}{}{}

    % ----  result ----
    \pin{37}{14}{R}{Z}

\end{circuitdiagram}

%%  ************    LibreSilicon's StdCellLibrary   *******************
%%
%%  Organisation:   Chipforge
%%                  Germany / European Union
%%
%%  Profile:        Chipforge focus on fine System-on-Chip Cores in
%%                  Verilog HDL Code which are easy understandable and
%%                  adjustable. For further information see
%%                          www.chipforge.org
%%                  there are projects from small cores up to PCBs, too.
%%
%%  File:           StdCellLib/Documents/Datasheets/Circuitry/AOAAOI2212.tex
%%
%%  Purpose:        Circuit File for AOAAOI2212
%%
%%  ************    LaTeX with circdia.sty package      ***************
%%
%%  ///////////////////////////////////////////////////////////////////
%%
%%  Copyright (c) 2018 - 2022 by
%%                  chipforge <stdcelllib@nospam.chipforge.org>
%%  All rights reserved.
%%
%%      This Standard Cell Library is licensed under the Libre Silicon
%%      public license; you can redistribute it and/or modify it under
%%      the terms of the Libre Silicon public license as published by
%%      the Libre Silicon alliance, either version 1 of the License, or
%%      (at your option) any later version.
%%
%%      This design is distributed in the hope that it will be useful,
%%      but WITHOUT ANY WARRANTY; without even the implied warranty of
%%      MERCHANTABILITY or FITNESS FOR A PARTICULAR PURPOSE.
%%      See the Libre Silicon Public License for more details.
%%
%%  ///////////////////////////////////////////////////////////////////
\begin{circuitdiagram}[draft]{32}{18}

    \usgate
    % ----  1st column  ----
    \pin{1}{1}{L}{A}
    \pin{1}{5}{L}{A1}
    \gate[\inputs{2}]{and}{5}{3}{R}{}{}

    % ----  2nd column  ----
    \wire{9}{3}{9}{5}
    \pin{8}{7}{L}{B}
    \pin{8}{9}{L}{B1}
    \gate[\inputs{3}]{or}{12}{7}{R}{}{}

    % ----  3rd column  ----
    \pin{15}{11}{L}{C}
    \gate[\inputs{2}]{and}{19}{9}{R}{}{}

    \pin{15}{17}{L}{D1}
    \pin{15}{13}{L}{D}
    \gate[\inputs{2}]{and}{19}{15}{R}{}{}

    % ----  4th column  ----
    \wire{23}{9}{23}{11}
    \gate[\inputs{2}]{nor}{26}{13}{R}{}{}


    % ----  result ----
    \pin{31}{13}{R}{Y}

\end{circuitdiagram}
 %%  ************    LibreSilicon's StdCellLibrary   *******************
%%
%%  Organisation:   Chipforge
%%                  Germany / European Union
%%
%%  Profile:        Chipforge focus on fine System-on-Chip Cores in
%%                  Verilog HDL Code which are easy understandable and
%%                  adjustable. For further information see
%%                          www.chipforge.org
%%                  there are projects from small cores up to PCBs, too.
%%
%%  File:           StdCellLib/Documents/Datasheets/Circuitry/AOAAO2212.tex
%%
%%  Purpose:        Circuit File for AOAAO2212
%%
%%  ************    LaTeX with circdia.sty package      ***************
%%
%%  ///////////////////////////////////////////////////////////////////
%%
%%  Copyright (c) 2018 - 2022 by
%%                  chipforge <stdcelllib@nospam.chipforge.org>
%%  All rights reserved.
%%
%%      This Standard Cell Library is licensed under the Libre Silicon
%%      public license; you can redistribute it and/or modify it under
%%      the terms of the Libre Silicon public license as published by
%%      the Libre Silicon alliance, either version 1 of the License, or
%%      (at your option) any later version.
%%
%%      This design is distributed in the hope that it will be useful,
%%      but WITHOUT ANY WARRANTY; without even the implied warranty of
%%      MERCHANTABILITY or FITNESS FOR A PARTICULAR PURPOSE.
%%      See the Libre Silicon Public License for more details.
%%
%%  ///////////////////////////////////////////////////////////////////
\begin{circuitdiagram}[draft]{38}{18}

    \usgate
    % ----  1st column  ----
    \pin{1}{1}{L}{A}
    \pin{1}{5}{L}{A1}
    \gate[\inputs{2}]{and}{5}{3}{R}{}{}

    % ----  2nd column  ----
    \wire{9}{3}{9}{5}
    \pin{8}{7}{L}{B}
    \pin{8}{9}{L}{B1}
    \gate[\inputs{3}]{or}{12}{7}{R}{}{}

    % ----  3rd column  ----
    \pin{15}{11}{L}{C}
    \gate[\inputs{2}]{and}{19}{9}{R}{}{}

    \pin{15}{17}{L}{D1}
    \pin{15}{13}{L}{D}
    \gate[\inputs{2}]{and}{19}{15}{R}{}{}

    % ----  4th column  ----
    \wire{23}{9}{23}{11}
    \gate[\inputs{2}]{nor}{26}{13}{R}{}{}

    % ----  last column ----
    \gate{not}{33}{13}{R}{}{}

    % ----  result ----
    \pin{37}{13}{R}{Z}

\end{circuitdiagram}

%%  ************    LibreSilicon's StdCellLibrary   *******************
%%
%%  Organisation:   Chipforge
%%                  Germany / European Union
%%
%%  Profile:        Chipforge focus on fine System-on-Chip Cores in
%%                  Verilog HDL Code which are easy understandable and
%%                  adjustable. For further information see
%%                          www.chipforge.org
%%                  there are projects from small cores up to PCBs, too.
%%
%%  File:           StdCellLib/Documents/Datasheets/Circuitry/AOAAOI3111.tex
%%
%%  Purpose:        Circuit File for AOAAOI3111
%%
%%  ************    LaTeX with circdia.sty package      ***************
%%
%%  ///////////////////////////////////////////////////////////////////
%%
%%  Copyright (c) 2018 - 2022 by
%%                  chipforge <stdcelllib@nospam.chipforge.org>
%%  All rights reserved.
%%
%%      This Standard Cell Library is licensed under the Libre Silicon
%%      public license; you can redistribute it and/or modify it under
%%      the terms of the Libre Silicon public license as published by
%%      the Libre Silicon alliance, either version 1 of the License, or
%%      (at your option) any later version.
%%
%%      This design is distributed in the hope that it will be useful,
%%      but WITHOUT ANY WARRANTY; without even the implied warranty of
%%      MERCHANTABILITY or FITNESS FOR A PARTICULAR PURPOSE.
%%      See the Libre Silicon Public License for more details.
%%
%%  ///////////////////////////////////////////////////////////////////
\begin{circuitdiagram}[draft]{32}{16}

    \usgate
    % ----  1st column  ----
    \pin{1}{1}{L}{A}
    \pin{1}{3}{L}{A1}
    \pin{1}{5}{L}{A2}
    \gate[\inputs{3}]{and}{5}{3}{R}{}{}

    % ----  2nd column  ----
    \pin{8}{7}{L}{B}
    \gate[\inputs{2}]{or}{12}{5}{R}{}{}

    % ----  3rd column  ----
    \pin{15}{9}{L}{C}
    \gate[\inputs{2}]{and}{19}{7}{R}{}{}

    \pin{15}{15}{L}{D1}
    \pin{15}{11}{L}{D}
    \gate[\inputs{2}]{and}{19}{13}{R}{}{}

    % ----  4th column  ----
    \wire{23}{7}{23}{9}
    \gate[\inputs{2}]{nor}{26}{11}{R}{}{}


    % ----  result ----
    \pin{31}{11}{R}{Y}

\end{circuitdiagram}
 %%  ************    LibreSilicon's StdCellLibrary   *******************
%%
%%  Organisation:   Chipforge
%%                  Germany / European Union
%%
%%  Profile:        Chipforge focus on fine System-on-Chip Cores in
%%                  Verilog HDL Code which are easy understandable and
%%                  adjustable. For further information see
%%                          www.chipforge.org
%%                  there are projects from small cores up to PCBs, too.
%%
%%  File:           StdCellLib/Documents/Datasheets/Circuitry/AOAAO3111.tex
%%
%%  Purpose:        Circuit File for AOAAO3111
%%
%%  ************    LaTeX with circdia.sty package      ***************
%%
%%  ///////////////////////////////////////////////////////////////////
%%
%%  Copyright (c) 2018 - 2022 by
%%                  chipforge <stdcelllib@nospam.chipforge.org>
%%  All rights reserved.
%%
%%      This Standard Cell Library is licensed under the Libre Silicon
%%      public license; you can redistribute it and/or modify it under
%%      the terms of the Libre Silicon public license as published by
%%      the Libre Silicon alliance, either version 1 of the License, or
%%      (at your option) any later version.
%%
%%      This design is distributed in the hope that it will be useful,
%%      but WITHOUT ANY WARRANTY; without even the implied warranty of
%%      MERCHANTABILITY or FITNESS FOR A PARTICULAR PURPOSE.
%%      See the Libre Silicon Public License for more details.
%%
%%  ///////////////////////////////////////////////////////////////////
\begin{circuitdiagram}[draft]{38}{16}

    \usgate
    % ----  1st column  ----
    \pin{1}{1}{L}{A}
    \pin{1}{3}{L}{A1}
    \pin{1}{5}{L}{A2}
    \gate[\inputs{3}]{and}{5}{3}{R}{}{}

    % ----  2nd column  ----
    \pin{8}{7}{L}{B}
    \gate[\inputs{2}]{or}{12}{5}{R}{}{}

    % ----  3rd column  ----
    \pin{15}{9}{L}{C}
    \gate[\inputs{2}]{and}{19}{7}{R}{}{}

    \pin{15}{15}{L}{D1}
    \pin{15}{11}{L}{D}
    \gate[\inputs{2}]{and}{19}{13}{R}{}{}

    % ----  4th column  ----
    \wire{23}{7}{23}{9}
    \gate[\inputs{2}]{nor}{26}{11}{R}{}{}

    % ----  5th column  ----
    \gate{not}{33}{11}{R}{}{}

    % ----  result ----
    \pin{37}{11}{R}{Z}

\end{circuitdiagram}

%%  ************    LibreSilicon's StdCellLibrary   *******************
%%
%%  Organisation:   Chipforge
%%                  Germany / European Union
%%
%%  Profile:        Chipforge focus on fine System-on-Chip Cores in
%%                  Verilog HDL Code which are easy understandable and
%%                  adjustable. For further information see
%%                          www.chipforge.org
%%                  there are projects from small cores up to PCBs, too.
%%
%%  File:           StdCellLib/Documents/Datasheets/Circuitry/AOAAOI3113.tex
%%
%%  Purpose:        Circuit File for AOAAOI3113
%%
%%  ************    LaTeX with circdia.sty package      ***************
%%
%%  ///////////////////////////////////////////////////////////////////
%%
%%  Copyright (c) 2018 - 2022 by
%%                  chipforge <stdcelllib@nospam.chipforge.org>
%%  All rights reserved.
%%
%%      This Standard Cell Library is licensed under the Libre Silicon
%%      public license; you can redistribute it and/or modify it under
%%      the terms of the Libre Silicon public license as published by
%%      the Libre Silicon alliance, either version 1 of the License, or
%%      (at your option) any later version.
%%
%%      This design is distributed in the hope that it will be useful,
%%      but WITHOUT ANY WARRANTY; without even the implied warranty of
%%      MERCHANTABILITY or FITNESS FOR A PARTICULAR PURPOSE.
%%      See the Libre Silicon Public License for more details.
%%
%%  ///////////////////////////////////////////////////////////////////
\begin{circuitdiagram}[draft]{32}{16}

    \usgate
    % ----  1st column  ----
    \pin{1}{1}{L}{A}
    \pin{1}{3}{L}{A1}
    \pin{1}{5}{L}{A2}
    \gate[\inputs{3}]{and}{5}{3}{R}{}{}

    % ----  2nd column  ----
    \pin{8}{7}{L}{B}
    \gate[\inputs{2}]{or}{12}{5}{R}{}{}

    % ----  3rd column  ----
    \pin{15}{9}{L}{C}
    \gate[\inputs{2}]{and}{19}{7}{R}{}{}

    \pin{15}{11}{L}{D}
    \pin{15}{13}{L}{D1}
    \pin{15}{15}{L}{D2}
    \gate[\inputs{3}]{and}{19}{13}{R}{}{}

    % ----  4th column  ----
    \wire{23}{7}{23}{9}
    \gate[\inputs{2}]{nor}{26}{11}{R}{}{}


    % ----  result ----
    \pin{31}{11}{R}{Y}

\end{circuitdiagram}
 %%  ************    LibreSilicon's StdCellLibrary   *******************
%%
%%  Organisation:   Chipforge
%%                  Germany / European Union
%%
%%  Profile:        Chipforge focus on fine System-on-Chip Cores in
%%                  Verilog HDL Code which are easy understandable and
%%                  adjustable. For further information see
%%                          www.chipforge.org
%%                  there are projects from small cores up to PCBs, too.
%%
%%  File:           StdCellLib/Documents/Datasheets/Circuitry/AOAAO3113.tex
%%
%%  Purpose:        Circuit File for AOAAO3113
%%
%%  ************    LaTeX with circdia.sty package      ***************
%%
%%  ///////////////////////////////////////////////////////////////////
%%
%%  Copyright (c) 2018 - 2022 by
%%                  chipforge <stdcelllib@nospam.chipforge.org>
%%  All rights reserved.
%%
%%      This Standard Cell Library is licensed under the Libre Silicon
%%      public license; you can redistribute it and/or modify it under
%%      the terms of the Libre Silicon public license as published by
%%      the Libre Silicon alliance, either version 1 of the License, or
%%      (at your option) any later version.
%%
%%      This design is distributed in the hope that it will be useful,
%%      but WITHOUT ANY WARRANTY; without even the implied warranty of
%%      MERCHANTABILITY or FITNESS FOR A PARTICULAR PURPOSE.
%%      See the Libre Silicon Public License for more details.
%%
%%  ///////////////////////////////////////////////////////////////////
\begin{circuitdiagram}[draft]{38}{16}

    \usgate
    % ----  1st column  ----
    \pin{1}{1}{L}{A}
    \pin{1}{3}{L}{A1}
    \pin{1}{5}{L}{A2}
    \gate[\inputs{3}]{and}{5}{3}{R}{}{}

    % ----  2nd column  ----
    \pin{8}{7}{L}{B}
    \gate[\inputs{2}]{or}{12}{5}{R}{}{}

    % ----  3rd column  ----
    \pin{15}{9}{L}{C}
    \gate[\inputs{2}]{and}{19}{7}{R}{}{}

    \pin{15}{11}{L}{D}
    \pin{15}{13}{L}{D1}
    \pin{15}{15}{L}{D2}
    \gate[\inputs{3}]{and}{19}{13}{R}{}{}

    % ----  4th column  ----
    \wire{23}{7}{23}{9}
    \gate[\inputs{2}]{nor}{26}{11}{R}{}{}

    % ----  last column ----
    \gate{not}{33}{11}{R}{}{}

    % ----  result ----
    \pin{37}{11}{R}{Z}

\end{circuitdiagram}

%%  ************    LibreSilicon's StdCellLibrary   *******************
%%
%%  Organisation:   Chipforge
%%                  Germany / European Union
%%
%%  Profile:        Chipforge focus on fine System-on-Chip Cores in
%%                  Verilog HDL Code which are easy understandable and
%%                  adjustable. For further information see
%%                          www.chipforge.org
%%                  there are projects from small cores up to PCBs, too.
%%
%%  File:           StdCellLib/Documents/Datasheets/Circuitry/AOAAOI3111.tex
%%
%%  Purpose:        Circuit File for AOAAOI3111
%%
%%  ************    LaTeX with circdia.sty package      ***************
%%
%%  ///////////////////////////////////////////////////////////////////
%%
%%  Copyright (c) 2018 - 2022 by
%%                  chipforge <stdcelllib@nospam.chipforge.org>
%%  All rights reserved.
%%
%%      This Standard Cell Library is licensed under the Libre Silicon
%%      public license; you can redistribute it and/or modify it under
%%      the terms of the Libre Silicon public license as published by
%%      the Libre Silicon alliance, either version 1 of the License, or
%%      (at your option) any later version.
%%
%%      This design is distributed in the hope that it will be useful,
%%      but WITHOUT ANY WARRANTY; without even the implied warranty of
%%      MERCHANTABILITY or FITNESS FOR A PARTICULAR PURPOSE.
%%      See the Libre Silicon Public License for more details.
%%
%%  ///////////////////////////////////////////////////////////////////
\begin{circuitdiagram}[draft]{32}{18}

    \usgate
    % ----  1st column  ----
    \pin{1}{1}{L}{A}
    \pin{1}{3}{L}{A1}
    \pin{1}{5}{L}{A2}
    \gate[\inputs{3}]{and}{5}{3}{R}{}{}

    % ----  2nd column  ----
    \wire{9}{3}{9}{5}
    \pin{8}{7}{L}{B}
    \pin{8}{9}{L}{B1}
    \gate[\inputs{3}]{or}{12}{7}{R}{}{}

    % ----  3rd column  ----
    \pin{15}{11}{L}{C}
    \gate[\inputs{2}]{and}{19}{9}{R}{}{}

    \pin{15}{17}{L}{D1}
    \pin{15}{13}{L}{D}
    \gate[\inputs{2}]{and}{19}{15}{R}{}{}

    % ----  4th column  ----
    \wire{23}{9}{23}{11}
    \gate[\inputs{2}]{nor}{26}{13}{R}{}{}


    % ----  result ----
    \pin{31}{13}{R}{Y}

\end{circuitdiagram}
 %%  ************    LibreSilicon's StdCellLibrary   *******************
%%
%%  Organisation:   Chipforge
%%                  Germany / European Union
%%
%%  Profile:        Chipforge focus on fine System-on-Chip Cores in
%%                  Verilog HDL Code which are easy understandable and
%%                  adjustable. For further information see
%%                          www.chipforge.org
%%                  there are projects from small cores up to PCBs, too.
%%
%%  File:           StdCellLib/Documents/Datasheets/Circuitry/AOAAO3111.tex
%%
%%  Purpose:        Circuit File for AOAAO3111
%%
%%  ************    LaTeX with circdia.sty package      ***************
%%
%%  ///////////////////////////////////////////////////////////////////
%%
%%  Copyright (c) 2018 - 2022 by
%%                  chipforge <stdcelllib@nospam.chipforge.org>
%%  All rights reserved.
%%
%%      This Standard Cell Library is licensed under the Libre Silicon
%%      public license; you can redistribute it and/or modify it under
%%      the terms of the Libre Silicon public license as published by
%%      the Libre Silicon alliance, either version 1 of the License, or
%%      (at your option) any later version.
%%
%%      This design is distributed in the hope that it will be useful,
%%      but WITHOUT ANY WARRANTY; without even the implied warranty of
%%      MERCHANTABILITY or FITNESS FOR A PARTICULAR PURPOSE.
%%      See the Libre Silicon Public License for more details.
%%
%%  ///////////////////////////////////////////////////////////////////
\begin{circuitdiagram}[draft]{38}{18}

    \usgate
    % ----  1st column  ----
    \pin{1}{1}{L}{A}
    \pin{1}{3}{L}{A1}
    \pin{1}{5}{L}{A2}
    \gate[\inputs{3}]{and}{5}{3}{R}{}{}

    % ----  2nd column  ----
    \wire{9}{3}{9}{5}
    \pin{8}{7}{L}{B}
    \pin{8}{9}{L}{B1}
    \gate[\inputs{3}]{or}{12}{7}{R}{}{}

    % ----  3rd column  ----
    \pin{15}{11}{L}{C}
    \gate[\inputs{2}]{and}{19}{9}{R}{}{}

    \pin{15}{17}{L}{D1}
    \pin{15}{13}{L}{D}
    \gate[\inputs{2}]{and}{19}{15}{R}{}{}

    % ----  4th column  ----
    \wire{23}{9}{23}{11}
    \gate[\inputs{2}]{nor}{26}{13}{R}{}{}

    % ----  5th column  ----
    \gate{not}{33}{13}{R}{}{}

    % ----  result ----
    \pin{37}{13}{R}{Z}

\end{circuitdiagram}


%%  ************    LibreSilicon's StdCellLibrary   *******************
%%
%%  Organisation:   Chipforge
%%                  Germany / European Union
%%
%%  Profile:        Chipforge focus on fine System-on-Chip Cores in
%%                  Verilog HDL Code which are easy understandable and
%%                  adjustable. For further information see
%%                          www.chipforge.org
%%                  there are projects from small cores up to PCBs, too.
%%
%%  File:           StdCellLib/Documents/Datasheets/Circuitry/AOAAOI21121.tex
%%
%%  Purpose:        Circuit File for AOAAOI21121
%%
%%  ************    LaTeX with circdia.sty package      ***************
%%
%%  ///////////////////////////////////////////////////////////////////
%%
%%  Copyright (c) 2018 - 2022 by
%%                  chipforge <stdcelllib@nospam.chipforge.org>
%%  All rights reserved.
%%
%%      This Standard Cell Library is licensed under the Libre Silicon
%%      public license; you can redistribute it and/or modify it under
%%      the terms of the Libre Silicon public license as published by
%%      the Libre Silicon alliance, either version 1 of the License, or
%%      (at your option) any later version.
%%
%%      This design is distributed in the hope that it will be useful,
%%      but WITHOUT ANY WARRANTY; without even the implied warranty of
%%      MERCHANTABILITY or FITNESS FOR A PARTICULAR PURPOSE.
%%      See the Libre Silicon Public License for more details.
%%
%%  ///////////////////////////////////////////////////////////////////
\begin{circuitdiagram}[draft]{32}{18}

    \usgate
    % ----  1st column  ----
    \pin{1}{1}{L}{A}
    \pin{1}{5}{L}{A1}
    \gate[\inputs{2}]{and}{5}{3}{R}{}{}

    % ----  2nd column  ----
    \pin{8}{7}{L}{B}
    \gate[\inputs{2}]{or}{12}{5}{R}{}{}

    % ----  3rd column  ----
    \pin{15}{9}{L}{C}
    \gate[\inputs{2}]{and}{19}{7}{R}{}{}

    \pin{15}{15}{L}{D1}
    \pin{15}{11}{L}{D}
    \gate[\inputs{2}]{and}{19}{13}{R}{}{}

    % ----  4th column  ----
    \wire{23}{7}{23}{11}
    \pin{22}{17}{L}{E}
    \wire{23}{15}{23}{17}
    \gate[\inputs{3}]{nor}{26}{13}{R}{}{}


    % ----  result ----
    \pin{31}{13}{R}{Y}

\end{circuitdiagram}
 %%  ************    LibreSilicon's StdCellLibrary   *******************
%%
%%  Organisation:   Chipforge
%%                  Germany / European Union
%%
%%  Profile:        Chipforge focus on fine System-on-Chip Cores in
%%                  Verilog HDL Code which are easy understandable and
%%                  adjustable. For further information see
%%                          www.chipforge.org
%%                  there are projects from small cores up to PCBs, too.
%%
%%  File:           StdCellLib/Documents/Datasheets/Circuitry/AOAAO21121.tex
%%
%%  Purpose:        Circuit File for AOAAO21121
%%
%%  ************    LaTeX with circdia.sty package      ***************
%%
%%  ///////////////////////////////////////////////////////////////////
%%
%%  Copyright (c) 2018 - 2022 by
%%                  chipforge <stdcelllib@nospam.chipforge.org>
%%  All rights reserved.
%%
%%      This Standard Cell Library is licensed under the Libre Silicon
%%      public license; you can redistribute it and/or modify it under
%%      the terms of the Libre Silicon public license as published by
%%      the Libre Silicon alliance, either version 1 of the License, or
%%      (at your option) any later version.
%%
%%      This design is distributed in the hope that it will be useful,
%%      but WITHOUT ANY WARRANTY; without even the implied warranty of
%%      MERCHANTABILITY or FITNESS FOR A PARTICULAR PURPOSE.
%%      See the Libre Silicon Public License for more details.
%%
%%  ///////////////////////////////////////////////////////////////////
\begin{circuitdiagram}[draft]{38}{18}

    \usgate
    % ----  1st column  ----
    \pin{1}{1}{L}{A}
    \pin{1}{5}{L}{A1}
    \gate[\inputs{2}]{and}{5}{3}{R}{}{}

    % ----  2nd column  ----
    \pin{8}{7}{L}{B}
    \gate[\inputs{2}]{or}{12}{5}{R}{}{}

    % ----  3rd column  ----
    \pin{15}{9}{L}{C}
    \gate[\inputs{2}]{and}{19}{7}{R}{}{}

    \pin{15}{15}{L}{D1}
    \pin{15}{11}{L}{D}
    \gate[\inputs{2}]{and}{19}{13}{R}{}{}

    % ----  4th column  ----
    \wire{23}{7}{23}{11}
    \pin{22}{17}{L}{E}
    \wire{23}{15}{23}{17}
    \gate[\inputs{3}]{nor}{26}{13}{R}{}{}

    % ----  5th column  ----
    \gate{not}{33}{13}{R}{}{}

    % ----  result ----
    \pin{37}{13}{R}{Z}

\end{circuitdiagram}

%%  ************    LibreSilicon's StdCellLibrary   *******************
%%
%%  Organisation:   Chipforge
%%                  Germany / European Union
%%
%%  Profile:        Chipforge focus on fine System-on-Chip Cores in
%%                  Verilog HDL Code which are easy understandable and
%%                  adjustable. For further information see
%%                          www.chipforge.org
%%                  there are projects from small cores up to PCBs, too.
%%
%%  File:           StdCellLib/Documents/Datasheets/Circuitry/AOAAOI21131.tex
%%
%%  Purpose:        Circuit File for AOAAOI21131
%%
%%  ************    LaTeX with circdia.sty package      ***************
%%
%%  ///////////////////////////////////////////////////////////////////
%%
%%  Copyright (c) 2018 - 2022 by
%%                  chipforge <stdcelllib@nospam.chipforge.org>
%%  All rights reserved.
%%
%%      This Standard Cell Library is licensed under the Libre Silicon
%%      public license; you can redistribute it and/or modify it under
%%      the terms of the Libre Silicon public license as published by
%%      the Libre Silicon alliance, either version 1 of the License, or
%%      (at your option) any later version.
%%
%%      This design is distributed in the hope that it will be useful,
%%      but WITHOUT ANY WARRANTY; without even the implied warranty of
%%      MERCHANTABILITY or FITNESS FOR A PARTICULAR PURPOSE.
%%      See the Libre Silicon Public License for more details.
%%
%%  ///////////////////////////////////////////////////////////////////
\begin{circuitdiagram}[draft]{32}{18}

    \usgate
    % ----  1st column  ----
    \pin{1}{1}{L}{A}
    \pin{1}{5}{L}{A1}
    \gate[\inputs{2}]{and}{5}{3}{R}{}{}

    % ----  2nd column  ----
    \pin{8}{7}{L}{B}
    \gate[\inputs{2}]{or}{12}{5}{R}{}{}

    % ----  3rd column  ----
    \pin{15}{9}{L}{C}
    \gate[\inputs{2}]{and}{19}{7}{R}{}{}

    \pin{15}{15}{L}{D2}
    \pin{15}{13}{L}{D1}
    \pin{15}{11}{L}{D}
    \gate[\inputs{3}]{and}{19}{13}{R}{}{}

    % ----  4th column  ----
    \wire{23}{7}{23}{11}
    \pin{22}{17}{L}{E}
    \wire{23}{15}{23}{17}
    \gate[\inputs{3}]{nor}{26}{13}{R}{}{}


    % ----  result ----
    \pin{31}{13}{R}{Y}

\end{circuitdiagram}
 %%  ************    LibreSilicon's StdCellLibrary   *******************
%%
%%  Organisation:   Chipforge
%%                  Germany / European Union
%%
%%  Profile:        Chipforge focus on fine System-on-Chip Cores in
%%                  Verilog HDL Code which are easy understandable and
%%                  adjustable. For further information see
%%                          www.chipforge.org
%%                  there are projects from small cores up to PCBs, too.
%%
%%  File:           StdCellLib/Documents/Datasheets/Circuitry/AOAAO21131.tex
%%
%%  Purpose:        Circuit File for AOAAO21131
%%
%%  ************    LaTeX with circdia.sty package      ***************
%%
%%  ///////////////////////////////////////////////////////////////////
%%
%%  Copyright (c) 2018 - 2022 by
%%                  chipforge <stdcelllib@nospam.chipforge.org>
%%  All rights reserved.
%%
%%      This Standard Cell Library is licensed under the Libre Silicon
%%      public license; you can redistribute it and/or modify it under
%%      the terms of the Libre Silicon public license as published by
%%      the Libre Silicon alliance, either version 1 of the License, or
%%      (at your option) any later version.
%%
%%      This design is distributed in the hope that it will be useful,
%%      but WITHOUT ANY WARRANTY; without even the implied warranty of
%%      MERCHANTABILITY or FITNESS FOR A PARTICULAR PURPOSE.
%%      See the Libre Silicon Public License for more details.
%%
%%  ///////////////////////////////////////////////////////////////////
\begin{circuitdiagram}[draft]{38}{18}

    \usgate
    % ----  1st column  ----
    \pin{1}{1}{L}{A}
    \pin{1}{5}{L}{A1}
    \gate[\inputs{2}]{and}{5}{3}{R}{}{}

    % ----  2nd column  ----
    \pin{8}{7}{L}{B}
    \gate[\inputs{2}]{or}{12}{5}{R}{}{}

    % ----  3rd column  ----
    \pin{15}{9}{L}{C}
    \gate[\inputs{2}]{and}{19}{7}{R}{}{}

    \pin{15}{15}{L}{D2}
    \pin{15}{13}{L}{D1}
    \pin{15}{11}{L}{D}
    \gate[\inputs{3}]{and}{19}{13}{R}{}{}

    % ----  4th column  ----
    \wire{23}{7}{23}{11}
    \pin{22}{17}{L}{E}
    \wire{23}{15}{23}{17}
    \gate[\inputs{3}]{nor}{26}{13}{R}{}{}

    % ----  5th column  ----
    \gate{not}{33}{13}{R}{}{}

    % ----  result ----
    \pin{37}{13}{R}{Z}

\end{circuitdiagram}

%%  ************    LibreSilicon's StdCellLibrary   *******************
%%
%%  Organisation:   Chipforge
%%                  Germany / European Union
%%
%%  Profile:        Chipforge focus on fine System-on-Chip Cores in
%%                  Verilog HDL Code which are easy understandable and
%%                  adjustable. For further information see
%%                          www.chipforge.org
%%                  there are projects from small cores up to PCBs, too.
%%
%%  File:           StdCellLib/Documents/Datasheets/Circuitry/AOAAOI21141.tex
%%
%%  Purpose:        Circuit File for AOAAOI21141
%%
%%  ************    LaTeX with circdia.sty package      ***************
%%
%%  ///////////////////////////////////////////////////////////////////
%%
%%  Copyright (c) 2018 - 2022 by
%%                  chipforge <stdcelllib@nospam.chipforge.org>
%%  All rights reserved.
%%
%%      This Standard Cell Library is licensed under the Libre Silicon
%%      public license; you can redistribute it and/or modify it under
%%      the terms of the Libre Silicon public license as published by
%%      the Libre Silicon alliance, either version 1 of the License, or
%%      (at your option) any later version.
%%
%%      This design is distributed in the hope that it will be useful,
%%      but WITHOUT ANY WARRANTY; without even the implied warranty of
%%      MERCHANTABILITY or FITNESS FOR A PARTICULAR PURPOSE.
%%      See the Libre Silicon Public License for more details.
%%
%%  ///////////////////////////////////////////////////////////////////
\begin{circuitdiagram}[draft]{32}{19}

    \usgate
    % ----  1st column  ----
    \pin{1}{1}{L}{A}
    \pin{1}{5}{L}{A1}
    \gate[\inputs{2}]{and}{5}{3}{R}{}{}

    % ----  2nd column  ----
    \pin{8}{7}{L}{B}
    \gate[\inputs{2}]{or}{12}{5}{R}{}{}

    % ----  3rd column  ----
    \pin{15}{9}{L}{C}
    \gate[\inputs{2}]{and}{19}{7}{R}{}{}

    \pin{15}{11}{L}{D}
    \pin{15}{13}{L}{D1}
    \pin{15}{15}{L}{D2}
    \pin{15}{17}{L}{D3}
    \gate[\inputs{4}]{and}{19}{14}{R}{}{}

    % ----  4th column  ----
    \wire{23}{7}{23}{12}
    \pin{22}{18}{L}{E}
    \wire{23}{16}{23}{18}
    \gate[\inputs{3}]{nor}{26}{14}{R}{}{}


    % ----  result ----
    \pin{31}{14}{R}{Y}

\end{circuitdiagram}
 %%  ************    LibreSilicon's StdCellLibrary   *******************
%%
%%  Organisation:   Chipforge
%%                  Germany / European Union
%%
%%  Profile:        Chipforge focus on fine System-on-Chip Cores in
%%                  Verilog HDL Code which are easy understandable and
%%                  adjustable. For further information see
%%                          www.chipforge.org
%%                  there are projects from small cores up to PCBs, too.
%%
%%  File:           StdCellLib/Documents/Datasheets/Circuitry/AOAAO21141.tex
%%
%%  Purpose:        Circuit File for AOAAO21141
%%
%%  ************    LaTeX with circdia.sty package      ***************
%%
%%  ///////////////////////////////////////////////////////////////////
%%
%%  Copyright (c) 2018 - 2022 by
%%                  chipforge <stdcelllib@nospam.chipforge.org>
%%  All rights reserved.
%%
%%      This Standard Cell Library is licensed under the Libre Silicon
%%      public license; you can redistribute it and/or modify it under
%%      the terms of the Libre Silicon public license as published by
%%      the Libre Silicon alliance, either version 1 of the License, or
%%      (at your option) any later version.
%%
%%      This design is distributed in the hope that it will be useful,
%%      but WITHOUT ANY WARRANTY; without even the implied warranty of
%%      MERCHANTABILITY or FITNESS FOR A PARTICULAR PURPOSE.
%%      See the Libre Silicon Public License for more details.
%%
%%  ///////////////////////////////////////////////////////////////////
\begin{circuitdiagram}[draft]{38}{19}

    \usgate
    % ----  1st column  ----
    \pin{1}{1}{L}{A}
    \pin{1}{5}{L}{A1}
    \gate[\inputs{2}]{and}{5}{3}{R}{}{}

    % ----  2nd column  ----
    \pin{8}{7}{L}{B}
    \gate[\inputs{2}]{or}{12}{5}{R}{}{}

    % ----  3rd column  ----
    \pin{15}{9}{L}{C}
    \gate[\inputs{2}]{and}{19}{7}{R}{}{}

    \pin{15}{11}{L}{D}
    \pin{15}{13}{L}{D1}
    \pin{15}{15}{L}{D2}
    \pin{15}{17}{L}{D2}
    \gate[\inputs{4}]{and}{19}{14}{R}{}{}

    % ----  4th column  ----
    \wire{23}{7}{23}{12}
    \pin{22}{18}{L}{E}
    \wire{23}{16}{23}{18}
    \gate[\inputs{3}]{nor}{26}{14}{R}{}{}

    % ----  last column ----
    \gate{not}{33}{14}{R}{}{}

    % ----  result ----
    \pin{37}{14}{R}{Z}

\end{circuitdiagram}

%%  ************    LibreSilicon's StdCellLibrary   *******************
%%
%%  Organisation:   Chipforge
%%                  Germany / European Union
%%
%%  Profile:        Chipforge focus on fine System-on-Chip Cores in
%%                  Verilog HDL Code which are easy understandable and
%%                  adjustable. For further information see
%%                          www.chipforge.org
%%                  there are projects from small cores up to PCBs, too.
%%
%%  File:           StdCellLib/Documents/Datasheets/Circuitry/AOAAOI21221.tex
%%
%%  Purpose:        Circuit File for AOAAOI21221
%%
%%  ************    LaTeX with circdia.sty package      ***************
%%
%%  ///////////////////////////////////////////////////////////////////
%%
%%  Copyright (c) 2018 - 2022 by
%%                  chipforge <stdcelllib@nospam.chipforge.org>
%%  All rights reserved.
%%
%%      This Standard Cell Library is licensed under the Libre Silicon
%%      public license; you can redistribute it and/or modify it under
%%      the terms of the Libre Silicon public license as published by
%%      the Libre Silicon alliance, either version 1 of the License, or
%%      (at your option) any later version.
%%
%%      This design is distributed in the hope that it will be useful,
%%      but WITHOUT ANY WARRANTY; without even the implied warranty of
%%      MERCHANTABILITY or FITNESS FOR A PARTICULAR PURPOSE.
%%      See the Libre Silicon Public License for more details.
%%
%%  ///////////////////////////////////////////////////////////////////
\begin{circuitdiagram}[draft]{32}{20}

    \usgate
    % ----  1st column  ----
    \pin{1}{1}{L}{A}
    \pin{1}{5}{L}{A1}
    \gate[\inputs{2}]{and}{5}{3}{R}{}{}

    % ----  2nd column  ----
    \pin{8}{7}{L}{B}
    \gate[\inputs{2}]{or}{12}{5}{R}{}{}

    % ----  3rd column  ----
    \wire{16}{5}{16}{7}
    \pin{15}{9}{L}{C}
    \pin{15}{11}{L}{C1}
    \gate[\inputs{3}]{and}{19}{9}{R}{}{}

    \pin{15}{13}{L}{D}
    \pin{15}{17}{L}{D1}
    \gate[\inputs{2}]{and}{19}{15}{R}{}{}

    % ----  4th column  ----
    \wire{23}{9}{23}{13}
    \pin{22}{19}{L}{E}
    \wire{23}{17}{23}{19}
    \gate[\inputs{3}]{nor}{26}{15}{R}{}{}


    % ----  result ----
    \pin{31}{15}{R}{Y}

\end{circuitdiagram}
 %%  ************    LibreSilicon's StdCellLibrary   *******************
%%
%%  Organisation:   Chipforge
%%                  Germany / European Union
%%
%%  Profile:        Chipforge focus on fine System-on-Chip Cores in
%%                  Verilog HDL Code which are easy understandable and
%%                  adjustable. For further information see
%%                          www.chipforge.org
%%                  there are projects from small cores up to PCBs, too.
%%
%%  File:           StdCellLib/Documents/Datasheets/Circuitry/AOAAO21221.tex
%%
%%  Purpose:        Circuit File for AOAAO21221
%%
%%  ************    LaTeX with circdia.sty package      ***************
%%
%%  ///////////////////////////////////////////////////////////////////
%%
%%  Copyright (c) 2018 - 2022 by
%%                  chipforge <stdcelllib@nospam.chipforge.org>
%%  All rights reserved.
%%
%%      This Standard Cell Library is licensed under the Libre Silicon
%%      public license; you can redistribute it and/or modify it under
%%      the terms of the Libre Silicon public license as published by
%%      the Libre Silicon alliance, either version 1 of the License, or
%%      (at your option) any later version.
%%
%%      This design is distributed in the hope that it will be useful,
%%      but WITHOUT ANY WARRANTY; without even the implied warranty of
%%      MERCHANTABILITY or FITNESS FOR A PARTICULAR PURPOSE.
%%      See the Libre Silicon Public License for more details.
%%
%%  ///////////////////////////////////////////////////////////////////
\begin{circuitdiagram}[draft]{38}{20}

    \usgate
    % ----  1st column  ----
    \pin{1}{1}{L}{A}
    \pin{1}{5}{L}{A1}
    \gate[\inputs{2}]{and}{5}{3}{R}{}{}

    % ----  2nd column  ----
    \pin{8}{7}{L}{B}
    \gate[\inputs{2}]{or}{12}{5}{R}{}{}

    % ----  3rd column  ----
    \wire{16}{5}{16}{7}
    \pin{15}{9}{L}{C}
    \pin{15}{11}{L}{C1}
    \gate[\inputs{3}]{and}{19}{9}{R}{}{}

    \pin{15}{13}{L}{D}
    \pin{15}{17}{L}{D1}
    \gate[\inputs{2}]{and}{19}{15}{R}{}{}

    % ----  4th column  ----
    \wire{23}{9}{23}{13}
    \pin{22}{19}{L}{E}
    \wire{23}{17}{23}{19}
    \gate[\inputs{3}]{nor}{26}{15}{R}{}{}

    % ----  last column ----
    \gate{not}{33}{15}{R}{}{}

    % ----  result ----
    \pin{37}{15}{R}{Z}

\end{circuitdiagram}

%%  ************    LibreSilicon's StdCellLibrary   *******************
%%
%%  Organisation:   Chipforge
%%                  Germany / European Union
%%
%%  Profile:        Chipforge focus on fine System-on-Chip Cores in
%%                  Verilog HDL Code which are easy understandable and
%%                  adjustable. For further information see
%%                          www.chipforge.org
%%                  there are projects from small cores up to PCBs, too.
%%
%%  File:           StdCellLib/Documents/Datasheets/Circuitry/AOAAOI21231.tex
%%
%%  Purpose:        Circuit File for AOAAOI21231
%%
%%  ************    LaTeX with circdia.sty package      ***************
%%
%%  ///////////////////////////////////////////////////////////////////
%%
%%  Copyright (c) 2018 - 2022 by
%%                  chipforge <stdcelllib@nospam.chipforge.org>
%%  All rights reserved.
%%
%%      This Standard Cell Library is licensed under the Libre Silicon
%%      public license; you can redistribute it and/or modify it under
%%      the terms of the Libre Silicon public license as published by
%%      the Libre Silicon alliance, either version 1 of the License, or
%%      (at your option) any later version.
%%
%%      This design is distributed in the hope that it will be useful,
%%      but WITHOUT ANY WARRANTY; without even the implied warranty of
%%      MERCHANTABILITY or FITNESS FOR A PARTICULAR PURPOSE.
%%      See the Libre Silicon Public License for more details.
%%
%%  ///////////////////////////////////////////////////////////////////
\begin{circuitdiagram}[draft]{32}{20}

    \usgate
    % ----  1st column  ----
    \pin{1}{1}{L}{A}
    \pin{1}{5}{L}{A1}
    \gate[\inputs{2}]{and}{5}{3}{R}{}{}

    % ----  2nd column  ----
    \pin{8}{7}{L}{B}
    \gate[\inputs{2}]{or}{12}{5}{R}{}{}

    % ----  3rd column  ----
    \wire{16}{5}{16}{7}
    \pin{15}{9}{L}{C}
    \pin{15}{11}{L}{C1}
    \gate[\inputs{3}]{and}{19}{9}{R}{}{}

    \pin{15}{13}{L}{D}
    \pin{15}{15}{L}{D1}
    \pin{15}{17}{L}{D2}
    \gate[\inputs{3}]{and}{19}{15}{R}{}{}

    % ----  4th column  ----
    \wire{23}{9}{23}{13}
    \pin{22}{19}{L}{E}
    \wire{23}{17}{23}{19}
    \gate[\inputs{3}]{nor}{26}{15}{R}{}{}


    % ----  result ----
    \pin{31}{15}{R}{Y}

\end{circuitdiagram}
 %%  ************    LibreSilicon's StdCellLibrary   *******************
%%
%%  Organisation:   Chipforge
%%                  Germany / European Union
%%
%%  Profile:        Chipforge focus on fine System-on-Chip Cores in
%%                  Verilog HDL Code which are easy understandable and
%%                  adjustable. For further information see
%%                          www.chipforge.org
%%                  there are projects from small cores up to PCBs, too.
%%
%%  File:           StdCellLib/Documents/Datasheets/Circuitry/AOAAO21231.tex
%%
%%  Purpose:        Circuit File for AOAAO21231
%%
%%  ************    LaTeX with circdia.sty package      ***************
%%
%%  ///////////////////////////////////////////////////////////////////
%%
%%  Copyright (c) 2018 - 2022 by
%%                  chipforge <stdcelllib@nospam.chipforge.org>
%%  All rights reserved.
%%
%%      This Standard Cell Library is licensed under the Libre Silicon
%%      public license; you can redistribute it and/or modify it under
%%      the terms of the Libre Silicon public license as published by
%%      the Libre Silicon alliance, either version 1 of the License, or
%%      (at your option) any later version.
%%
%%      This design is distributed in the hope that it will be useful,
%%      but WITHOUT ANY WARRANTY; without even the implied warranty of
%%      MERCHANTABILITY or FITNESS FOR A PARTICULAR PURPOSE.
%%      See the Libre Silicon Public License for more details.
%%
%%  ///////////////////////////////////////////////////////////////////
\begin{circuitdiagram}[draft]{38}{20}

    \usgate
    % ----  1st column  ----
    \pin{1}{1}{L}{A}
    \pin{1}{5}{L}{A1}
    \gate[\inputs{2}]{and}{5}{3}{R}{}{}

    % ----  2nd column  ----
    \pin{8}{7}{L}{B}
    \gate[\inputs{2}]{or}{12}{5}{R}{}{}

    % ----  3rd column  ----
    \wire{16}{5}{16}{7}
    \pin{15}{9}{L}{C}
    \pin{15}{11}{L}{C1}
    \gate[\inputs{3}]{and}{19}{9}{R}{}{}

    \pin{15}{13}{L}{D}
    \pin{15}{15}{L}{D1}
    \pin{15}{17}{L}{D2}
    \gate[\inputs{3}]{and}{19}{15}{R}{}{}

    % ----  4th column  ----
    \wire{23}{9}{23}{13}
    \pin{22}{19}{L}{E}
    \wire{23}{17}{23}{19}
    \gate[\inputs{3}]{nor}{26}{15}{R}{}{}

    % ----  last column ----
    \gate{not}{33}{15}{R}{}{}

    % ----  result ----
    \pin{37}{15}{R}{Z}

\end{circuitdiagram}

%%  ************    LibreSilicon's StdCellLibrary   *******************
%%
%%  Organisation:   Chipforge
%%                  Germany / European Union
%%
%%  Profile:        Chipforge focus on fine System-on-Chip Cores in
%%                  Verilog HDL Code which are easy understandable and
%%                  adjustable. For further information see
%%                          www.chipforge.org
%%                  there are projects from small cores up to PCBs, too.
%%
%%  File:           StdCellLib/Documents/Datasheets/Circuitry/AOAAOI31121.tex
%%
%%  Purpose:        Circuit File for AOAAOI31121
%%
%%  ************    LaTeX with circdia.sty package      ***************
%%
%%  ///////////////////////////////////////////////////////////////////
%%
%%  Copyright (c) 2018 - 2022 by
%%                  chipforge <stdcelllib@nospam.chipforge.org>
%%  All rights reserved.
%%
%%      This Standard Cell Library is licensed under the Libre Silicon
%%      public license; you can redistribute it and/or modify it under
%%      the terms of the Libre Silicon public license as published by
%%      the Libre Silicon alliance, either version 1 of the License, or
%%      (at your option) any later version.
%%
%%      This design is distributed in the hope that it will be useful,
%%      but WITHOUT ANY WARRANTY; without even the implied warranty of
%%      MERCHANTABILITY or FITNESS FOR A PARTICULAR PURPOSE.
%%      See the Libre Silicon Public License for more details.
%%
%%  ///////////////////////////////////////////////////////////////////
\begin{circuitdiagram}[draft]{32}{18}

    \usgate
    % ----  1st column  ----
    \pin{1}{1}{L}{A}
    \pin{1}{3}{L}{A1}
    \pin{1}{5}{L}{A2}
    \gate[\inputs{3}]{and}{5}{3}{R}{}{}

    % ----  2nd column  ----
    \pin{8}{7}{L}{B}
    \gate[\inputs{2}]{or}{12}{5}{R}{}{}

    % ----  3rd column  ----
    \pin{15}{9}{L}{C}
    \gate[\inputs{2}]{and}{19}{7}{R}{}{}

    \pin{15}{15}{L}{D1}
    \pin{15}{11}{L}{D}
    \gate[\inputs{2}]{and}{19}{13}{R}{}{}

    % ----  4th column  ----
    \wire{23}{7}{23}{11}
    \pin{22}{17}{L}{E}
    \wire{23}{15}{23}{17}
    \gate[\inputs{3}]{nor}{26}{13}{R}{}{}


    % ----  result ----
    \pin{31}{13}{R}{Y}

\end{circuitdiagram}
 %%  ************    LibreSilicon's StdCellLibrary   *******************
%%
%%  Organisation:   Chipforge
%%                  Germany / European Union
%%
%%  Profile:        Chipforge focus on fine System-on-Chip Cores in
%%                  Verilog HDL Code which are easy understandable and
%%                  adjustable. For further information see
%%                          www.chipforge.org
%%                  there are projects from small cores up to PCBs, too.
%%
%%  File:           StdCellLib/Documents/Datasheets/Circuitry/AOAAO31121.tex
%%
%%  Purpose:        Circuit File for AOAAO31121
%%
%%  ************    LaTeX with circdia.sty package      ***************
%%
%%  ///////////////////////////////////////////////////////////////////
%%
%%  Copyright (c) 2018 - 2022 by
%%                  chipforge <stdcelllib@nospam.chipforge.org>
%%  All rights reserved.
%%
%%      This Standard Cell Library is licensed under the Libre Silicon
%%      public license; you can redistribute it and/or modify it under
%%      the terms of the Libre Silicon public license as published by
%%      the Libre Silicon alliance, either version 1 of the License, or
%%      (at your option) any later version.
%%
%%      This design is distributed in the hope that it will be useful,
%%      but WITHOUT ANY WARRANTY; without even the implied warranty of
%%      MERCHANTABILITY or FITNESS FOR A PARTICULAR PURPOSE.
%%      See the Libre Silicon Public License for more details.
%%
%%  ///////////////////////////////////////////////////////////////////
\begin{circuitdiagram}[draft]{38}{18}

    \usgate
    % ----  1st column  ----
    \pin{1}{1}{L}{A}
    \pin{1}{3}{L}{A1}
    \pin{1}{5}{L}{A2}
    \gate[\inputs{3}]{and}{5}{3}{R}{}{}

    % ----  2nd column  ----
    \pin{8}{7}{L}{B}
    \gate[\inputs{2}]{or}{12}{5}{R}{}{}

    % ----  3rd column  ----
    \pin{15}{9}{L}{C}
    \gate[\inputs{2}]{and}{19}{7}{R}{}{}

    \pin{15}{15}{L}{D1}
    \pin{15}{11}{L}{D}
    \gate[\inputs{2}]{and}{19}{13}{R}{}{}

    % ----  4th column  ----
    \wire{23}{7}{23}{11}
    \pin{22}{17}{L}{E}
    \wire{23}{15}{23}{17}
    \gate[\inputs{3}]{nor}{26}{13}{R}{}{}

    % ----  5th column  ----
    \gate{not}{33}{13}{R}{}{}

    % ----  result ----
    \pin{37}{13}{R}{Z}

\end{circuitdiagram}


%%  ************    LibreSilicon's StdCellLibrary   *******************
%%
%%  Organisation:   Chipforge
%%                  Germany / European Union
%%
%%  Profile:        Chipforge focus on fine System-on-Chip Cores in
%%                  Verilog HDL Code which are easy understandable and
%%                  adjustable. For further information see
%%                          www.chipforge.org
%%                  there are projects from small cores up to PCBs, too.
%%
%%  File:           StdCellLib/Documents/section-OAOOAI_complex.tex
%%
%%  Purpose:        Section Level File for Standard Cell Library Documentation
%%
%%  ************    LaTeX with circdia.sty package      ***************
%%
%%  ///////////////////////////////////////////////////////////////////
%%
%%  Copyright (c) 2018 - 2022 by
%%                  chipforge <stdcelllib@nospam.chipforge.org>
%%  All rights reserved.
%%
%%      This Standard Cell Library is licensed under the Libre Silicon
%%      public license; you can redistribute it and/or modify it under
%%      the terms of the Libre Silicon public license as published by
%%      the Libre Silicon alliance, either version 1 of the License, or
%%      (at your option) any later version.
%%
%%      This design is distributed in the hope that it will be useful,
%%      but WITHOUT ANY WARRANTY; without even the implied warranty of
%%      MERCHANTABILITY or FITNESS FOR A PARTICULAR PURPOSE.
%%      See the Libre Silicon Public License for more details.
%%
%%  ///////////////////////////////////////////////////////////////////
\section{OR-AND-OR-OR-AND(-Invert) Complex Gates}

%%  ************    LibreSilicon's StdCellLibrary   *******************
%%
%%  Organisation:   Chipforge
%%                  Germany / European Union
%%
%%  Profile:        Chipforge focus on fine System-on-Chip Cores in
%%                  Verilog HDL Code which are easy understandable and
%%                  adjustable. For further information see
%%                          www.chipforge.org
%%                  there are projects from small cores up to PCBs, too.
%%
%%  File:           StdCellLib/Documents/Datasheets/Circuitry/OAOOAI2112.tex
%%
%%  Purpose:        Circuit File for OAOOAI2112
%%
%%  ************    LaTeX with circdia.sty package      ***************
%%
%%  ///////////////////////////////////////////////////////////////////
%%
%%  Copyright (c) 2018 - 2022 by
%%                  chipforge <stdcelllib@nospam.chipforge.org>
%%  All rights reserved.
%%
%%      This Standard Cell Library is licensed under the Libre Silicon
%%      public license; you can redistribute it and/or modify it under
%%      the terms of the Libre Silicon public license as published by
%%      the Libre Silicon alliance, either version 1 of the License, or
%%      (at your option) any later version.
%%
%%      This design is distributed in the hope that it will be useful,
%%      but WITHOUT ANY WARRANTY; without even the implied warranty of
%%      MERCHANTABILITY or FITNESS FOR A PARTICULAR PURPOSE.
%%      See the Libre Silicon Public License for more details.
%%
%%  ///////////////////////////////////////////////////////////////////
\begin{circuitdiagram}[draft]{32}{16}

    \usgate
    % ----  1st column  ----
    \pin{1}{1}{L}{A}
    \pin{1}{5}{L}{A1}
    \gate[\inputs{2}]{or}{5}{3}{R}{}{}

    % ----  2nd column  ----
    \pin{8}{7}{L}{B}
    \gate[\inputs{2}]{and}{12}{5}{R}{}{}

    % ----  3rd column  ----
    \pin{15}{9}{L}{C}
    \gate[\inputs{2}]{or}{19}{7}{R}{}{}

    \pin{15}{15}{L}{D1}
    \pin{15}{11}{L}{D}
    \gate[\inputs{2}]{or}{19}{13}{R}{}{}

    % ----  4th column  ----
    \wire{23}{7}{23}{9}
    \gate[\inputs{2}]{nand}{26}{11}{R}{}{}


    % ----  result ----
    \pin{31}{11}{R}{Y}

\end{circuitdiagram}
 %%  ************    LibreSilicon's StdCellLibrary   *******************
%%
%%  Organisation:   Chipforge
%%                  Germany / European Union
%%
%%  Profile:        Chipforge focus on fine System-on-Chip Cores in
%%                  Verilog HDL Code which are easy understandable and
%%                  adjustable. For further information see
%%                          www.chipforge.org
%%                  there are projects from small cores up to PCBs, too.
%%
%%  File:           StdCellLib/Documents/Datasheets/Circuitry/OAOOA2112.tex
%%
%%  Purpose:        Circuit File for OAOOA2112
%%
%%  ************    LaTeX with circdia.sty package      ***************
%%
%%  ///////////////////////////////////////////////////////////////////
%%
%%  Copyright (c) 2018 - 2022 by
%%                  chipforge <stdcelllib@nospam.chipforge.org>
%%  All rights reserved.
%%
%%      This Standard Cell Library is licensed under the Libre Silicon
%%      public license; you can redistribute it and/or modify it under
%%      the terms of the Libre Silicon public license as published by
%%      the Libre Silicon alliance, either version 1 of the License, or
%%      (at your option) any later version.
%%
%%      This design is distributed in the hope that it will be useful,
%%      but WITHOUT ANY WARRANTY; without even the implied warranty of
%%      MERCHANTABILITY or FITNESS FOR A PARTICULAR PURPOSE.
%%      See the Libre Silicon Public License for more details.
%%
%%  ///////////////////////////////////////////////////////////////////
\begin{circuitdiagram}[draft]{38}{16}

    \usgate
    % ----  1st column  ----
    \pin{1}{1}{L}{A}
    \pin{1}{5}{L}{A1}
    \gate[\inputs{2}]{or}{5}{3}{R}{}{}

    % ----  2nd column  ----
    \pin{8}{7}{L}{B}
    \gate[\inputs{2}]{and}{12}{5}{R}{}{}

    % ----  3rd column  ----
    \pin{15}{9}{L}{C}
    \gate[\inputs{2}]{or}{19}{7}{R}{}{}

    \pin{15}{15}{L}{D1}
    \pin{15}{11}{L}{D}
    \gate[\inputs{2}]{or}{19}{13}{R}{}{}

    % ----  4th column  ----
    \wire{23}{7}{23}{9}
    \gate[\inputs{2}]{nand}{26}{11}{R}{}{}

    % ----  5th column  ----
    \gate{not}{33}{11}{R}{}{}

    % ----  result ----
    \pin{37}{11}{R}{Z}

\end{circuitdiagram}

%%  ************    LibreSilicon's StdCellLibrary   *******************
%%
%%  Organisation:   Chipforge
%%                  Germany / European Union
%%
%%  Profile:        Chipforge focus on fine System-on-Chip Cores in
%%                  Verilog HDL Code which are easy understandable and
%%                  adjustable. For further information see
%%                          www.chipforge.org
%%                  there are projects from small cores up to PCBs, too.
%%
%%  File:           StdCellLib/Documents/Datasheets/Circuitry/OAOOAI2113.tex
%%
%%  Purpose:        Circuit File for OAOOAI2113
%%
%%  ************    LaTeX with circdia.sty package      ***************
%%
%%  ///////////////////////////////////////////////////////////////////
%%
%%  Copyright (c) 2018 - 2022 by
%%                  chipforge <stdcelllib@nospam.chipforge.org>
%%  All rights reserved.
%%
%%      This Standard Cell Library is licensed under the Libre Silicon
%%      public license; you can redistribute it and/or modify it under
%%      the terms of the Libre Silicon public license as published by
%%      the Libre Silicon alliance, either version 1 of the License, or
%%      (at your option) any later version.
%%
%%      This design is distributed in the hope that it will be useful,
%%      but WITHOUT ANY WARRANTY; without even the implied warranty of
%%      MERCHANTABILITY or FITNESS FOR A PARTICULAR PURPOSE.
%%      See the Libre Silicon Public License for more details.
%%
%%  ///////////////////////////////////////////////////////////////////
\begin{circuitdiagram}[draft]{32}{16}

    \usgate
    % ----  1st column  ----
    \pin{1}{1}{L}{A}
    \pin{1}{5}{L}{A1}
    \gate[\inputs{2}]{or}{5}{3}{R}{}{}

    % ----  2nd column  ----
    \pin{8}{7}{L}{B}
    \gate[\inputs{2}]{and}{12}{5}{R}{}{}

    % ----  3rd column  ----
    \pin{15}{9}{L}{C}
    \gate[\inputs{2}]{or}{19}{7}{R}{}{}

    \pin{15}{11}{L}{D}
    \pin{15}{13}{L}{D1}
    \pin{15}{15}{L}{D2}
    \gate[\inputs{3}]{or}{19}{13}{R}{}{}

    % ----  4th column  ----
    \wire{23}{7}{23}{9}
    \gate[\inputs{2}]{nand}{26}{11}{R}{}{}


    % ----  result ----
    \pin{31}{11}{R}{Y}

\end{circuitdiagram}
 %%  ************    LibreSilicon's StdCellLibrary   *******************
%%
%%  Organisation:   Chipforge
%%                  Germany / European Union
%%
%%  Profile:        Chipforge focus on fine System-on-Chip Cores in
%%                  Verilog HDL Code which are easy understandable and
%%                  adjustable. For further information see
%%                          www.chipforge.org
%%                  there are projects from small cores up to PCBs, too.
%%
%%  File:           StdCellLib/Documents/Datasheets/Circuitry/OAOOA2113.tex
%%
%%  Purpose:        Circuit File for OAOOA2113
%%
%%  ************    LaTeX with circdia.sty package      ***************
%%
%%  ///////////////////////////////////////////////////////////////////
%%
%%  Copyright (c) 2018 - 2022 by
%%                  chipforge <stdcelllib@nospam.chipforge.org>
%%  All rights reserved.
%%
%%      This Standard Cell Library is licensed under the Libre Silicon
%%      public license; you can redistribute it and/or modify it under
%%      the terms of the Libre Silicon public license as published by
%%      the Libre Silicon alliance, either version 1 of the License, or
%%      (at your option) any later version.
%%
%%      This design is distributed in the hope that it will be useful,
%%      but WITHOUT ANY WARRANTY; without even the implied warranty of
%%      MERCHANTABILITY or FITNESS FOR A PARTICULAR PURPOSE.
%%      See the Libre Silicon Public License for more details.
%%
%%  ///////////////////////////////////////////////////////////////////
\begin{circuitdiagram}[draft]{38}{16}

    \usgate
    % ----  1st column  ----
    \pin{1}{1}{L}{A}
    \pin{1}{5}{L}{A1}
    \gate[\inputs{2}]{or}{5}{3}{R}{}{}

    % ----  2nd column  ----
    \pin{8}{7}{L}{B}
    \gate[\inputs{2}]{and}{12}{5}{R}{}{}

    % ----  3rd column  ----
    \pin{15}{9}{L}{C}
    \gate[\inputs{2}]{or}{19}{7}{R}{}{}

    \pin{15}{11}{L}{D}
    \pin{15}{13}{L}{D1}
    \pin{15}{15}{L}{D2}
    \gate[\inputs{3}]{or}{19}{13}{R}{}{}

    % ----  4th column  ----
    \wire{23}{7}{23}{9}
    \gate[\inputs{2}]{nand}{26}{11}{R}{}{}

    % ----  5th column  ----
    \gate{not}{33}{11}{R}{}{}

    % ----  result ----
    \pin{37}{11}{R}{Z}

\end{circuitdiagram}

%%  ************    LibreSilicon's StdCellLibrary   *******************
%%
%%  Organisation:   Chipforge
%%                  Germany / European Union
%%
%%  Profile:        Chipforge focus on fine System-on-Chip Cores in
%%                  Verilog HDL Code which are easy understandable and
%%                  adjustable. For further information see
%%                          www.chipforge.org
%%                  there are projects from small cores up to PCBs, too.
%%
%%  File:           StdCellLib/Documents/Datasheets/Circuitry/OAOOAI2114.tex
%%
%%  Purpose:        Circuit File for OAOOAI2114
%%
%%  ************    LaTeX with circdia.sty package      ***************
%%
%%  ///////////////////////////////////////////////////////////////////
%%
%%  Copyright (c) 2018 - 2022 by
%%                  chipforge <stdcelllib@nospam.chipforge.org>
%%  All rights reserved.
%%
%%      This Standard Cell Library is licensed under the Libre Silicon
%%      public license; you can redistribute it and/or modify it under
%%      the terms of the Libre Silicon public license as published by
%%      the Libre Silicon alliance, either version 1 of the License, or
%%      (at your option) any later version.
%%
%%      This design is distributed in the hope that it will be useful,
%%      but WITHOUT ANY WARRANTY; without even the implied warranty of
%%      MERCHANTABILITY or FITNESS FOR A PARTICULAR PURPOSE.
%%      See the Libre Silicon Public License for more details.
%%
%%  ///////////////////////////////////////////////////////////////////
\begin{circuitdiagram}[draft]{32}{18}

    \usgate
    % ----  1st column  ----
    \pin{1}{1}{L}{A}
    \pin{1}{5}{L}{A1}
    \gate[\inputs{2}]{or}{5}{3}{R}{}{}

    % ----  2nd column  ----
    \pin{8}{7}{L}{B}
    \gate[\inputs{2}]{and}{12}{5}{R}{}{}

    % ----  3rd column  ----
    \pin{15}{9}{L}{C}
    \gate[\inputs{2}]{or}{19}{7}{R}{}{}

    \pin{15}{11}{L}{D}
    \pin{15}{13}{L}{D1}
    \pin{15}{15}{L}{D2}
    \pin{15}{17}{L}{D3}
    \gate[\inputs{4}]{or}{19}{14}{R}{}{}

    % ----  4th column  ----
    \wire{23}{7}{23}{10}
    \gate[\inputs{2}]{nand}{26}{12}{R}{}{}


    % ----  result ----
    \pin{31}{12}{R}{Y}

\end{circuitdiagram}
 %%  ************    LibreSilicon's StdCellLibrary   *******************
%%
%%  Organisation:   Chipforge
%%                  Germany / European Union
%%
%%  Profile:        Chipforge focus on fine System-on-Chip Cores in
%%                  Verilog HDL Code which are easy understandable and
%%                  adjustable. For further information see
%%                          www.chipforge.org
%%                  there are projects from small cores up to PCBs, too.
%%
%%  File:           StdCellLib/Documents/Datasheets/Circuitry/OAOOA2114.tex
%%
%%  Purpose:        Circuit File for OAOOA2114
%%
%%  ************    LaTeX with circdia.sty package      ***************
%%
%%  ///////////////////////////////////////////////////////////////////
%%
%%  Copyright (c) 2018 - 2022 by
%%                  chipforge <stdcelllib@nospam.chipforge.org>
%%  All rights reserved.
%%
%%      This Standard Cell Library is licensed under the Libre Silicon
%%      public license; you can redistribute it and/or modify it under
%%      the terms of the Libre Silicon public license as published by
%%      the Libre Silicon alliance, either version 1 of the License, or
%%      (at your option) any later version.
%%
%%      This design is distributed in the hope that it will be useful,
%%      but WITHOUT ANY WARRANTY; without even the implied warranty of
%%      MERCHANTABILITY or FITNESS FOR A PARTICULAR PURPOSE.
%%      See the Libre Silicon Public License for more details.
%%
%%  ///////////////////////////////////////////////////////////////////
\begin{circuitdiagram}[draft]{38}{18}

    \usgate
    % ----  1st column  ----
    \pin{1}{1}{L}{A}
    \pin{1}{5}{L}{A1}
    \gate[\inputs{2}]{or}{5}{3}{R}{}{}

    % ----  2nd column  ----
    \pin{8}{7}{L}{B}
    \gate[\inputs{2}]{and}{12}{5}{R}{}{}

    % ----  3rd column  ----
    \pin{15}{9}{L}{C}
    \gate[\inputs{2}]{or}{19}{7}{R}{}{}

    \pin{15}{11}{L}{D}
    \pin{15}{13}{L}{D1}
    \pin{15}{15}{L}{D2}
    \pin{15}{17}{L}{D3}
    \gate[\inputs{4}]{or}{19}{14}{R}{}{}

    % ----  4th column  ----
    \wire{23}{7}{23}{10}
    \gate[\inputs{2}]{nand}{26}{12}{R}{}{}

    % ----  5th column  ----
    \gate{not}{33}{12}{R}{}{}

    % ----  result ----
    \pin{37}{12}{R}{Z}

\end{circuitdiagram}

%%  ************    LibreSilicon's StdCellLibrary   *******************
%%
%%  Organisation:   Chipforge
%%                  Germany / European Union
%%
%%  Profile:        Chipforge focus on fine System-on-Chip Cores in
%%                  Verilog HDL Code which are easy understandable and
%%                  adjustable. For further information see
%%                          www.chipforge.org
%%                  there are projects from small cores up to PCBs, too.
%%
%%  File:           StdCellLib/Documents/Datasheets/Circuitry/OAOOAI2122.tex
%%
%%  Purpose:        Circuit File for OAOOAI2122
%%
%%  ************    LaTeX with circdia.sty package      ***************
%%
%%  ///////////////////////////////////////////////////////////////////
%%
%%  Copyright (c) 2018 - 2022 by
%%                  chipforge <stdcelllib@nospam.chipforge.org>
%%  All rights reserved.
%%
%%      This Standard Cell Library is licensed under the Libre Silicon
%%      public license; you can redistribute it and/or modify it under
%%      the terms of the Libre Silicon public license as published by
%%      the Libre Silicon alliance, either version 1 of the License, or
%%      (at your option) any later version.
%%
%%      This design is distributed in the hope that it will be useful,
%%      but WITHOUT ANY WARRANTY; without even the implied warranty of
%%      MERCHANTABILITY or FITNESS FOR A PARTICULAR PURPOSE.
%%      See the Libre Silicon Public License for more details.
%%
%%  ///////////////////////////////////////////////////////////////////
\begin{circuitdiagram}[draft]{32}{18}

    \usgate
    % ----  1st column  ----
    \pin{1}{1}{L}{A}
    \pin{1}{5}{L}{A1}
    \gate[\inputs{2}]{or}{5}{3}{R}{}{}

    % ----  2nd column  ----
    \pin{8}{7}{L}{B}
    \gate[\inputs{2}]{and}{12}{5}{R}{}{}

    % ----  3rd column  ----
    \wire{16}{5}{16}{7}
    \pin{15}{9}{L}{C}
    \pin{15}{11}{L}{C1}
    \gate[\inputs{3}]{or}{19}{9}{R}{}{}

    \pin{15}{13}{L}{D}
    \pin{15}{17}{L}{D1}
    \gate[\inputs{2}]{or}{19}{15}{R}{}{}

    % ----  4th column  ----
    \wire{23}{9}{23}{11}
    \gate[\inputs{2}]{nand}{26}{13}{R}{}{}


    % ----  result ----
    \pin{31}{13}{R}{Y}

\end{circuitdiagram}
 %%  ************    LibreSilicon's StdCellLibrary   *******************
%%
%%  Organisation:   Chipforge
%%                  Germany / European Union
%%
%%  Profile:        Chipforge focus on fine System-on-Chip Cores in
%%                  Verilog HDL Code which are easy understandable and
%%                  adjustable. For further information see
%%                          www.chipforge.org
%%                  there are projects from small cores up to PCBs, too.
%%
%%  File:           StdCellLib/Documents/Datasheets/Circuitry/OAOOA2122.tex
%%
%%  Purpose:        Circuit File for OAOOA2122 
%%
%%  ************    LaTeX with circdia.sty package      ***************
%%
%%  ///////////////////////////////////////////////////////////////////
%%
%%  Copyright (c) 2018 - 2022 by
%%                  chipforge <stdcelllib@nospam.chipforge.org>
%%  All rights reserved.
%%
%%      This Standard Cell Library is licensed under the Libre Silicon
%%      public license; you can redistribute it and/or modify it under
%%      the terms of the Libre Silicon public license as published by
%%      the Libre Silicon alliance, either version 1 of the License, or
%%      (at your option) any later version.
%%
%%      This design is distributed in the hope that it will be useful,
%%      but WITHOUT ANY WARRANTY; without even the implied warranty of
%%      MERCHANTABILITY or FITNESS FOR A PARTICULAR PURPOSE.
%%      See the Libre Silicon Public License for more details.
%%
%%  ///////////////////////////////////////////////////////////////////
\begin{circuitdiagram}[draft]{38}{18}

    \usgate
    % ----  1st column  ----
    \pin{1}{1}{L}{A}
    \pin{1}{5}{L}{A1}
    \gate[\inputs{2}]{or}{5}{3}{R}{}{}

    % ----  2nd column  ----
    \pin{8}{7}{L}{B}
    \gate[\inputs{2}]{and}{12}{5}{R}{}{}

    % ----  3rd column  ----
    \wire{16}{5}{16}{7}
    \pin{15}{9}{L}{C}
    \pin{15}{11}{L}{C1}
    \gate[\inputs{3}]{or}{19}{9}{R}{}{}

    \pin{15}{13}{L}{D}
    \pin{15}{17}{L}{D1}
    \gate[\inputs{2}]{or}{19}{15}{R}{}{}

    % ----  4th column  ----
    \wire{23}{9}{23}{11}
    \gate[\inputs{2}]{nand}{26}{13}{R}{}{}

    % ----  5th column  ----
    \gate{not}{33}{13}{R}{}{}

    % ----  result ----
    \pin{37}{13}{R}{Z}

\end{circuitdiagram}

%%  ************    LibreSilicon's StdCellLibrary   *******************
%%
%%  Organisation:   Chipforge
%%                  Germany / European Union
%%
%%  Profile:        Chipforge focus on fine System-on-Chip Cores in
%%                  Verilog HDL Code which are easy understandable and
%%                  adjustable. For further information see
%%                          www.chipforge.org
%%                  there are projects from small cores up to PCBs, too.
%%
%%  File:           StdCellLib/Documents/Datasheets/Circuitry/OAOOAI2123.tex
%%
%%  Purpose:        Circuit File for OAOOAI2123
%%
%%  ************    LaTeX with circdia.sty package      ***************
%%
%%  ///////////////////////////////////////////////////////////////////
%%
%%  Copyright (c) 2018 - 2022 by
%%                  chipforge <stdcelllib@nospam.chipforge.org>
%%  All rights reserved.
%%
%%      This Standard Cell Library is licensed under the Libre Silicon
%%      public license; you can redistribute it and/or modify it under
%%      the terms of the Libre Silicon public license as published by
%%      the Libre Silicon alliance, either version 1 of the License, or
%%      (at your option) any later version.
%%
%%      This design is distributed in the hope that it will be useful,
%%      but WITHOUT ANY WARRANTY; without even the implied warranty of
%%      MERCHANTABILITY or FITNESS FOR A PARTICULAR PURPOSE.
%%      See the Libre Silicon Public License for more details.
%%
%%  ///////////////////////////////////////////////////////////////////
\begin{circuitdiagram}[draft]{32}{18}

    \usgate
    % ----  1st column  ----
    \pin{1}{1}{L}{A}
    \pin{1}{5}{L}{A1}
    \gate[\inputs{2}]{or}{5}{3}{R}{}{}

    % ----  2nd column  ----
    \pin{8}{7}{L}{B}
    \gate[\inputs{2}]{and}{12}{5}{R}{}{}

    % ----  3rd column  ----
    \wire{16}{5}{16}{7}
    \pin{15}{9}{L}{C}
    \pin{15}{11}{L}{C1}
    \gate[\inputs{3}]{or}{19}{9}{R}{}{}

    \pin{15}{13}{L}{D}
    \pin{15}{15}{L}{D1}
    \pin{15}{17}{L}{D2}
    \gate[\inputs{3}]{or}{19}{15}{R}{}{}

    % ----  4th column  ----
    \wire{23}{9}{23}{11}
    \gate[\inputs{2}]{nand}{26}{13}{R}{}{}


    % ----  result ----
    \pin{31}{13}{R}{Y}

\end{circuitdiagram}
 %%  ************    LibreSilicon's StdCellLibrary   *******************
%%
%%  Organisation:   Chipforge
%%                  Germany / European Union
%%
%%  Profile:        Chipforge focus on fine System-on-Chip Cores in
%%                  Verilog HDL Code which are easy understandable and
%%                  adjustable. For further information see
%%                          www.chipforge.org
%%                  there are projects from small cores up to PCBs, too.
%%
%%  File:           StdCellLib/Documents/Datasheets/Circuitry/OAOOA2123.tex
%%
%%  Purpose:        Circuit File for OAOOA2123
%%
%%  ************    LaTeX with circdia.sty package      ***************
%%
%%  ///////////////////////////////////////////////////////////////////
%%
%%  Copyright (c) 2018 - 2022 by
%%                  chipforge <stdcelllib@nospam.chipforge.org>
%%  All rights reserved.
%%
%%      This Standard Cell Library is licensed under the Libre Silicon
%%      public license; you can redistribute it and/or modify it under
%%      the terms of the Libre Silicon public license as published by
%%      the Libre Silicon alliance, either version 1 of the License, or
%%      (at your option) any later version.
%%
%%      This design is distributed in the hope that it will be useful,
%%      but WITHOUT ANY WARRANTY; without even the implied warranty of
%%      MERCHANTABILITY or FITNESS FOR A PARTICULAR PURPOSE.
%%      See the Libre Silicon Public License for more details.
%%
%%  ///////////////////////////////////////////////////////////////////
\begin{circuitdiagram}[draft]{38}{18}

    \usgate
    % ----  1st column  ----
    \pin{1}{1}{L}{A}
    \pin{1}{5}{L}{A1}
    \gate[\inputs{2}]{or}{5}{3}{R}{}{}

    % ----  2nd column  ----
    \pin{8}{7}{L}{B}
    \gate[\inputs{2}]{and}{12}{5}{R}{}{}

    % ----  3rd column  ----
    \wire{16}{5}{16}{7}
    \pin{15}{9}{L}{C}
    \pin{15}{11}{L}{C1}
    \gate[\inputs{3}]{or}{19}{9}{R}{}{}

    \pin{15}{13}{L}{D}
    \pin{15}{15}{L}{D1}
    \pin{15}{17}{L}{D2}
    \gate[\inputs{3}]{or}{19}{15}{R}{}{}

    % ----  4th column  ----
    \wire{23}{9}{23}{11}
    \gate[\inputs{2}]{nand}{26}{13}{R}{}{}

    % ----  last column ----
    \gate{not}{33}{13}{R}{}{}

    % ----  result ----
    \pin{37}{13}{R}{Z}

\end{circuitdiagram}

\include{Datasheets/OAOOAI2124} \include{Datasheets/OAOOA2124}
%%  ************    LibreSilicon's StdCellLibrary   *******************
%%
%%  Organisation:   Chipforge
%%                  Germany / European Union
%%
%%  Profile:        Chipforge focus on fine System-on-Chip Cores in
%%                  Verilog HDL Code which are easy understandable and
%%                  adjustable. For further information see
%%                          www.chipforge.org
%%                  there are projects from small cores up to PCBs, too.
%%
%%  File:           StdCellLib/Documents/Datasheets/Circuitry/OAOOAI2212.tex
%%
%%  Purpose:        Circuit File for OAOOAI2212
%%
%%  ************    LaTeX with circdia.sty package      ***************
%%
%%  ///////////////////////////////////////////////////////////////////
%%
%%  Copyright (c) 2018 - 2022 by
%%                  chipforge <stdcelllib@nospam.chipforge.org>
%%  All rights reserved.
%%
%%      This Standard Cell Library is licensed under the Libre Silicon
%%      public license; you can redistribute it and/or modify it under
%%      the terms of the Libre Silicon public license as published by
%%      the Libre Silicon alliance, either version 1 of the License, or
%%      (at your option) any later version.
%%
%%      This design is distributed in the hope that it will be useful,
%%      but WITHOUT ANY WARRANTY; without even the implied warranty of
%%      MERCHANTABILITY or FITNESS FOR A PARTICULAR PURPOSE.
%%      See the Libre Silicon Public License for more details.
%%
%%  ///////////////////////////////////////////////////////////////////
\begin{circuitdiagram}[draft]{32}{18}

    \usgate
    % ----  1st column  ----
    \pin{1}{1}{L}{A}
    \pin{1}{5}{L}{A1}
    \gate[\inputs{2}]{or}{5}{3}{R}{}{}

    % ----  2nd column  ----
    \wire{9}{3}{9}{5}
    \pin{8}{7}{L}{B}
    \pin{8}{9}{L}{B1}
    \gate[\inputs{3}]{and}{12}{7}{R}{}{}

    % ----  3rd column  ----
    \pin{15}{11}{L}{C}
    \gate[\inputs{2}]{or}{19}{9}{R}{}{}

    \pin{15}{17}{L}{D1}
    \pin{15}{13}{L}{D}
    \gate[\inputs{2}]{or}{19}{15}{R}{}{}

    % ----  4th column  ----
    \wire{23}{9}{23}{11}
    \gate[\inputs{2}]{nand}{26}{13}{R}{}{}


    % ----  result ----
    \pin{31}{13}{R}{Y}

\end{circuitdiagram}
 %%  ************    LibreSilicon's StdCellLibrary   *******************
%%
%%  Organisation:   Chipforge
%%                  Germany / European Union
%%
%%  Profile:        Chipforge focus on fine System-on-Chip Cores in
%%                  Verilog HDL Code which are easy understandable and
%%                  adjustable. For further information see
%%                          www.chipforge.org
%%                  there are projects from small cores up to PCBs, too.
%%
%%  File:           StdCellLib/Documents/Datasheets/Circuitry/OAOOA2212.tex
%%
%%  Purpose:        Circuit File for OAOOA2212
%%
%%  ************    LaTeX with circdia.sty package      ***************
%%
%%  ///////////////////////////////////////////////////////////////////
%%
%%  Copyright (c) 2018 - 2022 by
%%                  chipforge <stdcelllib@nospam.chipforge.org>
%%  All rights reserved.
%%
%%      This Standard Cell Library is licensed under the Libre Silicon
%%      public license; you can redistribute it and/or modify it under
%%      the terms of the Libre Silicon public license as published by
%%      the Libre Silicon alliance, either version 1 of the License, or
%%      (at your option) any later version.
%%
%%      This design is distributed in the hope that it will be useful,
%%      but WITHOUT ANY WARRANTY; without even the implied warranty of
%%      MERCHANTABILITY or FITNESS FOR A PARTICULAR PURPOSE.
%%      See the Libre Silicon Public License for more details.
%%
%%  ///////////////////////////////////////////////////////////////////
\begin{circuitdiagram}[draft]{38}{18}

    \usgate
    % ----  1st column  ----
    \pin{1}{1}{L}{A}
    \pin{1}{5}{L}{A1}
    \gate[\inputs{2}]{or}{5}{3}{R}{}{}

    % ----  2nd column  ----
    \wire{9}{3}{9}{5}
    \pin{8}{7}{L}{B}
    \pin{8}{9}{L}{B1}
    \gate[\inputs{3}]{and}{12}{7}{R}{}{}

    % ----  3rd column  ----
    \pin{15}{11}{L}{C}
    \gate[\inputs{2}]{or}{19}{9}{R}{}{}

    \pin{15}{17}{L}{D1}
    \pin{15}{13}{L}{D}
    \gate[\inputs{2}]{or}{19}{15}{R}{}{}

    % ----  4th column  ----
    \wire{23}{9}{23}{11}
    \gate[\inputs{2}]{nand}{26}{13}{R}{}{}

    % ----  5th column  ----
    \gate{not}{33}{13}{R}{}{}

    % ----  result ----
    \pin{37}{13}{R}{Z}

\end{circuitdiagram}

%%  ************    LibreSilicon's StdCellLibrary   *******************
%%
%%  Organisation:   Chipforge
%%                  Germany / European Union
%%
%%  Profile:        Chipforge focus on fine System-on-Chip Cores in
%%                  Verilog HDL Code which are easy understandable and
%%                  adjustable. For further information see
%%                          www.chipforge.org
%%                  there are projects from small cores up to PCBs, too.
%%
%%  File:           StdCellLib/Documents/Datasheets/Circuitry/OAOOAI2213.tex
%%
%%  Purpose:        Circuit File for OAOOAI2213
%%
%%  ************    LaTeX with circdia.sty package      ***************
%%
%%  ///////////////////////////////////////////////////////////////////
%%
%%  Copyright (c) 2018 - 2022 by
%%                  chipforge <stdcelllib@nospam.chipforge.org>
%%  All rights reserved.
%%
%%      This Standard Cell Library is licensed under the Libre Silicon
%%      public license; you can redistribute it and/or modify it under
%%      the terms of the Libre Silicon public license as published by
%%      the Libre Silicon alliance, either version 1 of the License, or
%%      (at your option) any later version.
%%
%%      This design is distributed in the hope that it will be useful,
%%      but WITHOUT ANY WARRANTY; without even the implied warranty of
%%      MERCHANTABILITY or FITNESS FOR A PARTICULAR PURPOSE.
%%      See the Libre Silicon Public License for more details.
%%
%%  ///////////////////////////////////////////////////////////////////
\begin{circuitdiagram}[draft]{32}{18}

    \usgate
    % ----  1st column  ----
    \pin{1}{1}{L}{A}
    \pin{1}{5}{L}{A1}
    \gate[\inputs{2}]{or}{5}{3}{R}{}{}

    % ----  2nd column  ----
    \wire{9}{3}{9}{5}
    \pin{8}{7}{L}{B}
    \pin{8}{9}{L}{B1}
    \gate[\inputs{3}]{and}{12}{7}{R}{}{}

    % ----  3rd column  ----
    \pin{15}{11}{L}{C}
    \gate[\inputs{2}]{or}{19}{9}{R}{}{}

    \pin{15}{13}{L}{D}
    \pin{15}{15}{L}{D1}
    \pin{15}{17}{L}{D2}
    \gate[\inputs{3}]{or}{19}{15}{R}{}{}

    % ----  4th column  ----
    \wire{23}{9}{23}{11}
    \gate[\inputs{2}]{nand}{26}{13}{R}{}{}


    % ----  result ----
    \pin{31}{13}{R}{Y}

\end{circuitdiagram}
 %%  ************    LibreSilicon's StdCellLibrary   *******************
%%
%%  Organisation:   Chipforge
%%                  Germany / European Union
%%
%%  Profile:        Chipforge focus on fine System-on-Chip Cores in
%%                  Verilog HDL Code which are easy understandable and
%%                  adjustable. For further information see
%%                          www.chipforge.org
%%                  there are projects from small cores up to PCBs, too.
%%
%%  File:           StdCellLib/Documents/Datasheets/Circuitry/OAOOA2213.tex
%%
%%  Purpose:        Circuit File for OAOOA2213
%%
%%  ************    LaTeX with circdia.sty package      ***************
%%
%%  ///////////////////////////////////////////////////////////////////
%%
%%  Copyright (c) 2018 - 2022 by
%%                  chipforge <stdcelllib@nospam.chipforge.org>
%%  All rights reserved.
%%
%%      This Standard Cell Library is licensed under the Libre Silicon
%%      public license; you can redistribute it and/or modify it under
%%      the terms of the Libre Silicon public license as published by
%%      the Libre Silicon alliance, either version 1 of the License, or
%%      (at your option) any later version.
%%
%%      This design is distributed in the hope that it will be useful,
%%      but WITHOUT ANY WARRANTY; without even the implied warranty of
%%      MERCHANTABILITY or FITNESS FOR A PARTICULAR PURPOSE.
%%      See the Libre Silicon Public License for more details.
%%
%%  ///////////////////////////////////////////////////////////////////
\begin{circuitdiagram}[draft]{38}{18}

    \usgate
    % ----  1st column  ----
    \pin{1}{1}{L}{A}
    \pin{1}{5}{L}{A1}
    \gate[\inputs{2}]{or}{5}{3}{R}{}{}

    % ----  2nd column  ----
    \wire{9}{3}{9}{5}
    \pin{8}{7}{L}{B}
    \pin{8}{9}{L}{B1}
    \gate[\inputs{3}]{and}{12}{7}{R}{}{}

    % ----  3rd column  ----
    \pin{15}{11}{L}{C}
    \gate[\inputs{2}]{or}{19}{9}{R}{}{}

    \pin{15}{13}{L}{D}
    \pin{15}{15}{L}{D1}
    \pin{15}{17}{L}{D2}
    \gate[\inputs{3}]{or}{19}{15}{R}{}{}

    % ----  4th column  ----
    \wire{23}{9}{23}{11}
    \gate[\inputs{2}]{nand}{26}{13}{R}{}{}

    % ----  last column ----
    \gate{not}{33}{13}{R}{}{}

    % ----  result ----
    \pin{37}{13}{R}{Z}

\end{circuitdiagram}

\include{Datasheets/OAOOAI2214} \include{Datasheets/OAOOA2214}
%%  ************    LibreSilicon's StdCellLibrary   *******************
%%
%%  Organisation:   Chipforge
%%                  Germany / European Union
%%
%%  Profile:        Chipforge focus on fine System-on-Chip Cores in
%%                  Verilog HDL Code which are easy understandable and
%%                  adjustable. For further information see
%%                          www.chipforge.org
%%                  there are projects from small cores up to PCBs, too.
%%
%%  File:           StdCellLib/Documents/Datasheets/Circuitry/OAOOAI2222.tex
%%
%%  Purpose:        Circuit File for OAOOAI2222
%%
%%  ************    LaTeX with circdia.sty package      ***************
%%
%%  ///////////////////////////////////////////////////////////////////
%%
%%  Copyright (c) 2018 - 2022 by
%%                  chipforge <stdcelllib@nospam.chipforge.org>
%%  All rights reserved.
%%
%%      This Standard Cell Library is licensed under the Libre Silicon
%%      public license; you can redistribute it and/or modify it under
%%      the terms of the Libre Silicon public license as published by
%%      the Libre Silicon alliance, either version 1 of the License, or
%%      (at your option) any later version.
%%
%%      This design is distributed in the hope that it will be useful,
%%      but WITHOUT ANY WARRANTY; without even the implied warranty of
%%      MERCHANTABILITY or FITNESS FOR A PARTICULAR PURPOSE.
%%      See the Libre Silicon Public License for more details.
%%
%%  ///////////////////////////////////////////////////////////////////
\begin{circuitdiagram}[draft]{32}{20}

    \usgate
    % ----  1st column  ----
    \pin{1}{1}{L}{A}
    \pin{1}{5}{L}{A1}
    \gate[\inputs{2}]{or}{5}{3}{R}{}{}

    % ----  2nd column  ----
    \wire{9}{3}{9}{5}
    \pin{8}{7}{L}{B}
    \pin{8}{9}{L}{B1}
    \gate[\inputs{3}]{and}{12}{7}{R}{}{}

    % ----  3rd column  ----
    \wire{16}{7}{16}{9}
    \pin{15}{11}{L}{C}
    \pin{15}{13}{L}{C1}
    \gate[\inputs{3}]{or}{19}{11}{R}{}{}

    \pin{15}{15}{L}{D}
    \pin{15}{19}{L}{D1}
    \gate[\inputs{2}]{or}{19}{17}{R}{}{}

    % ----  4th column  ----
    \wire{23}{11}{23}{13}
    \gate[\inputs{2}]{nand}{26}{15}{R}{}{}


    % ----  result ----
    \pin{31}{15}{R}{Y}

\end{circuitdiagram}
 %%  ************    LibreSilicon's StdCellLibrary   *******************
%%
%%  Organisation:   Chipforge
%%                  Germany / European Union
%%
%%  Profile:        Chipforge focus on fine System-on-Chip Cores in
%%                  Verilog HDL Code which are easy understandable and
%%                  adjustable. For further information see
%%                          www.chipforge.org
%%                  there are projects from small cores up to PCBs, too.
%%
%%  File:           StdCellLib/Documents/Datasheets/Circuitry/OAOOA2222.tex
%%
%%  Purpose:        Circuit File for OAOOA2222
%%
%%  ************    LaTeX with circdia.sty package      ***************
%%
%%  ///////////////////////////////////////////////////////////////////
%%
%%  Copyright (c) 2018 - 2022 by
%%                  chipforge <stdcelllib@nospam.chipforge.org>
%%  All rights reserved.
%%
%%      This Standard Cell Library is licensed under the Libre Silicon
%%      public license; you can redistribute it and/or modify it under
%%      the terms of the Libre Silicon public license as published by
%%      the Libre Silicon alliance, either version 1 of the License, or
%%      (at your option) any later version.
%%
%%      This design is distributed in the hope that it will be useful,
%%      but WITHOUT ANY WARRANTY; without even the implied warranty of
%%      MERCHANTABILITY or FITNESS FOR A PARTICULAR PURPOSE.
%%      See the Libre Silicon Public License for more details.
%%
%%  ///////////////////////////////////////////////////////////////////
\begin{circuitdiagram}[draft]{38}{20}

    \usgate
    % ----  1st column  ----
    \pin{1}{1}{L}{A}
    \pin{1}{5}{L}{A1}
    \gate[\inputs{2}]{or}{5}{3}{R}{}{}

    % ----  2nd column  ----
    \wire{9}{3}{9}{5}
    \pin{8}{7}{L}{B}
    \pin{8}{9}{L}{B1}
    \gate[\inputs{3}]{and}{12}{7}{R}{}{}

    % ----  3rd column  ----
    \wire{16}{7}{16}{9}
    \pin{15}{11}{L}{C}
    \pin{15}{13}{L}{C1}
    \gate[\inputs{3}]{or}{19}{11}{R}{}{}

    \pin{15}{15}{L}{D}
    \pin{15}{19}{L}{D1}
    \gate[\inputs{2}]{or}{19}{17}{R}{}{}

    % ----  4th column  ----
    \wire{23}{11}{23}{13}
    \gate[\inputs{2}]{nand}{26}{15}{R}{}{}

    % ----  last column ----
    \gate{not}{33}{15}{R}{}{}

    % ----  result ----
    \pin{37}{15}{R}{Z}

\end{circuitdiagram}

\include{Datasheets/OAOOAI2223} \include{Datasheets/OAOOA2223}
%%  ************    LibreSilicon's StdCellLibrary   *******************
%%
%%  Organisation:   Chipforge
%%                  Germany / European Union
%%
%%  Profile:        Chipforge focus on fine System-on-Chip Cores in
%%                  Verilog HDL Code which are easy understandable and
%%                  adjustable. For further information see
%%                          www.chipforge.org
%%                  there are projects from small cores up to PCBs, too.
%%
%%  File:           StdCellLib/Documents/Datasheets/Circuitry/OAOOAI3111.tex
%%
%%  Purpose:        Circuit File for OAOOAI3111
%%
%%  ************    LaTeX with circdia.sty package      ***************
%%
%%  ///////////////////////////////////////////////////////////////////
%%
%%  Copyright (c) 2018 - 2022 by
%%                  chipforge <stdcelllib@nospam.chipforge.org>
%%  All rights reserved.
%%
%%      This Standard Cell Library is licensed under the Libre Silicon
%%      public license; you can redistribute it and/or modify it under
%%      the terms of the Libre Silicon public license as published by
%%      the Libre Silicon alliance, either version 1 of the License, or
%%      (at your option) any later version.
%%
%%      This design is distributed in the hope that it will be useful,
%%      but WITHOUT ANY WARRANTY; without even the implied warranty of
%%      MERCHANTABILITY or FITNESS FOR A PARTICULAR PURPOSE.
%%      See the Libre Silicon Public License for more details.
%%
%%  ///////////////////////////////////////////////////////////////////
\begin{circuitdiagram}[draft]{32}{16}

    \usgate
    % ----  1st column  ----
    \pin{1}{1}{L}{A}
    \pin{1}{3}{L}{A1}
    \pin{1}{5}{L}{A2}
    \gate[\inputs{3}]{or}{5}{3}{R}{}{}

    % ----  2nd column  ----
    \pin{8}{7}{L}{B}
    \gate[\inputs{2}]{and}{12}{5}{R}{}{}

    % ----  3rd column  ----
    \pin{15}{9}{L}{C}
    \gate[\inputs{2}]{or}{19}{7}{R}{}{}

    \pin{15}{15}{L}{D1}
    \pin{15}{11}{L}{D}
    \gate[\inputs{2}]{or}{19}{13}{R}{}{}

    % ----  4th column  ----
    \wire{23}{7}{23}{9}
    \gate[\inputs{2}]{nand}{26}{11}{R}{}{}


    % ----  result ----
    \pin{31}{11}{R}{Y}

\end{circuitdiagram}
 %%  ************    LibreSilicon's StdCellLibrary   *******************
%%
%%  Organisation:   Chipforge
%%                  Germany / European Union
%%
%%  Profile:        Chipforge focus on fine System-on-Chip Cores in
%%                  Verilog HDL Code which are easy understandable and
%%                  adjustable. For further information see
%%                          www.chipforge.org
%%                  there are projects from small cores up to PCBs, too.
%%
%%  File:           StdCellLib/Documents/Datasheets/Circuitry/OAOOA3111.tex
%%
%%  Purpose:        Circuit File for OAOOA3111
%%
%%  ************    LaTeX with circdia.sty package      ***************
%%
%%  ///////////////////////////////////////////////////////////////////
%%
%%  Copyright (c) 2018 - 2022 by
%%                  chipforge <stdcelllib@nospam.chipforge.org>
%%  All rights reserved.
%%
%%      This Standard Cell Library is licensed under the Libre Silicon
%%      public license; you can redistribute it and/or modify it under
%%      the terms of the Libre Silicon public license as published by
%%      the Libre Silicon alliance, either version 1 of the License, or
%%      (at your option) any later version.
%%
%%      This design is distributed in the hope that it will be useful,
%%      but WITHOUT ANY WARRANTY; without even the implied warranty of
%%      MERCHANTABILITY or FITNESS FOR A PARTICULAR PURPOSE.
%%      See the Libre Silicon Public License for more details.
%%
%%  ///////////////////////////////////////////////////////////////////
\begin{circuitdiagram}[draft]{38}{16}

    \usgate
    % ----  1st column  ----
    \pin{1}{1}{L}{A}
    \pin{1}{3}{L}{A1}
    \pin{1}{5}{L}{A2}
    \gate[\inputs{3}]{or}{5}{3}{R}{}{}

    % ----  2nd column  ----
    \pin{8}{7}{L}{B}
    \gate[\inputs{2}]{and}{12}{5}{R}{}{}

    % ----  3rd column  ----
    \pin{15}{9}{L}{C}
    \gate[\inputs{2}]{or}{19}{7}{R}{}{}

    \pin{15}{15}{L}{D1}
    \pin{15}{11}{L}{D}
    \gate[\inputs{2}]{or}{19}{13}{R}{}{}

    % ----  4th column  ----
    \wire{23}{7}{23}{9}
    \gate[\inputs{2}]{nand}{26}{11}{R}{}{}

    % ----  5th column  ----
    \gate{not}{33}{11}{R}{}{}

    % ----  result ----
    \pin{37}{11}{R}{Z}

\end{circuitdiagram}

%%  ************    LibreSilicon's StdCellLibrary   *******************
%%
%%  Organisation:   Chipforge
%%                  Germany / European Union
%%
%%  Profile:        Chipforge focus on fine System-on-Chip Cores in
%%                  Verilog HDL Code which are easy understandable and
%%                  adjustable. For further information see
%%                          www.chipforge.org
%%                  there are projects from small cores up to PCBs, too.
%%
%%  File:           StdCellLib/Documents/Datasheets/Circuitry/OAOOAI3113.tex
%%
%%  Purpose:        Circuit File for OAOOAI3113
%%
%%  ************    LaTeX with circdia.sty package      ***************
%%
%%  ///////////////////////////////////////////////////////////////////
%%
%%  Copyright (c) 2018 - 2022 by
%%                  chipforge <stdcelllib@nospam.chipforge.org>
%%  All rights reserved.
%%
%%      This Standard Cell Library is licensed under the Libre Silicon
%%      public license; you can redistribute it and/or modify it under
%%      the terms of the Libre Silicon public license as published by
%%      the Libre Silicon alliance, either version 1 of the License, or
%%      (at your option) any later version.
%%
%%      This design is distributed in the hope that it will be useful,
%%      but WITHOUT ANY WARRANTY; without even the implied warranty of
%%      MERCHANTABILITY or FITNESS FOR A PARTICULAR PURPOSE.
%%      See the Libre Silicon Public License for more details.
%%
%%  ///////////////////////////////////////////////////////////////////
\begin{circuitdiagram}[draft]{32}{16}

    \usgate
    % ----  1st column  ----
    \pin{1}{1}{L}{A}
    \pin{1}{3}{L}{A1}
    \pin{1}{5}{L}{A2}
    \gate[\inputs{3}]{or}{5}{3}{R}{}{}

    % ----  2nd column  ----
    \pin{8}{7}{L}{B}
    \gate[\inputs{2}]{and}{12}{5}{R}{}{}

    % ----  3rd column  ----
    \pin{15}{9}{L}{C}
    \gate[\inputs{2}]{or}{19}{7}{R}{}{}

    \pin{15}{11}{L}{D}
    \pin{15}{13}{L}{D1}
    \pin{15}{15}{L}{D2}
    \gate[\inputs{3}]{or}{19}{13}{R}{}{}

    % ----  4th column  ----
    \wire{23}{7}{23}{9}
    \gate[\inputs{2}]{nand}{26}{11}{R}{}{}


    % ----  result ----
    \pin{31}{11}{R}{Y}

\end{circuitdiagram}
 %%  ************    LibreSilicon's StdCellLibrary   *******************
%%
%%  Organisation:   Chipforge
%%                  Germany / European Union
%%
%%  Profile:        Chipforge focus on fine System-on-Chip Cores in
%%                  Verilog HDL Code which are easy understandable and
%%                  adjustable. For further information see
%%                          www.chipforge.org
%%                  there are projects from small cores up to PCBs, too.
%%
%%  File:           StdCellLib/Documents/Datasheets/Circuitry/OAOOA3113.tex
%%
%%  Purpose:        Circuit File for OAOOA3113
%%
%%  ************    LaTeX with circdia.sty package      ***************
%%
%%  ///////////////////////////////////////////////////////////////////
%%
%%  Copyright (c) 2018 - 2022 by
%%                  chipforge <stdcelllib@nospam.chipforge.org>
%%  All rights reserved.
%%
%%      This Standard Cell Library is licensed under the Libre Silicon
%%      public license; you can redistribute it and/or modify it under
%%      the terms of the Libre Silicon public license as published by
%%      the Libre Silicon alliance, either version 1 of the License, or
%%      (at your option) any later version.
%%
%%      This design is distributed in the hope that it will be useful,
%%      but WITHOUT ANY WARRANTY; without even the implied warranty of
%%      MERCHANTABILITY or FITNESS FOR A PARTICULAR PURPOSE.
%%      See the Libre Silicon Public License for more details.
%%
%%  ///////////////////////////////////////////////////////////////////
\begin{circuitdiagram}[draft]{38}{16}

    \usgate
    % ----  1st column  ----
    \pin{1}{1}{L}{A}
    \pin{1}{3}{L}{A1}
    \pin{1}{5}{L}{A2}
    \gate[\inputs{3}]{or}{5}{3}{R}{}{}

    % ----  2nd column  ----
    \pin{8}{7}{L}{B}
    \gate[\inputs{2}]{and}{12}{5}{R}{}{}

    % ----  3rd column  ----
    \pin{15}{9}{L}{C}
    \gate[\inputs{2}]{or}{19}{7}{R}{}{}

    \pin{15}{11}{L}{D}
    \pin{15}{13}{L}{D1}
    \pin{15}{15}{L}{D2}
    \gate[\inputs{3}]{or}{19}{13}{R}{}{}

    % ----  4th column  ----
    \wire{23}{7}{23}{9}
    \gate[\inputs{2}]{nand}{26}{11}{R}{}{}

    % ----  last column ----
    \gate{not}{33}{11}{R}{}{}

    % ----  result ----
    \pin{37}{11}{R}{Z}

\end{circuitdiagram}

%%  ************    LibreSilicon's StdCellLibrary   *******************
%%
%%  Organisation:   Chipforge
%%                  Germany / European Union
%%
%%  Profile:        Chipforge focus on fine System-on-Chip Cores in
%%                  Verilog HDL Code which are easy understandable and
%%                  adjustable. For further information see
%%                          www.chipforge.org
%%                  there are projects from small cores up to PCBs, too.
%%
%%  File:           StdCellLib/Documents/Datasheets/Circuitry/OAOOAI3212.tex
%%
%%  Purpose:        Circuit File for OAOOAI3212
%%
%%  ************    LaTeX with circdia.sty package      ***************
%%
%%  ///////////////////////////////////////////////////////////////////
%%
%%  Copyright (c) 2018 - 2022 by
%%                  chipforge <stdcelllib@nospam.chipforge.org>
%%  All rights reserved.
%%
%%      This Standard Cell Library is licensed under the Libre Silicon
%%      public license; you can redistribute it and/or modify it under
%%      the terms of the Libre Silicon public license as published by
%%      the Libre Silicon alliance, either version 1 of the License, or
%%      (at your option) any later version.
%%
%%      This design is distributed in the hope that it will be useful,
%%      but WITHOUT ANY WARRANTY; without even the implied warranty of
%%      MERCHANTABILITY or FITNESS FOR A PARTICULAR PURPOSE.
%%      See the Libre Silicon Public License for more details.
%%
%%  ///////////////////////////////////////////////////////////////////
\begin{circuitdiagram}[draft]{32}{18}

    \usgate
    % ----  1st column  ----
    \pin{1}{1}{L}{A}
    \pin{1}{3}{L}{A1}
    \pin{1}{5}{L}{A2}
    \gate[\inputs{3}]{or}{5}{3}{R}{}{}

    % ----  2nd column  ----
    \wire{9}{3}{9}{5}
    \pin{8}{7}{L}{B}
    \pin{8}{9}{L}{B1}
    \gate[\inputs{3}]{and}{12}{7}{R}{}{}

    % ----  3rd column  ----
    \pin{15}{11}{L}{C}
    \gate[\inputs{2}]{or}{19}{9}{R}{}{}

    \pin{15}{17}{L}{D1}
    \pin{15}{13}{L}{D}
    \gate[\inputs{2}]{or}{19}{15}{R}{}{}

    % ----  4th column  ----
    \wire{23}{9}{23}{11}
    \gate[\inputs{2}]{nand}{26}{13}{R}{}{}


    % ----  result ----
    \pin{31}{13}{R}{Y}

\end{circuitdiagram}
 %%  ************    LibreSilicon's StdCellLibrary   *******************
%%
%%  Organisation:   Chipforge
%%                  Germany / European Union
%%
%%  Profile:        Chipforge focus on fine System-on-Chip Cores in
%%                  Verilog HDL Code which are easy understandable and
%%                  adjustable. For further information see
%%                          www.chipforge.org
%%                  there are projects from small cores up to PCBs, too.
%%
%%  File:           StdCellLib/Documents/Datasheets/Circuitry/OAOOA3212.tex
%%
%%  Purpose:        Circuit File for OAOOA3212
%%
%%  ************    LaTeX with circdia.sty package      ***************
%%
%%  ///////////////////////////////////////////////////////////////////
%%
%%  Copyright (c) 2018 - 2022 by
%%                  chipforge <stdcelllib@nospam.chipforge.org>
%%  All rights reserved.
%%
%%      This Standard Cell Library is licensed under the Libre Silicon
%%      public license; you can redistribute it and/or modify it under
%%      the terms of the Libre Silicon public license as published by
%%      the Libre Silicon alliance, either version 1 of the License, or
%%      (at your option) any later version.
%%
%%      This design is distributed in the hope that it will be useful,
%%      but WITHOUT ANY WARRANTY; without even the implied warranty of
%%      MERCHANTABILITY or FITNESS FOR A PARTICULAR PURPOSE.
%%      See the Libre Silicon Public License for more details.
%%
%%  ///////////////////////////////////////////////////////////////////
\begin{circuitdiagram}[draft]{38}{18}

    \usgate
    % ----  1st column  ----
    \pin{1}{1}{L}{A}
    \pin{1}{3}{L}{A1}
    \pin{1}{5}{L}{A2}
    \gate[\inputs{3}]{or}{5}{3}{R}{}{}

    % ----  2nd column  ----
    \wire{9}{3}{9}{5}
    \pin{8}{7}{L}{B}
    \pin{8}{9}{L}{B1}
    \gate[\inputs{3}]{and}{12}{7}{R}{}{}

    % ----  3rd column  ----
    \pin{15}{11}{L}{C}
    \gate[\inputs{2}]{or}{19}{9}{R}{}{}

    \pin{15}{17}{L}{D1}
    \pin{15}{13}{L}{D}
    \gate[\inputs{2}]{or}{19}{15}{R}{}{}

    % ----  4th column  ----
    \wire{23}{9}{23}{11}
    \gate[\inputs{2}]{nand}{26}{13}{R}{}{}

    % ----  last column ----
    \gate{not}{33}{13}{R}{}{}

    % ----  result ----
    \pin{37}{13}{R}{Z}

\end{circuitdiagram}


%%  ************    LibreSilicon's StdCellLibrary   *******************
%%
%%  Organisation:   Chipforge
%%                  Germany / European Union
%%
%%  Profile:        Chipforge focus on fine System-on-Chip Cores in
%%                  Verilog HDL Code which are easy understandable and
%%                  adjustable. For further information see
%%                          www.chipforge.org
%%                  there are projects from small cores up to PCBs, too.
%%
%%  File:           StdCellLib/Documents/Datasheets/Circuitry/OAOOAI21121.tex
%%
%%  Purpose:        Circuit File for OAOOAI21121
%%
%%  ************    LaTeX with circdia.sty package      ***************
%%
%%  ///////////////////////////////////////////////////////////////////
%%
%%  Copyright (c) 2018 - 2022 by
%%                  chipforge <stdcelllib@nospam.chipforge.org>
%%  All rights reserved.
%%
%%      This Standard Cell Library is licensed under the Libre Silicon
%%      public license; you can redistribute it and/or modify it under
%%      the terms of the Libre Silicon public license as published by
%%      the Libre Silicon alliance, either version 1 of the License, or
%%      (at your option) any later version.
%%
%%      This design is distributed in the hope that it will be useful,
%%      but WITHOUT ANY WARRANTY; without even the implied warranty of
%%      MERCHANTABILITY or FITNESS FOR A PARTICULAR PURPOSE.
%%      See the Libre Silicon Public License for more details.
%%
%%  ///////////////////////////////////////////////////////////////////
\begin{circuitdiagram}[draft]{32}{18}

    \usgate
    % ----  1st column  ----
    \pin{1}{1}{L}{A}
    \pin{1}{5}{L}{A1}
    \gate[\inputs{2}]{or}{5}{3}{R}{}{}

    % ----  2nd column  ----
    \pin{8}{7}{L}{B}
    \gate[\inputs{2}]{and}{12}{5}{R}{}{}

    % ----  3rd column  ----
    \pin{15}{9}{L}{C}
    \gate[\inputs{2}]{or}{19}{7}{R}{}{}

    \pin{15}{15}{L}{D1}
    \pin{15}{11}{L}{D}
    \gate[\inputs{2}]{or}{19}{13}{R}{}{}

    % ----  4th column  ----
    \wire{23}{7}{23}{11}
    \pin{22}{17}{L}{E}
    \wire{23}{15}{23}{17}
    \gate[\inputs{3}]{nand}{26}{13}{R}{}{}


    % ----  result ----
    \pin{31}{13}{R}{Y}

\end{circuitdiagram}
 %%  ************    LibreSilicon's StdCellLibrary   *******************
%%
%%  Organisation:   Chipforge
%%                  Germany / European Union
%%
%%  Profile:        Chipforge focus on fine System-on-Chip Cores in
%%                  Verilog HDL Code which are easy understandable and
%%                  adjustable. For further information see
%%                          www.chipforge.org
%%                  there are projects from small cores up to PCBs, too.
%%
%%  File:           StdCellLib/Documents/Datasheets/Circuitry/OAOOA21121.tex
%%
%%  Purpose:        Circuit File for OAOOA21121
%%
%%  ************    LaTeX with circdia.sty package      ***************
%%
%%  ///////////////////////////////////////////////////////////////////
%%
%%  Copyright (c) 2018 - 2022 by
%%                  chipforge <stdcelllib@nospam.chipforge.org>
%%  All rights reserved.
%%
%%      This Standard Cell Library is licensed under the Libre Silicon
%%      public license; you can redistribute it and/or modify it under
%%      the terms of the Libre Silicon public license as published by
%%      the Libre Silicon alliance, either version 1 of the License, or
%%      (at your option) any later version.
%%
%%      This design is distributed in the hope that it will be useful,
%%      but WITHOUT ANY WARRANTY; without even the implied warranty of
%%      MERCHANTABILITY or FITNESS FOR A PARTICULAR PURPOSE.
%%      See the Libre Silicon Public License for more details.
%%
%%  ///////////////////////////////////////////////////////////////////
\begin{circuitdiagram}[draft]{38}{18}

    \usgate
    % ----  1st column  ----
    \pin{1}{1}{L}{A}
    \pin{1}{5}{L}{A1}
    \gate[\inputs{2}]{or}{5}{3}{R}{}{}

    % ----  2nd column  ----
    \pin{8}{7}{L}{B}
    \gate[\inputs{2}]{and}{12}{5}{R}{}{}

    % ----  3rd column  ----
    \pin{15}{9}{L}{C}
    \gate[\inputs{2}]{or}{19}{7}{R}{}{}

    \pin{15}{15}{L}{D1}
    \pin{15}{11}{L}{D}
    \gate[\inputs{2}]{or}{19}{13}{R}{}{}

    % ----  4th column  ----
    \wire{23}{7}{23}{11}
    \pin{22}{17}{L}{E}
    \wire{23}{15}{23}{17}
    \gate[\inputs{3}]{nand}{26}{13}{R}{}{}

    % ----  5th column  ----
    \gate{not}{33}{13}{R}{}{}

    % ----  result ----
    \pin{37}{13}{R}{Z}

\end{circuitdiagram}

%%  ************    LibreSilicon's StdCellLibrary   *******************
%%
%%  Organisation:   Chipforge
%%                  Germany / European Union
%%
%%  Profile:        Chipforge focus on fine System-on-Chip Cores in
%%                  Verilog HDL Code which are easy understandable and
%%                  adjustable. For further information see
%%                          www.chipforge.org
%%                  there are projects from small cores up to PCBs, too.
%%
%%  File:           StdCellLib/Documents/Datasheets/Circuitry/OAOOAI21131.tex
%%
%%  Purpose:        Circuit File for OAOOAI21131
%%
%%  ************    LaTeX with circdia.sty package      ***************
%%
%%  ///////////////////////////////////////////////////////////////////
%%
%%  Copyright (c) 2018 - 2022 by
%%                  chipforge <stdcelllib@nospam.chipforge.org>
%%  All rights reserved.
%%
%%      This Standard Cell Library is licensed under the Libre Silicon
%%      public license; you can redistribute it and/or modify it under
%%      the terms of the Libre Silicon public license as published by
%%      the Libre Silicon alliance, either version 1 of the License, or
%%      (at your option) any later version.
%%
%%      This design is distributed in the hope that it will be useful,
%%      but WITHOUT ANY WARRANTY; without even the implied warranty of
%%      MERCHANTABILITY or FITNESS FOR A PARTICULAR PURPOSE.
%%      See the Libre Silicon Public License for more details.
%%
%%  ///////////////////////////////////////////////////////////////////
\begin{circuitdiagram}[draft]{32}{18}

    \usgate
    % ----  1st column  ----
    \pin{1}{1}{L}{A}
    \pin{1}{5}{L}{A1}
    \gate[\inputs{2}]{or}{5}{3}{R}{}{}

    % ----  2nd column  ----
    \pin{8}{7}{L}{B}
    \gate[\inputs{2}]{and}{12}{5}{R}{}{}

    % ----  3rd column  ----
    \pin{15}{9}{L}{C}
    \gate[\inputs{2}]{or}{19}{7}{R}{}{}

    \pin{15}{15}{L}{D2}
    \pin{15}{13}{L}{D1}
    \pin{15}{11}{L}{D}
    \gate[\inputs{3}]{or}{19}{13}{R}{}{}

    % ----  4th column  ----
    \wire{23}{7}{23}{11}
    \pin{22}{17}{L}{E}
    \wire{23}{15}{23}{17}
    \gate[\inputs{3}]{nand}{26}{13}{R}{}{}


    % ----  result ----
    \pin{31}{13}{R}{Y}

\end{circuitdiagram}
 %%  ************    LibreSilicon's StdCellLibrary   *******************
%%
%%  Organisation:   Chipforge
%%                  Germany / European Union
%%
%%  Profile:        Chipforge focus on fine System-on-Chip Cores in
%%                  Verilog HDL Code which are easy understandable and
%%                  adjustable. For further information see
%%                          www.chipforge.org
%%                  there are projects from small cores up to PCBs, too.
%%
%%  File:           StdCellLib/Documents/Datasheets/Circuitry/OAOOA21131.tex
%%
%%  Purpose:        Circuit File for OAOOA21131
%%
%%  ************    LaTeX with circdia.sty package      ***************
%%
%%  ///////////////////////////////////////////////////////////////////
%%
%%  Copyright (c) 2018 - 2022 by
%%                  chipforge <stdcelllib@nospam.chipforge.org>
%%  All rights reserved.
%%
%%      This Standard Cell Library is licensed under the Libre Silicon
%%      public license; you can redistribute it and/or modify it under
%%      the terms of the Libre Silicon public license as published by
%%      the Libre Silicon alliance, either version 1 of the License, or
%%      (at your option) any later version.
%%
%%      This design is distributed in the hope that it will be useful,
%%      but WITHOUT ANY WARRANTY; without even the implied warranty of
%%      MERCHANTABILITY or FITNESS FOR A PARTICULAR PURPOSE.
%%      See the Libre Silicon Public License for more details.
%%
%%  ///////////////////////////////////////////////////////////////////
\begin{circuitdiagram}[draft]{38}{18}

    \usgate
    % ----  1st column  ----
    \pin{1}{1}{L}{A}
    \pin{1}{5}{L}{A1}
    \gate[\inputs{2}]{or}{5}{3}{R}{}{}

    % ----  2nd column  ----
    \pin{8}{7}{L}{B}
    \gate[\inputs{2}]{and}{12}{5}{R}{}{}

    % ----  3rd column  ----
    \pin{15}{9}{L}{C}
    \gate[\inputs{2}]{or}{19}{7}{R}{}{}

    \pin{15}{15}{L}{D2}
    \pin{15}{13}{L}{D1}
    \pin{15}{11}{L}{D}
    \gate[\inputs{3}]{or}{19}{13}{R}{}{}

    % ----  4th column  ----
    \wire{23}{7}{23}{11}
    \pin{22}{17}{L}{E}
    \wire{23}{15}{23}{17}
    \gate[\inputs{3}]{nand}{26}{13}{R}{}{}

    % ----  5th column  ----
    \gate{not}{33}{13}{R}{}{}

    % ----  result ----
    \pin{37}{13}{R}{Z}

\end{circuitdiagram}

%%  ************    LibreSilicon's StdCellLibrary   *******************
%%
%%  Organisation:   Chipforge
%%                  Germany / European Union
%%
%%  Profile:        Chipforge focus on fine System-on-Chip Cores in
%%                  Verilog HDL Code which are easy understandable and
%%                  adjustable. For further information see
%%                          www.chipforge.org
%%                  there are projects from small cores up to PCBs, too.
%%
%%  File:           StdCellLib/Documents/Datasheets/Circuitry/OAOOAI21141.tex
%%
%%  Purpose:        Circuit File for OAOOAI21141
%%
%%  ************    LaTeX with circdia.sty package      ***************
%%
%%  ///////////////////////////////////////////////////////////////////
%%
%%  Copyright (c) 2018 - 2022 by
%%                  chipforge <stdcelllib@nospam.chipforge.org>
%%  All rights reserved.
%%
%%      This Standard Cell Library is licensed under the Libre Silicon
%%      public license; you can redistribute it and/or modify it under
%%      the terms of the Libre Silicon public license as published by
%%      the Libre Silicon alliance, either version 1 of the License, or
%%      (at your option) any later version.
%%
%%      This design is distributed in the hope that it will be useful,
%%      but WITHOUT ANY WARRANTY; without even the implied warranty of
%%      MERCHANTABILITY or FITNESS FOR A PARTICULAR PURPOSE.
%%      See the Libre Silicon Public License for more details.
%%
%%  ///////////////////////////////////////////////////////////////////
\begin{circuitdiagram}[draft]{32}{19}

    \usgate
    % ----  1st column  ----
    \pin{1}{1}{L}{A}
    \pin{1}{5}{L}{A1}
    \gate[\inputs{2}]{or}{5}{3}{R}{}{}

    % ----  2nd column  ----
    \pin{8}{7}{L}{B}
    \gate[\inputs{2}]{and}{12}{5}{R}{}{}

    % ----  3rd column  ----
    \pin{15}{9}{L}{C}
    \gate[\inputs{2}]{or}{19}{7}{R}{}{}

    \pin{15}{11}{L}{D}
    \pin{15}{13}{L}{D1}
    \pin{15}{15}{L}{D2}
    \pin{15}{17}{L}{D3}
    \gate[\inputs{4}]{or}{19}{14}{R}{}{}

    % ----  4th column  ----
    \wire{23}{7}{23}{12}
    \pin{22}{18}{L}{E}
    \wire{23}{16}{23}{18}
    \gate[\inputs{3}]{nand}{26}{14}{R}{}{}


    % ----  result ----
    \pin{31}{14}{R}{Y}

\end{circuitdiagram}
 %%  ************    LibreSilicon's StdCellLibrary   *******************
%%
%%  Organisation:   Chipforge
%%                  Germany / European Union
%%
%%  Profile:        Chipforge focus on fine System-on-Chip Cores in
%%                  Verilog HDL Code which are easy understandable and
%%                  adjustable. For further information see
%%                          www.chipforge.org
%%                  there are projects from small cores up to PCBs, too.
%%
%%  File:           StdCellLib/Documents/Datasheets/Circuitry/OAOOA21141.tex
%%
%%  Purpose:        Circuit File for OAOOA21141
%%
%%  ************    LaTeX with circdia.sty package      ***************
%%
%%  ///////////////////////////////////////////////////////////////////
%%
%%  Copyright (c) 2018 - 2022 by
%%                  chipforge <stdcelllib@nospam.chipforge.org>
%%  All rights reserved.
%%
%%      This Standard Cell Library is licensed under the Libre Silicon
%%      public license; you can redistribute it and/or modify it under
%%      the terms of the Libre Silicon public license as published by
%%      the Libre Silicon alliance, either version 1 of the License, or
%%      (at your option) any later version.
%%
%%      This design is distributed in the hope that it will be useful,
%%      but WITHOUT ANY WARRANTY; without even the implied warranty of
%%      MERCHANTABILITY or FITNESS FOR A PARTICULAR PURPOSE.
%%      See the Libre Silicon Public License for more details.
%%
%%  ///////////////////////////////////////////////////////////////////
\begin{circuitdiagram}[draft]{38}{19}

    \usgate
    % ----  1st column  ----
    \pin{1}{1}{L}{A}
    \pin{1}{5}{L}{A1}
    \gate[\inputs{2}]{or}{5}{3}{R}{}{}

    % ----  2nd column  ----
    \pin{8}{7}{L}{B}
    \gate[\inputs{2}]{and}{12}{5}{R}{}{}

    % ----  3rd column  ----
    \pin{15}{9}{L}{C}
    \gate[\inputs{2}]{or}{19}{7}{R}{}{}

    \pin{15}{11}{L}{D}
    \pin{15}{13}{L}{D1}
    \pin{15}{15}{L}{D2}
    \pin{15}{17}{L}{D3}
    \gate[\inputs{4}]{or}{19}{14}{R}{}{}

    % ----  4th column  ----
    \wire{23}{7}{23}{12}
    \pin{22}{18}{L}{E}
    \wire{23}{16}{23}{18}
    \gate[\inputs{3}]{nand}{26}{14}{R}{}{}

    % ----  last column ----
    \gate{not}{33}{14}{R}{}{}

    % ----  result ----
    \pin{37}{14}{R}{Z}

\end{circuitdiagram}

%%  ************    LibreSilicon's StdCellLibrary   *******************
%%
%%  Organisation:   Chipforge
%%                  Germany / European Union
%%
%%  Profile:        Chipforge focus on fine System-on-Chip Cores in
%%                  Verilog HDL Code which are easy understandable and
%%                  adjustable. For further information see
%%                          www.chipforge.org
%%                  there are projects from small cores up to PCBs, too.
%%
%%  File:           StdCellLib/Documents/Datasheets/Circuitry/OAOOAI21221.tex
%%
%%  Purpose:        Circuit File for OAOOAI21221
%%
%%  ************    LaTeX with circdia.sty package      ***************
%%
%%  ///////////////////////////////////////////////////////////////////
%%
%%  Copyright (c) 2018 - 2022 by
%%                  chipforge <stdcelllib@nospam.chipforge.org>
%%  All rights reserved.
%%
%%      This Standard Cell Library is licensed under the Libre Silicon
%%      public license; you can redistribute it and/or modify it under
%%      the terms of the Libre Silicon public license as published by
%%      the Libre Silicon alliance, either version 1 of the License, or
%%      (at your option) any later version.
%%
%%      This design is distributed in the hope that it will be useful,
%%      but WITHOUT ANY WARRANTY; without even the implied warranty of
%%      MERCHANTABILITY or FITNESS FOR A PARTICULAR PURPOSE.
%%      See the Libre Silicon Public License for more details.
%%
%%  ///////////////////////////////////////////////////////////////////
\begin{circuitdiagram}[draft]{32}{20}

    \usgate
    % ----  1st column  ----
    \pin{1}{1}{L}{A}
    \pin{1}{5}{L}{A1}
    \gate[\inputs{2}]{or}{5}{3}{R}{}{}

    % ----  2nd column  ----
    \pin{8}{7}{L}{B}
    \gate[\inputs{2}]{and}{12}{5}{R}{}{}

    % ----  3rd column  ----
    \wire{16}{5}{16}{7}
    \pin{15}{9}{L}{C}
    \pin{15}{11}{L}{C1}
    \gate[\inputs{3}]{or}{19}{9}{R}{}{}

    \pin{15}{13}{L}{D}
    \pin{15}{17}{L}{D1}
    \gate[\inputs{2}]{or}{19}{15}{R}{}{}

    % ----  4th column  ----
    \wire{23}{9}{23}{13}
    \pin{22}{19}{L}{E}
    \wire{23}{17}{23}{19}
    \gate[\inputs{3}]{nand}{26}{15}{R}{}{}


    % ----  result ----
    \pin{31}{15}{R}{Y}

\end{circuitdiagram}
 %%  ************    LibreSilicon's StdCellLibrary   *******************
%%
%%  Organisation:   Chipforge
%%                  Germany / European Union
%%
%%  Profile:        Chipforge focus on fine System-on-Chip Cores in
%%                  Verilog HDL Code which are easy understandable and
%%                  adjustable. For further information see
%%                          www.chipforge.org
%%                  there are projects from small cores up to PCBs, too.
%%
%%  File:           StdCellLib/Documents/Datasheets/Circuitry/OAOOA21221.tex
%%
%%  Purpose:        Circuit File for OAOOA21221
%%
%%  ************    LaTeX with circdia.sty package      ***************
%%
%%  ///////////////////////////////////////////////////////////////////
%%
%%  Copyright (c) 2018 - 2022 by
%%                  chipforge <stdcelllib@nospam.chipforge.org>
%%  All rights reserved.
%%
%%      This Standard Cell Library is licensed under the Libre Silicon
%%      public license; you can redistribute it and/or modify it under
%%      the terms of the Libre Silicon public license as published by
%%      the Libre Silicon alliance, either version 1 of the License, or
%%      (at your option) any later version.
%%
%%      This design is distributed in the hope that it will be useful,
%%      but WITHOUT ANY WARRANTY; without even the implied warranty of
%%      MERCHANTABILITY or FITNESS FOR A PARTICULAR PURPOSE.
%%      See the Libre Silicon Public License for more details.
%%
%%  ///////////////////////////////////////////////////////////////////
\begin{circuitdiagram}[draft]{38}{20}

    \usgate
    % ----  1st column  ----
    \pin{1}{1}{L}{A}
    \pin{1}{5}{L}{A1}
    \gate[\inputs{2}]{or}{5}{3}{R}{}{}

    % ----  2nd column  ----
    \pin{8}{7}{L}{B}
    \gate[\inputs{2}]{and}{12}{5}{R}{}{}

    % ----  3rd column  ----
    \wire{16}{5}{16}{7}
    \pin{15}{9}{L}{C}
    \pin{15}{11}{L}{C1}
    \gate[\inputs{3}]{or}{19}{9}{R}{}{}

    \pin{15}{13}{L}{D}
    \pin{15}{17}{L}{D1}
    \gate[\inputs{2}]{or}{19}{15}{R}{}{}

    % ----  4th column  ----
    \wire{23}{9}{23}{13}
    \pin{22}{19}{L}{E}
    \wire{23}{17}{23}{19}
    \gate[\inputs{3}]{nand}{26}{15}{R}{}{}

    % ----  last column ----
    \gate{not}{33}{15}{R}{}{}

    % ----  result ----
    \pin{37}{15}{R}{Z}

\end{circuitdiagram}

\include{Datasheets/OAOOAI21231} \include{Datasheets/OAOOA21231}
%%  ************    LibreSilicon's StdCellLibrary   *******************
%%
%%  Organisation:   Chipforge
%%                  Germany / European Union
%%
%%  Profile:        Chipforge focus on fine System-on-Chip Cores in
%%                  Verilog HDL Code which are easy understandable and
%%                  adjustable. For further information see
%%                          www.chipforge.org
%%                  there are projects from small cores up to PCBs, too.
%%
%%  File:           StdCellLib/Documents/Datasheets/Circuitry/OAOOAI31121.tex
%%
%%  Purpose:        Circuit File for OAOOAI31121
%%
%%  ************    LaTeX with circdia.sty package      ***************
%%
%%  ///////////////////////////////////////////////////////////////////
%%
%%  Copyright (c) 2018 - 2022 by
%%                  chipforge <stdcelllib@nospam.chipforge.org>
%%  All rights reserved.
%%
%%      This Standard Cell Library is licensed under the Libre Silicon
%%      public license; you can redistribute it and/or modify it under
%%      the terms of the Libre Silicon public license as published by
%%      the Libre Silicon alliance, either version 1 of the License, or
%%      (at your option) any later version.
%%
%%      This design is distributed in the hope that it will be useful,
%%      but WITHOUT ANY WARRANTY; without even the implied warranty of
%%      MERCHANTABILITY or FITNESS FOR A PARTICULAR PURPOSE.
%%      See the Libre Silicon Public License for more details.
%%
%%  ///////////////////////////////////////////////////////////////////
\begin{circuitdiagram}[draft]{32}{18}

    \usgate
    % ----  1st column  ----
    \pin{1}{1}{L}{A}
    \pin{1}{3}{L}{A1}
    \pin{1}{5}{L}{A2}
    \gate[\inputs{3}]{or}{5}{3}{R}{}{}

    % ----  2nd column  ----
    \pin{8}{7}{L}{B}
    \gate[\inputs{2}]{and}{12}{5}{R}{}{}

    % ----  3rd column  ----
    \pin{15}{9}{L}{C}
    \gate[\inputs{2}]{or}{19}{7}{R}{}{}

    \pin{15}{15}{L}{D1}
    \pin{15}{11}{L}{D}
    \gate[\inputs{2}]{or}{19}{13}{R}{}{}

    % ----  4th column  ----
    \wire{23}{7}{23}{11}
    \pin{22}{17}{L}{E}
    \wire{23}{15}{23}{17}
    \gate[\inputs{3}]{nand}{26}{13}{R}{}{}


    % ----  result ----
    \pin{31}{13}{R}{Y}

\end{circuitdiagram}
 %%  ************    LibreSilicon's StdCellLibrary   *******************
%%
%%  Organisation:   Chipforge
%%                  Germany / European Union
%%
%%  Profile:        Chipforge focus on fine System-on-Chip Cores in
%%                  Verilog HDL Code which are easy understandable and
%%                  adjustable. For further information see
%%                          www.chipforge.org
%%                  there are projects from small cores up to PCBs, too.
%%
%%  File:           StdCellLib/Documents/Datasheets/Circuitry/OAOOA31121.tex
%%
%%  Purpose:        Circuit File for OAOOA31121
%%
%%  ************    LaTeX with circdia.sty package      ***************
%%
%%  ///////////////////////////////////////////////////////////////////
%%
%%  Copyright (c) 2018 - 2022 by
%%                  chipforge <stdcelllib@nospam.chipforge.org>
%%  All rights reserved.
%%
%%      This Standard Cell Library is licensed under the Libre Silicon
%%      public license; you can redistribute it and/or modify it under
%%      the terms of the Libre Silicon public license as published by
%%      the Libre Silicon alliance, either version 1 of the License, or
%%      (at your option) any later version.
%%
%%      This design is distributed in the hope that it will be useful,
%%      but WITHOUT ANY WARRANTY; without even the implied warranty of
%%      MERCHANTABILITY or FITNESS FOR A PARTICULAR PURPOSE.
%%      See the Libre Silicon Public License for more details.
%%
%%  ///////////////////////////////////////////////////////////////////
\begin{circuitdiagram}[draft]{38}{18}

    \usgate
    % ----  1st column  ----
    \pin{1}{1}{L}{A}
    \pin{1}{3}{L}{A1}
    \pin{1}{5}{L}{A2}
    \gate[\inputs{3}]{or}{5}{3}{R}{}{}

    % ----  2nd column  ----
    \pin{8}{7}{L}{B}
    \gate[\inputs{2}]{and}{12}{5}{R}{}{}

    % ----  3rd column  ----
    \pin{15}{9}{L}{C}
    \gate[\inputs{2}]{or}{19}{7}{R}{}{}

    \pin{15}{15}{L}{D1}
    \pin{15}{11}{L}{D}
    \gate[\inputs{2}]{or}{19}{13}{R}{}{}

    % ----  4th column  ----
    \wire{23}{7}{23}{11}
    \pin{22}{17}{L}{E}
    \wire{23}{15}{23}{17}
    \gate[\inputs{3}]{nand}{26}{13}{R}{}{}

    % ----  last column ----
    \gate{not}{33}{13}{R}{}{}

    % ----  result ----
    \pin{37}{13}{R}{Z}

\end{circuitdiagram}


%%  ************    LibreSilicon's StdCellLibrary   *******************
%%
%%  Organisation:   Chipforge
%%                  Germany / European Union
%%
%%  Profile:        Chipforge focus on fine System-on-Chip Cores in
%%                  Verilog HDL Code which are easy understandable and
%%                  adjustable. For further information see
%%                          www.chipforge.org
%%                  there are projects from small cores up to PCBs, too.
%%
%%  File:           StdCellLib/Documents/section-AOAAAOI_complex.tex
%%
%%  Purpose:        Section Level File for Standard Cell Library Documentation
%%
%%  ************    LaTeX with circdia.sty package      ***************
%%
%%  ///////////////////////////////////////////////////////////////////
%%
%%  Copyright (c) 2018 - 2022 by
%%                  chipforge <stdcelllib@nospam.chipforge.org>
%%  All rights reserved.
%%
%%      This Standard Cell Library is licensed under the Libre Silicon
%%      public license; you can redistribute it and/or modify it under
%%      the terms of the Libre Silicon public license as published by
%%      the Libre Silicon alliance, either version 1 of the License, or
%%      (at your option) any later version.
%%
%%      This design is distributed in the hope that it will be useful,
%%      but WITHOUT ANY WARRANTY; without even the implied warranty of
%%      MERCHANTABILITY or FITNESS FOR A PARTICULAR PURPOSE.
%%      See the Libre Silicon Public License for more details.
%%
%%  ///////////////////////////////////////////////////////////////////
\section{AND-OR-AND-AND-AND-OR(-Invert) Complex Gates}

%%  ************    LibreSilicon's StdCellLibrary   *******************
%%
%%  Organisation:   Chipforge
%%                  Germany / European Union
%%
%%  Profile:        Chipforge focus on fine System-on-Chip Cores in
%%                  Verilog HDL Code which are easy understandable and
%%                  adjustable. For further information see
%%                          www.chipforge.org
%%                  there are projects from small cores up to PCBs, too.
%%
%%  File:           StdCellLib/Documents/Datasheets/Circuitry/AOAAAOI21121.tex
%%
%%  Purpose:        Circuit File for AOAAAOI21121
%%
%%  ************    LaTeX with circdia.sty package      ***************
%%
%%  ///////////////////////////////////////////////////////////////////
%%
%%  Copyright (c) 2018 - 2022 by
%%                  chipforge <stdcelllib@nospam.chipforge.org>
%%  All rights reserved.
%%
%%      This Standard Cell Library is licensed under the Libre Silicon
%%      public license; you can redistribute it and/or modify it under
%%      the terms of the Libre Silicon public license as published by
%%      the Libre Silicon alliance, either version 1 of the License, or
%%      (at your option) any later version.
%%
%%      This design is distributed in the hope that it will be useful,
%%      but WITHOUT ANY WARRANTY; without even the implied warranty of
%%      MERCHANTABILITY or FITNESS FOR A PARTICULAR PURPOSE.
%%      See the Libre Silicon Public License for more details.
%%
%%  ///////////////////////////////////////////////////////////////////
\begin{circuitdiagram}[draft]{32}{22}

    \usgate
    % ----  1st column  ----
    \pin{1}{1}{L}{A}
    \pin{1}{5}{L}{A1}
    \gate[\inputs{2}]{and}{5}{3}{R}{}{}

    % ----  2nd column  ----
    \pin{8}{7}{L}{B}
    \gate[\inputs{2}]{or}{12}{5}{R}{}{}

    % ----  3rd column  ----
    \pin{15}{9}{L}{C}
    \gate[\inputs{2}]{and}{19}{7}{R}{}{}

    \pin{15}{15}{L}{D1}
    \pin{15}{11}{L}{D}
    \gate[\inputs{2}]{and}{19}{13}{R}{}{}

    \pin{15}{21}{L}{E1}
    \pin{15}{17}{L}{E}
    \gate[\inputs{2}]{and}{19}{19}{R}{}{}

    % ----  4th column  ----
    \wire{23}{7}{23}{11}
    \wire{23}{15}{23}{19}
    \gate[\inputs{3}]{nor}{26}{13}{R}{}{}


    % ----  result ----
    \pin{31}{13}{R}{Y}

\end{circuitdiagram}
 %%  ************    LibreSilicon's StdCellLibrary   *******************
%%
%%  Organisation:   Chipforge
%%                  Germany / European Union
%%
%%  Profile:        Chipforge focus on fine System-on-Chip Cores in
%%                  Verilog HDL Code which are easy understandable and
%%                  adjustable. For further information see
%%                          www.chipforge.org
%%                  there are projects from small cores up to PCBs, too.
%%
%%  File:           StdCellLib/Documents/Datasheets/Circuitry/AOAAAO21121.tex
%%
%%  Purpose:        Circuit File for AOAAAO21121
%%
%%  ************    LaTeX with circdia.sty package      ***************
%%
%%  ///////////////////////////////////////////////////////////////////
%%
%%  Copyright (c) 2018 - 2022 by
%%                  chipforge <stdcelllib@nospam.chipforge.org>
%%  All rights reserved.
%%
%%      This Standard Cell Library is licensed under the Libre Silicon
%%      public license; you can redistribute it and/or modify it under
%%      the terms of the Libre Silicon public license as published by
%%      the Libre Silicon alliance, either version 1 of the License, or
%%      (at your option) any later version.
%%
%%      This design is distributed in the hope that it will be useful,
%%      but WITHOUT ANY WARRANTY; without even the implied warranty of
%%      MERCHANTABILITY or FITNESS FOR A PARTICULAR PURPOSE.
%%      See the Libre Silicon Public License for more details.
%%
%%  ///////////////////////////////////////////////////////////////////
\begin{circuitdiagram}[draft]{38}{22}

    \usgate
    % ----  1st column  ----
    \pin{1}{1}{L}{A}
    \pin{1}{5}{L}{A1}
    \gate[\inputs{2}]{and}{5}{3}{R}{}{}

    % ----  2nd column  ----
    \pin{8}{7}{L}{B}
    \gate[\inputs{2}]{or}{12}{5}{R}{}{}

    % ----  3rd column  ----
    \pin{15}{9}{L}{C}
    \gate[\inputs{2}]{and}{19}{7}{R}{}{}

    \pin{15}{15}{L}{D1}
    \pin{15}{11}{L}{D}
    \gate[\inputs{2}]{and}{19}{13}{R}{}{}

    \pin{15}{21}{L}{E1}
    \pin{15}{17}{L}{E}
    \gate[\inputs{2}]{and}{19}{19}{R}{}{}

    % ----  4th column  ----
    \wire{23}{7}{23}{11}
    \wire{23}{15}{23}{19}
    \gate[\inputs{3}]{nor}{26}{13}{R}{}{}

    % ----  last column ----
    \gate{not}{33}{13}{R}{}{}

    % ----  result ----
    \pin{37}{13}{R}{Z}

\end{circuitdiagram}

%%  ************    LibreSilicon's StdCellLibrary   *******************
%%
%%  Organisation:   Chipforge
%%                  Germany / European Union
%%
%%  Profile:        Chipforge focus on fine System-on-Chip Cores in
%%                  Verilog HDL Code which are easy understandable and
%%                  adjustable. For further information see
%%                          www.chipforge.org
%%                  there are projects from small cores up to PCBs, too.
%%
%%  File:           StdCellLib/Documents/Datasheets/Circuitry/AOAAAOI21132.tex
%%
%%  Purpose:        Circuit File for AOAAAOI21132
%%
%%  ************    LaTeX with circdia.sty package      ***************
%%
%%  ///////////////////////////////////////////////////////////////////
%%
%%  Copyright (c) 2018 - 2022 by
%%                  chipforge <stdcelllib@nospam.chipforge.org>
%%  All rights reserved.
%%
%%      This Standard Cell Library is licensed under the Libre Silicon
%%      public license; you can redistribute it and/or modify it under
%%      the terms of the Libre Silicon public license as published by
%%      the Libre Silicon alliance, either version 1 of the License, or
%%      (at your option) any later version.
%%
%%      This design is distributed in the hope that it will be useful,
%%      but WITHOUT ANY WARRANTY; without even the implied warranty of
%%      MERCHANTABILITY or FITNESS FOR A PARTICULAR PURPOSE.
%%      See the Libre Silicon Public License for more details.
%%
%%  ///////////////////////////////////////////////////////////////////
\begin{circuitdiagram}[draft]{32}{22}

    \usgate
    % ----  1st column  ----
    \pin{1}{1}{L}{A}
    \pin{1}{5}{L}{A1}
    \gate[\inputs{2}]{and}{5}{3}{R}{}{}

    % ----  2nd column  ----
    \pin{8}{7}{L}{B}
    \gate[\inputs{2}]{or}{12}{5}{R}{}{}

    % ----  3rd column  ----
    \pin{15}{9}{L}{C}
    \gate[\inputs{2}]{and}{19}{7}{R}{}{}

    \pin{15}{11}{L}{D}
    \pin{15}{13}{L}{D1}
    \pin{15}{15}{L}{D2}
    \gate[\inputs{3}]{and}{19}{13}{R}{}{}

    \pin{15}{21}{L}{E1}
    \pin{15}{17}{L}{E}
    \gate[\inputs{2}]{and}{19}{19}{R}{}{}

    % ----  4th column  ----
    \wire{23}{7}{23}{11}
    \wire{23}{15}{23}{19}
    \gate[\inputs{3}]{nor}{26}{13}{R}{}{}


    % ----  result ----
    \pin{31}{13}{R}{Y}

\end{circuitdiagram}
 %%  ************    LibreSilicon's StdCellLibrary   *******************
%%
%%  Organisation:   Chipforge
%%                  Germany / European Union
%%
%%  Profile:        Chipforge focus on fine System-on-Chip Cores in
%%                  Verilog HDL Code which are easy understandable and
%%                  adjustable. For further information see
%%                          www.chipforge.org
%%                  there are projects from small cores up to PCBs, too.
%%
%%  File:           StdCellLib/Documents/Datasheets/Circuitry/AOAAAO21132.tex
%%
%%  Purpose:        Circuit File for AOAAAO21132
%%
%%  ************    LaTeX with circdia.sty package      ***************
%%
%%  ///////////////////////////////////////////////////////////////////
%%
%%  Copyright (c) 2018 - 2022 by
%%                  chipforge <stdcelllib@nospam.chipforge.org>
%%  All rights reserved.
%%
%%      This Standard Cell Library is licensed under the Libre Silicon
%%      public license; you can redistribute it and/or modify it under
%%      the terms of the Libre Silicon public license as published by
%%      the Libre Silicon alliance, either version 1 of the License, or
%%      (at your option) any later version.
%%
%%      This design is distributed in the hope that it will be useful,
%%      but WITHOUT ANY WARRANTY; without even the implied warranty of
%%      MERCHANTABILITY or FITNESS FOR A PARTICULAR PURPOSE.
%%      See the Libre Silicon Public License for more details.
%%
%%  ///////////////////////////////////////////////////////////////////
\begin{circuitdiagram}[draft]{38}{22}

    \usgate
    % ----  1st column  ----
    \pin{1}{1}{L}{A}
    \pin{1}{5}{L}{A1}
    \gate[\inputs{2}]{and}{5}{3}{R}{}{}

    % ----  2nd column  ----
    \pin{8}{7}{L}{B}
    \gate[\inputs{2}]{or}{12}{5}{R}{}{}

    % ----  3rd column  ----
    \pin{15}{9}{L}{C}
    \gate[\inputs{2}]{and}{19}{7}{R}{}{}

    \pin{15}{11}{L}{D}
    \pin{15}{13}{L}{D1}
    \pin{15}{15}{L}{D2}
    \gate[\inputs{3}]{and}{19}{13}{R}{}{}

    \pin{15}{21}{L}{E1}
    \pin{15}{17}{L}{E}
    \gate[\inputs{2}]{and}{19}{19}{R}{}{}

    % ----  4th column  ----
    \wire{23}{7}{23}{11}
    \wire{23}{15}{23}{19}
    \gate[\inputs{3}]{nor}{26}{13}{R}{}{}

    % ----  last column ----
    \gate{not}{33}{13}{R}{}{}

    % ----  result ----
    \pin{37}{13}{R}{Z}

\end{circuitdiagram}

%%  ************    LibreSilicon's StdCellLibrary   *******************
%%
%%  Organisation:   Chipforge
%%                  Germany / European Union
%%
%%  Profile:        Chipforge focus on fine System-on-Chip Cores in
%%                  Verilog HDL Code which are easy understandable and
%%                  adjustable. For further information see
%%                          www.chipforge.org
%%                  there are projects from small cores up to PCBs, too.
%%
%%  File:           StdCellLib/Documents/Datasheets/Circuitry/AOAAAOI21222.tex
%%
%%  Purpose:        Circuit File for AOAAAOI21222
%%
%%  ************    LaTeX with circdia.sty package      ***************
%%
%%  ///////////////////////////////////////////////////////////////////
%%
%%  Copyright (c) 2018 - 2022 by
%%                  chipforge <stdcelllib@nospam.chipforge.org>
%%  All rights reserved.
%%
%%      This Standard Cell Library is licensed under the Libre Silicon
%%      public license; you can redistribute it and/or modify it under
%%      the terms of the Libre Silicon public license as published by
%%      the Libre Silicon alliance, either version 1 of the License, or
%%      (at your option) any later version.
%%
%%      This design is distributed in the hope that it will be useful,
%%      but WITHOUT ANY WARRANTY; without even the implied warranty of
%%      MERCHANTABILITY or FITNESS FOR A PARTICULAR PURPOSE.
%%      See the Libre Silicon Public License for more details.
%%
%%  ///////////////////////////////////////////////////////////////////
\begin{circuitdiagram}[draft]{32}{24}

    \usgate
    % ----  1st column  ----
    \pin{1}{1}{L}{A}
    \pin{1}{5}{L}{A1}
    \gate[\inputs{2}]{and}{5}{3}{R}{}{}

    % ----  2nd column  ----
    \pin{8}{7}{L}{B}
    \gate[\inputs{2}]{or}{12}{5}{R}{}{}

    % ----  3rd column  ----
    \wire{16}{5}{16}{7}
    \pin{15}{9}{L}{C}
    \pin{15}{11}{L}{C1}
    \gate[\inputs{3}]{and}{19}{9}{R}{}{}

    \pin{15}{17}{L}{D1}
    \pin{15}{13}{L}{D}
    \gate[\inputs{2}]{and}{19}{15}{R}{}{}

    \pin{15}{23}{L}{E1}
    \pin{15}{19}{L}{E}
    \gate[\inputs{2}]{and}{19}{21}{R}{}{}

    % ----  4th column  ----
    \wire{23}{9}{23}{13}
    \wire{23}{17}{23}{21}
    \gate[\inputs{3}]{nor}{26}{15}{R}{}{}


    % ----  result ----
    \pin{31}{17}{R}{Y}

\end{circuitdiagram}
 %%  ************    LibreSilicon's StdCellLibrary   *******************
%%
%%  Organisation:   Chipforge
%%                  Germany / European Union
%%
%%  Profile:        Chipforge focus on fine System-on-Chip Cores in
%%                  Verilog HDL Code which are easy understandable and
%%                  adjustable. For further information see
%%                          www.chipforge.org
%%                  there are projects from small cores up to PCBs, too.
%%
%%  File:           StdCellLib/Documents/Datasheets/Circuitry/AOAAAO21222.tex
%%
%%  Purpose:        Circuit File for AOAAAO21222
%%
%%  ************    LaTeX with circdia.sty package      ***************
%%
%%  ///////////////////////////////////////////////////////////////////
%%
%%  Copyright (c) 2018 - 2022 by
%%                  chipforge <stdcelllib@nospam.chipforge.org>
%%  All rights reserved.
%%
%%      This Standard Cell Library is licensed under the Libre Silicon
%%      public license; you can redistribute it and/or modify it under
%%      the terms of the Libre Silicon public license as published by
%%      the Libre Silicon alliance, either version 1 of the License, or
%%      (at your option) any later version.
%%
%%      This design is distributed in the hope that it will be useful,
%%      but WITHOUT ANY WARRANTY; without even the implied warranty of
%%      MERCHANTABILITY or FITNESS FOR A PARTICULAR PURPOSE.
%%      See the Libre Silicon Public License for more details.
%%
%%  ///////////////////////////////////////////////////////////////////
\begin{circuitdiagram}[draft]{38}{24}

    \usgate
    % ----  1st column  ----
    \pin{1}{1}{L}{A}
    \pin{1}{5}{L}{A1}
    \gate[\inputs{2}]{and}{5}{3}{R}{}{}

    % ----  2nd column  ----
    \pin{8}{7}{L}{B}
    \gate[\inputs{2}]{or}{12}{5}{R}{}{}

    % ----  3rd column  ----
    \wire{16}{5}{16}{7}
    \pin{15}{9}{L}{C}
    \pin{15}{11}{L}{C1}
    \gate[\inputs{3}]{and}{19}{9}{R}{}{}

    \pin{15}{17}{L}{D1}
    \pin{15}{13}{L}{D}
    \gate[\inputs{2}]{and}{19}{15}{R}{}{}

    \pin{15}{23}{L}{E1}
    \pin{15}{19}{L}{E}
    \gate[\inputs{2}]{and}{19}{21}{R}{}{}

    % ----  4th column  ----
    \wire{23}{9}{23}{13}
    \wire{23}{17}{23}{21}
    \gate[\inputs{3}]{nor}{26}{15}{R}{}{}

    % ----  last column ----
    \gate{not}{33}{15}{R}{}{}

    % ----  result ----
    \pin{37}{15}{R}{Z}

\end{circuitdiagram}


%%  ************    LibreSilicon's StdCellLibrary   *******************
%%
%%  Organisation:   Chipforge
%%                  Germany / European Union
%%
%%  Profile:        Chipforge focus on fine System-on-Chip Cores in
%%                  Verilog HDL Code which are easy understandable and
%%                  adjustable. For further information see
%%                          www.chipforge.org
%%                  there are projects from small cores up to PCBs, too.
%%
%%  File:           StdCellLib/Documents/section-OAOOOAI_complex.tex
%%
%%  Purpose:        Section Level File for Standard Cell Library Documentation
%%
%%  ************    LaTeX with circdia.sty package      ***************
%%
%%  ///////////////////////////////////////////////////////////////////
%%
%%  Copyright (c) 2018 - 2022 by
%%                  chipforge <stdcelllib@nospam.chipforge.org>
%%  All rights reserved.
%%
%%      This Standard Cell Library is licensed under the Libre Silicon
%%      public license; you can redistribute it and/or modify it under
%%      the terms of the Libre Silicon public license as published by
%%      the Libre Silicon alliance, either version 1 of the License, or
%%      (at your option) any later version.
%%
%%      This design is distributed in the hope that it will be useful,
%%      but WITHOUT ANY WARRANTY; without even the implied warranty of
%%      MERCHANTABILITY or FITNESS FOR A PARTICULAR PURPOSE.
%%      See the Libre Silicon Public License for more details.
%%
%%  ///////////////////////////////////////////////////////////////////
\section{OR-AND-OR-OR-OR-AND(-Invert) Complex Gates}

%%  ************    LibreSilicon's StdCellLibrary   *******************
%%
%%  Organisation:   Chipforge
%%                  Germany / European Union
%%
%%  Profile:        Chipforge focus on fine System-on-Chip Cores in
%%                  Verilog HDL Code which are easy understandable and
%%                  adjustable. For further information see
%%                          www.chipforge.org
%%                  there are projects from small cores up to PCBs, too.
%%
%%  File:           StdCellLib/Documents/Datasheets/Circuitry/OAOOOAI21121.tex
%%
%%  Purpose:        Circuit File for OAOOOAI21121
%%
%%  ************    LaTeX with circdia.sty package      ***************
%%
%%  ///////////////////////////////////////////////////////////////////
%%
%%  Copyright (c) 2018 - 2022 by
%%                  chipforge <stdcelllib@nospam.chipforge.org>
%%  All rights reserved.
%%
%%      This Standard Cell Library is licensed under the Libre Silicon
%%      public license; you can redistribute it and/or modify it under
%%      the terms of the Libre Silicon public license as published by
%%      the Libre Silicon alliance, either version 1 of the License, or
%%      (at your option) any later version.
%%
%%      This design is distributed in the hope that it will be useful,
%%      but WITHOUT ANY WARRANTY; without even the implied warranty of
%%      MERCHANTABILITY or FITNESS FOR A PARTICULAR PURPOSE.
%%      See the Libre Silicon Public License for more details.
%%
%%  ///////////////////////////////////////////////////////////////////
\begin{circuitdiagram}[draft]{32}{22}

    \usgate
    % ----  1st column  ----
    \pin{1}{1}{L}{A}
    \pin{1}{5}{L}{A1}
    \gate[\inputs{2}]{or}{5}{3}{R}{}{}

    % ----  2nd column  ----
    \pin{8}{7}{L}{B}
    \gate[\inputs{2}]{and}{12}{5}{R}{}{}

    % ----  3rd column  ----
    \pin{15}{9}{L}{C}
    \gate[\inputs{2}]{or}{19}{7}{R}{}{}

    \pin{15}{15}{L}{D1}
    \pin{15}{11}{L}{D}
    \gate[\inputs{2}]{or}{19}{13}{R}{}{}

    \pin{15}{21}{L}{E1}
    \pin{15}{17}{L}{E}
    \gate[\inputs{2}]{or}{19}{19}{R}{}{}

    % ----  4th column  ----
    \wire{23}{7}{23}{11}
    \wire{23}{15}{23}{19}
    \gate[\inputs{3}]{nand}{26}{13}{R}{}{}


    % ----  result ----
    \pin{31}{13}{R}{Y}

\end{circuitdiagram}
 %%  ************    LibreSilicon's StdCellLibrary   *******************
%%
%%  Organisation:   Chipforge
%%                  Germany / European Union
%%
%%  Profile:        Chipforge focus on fine System-on-Chip Cores in
%%                  Verilog HDL Code which are easy understandable and
%%                  adjustable. For further information see
%%                          www.chipforge.org
%%                  there are projects from small cores up to PCBs, too.
%%
%%  File:           StdCellLib/Documents/Datasheets/Circuitry/OAOOOA21121.tex
%%
%%  Purpose:        Circuit File for OAOOOA21121
%%
%%  ************    LaTeX with circdia.sty package      ***************
%%
%%  ///////////////////////////////////////////////////////////////////
%%
%%  Copyright (c) 2018 - 2022 by
%%                  chipforge <stdcelllib@nospam.chipforge.org>
%%  All rights reserved.
%%
%%      This Standard Cell Library is licensed under the Libre Silicon
%%      public license; you can redistribute it and/or modify it under
%%      the terms of the Libre Silicon public license as published by
%%      the Libre Silicon alliance, either version 1 of the License, or
%%      (at your option) any later version.
%%
%%      This design is distributed in the hope that it will be useful,
%%      but WITHOUT ANY WARRANTY; without even the implied warranty of
%%      MERCHANTABILITY or FITNESS FOR A PARTICULAR PURPOSE.
%%      See the Libre Silicon Public License for more details.
%%
%%  ///////////////////////////////////////////////////////////////////
\begin{circuitdiagram}[draft]{38}{22}

    \usgate
    % ----  1st column  ----
    \pin{1}{1}{L}{A}
    \pin{1}{5}{L}{A1}
    \gate[\inputs{2}]{or}{5}{3}{R}{}{}

    % ----  2nd column  ----
    \pin{8}{7}{L}{B}
    \gate[\inputs{2}]{and}{12}{5}{R}{}{}

    % ----  3rd column  ----
    \pin{15}{9}{L}{C}
    \gate[\inputs{2}]{or}{19}{7}{R}{}{}

    \pin{15}{15}{L}{D1}
    \pin{15}{11}{L}{D}
    \gate[\inputs{2}]{or}{19}{13}{R}{}{}

    \pin{15}{21}{L}{E1}
    \pin{15}{17}{L}{E}
    \gate[\inputs{2}]{or}{19}{19}{R}{}{}

    % ----  4th column  ----
    \wire{23}{7}{23}{11}
    \wire{23}{15}{23}{19}
    \gate[\inputs{3}]{nand}{26}{13}{R}{}{}

    % ----  last column ----
    \gate{not}{33}{13}{R}{}{}

    % ----  result ----
    \pin{37}{13}{R}{Z}

\end{circuitdiagram}


%%  ************    LibreSilicon's StdCellLibrary   *******************
%%
%%  Organisation:   Chipforge
%%                  Germany / European Union
%%
%%  Profile:        Chipforge focus on fine System-on-Chip Cores in
%%                  Verilog HDL Code which are easy understandable and
%%                  adjustable. For further information see
%%                          www.chipforge.org
%%                  there are projects from small cores up to PCBs, too.
%%
%%  File:           StdCellLib/Documents/Book/section-AOOAOI_complex.tex
%%
%%  Purpose:        Section Level File for Standard Cell Library Documentation
%%
%%  ************    LaTeX with circdia.sty package      ***************
%%
%%  ///////////////////////////////////////////////////////////////////
%%
%%  Copyright (c) 2018 - 2022 by
%%                  chipforge <stdcelllib@nospam.chipforge.org>
%%  All rights reserved.
%%
%%      This Standard Cell Library is licensed under the Libre Silicon
%%      public license; you can redistribute it and/or modify it under
%%      the terms of the Libre Silicon public license as published by
%%      the Libre Silicon alliance, either version 1 of the License, or
%%      (at your option) any later version.
%%
%%      This design is distributed in the hope that it will be useful,
%%      but WITHOUT ANY WARRANTY; without even the implied warranty of
%%      MERCHANTABILITY or FITNESS FOR A PARTICULAR PURPOSE.
%%      See the Libre Silicon Public License for more details.
%%
%%  ///////////////////////////////////////////////////////////////////
\section{AND-OR-OR-AND-OR(-Invert) Complex Gates}


%%  ************    LibreSilicon's StdCellLibrary   *******************
%%
%%  Organisation:   Chipforge
%%                  Germany / European Union
%%
%%  Profile:        Chipforge focus on fine System-on-Chip Cores in
%%                  Verilog HDL Code which are easy understandable and
%%                  adjustable. For further information see
%%                          www.chipforge.org
%%                  there are projects from small cores up to PCBs, too.
%%
%%  File:           StdCellLib/Documents/LaTeX/section-OAAOAI_complex.tex
%%
%%  Purpose:        Section Level File for Standard Cell Library Documentation
%%
%%  ************    LaTeX with circdia.sty package      ***************
%%
%%  ///////////////////////////////////////////////////////////////////
%%
%%  Copyright (c) 2018 - 2021 by
%%                  chipforge <stdcelllib@nospam.chipforge.org>
%%  All rights reserved.
%%
%%      This Standard Cell Library is licensed under the Libre Silicon
%%      public license; you can redistribute it and/or modify it under
%%      the terms of the Libre Silicon public license as published by
%%      the Libre Silicon alliance, either version 1 of the License, or
%%      (at your option) any later version.
%%
%%      This design is distributed in the hope that it will be useful,
%%      but WITHOUT ANY WARRANTY; without even the implied warranty of
%%      MERCHANTABILITY or FITNESS FOR A PARTICULAR PURPOSE.
%%      See the Libre Silicon Public License for more details.
%%
\section{OR-AND-AND-OR-AND(-Invert) Complex Gates}


%%  ************    LibreSilicon's StdCellLibrary   *******************
%%
%%  Organisation:   Chipforge
%%                  Germany / European Union
%%
%%  Profile:        Chipforge focus on fine System-on-Chip Cores in
%%                  Verilog HDL Code which are easy understandable and
%%                  adjustable. For further information see
%%                          www.chipforge.org
%%                  there are projects from small cores up to PCBs, too.
%%
%%  File:           StdCellLib/Documents/section-AAOAAOI_complex.tex
%%
%%  Purpose:        Section Level File for Standard Cell Library Documentation
%%
%%  ************    LaTeX with circdia.sty package      ***************
%%
%%  ///////////////////////////////////////////////////////////////////
%%
%%  Copyright (c) 2018 - 2022 by
%%                  chipforge <stdcelllib@nospam.chipforge.org>
%%  All rights reserved.
%%
%%      This Standard Cell Library is licensed under the Libre Silicon
%%      public license; you can redistribute it and/or modify it under
%%      the terms of the Libre Silicon public license as published by
%%      the Libre Silicon alliance, either version 1 of the License, or
%%      (at your option) any later version.
%%
%%      This design is distributed in the hope that it will be useful,
%%      but WITHOUT ANY WARRANTY; without even the implied warranty of
%%      MERCHANTABILITY or FITNESS FOR A PARTICULAR PURPOSE.
%%      See the Libre Silicon Public License for more details.
%%
%%  ///////////////////////////////////////////////////////////////////
\section{AND-AND-OR-AND-AND-OR(-Invert) Complex Gates}

%%  ************    LibreSilicon's StdCellLibrary   *******************
%%
%%  Organisation:   Chipforge
%%                  Germany / European Union
%%
%%  Profile:        Chipforge focus on fine System-on-Chip Cores in
%%                  Verilog HDL Code which are easy understandable and
%%                  adjustable. For further information see
%%                          www.chipforge.org
%%                  there are projects from small cores up to PCBs, too.
%%
%%  File:           StdCellLib/Documents/Datasheets/Circuitry/AAOAAOI2212.tex
%%
%%  Purpose:        Circuit File for AAOAAOI2212
%%
%%  ************    LaTeX with circdia.sty package      ***************
%%
%%  ///////////////////////////////////////////////////////////////////
%%
%%  Copyright (c) 2018 - 2022 by
%%                  chipforge <stdcelllib@nospam.chipforge.org>
%%  All rights reserved.
%%
%%      This Standard Cell Library is licensed under the Libre Silicon
%%      public license; you can redistribute it and/or modify it under
%%      the terms of the Libre Silicon public license as published by
%%      the Libre Silicon alliance, either version 1 of the License, or
%%      (at your option) any later version.
%%
%%      This design is distributed in the hope that it will be useful,
%%      but WITHOUT ANY WARRANTY; without even the implied warranty of
%%      MERCHANTABILITY or FITNESS FOR A PARTICULAR PURPOSE.
%%      See the Libre Silicon Public License for more details.
%%
%%  ///////////////////////////////////////////////////////////////////
\begin{circuitdiagram}[draft]{32}{18}

    \usgate
    % ----  1st column  ----
    \pin{1}{1}{L}{A}
    \pin{1}{5}{L}{A1}
    \gate[\inputs{2}]{and}{5}{3}{R}{}{}

    \pin{1}{7}{L}{B}
    \pin{1}{11}{L}{B1}
    \gate[\inputs{2}]{and}{5}{9}{R}{}{}

    % ----  2nd column  ----
    \wire{9}{3}{9}{5}
    \gate[\inputs{2}]{or}{12}{7}{R}{}{}

    % ----  3rd column  ----
    \pin{15}{11}{L}{C}
    \gate[\inputs{2}]{and}{19}{9}{R}{}{}

    \pin{15}{13}{L}{D}
    \pin{15}{17}{L}{D1}
    \gate[\inputs{2}]{and}{19}{15}{R}{}{}

    % ----  4th column  ----
    \wire{23}{9}{23}{11}
    \gate[\inputs{2}]{nor}{26}{13}{R}{}{}

    % ----  result ----
    \pin{31}{13}{R}{Y}

\end{circuitdiagram}
 %%  ************    LibreSilicon's StdCellLibrary   *******************
%%
%%  Organisation:   Chipforge
%%                  Germany / European Union
%%
%%  Profile:        Chipforge focus on fine System-on-Chip Cores in
%%                  Verilog HDL Code which are easy understandable and
%%                  adjustable. For further information see
%%                          www.chipforge.org
%%                  there are projects from small cores up to PCBs, too.
%%
%%  File:           StdCellLib/Documents/Datasheets/Circuitry/AAOAAO2212.tex
%%
%%  Purpose:        Circuit File for AAOAAO2212
%%
%%  ************    LaTeX with circdia.sty package      ***************
%%
%%  ///////////////////////////////////////////////////////////////////
%%
%%  Copyright (c) 2018 - 2022 by
%%                  chipforge <stdcelllib@nospam.chipforge.org>
%%  All rights reserved.
%%
%%      This Standard Cell Library is licensed under the Libre Silicon
%%      public license; you can redistribute it and/or modify it under
%%      the terms of the Libre Silicon public license as published by
%%      the Libre Silicon alliance, either version 1 of the License, or
%%      (at your option) any later version.
%%
%%      This design is distributed in the hope that it will be useful,
%%      but WITHOUT ANY WARRANTY; without even the implied warranty of
%%      MERCHANTABILITY or FITNESS FOR A PARTICULAR PURPOSE.
%%      See the Libre Silicon Public License for more details.
%%
%%  ///////////////////////////////////////////////////////////////////
\begin{circuitdiagram}[draft]{38}{18}

    \usgate
    % ----  1st column  ----
    \pin{1}{1}{L}{A}
    \pin{1}{5}{L}{A1}
    \gate[\inputs{2}]{and}{5}{3}{R}{}{}

    \pin{1}{7}{L}{B}
    \pin{1}{11}{L}{B1}
    \gate[\inputs{2}]{and}{5}{9}{R}{}{}

    % ----  2nd column  ----
    \wire{9}{3}{9}{5}
    \gate[\inputs{2}]{or}{12}{7}{R}{}{}

    % ----  3rd column  ----
    \pin{15}{11}{L}{C}
    \gate[\inputs{2}]{and}{19}{9}{R}{}{}

    \pin{15}{13}{L}{D}
    \pin{15}{17}{L}{D1}
    \gate[\inputs{2}]{and}{19}{15}{R}{}{}

    % ----  4th column  ----
    \wire{23}{9}{23}{11}
    \gate[\inputs{2}]{nor}{26}{13}{R}{}{}

    % ----  4th column  ----
    \gate{not}{33}{13}{R}{}{}

    % ----  result ----
    \pin{37}{13}{R}{Z}

\end{circuitdiagram}

%%  ************    LibreSilicon's StdCellLibrary   *******************
%%
%%  Organisation:   Chipforge
%%                  Germany / European Union
%%
%%  Profile:        Chipforge focus on fine System-on-Chip Cores in
%%                  Verilog HDL Code which are easy understandable and
%%                  adjustable. For further information see
%%                          www.chipforge.org
%%                  there are projects from small cores up to PCBs, too.
%%
%%  File:           StdCellLib/Documents/Datasheets/Circuitry/AAOAAOI2213.tex
%%
%%  Purpose:        Circuit File for AAOAAOI2213
%%
%%  ************    LaTeX with circdia.sty package      ***************
%%
%%  ///////////////////////////////////////////////////////////////////
%%
%%  Copyright (c) 2018 - 2022 by
%%                  chipforge <stdcelllib@nospam.chipforge.org>
%%  All rights reserved.
%%
%%      This Standard Cell Library is licensed under the Libre Silicon
%%      public license; you can redistribute it and/or modify it under
%%      the terms of the Libre Silicon public license as published by
%%      the Libre Silicon alliance, either version 1 of the License, or
%%      (at your option) any later version.
%%
%%      This design is distributed in the hope that it will be useful,
%%      but WITHOUT ANY WARRANTY; without even the implied warranty of
%%      MERCHANTABILITY or FITNESS FOR A PARTICULAR PURPOSE.
%%      See the Libre Silicon Public License for more details.
%%
%%  ///////////////////////////////////////////////////////////////////
\begin{circuitdiagram}[draft]{32}{18}

    \usgate
    % ----  1st column  ----
    \pin{1}{1}{L}{A}
    \pin{1}{5}{L}{A1}
    \gate[\inputs{2}]{and}{5}{3}{R}{}{}

    \pin{1}{7}{L}{B}
    \pin{1}{11}{L}{B1}
    \gate[\inputs{2}]{and}{5}{9}{R}{}{}

    % ----  2nd column  ----
    \wire{9}{3}{9}{5}
    \gate[\inputs{2}]{or}{12}{7}{R}{}{}

    % ----  3rd column  ----
    \pin{15}{11}{L}{C}
    \gate[\inputs{2}]{and}{19}{9}{R}{}{}

    \pin{15}{13}{L}{D}
    \pin{15}{15}{L}{D1}
    \pin{15}{17}{L}{D2}
    \gate[\inputs{3}]{and}{19}{15}{R}{}{}

    % ----  4th column  ----
    \wire{23}{9}{23}{11}
    \gate[\inputs{2}]{nor}{26}{13}{R}{}{}

    % ----  result ----
    \pin{31}{13}{R}{Y}

\end{circuitdiagram}
 %%  ************    LibreSilicon's StdCellLibrary   *******************
%%
%%  Organisation:   Chipforge
%%                  Germany / European Union
%%
%%  Profile:        Chipforge focus on fine System-on-Chip Cores in
%%                  Verilog HDL Code which are easy understandable and
%%                  adjustable. For further information see
%%                          www.chipforge.org
%%                  there are projects from small cores up to PCBs, too.
%%
%%  File:           StdCellLib/Documents/Datasheets/Circuitry/AAOAAO2213.tex
%%
%%  Purpose:        Circuit File for AAOAAO2213
%%
%%  ************    LaTeX with circdia.sty package      ***************
%%
%%  ///////////////////////////////////////////////////////////////////
%%
%%  Copyright (c) 2018 - 2022 by
%%                  chipforge <stdcelllib@nospam.chipforge.org>
%%  All rights reserved.
%%
%%      This Standard Cell Library is licensed under the Libre Silicon
%%      public license; you can redistribute it and/or modify it under
%%      the terms of the Libre Silicon public license as published by
%%      the Libre Silicon alliance, either version 1 of the License, or
%%      (at your option) any later version.
%%
%%      This design is distributed in the hope that it will be useful,
%%      but WITHOUT ANY WARRANTY; without even the implied warranty of
%%      MERCHANTABILITY or FITNESS FOR A PARTICULAR PURPOSE.
%%      See the Libre Silicon Public License for more details.
%%
%%  ///////////////////////////////////////////////////////////////////
\begin{circuitdiagram}[draft]{38}{18}

    \usgate
    % ----  1st column  ----
    \pin{1}{1}{L}{A}
    \pin{1}{5}{L}{A1}
    \gate[\inputs{2}]{and}{5}{3}{R}{}{}

    \pin{1}{7}{L}{B}
    \pin{1}{11}{L}{B1}
    \gate[\inputs{2}]{and}{5}{9}{R}{}{}

    % ----  2nd column  ----
    \wire{9}{3}{9}{5}
    \gate[\inputs{2}]{or}{12}{7}{R}{}{}

    % ----  3rd column  ----
    \pin{15}{11}{L}{C}
    \gate[\inputs{2}]{and}{19}{9}{R}{}{}

    \pin{15}{13}{L}{D}
    \pin{15}{15}{L}{D1}
    \pin{15}{17}{L}{D2}
    \gate[\inputs{3}]{and}{19}{15}{R}{}{}

    % ----  4th column  ----
    \wire{23}{9}{23}{11}
    \gate[\inputs{2}]{nor}{26}{13}{R}{}{}

    % ----  4th column  ----
    \gate{not}{33}{13}{R}{}{}

    % ----  result ----
    \pin{37}{13}{R}{Z}

\end{circuitdiagram}

%%  ************    LibreSilicon's StdCellLibrary   *******************
%%
%%  Organisation:   Chipforge
%%                  Germany / European Union
%%
%%  Profile:        Chipforge focus on fine System-on-Chip Cores in
%%                  Verilog HDL Code which are easy understandable and
%%                  adjustable. For further information see
%%                          www.chipforge.org
%%                  there are projects from small cores up to PCBs, too.
%%
%%  File:           StdCellLib/Documents/Datasheets/Circuitry/AAOAAOI2214.tex
%%
%%  Purpose:        Circuit File for AAOAAOI2214
%%
%%  ************    LaTeX with circdia.sty package      ***************
%%
%%  ///////////////////////////////////////////////////////////////////
%%
%%  Copyright (c) 2018 - 2022 by
%%                  chipforge <stdcelllib@nospam.chipforge.org>
%%  All rights reserved.
%%
%%      This Standard Cell Library is licensed under the Libre Silicon
%%      public license; you can redistribute it and/or modify it under
%%      the terms of the Libre Silicon public license as published by
%%      the Libre Silicon alliance, either version 1 of the License, or
%%      (at your option) any later version.
%%
%%      This design is distributed in the hope that it will be useful,
%%      but WITHOUT ANY WARRANTY; without even the implied warranty of
%%      MERCHANTABILITY or FITNESS FOR A PARTICULAR PURPOSE.
%%      See the Libre Silicon Public License for more details.
%%
%%  ///////////////////////////////////////////////////////////////////
\begin{circuitdiagram}[draft]{32}{20}

    \usgate
    % ----  1st column  ----
    \pin{1}{1}{L}{A}
    \pin{1}{5}{L}{A1}
    \gate[\inputs{2}]{and}{5}{3}{R}{}{}

    \pin{1}{7}{L}{B}
    \pin{1}{11}{L}{B1}
    \gate[\inputs{2}]{and}{5}{9}{R}{}{}

    % ----  2nd column  ----
    \wire{9}{3}{9}{5}
    \gate[\inputs{2}]{or}{12}{7}{R}{}{}

    % ----  3rd column  ----
    \pin{15}{11}{L}{C}
    \gate[\inputs{2}]{and}{19}{9}{R}{}{}

    \pin{15}{13}{L}{D}
    \pin{15}{15}{L}{D1}
    \pin{15}{17}{L}{D2}
    \pin{15}{19}{L}{D3}
    \gate[\inputs{4}]{and}{19}{16}{R}{}{}

    % ----  4th column  ----
    \wire{23}{9}{23}{12}
    \gate[\inputs{2}]{nor}{26}{14}{R}{}{}

    % ----  result ----
    \pin{31}{14}{R}{Y}

\end{circuitdiagram}
 %%  ************    LibreSilicon's StdCellLibrary   *******************
%%
%%  Organisation:   Chipforge
%%                  Germany / European Union
%%
%%  Profile:        Chipforge focus on fine System-on-Chip Cores in
%%                  Verilog HDL Code which are easy understandable and
%%                  adjustable. For further information see
%%                          www.chipforge.org
%%                  there are projects from small cores up to PCBs, too.
%%
%%  File:           StdCellLib/Documents/Datasheets/Circuitry/AAOAAO2214.tex
%%
%%  Purpose:        Circuit File for AAOAAO2214
%%
%%  ************    LaTeX with circdia.sty package      ***************
%%
%%  ///////////////////////////////////////////////////////////////////
%%
%%  Copyright (c) 2018 - 2022 by
%%                  chipforge <stdcelllib@nospam.chipforge.org>
%%  All rights reserved.
%%
%%      This Standard Cell Library is licensed under the Libre Silicon
%%      public license; you can redistribute it and/or modify it under
%%      the terms of the Libre Silicon public license as published by
%%      the Libre Silicon alliance, either version 1 of the License, or
%%      (at your option) any later version.
%%
%%      This design is distributed in the hope that it will be useful,
%%      but WITHOUT ANY WARRANTY; without even the implied warranty of
%%      MERCHANTABILITY or FITNESS FOR A PARTICULAR PURPOSE.
%%      See the Libre Silicon Public License for more details.
%%
%%  ///////////////////////////////////////////////////////////////////
\begin{circuitdiagram}[draft]{38}{20}

    \usgate
    % ----  1st column  ----
    \pin{1}{1}{L}{A}
    \pin{1}{5}{L}{A1}
    \gate[\inputs{2}]{and}{5}{3}{R}{}{}

    \pin{1}{7}{L}{B}
    \pin{1}{11}{L}{B1}
    \gate[\inputs{2}]{and}{5}{9}{R}{}{}

    % ----  2nd column  ----
    \wire{9}{3}{9}{5}
    \gate[\inputs{2}]{or}{12}{7}{R}{}{}

    % ----  3rd column  ----
    \pin{15}{11}{L}{C}
    \gate[\inputs{2}]{and}{19}{9}{R}{}{}

    \pin{15}{13}{L}{D}
    \pin{15}{15}{L}{D1}
    \pin{15}{17}{L}{D2}
    \pin{15}{19}{L}{D3}
    \gate[\inputs{4}]{and}{19}{16}{R}{}{}

    % ----  4th column  ----
    \wire{23}{9}{23}{12}
    \gate[\inputs{2}]{nor}{26}{14}{R}{}{}

    % ----  4th column  ----
    \gate{not}{33}{14}{R}{}{}

    % ----  result ----
    \pin{37}{14}{R}{Z}

\end{circuitdiagram}


%%  ************    LibreSilicon's StdCellLibrary   *******************
%%
%%  Organisation:   Chipforge
%%                  Germany / European Union
%%
%%  Profile:        Chipforge focus on fine System-on-Chip Cores in
%%                  Verilog HDL Code which are easy understandable and
%%                  adjustable. For further information see
%%                          www.chipforge.org
%%                  there are projects from small cores up to PCBs, too.
%%
%%  File:           StdCellLib/Documents/Datasheets/Circuitry/AAOAAOI22121.tex
%%
%%  Purpose:        Circuit File for AAOAAOI22121
%%
%%  ************    LaTeX with circdia.sty package      ***************
%%
%%  ///////////////////////////////////////////////////////////////////
%%
%%  Copyright (c) 2018 - 2022 by
%%                  chipforge <stdcelllib@nospam.chipforge.org>
%%  All rights reserved.
%%
%%      This Standard Cell Library is licensed under the Libre Silicon
%%      public license; you can redistribute it and/or modify it under
%%      the terms of the Libre Silicon public license as published by
%%      the Libre Silicon alliance, either version 1 of the License, or
%%      (at your option) any later version.
%%
%%      This design is distributed in the hope that it will be useful,
%%      but WITHOUT ANY WARRANTY; without even the implied warranty of
%%      MERCHANTABILITY or FITNESS FOR A PARTICULAR PURPOSE.
%%      See the Libre Silicon Public License for more details.
%%
%%  ///////////////////////////////////////////////////////////////////
\begin{circuitdiagram}[draft]{32}{20}

    \usgate
    % ----  1st column  ----
    \pin{1}{1}{L}{A}
    \pin{1}{5}{L}{A1}
    \gate[\inputs{2}]{and}{5}{3}{R}{}{}

    \pin{1}{7}{L}{B}
    \pin{1}{11}{L}{B1}
    \gate[\inputs{2}]{and}{5}{9}{R}{}{}

    % ----  2nd column  ----
    \wire{9}{3}{9}{5}
    \gate[\inputs{2}]{or}{12}{7}{R}{}{}

    % ----  3rd column  ----
    \pin{15}{11}{L}{C}
    \gate[\inputs{2}]{and}{19}{9}{R}{}{}

    \pin{15}{13}{L}{D}
    \pin{15}{17}{L}{D1}
    \gate[\inputs{2}]{and}{19}{15}{R}{}{}

    % ----  4th column  ----
    \wire{23}{9}{23}{13}
    \pin{22}{19}{L}{E}
    \wire{23}{17}{23}{19}
    \gate[\inputs{3}]{nor}{26}{15}{R}{}{}

    % ----  result ----
    \pin{31}{15}{R}{Y}

\end{circuitdiagram}
 %%  ************    LibreSilicon's StdCellLibrary   *******************
%%
%%  Organisation:   Chipforge
%%                  Germany / European Union
%%
%%  Profile:        Chipforge focus on fine System-on-Chip Cores in
%%                  Verilog HDL Code which are easy understandable and
%%                  adjustable. For further information see
%%                          www.chipforge.org
%%                  there are projects from small cores up to PCBs, too.
%%
%%  File:           StdCellLib/Documents/Datasheets/Circuitry/AAOAAO22121.tex
%%
%%  Purpose:        Circuit File for AAOAAO22121
%%
%%  ************    LaTeX with circdia.sty package      ***************
%%
%%  ///////////////////////////////////////////////////////////////////
%%
%%  Copyright (c) 2018 - 2022 by
%%                  chipforge <stdcelllib@nospam.chipforge.org>
%%  All rights reserved.
%%
%%      This Standard Cell Library is licensed under the Libre Silicon
%%      public license; you can redistribute it and/or modify it under
%%      the terms of the Libre Silicon public license as published by
%%      the Libre Silicon alliance, either version 1 of the License, or
%%      (at your option) any later version.
%%
%%      This design is distributed in the hope that it will be useful,
%%      but WITHOUT ANY WARRANTY; without even the implied warranty of
%%      MERCHANTABILITY or FITNESS FOR A PARTICULAR PURPOSE.
%%      See the Libre Silicon Public License for more details.
%%
%%  ///////////////////////////////////////////////////////////////////
\begin{circuitdiagram}[draft]{38}{20}

    \usgate
    % ----  1st column  ----
    \pin{1}{1}{L}{A}
    \pin{1}{5}{L}{A1}
    \gate[\inputs{2}]{and}{5}{3}{R}{}{}

    \pin{1}{7}{L}{B}
    \pin{1}{11}{L}{B1}
    \gate[\inputs{2}]{and}{5}{9}{R}{}{}

    % ----  2nd column  ----
    \wire{9}{3}{9}{5}
    \gate[\inputs{2}]{or}{12}{7}{R}{}{}

    % ----  3rd column  ----
    \pin{15}{11}{L}{C}
    \gate[\inputs{2}]{and}{19}{9}{R}{}{}

    \pin{15}{13}{L}{D}
    \pin{15}{17}{L}{D1}
    \gate[\inputs{2}]{and}{19}{15}{R}{}{}

    % ----  4th column  ----
    \wire{23}{9}{23}{13}
    \pin{22}{19}{L}{E}
    \wire{23}{17}{23}{19}
    \gate[\inputs{3}]{nor}{26}{15}{R}{}{}

    % ----  4th column  ----
    \gate{not}{33}{15}{R}{}{}

    % ----  result ----
    \pin{37}{15}{R}{Z}

\end{circuitdiagram}

%%  ************    LibreSilicon's StdCellLibrary   *******************
%%
%%  Organisation:   Chipforge
%%                  Germany / European Union
%%
%%  Profile:        Chipforge focus on fine System-on-Chip Cores in
%%                  Verilog HDL Code which are easy understandable and
%%                  adjustable. For further information see
%%                          www.chipforge.org
%%                  there are projects from small cores up to PCBs, too.
%%
%%  File:           StdCellLib/Documents/Datasheets/Circuitry/AAOAAOI22131.tex
%%
%%  Purpose:        Circuit File for AAOAAOI22131
%%
%%  ************    LaTeX with circdia.sty package      ***************
%%
%%  ///////////////////////////////////////////////////////////////////
%%
%%  Copyright (c) 2018 - 2022 by
%%                  chipforge <stdcelllib@nospam.chipforge.org>
%%  All rights reserved.
%%
%%      This Standard Cell Library is licensed under the Libre Silicon
%%      public license; you can redistribute it and/or modify it under
%%      the terms of the Libre Silicon public license as published by
%%      the Libre Silicon alliance, either version 1 of the License, or
%%      (at your option) any later version.
%%
%%      This design is distributed in the hope that it will be useful,
%%      but WITHOUT ANY WARRANTY; without even the implied warranty of
%%      MERCHANTABILITY or FITNESS FOR A PARTICULAR PURPOSE.
%%      See the Libre Silicon Public License for more details.
%%
%%  ///////////////////////////////////////////////////////////////////
\begin{circuitdiagram}[draft]{32}{20}

    \usgate
    % ----  1st column  ----
    \pin{1}{1}{L}{A}
    \pin{1}{5}{L}{A1}
    \gate[\inputs{2}]{and}{5}{3}{R}{}{}

    \pin{1}{7}{L}{B}
    \pin{1}{11}{L}{B1}
    \gate[\inputs{2}]{and}{5}{9}{R}{}{}

    % ----  2nd column  ----
    \wire{9}{3}{9}{5}
    \gate[\inputs{2}]{or}{12}{7}{R}{}{}

    % ----  3rd column  ----
    \pin{15}{11}{L}{C}
    \gate[\inputs{2}]{and}{19}{9}{R}{}{}

    \pin{15}{13}{L}{D}
    \pin{15}{15}{L}{D1}
    \pin{15}{17}{L}{D2}
    \gate[\inputs{3}]{and}{19}{15}{R}{}{}

    % ----  4th column  ----
    \wire{23}{9}{23}{13}
    \pin{22}{19}{L}{E}
    \wire{23}{17}{23}{19}
    \gate[\inputs{3}]{nor}{26}{15}{R}{}{}

    % ----  result ----
    \pin{31}{15}{R}{Y}

\end{circuitdiagram}
 %%  ************    LibreSilicon's StdCellLibrary   *******************
%%
%%  Organisation:   Chipforge
%%                  Germany / European Union
%%
%%  Profile:        Chipforge focus on fine System-on-Chip Cores in
%%                  Verilog HDL Code which are easy understandable and
%%                  adjustable. For further information see
%%                          www.chipforge.org
%%                  there are projects from small cores up to PCBs, too.
%%
%%  File:           StdCellLib/Documents/Datasheets/Circuitry/AAOAAO22131.tex
%%
%%  Purpose:        Circuit File for AAOAAO22131
%%
%%  ************    LaTeX with circdia.sty package      ***************
%%
%%  ///////////////////////////////////////////////////////////////////
%%
%%  Copyright (c) 2018 - 2022 by
%%                  chipforge <stdcelllib@nospam.chipforge.org>
%%  All rights reserved.
%%
%%      This Standard Cell Library is licensed under the Libre Silicon
%%      public license; you can redistribute it and/or modify it under
%%      the terms of the Libre Silicon public license as published by
%%      the Libre Silicon alliance, either version 1 of the License, or
%%      (at your option) any later version.
%%
%%      This design is distributed in the hope that it will be useful,
%%      but WITHOUT ANY WARRANTY; without even the implied warranty of
%%      MERCHANTABILITY or FITNESS FOR A PARTICULAR PURPOSE.
%%      See the Libre Silicon Public License for more details.
%%
%%  ///////////////////////////////////////////////////////////////////
\begin{circuitdiagram}[draft]{38}{20}

    \usgate
    % ----  1st column  ----
    \pin{1}{1}{L}{A}
    \pin{1}{5}{L}{A1}
    \gate[\inputs{2}]{and}{5}{3}{R}{}{}

    \pin{1}{7}{L}{B}
    \pin{1}{11}{L}{B1}
    \gate[\inputs{2}]{and}{5}{9}{R}{}{}

    % ----  2nd column  ----
    \wire{9}{3}{9}{5}
    \gate[\inputs{2}]{or}{12}{7}{R}{}{}

    % ----  3rd column  ----
    \pin{15}{11}{L}{C}
    \gate[\inputs{2}]{and}{19}{9}{R}{}{}

    \pin{15}{13}{L}{D}
    \pin{15}{15}{L}{D1}
    \pin{15}{17}{L}{D2}
    \gate[\inputs{3}]{and}{19}{15}{R}{}{}

    % ----  4th column  ----
    \wire{23}{9}{23}{13}
    \pin{22}{19}{L}{E}
    \wire{23}{17}{23}{19}
    \gate[\inputs{3}]{nor}{26}{15}{R}{}{}

    % ----  4th column  ----
    \gate{not}{33}{15}{R}{}{}

    % ----  result ----
    \pin{37}{15}{R}{Z}

\end{circuitdiagram}


%%  ************    LibreSilicon's StdCellLibrary   *******************
%%
%%  Organisation:   Chipforge
%%                  Germany / European Union
%%
%%  Profile:        Chipforge focus on fine System-on-Chip Cores in
%%                  Verilog HDL Code which are easy understandable and
%%                  adjustable. For further information see
%%                          www.chipforge.org
%%                  there are projects from small cores up to PCBs, too.
%%
%%  File:           StdCellLib/Documents/section-OOAOOAI_complex.tex
%%
%%  Purpose:        Section Level File for Standard Cell Library Documentation
%%
%%  ************    LaTeX with circdia.sty package      ***************
%%
%%  ///////////////////////////////////////////////////////////////////
%%
%%  Copyright (c) 2018 - 2022 by
%%                  chipforge <stdcelllib@nospam.chipforge.org>
%%  All rights reserved.
%%
%%      This Standard Cell Library is licensed under the Libre Silicon
%%      public license; you can redistribute it and/or modify it under
%%      the terms of the Libre Silicon public license as published by
%%      the Libre Silicon alliance, either version 1 of the License, or
%%      (at your option) any later version.
%%
%%      This design is distributed in the hope that it will be useful,
%%      but WITHOUT ANY WARRANTY; without even the implied warranty of
%%      MERCHANTABILITY or FITNESS FOR A PARTICULAR PURPOSE.
%%      See the Libre Silicon Public License for more details.
%%
%%  ///////////////////////////////////////////////////////////////////
\section{OR-OR-AND-OR-OR-AND(-Invert) Complex Gates}

%%  ************    LibreSilicon's StdCellLibrary   *******************
%%
%%  Organisation:   Chipforge
%%                  Germany / European Union
%%
%%  Profile:        Chipforge focus on fine System-on-Chip Cores in
%%                  Verilog HDL Code which are easy understandable and
%%                  adjustable. For further information see
%%                          www.chipforge.org
%%                  there are projects from small cores up to PCBs, too.
%%
%%  File:           StdCellLib/Documents/Datasheets/Circuitry/OOAOOAI2212.tex
%%
%%  Purpose:        Circuit File for OOAOOAI2212
%%
%%  ************    LaTeX with circdia.sty package      ***************
%%
%%  ///////////////////////////////////////////////////////////////////
%%
%%  Copyright (c) 2018 - 2022 by
%%                  chipforge <stdcelllib@nospam.chipforge.org>
%%  All rights reserved.
%%
%%      This Standard Cell Library is licensed under the Libre Silicon
%%      public license; you can redistribute it and/or modify it under
%%      the terms of the Libre Silicon public license as published by
%%      the Libre Silicon alliance, either version 1 of the License, or
%%      (at your option) any later version.
%%
%%      This design is distributed in the hope that it will be useful,
%%      but WITHOUT ANY WARRANTY; without even the implied warranty of
%%      MERCHANTABILITY or FITNESS FOR A PARTICULAR PURPOSE.
%%      See the Libre Silicon Public License for more details.
%%
%%  ///////////////////////////////////////////////////////////////////
\begin{circuitdiagram}[draft]{32}{18}

    \usgate
    % ----  1st column  ----
    \pin{1}{1}{L}{A}
    \pin{1}{5}{L}{A1}
    \gate[\inputs{2}]{or}{5}{3}{R}{}{}

    \pin{1}{7}{L}{B}
    \pin{1}{11}{L}{B1}
    \gate[\inputs{2}]{or}{5}{9}{R}{}{}

    % ----  2nd column  ----
    \wire{9}{3}{9}{5}
    \gate[\inputs{2}]{and}{12}{7}{R}{}{}

    % ----  3rd column  ----
    \pin{15}{11}{L}{C}
    \gate[\inputs{2}]{or}{19}{9}{R}{}{}

    \pin{15}{13}{L}{D}
    \pin{15}{17}{L}{D1}
    \gate[\inputs{2}]{or}{19}{15}{R}{}{}

    % ----  4th column  ----
    \wire{23}{9}{23}{11}
    \gate[\inputs{2}]{nand}{26}{13}{R}{}{}

    % ----  result ----
    \pin{31}{13}{R}{Y}

\end{circuitdiagram}
 %%  ************    LibreSilicon's StdCellLibrary   *******************
%%
%%  Organisation:   Chipforge
%%                  Germany / European Union
%%
%%  Profile:        Chipforge focus on fine System-on-Chip Cores in
%%                  Verilog HDL Code which are easy understandable and
%%                  adjustable. For further information see
%%                          www.chipforge.org
%%                  there are projects from small cores up to PCBs, too.
%%
%%  File:           StdCellLib/Documents/Datasheets/Circuitry/OOAOOA2212.tex
%%
%%  Purpose:        Circuit File for OOAOOA2212
%%
%%  ************    LaTeX with circdia.sty package      ***************
%%
%%  ///////////////////////////////////////////////////////////////////
%%
%%  Copyright (c) 2018 - 2022 by
%%                  chipforge <stdcelllib@nospam.chipforge.org>
%%  All rights reserved.
%%
%%      This Standard Cell Library is licensed under the Libre Silicon
%%      public license; you can redistribute it and/or modify it under
%%      the terms of the Libre Silicon public license as published by
%%      the Libre Silicon alliance, either version 1 of the License, or
%%      (at your option) any later version.
%%
%%      This design is distributed in the hope that it will be useful,
%%      but WITHOUT ANY WARRANTY; without even the implied warranty of
%%      MERCHANTABILITY or FITNESS FOR A PARTICULAR PURPOSE.
%%      See the Libre Silicon Public License for more details.
%%
%%  ///////////////////////////////////////////////////////////////////
\begin{circuitdiagram}[draft]{38}{18}

    \usgate
    % ----  1st column  ----
    \pin{1}{1}{L}{A}
    \pin{1}{5}{L}{A1}
    \gate[\inputs{2}]{or}{5}{3}{R}{}{}

    \pin{1}{7}{L}{B}
    \pin{1}{11}{L}{B1}
    \gate[\inputs{2}]{or}{5}{9}{R}{}{}

    % ----  2nd column  ----
    \wire{9}{3}{9}{5}
    \gate[\inputs{2}]{and}{12}{7}{R}{}{}

    % ----  3rd column  ----
    \pin{15}{11}{L}{C}
    \gate[\inputs{2}]{or}{19}{9}{R}{}{}

    \pin{15}{13}{L}{D}
    \pin{15}{17}{L}{D1}
    \gate[\inputs{2}]{or}{19}{15}{R}{}{}

    % ----  4th column  ----
    \wire{23}{9}{23}{11}
    \gate[\inputs{2}]{nand}{26}{13}{R}{}{}

    % ----  4th column  ----
    \gate{not}{33}{13}{R}{}{}

    % ----  result ----
    \pin{37}{13}{R}{Z}

\end{circuitdiagram}

%%  ************    LibreSilicon's StdCellLibrary   *******************
%%
%%  Organisation:   Chipforge
%%                  Germany / European Union
%%
%%  Profile:        Chipforge focus on fine System-on-Chip Cores in
%%                  Verilog HDL Code which are easy understandable and
%%                  adjustable. For further information see
%%                          www.chipforge.org
%%                  there are projects from small cores up to PCBs, too.
%%
%%  File:           StdCellLib/Documents/Datasheets/Circuitry/OOAOOAI2213.tex
%%
%%  Purpose:        Circuit File for OOAOOAI2213
%%
%%  ************    LaTeX with circdia.sty package      ***************
%%
%%  ///////////////////////////////////////////////////////////////////
%%
%%  Copyright (c) 2018 - 2022 by
%%                  chipforge <stdcelllib@nospam.chipforge.org>
%%  All rights reserved.
%%
%%      This Standard Cell Library is licensed under the Libre Silicon
%%      public license; you can redistribute it and/or modify it under
%%      the terms of the Libre Silicon public license as published by
%%      the Libre Silicon alliance, either version 1 of the License, or
%%      (at your option) any later version.
%%
%%      This design is distributed in the hope that it will be useful,
%%      but WITHOUT ANY WARRANTY; without even the implied warranty of
%%      MERCHANTABILITY or FITNESS FOR A PARTICULAR PURPOSE.
%%      See the Libre Silicon Public License for more details.
%%
%%  ///////////////////////////////////////////////////////////////////
\begin{circuitdiagram}[draft]{32}{18}

    \usgate
    % ----  1st column  ----
    \pin{1}{1}{L}{A}
    \pin{1}{5}{L}{A1}
    \gate[\inputs{2}]{or}{5}{3}{R}{}{}

    \pin{1}{7}{L}{B}
    \pin{1}{11}{L}{B1}
    \gate[\inputs{2}]{or}{5}{9}{R}{}{}

    % ----  2nd column  ----
    \wire{9}{3}{9}{5}
    \gate[\inputs{2}]{and}{12}{7}{R}{}{}

    % ----  3rd column  ----
    \pin{15}{11}{L}{C}
    \gate[\inputs{2}]{or}{19}{9}{R}{}{}

    \pin{15}{13}{L}{D}
    \pin{15}{15}{L}{D1}
    \pin{15}{17}{L}{D2}
    \gate[\inputs{3}]{or}{19}{15}{R}{}{}

    % ----  4th column  ----
    \wire{23}{9}{23}{11}
    \gate[\inputs{2}]{nand}{26}{13}{R}{}{}

    % ----  result ----
    \pin{31}{13}{R}{Y}

\end{circuitdiagram}
 %%  ************    LibreSilicon's StdCellLibrary   *******************
%%
%%  Organisation:   Chipforge
%%                  Germany / European Union
%%
%%  Profile:        Chipforge focus on fine System-on-Chip Cores in
%%                  Verilog HDL Code which are easy understandable and
%%                  adjustable. For further information see
%%                          www.chipforge.org
%%                  there are projects from small cores up to PCBs, too.
%%
%%  File:           StdCellLib/Documents/Datasheets/Circuitry/OOAOOA2213.tex
%%
%%  Purpose:        Circuit File for OOAOOA2213
%%
%%  ************    LaTeX with circdia.sty package      ***************
%%
%%  ///////////////////////////////////////////////////////////////////
%%
%%  Copyright (c) 2018 - 2022 by
%%                  chipforge <stdcelllib@nospam.chipforge.org>
%%  All rights reserved.
%%
%%      This Standard Cell Library is licensed under the Libre Silicon
%%      public license; you can redistribute it and/or modify it under
%%      the terms of the Libre Silicon public license as published by
%%      the Libre Silicon alliance, either version 1 of the License, or
%%      (at your option) any later version.
%%
%%      This design is distributed in the hope that it will be useful,
%%      but WITHOUT ANY WARRANTY; without even the implied warranty of
%%      MERCHANTABILITY or FITNESS FOR A PARTICULAR PURPOSE.
%%      See the Libre Silicon Public License for more details.
%%
%%  ///////////////////////////////////////////////////////////////////
\begin{circuitdiagram}[draft]{38}{18}

    \usgate
    % ----  1st column  ----
    \pin{1}{1}{L}{A}
    \pin{1}{5}{L}{A1}
    \gate[\inputs{2}]{or}{5}{3}{R}{}{}

    \pin{1}{7}{L}{B}
    \pin{1}{11}{L}{B1}
    \gate[\inputs{2}]{or}{5}{9}{R}{}{}

    % ----  2nd column  ----
    \wire{9}{3}{9}{5}
    \gate[\inputs{2}]{and}{12}{7}{R}{}{}

    % ----  3rd column  ----
    \pin{15}{11}{L}{C}
    \gate[\inputs{2}]{or}{19}{9}{R}{}{}

    \pin{15}{13}{L}{D}
    \pin{15}{15}{L}{D1}
    \pin{15}{17}{L}{D2}
    \gate[\inputs{3}]{or}{19}{15}{R}{}{}

    % ----  4th column  ----
    \wire{23}{9}{23}{11}
    \gate[\inputs{2}]{nand}{26}{13}{R}{}{}

    % ----  4th column  ----
    \gate{not}{33}{13}{R}{}{}

    % ----  result ----
    \pin{37}{13}{R}{Z}

\end{circuitdiagram}

%%  ************    LibreSilicon's StdCellLibrary   *******************
%%
%%  Organisation:   Chipforge
%%                  Germany / European Union
%%
%%  Profile:        Chipforge focus on fine System-on-Chip Cores in
%%                  Verilog HDL Code which are easy understandable and
%%                  adjustable. For further information see
%%                          www.chipforge.org
%%                  there are projects from small cores up to PCBs, too.
%%
%%  File:           StdCellLib/Documents/Datasheets/Circuitry/OOAOOAI2214.tex
%%
%%  Purpose:        Circuit File for OOAOOAI2214
%%
%%  ************    LaTeX with circdia.sty package      ***************
%%
%%  ///////////////////////////////////////////////////////////////////
%%
%%  Copyright (c) 2018 - 2022 by
%%                  chipforge <stdcelllib@nospam.chipforge.org>
%%  All rights reserved.
%%
%%      This Standard Cell Library is licensed under the Libre Silicon
%%      public license; you can redistribute it and/or modify it under
%%      the terms of the Libre Silicon public license as published by
%%      the Libre Silicon alliance, either version 1 of the License, or
%%      (at your option) any later version.
%%
%%      This design is distributed in the hope that it will be useful,
%%      but WITHOUT ANY WARRANTY; without even the implied warranty of
%%      MERCHANTABILITY or FITNESS FOR A PARTICULAR PURPOSE.
%%      See the Libre Silicon Public License for more details.
%%
%%  ///////////////////////////////////////////////////////////////////
\begin{circuitdiagram}[draft]{32}{20}

    \usgate
    % ----  1st column  ----
    \pin{1}{1}{L}{A}
    \pin{1}{5}{L}{A1}
    \gate[\inputs{2}]{or}{5}{3}{R}{}{}

    \pin{1}{7}{L}{B}
    \pin{1}{11}{L}{B1}
    \gate[\inputs{2}]{or}{5}{9}{R}{}{}

    % ----  2nd column  ----
    \wire{9}{3}{9}{5}
    \gate[\inputs{2}]{and}{12}{7}{R}{}{}

    % ----  3rd column  ----
    \pin{15}{11}{L}{C}
    \gate[\inputs{2}]{or}{19}{9}{R}{}{}

    \pin{15}{13}{L}{D}
    \pin{15}{15}{L}{D1}
    \pin{15}{17}{L}{D2}
    \pin{15}{19}{L}{D3}
    \gate[\inputs{4}]{or}{19}{16}{R}{}{}

    % ----  4th column  ----
    \wire{23}{9}{23}{12}
    \gate[\inputs{2}]{nand}{26}{14}{R}{}{}

    % ----  result ----
    \pin{31}{14}{R}{Y}

\end{circuitdiagram}
 %%  ************    LibreSilicon's StdCellLibrary   *******************
%%
%%  Organisation:   Chipforge
%%                  Germany / European Union
%%
%%  Profile:        Chipforge focus on fine System-on-Chip Cores in
%%                  Verilog HDL Code which are easy understandable and
%%                  adjustable. For further information see
%%                          www.chipforge.org
%%                  there are projects from small cores up to PCBs, too.
%%
%%  File:           StdCellLib/Documents/Datasheets/Circuitry/OOAOOA2214.tex
%%
%%  Purpose:        Circuit File for OOAOOA2214
%%
%%  ************    LaTeX with circdia.sty package      ***************
%%
%%  ///////////////////////////////////////////////////////////////////
%%
%%  Copyright (c) 2018 - 2022 by
%%                  chipforge <stdcelllib@nospam.chipforge.org>
%%  All rights reserved.
%%
%%      This Standard Cell Library is licensed under the Libre Silicon
%%      public license; you can redistribute it and/or modify it under
%%      the terms of the Libre Silicon public license as published by
%%      the Libre Silicon alliance, either version 1 of the License, or
%%      (at your option) any later version.
%%
%%      This design is distributed in the hope that it will be useful,
%%      but WITHOUT ANY WARRANTY; without even the implied warranty of
%%      MERCHANTABILITY or FITNESS FOR A PARTICULAR PURPOSE.
%%      See the Libre Silicon Public License for more details.
%%
%%  ///////////////////////////////////////////////////////////////////
\begin{circuitdiagram}[draft]{38}{20}

    \usgate
    % ----  1st column  ----
    \pin{1}{1}{L}{A}
    \pin{1}{5}{L}{A1}
    \gate[\inputs{2}]{or}{5}{3}{R}{}{}

    \pin{1}{7}{L}{B}
    \pin{1}{11}{L}{B1}
    \gate[\inputs{2}]{or}{5}{9}{R}{}{}

    % ----  2nd column  ----
    \wire{9}{3}{9}{5}
    \gate[\inputs{2}]{and}{12}{7}{R}{}{}

    % ----  3rd column  ----
    \pin{15}{11}{L}{C}
    \gate[\inputs{2}]{or}{19}{9}{R}{}{}

    \pin{15}{13}{L}{D}
    \pin{15}{15}{L}{D1}
    \pin{15}{17}{L}{D2}
    \pin{15}{19}{L}{D3}
    \gate[\inputs{4}]{or}{19}{16}{R}{}{}

    % ----  4th column  ----
    \wire{23}{9}{23}{12}
    \gate[\inputs{2}]{nand}{26}{14}{R}{}{}

    % ----  4th column  ----
    \gate{not}{33}{14}{R}{}{}

    % ----  result ----
    \pin{37}{14}{R}{Z}

\end{circuitdiagram}

%%  ************    LibreSilicon's StdCellLibrary   *******************
%%
%%  Organisation:   Chipforge
%%                  Germany / European Union
%%
%%  Profile:        Chipforge focus on fine System-on-Chip Cores in
%%                  Verilog HDL Code which are easy understandable and
%%                  adjustable. For further information see
%%                          www.chipforge.org
%%                  there are projects from small cores up to PCBs, too.
%%
%%  File:           StdCellLib/Documents/Datasheets/Circuitry/OOAOOAI2222.tex
%%
%%  Purpose:        Circuit File for OOAOOAI2222
%%
%%  ************    LaTeX with circdia.sty package      ***************
%%
%%  ///////////////////////////////////////////////////////////////////
%%
%%  Copyright (c) 2018 - 2022 by
%%                  chipforge <stdcelllib@nospam.chipforge.org>
%%  All rights reserved.
%%
%%      This Standard Cell Library is licensed under the Libre Silicon
%%      public license; you can redistribute it and/or modify it under
%%      the terms of the Libre Silicon public license as published by
%%      the Libre Silicon alliance, either version 1 of the License, or
%%      (at your option) any later version.
%%
%%      This design is distributed in the hope that it will be useful,
%%      but WITHOUT ANY WARRANTY; without even the implied warranty of
%%      MERCHANTABILITY or FITNESS FOR A PARTICULAR PURPOSE.
%%      See the Libre Silicon Public License for more details.
%%
%%  ///////////////////////////////////////////////////////////////////
\begin{circuitdiagram}[draft]{32}{20}

    \usgate
    % ----  1st column  ----
    \pin{1}{1}{L}{A}
    \pin{1}{5}{L}{A1}
    \gate[\inputs{2}]{or}{5}{3}{R}{}{}

    \pin{1}{7}{L}{B}
    \pin{1}{11}{L}{B1}
    \gate[\inputs{2}]{or}{5}{9}{R}{}{}

    % ----  2nd column  ----
    \wire{9}{3}{9}{5}
    \gate[\inputs{2}]{and}{12}{7}{R}{}{}

    % ----  3rd column  ----
    \wire{16}{7}{16}{9}
    \pin{15}{11}{L}{C}
    \pin{15}{13}{L}{C1}
    \gate[\inputs{3}]{or}{19}{11}{R}{}{}

    \pin{15}{15}{L}{D}
    \pin{15}{19}{L}{D1}
    \gate[\inputs{2}]{or}{19}{17}{R}{}{}

    % ----  4th column  ----
    \wire{23}{11}{23}{13}
    \gate[\inputs{2}]{nand}{26}{15}{R}{}{}

    % ----  result ----
    \pin{31}{15}{R}{Y}

\end{circuitdiagram}
 %%  ************    LibreSilicon's StdCellLibrary   *******************
%%
%%  Organisation:   Chipforge
%%                  Germany / European Union
%%
%%  Profile:        Chipforge focus on fine System-on-Chip Cores in
%%                  Verilog HDL Code which are easy understandable and
%%                  adjustable. For further information see
%%                          www.chipforge.org
%%                  there are projects from small cores up to PCBs, too.
%%
%%  File:           StdCellLib/Documents/Datasheets/Circuitry/OOAOOA2222.tex
%%
%%  Purpose:        Circuit File for OOAOOA2222
%%
%%  ************    LaTeX with circdia.sty package      ***************
%%
%%  ///////////////////////////////////////////////////////////////////
%%
%%  Copyright (c) 2018 - 2022 by
%%                  chipforge <stdcelllib@nospam.chipforge.org>
%%  All rights reserved.
%%
%%      This Standard Cell Library is licensed under the Libre Silicon
%%      public license; you can redistribute it and/or modify it under
%%      the terms of the Libre Silicon public license as published by
%%      the Libre Silicon alliance, either version 1 of the License, or
%%      (at your option) any later version.
%%
%%      This design is distributed in the hope that it will be useful,
%%      but WITHOUT ANY WARRANTY; without even the implied warranty of
%%      MERCHANTABILITY or FITNESS FOR A PARTICULAR PURPOSE.
%%      See the Libre Silicon Public License for more details.
%%
%%  ///////////////////////////////////////////////////////////////////
\begin{circuitdiagram}[draft]{38}{20}

    \usgate
    % ----  1st column  ----
    \pin{1}{1}{L}{A}
    \pin{1}{5}{L}{A1}
    \gate[\inputs{2}]{or}{5}{3}{R}{}{}

    \pin{1}{7}{L}{B}
    \pin{1}{11}{L}{B1}
    \gate[\inputs{2}]{or}{5}{9}{R}{}{}

    % ----  2nd column  ----
    \wire{9}{3}{9}{5}
    \gate[\inputs{2}]{and}{12}{7}{R}{}{}

    % ----  3rd column  ----
    \wire{16}{7}{16}{9}
    \pin{15}{11}{L}{C}
    \pin{15}{13}{L}{C1}
    \gate[\inputs{3}]{or}{19}{11}{R}{}{}

    \pin{15}{15}{L}{D}
    \pin{15}{19}{L}{D1}
    \gate[\inputs{2}]{or}{19}{17}{R}{}{}

    % ----  4th column  ----
    \wire{23}{11}{23}{13}
    \gate[\inputs{2}]{nand}{26}{15}{R}{}{}

    % ----  last column ----
    \gate{not}{33}{15}{R}{}{}

    % ----  result ----
    \pin{37}{15}{R}{Z}

\end{circuitdiagram}

%%  ************    LibreSilicon's StdCellLibrary   *******************
%%
%%  Organisation:   Chipforge
%%                  Germany / European Union
%%
%%  Profile:        Chipforge focus on fine System-on-Chip Cores in
%%                  Verilog HDL Code which are easy understandable and
%%                  adjustable. For further information see
%%                          www.chipforge.org
%%                  there are projects from small cores up to PCBs, too.
%%
%%  File:           StdCellLib/Documents/Datasheets/Circuitry/OOAOOAI3212.tex
%%
%%  Purpose:        Circuit File for OOAOOAI3212
%%
%%  ************    LaTeX with circdia.sty package      ***************
%%
%%  ///////////////////////////////////////////////////////////////////
%%
%%  Copyright (c) 2018 - 2022 by
%%                  chipforge <stdcelllib@nospam.chipforge.org>
%%  All rights reserved.
%%
%%      This Standard Cell Library is licensed under the Libre Silicon
%%      public license; you can redistribute it and/or modify it under
%%      the terms of the Libre Silicon public license as published by
%%      the Libre Silicon alliance, either version 1 of the License, or
%%      (at your option) any later version.
%%
%%      This design is distributed in the hope that it will be useful,
%%      but WITHOUT ANY WARRANTY; without even the implied warranty of
%%      MERCHANTABILITY or FITNESS FOR A PARTICULAR PURPOSE.
%%      See the Libre Silicon Public License for more details.
%%
%%  ///////////////////////////////////////////////////////////////////
\begin{circuitdiagram}[draft]{32}{18}

    \usgate
    % ----  1st column  ----
    \pin{1}{1}{L}{A}
    \pin{1}{3}{L}{A1}
    \pin{1}{5}{L}{A2}
    \gate[\inputs{3}]{or}{5}{3}{R}{}{}

    \pin{1}{7}{L}{B}
    \pin{1}{11}{L}{B1}
    \gate[\inputs{2}]{or}{5}{9}{R}{}{}

    % ----  2nd column  ----
    \wire{9}{3}{9}{5}
    \gate[\inputs{2}]{and}{12}{7}{R}{}{}

    % ----  3rd column  ----
    \pin{15}{11}{L}{C}
    \gate[\inputs{2}]{or}{19}{9}{R}{}{}

    \pin{15}{13}{L}{D}
    \pin{15}{17}{L}{D1}
    \gate[\inputs{2}]{or}{19}{15}{R}{}{}

    % ----  4th column  ----
    \wire{23}{9}{23}{11}
    \gate[\inputs{2}]{nand}{26}{13}{R}{}{}

    % ----  result ----
    \pin{31}{13}{R}{Y}

\end{circuitdiagram}
 %%  ************    LibreSilicon's StdCellLibrary   *******************
%%
%%  Organisation:   Chipforge
%%                  Germany / European Union
%%
%%  Profile:        Chipforge focus on fine System-on-Chip Cores in
%%                  Verilog HDL Code which are easy understandable and
%%                  adjustable. For further information see
%%                          www.chipforge.org
%%                  there are projects from small cores up to PCBs, too.
%%
%%  File:           StdCellLib/Documents/Datasheets/Circuitry/OOAOOA3212.tex
%%
%%  Purpose:        Circuit File for OOAOOA3212
%%
%%  ************    LaTeX with circdia.sty package      ***************
%%
%%  ///////////////////////////////////////////////////////////////////
%%
%%  Copyright (c) 2018 - 2022 by
%%                  chipforge <stdcelllib@nospam.chipforge.org>
%%  All rights reserved.
%%
%%      This Standard Cell Library is licensed under the Libre Silicon
%%      public license; you can redistribute it and/or modify it under
%%      the terms of the Libre Silicon public license as published by
%%      the Libre Silicon alliance, either version 1 of the License, or
%%      (at your option) any later version.
%%
%%      This design is distributed in the hope that it will be useful,
%%      but WITHOUT ANY WARRANTY; without even the implied warranty of
%%      MERCHANTABILITY or FITNESS FOR A PARTICULAR PURPOSE.
%%      See the Libre Silicon Public License for more details.
%%
%%  ///////////////////////////////////////////////////////////////////
\begin{circuitdiagram}[draft]{38}{18}

    \usgate
    % ----  1st column  ----
    \pin{1}{1}{L}{A}
    \pin{1}{3}{L}{A1}
    \pin{1}{5}{L}{A2}
    \gate[\inputs{3}]{or}{5}{3}{R}{}{}

    \pin{1}{7}{L}{B}
    \pin{1}{11}{L}{B1}
    \gate[\inputs{2}]{or}{5}{9}{R}{}{}

    % ----  2nd column  ----
    \wire{9}{3}{9}{5}
    \gate[\inputs{2}]{and}{12}{7}{R}{}{}

    % ----  3rd column  ----
    \pin{15}{11}{L}{C}
    \gate[\inputs{2}]{or}{19}{9}{R}{}{}

    \pin{15}{13}{L}{D}
    \pin{15}{17}{L}{D1}
    \gate[\inputs{2}]{or}{19}{15}{R}{}{}

    % ----  4th column  ----
    \wire{23}{9}{23}{11}
    \gate[\inputs{2}]{nand}{26}{13}{R}{}{}

    % ----  last column ----
    \gate{not}{33}{13}{R}{}{}

    % ----  result ----
    \pin{37}{13}{R}{Z}

\end{circuitdiagram}


%%  ************    LibreSilicon's StdCellLibrary   *******************
%%
%%  Organisation:   Chipforge
%%                  Germany / European Union
%%
%%  Profile:        Chipforge focus on fine System-on-Chip Cores in
%%                  Verilog HDL Code which are easy understandable and
%%                  adjustable. For further information see
%%                          www.chipforge.org
%%                  there are projects from small cores up to PCBs, too.
%%
%%  File:           StdCellLib/Documents/Datasheets/Circuitry/OOAOOAI22112.tex
%%
%%  Purpose:        Circuit File for OOAOOAI22112
%%
%%  ************    LaTeX with circdia.sty package      ***************
%%
%%  ///////////////////////////////////////////////////////////////////
%%
%%  Copyright (c) 2018 - 2022 by
%%                  chipforge <stdcelllib@nospam.chipforge.org>
%%  All rights reserved.
%%
%%      This Standard Cell Library is licensed under the Libre Silicon
%%      public license; you can redistribute it and/or modify it under
%%      the terms of the Libre Silicon public license as published by
%%      the Libre Silicon alliance, either version 1 of the License, or
%%      (at your option) any later version.
%%
%%      This design is distributed in the hope that it will be useful,
%%      but WITHOUT ANY WARRANTY; without even the implied warranty of
%%      MERCHANTABILITY or FITNESS FOR A PARTICULAR PURPOSE.
%%      See the Libre Silicon Public License for more details.
%%
%%  ///////////////////////////////////////////////////////////////////
\begin{circuitdiagram}[draft]{32}{20}

    \usgate
    % ----  1st column  ----
    \pin{1}{1}{L}{A}
    \pin{1}{5}{L}{A1}
    \gate[\inputs{2}]{or}{5}{3}{R}{}{}

    \pin{1}{7}{L}{B}
    \pin{1}{11}{L}{B1}
    \gate[\inputs{2}]{or}{5}{9}{R}{}{}

    % ----  2nd column  ----
    \wire{9}{3}{9}{7}
    \pin{8}{13}{L}{C}
    \wire{9}{11}{9}{13}
    \gate[\inputs{3}]{and}{12}{9}{R}{}{}

    % ----  3rd column  ----
    \pin{15}{13}{L}{D}
    \gate[\inputs{2}]{or}{19}{11}{R}{}{}

    \pin{15}{15}{L}{E}
    \pin{15}{19}{L}{E1}
    \gate[\inputs{2}]{or}{19}{17}{R}{}{}

    % ----  4th column  ----
    \wire{23}{11}{23}{13}
    \gate[\inputs{2}]{nand}{26}{15}{R}{}{}

    % ----  result ----
    \pin{31}{15}{R}{Y}

\end{circuitdiagram}
 %%  ************    LibreSilicon's StdCellLibrary   *******************
%%
%%  Organisation:   Chipforge
%%                  Germany / European Union
%%
%%  Profile:        Chipforge focus on fine System-on-Chip Cores in
%%                  Verilog HDL Code which are easy understandable and
%%                  adjustable. For further information see
%%                          www.chipforge.org
%%                  there are projects from small cores up to PCBs, too.
%%
%%  File:           StdCellLib/Documents/Datasheets/Circuitry/OOAOOA22112.tex
%%
%%  Purpose:        Circuit File for OOAOOA22112
%%
%%  ************    LaTeX with circdia.sty package      ***************
%%
%%  ///////////////////////////////////////////////////////////////////
%%
%%  Copyright (c) 2018 - 2022 by
%%                  chipforge <stdcelllib@nospam.chipforge.org>
%%  All rights reserved.
%%
%%      This Standard Cell Library is licensed under the Libre Silicon
%%      public license; you can redistribute it and/or modify it under
%%      the terms of the Libre Silicon public license as published by
%%      the Libre Silicon alliance, either version 1 of the License, or
%%      (at your option) any later version.
%%
%%      This design is distributed in the hope that it will be useful,
%%      but WITHOUT ANY WARRANTY; without even the implied warranty of
%%      MERCHANTABILITY or FITNESS FOR A PARTICULAR PURPOSE.
%%      See the Libre Silicon Public License for more details.
%%
%%  ///////////////////////////////////////////////////////////////////
\begin{circuitdiagram}[draft]{38}{20}

    \usgate
    % ----  1st column  ----
    \pin{1}{1}{L}{A}
    \pin{1}{5}{L}{A1}
    \gate[\inputs{2}]{or}{5}{3}{R}{}{}

    \pin{1}{7}{L}{B}
    \pin{1}{11}{L}{B1}
    \gate[\inputs{2}]{or}{5}{9}{R}{}{}

    % ----  2nd column  ----
    \wire{9}{3}{9}{7}
    \pin{8}{13}{L}{C}
    \wire{9}{11}{9}{13}
    \gate[\inputs{3}]{and}{12}{9}{R}{}{}

    % ----  3rd column  ----
    \pin{15}{13}{L}{D}
    \gate[\inputs{2}]{or}{19}{11}{R}{}{}

    \pin{15}{15}{L}{E}
    \pin{15}{19}{L}{E1}
    \gate[\inputs{2}]{or}{19}{17}{R}{}{}

    % ----  4th column  ----
    \wire{23}{11}{23}{13}
    \gate[\inputs{2}]{nand}{26}{15}{R}{}{}

    % ----  last column ----
    \gate{not}{33}{15}{R}{}{}

    % ----  result ----
    \pin{37}{15}{R}{Z}

\end{circuitdiagram}

%%  ************    LibreSilicon's StdCellLibrary   *******************
%%
%%  Organisation:   Chipforge
%%                  Germany / European Union
%%
%%  Profile:        Chipforge focus on fine System-on-Chip Cores in
%%                  Verilog HDL Code which are easy understandable and
%%                  adjustable. For further information see
%%                          www.chipforge.org
%%                  there are projects from small cores up to PCBs, too.
%%
%%  File:           StdCellLib/Documents/Datasheets/Circuitry/OOAOOAI22121.tex
%%
%%  Purpose:        Circuit File for OOAOOAI22121
%%
%%  ************    LaTeX with circdia.sty package      ***************
%%
%%  ///////////////////////////////////////////////////////////////////
%%
%%  Copyright (c) 2018 - 2022 by
%%                  chipforge <stdcelllib@nospam.chipforge.org>
%%  All rights reserved.
%%
%%      This Standard Cell Library is licensed under the Libre Silicon
%%      public license; you can redistribute it and/or modify it under
%%      the terms of the Libre Silicon public license as published by
%%      the Libre Silicon alliance, either version 1 of the License, or
%%      (at your option) any later version.
%%
%%      This design is distributed in the hope that it will be useful,
%%      but WITHOUT ANY WARRANTY; without even the implied warranty of
%%      MERCHANTABILITY or FITNESS FOR A PARTICULAR PURPOSE.
%%      See the Libre Silicon Public License for more details.
%%
%%  ///////////////////////////////////////////////////////////////////
\begin{circuitdiagram}[draft]{32}{20}

    \usgate
    % ----  1st column  ----
    \pin{1}{1}{L}{A}
    \pin{1}{5}{L}{A1}
    \gate[\inputs{2}]{or}{5}{3}{R}{}{}

    \pin{1}{7}{L}{B}
    \pin{1}{11}{L}{B1}
    \gate[\inputs{2}]{or}{5}{9}{R}{}{}

    % ----  2nd column  ----
    \wire{9}{3}{9}{5}
    \gate[\inputs{2}]{and}{12}{7}{R}{}{}

    % ----  3rd column  ----
    \pin{15}{11}{L}{C}
    \gate[\inputs{2}]{or}{19}{9}{R}{}{}

    \pin{15}{13}{L}{D}
    \pin{15}{17}{L}{D1}
    \gate[\inputs{2}]{or}{19}{15}{R}{}{}

    % ----  4th column  ----
    \wire{23}{9}{23}{13}
    \pin{22}{19}{L}{E}
    \wire{23}{17}{23}{19}
    \gate[\inputs{3}]{nand}{26}{15}{R}{}{}

    % ----  result ----
    \pin{31}{15}{R}{Y}

\end{circuitdiagram}
 %%  ************    LibreSilicon's StdCellLibrary   *******************
%%
%%  Organisation:   Chipforge
%%                  Germany / European Union
%%
%%  Profile:        Chipforge focus on fine System-on-Chip Cores in
%%                  Verilog HDL Code which are easy understandable and
%%                  adjustable. For further information see
%%                          www.chipforge.org
%%                  there are projects from small cores up to PCBs, too.
%%
%%  File:           StdCellLib/Documents/Datasheets/Circuitry/OOAOOA22121.tex
%%
%%  Purpose:        Circuit File for OOAOOA22121
%%
%%  ************    LaTeX with circdia.sty package      ***************
%%
%%  ///////////////////////////////////////////////////////////////////
%%
%%  Copyright (c) 2018 - 2022 by
%%                  chipforge <stdcelllib@nospam.chipforge.org>
%%  All rights reserved.
%%
%%      This Standard Cell Library is licensed under the Libre Silicon
%%      public license; you can redistribute it and/or modify it under
%%      the terms of the Libre Silicon public license as published by
%%      the Libre Silicon alliance, either version 1 of the License, or
%%      (at your option) any later version.
%%
%%      This design is distributed in the hope that it will be useful,
%%      but WITHOUT ANY WARRANTY; without even the implied warranty of
%%      MERCHANTABILITY or FITNESS FOR A PARTICULAR PURPOSE.
%%      See the Libre Silicon Public License for more details.
%%
%%  ///////////////////////////////////////////////////////////////////
\begin{circuitdiagram}[draft]{38}{20}

    \usgate
    % ----  1st column  ----
    \pin{1}{1}{L}{A}
    \pin{1}{5}{L}{A1}
    \gate[\inputs{2}]{or}{5}{3}{R}{}{}

    \pin{1}{7}{L}{B}
    \pin{1}{11}{L}{B1}
    \gate[\inputs{2}]{or}{5}{9}{R}{}{}

    % ----  2nd column  ----
    \wire{9}{3}{9}{5}
    \gate[\inputs{2}]{and}{12}{7}{R}{}{}

    % ----  3rd column  ----
    \pin{15}{11}{L}{C}
    \gate[\inputs{2}]{or}{19}{9}{R}{}{}

    \pin{15}{13}{L}{D}
    \pin{15}{17}{L}{D1}
    \gate[\inputs{2}]{or}{19}{15}{R}{}{}

    % ----  4th column  ----
    \wire{23}{9}{23}{13}
    \pin{22}{19}{L}{E}
    \wire{23}{17}{23}{19}
    \gate[\inputs{3}]{nand}{26}{15}{R}{}{}

    % ----  4th column  ----
    \gate{not}{33}{15}{R}{}{}

    % ----  result ----
    \pin{37}{15}{R}{Z}

\end{circuitdiagram}

%%  ************    LibreSilicon's StdCellLibrary   *******************
%%
%%  Organisation:   Chipforge
%%                  Germany / European Union
%%
%%  Profile:        Chipforge focus on fine System-on-Chip Cores in
%%                  Verilog HDL Code which are easy understandable and
%%                  adjustable. For further information see
%%                          www.chipforge.org
%%                  there are projects from small cores up to PCBs, too.
%%
%%  File:           StdCellLib/Documents/Datasheets/Circuitry/OOAOOAI22131.tex
%%
%%  Purpose:        Circuit File for OOAOOAI22131
%%
%%  ************    LaTeX with circdia.sty package      ***************
%%
%%  ///////////////////////////////////////////////////////////////////
%%
%%  Copyright (c) 2018 - 2022 by
%%                  chipforge <stdcelllib@nospam.chipforge.org>
%%  All rights reserved.
%%
%%      This Standard Cell Library is licensed under the Libre Silicon
%%      public license; you can redistribute it and/or modify it under
%%      the terms of the Libre Silicon public license as published by
%%      the Libre Silicon alliance, either version 1 of the License, or
%%      (at your option) any later version.
%%
%%      This design is distributed in the hope that it will be useful,
%%      but WITHOUT ANY WARRANTY; without even the implied warranty of
%%      MERCHANTABILITY or FITNESS FOR A PARTICULAR PURPOSE.
%%      See the Libre Silicon Public License for more details.
%%
%%  ///////////////////////////////////////////////////////////////////
\begin{circuitdiagram}[draft]{32}{20}

    \usgate
    % ----  1st column  ----
    \pin{1}{1}{L}{A}
    \pin{1}{5}{L}{A1}
    \gate[\inputs{2}]{or}{5}{3}{R}{}{}

    \pin{1}{7}{L}{B}
    \pin{1}{11}{L}{B1}
    \gate[\inputs{2}]{or}{5}{9}{R}{}{}

    % ----  2nd column  ----
    \wire{9}{3}{9}{5}
    \gate[\inputs{2}]{and}{12}{7}{R}{}{}

    % ----  3rd column  ----
    \pin{15}{11}{L}{C}
    \gate[\inputs{2}]{or}{19}{9}{R}{}{}

    \pin{15}{13}{L}{D}
    \pin{15}{15}{L}{D1}
    \pin{15}{17}{L}{D2}
    \gate[\inputs{3}]{or}{19}{15}{R}{}{}

    % ----  4th column  ----
    \wire{23}{9}{23}{13}
    \pin{22}{19}{L}{E}
    \wire{23}{17}{23}{19}
    \gate[\inputs{3}]{nand}{26}{15}{R}{}{}

    % ----  result ----
    \pin{31}{15}{R}{Y}

\end{circuitdiagram}
 %%  ************    LibreSilicon's StdCellLibrary   *******************
%%
%%  Organisation:   Chipforge
%%                  Germany / European Union
%%
%%  Profile:        Chipforge focus on fine System-on-Chip Cores in
%%                  Verilog HDL Code which are easy understandable and
%%                  adjustable. For further information see
%%                          www.chipforge.org
%%                  there are projects from small cores up to PCBs, too.
%%
%%  File:           StdCellLib/Documents/Datasheets/Circuitry/OOAOOA22131.tex
%%
%%  Purpose:        Circuit File for OOAOOA22131
%%
%%  ************    LaTeX with circdia.sty package      ***************
%%
%%  ///////////////////////////////////////////////////////////////////
%%
%%  Copyright (c) 2018 - 2022 by
%%                  chipforge <stdcelllib@nospam.chipforge.org>
%%  All rights reserved.
%%
%%      This Standard Cell Library is licensed under the Libre Silicon
%%      public license; you can redistribute it and/or modify it under
%%      the terms of the Libre Silicon public license as published by
%%      the Libre Silicon alliance, either version 1 of the License, or
%%      (at your option) any later version.
%%
%%      This design is distributed in the hope that it will be useful,
%%      but WITHOUT ANY WARRANTY; without even the implied warranty of
%%      MERCHANTABILITY or FITNESS FOR A PARTICULAR PURPOSE.
%%      See the Libre Silicon Public License for more details.
%%
%%  ///////////////////////////////////////////////////////////////////
\begin{circuitdiagram}[draft]{38}{20}

    \usgate
    % ----  1st column  ----
    \pin{1}{1}{L}{A}
    \pin{1}{5}{L}{A1}
    \gate[\inputs{2}]{or}{5}{3}{R}{}{}

    \pin{1}{7}{L}{B}
    \pin{1}{11}{L}{B1}
    \gate[\inputs{2}]{or}{5}{9}{R}{}{}

    % ----  2nd column  ----
    \wire{9}{3}{9}{5}
    \gate[\inputs{2}]{and}{12}{7}{R}{}{}

    % ----  3rd column  ----
    \pin{15}{11}{L}{C}
    \gate[\inputs{2}]{or}{19}{9}{R}{}{}

    \pin{15}{13}{L}{D}
    \pin{15}{15}{L}{D1}
    \pin{15}{17}{L}{D2}
    \gate[\inputs{3}]{or}{19}{15}{R}{}{}

    % ----  4th column  ----
    \wire{23}{9}{23}{13}
    \pin{22}{19}{L}{E}
    \wire{23}{17}{23}{19}
    \gate[\inputs{3}]{nand}{26}{15}{R}{}{}

    % ----  4th column  ----
    \gate{not}{33}{15}{R}{}{}

    % ----  result ----
    \pin{37}{15}{R}{Z}

\end{circuitdiagram}

%%  ************    LibreSilicon's StdCellLibrary   *******************
%%
%%  Organisation:   Chipforge
%%                  Germany / European Union
%%
%%  Profile:        Chipforge focus on fine System-on-Chip Cores in
%%                  Verilog HDL Code which are easy understandable and
%%                  adjustable. For further information see
%%                          www.chipforge.org
%%                  there are projects from small cores up to PCBs, too.
%%
%%  File:           StdCellLib/Documents/Datasheets/Circuitry/OOAOOAI22141.tex
%%
%%  Purpose:        Circuit File for OOAOOAI22141
%%
%%  ************    LaTeX with circdia.sty package      ***************
%%
%%  ///////////////////////////////////////////////////////////////////
%%
%%  Copyright (c) 2018 - 2022 by
%%                  chipforge <stdcelllib@nospam.chipforge.org>
%%  All rights reserved.
%%
%%      This Standard Cell Library is licensed under the Libre Silicon
%%      public license; you can redistribute it and/or modify it under
%%      the terms of the Libre Silicon public license as published by
%%      the Libre Silicon alliance, either version 1 of the License, or
%%      (at your option) any later version.
%%
%%      This design is distributed in the hope that it will be useful,
%%      but WITHOUT ANY WARRANTY; without even the implied warranty of
%%      MERCHANTABILITY or FITNESS FOR A PARTICULAR PURPOSE.
%%      See the Libre Silicon Public License for more details.
%%
%%  ///////////////////////////////////////////////////////////////////
\begin{circuitdiagram}[draft]{32}{21}

    \usgate
    % ----  1st column  ----
    \pin{1}{1}{L}{A}
    \pin{1}{5}{L}{A1}
    \gate[\inputs{2}]{or}{5}{3}{R}{}{}

    \pin{1}{7}{L}{B}
    \pin{1}{11}{L}{B1}
    \gate[\inputs{2}]{or}{5}{9}{R}{}{}

    % ----  2nd column  ----
    \wire{9}{3}{9}{5}
    \gate[\inputs{2}]{and}{12}{7}{R}{}{}

    % ----  3rd column  ----
    \pin{15}{11}{L}{C}
    \gate[\inputs{2}]{or}{19}{9}{R}{}{}

    \pin{15}{13}{L}{D}
    \pin{15}{15}{L}{D1}
    \pin{15}{17}{L}{D2}
    \pin{15}{19}{L}{D3}
    \gate[\inputs{4}]{or}{19}{16}{R}{}{}

    % ----  4th column  ----
    \wire{23}{9}{23}{14}
    \pin{22}{20}{L}{E}
    \wire{23}{18}{23}{20}
    \gate[\inputs{3}]{nand}{26}{16}{R}{}{}

    % ----  result ----
    \pin{31}{16}{R}{Y}

\end{circuitdiagram}
 %%  ************    LibreSilicon's StdCellLibrary   *******************
%%
%%  Organisation:   Chipforge
%%                  Germany / European Union
%%
%%  Profile:        Chipforge focus on fine System-on-Chip Cores in
%%                  Verilog HDL Code which are easy understandable and
%%                  adjustable. For further information see
%%                          www.chipforge.org
%%                  there are projects from small cores up to PCBs, too.
%%
%%  File:           StdCellLib/Documents/Datasheets/Circuitry/OOAOOA22141.tex
%%
%%  Purpose:        Circuit File for OOAOOA22141
%%
%%  ************    LaTeX with circdia.sty package      ***************
%%
%%  ///////////////////////////////////////////////////////////////////
%%
%%  Copyright (c) 2018 - 2022 by
%%                  chipforge <stdcelllib@nospam.chipforge.org>
%%  All rights reserved.
%%
%%      This Standard Cell Library is licensed under the Libre Silicon
%%      public license; you can redistribute it and/or modify it under
%%      the terms of the Libre Silicon public license as published by
%%      the Libre Silicon alliance, either version 1 of the License, or
%%      (at your option) any later version.
%%
%%      This design is distributed in the hope that it will be useful,
%%      but WITHOUT ANY WARRANTY; without even the implied warranty of
%%      MERCHANTABILITY or FITNESS FOR A PARTICULAR PURPOSE.
%%      See the Libre Silicon Public License for more details.
%%
%%  ///////////////////////////////////////////////////////////////////
\begin{circuitdiagram}[draft]{38}{21}

    \usgate
    % ----  1st column  ----
    \pin{1}{1}{L}{A}
    \pin{1}{5}{L}{A1}
    \gate[\inputs{2}]{or}{5}{3}{R}{}{}

    \pin{1}{7}{L}{B}
    \pin{1}{11}{L}{B1}
    \gate[\inputs{2}]{or}{5}{9}{R}{}{}

    % ----  2nd column  ----
    \wire{9}{3}{9}{5}
    \gate[\inputs{2}]{and}{12}{7}{R}{}{}

    % ----  3rd column  ----
    \pin{15}{11}{L}{C}
    \gate[\inputs{2}]{or}{19}{9}{R}{}{}

    \pin{15}{13}{L}{D}
    \pin{15}{15}{L}{D1}
    \pin{15}{17}{L}{D2}
    \pin{15}{19}{L}{D3}
    \gate[\inputs{4}]{or}{19}{16}{R}{}{}

    % ----  4th column  ----
    \wire{23}{9}{23}{14}
    \pin{22}{20}{L}{E}
    \wire{23}{18}{23}{20}
    \gate[\inputs{3}]{nand}{26}{16}{R}{}{}

    % ----  last column ----
    \gate{not}{33}{16}{R}{}{}

    % ----  result ----
    \pin{37}{16}{R}{Z}

\end{circuitdiagram}


%%  ************    LibreSilicon's StdCellLibrary   *******************
%%
%%  Organisation:   Chipforge
%%                  Germany / European Union
%%
%%  Profile:        Chipforge focus on fine System-on-Chip Cores in
%%                  Verilog HDL Code which are easy understandable and
%%                  adjustable. For further information see
%%                          www.chipforge.org
%%                  there are projects from small cores up to PCBs, too.
%%
%%  File:           StdCellLib/Documents/section-AOOAAOI_complex.tex
%%
%%  Purpose:        Section Level File for Standard Cell Library Documentation
%%
%%  ************    LaTeX with circdia.sty package      ***************
%%
%%  ///////////////////////////////////////////////////////////////////
%%
%%  Copyright (c) 2018 - 2022 by
%%                  chipforge <stdcelllib@nospam.chipforge.org>
%%  All rights reserved.
%%
%%      This Standard Cell Library is licensed under the Libre Silicon
%%      public license; you can redistribute it and/or modify it under
%%      the terms of the Libre Silicon public license as published by
%%      the Libre Silicon alliance, either version 1 of the License, or
%%      (at your option) any later version.
%%
%%      This design is distributed in the hope that it will be useful,
%%      but WITHOUT ANY WARRANTY; without even the implied warranty of
%%      MERCHANTABILITY or FITNESS FOR A PARTICULAR PURPOSE.
%%      See the Libre Silicon Public License for more details.
%%
%%  ///////////////////////////////////////////////////////////////////
\section{AND-OR-OR-AND-AND-OR(-Invert) Complex Gates}

%%  ************    LibreSilicon's StdCellLibrary   *******************
%%
%%  Organisation:   Chipforge
%%                  Germany / European Union
%%
%%  Profile:        Chipforge focus on fine System-on-Chip Cores in
%%                  Verilog HDL Code which are easy understandable and
%%                  adjustable. For further information see
%%                          www.chipforge.org
%%                  there are projects from small cores up to PCBs, too.
%%
%%  File:           StdCellLib/Documents/Datasheets/Circuitry/AOOAAOI2122.tex
%%
%%  Purpose:        Circuit File for AOOAAOI2122
%%
%%  ************    LaTeX with circdia.sty package      ***************
%%
%%  ///////////////////////////////////////////////////////////////////
%%
%%  Copyright (c) 2018 - 2022 by
%%                  chipforge <stdcelllib@nospam.chipforge.org>
%%  All rights reserved.
%%
%%      This Standard Cell Library is licensed under the Libre Silicon
%%      public license; you can redistribute it and/or modify it under
%%      the terms of the Libre Silicon public license as published by
%%      the Libre Silicon alliance, either version 1 of the License, or
%%      (at your option) any later version.
%%
%%      This design is distributed in the hope that it will be useful,
%%      but WITHOUT ANY WARRANTY; without even the implied warranty of
%%      MERCHANTABILITY or FITNESS FOR A PARTICULAR PURPOSE.
%%      See the Libre Silicon Public License for more details.
%%
%%  ///////////////////////////////////////////////////////////////////
\begin{circuitdiagram}[draft]{32}{20}

    \usgate
    % ----  1st column  ----
    \pin{1}{1}{L}{A}
    \pin{1}{5}{L}{A1}
    \gate[\inputs{2}]{and}{5}{3}{R}{}{}

    % ----  2nd column  ----
    \pin{8}{7}{L}{B}
    \gate[\inputs{2}]{or}{12}{5}{R}{}{}

    \pin{8}{9}{L}{C}
    \pin{8}{13}{L}{C1}
    \gate[\inputs{2}]{or}{12}{11}{R}{}{}

    % ----  3rd column  ----
    \wire{16}{5}{16}{7}
    \gate[\inputs{2}]{and}{19}{9}{R}{}{}

    \pin{15}{15}{L}{D}
    \pin{15}{19}{L}{D1}
    \gate[\inputs{2}]{and}{19}{17}{R}{}{}

    % ----  4th column  ----
    \wire{23}{9}{23}{13}
    \gate[\inputs{2}]{nor}{26}{15}{R}{}{}

    % ----  result ----
    \pin{31}{15}{R}{Y}

\end{circuitdiagram}
 %%  ************    LibreSilicon's StdCellLibrary   *******************
%%
%%  Organisation:   Chipforge
%%                  Germany / European Union
%%
%%  Profile:        Chipforge focus on fine System-on-Chip Cores in
%%                  Verilog HDL Code which are easy understandable and
%%                  adjustable. For further information see
%%                          www.chipforge.org
%%                  there are projects from small cores up to PCBs, too.
%%
%%  File:           StdCellLib/Documents/Datasheets/Circuitry/AOOAAO2122.tex
%%
%%  Purpose:        Circuit File for AOOAAO2122
%%
%%  ************    LaTeX with circdia.sty package      ***************
%%
%%  ///////////////////////////////////////////////////////////////////
%%
%%  Copyright (c) 2018 - 2022 by
%%                  chipforge <stdcelllib@nospam.chipforge.org>
%%  All rights reserved.
%%
%%      This Standard Cell Library is licensed under the Libre Silicon
%%      public license; you can redistribute it and/or modify it under
%%      the terms of the Libre Silicon public license as published by
%%      the Libre Silicon alliance, either version 1 of the License, or
%%      (at your option) any later version.
%%
%%      This design is distributed in the hope that it will be useful,
%%      but WITHOUT ANY WARRANTY; without even the implied warranty of
%%      MERCHANTABILITY or FITNESS FOR A PARTICULAR PURPOSE.
%%      See the Libre Silicon Public License for more details.
%%
%%  ///////////////////////////////////////////////////////////////////
\begin{circuitdiagram}[draft]{38}{20}

    \usgate
    % ----  1st column  ----
    \pin{1}{1}{L}{A}
    \pin{1}{5}{L}{A1}
    \gate[\inputs{2}]{and}{5}{3}{R}{}{}

    % ----  2nd column  ----
    \pin{8}{7}{L}{B}
    \gate[\inputs{2}]{or}{12}{5}{R}{}{}

    \pin{8}{9}{L}{C}
    \pin{8}{13}{L}{C1}
    \gate[\inputs{2}]{or}{12}{11}{R}{}{}

    % ----  3rd column  ----
    \wire{16}{5}{16}{7}
    \gate[\inputs{2}]{and}{19}{9}{R}{}{}

    \pin{15}{15}{L}{D}
    \pin{15}{19}{L}{D1}
    \gate[\inputs{2}]{and}{19}{17}{R}{}{}

    % ----  4th column  ----
    \wire{23}{9}{23}{13}
    \gate[\inputs{2}]{nor}{26}{15}{R}{}{}

    % ----  5th column  ----
    \gate{not}{33}{15}{R}{}{}

    % ----  result ----
    \pin{37}{15}{R}{Z}

\end{circuitdiagram}

%%  ************    LibreSilicon's StdCellLibrary   *******************
%%
%%  Organisation:   Chipforge
%%                  Germany / European Union
%%
%%  Profile:        Chipforge focus on fine System-on-Chip Cores in
%%                  Verilog HDL Code which are easy understandable and
%%                  adjustable. For further information see
%%                          www.chipforge.org
%%                  there are projects from small cores up to PCBs, too.
%%
%%  File:           StdCellLib/Documents/Datasheets/Circuitry/AOOAAOI2123.tex
%%
%%  Purpose:        Circuit File for AOOAAOI2123
%%
%%  ************    LaTeX with circdia.sty package      ***************
%%
%%  ///////////////////////////////////////////////////////////////////
%%
%%  Copyright (c) 2018 - 2022 by
%%                  chipforge <stdcelllib@nospam.chipforge.org>
%%  All rights reserved.
%%
%%      This Standard Cell Library is licensed under the Libre Silicon
%%      public license; you can redistribute it and/or modify it under
%%      the terms of the Libre Silicon public license as published by
%%      the Libre Silicon alliance, either version 1 of the License, or
%%      (at your option) any later version.
%%
%%      This design is distributed in the hope that it will be useful,
%%      but WITHOUT ANY WARRANTY; without even the implied warranty of
%%      MERCHANTABILITY or FITNESS FOR A PARTICULAR PURPOSE.
%%      See the Libre Silicon Public License for more details.
%%
%%  ///////////////////////////////////////////////////////////////////
\begin{circuitdiagram}[draft]{32}{20}

    \usgate
    % ----  1st column  ----
    \pin{1}{1}{L}{A}
    \pin{1}{5}{L}{A1}
    \gate[\inputs{2}]{and}{5}{3}{R}{}{}

    % ----  2nd column  ----
    \pin{8}{7}{L}{B}
    \gate[\inputs{2}]{or}{12}{5}{R}{}{}

    \pin{8}{9}{L}{C}
    \pin{8}{13}{L}{C1}
    \gate[\inputs{2}]{or}{12}{11}{R}{}{}

    % ----  3rd column  ----
    \wire{16}{5}{16}{7}
    \gate[\inputs{2}]{and}{19}{9}{R}{}{}

    \pin{15}{15}{L}{D}
    \pin{15}{17}{L}{D1}
    \pin{15}{19}{L}{D2}
    \gate[\inputs{3}]{and}{19}{17}{R}{}{}

    % ----  4th column  ----
    \wire{23}{9}{23}{13}
    \gate[\inputs{2}]{nor}{26}{15}{R}{}{}

    % ----  result ----
    \pin{31}{15}{R}{Y}

\end{circuitdiagram}
 %%  ************    LibreSilicon's StdCellLibrary   *******************
%%
%%  Organisation:   Chipforge
%%                  Germany / European Union
%%
%%  Profile:        Chipforge focus on fine System-on-Chip Cores in
%%                  Verilog HDL Code which are easy understandable and
%%                  adjustable. For further information see
%%                          www.chipforge.org
%%                  there are projects from small cores up to PCBs, too.
%%
%%  File:           StdCellLib/Documents/Datasheets/Circuitry/AOOAAO2123.tex
%%
%%  Purpose:        Circuit File for AOOAAO2123
%%
%%  ************    LaTeX with circdia.sty package      ***************
%%
%%  ///////////////////////////////////////////////////////////////////
%%
%%  Copyright (c) 2018 - 2022 by
%%                  chipforge <stdcelllib@nospam.chipforge.org>
%%  All rights reserved.
%%
%%      This Standard Cell Library is licensed under the Libre Silicon
%%      public license; you can redistribute it and/or modify it under
%%      the terms of the Libre Silicon public license as published by
%%      the Libre Silicon alliance, either version 1 of the License, or
%%      (at your option) any later version.
%%
%%      This design is distributed in the hope that it will be useful,
%%      but WITHOUT ANY WARRANTY; without even the implied warranty of
%%      MERCHANTABILITY or FITNESS FOR A PARTICULAR PURPOSE.
%%      See the Libre Silicon Public License for more details.
%%
%%  ///////////////////////////////////////////////////////////////////
\begin{circuitdiagram}[draft]{38}{20}

    \usgate
    % ----  1st column  ----
    \pin{1}{1}{L}{A}
    \pin{1}{5}{L}{A1}
    \gate[\inputs{2}]{and}{5}{3}{R}{}{}

    % ----  2nd column  ----
    \pin{8}{7}{L}{B}
    \gate[\inputs{2}]{or}{12}{5}{R}{}{}

    \pin{8}{9}{L}{C}
    \pin{8}{13}{L}{C1}
    \gate[\inputs{2}]{or}{12}{11}{R}{}{}

    % ----  3rd column  ----
    \wire{16}{5}{16}{7}
    \gate[\inputs{2}]{and}{19}{9}{R}{}{}

    \pin{15}{15}{L}{D}
    \pin{15}{17}{L}{D1}
    \pin{15}{19}{L}{D2}
    \gate[\inputs{3}]{and}{19}{17}{R}{}{}

    % ----  4th column  ----
    \wire{23}{9}{23}{13}
    \gate[\inputs{2}]{nor}{26}{15}{R}{}{}

    % ----  5th column  ----
    \gate{not}{33}{15}{R}{}{}

    % ----  result ----
    \pin{37}{15}{R}{Z}

\end{circuitdiagram}

%%  ************    LibreSilicon's StdCellLibrary   *******************
%%
%%  Organisation:   Chipforge
%%                  Germany / European Union
%%
%%  Profile:        Chipforge focus on fine System-on-Chip Cores in
%%                  Verilog HDL Code which are easy understandable and
%%                  adjustable. For further information see
%%                          www.chipforge.org
%%                  there are projects from small cores up to PCBs, too.
%%
%%  File:           StdCellLib/Documents/Datasheets/Circuitry/AOOAAOI2124.tex
%%
%%  Purpose:        Circuit File for AOOAAOI2124
%%
%%  ************    LaTeX with circdia.sty package      ***************
%%
%%  ///////////////////////////////////////////////////////////////////
%%
%%  Copyright (c) 2018 - 2022 by
%%                  chipforge <stdcelllib@nospam.chipforge.org>
%%  All rights reserved.
%%
%%      This Standard Cell Library is licensed under the Libre Silicon
%%      public license; you can redistribute it and/or modify it under
%%      the terms of the Libre Silicon public license as published by
%%      the Libre Silicon alliance, either version 1 of the License, or
%%      (at your option) any later version.
%%
%%      This design is distributed in the hope that it will be useful,
%%      but WITHOUT ANY WARRANTY; without even the implied warranty of
%%      MERCHANTABILITY or FITNESS FOR A PARTICULAR PURPOSE.
%%      See the Libre Silicon Public License for more details.
%%
%%  ///////////////////////////////////////////////////////////////////
\begin{circuitdiagram}[draft]{32}{22}

    \usgate
    % ----  1st column  ----
    \pin{1}{1}{L}{A}
    \pin{1}{5}{L}{A1}
    \gate[\inputs{2}]{and}{5}{3}{R}{}{}

    % ----  2nd column  ----
    \pin{8}{7}{L}{B}
    \gate[\inputs{2}]{or}{12}{5}{R}{}{}

    \pin{8}{9}{L}{C}
    \pin{8}{13}{L}{C1}
    \gate[\inputs{2}]{or}{12}{11}{R}{}{}

    % ----  3rd column  ----
    \wire{16}{5}{16}{7}
    \gate[\inputs{2}]{and}{19}{9}{R}{}{}

    \pin{15}{15}{L}{D}
    \pin{15}{17}{L}{D1}
    \pin{15}{19}{L}{D2}
    \pin{15}{21}{L}{D3}
    \gate[\inputs{4}]{and}{19}{18}{R}{}{}

    % ----  4th column  ----
    \wire{23}{9}{23}{14}
    \gate[\inputs{2}]{nor}{26}{16}{R}{}{}

    % ----  result ----
    \pin{31}{16}{R}{Y}

\end{circuitdiagram}
 %%  ************    LibreSilicon's StdCellLibrary   *******************
%%
%%  Organisation:   Chipforge
%%                  Germany / European Union
%%
%%  Profile:        Chipforge focus on fine System-on-Chip Cores in
%%                  Verilog HDL Code which are easy understandable and
%%                  adjustable. For further information see
%%                          www.chipforge.org
%%                  there are projects from small cores up to PCBs, too.
%%
%%  File:           StdCellLib/Documents/Datasheets/Circuitry/AOOAAO2124.tex
%%
%%  Purpose:        Circuit File for AOOAAO2124
%%
%%  ************    LaTeX with circdia.sty package      ***************
%%
%%  ///////////////////////////////////////////////////////////////////
%%
%%  Copyright (c) 2018 - 2022 by
%%                  chipforge <stdcelllib@nospam.chipforge.org>
%%  All rights reserved.
%%
%%      This Standard Cell Library is licensed under the Libre Silicon
%%      public license; you can redistribute it and/or modify it under
%%      the terms of the Libre Silicon public license as published by
%%      the Libre Silicon alliance, either version 1 of the License, or
%%      (at your option) any later version.
%%
%%      This design is distributed in the hope that it will be useful,
%%      but WITHOUT ANY WARRANTY; without even the implied warranty of
%%      MERCHANTABILITY or FITNESS FOR A PARTICULAR PURPOSE.
%%      See the Libre Silicon Public License for more details.
%%
%%  ///////////////////////////////////////////////////////////////////
\begin{circuitdiagram}[draft]{38}{22}

    \usgate
    % ----  1st column  ----
    \pin{1}{1}{L}{A}
    \pin{1}{5}{L}{A1}
    \gate[\inputs{2}]{and}{5}{3}{R}{}{}

    % ----  2nd column  ----
    \pin{8}{7}{L}{B}
    \gate[\inputs{2}]{or}{12}{5}{R}{}{}

    \pin{8}{9}{L}{C}
    \pin{8}{13}{L}{C1}
    \gate[\inputs{2}]{or}{12}{11}{R}{}{}

    % ----  3rd column  ----
    \wire{16}{5}{16}{7}
    \gate[\inputs{2}]{and}{19}{9}{R}{}{}

    \pin{15}{15}{L}{D}
    \pin{15}{17}{L}{D1}
    \pin{15}{19}{L}{D2}
    \pin{15}{21}{L}{D3}
    \gate[\inputs{4}]{and}{19}{18}{R}{}{}

    % ----  4th column  ----
    \wire{23}{9}{23}{14}
    \gate[\inputs{2}]{nor}{26}{16}{R}{}{}

    % ----  5th column  ----
    \gate{not}{33}{16}{R}{}{}

    % ----  result ----
    \pin{37}{16}{R}{Z}

\end{circuitdiagram}

%%  ************    LibreSilicon's StdCellLibrary   *******************
%%
%%  Organisation:   Chipforge
%%                  Germany / European Union
%%
%%  Profile:        Chipforge focus on fine System-on-Chip Cores in
%%                  Verilog HDL Code which are easy understandable and
%%                  adjustable. For further information see
%%                          www.chipforge.org
%%                  there are projects from small cores up to PCBs, too.
%%
%%  File:           StdCellLib/Documents/Datasheets/Circuitry/AOOAAOI2132.tex
%%
%%  Purpose:        Circuit File for AOOAAOI2132
%%
%%  ************    LaTeX with circdia.sty package      ***************
%%
%%  ///////////////////////////////////////////////////////////////////
%%
%%  Copyright (c) 2018 - 2022 by
%%                  chipforge <stdcelllib@nospam.chipforge.org>
%%  All rights reserved.
%%
%%      This Standard Cell Library is licensed under the Libre Silicon
%%      public license; you can redistribute it and/or modify it under
%%      the terms of the Libre Silicon public license as published by
%%      the Libre Silicon alliance, either version 1 of the License, or
%%      (at your option) any later version.
%%
%%      This design is distributed in the hope that it will be useful,
%%      but WITHOUT ANY WARRANTY; without even the implied warranty of
%%      MERCHANTABILITY or FITNESS FOR A PARTICULAR PURPOSE.
%%      See the Libre Silicon Public License for more details.
%%
%%  ///////////////////////////////////////////////////////////////////
\begin{circuitdiagram}[draft]{32}{20}

    \usgate
    % ----  1st column  ----
    \pin{1}{1}{L}{A}
    \pin{1}{5}{L}{A1}
    \gate[\inputs{2}]{and}{5}{3}{R}{}{}

    % ----  2nd column  ----
    \pin{8}{7}{L}{B}
    \gate[\inputs{2}]{or}{12}{5}{R}{}{}

    \pin{8}{9}{L}{C}
    \pin{8}{11}{L}{C1}
    \pin{8}{13}{L}{C2}
    \gate[\inputs{3}]{or}{12}{11}{R}{}{}

    % ----  3rd column  ----
    \wire{16}{5}{16}{7}
    \gate[\inputs{2}]{and}{19}{9}{R}{}{}

    \pin{15}{15}{L}{D}
    \pin{15}{19}{L}{D1}
    \gate[\inputs{2}]{and}{19}{17}{R}{}{}

    % ----  4th column  ----
    \wire{23}{9}{23}{13}
    \gate[\inputs{2}]{nor}{26}{15}{R}{}{}

    % ----  result ----
    \pin{31}{15}{R}{Y}

\end{circuitdiagram}
 %%  ************    LibreSilicon's StdCellLibrary   *******************
%%
%%  Organisation:   Chipforge
%%                  Germany / European Union
%%
%%  Profile:        Chipforge focus on fine System-on-Chip Cores in
%%                  Verilog HDL Code which are easy understandable and
%%                  adjustable. For further information see
%%                          www.chipforge.org
%%                  there are projects from small cores up to PCBs, too.
%%
%%  File:           StdCellLib/Documents/Datasheets/Circuitry/AOOAAO2132.tex
%%
%%  Purpose:        Circuit File for AOOAAO2132
%%
%%  ************    LaTeX with circdia.sty package      ***************
%%
%%  ///////////////////////////////////////////////////////////////////
%%
%%  Copyright (c) 2018 - 2022 by
%%                  chipforge <stdcelllib@nospam.chipforge.org>
%%  All rights reserved.
%%
%%      This Standard Cell Library is licensed under the Libre Silicon
%%      public license; you can redistribute it and/or modify it under
%%      the terms of the Libre Silicon public license as published by
%%      the Libre Silicon alliance, either version 1 of the License, or
%%      (at your option) any later version.
%%
%%      This design is distributed in the hope that it will be useful,
%%      but WITHOUT ANY WARRANTY; without even the implied warranty of
%%      MERCHANTABILITY or FITNESS FOR A PARTICULAR PURPOSE.
%%      See the Libre Silicon Public License for more details.
%%
%%  ///////////////////////////////////////////////////////////////////
\begin{circuitdiagram}[draft]{38}{20}

    \usgate
    % ----  1st column  ----
    \pin{1}{1}{L}{A}
    \pin{1}{5}{L}{A1}
    \gate[\inputs{2}]{and}{5}{3}{R}{}{}

    % ----  2nd column  ----
    \pin{8}{7}{L}{B}
    \gate[\inputs{2}]{or}{12}{5}{R}{}{}

    \pin{8}{9}{L}{C}
    \pin{8}{11}{L}{C1}
    \pin{8}{13}{L}{C2}
    \gate[\inputs{3}]{or}{12}{11}{R}{}{}

    % ----  3rd column  ----
    \wire{16}{5}{16}{7}
    \gate[\inputs{2}]{and}{19}{9}{R}{}{}

    \pin{15}{15}{L}{D}
    \pin{15}{19}{L}{D1}
    \gate[\inputs{2}]{and}{19}{17}{R}{}{}

    % ----  4th column  ----
    \wire{23}{9}{23}{13}
    \gate[\inputs{2}]{nor}{26}{15}{R}{}{}

    % ----  5th column  ----
    \gate{not}{33}{15}{R}{}{}

    % ----  result ----
    \pin{37}{15}{R}{Z}

\end{circuitdiagram}

%%  ************    LibreSilicon's StdCellLibrary   *******************
%%
%%  Organisation:   Chipforge
%%                  Germany / European Union
%%
%%  Profile:        Chipforge focus on fine System-on-Chip Cores in
%%                  Verilog HDL Code which are easy understandable and
%%                  adjustable. For further information see
%%                          www.chipforge.org
%%                  there are projects from small cores up to PCBs, too.
%%
%%  File:           StdCellLib/Documents/Datasheets/Circuitry/AOOAAOI3122.tex
%%
%%  Purpose:        Circuit File for AOOAAOI3122
%%
%%  ************    LaTeX with circdia.sty package      ***************
%%
%%  ///////////////////////////////////////////////////////////////////
%%
%%  Copyright (c) 2018 - 2022 by
%%                  chipforge <stdcelllib@nospam.chipforge.org>
%%  All rights reserved.
%%
%%      This Standard Cell Library is licensed under the Libre Silicon
%%      public license; you can redistribute it and/or modify it under
%%      the terms of the Libre Silicon public license as published by
%%      the Libre Silicon alliance, either version 1 of the License, or
%%      (at your option) any later version.
%%
%%      This design is distributed in the hope that it will be useful,
%%      but WITHOUT ANY WARRANTY; without even the implied warranty of
%%      MERCHANTABILITY or FITNESS FOR A PARTICULAR PURPOSE.
%%      See the Libre Silicon Public License for more details.
%%
%%  ///////////////////////////////////////////////////////////////////
\begin{circuitdiagram}[draft]{32}{20}

    \usgate
    % ----  1st column  ----
    \pin{1}{1}{L}{A}
    \pin{1}{3}{L}{A1}
    \pin{1}{5}{L}{A2}
    \gate[\inputs{3}]{and}{5}{3}{R}{}{}

    % ----  2nd column  ----
    \pin{8}{7}{L}{B}
    \gate[\inputs{2}]{or}{12}{5}{R}{}{}

    \pin{8}{9}{L}{C}
    \pin{8}{13}{L}{C1}
    \gate[\inputs{2}]{or}{12}{11}{R}{}{}

    % ----  3rd column  ----
    \wire{16}{5}{16}{7}
    \gate[\inputs{2}]{and}{19}{9}{R}{}{}

    \pin{15}{15}{L}{D}
    \pin{15}{19}{L}{D1}
    \gate[\inputs{2}]{and}{19}{17}{R}{}{}

    % ----  4th column  ----
    \wire{23}{9}{23}{13}
    \gate[\inputs{2}]{nor}{26}{15}{R}{}{}

    % ----  result ----
    \pin{31}{15}{R}{Y}

\end{circuitdiagram}
 %%  ************    LibreSilicon's StdCellLibrary   *******************
%%
%%  Organisation:   Chipforge
%%                  Germany / European Union
%%
%%  Profile:        Chipforge focus on fine System-on-Chip Cores in
%%                  Verilog HDL Code which are easy understandable and
%%                  adjustable. For further information see
%%                          www.chipforge.org
%%                  there are projects from small cores up to PCBs, too.
%%
%%  File:           StdCellLib/Documents/Datasheets/Circuitry/AOOAAO3122.tex
%%
%%  Purpose:        Circuit File for AOOAAO3122
%%
%%  ************    LaTeX with circdia.sty package      ***************
%%
%%  ///////////////////////////////////////////////////////////////////
%%
%%  Copyright (c) 2018 - 2022 by
%%                  chipforge <stdcelllib@nospam.chipforge.org>
%%  All rights reserved.
%%
%%      This Standard Cell Library is licensed under the Libre Silicon
%%      public license; you can redistribute it and/or modify it under
%%      the terms of the Libre Silicon public license as published by
%%      the Libre Silicon alliance, either version 1 of the License, or
%%      (at your option) any later version.
%%
%%      This design is distributed in the hope that it will be useful,
%%      but WITHOUT ANY WARRANTY; without even the implied warranty of
%%      MERCHANTABILITY or FITNESS FOR A PARTICULAR PURPOSE.
%%      See the Libre Silicon Public License for more details.
%%
%%  ///////////////////////////////////////////////////////////////////
\begin{circuitdiagram}[draft]{38}{20}

    \usgate
    % ----  1st column  ----
    \pin{1}{1}{L}{A}
    \pin{1}{3}{L}{A1}
    \pin{1}{5}{L}{A2}
    \gate[\inputs{3}]{and}{5}{3}{R}{}{}

    % ----  2nd column  ----
    \pin{8}{7}{L}{B}
    \gate[\inputs{2}]{or}{12}{5}{R}{}{}

    \pin{8}{9}{L}{C}
    \pin{8}{13}{L}{C1}
    \gate[\inputs{2}]{or}{12}{11}{R}{}{}

    % ----  3rd column  ----
    \wire{16}{5}{16}{7}
    \gate[\inputs{2}]{and}{19}{9}{R}{}{}

    \pin{15}{15}{L}{D}
    \pin{15}{19}{L}{D1}
    \gate[\inputs{2}]{and}{19}{17}{R}{}{}

    % ----  4th column  ----
    \wire{23}{9}{23}{13}
    \gate[\inputs{2}]{nor}{26}{15}{R}{}{}

    % ----  last column ----
    \gate{not}{33}{15}{R}{}{}

    % ----  result ----
    \pin{37}{15}{R}{Z}

\end{circuitdiagram}


%%  ************    LibreSilicon's StdCellLibrary   *******************
%%
%%  Organisation:   Chipforge
%%                  Germany / European Union
%%
%%  Profile:        Chipforge focus on fine System-on-Chip Cores in
%%                  Verilog HDL Code which are easy understandable and
%%                  adjustable. For further information see
%%                          www.chipforge.org
%%                  there are projects from small cores up to PCBs, too.
%%
%%  File:           StdCellLib/Documents/Datasheets/Circuitry/AOOAAOI21212.tex
%%
%%  Purpose:        Circuit File for AOOAAOI21212
%%
%%  ************    LaTeX with circdia.sty package      ***************
%%
%%  ///////////////////////////////////////////////////////////////////
%%
%%  Copyright (c) 2018 - 2022 by
%%                  chipforge <stdcelllib@nospam.chipforge.org>
%%  All rights reserved.
%%
%%      This Standard Cell Library is licensed under the Libre Silicon
%%      public license; you can redistribute it and/or modify it under
%%      the terms of the Libre Silicon public license as published by
%%      the Libre Silicon alliance, either version 1 of the License, or
%%      (at your option) any later version.
%%
%%      This design is distributed in the hope that it will be useful,
%%      but WITHOUT ANY WARRANTY; without even the implied warranty of
%%      MERCHANTABILITY or FITNESS FOR A PARTICULAR PURPOSE.
%%      See the Libre Silicon Public License for more details.
%%
%%  ///////////////////////////////////////////////////////////////////
\begin{circuitdiagram}[draft]{32}{22}

    \usgate
    % ----  1st column  ----
    \pin{1}{1}{L}{A}
    \pin{1}{5}{L}{A1}
    \gate[\inputs{2}]{and}{5}{3}{R}{}{}

    % ----  2nd column  ----
    \pin{8}{7}{L}{B}
    \gate[\inputs{2}]{or}{12}{5}{R}{}{}

    \pin{8}{9}{L}{C}
    \pin{8}{13}{L}{C1}
    \gate[\inputs{2}]{or}{12}{11}{R}{}{}

    % ----  3rd column  ----
    \wire{16}{5}{16}{9}
    \pin{15}{15}{L}{D}
    \wire{16}{13}{16}{15}
    \gate[\inputs{3}]{and}{19}{11}{R}{}{}

    \pin{15}{17}{L}{E}
    \pin{15}{21}{L}{E1}
    \gate[\inputs{2}]{and}{19}{19}{R}{}{}

    % ----  4th column  ----
    \wire{23}{11}{23}{15}
    \gate[\inputs{2}]{nor}{26}{17}{R}{}{}

    % ----  result ----
    \pin{31}{17}{R}{Y}

\end{circuitdiagram}
 %%  ************    LibreSilicon's StdCellLibrary   *******************
%%
%%  Organisation:   Chipforge
%%                  Germany / European Union
%%
%%  Profile:        Chipforge focus on fine System-on-Chip Cores in
%%                  Verilog HDL Code which are easy understandable and
%%                  adjustable. For further information see
%%                          www.chipforge.org
%%                  there are projects from small cores up to PCBs, too.
%%
%%  File:           StdCellLib/Documents/Datasheets/Circuitry/AOOAAO21212.tex
%%
%%  Purpose:        Circuit File for AOOAAO21212
%%
%%  ************    LaTeX with circdia.sty package      ***************
%%
%%  ///////////////////////////////////////////////////////////////////
%%
%%  Copyright (c) 2018 - 2022 by
%%                  chipforge <stdcelllib@nospam.chipforge.org>
%%  All rights reserved.
%%
%%      This Standard Cell Library is licensed under the Libre Silicon
%%      public license; you can redistribute it and/or modify it under
%%      the terms of the Libre Silicon public license as published by
%%      the Libre Silicon alliance, either version 1 of the License, or
%%      (at your option) any later version.
%%
%%      This design is distributed in the hope that it will be useful,
%%      but WITHOUT ANY WARRANTY; without even the implied warranty of
%%      MERCHANTABILITY or FITNESS FOR A PARTICULAR PURPOSE.
%%      See the Libre Silicon Public License for more details.
%%
%%  ///////////////////////////////////////////////////////////////////
\begin{circuitdiagram}[draft]{38}{22}

    \usgate
    % ----  1st column  ----
    \pin{1}{1}{L}{A}
    \pin{1}{5}{L}{A1}
    \gate[\inputs{2}]{and}{5}{3}{R}{}{}

    % ----  2nd column  ----
    \pin{8}{7}{L}{B}
    \gate[\inputs{2}]{or}{12}{5}{R}{}{}

    \pin{8}{9}{L}{C}
    \pin{8}{13}{L}{C1}
    \gate[\inputs{2}]{or}{12}{11}{R}{}{}

    % ----  3rd column  ----
    \wire{16}{5}{16}{9}
    \pin{15}{15}{L}{D}
    \wire{16}{13}{16}{15}
    \gate[\inputs{3}]{and}{19}{11}{R}{}{}

    \pin{15}{17}{L}{E}
    \pin{15}{21}{L}{E1}
    \gate[\inputs{2}]{and}{19}{19}{R}{}{}

    % ----  4th column  ----
    \wire{23}{11}{23}{15}
    \gate[\inputs{2}]{nor}{26}{17}{R}{}{}

    % ----  last column ----
    \gate{not}{33}{17}{R}{}{}

    % ----  result ----
    \pin{37}{17}{R}{Z}

\end{circuitdiagram}

%%  ************    LibreSilicon's StdCellLibrary   *******************
%%
%%  Organisation:   Chipforge
%%                  Germany / European Union
%%
%%  Profile:        Chipforge focus on fine System-on-Chip Cores in
%%                  Verilog HDL Code which are easy understandable and
%%                  adjustable. For further information see
%%                          www.chipforge.org
%%                  there are projects from small cores up to PCBs, too.
%%
%%  File:           StdCellLib/Documents/Datasheets/Circuitry/AOOAAOI21221.tex
%%
%%  Purpose:        Circuit File for AOOAAOI21221
%%
%%  ************    LaTeX with circdia.sty package      ***************
%%
%%  ///////////////////////////////////////////////////////////////////
%%
%%  Copyright (c) 2018 - 2022 by
%%                  chipforge <stdcelllib@nospam.chipforge.org>
%%  All rights reserved.
%%
%%      This Standard Cell Library is licensed under the Libre Silicon
%%      public license; you can redistribute it and/or modify it under
%%      the terms of the Libre Silicon public license as published by
%%      the Libre Silicon alliance, either version 1 of the License, or
%%      (at your option) any later version.
%%
%%      This design is distributed in the hope that it will be useful,
%%      but WITHOUT ANY WARRANTY; without even the implied warranty of
%%      MERCHANTABILITY or FITNESS FOR A PARTICULAR PURPOSE.
%%      See the Libre Silicon Public License for more details.
%%
%%  ///////////////////////////////////////////////////////////////////
\begin{circuitdiagram}[draft]{32}{22}

    \usgate
    % ----  1st column  ----
    \pin{1}{1}{L}{A}
    \pin{1}{5}{L}{A1}
    \gate[\inputs{2}]{and}{5}{3}{R}{}{}

    % ----  2nd column  ----
    \pin{8}{7}{L}{B}
    \gate[\inputs{2}]{or}{12}{5}{R}{}{}

    \pin{8}{9}{L}{C}
    \pin{8}{13}{L}{C1}
    \gate[\inputs{2}]{or}{12}{11}{R}{}{}

    % ----  3rd column  ----
    \wire{16}{5}{16}{7}
    \gate[\inputs{2}]{and}{19}{9}{R}{}{}

    \pin{15}{15}{L}{D}
    \pin{15}{19}{L}{D1}
    \gate[\inputs{2}]{and}{19}{17}{R}{}{}

    % ----  4th column  ----
    \pin{22}{21}{L}{E}
    \wire{23}{19}{23}{21}
    \wire{23}{9}{23}{15}
    \gate[\inputs{3}]{nor}{26}{17}{R}{}{}

    % ----  result ----
    \pin{31}{17}{R}{Y}

\end{circuitdiagram}
 %%  ************    LibreSilicon's StdCellLibrary   *******************
%%
%%  Organisation:   Chipforge
%%                  Germany / European Union
%%
%%  Profile:        Chipforge focus on fine System-on-Chip Cores in
%%                  Verilog HDL Code which are easy understandable and
%%                  adjustable. For further information see
%%                          www.chipforge.org
%%                  there are projects from small cores up to PCBs, too.
%%
%%  File:           StdCellLib/Documents/Datasheets/Circuitry/AOOAAO21221.tex
%%
%%  Purpose:        Circuit File for AOOAAO21221
%%
%%  ************    LaTeX with circdia.sty package      ***************
%%
%%  ///////////////////////////////////////////////////////////////////
%%
%%  Copyright (c) 2018 - 2022 by
%%                  chipforge <stdcelllib@nospam.chipforge.org>
%%  All rights reserved.
%%
%%      This Standard Cell Library is licensed under the Libre Silicon
%%      public license; you can redistribute it and/or modify it under
%%      the terms of the Libre Silicon public license as published by
%%      the Libre Silicon alliance, either version 1 of the License, or
%%      (at your option) any later version.
%%
%%      This design is distributed in the hope that it will be useful,
%%      but WITHOUT ANY WARRANTY; without even the implied warranty of
%%      MERCHANTABILITY or FITNESS FOR A PARTICULAR PURPOSE.
%%      See the Libre Silicon Public License for more details.
%%
%%  ///////////////////////////////////////////////////////////////////
\begin{circuitdiagram}[draft]{38}{22}

    \usgate
    % ----  1st column  ----
    \pin{1}{1}{L}{A}
    \pin{1}{5}{L}{A1}
    \gate[\inputs{2}]{and}{5}{3}{R}{}{}

    % ----  2nd column  ----
    \pin{8}{7}{L}{B}
    \gate[\inputs{2}]{or}{12}{5}{R}{}{}

    \pin{8}{9}{L}{C}
    \pin{8}{13}{L}{C1}
    \gate[\inputs{2}]{or}{12}{11}{R}{}{}

    % ----  3rd column  ----
    \wire{16}{5}{16}{7}
    \gate[\inputs{2}]{and}{19}{9}{R}{}{}

    \pin{15}{15}{L}{D}
    \pin{15}{19}{L}{D1}
    \gate[\inputs{2}]{and}{19}{17}{R}{}{}

    % ----  4th column  ----
    \pin{22}{21}{L}{E}
    \wire{23}{19}{23}{21}
    \wire{23}{9}{23}{15}
    \gate[\inputs{3}]{nor}{26}{17}{R}{}{}

    % ----  5th column  ----
    \gate{not}{33}{17}{R}{}{}

    % ----  result ----
    \pin{37}{17}{R}{Z}

\end{circuitdiagram}

%%  ************    LibreSilicon's StdCellLibrary   *******************
%%
%%  Organisation:   Chipforge
%%                  Germany / European Union
%%
%%  Profile:        Chipforge focus on fine System-on-Chip Cores in
%%                  Verilog HDL Code which are easy understandable and
%%                  adjustable. For further information see
%%                          www.chipforge.org
%%                  there are projects from small cores up to PCBs, too.
%%
%%  File:           StdCellLib/Documents/Datasheets/Circuitry/AOOAAOI21231.tex
%%
%%  Purpose:        Circuit File for AOOAAOI21231
%%
%%  ************    LaTeX with circdia.sty package      ***************
%%
%%  ///////////////////////////////////////////////////////////////////
%%
%%  Copyright (c) 2018 - 2022 by
%%                  chipforge <stdcelllib@nospam.chipforge.org>
%%  All rights reserved.
%%
%%      This Standard Cell Library is licensed under the Libre Silicon
%%      public license; you can redistribute it and/or modify it under
%%      the terms of the Libre Silicon public license as published by
%%      the Libre Silicon alliance, either version 1 of the License, or
%%      (at your option) any later version.
%%
%%      This design is distributed in the hope that it will be useful,
%%      but WITHOUT ANY WARRANTY; without even the implied warranty of
%%      MERCHANTABILITY or FITNESS FOR A PARTICULAR PURPOSE.
%%      See the Libre Silicon Public License for more details.
%%
%%  ///////////////////////////////////////////////////////////////////
\begin{circuitdiagram}[draft]{32}{22}

    \usgate
    % ----  1st column  ----
    \pin{1}{1}{L}{A}
    \pin{1}{5}{L}{A1}
    \gate[\inputs{2}]{and}{5}{3}{R}{}{}

    % ----  2nd column  ----
    \pin{8}{7}{L}{B}
    \gate[\inputs{2}]{or}{12}{5}{R}{}{}

    \pin{8}{9}{L}{C}
    \pin{8}{13}{L}{C1}
    \gate[\inputs{2}]{or}{12}{11}{R}{}{}

    % ----  3rd column  ----
    \wire{16}{5}{16}{7}
    \gate[\inputs{2}]{and}{19}{9}{R}{}{}

    \pin{15}{15}{L}{D}
    \pin{15}{17}{L}{D1}
    \pin{15}{19}{L}{D2}
    \gate[\inputs{3}]{and}{19}{17}{R}{}{}

    % ----  4th column  ----
    \wire{23}{9}{23}{15}
    \pin{22}{21}{L}{E}
    \wire{23}{19}{23}{21}
    \gate[\inputs{3}]{nor}{26}{17}{R}{}{}

    % ----  result ----
    \pin{31}{17}{R}{Y}

\end{circuitdiagram}
 %%  ************    LibreSilicon's StdCellLibrary   *******************
%%
%%  Organisation:   Chipforge
%%                  Germany / European Union
%%
%%  Profile:        Chipforge focus on fine System-on-Chip Cores in
%%                  Verilog HDL Code which are easy understandable and
%%                  adjustable. For further information see
%%                          www.chipforge.org
%%                  there are projects from small cores up to PCBs, too.
%%
%%  File:           StdCellLib/Documents/Datasheets/Circuitry/AOOAAO21231.tex
%%
%%  Purpose:        Circuit File for AOOAAO21231
%%
%%  ************    LaTeX with circdia.sty package      ***************
%%
%%  ///////////////////////////////////////////////////////////////////
%%
%%  Copyright (c) 2018 - 2022 by
%%                  chipforge <stdcelllib@nospam.chipforge.org>
%%  All rights reserved.
%%
%%      This Standard Cell Library is licensed under the Libre Silicon
%%      public license; you can redistribute it and/or modify it under
%%      the terms of the Libre Silicon public license as published by
%%      the Libre Silicon alliance, either version 1 of the License, or
%%      (at your option) any later version.
%%
%%      This design is distributed in the hope that it will be useful,
%%      but WITHOUT ANY WARRANTY; without even the implied warranty of
%%      MERCHANTABILITY or FITNESS FOR A PARTICULAR PURPOSE.
%%      See the Libre Silicon Public License for more details.
%%
%%  ///////////////////////////////////////////////////////////////////
\begin{circuitdiagram}[draft]{38}{22}

    \usgate
    % ----  1st column  ----
    \pin{1}{1}{L}{A}
    \pin{1}{5}{L}{A1}
    \gate[\inputs{2}]{and}{5}{3}{R}{}{}

    % ----  2nd column  ----
    \pin{8}{7}{L}{B}
    \gate[\inputs{2}]{or}{12}{5}{R}{}{}

    \pin{8}{9}{L}{C}
    \pin{8}{13}{L}{C1}
    \gate[\inputs{2}]{or}{12}{11}{R}{}{}

    % ----  3rd column  ----
    \wire{16}{5}{16}{7}
    \gate[\inputs{2}]{and}{19}{9}{R}{}{}

    \pin{15}{15}{L}{D}
    \pin{15}{17}{L}{D1}
    \pin{15}{19}{L}{D2}
    \gate[\inputs{3}]{and}{19}{17}{R}{}{}

    % ----  4th column  ----
    \wire{23}{9}{23}{15}
    \pin{22}{21}{L}{E}
    \wire{23}{19}{23}{21}
    \gate[\inputs{3}]{nor}{26}{17}{R}{}{}

    % ----  last column  ----
    \gate{not}{33}{17}{R}{}{}

    % ----  result ----
    \pin{37}{17}{R}{Z}

\end{circuitdiagram}


%%  ************    LibreSilicon's StdCellLibrary   *******************
%%
%%  Organisation:   Chipforge
%%                  Germany / European Union
%%
%%  Profile:        Chipforge focus on fine System-on-Chip Cores in
%%                  Verilog HDL Code which are easy understandable and
%%                  adjustable. For further information see
%%                          www.chipforge.org
%%                  there are projects from small cores up to PCBs, too.
%%
%%  File:           StdCellLib/Documents/section-OAAOOAI_complex.tex
%%
%%  Purpose:        Section Level File for Standard Cell Library Documentation
%%
%%  ************    LaTeX with circdia.sty package      ***************
%%
%%  ///////////////////////////////////////////////////////////////////
%%
%%  Copyright (c) 2018 - 2022 by
%%                  chipforge <stdcelllib@nospam.chipforge.org>
%%  All rights reserved.
%%
%%      This Standard Cell Library is licensed under the Libre Silicon
%%      public license; you can redistribute it and/or modify it under
%%      the terms of the Libre Silicon public license as published by
%%      the Libre Silicon alliance, either version 1 of the License, or
%%      (at your option) any later version.
%%
%%      This design is distributed in the hope that it will be useful,
%%      but WITHOUT ANY WARRANTY; without even the implied warranty of
%%      MERCHANTABILITY or FITNESS FOR A PARTICULAR PURPOSE.
%%      See the Libre Silicon Public License for more details.
%%
%%  ///////////////////////////////////////////////////////////////////
\section{OR-AND-AND-OR-OR-AND(-Invert) Complex Gates}

%%  ************    LibreSilicon's StdCellLibrary   *******************
%%
%%  Organisation:   Chipforge
%%                  Germany / European Union
%%
%%  Profile:        Chipforge focus on fine System-on-Chip Cores in
%%                  Verilog HDL Code which are easy understandable and
%%                  adjustable. For further information see
%%                          www.chipforge.org
%%                  there are projects from small cores up to PCBs, too.
%%
%%  File:           StdCellLib/Documents/Datasheets/Circuitry/OAAOOAI2122.tex
%%
%%  Purpose:        Circuit File for OAAOOAI2122
%%
%%  ************    LaTeX with circdia.sty package      ***************
%%
%%  ///////////////////////////////////////////////////////////////////
%%
%%  Copyright (c) 2018 - 2022 by
%%                  chipforge <stdcelllib@nospam.chipforge.org>
%%  All rights reserved.
%%
%%      This Standard Cell Library is licensed under the Libre Silicon
%%      public license; you can redistribute it and/or modify it under
%%      the terms of the Libre Silicon public license as published by
%%      the Libre Silicon alliance, either version 1 of the License, or
%%      (at your option) any later version.
%%
%%      This design is distributed in the hope that it will be useful,
%%      but WITHOUT ANY WARRANTY; without even the implied warranty of
%%      MERCHANTABILITY or FITNESS FOR A PARTICULAR PURPOSE.
%%      See the Libre Silicon Public License for more details.
%%
%%  ///////////////////////////////////////////////////////////////////
\begin{circuitdiagram}[draft]{32}{20}

    \usgate
    % ----  1st column  ----
    \pin{1}{1}{L}{A}
    \pin{1}{5}{L}{A1}
    \gate[\inputs{2}]{or}{5}{3}{R}{}{}

    % ----  2nd column  ----
    \pin{8}{7}{L}{B}
    \gate[\inputs{2}]{and}{12}{5}{R}{}{}

    \pin{8}{9}{L}{C}
    \pin{8}{13}{L}{C1}
    \gate[\inputs{2}]{and}{12}{11}{R}{}{}

    % ----  3rd column  ----
    \wire{16}{5}{16}{7}
    \gate[\inputs{2}]{or}{19}{9}{R}{}{}

    \pin{15}{15}{L}{D}
    \pin{15}{19}{L}{D1}
    \gate[\inputs{2}]{or}{19}{17}{R}{}{}

    % ----  4th column  ----
    \wire{23}{9}{23}{13}
    \gate[\inputs{2}]{nand}{26}{15}{R}{}{}

    % ----  result ----
    \pin{31}{15}{R}{Y}

\end{circuitdiagram}
 %%  ************    LibreSilicon's StdCellLibrary   *******************
%%
%%  Organisation:   Chipforge
%%                  Germany / European Union
%%
%%  Profile:        Chipforge focus on fine System-on-Chip Cores in
%%                  Verilog HDL Code which are easy understandable and
%%                  adjustable. For further information see
%%                          www.chipforge.org
%%                  there are projects from small cores up to PCBs, too.
%%
%%  File:           StdCellLib/Documents/Datasheets/Circuitry/OAAOOA2122.tex
%%
%%  Purpose:        Circuit File for OAAOOA2122
%%
%%  ************    LaTeX with circdia.sty package      ***************
%%
%%  ///////////////////////////////////////////////////////////////////
%%
%%  Copyright (c) 2018 - 2022 by
%%                  chipforge <stdcelllib@nospam.chipforge.org>
%%  All rights reserved.
%%
%%      This Standard Cell Library is licensed under the Libre Silicon
%%      public license; you can redistribute it and/or modify it under
%%      the terms of the Libre Silicon public license as published by
%%      the Libre Silicon alliance, either version 1 of the License, or
%%      (at your option) any later version.
%%
%%      This design is distributed in the hope that it will be useful,
%%      but WITHOUT ANY WARRANTY; without even the implied warranty of
%%      MERCHANTABILITY or FITNESS FOR A PARTICULAR PURPOSE.
%%      See the Libre Silicon Public License for more details.
%%
%%  ///////////////////////////////////////////////////////////////////
\begin{circuitdiagram}[draft]{38}{20}

    \usgate
    % ----  1st column  ----
    \pin{1}{1}{L}{A}
    \pin{1}{5}{L}{A1}
    \gate[\inputs{2}]{or}{5}{3}{R}{}{}

    % ----  2nd column  ----
    \pin{8}{7}{L}{B}
    \gate[\inputs{2}]{and}{12}{5}{R}{}{}

    \pin{8}{9}{L}{C}
    \pin{8}{13}{L}{C1}
    \gate[\inputs{2}]{and}{12}{11}{R}{}{}

    % ----  3rd column  ----
    \wire{16}{5}{16}{7}
    \gate[\inputs{2}]{or}{19}{9}{R}{}{}

    \pin{15}{15}{L}{D}
    \pin{15}{19}{L}{D1}
    \gate[\inputs{2}]{or}{19}{17}{R}{}{}

    % ----  4th column  ----
    \wire{23}{9}{23}{13}
    \gate[\inputs{2}]{nand}{26}{15}{R}{}{}

    % ----  5th column  ----
    \gate{not}{33}{15}{R}{}{}

    % ----  result ----
    \pin{37}{15}{R}{Z}

\end{circuitdiagram}

%%  ************    LibreSilicon's StdCellLibrary   *******************
%%
%%  Organisation:   Chipforge
%%                  Germany / European Union
%%
%%  Profile:        Chipforge focus on fine System-on-Chip Cores in
%%                  Verilog HDL Code which are easy understandable and
%%                  adjustable. For further information see
%%                          www.chipforge.org
%%                  there are projects from small cores up to PCBs, too.
%%
%%  File:           StdCellLib/Documents/Datasheets/Circuitry/OAAOOAI2123.tex
%%
%%  Purpose:        Circuit File for OAAOOAI2123
%%
%%  ************    LaTeX with circdia.sty package      ***************
%%
%%  ///////////////////////////////////////////////////////////////////
%%
%%  Copyright (c) 2018 - 2022 by
%%                  chipforge <stdcelllib@nospam.chipforge.org>
%%  All rights reserved.
%%
%%      This Standard Cell Library is licensed under the Libre Silicon
%%      public license; you can redistribute it and/or modify it under
%%      the terms of the Libre Silicon public license as published by
%%      the Libre Silicon alliance, either version 1 of the License, or
%%      (at your option) any later version.
%%
%%      This design is distributed in the hope that it will be useful,
%%      but WITHOUT ANY WARRANTY; without even the implied warranty of
%%      MERCHANTABILITY or FITNESS FOR A PARTICULAR PURPOSE.
%%      See the Libre Silicon Public License for more details.
%%
%%  ///////////////////////////////////////////////////////////////////
\begin{circuitdiagram}[draft]{32}{20}

    \usgate
    % ----  1st column  ----
    \pin{1}{1}{L}{A}
    \pin{1}{5}{L}{A1}
    \gate[\inputs{2}]{or}{5}{3}{R}{}{}

    % ----  2nd column  ----
    \pin{8}{7}{L}{B}
    \gate[\inputs{2}]{and}{12}{5}{R}{}{}

    \pin{8}{9}{L}{C}
    \pin{8}{13}{L}{C1}
    \gate[\inputs{2}]{and}{12}{11}{R}{}{}

    % ----  3rd column  ----
    \wire{16}{5}{16}{7}
    \gate[\inputs{2}]{or}{19}{9}{R}{}{}

    \pin{15}{15}{L}{D}
    \pin{15}{17}{L}{D1}
    \pin{15}{19}{L}{D2}
    \gate[\inputs{3}]{or}{19}{17}{R}{}{}

    % ----  4th column  ----
    \wire{23}{9}{23}{13}
    \gate[\inputs{2}]{nand}{26}{15}{R}{}{}

    % ----  result ----
    \pin{31}{15}{R}{Y}

\end{circuitdiagram}
 %%  ************    LibreSilicon's StdCellLibrary   *******************
%%
%%  Organisation:   Chipforge
%%                  Germany / European Union
%%
%%  Profile:        Chipforge focus on fine System-on-Chip Cores in
%%                  Verilog HDL Code which are easy understandable and
%%                  adjustable. For further information see
%%                          www.chipforge.org
%%                  there are projects from small cores up to PCBs, too.
%%
%%  File:           StdCellLib/Documents/Datasheets/Circuitry/OAAOOA2123.tex
%%
%%  Purpose:        Circuit File for OAAOOA2123
%%
%%  ************    LaTeX with circdia.sty package      ***************
%%
%%  ///////////////////////////////////////////////////////////////////
%%
%%  Copyright (c) 2018 - 2022 by
%%                  chipforge <stdcelllib@nospam.chipforge.org>
%%  All rights reserved.
%%
%%      This Standard Cell Library is licensed under the Libre Silicon
%%      public license; you can redistribute it and/or modify it under
%%      the terms of the Libre Silicon public license as published by
%%      the Libre Silicon alliance, either version 1 of the License, or
%%      (at your option) any later version.
%%
%%      This design is distributed in the hope that it will be useful,
%%      but WITHOUT ANY WARRANTY; without even the implied warranty of
%%      MERCHANTABILITY or FITNESS FOR A PARTICULAR PURPOSE.
%%      See the Libre Silicon Public License for more details.
%%
%%  ///////////////////////////////////////////////////////////////////
\begin{circuitdiagram}[draft]{38}{20}

    \usgate
    % ----  1st column  ----
    \pin{1}{1}{L}{A}
    \pin{1}{5}{L}{A1}
    \gate[\inputs{2}]{or}{5}{3}{R}{}{}

    % ----  2nd column  ----
    \pin{8}{7}{L}{B}
    \gate[\inputs{2}]{and}{12}{5}{R}{}{}

    \pin{8}{9}{L}{C}
    \pin{8}{13}{L}{C1}
    \gate[\inputs{2}]{and}{12}{11}{R}{}{}

    % ----  3rd column  ----
    \wire{16}{5}{16}{7}
    \gate[\inputs{2}]{or}{19}{9}{R}{}{}

    \pin{15}{15}{L}{D}
    \pin{15}{17}{L}{D1}
    \pin{15}{19}{L}{D2}
    \gate[\inputs{3}]{or}{19}{17}{R}{}{}

    % ----  4th column  ----
    \wire{23}{9}{23}{13}
    \gate[\inputs{2}]{nand}{26}{15}{R}{}{}

    % ----  5th column  ----
    \gate{not}{33}{15}{R}{}{}

    % ----  result ----
    \pin{37}{15}{R}{Z}

\end{circuitdiagram}

%%  ************    LibreSilicon's StdCellLibrary   *******************
%%
%%  Organisation:   Chipforge
%%                  Germany / European Union
%%
%%  Profile:        Chipforge focus on fine System-on-Chip Cores in
%%                  Verilog HDL Code which are easy understandable and
%%                  adjustable. For further information see
%%                          www.chipforge.org
%%                  there are projects from small cores up to PCBs, too.
%%
%%  File:           StdCellLib/Documents/Datasheets/Circuitry/OAAOOAI2124.tex
%%
%%  Purpose:        Circuit File for OAAOOAI2124
%%
%%  ************    LaTeX with circdia.sty package      ***************
%%
%%  ///////////////////////////////////////////////////////////////////
%%
%%  Copyright (c) 2018 - 2022 by
%%                  chipforge <stdcelllib@nospam.chipforge.org>
%%  All rights reserved.
%%
%%      This Standard Cell Library is licensed under the Libre Silicon
%%      public license; you can redistribute it and/or modify it under
%%      the terms of the Libre Silicon public license as published by
%%      the Libre Silicon alliance, either version 1 of the License, or
%%      (at your option) any later version.
%%
%%      This design is distributed in the hope that it will be useful,
%%      but WITHOUT ANY WARRANTY; without even the implied warranty of
%%      MERCHANTABILITY or FITNESS FOR A PARTICULAR PURPOSE.
%%      See the Libre Silicon Public License for more details.
%%
%%  ///////////////////////////////////////////////////////////////////
\begin{circuitdiagram}[draft]{32}{22}

    \usgate
    % ----  1st column  ----
    \pin{1}{1}{L}{A}
    \pin{1}{5}{L}{A1}
    \gate[\inputs{2}]{or}{5}{3}{R}{}{}

    % ----  2nd column  ----
    \pin{8}{7}{L}{B}
    \gate[\inputs{2}]{and}{12}{5}{R}{}{}

    \pin{8}{9}{L}{C}
    \pin{8}{13}{L}{C1}
    \gate[\inputs{2}]{and}{12}{11}{R}{}{}

    % ----  3rd column  ----
    \wire{16}{5}{16}{7}
    \gate[\inputs{2}]{or}{19}{9}{R}{}{}

    \pin{15}{15}{L}{D}
    \pin{15}{17}{L}{D1}
    \pin{15}{19}{L}{D2}
    \pin{15}{21}{L}{D3}
    \gate[\inputs{4}]{or}{19}{18}{R}{}{}

    % ----  4th column  ----
    \wire{23}{9}{23}{14}
    \gate[\inputs{2}]{nand}{26}{16}{R}{}{}

    % ----  result ----
    \pin{31}{16}{R}{Y}

\end{circuitdiagram}
 %%  ************    LibreSilicon's StdCellLibrary   *******************
%%
%%  Organisation:   Chipforge
%%                  Germany / European Union
%%
%%  Profile:        Chipforge focus on fine System-on-Chip Cores in
%%                  Verilog HDL Code which are easy understandable and
%%                  adjustable. For further information see
%%                          www.chipforge.org
%%                  there are projects from small cores up to PCBs, too.
%%
%%  File:           StdCellLib/Documents/Datasheets/Circuitry/OAAOOA2124.tex
%%
%%  Purpose:        Circuit File for OAAOOA2124
%%
%%  ************    LaTeX with circdia.sty package      ***************
%%
%%  ///////////////////////////////////////////////////////////////////
%%
%%  Copyright (c) 2018 - 2022 by
%%                  chipforge <stdcelllib@nospam.chipforge.org>
%%  All rights reserved.
%%
%%      This Standard Cell Library is licensed under the Libre Silicon
%%      public license; you can redistribute it and/or modify it under
%%      the terms of the Libre Silicon public license as published by
%%      the Libre Silicon alliance, either version 1 of the License, or
%%      (at your option) any later version.
%%
%%      This design is distributed in the hope that it will be useful,
%%      but WITHOUT ANY WARRANTY; without even the implied warranty of
%%      MERCHANTABILITY or FITNESS FOR A PARTICULAR PURPOSE.
%%      See the Libre Silicon Public License for more details.
%%
%%  ///////////////////////////////////////////////////////////////////
\begin{circuitdiagram}[draft]{38}{22}

    \usgate
    % ----  1st column  ----
    \pin{1}{1}{L}{A}
    \pin{1}{5}{L}{A1}
    \gate[\inputs{2}]{or}{5}{3}{R}{}{}

    % ----  2nd column  ----
    \pin{8}{7}{L}{B}
    \gate[\inputs{2}]{and}{12}{5}{R}{}{}

    \pin{8}{9}{L}{C}
    \pin{8}{13}{L}{C1}
    \gate[\inputs{2}]{and}{12}{11}{R}{}{}

    % ----  3rd column  ----
    \wire{16}{5}{16}{7}
    \gate[\inputs{2}]{or}{19}{9}{R}{}{}

    \pin{15}{15}{L}{D}
    \pin{15}{17}{L}{D1}
    \pin{15}{19}{L}{D2}
    \pin{15}{21}{L}{D3}
    \gate[\inputs{4}]{or}{19}{18}{R}{}{}

    % ----  4th column  ----
    \wire{23}{9}{23}{14}
    \gate[\inputs{2}]{nand}{26}{16}{R}{}{}

    % ----  5th column  ----
    \gate{not}{33}{16}{R}{}{}

    % ----  result ----
    \pin{37}{16}{R}{Z}

\end{circuitdiagram}

%%  ************    LibreSilicon's StdCellLibrary   *******************
%%
%%  Organisation:   Chipforge
%%                  Germany / European Union
%%
%%  Profile:        Chipforge focus on fine System-on-Chip Cores in
%%                  Verilog HDL Code which are easy understandable and
%%                  adjustable. For further information see
%%                          www.chipforge.org
%%                  there are projects from small cores up to PCBs, too.
%%
%%  File:           StdCellLib/Documents/Datasheets/Circuitry/OAAOOAI2132.tex
%%
%%  Purpose:        Circuit File for OAAOOAI2132
%%
%%  ************    LaTeX with circdia.sty package      ***************
%%
%%  ///////////////////////////////////////////////////////////////////
%%
%%  Copyright (c) 2018 - 2022 by
%%                  chipforge <stdcelllib@nospam.chipforge.org>
%%  All rights reserved.
%%
%%      This Standard Cell Library is licensed under the Libre Silicon
%%      public license; you can redistribute it and/or modify it under
%%      the terms of the Libre Silicon public license as published by
%%      the Libre Silicon alliance, either version 1 of the License, or
%%      (at your option) any later version.
%%
%%      This design is distributed in the hope that it will be useful,
%%      but WITHOUT ANY WARRANTY; without even the implied warranty of
%%      MERCHANTABILITY or FITNESS FOR A PARTICULAR PURPOSE.
%%      See the Libre Silicon Public License for more details.
%%
%%  ///////////////////////////////////////////////////////////////////
\begin{circuitdiagram}[draft]{32}{20}

    \usgate
    % ----  1st column  ----
    \pin{1}{1}{L}{A}
    \pin{1}{5}{L}{A1}
    \gate[\inputs{2}]{or}{5}{3}{R}{}{}

    % ----  2nd column  ----
    \pin{8}{7}{L}{B}
    \gate[\inputs{2}]{and}{12}{5}{R}{}{}

    \pin{8}{9}{L}{C}
    \pin{8}{11}{L}{C1}
    \pin{8}{13}{L}{C2}
    \gate[\inputs{3}]{and}{12}{11}{R}{}{}

    % ----  3rd column  ----
    \wire{16}{5}{16}{7}
    \gate[\inputs{2}]{or}{19}{9}{R}{}{}

    \pin{15}{15}{L}{D}
    \pin{15}{19}{L}{D1}
    \gate[\inputs{2}]{or}{19}{17}{R}{}{}

    % ----  4th column  ----
    \wire{23}{9}{23}{13}
    \gate[\inputs{2}]{nand}{26}{15}{R}{}{}

    % ----  result ----
    \pin{31}{15}{R}{Y}

\end{circuitdiagram}
 %%  ************    LibreSilicon's StdCellLibrary   *******************
%%
%%  Organisation:   Chipforge
%%                  Germany / European Union
%%
%%  Profile:        Chipforge focus on fine System-on-Chip Cores in
%%                  Verilog HDL Code which are easy understandable and
%%                  adjustable. For further information see
%%                          www.chipforge.org
%%                  there are projects from small cores up to PCBs, too.
%%
%%  File:           StdCellLib/Documents/Datasheets/Circuitry/OAAOOA2132.tex
%%
%%  Purpose:        Circuit File for OAAOOA2132
%%
%%  ************    LaTeX with circdia.sty package      ***************
%%
%%  ///////////////////////////////////////////////////////////////////
%%
%%  Copyright (c) 2018 - 2022 by
%%                  chipforge <stdcelllib@nospam.chipforge.org>
%%  All rights reserved.
%%
%%      This Standard Cell Library is licensed under the Libre Silicon
%%      public license; you can redistribute it and/or modify it under
%%      the terms of the Libre Silicon public license as published by
%%      the Libre Silicon alliance, either version 1 of the License, or
%%      (at your option) any later version.
%%
%%      This design is distributed in the hope that it will be useful,
%%      but WITHOUT ANY WARRANTY; without even the implied warranty of
%%      MERCHANTABILITY or FITNESS FOR A PARTICULAR PURPOSE.
%%      See the Libre Silicon Public License for more details.
%%
%%  ///////////////////////////////////////////////////////////////////
\begin{circuitdiagram}[draft]{38}{20}

    \usgate
    % ----  1st column  ----
    \pin{1}{1}{L}{A}
    \pin{1}{5}{L}{A1}
    \gate[\inputs{2}]{or}{5}{3}{R}{}{}

    % ----  2nd column  ----
    \pin{8}{7}{L}{B}
    \gate[\inputs{2}]{and}{12}{5}{R}{}{}

    \pin{8}{9}{L}{C}
    \pin{8}{11}{L}{C1}
    \pin{8}{13}{L}{C2}
    \gate[\inputs{3}]{and}{12}{11}{R}{}{}

    % ----  3rd column  ----
    \wire{16}{5}{16}{7}
    \gate[\inputs{2}]{or}{19}{9}{R}{}{}

    \pin{15}{15}{L}{D}
    \pin{15}{19}{L}{D1}
    \gate[\inputs{2}]{or}{19}{17}{R}{}{}

    % ----  4th column  ----
    \wire{23}{9}{23}{13}
    \gate[\inputs{2}]{nand}{26}{15}{R}{}{}

    % ----  5th column  ----
    \gate{not}{33}{15}{R}{}{}

    % ----  result ----
    \pin{37}{15}{R}{Z}

\end{circuitdiagram}

%%  ************    LibreSilicon's StdCellLibrary   *******************
%%
%%  Organisation:   Chipforge
%%                  Germany / European Union
%%
%%  Profile:        Chipforge focus on fine System-on-Chip Cores in
%%                  Verilog HDL Code which are easy understandable and
%%                  adjustable. For further information see
%%                          www.chipforge.org
%%                  there are projects from small cores up to PCBs, too.
%%
%%  File:           StdCellLib/Documents/Datasheets/Circuitry/OAAOOAI2222.tex
%%
%%  Purpose:        Circuit File for OAAOOAI2222
%%
%%  ************    LaTeX with circdia.sty package      ***************
%%
%%  ///////////////////////////////////////////////////////////////////
%%
%%  Copyright (c) 2018 - 2022 by
%%                  chipforge <stdcelllib@nospam.chipforge.org>
%%  All rights reserved.
%%
%%      This Standard Cell Library is licensed under the Libre Silicon
%%      public license; you can redistribute it and/or modify it under
%%      the terms of the Libre Silicon public license as published by
%%      the Libre Silicon alliance, either version 1 of the License, or
%%      (at your option) any later version.
%%
%%      This design is distributed in the hope that it will be useful,
%%      but WITHOUT ANY WARRANTY; without even the implied warranty of
%%      MERCHANTABILITY or FITNESS FOR A PARTICULAR PURPOSE.
%%      See the Libre Silicon Public License for more details.
%%
%%  ///////////////////////////////////////////////////////////////////
\begin{circuitdiagram}[draft]{32}{22}

    \usgate
    % ----  1st column  ----
    \pin{1}{1}{L}{A}
    \pin{1}{5}{L}{A1}
    \gate[\inputs{2}]{or}{5}{3}{R}{}{}

    % ----  2nd column  ----
    \wire{9}{3}{9}{5}
    \pin{8}{7}{L}{B}
    \pin{8}{9}{L}{B1}
    \gate[\inputs{3}]{and}{12}{7}{R}{}{}

    \pin{8}{11}{L}{C}
    \pin{8}{15}{L}{C1}
    \gate[\inputs{2}]{and}{12}{13}{R}{}{}

    % ----  3rd column  ----
    \wire{16}{7}{16}{9}
    \gate[\inputs{2}]{or}{19}{11}{R}{}{}

    \pin{15}{17}{L}{D}
    \pin{15}{21}{L}{D1}
    \gate[\inputs{2}]{or}{19}{19}{R}{}{}

    % ----  4th column  ----
    \wire{23}{11}{23}{15}
    \gate[\inputs{2}]{nand}{26}{17}{R}{}{}

    % ----  result ----
    \pin{31}{17}{R}{Y}

\end{circuitdiagram}
 %%  ************    LibreSilicon's StdCellLibrary   *******************
%%
%%  Organisation:   Chipforge
%%                  Germany / European Union
%%
%%  Profile:        Chipforge focus on fine System-on-Chip Cores in
%%                  Verilog HDL Code which are easy understandable and
%%                  adjustable. For further information see
%%                          www.chipforge.org
%%                  there are projects from small cores up to PCBs, too.
%%
%%  File:           StdCellLib/Documents/Datasheets/Circuitry/OAAOOA2222.tex
%%
%%  Purpose:        Circuit File for OAAOOA2222
%%
%%  ************    LaTeX with circdia.sty package      ***************
%%
%%  ///////////////////////////////////////////////////////////////////
%%
%%  Copyright (c) 2018 - 2022 by
%%                  chipforge <stdcelllib@nospam.chipforge.org>
%%  All rights reserved.
%%
%%      This Standard Cell Library is licensed under the Libre Silicon
%%      public license; you can redistribute it and/or modify it under
%%      the terms of the Libre Silicon public license as published by
%%      the Libre Silicon alliance, either version 1 of the License, or
%%      (at your option) any later version.
%%
%%      This design is distributed in the hope that it will be useful,
%%      but WITHOUT ANY WARRANTY; without even the implied warranty of
%%      MERCHANTABILITY or FITNESS FOR A PARTICULAR PURPOSE.
%%      See the Libre Silicon Public License for more details.
%%
%%  ///////////////////////////////////////////////////////////////////
\begin{circuitdiagram}[draft]{38}{22}

    \usgate
    % ----  1st column  ----
    \pin{1}{1}{L}{A}
    \pin{1}{5}{L}{A1}
    \gate[\inputs{2}]{or}{5}{3}{R}{}{}

    % ----  2nd column  ----
    \wire{9}{3}{9}{5}
    \pin{8}{7}{L}{B}
    \pin{8}{9}{L}{B1}
    \gate[\inputs{3}]{and}{12}{7}{R}{}{}

    \pin{8}{11}{L}{C}
    \pin{8}{15}{L}{C1}
    \gate[\inputs{2}]{and}{12}{13}{R}{}{}

    % ----  3rd column  ----
    \wire{16}{7}{16}{9}
    \gate[\inputs{2}]{or}{19}{11}{R}{}{}

    \pin{15}{17}{L}{D}
    \pin{15}{21}{L}{D1}
    \gate[\inputs{2}]{or}{19}{19}{R}{}{}

    % ----  4th column  ----
    \wire{23}{11}{23}{15}
    \gate[\inputs{2}]{nand}{26}{17}{R}{}{}

    % ----  last column ----
    \gate{not}{33}{17}{R}{}{}

    % ----  result ----
    \pin{37}{17}{R}{Z}

\end{circuitdiagram}

%%  ************    LibreSilicon's StdCellLibrary   *******************
%%
%%  Organisation:   Chipforge
%%                  Germany / European Union
%%
%%  Profile:        Chipforge focus on fine System-on-Chip Cores in
%%                  Verilog HDL Code which are easy understandable and
%%                  adjustable. For further information see
%%                          www.chipforge.org
%%                  there are projects from small cores up to PCBs, too.
%%
%%  File:           StdCellLib/Documents/Datasheets/Circuitry/OAAOOAI2232.tex
%%
%%  Purpose:        Circuit File for OAAOOAI2232
%%
%%  ************    LaTeX with circdia.sty package      ***************
%%
%%  ///////////////////////////////////////////////////////////////////
%%
%%  Copyright (c) 2018 - 2022 by
%%                  chipforge <stdcelllib@nospam.chipforge.org>
%%  All rights reserved.
%%
%%      This Standard Cell Library is licensed under the Libre Silicon
%%      public license; you can redistribute it and/or modify it under
%%      the terms of the Libre Silicon public license as published by
%%      the Libre Silicon alliance, either version 1 of the License, or
%%      (at your option) any later version.
%%
%%      This design is distributed in the hope that it will be useful,
%%      but WITHOUT ANY WARRANTY; without even the implied warranty of
%%      MERCHANTABILITY or FITNESS FOR A PARTICULAR PURPOSE.
%%      See the Libre Silicon Public License for more details.
%%
%%  ///////////////////////////////////////////////////////////////////
\begin{circuitdiagram}[draft]{32}{22}

    \usgate
    % ----  1st column  ----
    \pin{1}{1}{L}{A}
    \pin{1}{5}{L}{A1}
    \gate[\inputs{2}]{or}{5}{3}{R}{}{}

    % ----  2nd column  ----
    \wire{9}{3}{9}{5}
    \pin{8}{7}{L}{B}
    \pin{8}{9}{L}{B1}
    \gate[\inputs{3}]{and}{12}{7}{R}{}{}

    \pin{8}{11}{L}{C}
    \pin{8}{13}{L}{C1}
    \pin{8}{15}{L}{C2}
    \gate[\inputs{3}]{and}{12}{13}{R}{}{}

    % ----  3rd column  ----
    \wire{16}{7}{16}{9}
    \gate[\inputs{2}]{or}{19}{11}{R}{}{}

    \pin{15}{17}{L}{D}
    \pin{15}{21}{L}{D1}
    \gate[\inputs{2}]{or}{19}{19}{R}{}{}

    % ----  4th column  ----
    \wire{23}{11}{23}{15}
    \gate[\inputs{2}]{nand}{26}{17}{R}{}{}

    % ----  result ----
    \pin{31}{17}{R}{Y}

\end{circuitdiagram}
 %%  ************    LibreSilicon's StdCellLibrary   *******************
%%
%%  Organisation:   Chipforge
%%                  Germany / European Union
%%
%%  Profile:        Chipforge focus on fine System-on-Chip Cores in
%%                  Verilog HDL Code which are easy understandable and
%%                  adjustable. For further information see
%%                          www.chipforge.org
%%                  there are projects from small cores up to PCBs, too.
%%
%%  File:           StdCellLib/Documents/Datasheets/Circuitry/OAAOOA2232.tex
%%
%%  Purpose:        Circuit File for OAAOOA2232
%%
%%  ************    LaTeX with circdia.sty package      ***************
%%
%%  ///////////////////////////////////////////////////////////////////
%%
%%  Copyright (c) 2018 - 2022 by
%%                  chipforge <stdcelllib@nospam.chipforge.org>
%%  All rights reserved.
%%
%%      This Standard Cell Library is licensed under the Libre Silicon
%%      public license; you can redistribute it and/or modify it under
%%      the terms of the Libre Silicon public license as published by
%%      the Libre Silicon alliance, either version 1 of the License, or
%%      (at your option) any later version.
%%
%%      This design is distributed in the hope that it will be useful,
%%      but WITHOUT ANY WARRANTY; without even the implied warranty of
%%      MERCHANTABILITY or FITNESS FOR A PARTICULAR PURPOSE.
%%      See the Libre Silicon Public License for more details.
%%
%%  ///////////////////////////////////////////////////////////////////
\begin{circuitdiagram}[draft]{38}{22}

    \usgate
    % ----  1st column  ----
    \pin{1}{1}{L}{A}
    \pin{1}{5}{L}{A1}
    \gate[\inputs{2}]{or}{5}{3}{R}{}{}

    % ----  2nd column  ----
    \wire{9}{3}{9}{5}
    \pin{8}{7}{L}{B}
    \pin{8}{9}{L}{B1}
    \gate[\inputs{3}]{and}{12}{7}{R}{}{}

    \pin{8}{11}{L}{C}
    \pin{8}{13}{L}{C1}
    \pin{8}{15}{L}{C2}
    \gate[\inputs{3}]{and}{12}{13}{R}{}{}

    % ----  3rd column  ----
    \wire{16}{7}{16}{9}
    \gate[\inputs{2}]{or}{19}{11}{R}{}{}

    \pin{15}{17}{L}{D}
    \pin{15}{21}{L}{D1}
    \gate[\inputs{2}]{or}{19}{19}{R}{}{}

    % ----  4th column  ----
    \wire{23}{11}{23}{15}
    \gate[\inputs{2}]{nand}{26}{17}{R}{}{}

    % ----  last column ----
    \gate{not}{33}{17}{R}{}{}

    % ----  result ----
    \pin{37}{17}{R}{Z}

\end{circuitdiagram}

%%  ************    LibreSilicon's StdCellLibrary   *******************
%%
%%  Organisation:   Chipforge
%%                  Germany / European Union
%%
%%  Profile:        Chipforge focus on fine System-on-Chip Cores in
%%                  Verilog HDL Code which are easy understandable and
%%                  adjustable. For further information see
%%                          www.chipforge.org
%%                  there are projects from small cores up to PCBs, too.
%%
%%  File:           StdCellLib/Documents/Datasheets/Circuitry/OAAOOAI3122.tex
%%
%%  Purpose:        Circuit File for OAAOOAI3122
%%
%%  ************    LaTeX with circdia.sty package      ***************
%%
%%  ///////////////////////////////////////////////////////////////////
%%
%%  Copyright (c) 2018 - 2022 by
%%                  chipforge <stdcelllib@nospam.chipforge.org>
%%  All rights reserved.
%%
%%      This Standard Cell Library is licensed under the Libre Silicon
%%      public license; you can redistribute it and/or modify it under
%%      the terms of the Libre Silicon public license as published by
%%      the Libre Silicon alliance, either version 1 of the License, or
%%      (at your option) any later version.
%%
%%      This design is distributed in the hope that it will be useful,
%%      but WITHOUT ANY WARRANTY; without even the implied warranty of
%%      MERCHANTABILITY or FITNESS FOR A PARTICULAR PURPOSE.
%%      See the Libre Silicon Public License for more details.
%%
%%  ///////////////////////////////////////////////////////////////////
\begin{circuitdiagram}[draft]{32}{20}

    \usgate
    % ----  1st column  ----
    \pin{1}{1}{L}{A}
    \pin{1}{3}{L}{A1}
    \pin{1}{5}{L}{A2}
    \gate[\inputs{3}]{or}{5}{3}{R}{}{}

    % ----  2nd column  ----
    \pin{8}{7}{L}{B}
    \gate[\inputs{2}]{and}{12}{5}{R}{}{}

    \pin{8}{9}{L}{C}
    \pin{8}{13}{L}{C1}
    \gate[\inputs{2}]{and}{12}{11}{R}{}{}

    % ----  3rd column  ----
    \wire{16}{5}{16}{7}
    \gate[\inputs{2}]{or}{19}{9}{R}{}{}

    \pin{15}{15}{L}{D}
    \pin{15}{19}{L}{D1}
    \gate[\inputs{2}]{or}{19}{17}{R}{}{}

    % ----  4th column  ----
    \wire{23}{9}{23}{13}
    \gate[\inputs{2}]{nand}{26}{15}{R}{}{}

    % ----  result ----
    \pin{31}{15}{R}{Y}

\end{circuitdiagram}
 %%  ************    LibreSilicon's StdCellLibrary   *******************
%%
%%  Organisation:   Chipforge
%%                  Germany / European Union
%%
%%  Profile:        Chipforge focus on fine System-on-Chip Cores in
%%                  Verilog HDL Code which are easy understandable and
%%                  adjustable. For further information see
%%                          www.chipforge.org
%%                  there are projects from small cores up to PCBs, too.
%%
%%  File:           StdCellLib/Documents/Datasheets/Circuitry/OAAOOA3122.tex
%%
%%  Purpose:        Circuit File for OAAOOA3122
%%
%%  ************    LaTeX with circdia.sty package      ***************
%%
%%  ///////////////////////////////////////////////////////////////////
%%
%%  Copyright (c) 2018 - 2022 by
%%                  chipforge <stdcelllib@nospam.chipforge.org>
%%  All rights reserved.
%%
%%      This Standard Cell Library is licensed under the Libre Silicon
%%      public license; you can redistribute it and/or modify it under
%%      the terms of the Libre Silicon public license as published by
%%      the Libre Silicon alliance, either version 1 of the License, or
%%      (at your option) any later version.
%%
%%      This design is distributed in the hope that it will be useful,
%%      but WITHOUT ANY WARRANTY; without even the implied warranty of
%%      MERCHANTABILITY or FITNESS FOR A PARTICULAR PURPOSE.
%%      See the Libre Silicon Public License for more details.
%%
%%  ///////////////////////////////////////////////////////////////////
\begin{circuitdiagram}[draft]{38}{20}

    \usgate
    % ----  1st column  ----
    \pin{1}{1}{L}{A}
    \pin{1}{3}{L}{A1}
    \pin{1}{5}{L}{A2}
    \gate[\inputs{3}]{or}{5}{3}{R}{}{}

    % ----  2nd column  ----
    \pin{8}{7}{L}{B}
    \gate[\inputs{2}]{and}{12}{5}{R}{}{}

    \pin{8}{9}{L}{C}
    \pin{8}{13}{L}{C1}
    \gate[\inputs{2}]{and}{12}{11}{R}{}{}

    % ----  3rd column  ----
    \wire{16}{5}{16}{7}
    \gate[\inputs{2}]{or}{19}{9}{R}{}{}

    \pin{15}{15}{L}{D}
    \pin{15}{19}{L}{D1}
    \gate[\inputs{2}]{or}{19}{17}{R}{}{}

    % ----  4th column  ----
    \wire{23}{9}{23}{13}
    \gate[\inputs{2}]{nand}{26}{15}{R}{}{}

    % ----  last column ----
    \gate{not}{33}{15}{R}{}{}

    % ----  result ----
    \pin{37}{15}{R}{Z}

\end{circuitdiagram}


%%  ************    LibreSilicon's StdCellLibrary   *******************
%%
%%  Organisation:   Chipforge
%%                  Germany / European Union
%%
%%  Profile:        Chipforge focus on fine System-on-Chip Cores in
%%                  Verilog HDL Code which are easy understandable and
%%                  adjustable. For further information see
%%                          www.chipforge.org
%%                  there are projects from small cores up to PCBs, too.
%%
%%  File:           StdCellLib/Documents/Datasheets/Circuitry/OAAOOAI21212.tex
%%
%%  Purpose:        Circuit File for OAAOOAI21212
%%
%%  ************    LaTeX with circdia.sty package      ***************
%%
%%  ///////////////////////////////////////////////////////////////////
%%
%%  Copyright (c) 2018 - 2022 by
%%                  chipforge <stdcelllib@nospam.chipforge.org>
%%  All rights reserved.
%%
%%      This Standard Cell Library is licensed under the Libre Silicon
%%      public license; you can redistribute it and/or modify it under
%%      the terms of the Libre Silicon public license as published by
%%      the Libre Silicon alliance, either version 1 of the License, or
%%      (at your option) any later version.
%%
%%      This design is distributed in the hope that it will be useful,
%%      but WITHOUT ANY WARRANTY; without even the implied warranty of
%%      MERCHANTABILITY or FITNESS FOR A PARTICULAR PURPOSE.
%%      See the Libre Silicon Public License for more details.
%%
%%  ///////////////////////////////////////////////////////////////////
\begin{circuitdiagram}[draft]{32}{22}

    \usgate
    % ----  1st column  ----
    \pin{1}{1}{L}{A}
    \pin{1}{5}{L}{A1}
    \gate[\inputs{2}]{or}{5}{3}{R}{}{}

    % ----  2nd column  ----
    \pin{8}{7}{L}{B}
    \gate[\inputs{2}]{and}{12}{5}{R}{}{}

    \pin{8}{9}{L}{C}
    \pin{8}{13}{L}{C1}
    \gate[\inputs{2}]{and}{12}{11}{R}{}{}

    % ----  3rd column  ----
    \wire{16}{5}{16}{9}
    \pin{15}{15}{L}{D}
    \wire{16}{13}{16}{15}
    \gate[\inputs{3}]{or}{19}{11}{R}{}{}

    \pin{15}{17}{L}{E}
    \pin{15}{21}{L}{E1}
    \gate[\inputs{2}]{or}{19}{19}{R}{}{}

    % ----  4th column  ----
    \wire{23}{11}{23}{15}
    \gate[\inputs{2}]{nand}{26}{17}{R}{}{}

    % ----  result ----
    \pin{31}{17}{R}{Y}

\end{circuitdiagram}
 %%  ************    LibreSilicon's StdCellLibrary   *******************
%%
%%  Organisation:   Chipforge
%%                  Germany / European Union
%%
%%  Profile:        Chipforge focus on fine System-on-Chip Cores in
%%                  Verilog HDL Code which are easy understandable and
%%                  adjustable. For further information see
%%                          www.chipforge.org
%%                  there are projects from small cores up to PCBs, too.
%%
%%  File:           StdCellLib/Documents/Datasheets/Circuitry/OAAOOA21212.tex
%%
%%  Purpose:        Circuit File for OAAOOA21212
%%
%%  ************    LaTeX with circdia.sty package      ***************
%%
%%  ///////////////////////////////////////////////////////////////////
%%
%%  Copyright (c) 2018 - 2022 by
%%                  chipforge <stdcelllib@nospam.chipforge.org>
%%  All rights reserved.
%%
%%      This Standard Cell Library is licensed under the Libre Silicon
%%      public license; you can redistribute it and/or modify it under
%%      the terms of the Libre Silicon public license as published by
%%      the Libre Silicon alliance, either version 1 of the License, or
%%      (at your option) any later version.
%%
%%      This design is distributed in the hope that it will be useful,
%%      but WITHOUT ANY WARRANTY; without even the implied warranty of
%%      MERCHANTABILITY or FITNESS FOR A PARTICULAR PURPOSE.
%%      See the Libre Silicon Public License for more details.
%%
%%  ///////////////////////////////////////////////////////////////////
\begin{circuitdiagram}[draft]{38}{22}

    \usgate
    % ----  1st column  ----
    \pin{1}{1}{L}{A}
    \pin{1}{5}{L}{A1}
    \gate[\inputs{2}]{or}{5}{3}{R}{}{}

    % ----  2nd column  ----
    \pin{8}{7}{L}{B}
    \gate[\inputs{2}]{and}{12}{5}{R}{}{}

    \pin{8}{9}{L}{C}
    \pin{8}{13}{L}{C1}
    \gate[\inputs{2}]{and}{12}{11}{R}{}{}

    % ----  3rd column  ----
    \wire{16}{5}{16}{9}
    \pin{15}{15}{L}{D}
    \wire{16}{13}{16}{15}
    \gate[\inputs{3}]{or}{19}{11}{R}{}{}

    \pin{15}{17}{L}{E}
    \pin{15}{21}{L}{E1}
    \gate[\inputs{2}]{or}{19}{19}{R}{}{}

    % ----  4th column  ----
    \wire{23}{11}{23}{15}
    \gate[\inputs{2}]{nand}{26}{17}{R}{}{}

    % ----  last column ----
    \gate{not}{33}{17}{R}{}{}

    % ----  result ----
    \pin{37}{17}{R}{Z}

\end{circuitdiagram}

%%  ************    LibreSilicon's StdCellLibrary   *******************
%%
%%  Organisation:   Chipforge
%%                  Germany / European Union
%%
%%  Profile:        Chipforge focus on fine System-on-Chip Cores in
%%                  Verilog HDL Code which are easy understandable and
%%                  adjustable. For further information see
%%                          www.chipforge.org
%%                  there are projects from small cores up to PCBs, too.
%%
%%  File:           StdCellLib/Documents/Datasheets/Circuitry/OAAOOAI21221.tex
%%
%%  Purpose:        Circuit File for OAAOOAI21221
%%
%%  ************    LaTeX with circdia.sty package      ***************
%%
%%  ///////////////////////////////////////////////////////////////////
%%
%%  Copyright (c) 2018 - 2022 by
%%                  chipforge <stdcelllib@nospam.chipforge.org>
%%  All rights reserved.
%%
%%      This Standard Cell Library is licensed under the Libre Silicon
%%      public license; you can redistribute it and/or modify it under
%%      the terms of the Libre Silicon public license as published by
%%      the Libre Silicon alliance, either version 1 of the License, or
%%      (at your option) any later version.
%%
%%      This design is distributed in the hope that it will be useful,
%%      but WITHOUT ANY WARRANTY; without even the implied warranty of
%%      MERCHANTABILITY or FITNESS FOR A PARTICULAR PURPOSE.
%%      See the Libre Silicon Public License for more details.
%%
%%  ///////////////////////////////////////////////////////////////////
\begin{circuitdiagram}[draft]{32}{22}

    \usgate
    % ----  1st column  ----
    \pin{1}{1}{L}{A}
    \pin{1}{5}{L}{A1}
    \gate[\inputs{2}]{or}{5}{3}{R}{}{}

    % ----  2nd column  ----
    \pin{8}{7}{L}{B}
    \gate[\inputs{2}]{and}{12}{5}{R}{}{}

    \pin{8}{9}{L}{C}
    \pin{8}{13}{L}{C1}
    \gate[\inputs{2}]{and}{12}{11}{R}{}{}

    % ----  3rd column  ----
    \wire{16}{5}{16}{7}
    \gate[\inputs{2}]{or}{19}{9}{R}{}{}

    \pin{15}{15}{L}{D}
    \pin{15}{19}{L}{D1}
    \gate[\inputs{2}]{or}{19}{17}{R}{}{}

    % ----  4th column  ----
    \pin{22}{21}{L}{E}
    \wire{23}{19}{23}{21}
    \wire{23}{9}{23}{15}
    \gate[\inputs{3}]{nand}{26}{17}{R}{}{}

    % ----  result ----
    \pin{31}{17}{R}{Y}

\end{circuitdiagram}
 %%  ************    LibreSilicon's StdCellLibrary   *******************
%%
%%  Organisation:   Chipforge
%%                  Germany / European Union
%%
%%  Profile:        Chipforge focus on fine System-on-Chip Cores in
%%                  Verilog HDL Code which are easy understandable and
%%                  adjustable. For further information see
%%                          www.chipforge.org
%%                  there are projects from small cores up to PCBs, too.
%%
%%  File:           StdCellLib/Documents/Datasheets/Circuitry/OAAOOA21221.tex
%%
%%  Purpose:        Circuit File for OAAOOA21221
%%
%%  ************    LaTeX with circdia.sty package      ***************
%%
%%  ///////////////////////////////////////////////////////////////////
%%
%%  Copyright (c) 2018 - 2022 by
%%                  chipforge <stdcelllib@nospam.chipforge.org>
%%  All rights reserved.
%%
%%      This Standard Cell Library is licensed under the Libre Silicon
%%      public license; you can redistribute it and/or modify it under
%%      the terms of the Libre Silicon public license as published by
%%      the Libre Silicon alliance, either version 1 of the License, or
%%      (at your option) any later version.
%%
%%      This design is distributed in the hope that it will be useful,
%%      but WITHOUT ANY WARRANTY; without even the implied warranty of
%%      MERCHANTABILITY or FITNESS FOR A PARTICULAR PURPOSE.
%%      See the Libre Silicon Public License for more details.
%%
%%  ///////////////////////////////////////////////////////////////////
\begin{circuitdiagram}[draft]{38}{22}

    \usgate
    % ----  1st column  ----
    \pin{1}{1}{L}{A}
    \pin{1}{5}{L}{A1}
    \gate[\inputs{2}]{or}{5}{3}{R}{}{}

    % ----  2nd column  ----
    \pin{8}{7}{L}{B}
    \gate[\inputs{2}]{and}{12}{5}{R}{}{}

    \pin{8}{9}{L}{C}
    \pin{8}{13}{L}{C1}
    \gate[\inputs{2}]{and}{12}{11}{R}{}{}

    % ----  3rd column  ----
    \wire{16}{5}{16}{7}
    \gate[\inputs{2}]{or}{19}{9}{R}{}{}

    \pin{15}{15}{L}{D}
    \pin{15}{19}{L}{D1}
    \gate[\inputs{2}]{or}{19}{17}{R}{}{}

    % ----  4th column  ----
    \pin{22}{21}{L}{E}
    \wire{23}{19}{23}{21}
    \wire{23}{9}{23}{15}
    \gate[\inputs{3}]{nand}{26}{17}{R}{}{}

    % ----  5th column  ----
    \gate{not}{33}{17}{R}{}{}

    % ----  result ----
    \pin{37}{17}{R}{Z}

\end{circuitdiagram}

%%  ************    LibreSilicon's StdCellLibrary   *******************
%%
%%  Organisation:   Chipforge
%%                  Germany / European Union
%%
%%  Profile:        Chipforge focus on fine System-on-Chip Cores in
%%                  Verilog HDL Code which are easy understandable and
%%                  adjustable. For further information see
%%                          www.chipforge.org
%%                  there are projects from small cores up to PCBs, too.
%%
%%  File:           StdCellLib/Documents/Datasheets/Circuitry/OAAOOAI21231.tex
%%
%%  Purpose:        Circuit File for OAAOOAI21231
%%
%%  ************    LaTeX with circdia.sty package      ***************
%%
%%  ///////////////////////////////////////////////////////////////////
%%
%%  Copyright (c) 2018 - 2022 by
%%                  chipforge <stdcelllib@nospam.chipforge.org>
%%  All rights reserved.
%%
%%      This Standard Cell Library is licensed under the Libre Silicon
%%      public license; you can redistribute it and/or modify it under
%%      the terms of the Libre Silicon public license as published by
%%      the Libre Silicon alliance, either version 1 of the License, or
%%      (at your option) any later version.
%%
%%      This design is distributed in the hope that it will be useful,
%%      but WITHOUT ANY WARRANTY; without even the implied warranty of
%%      MERCHANTABILITY or FITNESS FOR A PARTICULAR PURPOSE.
%%      See the Libre Silicon Public License for more details.
%%
%%  ///////////////////////////////////////////////////////////////////
\begin{circuitdiagram}[draft]{32}{22}

    \usgate
    % ----  1st column  ----
    \pin{1}{1}{L}{A}
    \pin{1}{5}{L}{A1}
    \gate[\inputs{2}]{or}{5}{3}{R}{}{}

    % ----  2nd column  ----
    \pin{8}{7}{L}{B}
    \gate[\inputs{2}]{and}{12}{5}{R}{}{}

    \pin{8}{9}{L}{C}
    \pin{8}{13}{L}{C1}
    \gate[\inputs{2}]{and}{12}{11}{R}{}{}

    % ----  3rd column  ----
    \wire{16}{5}{16}{7}
    \gate[\inputs{2}]{or}{19}{9}{R}{}{}

    \pin{15}{15}{L}{D}
    \pin{15}{17}{L}{D1}
    \pin{15}{19}{L}{D2}
    \gate[\inputs{3}]{or}{19}{17}{R}{}{}

    % ----  4th column  ----
    \pin{22}{21}{L}{E}
    \wire{23}{19}{23}{21}
    \wire{23}{9}{23}{15}
    \gate[\inputs{3}]{nand}{26}{17}{R}{}{}

    % ----  result ----
    \pin{31}{17}{R}{Y}

\end{circuitdiagram}
 %%  ************    LibreSilicon's StdCellLibrary   *******************
%%
%%  Organisation:   Chipforge
%%                  Germany / European Union
%%
%%  Profile:        Chipforge focus on fine System-on-Chip Cores in
%%                  Verilog HDL Code which are easy understandable and
%%                  adjustable. For further information see
%%                          www.chipforge.org
%%                  there are projects from small cores up to PCBs, too.
%%
%%  File:           StdCellLib/Documents/Datasheets/Circuitry/OAAOOA21231.tex
%%
%%  Purpose:        Circuit File for OAAOOA21231
%%
%%  ************    LaTeX with circdia.sty package      ***************
%%
%%  ///////////////////////////////////////////////////////////////////
%%
%%  Copyright (c) 2018 - 2022 by
%%                  chipforge <stdcelllib@nospam.chipforge.org>
%%  All rights reserved.
%%
%%      This Standard Cell Library is licensed under the Libre Silicon
%%      public license; you can redistribute it and/or modify it under
%%      the terms of the Libre Silicon public license as published by
%%      the Libre Silicon alliance, either version 1 of the License, or
%%      (at your option) any later version.
%%
%%      This design is distributed in the hope that it will be useful,
%%      but WITHOUT ANY WARRANTY; without even the implied warranty of
%%      MERCHANTABILITY or FITNESS FOR A PARTICULAR PURPOSE.
%%      See the Libre Silicon Public License for more details.
%%
%%  ///////////////////////////////////////////////////////////////////
\begin{circuitdiagram}[draft]{38}{22}

    \usgate
    % ----  1st column  ----
    \pin{1}{1}{L}{A}
    \pin{1}{5}{L}{A1}
    \gate[\inputs{2}]{or}{5}{3}{R}{}{}

    % ----  2nd column  ----
    \pin{8}{7}{L}{B}
    \gate[\inputs{2}]{and}{12}{5}{R}{}{}

    \pin{8}{9}{L}{C}
    \pin{8}{13}{L}{C1}
    \gate[\inputs{2}]{and}{12}{11}{R}{}{}

    % ----  3rd column  ----
    \wire{16}{5}{16}{7}
    \gate[\inputs{2}]{or}{19}{9}{R}{}{}

    \pin{15}{15}{L}{D}
    \pin{15}{17}{L}{D1}
    \pin{15}{19}{L}{D2}
    \gate[\inputs{3}]{or}{19}{17}{R}{}{}

    % ----  4th column  ----
    \pin{22}{21}{L}{E}
    \wire{23}{19}{23}{21}
    \wire{23}{9}{23}{15}
    \gate[\inputs{3}]{nand}{26}{17}{R}{}{}

    % ----  5th column  ----
    \gate{not}{33}{17}{R}{}{}

    % ----  result ----
    \pin{37}{17}{R}{Z}

\end{circuitdiagram}

%%  ************    LibreSilicon's StdCellLibrary   *******************
%%
%%  Organisation:   Chipforge
%%                  Germany / European Union
%%
%%  Profile:        Chipforge focus on fine System-on-Chip Cores in
%%                  Verilog HDL Code which are easy understandable and
%%                  adjustable. For further information see
%%                          www.chipforge.org
%%                  there are projects from small cores up to PCBs, too.
%%
%%  File:           StdCellLib/Documents/Datasheets/Circuitry/OAAOOAI21241.tex
%%
%%  Purpose:        Circuit File for OAAOOAI21241
%%
%%  ************    LaTeX with circdia.sty package      ***************
%%
%%  ///////////////////////////////////////////////////////////////////
%%
%%  Copyright (c) 2018 - 2022 by
%%                  chipforge <stdcelllib@nospam.chipforge.org>
%%  All rights reserved.
%%
%%      This Standard Cell Library is licensed under the Libre Silicon
%%      public license; you can redistribute it and/or modify it under
%%      the terms of the Libre Silicon public license as published by
%%      the Libre Silicon alliance, either version 1 of the License, or
%%      (at your option) any later version.
%%
%%      This design is distributed in the hope that it will be useful,
%%      but WITHOUT ANY WARRANTY; without even the implied warranty of
%%      MERCHANTABILITY or FITNESS FOR A PARTICULAR PURPOSE.
%%      See the Libre Silicon Public License for more details.
%%
%%  ///////////////////////////////////////////////////////////////////
\begin{circuitdiagram}[draft]{32}{23}

    \usgate
    % ----  1st column  ----
    \pin{1}{1}{L}{A}
    \pin{1}{5}{L}{A1}
    \gate[\inputs{2}]{or}{5}{3}{R}{}{}

    % ----  2nd column  ----
    \pin{8}{7}{L}{B}
    \gate[\inputs{2}]{and}{12}{5}{R}{}{}

    \pin{8}{9}{L}{C}
    \pin{8}{13}{L}{C1}
    \gate[\inputs{2}]{and}{12}{11}{R}{}{}

    % ----  3rd column  ----
    \wire{16}{5}{16}{7}
    \gate[\inputs{2}]{or}{19}{9}{R}{}{}

    \pin{15}{15}{L}{D}
    \pin{15}{17}{L}{D1}
    \pin{15}{19}{L}{D2}
    \pin{15}{21}{L}{D3}
    \gate[\inputs{4}]{or}{19}{18}{R}{}{}

    % ----  4th column  ----
    \wire{23}{9}{23}{16}
    \pin{22}{22}{L}{E}
    \wire{23}{20}{23}{22}
    \gate[\inputs{3}]{nand}{26}{18}{R}{}{}

    % ----  result ----
    \pin{31}{18}{R}{Y}

\end{circuitdiagram}
 %%  ************    LibreSilicon's StdCellLibrary   *******************
%%
%%  Organisation:   Chipforge
%%                  Germany / European Union
%%
%%  Profile:        Chipforge focus on fine System-on-Chip Cores in
%%                  Verilog HDL Code which are easy understandable and
%%                  adjustable. For further information see
%%                          www.chipforge.org
%%                  there are projects from small cores up to PCBs, too.
%%
%%  File:           StdCellLib/Documents/Datasheets/Circuitry/OAAOOA21241.tex
%%
%%  Purpose:        Circuit File for OAAOOA21241
%%
%%  ************    LaTeX with circdia.sty package      ***************
%%
%%  ///////////////////////////////////////////////////////////////////
%%
%%  Copyright (c) 2018 - 2022 by
%%                  chipforge <stdcelllib@nospam.chipforge.org>
%%  All rights reserved.
%%
%%      This Standard Cell Library is licensed under the Libre Silicon
%%      public license; you can redistribute it and/or modify it under
%%      the terms of the Libre Silicon public license as published by
%%      the Libre Silicon alliance, either version 1 of the License, or
%%      (at your option) any later version.
%%
%%      This design is distributed in the hope that it will be useful,
%%      but WITHOUT ANY WARRANTY; without even the implied warranty of
%%      MERCHANTABILITY or FITNESS FOR A PARTICULAR PURPOSE.
%%      See the Libre Silicon Public License for more details.
%%
%%  ///////////////////////////////////////////////////////////////////
\begin{circuitdiagram}[draft]{38}{23}

    \usgate
    % ----  1st column  ----
    \pin{1}{1}{L}{A}
    \pin{1}{5}{L}{A1}
    \gate[\inputs{2}]{or}{5}{3}{R}{}{}

    % ----  2nd column  ----
    \pin{8}{7}{L}{B}
    \gate[\inputs{2}]{and}{12}{5}{R}{}{}

    \pin{8}{9}{L}{C}
    \pin{8}{13}{L}{C1}
    \gate[\inputs{2}]{and}{12}{11}{R}{}{}

    % ----  3rd column  ----
    \wire{16}{5}{16}{7}
    \gate[\inputs{2}]{or}{19}{9}{R}{}{}

    \pin{15}{15}{L}{D}
    \pin{15}{17}{L}{D1}
    \pin{15}{19}{L}{D2}
    \pin{15}{21}{L}{D3}
    \gate[\inputs{4}]{or}{19}{18}{R}{}{}

    % ----  4th column  ----
    \wire{23}{9}{23}{16}
    \pin{22}{22}{L}{E}
    \wire{23}{20}{23}{22}
    \gate[\inputs{3}]{nand}{26}{18}{R}{}{}

    % ----  last column  ----
    \gate{not}{33}{18}{R}{}{}

    % ----  result ----
    \pin{37}{18}{R}{Z}

\end{circuitdiagram}



%%  ------------    five phases     -----------------------------------

%%  ************    LibreSilicon's StdCellLibrary   *******************
%%
%%  Organisation:   Chipforge
%%                  Germany / European Union
%%
%%  Profile:        Chipforge focus on fine System-on-Chip Cores in
%%                  Verilog HDL Code which are easy understandable and
%%                  adjustable. For further information see
%%                          www.chipforge.org
%%                  there are projects from small cores up to PCBs, too.
%%
%%  File:           StdCellLib/Documents/Book/section-OAOAOI_complex.tex
%%
%%  Purpose:        Section Level File for Standard Cell Library Documentation
%%
%%  ************    LaTeX with circdia.sty package      ***************
%%
%%  ///////////////////////////////////////////////////////////////////
%%
%%  Copyright (c) 2018 - 2022 by
%%                  chipforge <stdcelllib@nospam.chipforge.org>
%%  All rights reserved.
%%
%%      This Standard Cell Library is licensed under the Libre Silicon
%%      public license; you can redistribute it and/or modify it under
%%      the terms of the Libre Silicon public license as published by
%%      the Libre Silicon alliance, either version 1 of the License, or
%%      (at your option) any later version.
%%
%%      This design is distributed in the hope that it will be useful,
%%      but WITHOUT ANY WARRANTY; without even the implied warranty of
%%      MERCHANTABILITY or FITNESS FOR A PARTICULAR PURPOSE.
%%      See the Libre Silicon Public License for more details.
%%
%%  ///////////////////////////////////////////////////////////////////
\section{OR-AND-OR-AND-OR(-Invert) Complex Gates}


%%  ************    LibreSilicon's StdCellLibrary   *******************
%%
%%  Organisation:   Chipforge
%%                  Germany / European Union
%%
%%  Profile:        Chipforge focus on fine System-on-Chip Cores in
%%                  Verilog HDL Code which are easy understandable and
%%                  adjustable. For further information see
%%                          www.chipforge.org
%%                  there are projects from small cores up to PCBs, too.
%%
%%  File:           StdCellLib/Documents/section-AOAOAI_complex.tex
%%
%%  Purpose:        Section Level File for Standard Cell Library Documentation
%%
%%  ************    LaTeX with circdia.sty package      ***************
%%
%%  ///////////////////////////////////////////////////////////////////
%%
%%  Copyright (c) 2018 - 2022 by
%%                  chipforge <stdcelllib@nospam.chipforge.org>
%%  All rights reserved.
%%
%%      This Standard Cell Library is licensed under the Libre Silicon
%%      public license; you can redistribute it and/or modify it under
%%      the terms of the Libre Silicon public license as published by
%%      the Libre Silicon alliance, either version 1 of the License, or
%%      (at your option) any later version.
%%
%%      This design is distributed in the hope that it will be useful,
%%      but WITHOUT ANY WARRANTY; without even the implied warranty of
%%      MERCHANTABILITY or FITNESS FOR A PARTICULAR PURPOSE.
%%      See the Libre Silicon Public License for more details.
%%
%%  ///////////////////////////////////////////////////////////////////
\section{AND-OR-AND-OR-AND(-Invert) Complex Gates}

%%  ************    LibreSilicon's StdCellLibrary   *******************
%%
%%  Organisation:   Chipforge
%%                  Germany / European Union
%%
%%  Profile:        Chipforge focus on fine System-on-Chip Cores in
%%                  Verilog HDL Code which are easy understandable and
%%                  adjustable. For further information see
%%                          www.chipforge.org
%%                  there are projects from small cores up to PCBs, too.
%%
%%  File:           StdCellLib/Documents/Datasheets/Circuitry/AOAOAI21111.tex
%%
%%  Purpose:        Circuit File for AOAOAI21111
%%
%%  ************    LaTeX with circdia.sty package      ***************
%%
%%  ///////////////////////////////////////////////////////////////////
%%
%%  Copyright (c) 2018 - 2022 by
%%                  chipforge <stdcelllib@nospam.chipforge.org>
%%  All rights reserved.
%%
%%      This Standard Cell Library is licensed under the Libre Silicon
%%      public license; you can redistribute it and/or modify it under
%%      the terms of the Libre Silicon public license as published by
%%      the Libre Silicon alliance, either version 1 of the License, or
%%      (at your option) any later version.
%%
%%      This design is distributed in the hope that it will be useful,
%%      but WITHOUT ANY WARRANTY; without even the implied warranty of
%%      MERCHANTABILITY or FITNESS FOR A PARTICULAR PURPOSE.
%%      See the Libre Silicon Public License for more details.
%%
%%  ///////////////////////////////////////////////////////////////////
\begin{circuitdiagram}[draft]{39}{14}

    \usgate
    % ----  1st column  ----
    \pin{1}{1}{L}{A}
    \pin{1}{5}{L}{A1}
    \gate[\inputs{2}]{and}{5}{3}{R}{}{}

    % ----  2nd column  ----
    \pin{8}{7}{L}{B}
    \gate[\inputs{2}]{or}{12}{5}{R}{}{}

    % ----  3rd column  ----
    \pin{15}{9}{L}{C}
    \gate[\inputs{2}]{and}{19}{7}{R}{}{}

    % ----  4th column  ----
    \pin{22}{11}{L}{D}
    \gate[\inputs{2}]{or}{26}{9}{R}{}{}

    % ----  5th column  ----
    \pin{29}{13}{L}{E}
    \gate[\inputs{2}]{nand}{33}{11}{R}{}{}

    % ----  result ----
    \pin{38}{11}{R}{Y}

\end{circuitdiagram}
 %%  ************    LibreSilicon's StdCellLibrary   *******************
%%
%%  Organisation:   Chipforge
%%                  Germany / European Union
%%
%%  Profile:        Chipforge focus on fine System-on-Chip Cores in
%%                  Verilog HDL Code which are easy understandable and
%%                  adjustable. For further information see
%%                          www.chipforge.org
%%                  there are projects from small cores up to PCBs, too.
%%
%%  File:           StdCellLib/Documents/Datasheets/Circuitry/AOAOA21111.tex
%%
%%  Purpose:        Circuit File for AOAOA21111
%%
%%  ************    LaTeX with circdia.sty package      ***************
%%
%%  ///////////////////////////////////////////////////////////////////
%%
%%  Copyright (c) 2018 - 2022 by
%%                  chipforge <stdcelllib@nospam.chipforge.org>
%%  All rights reserved.
%%
%%      This Standard Cell Library is licensed under the Libre Silicon
%%      public license; you can redistribute it and/or modify it under
%%      the terms of the Libre Silicon public license as published by
%%      the Libre Silicon alliance, either version 1 of the License, or
%%      (at your option) any later version.
%%
%%      This design is distributed in the hope that it will be useful,
%%      but WITHOUT ANY WARRANTY; without even the implied warranty of
%%      MERCHANTABILITY or FITNESS FOR A PARTICULAR PURPOSE.
%%      See the Libre Silicon Public License for more details.
%%
%%  ///////////////////////////////////////////////////////////////////
\begin{circuitdiagram}[draft]{45}{14}

    \usgate
    % ----  1st column  ----
    \pin{1}{1}{L}{A}
    \pin{1}{5}{L}{A1}
    \gate[\inputs{2}]{and}{5}{3}{R}{}{}

    % ----  2nd column  ----
    \pin{8}{7}{L}{B}
    \gate[\inputs{2}]{or}{12}{5}{R}{}{}

    % ----  3rd column  ----
    \pin{15}{9}{L}{C}
    \gate[\inputs{2}]{and}{19}{7}{R}{}{}

    % ----  4th column  ----
    \pin{22}{11}{L}{D}
    \gate[\inputs{2}]{or}{26}{9}{R}{}{}

    % ----  5th column  ----
    \pin{29}{13}{L}{E}
    \gate[\inputs{2}]{nand}{33}{11}{R}{}{}

    % ----  6th column  ----
    \gate{not}{40}{11}{R}{}{}

    % ----  result ----
    \pin{44}{11}{R}{Z}

\end{circuitdiagram}

%%  ************    LibreSilicon's StdCellLibrary   *******************
%%
%%  Organisation:   Chipforge
%%                  Germany / European Union
%%
%%  Profile:        Chipforge focus on fine System-on-Chip Cores in
%%                  Verilog HDL Code which are easy understandable and
%%                  adjustable. For further information see
%%                          www.chipforge.org
%%                  there are projects from small cores up to PCBs, too.
%%
%%  File:           StdCellLib/Documents/Datasheets/Circuitry/AOAOAI21121.tex
%%
%%  Purpose:        Circuit File for AOAOAI21121
%%
%%  ************    LaTeX with circdia.sty package      ***************
%%
%%  ///////////////////////////////////////////////////////////////////
%%
%%  Copyright (c) 2018 - 2022 by
%%                  chipforge <stdcelllib@nospam.chipforge.org>
%%  All rights reserved.
%%
%%      This Standard Cell Library is licensed under the Libre Silicon
%%      public license; you can redistribute it and/or modify it under
%%      the terms of the Libre Silicon public license as published by
%%      the Libre Silicon alliance, either version 1 of the License, or
%%      (at your option) any later version.
%%
%%      This design is distributed in the hope that it will be useful,
%%      but WITHOUT ANY WARRANTY; without even the implied warranty of
%%      MERCHANTABILITY or FITNESS FOR A PARTICULAR PURPOSE.
%%      See the Libre Silicon Public License for more details.
%%
%%  ///////////////////////////////////////////////////////////////////
\begin{circuitdiagram}[draft]{39}{16}

    \usgate
    % ----  1st column  ----
    \pin{1}{1}{L}{A}
    \pin{1}{5}{L}{A1}
    \gate[\inputs{2}]{and}{5}{3}{R}{}{}

    % ----  2nd column  ----
    \pin{8}{7}{L}{B}
    \gate[\inputs{2}]{or}{12}{5}{R}{}{}

    % ----  3rd column  ----
    \pin{15}{9}{L}{C}
    \gate[\inputs{2}]{and}{19}{7}{R}{}{}

    % ----  4th column  ----
    \wire{23}{7}{23}{9}
    \pin{22}{11}{L}{D}
    \pin{22}{13}{L}{D1}
    \gate[\inputs{3}]{or}{26}{11}{R}{}{}

    % ----  5th column  ----
    \pin{29}{15}{L}{E}
    \gate[\inputs{2}]{nand}{33}{13}{R}{}{}

    % ----  result ----
    \pin{38}{13}{R}{Y}

\end{circuitdiagram}
 %%  ************    LibreSilicon's StdCellLibrary   *******************
%%
%%  Organisation:   Chipforge
%%                  Germany / European Union
%%
%%  Profile:        Chipforge focus on fine System-on-Chip Cores in
%%                  Verilog HDL Code which are easy understandable and
%%                  adjustable. For further information see
%%                          www.chipforge.org
%%                  there are projects from small cores up to PCBs, too.
%%
%%  File:           StdCellLib/Documents/Datasheets/Circuitry/AOAOA21121.tex
%%
%%  Purpose:        Circuit File for AOAOA21121
%%
%%  ************    LaTeX with circdia.sty package      ***************
%%
%%  ///////////////////////////////////////////////////////////////////
%%
%%  Copyright (c) 2018 - 2022 by
%%                  chipforge <stdcelllib@nospam.chipforge.org>
%%  All rights reserved.
%%
%%      This Standard Cell Library is licensed under the Libre Silicon
%%      public license; you can redistribute it and/or modify it under
%%      the terms of the Libre Silicon public license as published by
%%      the Libre Silicon alliance, either version 1 of the License, or
%%      (at your option) any later version.
%%
%%      This design is distributed in the hope that it will be useful,
%%      but WITHOUT ANY WARRANTY; without even the implied warranty of
%%      MERCHANTABILITY or FITNESS FOR A PARTICULAR PURPOSE.
%%      See the Libre Silicon Public License for more details.
%%
%%  ///////////////////////////////////////////////////////////////////
\begin{circuitdiagram}[draft]{45}{16}

    \usgate
    % ----  1st column  ----
    \pin{1}{1}{L}{A}
    \pin{1}{5}{L}{A1}
    \gate[\inputs{2}]{and}{5}{3}{R}{}{}

    % ----  2nd column  ----
    \pin{8}{7}{L}{B}
    \gate[\inputs{2}]{or}{12}{5}{R}{}{}

    % ----  3rd column  ----
    \pin{15}{9}{L}{C}
    \gate[\inputs{2}]{and}{19}{7}{R}{}{}

    % ----  4th column  ----
    \wire{23}{7}{23}{9}
    \pin{22}{11}{L}{D}
    \pin{22}{13}{L}{D1}
    \gate[\inputs{3}]{or}{26}{11}{R}{}{}

    % ----  5th column  ----
    \pin{29}{15}{L}{E}
    \gate[\inputs{2}]{nand}{33}{13}{R}{}{}

    % ----  6th column  ----
    \gate{not}{40}{13}{R}{}{}

    % ----  result ----
    \pin{44}{13}{R}{Z}

\end{circuitdiagram}


%%  ************    LibreSilicon's StdCellLibrary   *******************
%%
%%  Organisation:   Chipforge
%%                  Germany / European Union
%%
%%  Profile:        Chipforge focus on fine System-on-Chip Cores in
%%                  Verilog HDL Code which are easy understandable and
%%                  adjustable. For further information see
%%                          www.chipforge.org
%%                  there are projects from small cores up to PCBs, too.
%%
%%  File:           StdCellLib/Documents/LaTeX/section-AOAOOAI_complex.tex
%%
%%  Purpose:        Section Level File for Standard Cell Library Documentation
%%
%%  ************    LaTeX with circdia.sty package      ***************
%%
%%  ///////////////////////////////////////////////////////////////////
%%
%%  Copyright (c) 2018 - 2021 by
%%                  chipforge <stdcelllib@nospam.chipforge.org>
%%  All rights reserved.
%%
%%      This Standard Cell Library is licensed under the Libre Silicon
%%      public license; you can redistribute it and/or modify it under
%%      the terms of the Libre Silicon public license as published by
%%      the Libre Silicon alliance, either version 1 of the License, or
%%      (at your option) any later version.
%%
%%      This design is distributed in the hope that it will be useful,
%%      but WITHOUT ANY WARRANTY; without even the implied warranty of
%%      MERCHANTABILITY or FITNESS FOR A PARTICULAR PURPOSE.
%%      See the Libre Silicon Public License for more details.
%%
\section{AND-OR-AND-OR-OR-AND(-Invert) Complex Gates}


%%  ************    LibreSilicon's StdCellLibrary   *******************
%%
%%  Organisation:   Chipforge
%%                  Germany / European Union
%%
%%  Profile:        Chipforge focus on fine System-on-Chip Cores in
%%                  Verilog HDL Code which are easy understandable and
%%                  adjustable. For further information see
%%                          www.chipforge.org
%%                  there are projects from small cores up to PCBs, too.
%%
%%  File:           StdCellLib/Documents/section-OAOAAOI_complex.tex
%%
%%  Purpose:        Section Level File for Standard Cell Library Documentation
%%
%%  ************    LaTeX with circdia.sty package      ***************
%%
%%  ///////////////////////////////////////////////////////////////////
%%
%%  Copyright (c) 2018 - 2022 by
%%                  chipforge <stdcelllib@nospam.chipforge.org>
%%  All rights reserved.
%%
%%      This Standard Cell Library is licensed under the Libre Silicon
%%      public license; you can redistribute it and/or modify it under
%%      the terms of the Libre Silicon public license as published by
%%      the Libre Silicon alliance, either version 1 of the License, or
%%      (at your option) any later version.
%%
%%      This design is distributed in the hope that it will be useful,
%%      but WITHOUT ANY WARRANTY; without even the implied warranty of
%%      MERCHANTABILITY or FITNESS FOR A PARTICULAR PURPOSE.
%%      See the Libre Silicon Public License for more details.
%%
%%  ///////////////////////////////////////////////////////////////////
\section{OR-AND-OR-AND-AND-OR(-Invert) Complex Gates}

%%  ************    LibreSilicon's StdCellLibrary   *******************
%%
%%  Organisation:   Chipforge
%%                  Germany / European Union
%%
%%  Profile:        Chipforge focus on fine System-on-Chip Cores in
%%                  Verilog HDL Code which are easy understandable and
%%                  adjustable. For further information see
%%                          www.chipforge.org
%%                  there are projects from small cores up to PCBs, too.
%%
%%  File:           StdCellLib/Documents/Datasheets/Circuitry/OAOAOAI21112.tex
%%
%%  Purpose:        Circuit File for OAOAOAI21112
%%
%%  ************    LaTeX with circdia.sty package      ***************
%%
%%  ///////////////////////////////////////////////////////////////////
%%
%%  Copyright (c) 2018 - 2022 by
%%                  chipforge <stdcelllib@nospam.chipforge.org>
%%  All rights reserved.
%%
%%      This Standard Cell Library is licensed under the Libre Silicon
%%      public license; you can redistribute it and/or modify it under
%%      the terms of the Libre Silicon public license as published by
%%      the Libre Silicon alliance, either version 1 of the License, or
%%      (at your option) any later version.
%%
%%      This design is distributed in the hope that it will be useful,
%%      but WITHOUT ANY WARRANTY; without even the implied warranty of
%%      MERCHANTABILITY or FITNESS FOR A PARTICULAR PURPOSE.
%%      See the Libre Silicon Public License for more details.
%%
%%  ///////////////////////////////////////////////////////////////////
\begin{circuitdiagram}[draft]{39}{18}

    \usgate
    % ----  1st column  ----
    \pin{1}{1}{L}{A}
    \pin{1}{5}{L}{A1}
    \gate[\inputs{2}]{or}{5}{3}{R}{}{}

    % ----  2nd column  ----
    \pin{8}{7}{L}{B}
    \gate[\inputs{2}]{and}{12}{5}{R}{}{}

    % ----  3rd column  ----
    \pin{15}{9}{L}{C}
    \gate[\inputs{2}]{or}{19}{7}{R}{}{}

    % ----  4th column  ----
    \pin{22}{11}{L}{D}
    \gate[\inputs{2}]{and}{26}{9}{R}{}{}

    \wire{30}{9}{30}{11}
    \pin{22}{13}{L}{E}
    \pin{22}{17}{L}{E1}
    \gate[\inputs{2}]{and}{26}{15}{R}{}{}

    % ----  5th column  ----
    \gate[\inputs{2}]{nor}{33}{13}{R}{}{}

    % ----  result ----
    \pin{38}{13}{R}{Y}

\end{circuitdiagram}
 %%  ************    LibreSilicon's StdCellLibrary   *******************
%%
%%  Organisation:   Chipforge
%%                  Germany / European Union
%%
%%  Profile:        Chipforge focus on fine System-on-Chip Cores in
%%                  Verilog HDL Code which are easy understandable and
%%                  adjustable. For further information see
%%                          www.chipforge.org
%%                  there are projects from small cores up to PCBs, too.
%%
%%  File:           StdCellLib/Documents/Datasheets/Circuitry/OAOAOA21112.tex
%%
%%  Purpose:        Circuit File for OAOAOA21112
%%
%%  ************    LaTeX with circdia.sty package      ***************
%%
%%  ///////////////////////////////////////////////////////////////////
%%
%%  Copyright (c) 2018 - 2022 by
%%                  chipforge <stdcelllib@nospam.chipforge.org>
%%  All rights reserved.
%%
%%      This Standard Cell Library is licensed under the Libre Silicon
%%      public license; you can redistribute it and/or modify it under
%%      the terms of the Libre Silicon public license as published by
%%      the Libre Silicon alliance, either version 1 of the License, or
%%      (at your option) any later version.
%%
%%      This design is distributed in the hope that it will be useful,
%%      but WITHOUT ANY WARRANTY; without even the implied warranty of
%%      MERCHANTABILITY or FITNESS FOR A PARTICULAR PURPOSE.
%%      See the Libre Silicon Public License for more details.
%%
%%  ///////////////////////////////////////////////////////////////////
\begin{circuitdiagram}[draft]{45}{18}

    \usgate
    % ----  1st column  ----
    \pin{1}{1}{L}{A}
    \pin{1}{5}{L}{A1}
    \gate[\inputs{2}]{or}{5}{3}{R}{}{}

    % ----  2nd column  ----
    \pin{8}{7}{L}{B}
    \gate[\inputs{2}]{and}{12}{5}{R}{}{}

    % ----  3rd column  ----
    \pin{15}{9}{L}{C}
    \gate[\inputs{2}]{or}{19}{7}{R}{}{}

    % ----  4th column  ----
    \pin{22}{11}{L}{D}
    \gate[\inputs{2}]{and}{26}{9}{R}{}{}

    \wire{30}{9}{30}{11}
    \pin{22}{13}{L}{E}
    \pin{22}{17}{L}{E1}
    \gate[\inputs{2}]{and}{26}{15}{R}{}{}

    % ----  5th column  ----
    \gate[\inputs{2}]{nor}{33}{13}{R}{}{}

    % ----  6th column  ----
    \gate{not}{40}{13}{R}{}{}

    % ----  result ----
    \pin{44}{13}{R}{Z}

\end{circuitdiagram}


%%  ************    LibreSilicon's StdCellLibrary   *******************
%%
%%  Organisation:   Chipforge
%%                  Germany / European Union
%%
%%  Profile:        Chipforge focus on fine System-on-Chip Cores in
%%                  Verilog HDL Code which are easy understandable and
%%                  adjustable. For further information see
%%                          www.chipforge.org
%%                  there are projects from small cores up to PCBs, too.
%%
%%  File:           StdCellLib/Documents/section-AOOAOOAI_complex.tex
%%
%%  Purpose:        Section Level File for Standard Cell Library Documentation
%%
%%  ************    LaTeX with circdia.sty package      ***************
%%
%%  ///////////////////////////////////////////////////////////////////
%%
%%  Copyright (c) 2018 - 2022 by
%%                  chipforge <stdcelllib@nospam.chipforge.org>
%%  All rights reserved.
%%
%%      This Standard Cell Library is licensed under the Libre Silicon
%%      public license; you can redistribute it and/or modify it under
%%      the terms of the Libre Silicon public license as published by
%%      the Libre Silicon alliance, either version 1 of the License, or
%%      (at your option) any later version.
%%
%%      This design is distributed in the hope that it will be useful,
%%      but WITHOUT ANY WARRANTY; without even the implied warranty of
%%      MERCHANTABILITY or FITNESS FOR A PARTICULAR PURPOSE.
%%      See the Libre Silicon Public License for more details.
%%
%%  ///////////////////////////////////////////////////////////////////
\section{AND-OR-OR-AND-OR-OR-AND(-Invert) Complex Gates}

%%  ************    LibreSilicon's StdCellLibrary   *******************
%%
%%  Organisation:   Chipforge
%%                  Germany / European Union
%%
%%  Profile:        Chipforge focus on fine System-on-Chip Cores in
%%                  Verilog HDL Code which are easy understandable and
%%                  adjustable. For further information see
%%                          www.chipforge.org
%%                  there are projects from small cores up to PCBs, too.
%%
%%  File:           StdCellLib/Documents/Datasheets/Circuitry/AOOAOOAI21212.tex
%%
%%  Purpose:        Circuit File for AOOAOOAI21212
%%
%%  ************    LaTeX with circdia.sty package      ***************
%%
%%  ///////////////////////////////////////////////////////////////////
%%
%%  Copyright (c) 2018 - 2022 by
%%                  chipforge <stdcelllib@nospam.chipforge.org>
%%  All rights reserved.
%%
%%      This Standard Cell Library is licensed under the Libre Silicon
%%      public license; you can redistribute it and/or modify it under
%%      the terms of the Libre Silicon public license as published by
%%      the Libre Silicon alliance, either version 1 of the License, or
%%      (at your option) any later version.
%%
%%      This design is distributed in the hope that it will be useful,
%%      but WITHOUT ANY WARRANTY; without even the implied warranty of
%%      MERCHANTABILITY or FITNESS FOR A PARTICULAR PURPOSE.
%%      See the Libre Silicon Public License for more details.
%%
%%  ///////////////////////////////////////////////////////////////////
\begin{circuitdiagram}[draft]{39}{20}

    \usgate
    % ----  1st column  ----
    \pin{1}{1}{L}{A}
    \pin{1}{5}{L}{A1}
    \gate[\inputs{2}]{and}{5}{3}{R}{}{}

    % ----  2nd column  ----
    \pin{8}{7}{L}{B}
    \gate[\inputs{2}]{or}{12}{5}{R}{}{}

    \pin{8}{9}{L}{C}
    \pin{8}{13}{L}{C1}
    \gate[\inputs{2}]{or}{12}{11}{R}{}{}

    % ----  3rd column  ----
    \wire{16}{5}{16}{7}
    \gate[\inputs{2}]{and}{19}{9}{R}{}{}

    % ----  4th column  ----
    \pin{22}{13}{L}{D}
    \gate[\inputs{2}]{or}{26}{11}{R}{}{}

    \pin{22}{15}{L}{E}
    \pin{22}{19}{L}{E1}
    \gate[\inputs{2}]{or}{26}{17}{R}{}{}

    % ----  5th column  ----
    \wire{30}{11}{30}{13}
    \gate[\inputs{2}]{nand}{33}{15}{R}{}{}
    % ----  result ----
    \pin{38}{15}{R}{Y}

\end{circuitdiagram}
 %%  ************    LibreSilicon's StdCellLibrary   *******************
%%
%%  Organisation:   Chipforge
%%                  Germany / European Union
%%
%%  Profile:        Chipforge focus on fine System-on-Chip Cores in
%%                  Verilog HDL Code which are easy understandable and
%%                  adjustable. For further information see
%%                          www.chipforge.org
%%                  there are projects from small cores up to PCBs, too.
%%
%%  File:           StdCellLib/Documents/Datasheets/Circuitry/AOOAOOA21212.tex
%%
%%  Purpose:        Circuit File for AOOAOOA21212
%%
%%  ************    LaTeX with circdia.sty package      ***************
%%
%%  ///////////////////////////////////////////////////////////////////
%%
%%  Copyright (c) 2018 - 2022 by
%%                  chipforge <stdcelllib@nospam.chipforge.org>
%%  All rights reserved.
%%
%%      This Standard Cell Library is licensed under the Libre Silicon
%%      public license; you can redistribute it and/or modify it under
%%      the terms of the Libre Silicon public license as published by
%%      the Libre Silicon alliance, either version 1 of the License, or
%%      (at your option) any later version.
%%
%%      This design is distributed in the hope that it will be useful,
%%      but WITHOUT ANY WARRANTY; without even the implied warranty of
%%      MERCHANTABILITY or FITNESS FOR A PARTICULAR PURPOSE.
%%      See the Libre Silicon Public License for more details.
%%
%%  ///////////////////////////////////////////////////////////////////
\begin{circuitdiagram}[draft]{45}{20}

    \usgate
    % ----  1st column  ----
    \pin{1}{1}{L}{A}
    \pin{1}{5}{L}{A1}
    \gate[\inputs{2}]{and}{5}{3}{R}{}{}

    % ----  2nd column  ----
    \pin{8}{7}{L}{B}
    \gate[\inputs{2}]{or}{12}{5}{R}{}{}

    \pin{8}{9}{L}{C}
    \pin{8}{13}{L}{C1}
    \gate[\inputs{2}]{or}{12}{11}{R}{}{}

    % ----  3rd column  ----
    \wire{16}{5}{16}{7}
    \gate[\inputs{2}]{and}{19}{9}{R}{}{}

    % ----  4th column  ----
    \pin{22}{13}{L}{D}
    \gate[\inputs{2}]{or}{26}{11}{R}{}{}

    \pin{22}{15}{L}{E}
    \pin{22}{19}{L}{E1}
    \gate[\inputs{2}]{or}{26}{17}{R}{}{}

    % ----  5th column  ----
    \wire{30}{11}{30}{13}
    \gate[\inputs{2}]{nand}{33}{15}{R}{}{}

    % ----  last column ----
    \gate{not}{40}{15}{R}{}{}

    % ----  result ----
    \pin{44}{15}{R}{Z}

\end{circuitdiagram}


%%  ************    LibreSilicon's StdCellLibrary   *******************
%%
%%  Organisation:   Chipforge
%%                  Germany / European Union
%%
%%  Profile:        Chipforge focus on fine System-on-Chip Cores in
%%                  Verilog HDL Code which are easy understandable and
%%                  adjustable. For further information see
%%                          www.chipforge.org
%%                  there are projects from small cores up to PCBs, too.
%%
%%  File:           StdCellLib/Documents/Book/section-OAAOAAOI_complex.tex
%%
%%  Purpose:        Section Level File for Standard Cell Library Documentation
%%
%%  ************    LaTeX with circdia.sty package      ***************
%%
%%  ///////////////////////////////////////////////////////////////////
%%
%%  Copyright (c) 2018 - 2022 by
%%                  chipforge <stdcelllib@nospam.chipforge.org>
%%  All rights reserved.
%%
%%      This Standard Cell Library is licensed under the Libre Silicon
%%      public license; you can redistribute it and/or modify it under
%%      the terms of the Libre Silicon public license as published by
%%      the Libre Silicon alliance, either version 1 of the License, or
%%      (at your option) any later version.
%%
%%      This design is distributed in the hope that it will be useful,
%%      but WITHOUT ANY WARRANTY; without even the implied warranty of
%%      MERCHANTABILITY or FITNESS FOR A PARTICULAR PURPOSE.
%%      See the Libre Silicon Public License for more details.
%%
%%  ///////////////////////////////////////////////////////////////////
\section{OR-AND-AND-OR-AND-AND-OR(-Invert) Complex Gates}


%%  ************    LibreSilicon's StdCellLibrary   *******************
%%
%%  Organisation:   Chipforge
%%                  Germany / European Union
%%
%%  Profile:        Chipforge focus on fine System-on-Chip Cores in
%%                  Verilog HDL Code which are easy understandable and
%%                  adjustable. For further information see
%%                          www.chipforge.org
%%                  there are projects from small cores up to PCBs, too.
%%
%%  File:           StdCellLib/Documents/section-AOAAOOAI_complex.tex
%%
%%  Purpose:        Section Level File for Standard Cell Library Documentation
%%
%%  ************    LaTeX with circdia.sty package      ***************
%%
%%  ///////////////////////////////////////////////////////////////////
%%
%%  Copyright (c) 2018 - 2022 by
%%                  chipforge <stdcelllib@nospam.chipforge.org>
%%  All rights reserved.
%%
%%      This Standard Cell Library is licensed under the Libre Silicon
%%      public license; you can redistribute it and/or modify it under
%%      the terms of the Libre Silicon public license as published by
%%      the Libre Silicon alliance, either version 1 of the License, or
%%      (at your option) any later version.
%%
%%      This design is distributed in the hope that it will be useful,
%%      but WITHOUT ANY WARRANTY; without even the implied warranty of
%%      MERCHANTABILITY or FITNESS FOR A PARTICULAR PURPOSE.
%%      See the Libre Silicon Public License for more details.
%%
%%  ///////////////////////////////////////////////////////////////////
\section{AND-OR-AND-AND-OR-OR-AND(-Invert) Complex Gates}

%%  ************    LibreSilicon's StdCellLibrary   *******************
%%
%%  Organisation:   Chipforge
%%                  Germany / European Union
%%
%%  Profile:        Chipforge focus on fine System-on-Chip Cores in
%%                  Verilog HDL Code which are easy understandable and
%%                  adjustable. For further information see
%%                          www.chipforge.org
%%                  there are projects from small cores up to PCBs, too.
%%
%%  File:           StdCellLib/Documents/Datasheets/Circuitry/AOAAOOAI21121.tex
%%
%%  Purpose:        Circuit File for AOAAOOAI21121
%%
%%  ************    LaTeX with circdia.sty package      ***************
%%
%%  ///////////////////////////////////////////////////////////////////
%%
%%  Copyright (c) 2018 - 2022 by
%%                  chipforge <stdcelllib@nospam.chipforge.org>
%%  All rights reserved.
%%
%%      This Standard Cell Library is licensed under the Libre Silicon
%%      public license; you can redistribute it and/or modify it under
%%      the terms of the Libre Silicon public license as published by
%%      the Libre Silicon alliance, either version 1 of the License, or
%%      (at your option) any later version.
%%
%%      This design is distributed in the hope that it will be useful,
%%      but WITHOUT ANY WARRANTY; without even the implied warranty of
%%      MERCHANTABILITY or FITNESS FOR A PARTICULAR PURPOSE.
%%      See the Libre Silicon Public License for more details.
%%
%%  ///////////////////////////////////////////////////////////////////
\begin{circuitdiagram}[draft]{39}{22}

    \usgate
    % ----  1st column  ----
    \pin{1}{1}{L}{A}
    \pin{1}{5}{L}{A1}
    \gate[\inputs{2}]{and}{5}{3}{R}{}{}

    % ----  2nd column  ----
    \pin{8}{7}{L}{B}
    \gate[\inputs{2}]{or}{12}{5}{R}{}{}

    % ----  3rd column  ----
    \pin{15}{9}{L}{C}
    \gate[\inputs{2}]{and}{19}{7}{R}{}{}

    \pin{15}{15}{L}{D1}
    \pin{15}{11}{L}{D}
    \gate[\inputs{2}]{and}{19}{13}{R}{}{}

    % ----  4th column  ----
    \wire{23}{7}{23}{9}
    \gate[\inputs{2}]{or}{26}{11}{R}{}{}
    
    \pin{22}{17}{L}{E}
    \pin{22}{21}{L}{E1}
    \gate[\inputs{2}]{or}{26}{19}{R}{}{}

    % ----  5th column  ----
    \wire{30}{11}{30}{15}
    \gate[\inputs{2}]{nand}{33}{17}{R}{}{}


    % ----  result ----
    \pin{38}{17}{R}{Y}

\end{circuitdiagram}
 %%  ************    LibreSilicon's StdCellLibrary   *******************
%%
%%  Organisation:   Chipforge
%%                  Germany / European Union
%%
%%  Profile:        Chipforge focus on fine System-on-Chip Cores in
%%                  Verilog HDL Code which are easy understandable and
%%                  adjustable. For further information see
%%                          www.chipforge.org
%%                  there are projects from small cores up to PCBs, too.
%%
%%  File:           StdCellLib/Documents/Datasheets/Circuitry/AOAAOOA21121.tex
%%
%%  Purpose:        Circuit File for AOAAOOA21121
%%
%%  ************    LaTeX with circdia.sty package      ***************
%%
%%  ///////////////////////////////////////////////////////////////////
%%
%%  Copyright (c) 2018 - 2022 by
%%                  chipforge <stdcelllib@nospam.chipforge.org>
%%  All rights reserved.
%%
%%      This Standard Cell Library is licensed under the Libre Silicon
%%      public license; you can redistribute it and/or modify it under
%%      the terms of the Libre Silicon public license as published by
%%      the Libre Silicon alliance, either version 1 of the License, or
%%      (at your option) any later version.
%%
%%      This design is distributed in the hope that it will be useful,
%%      but WITHOUT ANY WARRANTY; without even the implied warranty of
%%      MERCHANTABILITY or FITNESS FOR A PARTICULAR PURPOSE.
%%      See the Libre Silicon Public License for more details.
%%
%%  ///////////////////////////////////////////////////////////////////
\begin{circuitdiagram}[draft]{45}{22}

    \usgate
    % ----  1st column  ----
    \pin{1}{1}{L}{A}
    \pin{1}{5}{L}{A1}
    \gate[\inputs{2}]{and}{5}{3}{R}{}{}

    % ----  2nd column  ----
    \pin{8}{7}{L}{B}
    \gate[\inputs{2}]{or}{12}{5}{R}{}{}

    % ----  3rd column  ----
    \pin{15}{9}{L}{C}
    \gate[\inputs{2}]{and}{19}{7}{R}{}{}

    \pin{15}{15}{L}{D1}
    \pin{15}{11}{L}{D}
    \gate[\inputs{2}]{and}{19}{13}{R}{}{}

    % ----  4th column  ----
    \wire{23}{7}{23}{9}
    \gate[\inputs{2}]{or}{26}{11}{R}{}{}
    
    \pin{22}{17}{L}{E}
    \pin{22}{21}{L}{E1}
    \gate[\inputs{2}]{or}{26}{19}{R}{}{}

    % ----  5th column  ----
    \wire{30}{11}{30}{15}
    \gate[\inputs{2}]{nand}{33}{17}{R}{}{}

    % ----  last column ----
    \gate{not}{40}{17}{R}{}{}

    % ----  result ----
    \pin{44}{17}{R}{Z}

\end{circuitdiagram}

%%  ************    LibreSilicon's StdCellLibrary   *******************
%%
%%  Organisation:   Chipforge
%%                  Germany / European Union
%%
%%  Profile:        Chipforge focus on fine System-on-Chip Cores in
%%                  Verilog HDL Code which are easy understandable and
%%                  adjustable. For further information see
%%                          www.chipforge.org
%%                  there are projects from small cores up to PCBs, too.
%%
%%  File:           StdCellLib/Documents/Datasheets/Circuitry/AOAAOOAI21123.tex
%%
%%  Purpose:        Circuit File for AOAAOOAI21123
%%
%%  ************    LaTeX with circdia.sty package      ***************
%%
%%  ///////////////////////////////////////////////////////////////////
%%
%%  Copyright (c) 2018 - 2022 by
%%                  chipforge <stdcelllib@nospam.chipforge.org>
%%  All rights reserved.
%%
%%      This Standard Cell Library is licensed under the Libre Silicon
%%      public license; you can redistribute it and/or modify it under
%%      the terms of the Libre Silicon public license as published by
%%      the Libre Silicon alliance, either version 1 of the License, or
%%      (at your option) any later version.
%%
%%      This design is distributed in the hope that it will be useful,
%%      but WITHOUT ANY WARRANTY; without even the implied warranty of
%%      MERCHANTABILITY or FITNESS FOR A PARTICULAR PURPOSE.
%%      See the Libre Silicon Public License for more details.
%%
%%  ///////////////////////////////////////////////////////////////////
\begin{circuitdiagram}[draft]{39}{22}

    \usgate
    % ----  1st column  ----
    \pin{1}{1}{L}{A}
    \pin{1}{5}{L}{A1}
    \gate[\inputs{2}]{and}{5}{3}{R}{}{}

    % ----  2nd column  ----
    \pin{8}{7}{L}{B}
    \gate[\inputs{2}]{or}{12}{5}{R}{}{}

    % ----  3rd column  ----
    \pin{15}{9}{L}{C}
    \gate[\inputs{2}]{and}{19}{7}{R}{}{}

    \pin{15}{15}{L}{D1}
    \pin{15}{11}{L}{D}
    \gate[\inputs{2}]{and}{19}{13}{R}{}{}

    % ----  4th column  ----
    \wire{23}{7}{23}{9}
    \gate[\inputs{2}]{or}{26}{11}{R}{}{}
    
    \pin{22}{17}{L}{E}
    \pin{22}{19}{L}{E1}
    \pin{22}{21}{L}{E2}
    \gate[\inputs{3}]{or}{26}{19}{R}{}{}

    % ----  5th column  ----
    \wire{30}{11}{30}{15}
    \gate[\inputs{2}]{nand}{33}{17}{R}{}{}


    % ----  result ----
    \pin{38}{17}{R}{Y}

\end{circuitdiagram}
 %%  ************    LibreSilicon's StdCellLibrary   *******************
%%
%%  Organisation:   Chipforge
%%                  Germany / European Union
%%
%%  Profile:        Chipforge focus on fine System-on-Chip Cores in
%%                  Verilog HDL Code which are easy understandable and
%%                  adjustable. For further information see
%%                          www.chipforge.org
%%                  there are projects from small cores up to PCBs, too.
%%
%%  File:           StdCellLib/Documents/Datasheets/Circuitry/AOAAOOA21123.tex
%%
%%  Purpose:        Circuit File for AOAAOOA21123
%%
%%  ************    LaTeX with circdia.sty package      ***************
%%
%%  ///////////////////////////////////////////////////////////////////
%%
%%  Copyright (c) 2018 - 2022 by
%%                  chipforge <stdcelllib@nospam.chipforge.org>
%%  All rights reserved.
%%
%%      This Standard Cell Library is licensed under the Libre Silicon
%%      public license; you can redistribute it and/or modify it under
%%      the terms of the Libre Silicon public license as published by
%%      the Libre Silicon alliance, either version 1 of the License, or
%%      (at your option) any later version.
%%
%%      This design is distributed in the hope that it will be useful,
%%      but WITHOUT ANY WARRANTY; without even the implied warranty of
%%      MERCHANTABILITY or FITNESS FOR A PARTICULAR PURPOSE.
%%      See the Libre Silicon Public License for more details.
%%
%%  ///////////////////////////////////////////////////////////////////
\begin{circuitdiagram}[draft]{45}{22}

    \usgate
    % ----  1st column  ----
    \pin{1}{1}{L}{A}
    \pin{1}{5}{L}{A1}
    \gate[\inputs{2}]{and}{5}{3}{R}{}{}

    % ----  2nd column  ----
    \pin{8}{7}{L}{B}
    \gate[\inputs{2}]{or}{12}{5}{R}{}{}

    % ----  3rd column  ----
    \pin{15}{9}{L}{C}
    \gate[\inputs{2}]{and}{19}{7}{R}{}{}

    \pin{15}{15}{L}{D1}
    \pin{15}{11}{L}{D}
    \gate[\inputs{2}]{and}{19}{13}{R}{}{}

    % ----  4th column  ----
    \wire{23}{7}{23}{9}
    \gate[\inputs{2}]{or}{26}{11}{R}{}{}
    
    \pin{22}{17}{L}{E}
    \pin{22}{19}{L}{E1}
    \pin{22}{21}{L}{E2}
    \gate[\inputs{3}]{or}{26}{19}{R}{}{}

    % ----  5th column  ----
    \wire{30}{11}{30}{15}
    \gate[\inputs{2}]{nand}{33}{17}{R}{}{}

    % ----  last column ----
    \gate{not}{40}{17}{R}{}{}

    % ----  result ----
    \pin{44}{17}{R}{Z}

\end{circuitdiagram}

%%  ************    LibreSilicon's StdCellLibrary   *******************
%%
%%  Organisation:   Chipforge
%%                  Germany / European Union
%%
%%  Profile:        Chipforge focus on fine System-on-Chip Cores in
%%                  Verilog HDL Code which are easy understandable and
%%                  adjustable. For further information see
%%                          www.chipforge.org
%%                  there are projects from small cores up to PCBs, too.
%%
%%  File:           StdCellLib/Documents/Datasheets/Circuitry/AOAAOOAI21132.tex
%%
%%  Purpose:        Circuit File for AOAAOOAI21132
%%
%%  ************    LaTeX with circdia.sty package      ***************
%%
%%  ///////////////////////////////////////////////////////////////////
%%
%%  Copyright (c) 2018 - 2022 by
%%                  chipforge <stdcelllib@nospam.chipforge.org>
%%  All rights reserved.
%%
%%      This Standard Cell Library is licensed under the Libre Silicon
%%      public license; you can redistribute it and/or modify it under
%%      the terms of the Libre Silicon public license as published by
%%      the Libre Silicon alliance, either version 1 of the License, or
%%      (at your option) any later version.
%%
%%      This design is distributed in the hope that it will be useful,
%%      but WITHOUT ANY WARRANTY; without even the implied warranty of
%%      MERCHANTABILITY or FITNESS FOR A PARTICULAR PURPOSE.
%%      See the Libre Silicon Public License for more details.
%%
%%  ///////////////////////////////////////////////////////////////////
\begin{circuitdiagram}[draft]{39}{22}

    \usgate
    % ----  1st column  ----
    \pin{1}{1}{L}{A}
    \pin{1}{5}{L}{A1}
    \gate[\inputs{2}]{and}{5}{3}{R}{}{}

    % ----  2nd column  ----
    \pin{8}{7}{L}{B}
    \gate[\inputs{2}]{or}{12}{5}{R}{}{}

    % ----  3rd column  ----
    \pin{15}{9}{L}{C}
    \gate[\inputs{2}]{and}{19}{7}{R}{}{}

    \pin{15}{11}{L}{D}
    \pin{15}{13}{L}{D1}
    \pin{15}{15}{L}{D2}
    \gate[\inputs{3}]{and}{19}{13}{R}{}{}

    % ----  4th column  ----
    \wire{23}{7}{23}{9}
    \gate[\inputs{2}]{or}{26}{11}{R}{}{}
    
    \pin{22}{17}{L}{E}
    \pin{22}{21}{L}{E1}
    \gate[\inputs{2}]{or}{26}{19}{R}{}{}

    % ----  5th column  ----
    \wire{30}{11}{30}{15}
    \gate[\inputs{2}]{nand}{33}{17}{R}{}{}


    % ----  result ----
    \pin{38}{17}{R}{Y}

\end{circuitdiagram}
 %%  ************    LibreSilicon's StdCellLibrary   *******************
%%
%%  Organisation:   Chipforge
%%                  Germany / European Union
%%
%%  Profile:        Chipforge focus on fine System-on-Chip Cores in
%%                  Verilog HDL Code which are easy understandable and
%%                  adjustable. For further information see
%%                          www.chipforge.org
%%                  there are projects from small cores up to PCBs, too.
%%
%%  File:           StdCellLib/Documents/Datasheets/Circuitry/AOAAOOA21132.tex
%%
%%  Purpose:        Circuit File for AOAAOOA21132
%%
%%  ************    LaTeX with circdia.sty package      ***************
%%
%%  ///////////////////////////////////////////////////////////////////
%%
%%  Copyright (c) 2018 - 2022 by
%%                  chipforge <stdcelllib@nospam.chipforge.org>
%%  All rights reserved.
%%
%%      This Standard Cell Library is licensed under the Libre Silicon
%%      public license; you can redistribute it and/or modify it under
%%      the terms of the Libre Silicon public license as published by
%%      the Libre Silicon alliance, either version 1 of the License, or
%%      (at your option) any later version.
%%
%%      This design is distributed in the hope that it will be useful,
%%      but WITHOUT ANY WARRANTY; without even the implied warranty of
%%      MERCHANTABILITY or FITNESS FOR A PARTICULAR PURPOSE.
%%      See the Libre Silicon Public License for more details.
%%
%%  ///////////////////////////////////////////////////////////////////
\begin{circuitdiagram}[draft]{45}{22}

    \usgate
    % ----  1st column  ----
    \pin{1}{1}{L}{A}
    \pin{1}{5}{L}{A1}
    \gate[\inputs{2}]{and}{5}{3}{R}{}{}

    % ----  2nd column  ----
    \pin{8}{7}{L}{B}
    \gate[\inputs{2}]{or}{12}{5}{R}{}{}

    % ----  3rd column  ----
    \pin{15}{9}{L}{C}
    \gate[\inputs{2}]{and}{19}{7}{R}{}{}

    \pin{15}{11}{L}{D}
    \pin{15}{13}{L}{D1}
    \pin{15}{15}{L}{D2}
    \gate[\inputs{3}]{and}{19}{13}{R}{}{}

    % ----  4th column  ----
    \wire{23}{7}{23}{9}
    \gate[\inputs{2}]{or}{26}{11}{R}{}{}
    
    \pin{22}{17}{L}{E}
    \pin{22}{21}{L}{E1}
    \gate[\inputs{2}]{or}{26}{19}{R}{}{}

    % ----  5th column  ----
    \wire{30}{11}{30}{15}
    \gate[\inputs{2}]{nand}{33}{17}{R}{}{}

    % ----  last column ----
    \gate{not}{40}{17}{R}{}{}

    % ----  result ----
    \pin{44}{17}{R}{Z}

\end{circuitdiagram}


%%  ************    LibreSilicon's StdCellLibrary   *******************
%%
%%  Organisation:   Chipforge
%%                  Germany / European Union
%%
%%  Profile:        Chipforge focus on fine System-on-Chip Cores in
%%                  Verilog HDL Code which are easy understandable and
%%                  adjustable. For further information see
%%                          www.chipforge.org
%%                  there are projects from small cores up to PCBs, too.
%%
%%  File:           StdCellLib/Documents/Datasheets/Circuitry/AOAAOOAI211212.tex
%%
%%  Purpose:        Circuit File for AOAAOOAI211212
%%
%%  ************    LaTeX with circdia.sty package      ***************
%%
%%  ///////////////////////////////////////////////////////////////////
%%
%%  Copyright (c) 2018 - 2022 by
%%                  chipforge <stdcelllib@nospam.chipforge.org>
%%  All rights reserved.
%%
%%      This Standard Cell Library is licensed under the Libre Silicon
%%      public license; you can redistribute it and/or modify it under
%%      the terms of the Libre Silicon public license as published by
%%      the Libre Silicon alliance, either version 1 of the License, or
%%      (at your option) any later version.
%%
%%      This design is distributed in the hope that it will be useful,
%%      but WITHOUT ANY WARRANTY; without even the implied warranty of
%%      MERCHANTABILITY or FITNESS FOR A PARTICULAR PURPOSE.
%%      See the Libre Silicon Public License for more details.
%%
%%  ///////////////////////////////////////////////////////////////////
\begin{circuitdiagram}[draft]{39}{24}

    \usgate
    % ----  1st column  ----
    \pin{1}{1}{L}{A}
    \pin{1}{5}{L}{A1}
    \gate[\inputs{2}]{and}{5}{3}{R}{}{}

    % ----  2nd column  ----
    \pin{8}{7}{L}{B}
    \gate[\inputs{2}]{or}{12}{5}{R}{}{}

    % ----  3rd column  ----
    \pin{15}{9}{L}{C}
    \gate[\inputs{2}]{and}{19}{7}{R}{}{}

    \pin{15}{15}{L}{D1}
    \pin{15}{11}{L}{D}
    \gate[\inputs{2}]{and}{19}{13}{R}{}{}

    % ----  4th column  ----
    \wire{23}{7}{23}{11}
    \pin{22}{17}{L}{E}
    \wire{23}{15}{23}{17}
    \gate[\inputs{3}]{or}{26}{13}{R}{}{}

    \pin{22}{19}{L}{F}
    \pin{22}{23}{L}{F1}
    \gate[\inputs{2}]{or}{26}{21}{R}{}{}

    % ----  5th column  ----
    \wire{30}{13}{30}{17}
    \gate[\inputs{2}]{nand}{33}{19}{R}{}{}


    % ----  result ----
    \pin{38}{19}{R}{Y}

\end{circuitdiagram}
 %%  ************    LibreSilicon's StdCellLibrary   *******************
%%
%%  Organisation:   Chipforge
%%                  Germany / European Union
%%
%%  Profile:        Chipforge focus on fine System-on-Chip Cores in
%%                  Verilog HDL Code which are easy understandable and
%%                  adjustable. For further information see
%%                          www.chipforge.org
%%                  there are projects from small cores up to PCBs, too.
%%
%%  File:           StdCellLib/Documents/Datasheets/Circuitry/AOAAOOA211212.tex
%%
%%  Purpose:        Circuit File for AOAAOOA211212
%%
%%  ************    LaTeX with circdia.sty package      ***************
%%
%%  ///////////////////////////////////////////////////////////////////
%%
%%  Copyright (c) 2018 - 2022 by
%%                  chipforge <stdcelllib@nospam.chipforge.org>
%%  All rights reserved.
%%
%%      This Standard Cell Library is licensed under the Libre Silicon
%%      public license; you can redistribute it and/or modify it under
%%      the terms of the Libre Silicon public license as published by
%%      the Libre Silicon alliance, either version 1 of the License, or
%%      (at your option) any later version.
%%
%%      This design is distributed in the hope that it will be useful,
%%      but WITHOUT ANY WARRANTY; without even the implied warranty of
%%      MERCHANTABILITY or FITNESS FOR A PARTICULAR PURPOSE.
%%      See the Libre Silicon Public License for more details.
%%
%%  ///////////////////////////////////////////////////////////////////
\begin{circuitdiagram}[draft]{45}{24}

    \usgate
    % ----  1st column  ----
    \pin{1}{1}{L}{A}
    \pin{1}{5}{L}{A1}
    \gate[\inputs{2}]{and}{5}{3}{R}{}{}

    % ----  2nd column  ----
    \pin{8}{7}{L}{B}
    \gate[\inputs{2}]{or}{12}{5}{R}{}{}

    % ----  3rd column  ----
    \pin{15}{9}{L}{C}
    \gate[\inputs{2}]{and}{19}{7}{R}{}{}

    \pin{15}{15}{L}{D1}
    \pin{15}{11}{L}{D}
    \gate[\inputs{2}]{and}{19}{13}{R}{}{}

    % ----  4th column  ----
    \wire{23}{7}{23}{11}
    \pin{22}{17}{L}{E}
    \wire{23}{15}{23}{17}
    \gate[\inputs{3}]{or}{26}{13}{R}{}{}

    \pin{22}{19}{L}{F}
    \pin{22}{23}{L}{F1}
    \gate[\inputs{2}]{or}{26}{21}{R}{}{}

    % ----  5th column  ----
    \wire{30}{13}{30}{17}
    \gate[\inputs{2}]{nand}{33}{19}{R}{}{}

    % ----  last column ----
    \gate{not}{40}{19}{R}{}{}

    % ----  result ----
    \pin{44}{19}{R}{Z}

\end{circuitdiagram}


%%  ************    LibreSilicon's StdCellLibrary   *******************
%%
%%  Organisation:   Chipforge
%%                  Germany / European Union
%%
%%  Profile:        Chipforge focus on fine System-on-Chip Cores in
%%                  Verilog HDL Code which are easy understandable and
%%                  adjustable. For further information see
%%                          www.chipforge.org
%%                  there are projects from small cores up to PCBs, too.
%%
%%  File:           StdCellLib/Documents/section-OAOOAAOI_complex.tex
%%
%%  Purpose:        Section Level File for Standard Cell Library Documentation
%%
%%  ************    LaTeX with circdia.sty package      ***************
%%
%%  ///////////////////////////////////////////////////////////////////
%%
%%  Copyright (c) 2018 - 2022 by
%%                  chipforge <stdcelllib@nospam.chipforge.org>
%%  All rights reserved.
%%
%%      This Standard Cell Library is licensed under the Libre Silicon
%%      public license; you can redistribute it and/or modify it under
%%      the terms of the Libre Silicon public license as published by
%%      the Libre Silicon alliance, either version 1 of the License, or
%%      (at your option) any later version.
%%
%%      This design is distributed in the hope that it will be useful,
%%      but WITHOUT ANY WARRANTY; without even the implied warranty of
%%      MERCHANTABILITY or FITNESS FOR A PARTICULAR PURPOSE.
%%      See the Libre Silicon Public License for more details.
%%
%%  ///////////////////////////////////////////////////////////////////
\section{OR-AND-OR-OR-AND-AND-OR(-Invert) Complex Gates}


%%  ************    LibreSilicon's StdCellLibrary   *******************
%%
%%  Organisation:   Chipforge
%%                  Germany / European Union
%%
%%  Profile:        Chipforge focus on fine System-on-Chip Cores in
%%                  Verilog HDL Code which are easy understandable and
%%                  adjustable. For further information see
%%                          www.chipforge.org
%%                  there are projects from small cores up to PCBs, too.
%%
%%  File:           StdCellLib/Documents/Book/section-AAOAAOOAI_complex.tex
%%
%%  Purpose:        Section Level File for Standard Cell Library Documentation
%%
%%  ************    LaTeX with circdia.sty package      ***************
%%
%%  ///////////////////////////////////////////////////////////////////
%%
%%  Copyright (c) 2018 - 2022 by
%%                  chipforge <stdcelllib@nospam.chipforge.org>
%%  All rights reserved.
%%
%%      This Standard Cell Library is licensed under the Libre Silicon
%%      public license; you can redistribute it and/or modify it under
%%      the terms of the Libre Silicon public license as published by
%%      the Libre Silicon alliance, either version 1 of the License, or
%%      (at your option) any later version.
%%
%%      This design is distributed in the hope that it will be useful,
%%      but WITHOUT ANY WARRANTY; without even the implied warranty of
%%      MERCHANTABILITY or FITNESS FOR A PARTICULAR PURPOSE.
%%      See the Libre Silicon Public License for more details.
%%
%%  ///////////////////////////////////////////////////////////////////
\section{AND-AND-OR-AND-AND-OR-OR-AND(-Invert) Complex Gates}


%%  ************    LibreSilicon's StdCellLibrary   *******************
%%
%%  Organisation:   Chipforge
%%                  Germany / European Union
%%
%%  Profile:        Chipforge focus on fine System-on-Chip Cores in
%%                  Verilog HDL Code which are easy understandable and
%%                  adjustable. For further information see
%%                          www.chipforge.org
%%                  there are projects from small cores up to PCBs, too.
%%
%%  File:           StdCellLib/Documents/Book/section-OOAOOAAOI_complex.tex
%%
%%  Purpose:        Section Level File for Standard Cell Library Documentation
%%
%%  ************    LaTeX with circdia.sty package      ***************
%%
%%  ///////////////////////////////////////////////////////////////////
%%
%%  Copyright (c) 2018 - 2022 by
%%                  chipforge <stdcelllib@nospam.chipforge.org>
%%  All rights reserved.
%%
%%      This Standard Cell Library is licensed under the Libre Silicon
%%      public license; you can redistribute it and/or modify it under
%%      the terms of the Libre Silicon public license as published by
%%      the Libre Silicon alliance, either version 1 of the License, or
%%      (at your option) any later version.
%%
%%      This design is distributed in the hope that it will be useful,
%%      but WITHOUT ANY WARRANTY; without even the implied warranty of
%%      MERCHANTABILITY or FITNESS FOR A PARTICULAR PURPOSE.
%%      See the Libre Silicon Public License for more details.
%%
%%  ///////////////////////////////////////////////////////////////////
\section{OR-OR-AND-OR-OR-AND-AND-OR(-Invert) Complex Gates}


%%  ************    LibreSilicon's StdCellLibrary   *******************
%%
%%  Organisation:   Chipforge
%%                  Germany / European Union
%%
%%  Profile:        Chipforge focus on fine System-on-Chip Cores in
%%                  Verilog HDL Code which are easy understandable and
%%                  adjustable. For further information see
%%                          www.chipforge.org
%%                  there are projects from small cores up to PCBs, too.
%%
%%  File:           StdCellLib/Documents/LaTeX/section-AAOAOOAI_complex.tex
%%
%%  Purpose:        Section Level File for Standard Cell Library Documentation
%%
%%  ************    LaTeX with circdia.sty package      ***************
%%
%%  ///////////////////////////////////////////////////////////////////
%%
%%  Copyright (c) 2018 - 2021 by
%%                  chipforge <stdcelllib@nospam.chipforge.org>
%%  All rights reserved.
%%
%%      This Standard Cell Library is licensed under the Libre Silicon
%%      public license; you can redistribute it and/or modify it under
%%      the terms of the Libre Silicon public license as published by
%%      the Libre Silicon alliance, either version 1 of the License, or
%%      (at your option) any later version.
%%
%%      This design is distributed in the hope that it will be useful,
%%      but WITHOUT ANY WARRANTY; without even the implied warranty of
%%      MERCHANTABILITY or FITNESS FOR A PARTICULAR PURPOSE.
%%      See the Libre Silicon Public License for more details.
%%
\section{AND-AND-OR-AND-OR-OR-AND(-Invert) Complex Gates}


%%  ************    LibreSilicon's StdCellLibrary   *******************
%%
%%  Organisation:   Chipforge
%%                  Germany / European Union
%%
%%  Profile:        Chipforge focus on fine System-on-Chip Cores in
%%                  Verilog HDL Code which are easy understandable and
%%                  adjustable. For further information see
%%                          www.chipforge.org
%%                  there are projects from small cores up to PCBs, too.
%%
%%  File:           StdCellLib/Documents/Book/section-OOAOAAOI_complex.tex
%%
%%  Purpose:        Section Level File for Standard Cell Library Documentation
%%
%%  ************    LaTeX with circdia.sty package      ***************
%%
%%  ///////////////////////////////////////////////////////////////////
%%
%%  Copyright (c) 2018 - 2022 by
%%                  chipforge <stdcelllib@nospam.chipforge.org>
%%  All rights reserved.
%%
%%      This Standard Cell Library is licensed under the Libre Silicon
%%      public license; you can redistribute it and/or modify it under
%%      the terms of the Libre Silicon public license as published by
%%      the Libre Silicon alliance, either version 1 of the License, or
%%      (at your option) any later version.
%%
%%      This design is distributed in the hope that it will be useful,
%%      but WITHOUT ANY WARRANTY; without even the implied warranty of
%%      MERCHANTABILITY or FITNESS FOR A PARTICULAR PURPOSE.
%%      See the Libre Silicon Public License for more details.
%%
%%  ///////////////////////////////////////////////////////////////////
\section{OR-OR-AND-OR-AND-AND-OR(-Invert) Complex Gates}



%%  ------------    six phases      -----------------------------------

%%  ************    LibreSilicon's StdCellLibrary   *******************
%%
%%  Organisation:   Chipforge
%%                  Germany / European Union
%%
%%  Profile:        Chipforge focus on fine System-on-Chip Cores in
%%                  Verilog HDL Code which are easy understandable and
%%                  adjustable. For further information see
%%                          www.chipforge.org
%%                  there are projects from small cores up to PCBs, too.
%%
%%  File:           StdCellLib/Documents/Book/section-OAOAOAI_complex.tex
%%
%%  Purpose:        Section Level File for Standard Cell Library Documentation
%%
%%  ************    LaTeX with circdia.sty package      ***************
%%
%%  ///////////////////////////////////////////////////////////////////
%%
%%  Copyright (c) 2018 - 2022 by
%%                  chipforge <stdcelllib@nospam.chipforge.org>
%%  All rights reserved.
%%
%%      This Standard Cell Library is licensed under the Libre Silicon
%%      public license; you can redistribute it and/or modify it under
%%      the terms of the Libre Silicon public license as published by
%%      the Libre Silicon alliance, either version 1 of the License, or
%%      (at your option) any later version.
%%
%%      This design is distributed in the hope that it will be useful,
%%      but WITHOUT ANY WARRANTY; without even the implied warranty of
%%      MERCHANTABILITY or FITNESS FOR A PARTICULAR PURPOSE.
%%      See the Libre Silicon Public License for more details.
%%
%%  ///////////////////////////////////////////////////////////////////
\section{OR-AND-OR-AND-OR-AND(-Invert) Complex Gates}

%%  ************    LibreSilicon's StdCellLibrary   *******************
%%
%%  Organisation:   Chipforge
%%                  Germany / European Union
%%
%%  Profile:        Chipforge focus on fine System-on-Chip Cores in
%%                  Verilog HDL Code which are easy understandable and
%%                  adjustable. For further information see
%%                          www.chipforge.org
%%                  there are projects from small cores up to PCBs, too.
%%
%%  File:           StdCellLib/Documents/Datasheets/Circuitry/OAOAOAI211111.tex
%%
%%  Purpose:        Circuit File for OAOAOAI211111
%%
%%  ************    LaTeX with circdia.sty package      ***************
%%
%%  ///////////////////////////////////////////////////////////////////
%%
%%  Copyright (c) 2018 - 2022 by
%%                  chipforge <stdcelllib@nospam.chipforge.org>
%%  All rights reserved.
%%
%%      This Standard Cell Library is licensed under the Libre Silicon
%%      public license; you can redistribute it and/or modify it under
%%      the terms of the Libre Silicon public license as published by
%%      the Libre Silicon alliance, either version 1 of the License, or
%%      (at your option) any later version.
%%
%%      This design is distributed in the hope that it will be useful,
%%      but WITHOUT ANY WARRANTY; without even the implied warranty of
%%      MERCHANTABILITY or FITNESS FOR A PARTICULAR PURPOSE.
%%      See the Libre Silicon Public License for more details.
%%
%%  ///////////////////////////////////////////////////////////////////
\begin{circuitdiagram}[draft]{46}{16}

    \usgate
    % ----  1st column  ----
    \pin{1}{1}{L}{A}
    \pin{1}{5}{L}{A1}
    \gate[\inputs{2}]{or}{5}{3}{R}{}{}

    % ----  2nd column  ----
    \pin{8}{7}{L}{B}
    \gate[\inputs{2}]{and}{12}{5}{R}{}{}

    % ----  3rd column  ----
    \pin{15}{9}{L}{C}
    \gate[\inputs{2}]{or}{19}{7}{R}{}{}

    % ----  4th column  ----
    \pin{22}{11}{L}{D}
    \gate[\inputs{2}]{and}{26}{9}{R}{}{}

    % ----  5th column  ----
    \pin{29}{13}{L}{E}
    \gate[\inputs{2}]{or}{33}{11}{R}{}{}

    % ----  6th column  ----
    \pin{36}{15}{L}{F}
    \gate[\inputs{2}]{nand}{40}{13}{R}{}{}


    % ----  result ----
    \pin{45}{13}{R}{Y}

\end{circuitdiagram}
 %%  ************    LibreSilicon's StdCellLibrary   *******************
%%
%%  Organisation:   Chipforge
%%                  Germany / European Union
%%
%%  Profile:        Chipforge focus on fine System-on-Chip Cores in
%%                  Verilog HDL Code which are easy understandable and
%%                  adjustable. For further information see
%%                          www.chipforge.org
%%                  there are projects from small cores up to PCBs, too.
%%
%%  File:           StdCellLib/Documents/Datasheets/Circuitry/OAOAOA211111.tex
%%
%%  Purpose:        Circuit File for OAOAOA211111
%%
%%  ************    LaTeX with circdia.sty package      ***************
%%
%%  ///////////////////////////////////////////////////////////////////
%%
%%  Copyright (c) 2018 - 2022 by
%%                  chipforge <stdcelllib@nospam.chipforge.org>
%%  All rights reserved.
%%
%%      This Standard Cell Library is licensed under the Libre Silicon
%%      public license; you can redistribute it and/or modify it under
%%      the terms of the Libre Silicon public license as published by
%%      the Libre Silicon alliance, either version 1 of the License, or
%%      (at your option) any later version.
%%
%%      This design is distributed in the hope that it will be useful,
%%      but WITHOUT ANY WARRANTY; without even the implied warranty of
%%      MERCHANTABILITY or FITNESS FOR A PARTICULAR PURPOSE.
%%      See the Libre Silicon Public License for more details.
%%
%%  ///////////////////////////////////////////////////////////////////
\begin{circuitdiagram}[draft]{52}{16}

    \usgate
    % ----  1st column  ----
    \pin{1}{1}{L}{A}
    \pin{1}{5}{L}{A1}
    \gate[\inputs{2}]{or}{5}{3}{R}{}{}

    % ----  2nd column  ----
    \pin{8}{7}{L}{B}
    \gate[\inputs{2}]{and}{12}{5}{R}{}{}

    % ----  3rd column  ----
    \pin{15}{9}{L}{C}
    \gate[\inputs{2}]{or}{19}{7}{R}{}{}

    % ----  4th column  ----
    \pin{22}{11}{L}{D}
    \gate[\inputs{2}]{and}{26}{9}{R}{}{}

    % ----  5th column  ----
    \pin{29}{13}{L}{E}
    \gate[\inputs{2}]{or}{33}{11}{R}{}{}

    % ----  6th column  ----
    \pin{36}{15}{L}{E}
    \gate[\inputs{2}]{nand}{40}{13}{R}{}{}

    % ----  7th column  ----
    \gate{not}{47}{13}{R}{}{}

    % ----  result ----
    \pin{51}{13}{R}{Z}

\end{circuitdiagram}


%%  ************    LibreSilicon's StdCellLibrary   *******************
%%
%%  Organisation:   Chipforge
%%                  Germany / European Union
%%
%%  Profile:        Chipforge focus on fine System-on-Chip Cores in
%%                  Verilog HDL Code which are easy understandable and
%%                  adjustable. For further information see
%%                          www.chipforge.org
%%                  there are projects from small cores up to PCBs, too.
%%
%%  File:           StdCellLib/Documents/LaTeX/section-AOAOAOI_complex.tex
%%
%%  Purpose:        Section Level File for Standard Cell Library Documentation
%%
%%  ************    LaTeX with circdia.sty package      ***************
%%
%%  ///////////////////////////////////////////////////////////////////
%%
%%  Copyright (c) 2018 - 2021 by
%%                  chipforge <stdcelllib@nospam.chipforge.org>
%%  All rights reserved.
%%
%%      This Standard Cell Library is licensed under the Libre Silicon
%%      public license; you can redistribute it and/or modify it under
%%      the terms of the Libre Silicon public license as published by
%%      the Libre Silicon alliance, either version 1 of the License, or
%%      (at your option) any later version.
%%
%%      This design is distributed in the hope that it will be useful,
%%      but WITHOUT ANY WARRANTY; without even the implied warranty of
%%      MERCHANTABILITY or FITNESS FOR A PARTICULAR PURPOSE.
%%      See the Libre Silicon Public License for more details.
%%
\section{AND-OR-AND-OR-AND-OR(-Invert) Complex Gates}

\include{AOAOAOI211111_datasheet} \include{AOAOAO211111_datasheet}

%%  ************    LibreSilicon's StdCellLibrary   *******************
%%
%%  Organisation:   Chipforge
%%                  Germany / European Union
%%
%%  Profile:        Chipforge focus on fine System-on-Chip Cores in
%%                  Verilog HDL Code which are easy understandable and
%%                  adjustable. For further information see
%%                          www.chipforge.org
%%                  there are projects from small cores up to PCBs, too.
%%
%%  File:           StdCellLib/Documents/section-OAOAOOAI.tex
%%
%%  Purpose:        Section Level File for Standard Cell Library Documentation
%%
%%  ************    LaTeX with circdia.sty package      ***************
%%
%%  ///////////////////////////////////////////////////////////////////
%%
%%  Copyright (c) 2018 - 2022 by
%%                  chipforge <stdcelllib@nospam.chipforge.org>
%%  All rights reserved.
%%
%%      This Standard Cell Library is licensed under the Libre Silicon
%%      public license; you can redistribute it and/or modify it under
%%      the terms of the Libre Silicon public license as published by
%%      the Libre Silicon alliance, either version 1 of the License, or
%%      (at your option) any later version.
%%
%%      This design is distributed in the hope that it will be useful,
%%      but WITHOUT ANY WARRANTY; without even the implied warranty of
%%      MERCHANTABILITY or FITNESS FOR A PARTICULAR PURPOSE.
%%      See the Libre Silicon Public License for more details.
%%
%%  ///////////////////////////////////////////////////////////////////
\section{OR-AND-OR-AND-OR-OR-AND(-Invert) Complex Gates}

%%  ************    LibreSilicon's StdCellLibrary   *******************
%%
%%  Organisation:   Chipforge
%%                  Germany / European Union
%%
%%  Profile:        Chipforge focus on fine System-on-Chip Cores in
%%                  Verilog HDL Code which are easy understandable and
%%                  adjustable. For further information see
%%                          www.chipforge.org
%%                  there are projects from small cores up to PCBs, too.
%%
%%  File:           StdCellLib/Documents/Datasheets/Circuitry/OAOAOOAI211112.tex
%%
%%  Purpose:        Circuit File for OAOAOOAI211112
%%
%%  ************    LaTeX with circdia.sty package      ***************
%%
%%  ///////////////////////////////////////////////////////////////////
%%
%%  Copyright (c) 2018 - 2022 by
%%                  chipforge <stdcelllib@nospam.chipforge.org>
%%  All rights reserved.
%%
%%      This Standard Cell Library is licensed under the Libre Silicon
%%      public license; you can redistribute it and/or modify it under
%%      the terms of the Libre Silicon public license as published by
%%      the Libre Silicon alliance, either version 1 of the License, or
%%      (at your option) any later version.
%%
%%      This design is distributed in the hope that it will be useful,
%%      but WITHOUT ANY WARRANTY; without even the implied warranty of
%%      MERCHANTABILITY or FITNESS FOR A PARTICULAR PURPOSE.
%%      See the Libre Silicon Public License for more details.
%%
%%  ///////////////////////////////////////////////////////////////////
\begin{circuitdiagram}[draft]{46}{20}

    \usgate
    % ----  1st column  ----
    \pin{1}{1}{L}{A}
    \pin{1}{5}{L}{A1}
    \gate[\inputs{2}]{or}{5}{3}{R}{}{}

    % ----  2nd column  ----
    \pin{8}{7}{L}{B}
    \gate[\inputs{2}]{and}{12}{5}{R}{}{}

    % ----  3rd column  ----
    \pin{15}{9}{L}{C}
    \gate[\inputs{2}]{or}{19}{7}{R}{}{}

    % ----  4th column  ----
    \pin{22}{11}{L}{D}
    \gate[\inputs{2}]{and}{26}{9}{R}{}{}

    % ----  5th column  ----
    \pin{29}{13}{L}{E}
    \gate[\inputs{2}]{or}{33}{11}{R}{}{}

    \pin{29}{15}{L}{F}
    \pin{29}{19}{L}{F1}
    \gate[\inputs{2}]{or}{33}{17}{R}{}{}

    % ----  6th column  ----
    \wire{37}{11}{37}{13}
    \gate[\inputs{2}]{nand}{40}{15}{R}{}{}


    % ----  result ----
    \pin{45}{15}{R}{Y}

\end{circuitdiagram}
 %%  ************    LibreSilicon's StdCellLibrary   *******************
%%
%%  Organisation:   Chipforge
%%                  Germany / European Union
%%
%%  Profile:        Chipforge focus on fine System-on-Chip Cores in
%%                  Verilog HDL Code which are easy understandable and
%%                  adjustable. For further information see
%%                          www.chipforge.org
%%                  there are projects from small cores up to PCBs, too.
%%
%%  File:           StdCellLib/Documents/Datasheets/Circuitry/OAOAOOA211112.tex
%%
%%  Purpose:        Circuit File for OAOAOOA211112
%%
%%  ************    LaTeX with circdia.sty package      ***************
%%
%%  ///////////////////////////////////////////////////////////////////
%%
%%  Copyright (c) 2018 - 2022 by
%%                  chipforge <stdcelllib@nospam.chipforge.org>
%%  All rights reserved.
%%
%%      This Standard Cell Library is licensed under the Libre Silicon
%%      public license; you can redistribute it and/or modify it under
%%      the terms of the Libre Silicon public license as published by
%%      the Libre Silicon alliance, either version 1 of the License, or
%%      (at your option) any later version.
%%
%%      This design is distributed in the hope that it will be useful,
%%      but WITHOUT ANY WARRANTY; without even the implied warranty of
%%      MERCHANTABILITY or FITNESS FOR A PARTICULAR PURPOSE.
%%      See the Libre Silicon Public License for more details.
%%
%%  ///////////////////////////////////////////////////////////////////
\begin{circuitdiagram}[draft]{52}{20}

    \usgate
    % ----  1st column  ----
    \pin{1}{1}{L}{A}
    \pin{1}{5}{L}{A1}
    \gate[\inputs{2}]{or}{5}{3}{R}{}{}

    % ----  2nd column  ----
    \pin{8}{7}{L}{B}
    \gate[\inputs{2}]{and}{12}{5}{R}{}{}

    % ----  3rd column  ----
    \pin{15}{9}{L}{C}
    \gate[\inputs{2}]{or}{19}{7}{R}{}{}

    % ----  4th column  ----
    \pin{22}{11}{L}{D}
    \gate[\inputs{2}]{and}{26}{9}{R}{}{}

    % ----  5th column  ----
    \pin{29}{13}{L}{E}
    \gate[\inputs{2}]{or}{33}{11}{R}{}{}

    \pin{29}{15}{L}{F}
    \pin{29}{19}{L}{F1}
    \gate[\inputs{2}]{or}{33}{17}{R}{}{}

    % ----  6th column  ----
    \wire{37}{11}{37}{13}
    \gate[\inputs{2}]{nand}{40}{15}{R}{}{}

    % ----  last column ----
    \gate{not}{47}{15}{R}{}{}

    % ----  result ----
    \pin{51}{15}{R}{Z}

\end{circuitdiagram}


%%  ************    LibreSilicon's StdCellLibrary   *******************
%%
%%  Organisation:   Chipforge
%%                  Germany / European Union
%%
%%  Profile:        Chipforge focus on fine System-on-Chip Cores in
%%                  Verilog HDL Code which are easy understandable and
%%                  adjustable. For further information see
%%                          www.chipforge.org
%%                  there are projects from small cores up to PCBs, too.
%%
%%  File:           StdCellLib/Documents/LaTeX/section-AOAOAAOI_complex.tex
%%
%%  Purpose:        Section Level File for Standard Cell Library Documentation
%%
%%  ************    LaTeX with circdia.sty package      ***************
%%
%%  ///////////////////////////////////////////////////////////////////
%%
%%  Copyright (c) 2018 - 2021 by
%%                  chipforge <stdcelllib@nospam.chipforge.org>
%%  All rights reserved.
%%
%%      This Standard Cell Library is licensed under the Libre Silicon
%%      public license; you can redistribute it and/or modify it under
%%      the terms of the Libre Silicon public license as published by
%%      the Libre Silicon alliance, either version 1 of the License, or
%%      (at your option) any later version.
%%
%%      This design is distributed in the hope that it will be useful,
%%      but WITHOUT ANY WARRANTY; without even the implied warranty of
%%      MERCHANTABILITY or FITNESS FOR A PARTICULAR PURPOSE.
%%      See the Libre Silicon Public License for more details.
%%
\section{AND-OR-AND-OR-AND-AND-OR(-Invert) Complex Gates}


%%  ************    LibreSilicon's StdCellLibrary   *******************
%%
%%  Organisation:   Chipforge
%%                  Germany / European Union
%%
%%  Profile:        Chipforge focus on fine System-on-Chip Cores in
%%                  Verilog HDL Code which are easy understandable and
%%                  adjustable. For further information see
%%                          www.chipforge.org
%%                  there are projects from small cores up to PCBs, too.
%%
%%  File:           StdCellLib/Documents/Book/section-OAAOAOOAI_complex.tex
%%
%%  Purpose:        Section Level File for Standard Cell Library Documentation
%%
%%  ************    LaTeX with circdia.sty package      ***************
%%
%%  ///////////////////////////////////////////////////////////////////
%%
%%  Copyright (c) 2018 - 2022 by
%%                  chipforge <stdcelllib@nospam.chipforge.org>
%%  All rights reserved.
%%
%%      This Standard Cell Library is licensed under the Libre Silicon
%%      public license; you can redistribute it and/or modify it under
%%      the terms of the Libre Silicon public license as published by
%%      the Libre Silicon alliance, either version 1 of the License, or
%%      (at your option) any later version.
%%
%%      This design is distributed in the hope that it will be useful,
%%      but WITHOUT ANY WARRANTY; without even the implied warranty of
%%      MERCHANTABILITY or FITNESS FOR A PARTICULAR PURPOSE.
%%      See the Libre Silicon Public License for more details.
%%
%%  ///////////////////////////////////////////////////////////////////
\section{OR-AND-AND-OR-AND-OR-OR-AND(-Invert) Complex Gates}


%%  ************    LibreSilicon's StdCellLibrary   *******************
%%
%%  Organisation:   Chipforge
%%                  Germany / European Union
%%
%%  Profile:        Chipforge focus on fine System-on-Chip Cores in
%%                  Verilog HDL Code which are easy understandable and
%%                  adjustable. For further information see
%%                          www.chipforge.org
%%                  there are projects from small cores up to PCBs, too.
%%
%%  File:           StdCellLib/Documents/section-AOOAOAAOI_complex.tex
%%
%%  Purpose:        Section Level File for Standard Cell Library Documentation
%%
%%  ************    LaTeX with circdia.sty package      ***************
%%
%%  ///////////////////////////////////////////////////////////////////
%%
%%  Copyright (c) 2018 - 2022 by
%%                  chipforge <stdcelllib@nospam.chipforge.org>
%%  All rights reserved.
%%
%%      This Standard Cell Library is licensed under the Libre Silicon
%%      public license; you can redistribute it and/or modify it under
%%      the terms of the Libre Silicon public license as published by
%%      the Libre Silicon alliance, either version 1 of the License, or
%%      (at your option) any later version.
%%
%%      This design is distributed in the hope that it will be useful,
%%      but WITHOUT ANY WARRANTY; without even the implied warranty of
%%      MERCHANTABILITY or FITNESS FOR A PARTICULAR PURPOSE.
%%      See the Libre Silicon Public License for more details.
%%
%%  ///////////////////////////////////////////////////////////////////
\section{AND-OR-OR-AND-OR-AND-AND-OR(-Invert) Complex Gates}


%%  ************    LibreSilicon's StdCellLibrary   *******************
%%
%%  Organisation:   Chipforge
%%                  Germany / European Union
%%
%%  Profile:        Chipforge focus on fine System-on-Chip Cores in
%%                  Verilog HDL Code which are easy understandable and
%%                  adjustable. For further information see
%%                          www.chipforge.org
%%                  there are projects from small cores up to PCBs, too.
%%
%%  File:           StdCellLib/Documents/section-OAOOAOOAI_complex.tex
%%
%%  Purpose:        Section Level File for Standard Cell Library Documentation
%%
%%  ************    LaTeX with circdia.sty package      ***************
%%
%%  ///////////////////////////////////////////////////////////////////
%%
%%  Copyright (c) 2018 - 2022 by
%%                  chipforge <stdcelllib@nospam.chipforge.org>
%%  All rights reserved.
%%
%%      This Standard Cell Library is licensed under the Libre Silicon
%%      public license; you can redistribute it and/or modify it under
%%      the terms of the Libre Silicon public license as published by
%%      the Libre Silicon alliance, either version 1 of the License, or
%%      (at your option) any later version.
%%
%%      This design is distributed in the hope that it will be useful,
%%      but WITHOUT ANY WARRANTY; without even the implied warranty of
%%      MERCHANTABILITY or FITNESS FOR A PARTICULAR PURPOSE.
%%      See the Libre Silicon Public License for more details.
%%
%%  ///////////////////////////////////////////////////////////////////
\section{OR-AND-OR-OR-AND-OR-OR-AND(-Invert) Complex Gates}

\include{Datasheets/OAOOAOOAI211212} \include{Datasheets/OAOOAOOA211212}

%%  ************    LibreSilicon's StdCellLibrary   *******************
%%
%%  Organisation:   Chipforge
%%                  Germany / European Union
%%
%%  Profile:        Chipforge focus on fine System-on-Chip Cores in
%%                  Verilog HDL Code which are easy understandable and
%%                  adjustable. For further information see
%%                          www.chipforge.org
%%                  there are projects from small cores up to PCBs, too.
%%
%%  File:           StdCellLib/Documents/LaTeX/section-AOAAOAAOI_complex.tex
%%
%%  Purpose:        Section Level File for Standard Cell Library Documentation
%%
%%  ************    LaTeX with circdia.sty package      ***************
%%
%%  ///////////////////////////////////////////////////////////////////
%%
%%  Copyright (c) 2018 - 2021 by
%%                  chipforge <stdcelllib@nospam.chipforge.org>
%%  All rights reserved.
%%
%%      This Standard Cell Library is licensed under the Libre Silicon
%%      public license; you can redistribute it and/or modify it under
%%      the terms of the Libre Silicon public license as published by
%%      the Libre Silicon alliance, either version 1 of the License, or
%%      (at your option) any later version.
%%
%%      This design is distributed in the hope that it will be useful,
%%      but WITHOUT ANY WARRANTY; without even the implied warranty of
%%      MERCHANTABILITY or FITNESS FOR A PARTICULAR PURPOSE.
%%      See the Libre Silicon Public License for more details.
%%
\section{AND-OR-AND-AND-OR-AND-AND-OR(-Invert) Complex Gates}


%%  ************    LibreSilicon's StdCellLibrary   *******************
%%
%%  Organisation:   Chipforge
%%                  Germany / European Union
%%
%%  Profile:        Chipforge focus on fine System-on-Chip Cores in
%%                  Verilog HDL Code which are easy understandable and
%%                  adjustable. For further information see
%%                          www.chipforge.org
%%                  there are projects from small cores up to PCBs, too.
%%
%%  File:           StdCellLib/Documents/section-OAOAAOOAI_complex.tex
%%
%%  Purpose:        Section Level File for Standard Cell Library Documentation
%%
%%  ************    LaTeX with circdia.sty package      ***************
%%
%%  ///////////////////////////////////////////////////////////////////
%%
%%  Copyright (c) 2018 - 2022 by
%%                  chipforge <stdcelllib@nospam.chipforge.org>
%%  All rights reserved.
%%
%%      This Standard Cell Library is licensed under the Libre Silicon
%%      public license; you can redistribute it and/or modify it under
%%      the terms of the Libre Silicon public license as published by
%%      the Libre Silicon alliance, either version 1 of the License, or
%%      (at your option) any later version.
%%
%%      This design is distributed in the hope that it will be useful,
%%      but WITHOUT ANY WARRANTY; without even the implied warranty of
%%      MERCHANTABILITY or FITNESS FOR A PARTICULAR PURPOSE.
%%      See the Libre Silicon Public License for more details.
%%
%%  ///////////////////////////////////////////////////////////////////
\section{OR-AND-OR-AND-AND-OR-OR-AND(-Invert) Complex Gates}


%%  ************    LibreSilicon's StdCellLibrary   *******************
%%
%%  Organisation:   Chipforge
%%                  Germany / European Union
%%
%%  Profile:        Chipforge focus on fine System-on-Chip Cores in
%%                  Verilog HDL Code which are easy understandable and
%%                  adjustable. For further information see
%%                          www.chipforge.org
%%                  there are projects from small cores up to PCBs, too.
%%
%%  File:           StdCellLib/Documents/section-AOAOOAAOI_complex.tex
%%
%%  Purpose:        Section Level File for Standard Cell Library Documentation
%%
%%  ************    LaTeX with circdia.sty package      ***************
%%
%%  ///////////////////////////////////////////////////////////////////
%%
%%  Copyright (c) 2018 - 2022 by
%%                  chipforge <stdcelllib@nospam.chipforge.org>
%%  All rights reserved.
%%
%%      This Standard Cell Library is licensed under the Libre Silicon
%%      public license; you can redistribute it and/or modify it under
%%      the terms of the Libre Silicon public license as published by
%%      the Libre Silicon alliance, either version 1 of the License, or
%%      (at your option) any later version.
%%
%%      This design is distributed in the hope that it will be useful,
%%      but WITHOUT ANY WARRANTY; without even the implied warranty of
%%      MERCHANTABILITY or FITNESS FOR A PARTICULAR PURPOSE.
%%      See the Libre Silicon Public License for more details.
%%
%%  ///////////////////////////////////////////////////////////////////
\section{AND-OR-AND-OR-OR-AND-AND-OR(-Invert) Complex Gates}

%%  ************    LibreSilicon's StdCellLibrary   *******************
%%
%%  Organisation:   Chipforge
%%                  Germany / European Union
%%
%%  Profile:        Chipforge focus on fine System-on-Chip Cores in
%%                  Verilog HDL Code which are easy understandable and
%%                  adjustable. For further information see
%%                          www.chipforge.org
%%                  there are projects from small cores up to PCBs, too.
%%
%%  File:           StdCellLib/Documents/Datasheets/Circuitry/AOAOOAAOI211122.tex
%%
%%  Purpose:        Circuit File for AOAOOAAOI211122 
%%
%%  ************    LaTeX with circdia.sty package      ***************
%%
%%  ///////////////////////////////////////////////////////////////////
%%
%%  Copyright (c) 2018 - 2022 by
%%                  chipforge <stdcelllib@nospam.chipforge.org>
%%  All rights reserved.
%%
%%      This Standard Cell Library is licensed under the Libre Silicon
%%      public license; you can redistribute it and/or modify it under
%%      the terms of the Libre Silicon public license as published by
%%      the Libre Silicon alliance, either version 1 of the License, or
%%      (at your option) any later version.
%%
%%      This design is distributed in the hope that it will be useful,
%%      but WITHOUT ANY WARRANTY; without even the implied warranty of
%%      MERCHANTABILITY or FITNESS FOR A PARTICULAR PURPOSE.
%%      See the Libre Silicon Public License for more details.
%%
%%  ///////////////////////////////////////////////////////////////////
\begin{circuitdiagram}[draft]{46}{24}

    \usgate
    % ----  1st column  ----
    \pin{1}{1}{L}{A}
    \pin{1}{5}{L}{A1}
    \gate[\inputs{2}]{and}{5}{3}{R}{}{}

    % ----  2nd column  ----
    \pin{8}{7}{L}{B}
    \gate[\inputs{2}]{or}{12}{5}{R}{}{}

    % ----  3rd column  ----
    \pin{15}{9}{L}{C}
    \gate[\inputs{2}]{and}{19}{7}{R}{}{}

    % ----  4th column  ----
    \pin{22}{11}{L}{D}
    \gate[\inputs{2}]{or}{26}{9}{R}{}{}

    \pin{22}{13}{L}{E}
    \pin{22}{17}{L}{E1}
    \gate[\inputs{2}]{or}{26}{15}{R}{}{}

    % ----  5th column  ----
    \wire{30}{9}{30}{11}
    \gate[\inputs{2}]{and}{33}{13}{R}{}{}

    \pin{29}{19}{L}{F}
    \pin{29}{23}{L}{F1}
    \gate[\inputs{2}]{and}{33}{21}{R}{}{}

    % ----  6th column  ----
    \wire{37}{13}{37}{17}
    \gate[\inputs{2}]{nor}{40}{19}{R}{}{}


    % ----  result ----
    \pin{45}{19}{R}{Y}

\end{circuitdiagram}
 %%  ************    LibreSilicon's StdCellLibrary   *******************
%%
%%  Organisation:   Chipforge
%%                  Germany / European Union
%%
%%  Profile:        Chipforge focus on fine System-on-Chip Cores in
%%                  Verilog HDL Code which are easy understandable and
%%                  adjustable. For further information see
%%                          www.chipforge.org
%%                  there are projects from small cores up to PCBs, too.
%%
%%  File:           StdCellLib/Documents/Datasheets/Circuitry/AOAOOAAO211122.tex
%%
%%  Purpose:        Circuit File for AOAOOAAO211122 
%%
%%  ************    LaTeX with circdia.sty package      ***************
%%
%%  ///////////////////////////////////////////////////////////////////
%%
%%  Copyright (c) 2018 - 2022 by
%%                  chipforge <stdcelllib@nospam.chipforge.org>
%%  All rights reserved.
%%
%%      This Standard Cell Library is licensed under the Libre Silicon
%%      public license; you can redistribute it and/or modify it under
%%      the terms of the Libre Silicon public license as published by
%%      the Libre Silicon alliance, either version 1 of the License, or
%%      (at your option) any later version.
%%
%%      This design is distributed in the hope that it will be useful,
%%      but WITHOUT ANY WARRANTY; without even the implied warranty of
%%      MERCHANTABILITY or FITNESS FOR A PARTICULAR PURPOSE.
%%      See the Libre Silicon Public License for more details.
%%
%%  ///////////////////////////////////////////////////////////////////
\begin{circuitdiagram}[draft]{52}{24}

    \usgate
    % ----  1st column  ----
    \pin{1}{1}{L}{A}
    \pin{1}{5}{L}{A1}
    \gate[\inputs{2}]{and}{5}{3}{R}{}{}

    % ----  2nd column  ----
    \pin{8}{7}{L}{B}
    \gate[\inputs{2}]{or}{12}{5}{R}{}{}

    % ----  3rd column  ----
    \pin{15}{9}{L}{C}
    \gate[\inputs{2}]{and}{19}{7}{R}{}{}

    % ----  4th column  ----
    \pin{22}{11}{L}{D}
    \gate[\inputs{2}]{or}{26}{9}{R}{}{}

    \pin{22}{13}{L}{E}
    \pin{22}{17}{L}{E1}
    \gate[\inputs{2}]{or}{26}{15}{R}{}{}

    % ----  5th column  ----
    \wire{30}{9}{30}{11}
    \gate[\inputs{2}]{and}{33}{13}{R}{}{}

    \pin{29}{19}{L}{F}
    \pin{29}{23}{L}{F1}
    \gate[\inputs{2}]{and}{33}{21}{R}{}{}

    % ----  6th column  ----
    \wire{37}{13}{37}{17}
    \gate[\inputs{2}]{nor}{40}{19}{R}{}{}

    % ----  last column  ----
    \gate{not}{47}{19}{R}{}{}

    % ----  result ----
    \pin{51}{19}{R}{Z}

\end{circuitdiagram}


%%  ************    LibreSilicon's StdCellLibrary   *******************
%%
%%  Organisation:   Chipforge
%%                  Germany / European Union
%%
%%  Profile:        Chipforge focus on fine System-on-Chip Cores in
%%                  Verilog HDL Code which are easy understandable and
%%                  adjustable. For further information see
%%                          www.chipforge.org
%%                  there are projects from small cores up to PCBs, too.
%%
%%  File:           StdCellLib/Documents/section-OAAOAAOOAI_complex.tex
%%
%%  Purpose:        Section Level File for Standard Cell Library Documentation
%%
%%  ************    LaTeX with circdia.sty package      ***************
%%
%%  ///////////////////////////////////////////////////////////////////
%%
%%  Copyright (c) 2018 - 2022 by
%%                  chipforge <stdcelllib@nospam.chipforge.org>
%%  All rights reserved.
%%
%%      This Standard Cell Library is licensed under the Libre Silicon
%%      public license; you can redistribute it and/or modify it under
%%      the terms of the Libre Silicon public license as published by
%%      the Libre Silicon alliance, either version 1 of the License, or
%%      (at your option) any later version.
%%
%%      This design is distributed in the hope that it will be useful,
%%      but WITHOUT ANY WARRANTY; without even the implied warranty of
%%      MERCHANTABILITY or FITNESS FOR A PARTICULAR PURPOSE.
%%      See the Libre Silicon Public License for more details.
%%
%%  ///////////////////////////////////////////////////////////////////
\section{OR-AND-AND-OR-AND-AND-OR-OR-AND(-Invert) Complex Gates}


%%  ************    LibreSilicon's StdCellLibrary   *******************
%%
%%  Organisation:   Chipforge
%%                  Germany / European Union
%%
%%  Profile:        Chipforge focus on fine System-on-Chip Cores in
%%                  Verilog HDL Code which are easy understandable and
%%                  adjustable. For further information see
%%                          www.chipforge.org
%%                  there are projects from small cores up to PCBs, too.
%%
%%  File:           StdCellLib/Documents/LaTeX/section-AOOAOOAAOI_complex.tex
%%
%%  Purpose:        Section Level File for Standard Cell Library Documentation
%%
%%  ************    LaTeX with circdia.sty package      ***************
%%
%%  ///////////////////////////////////////////////////////////////////
%%
%%  Copyright (c) 2018 - 2021 by
%%                  chipforge <stdcelllib@nospam.chipforge.org>
%%  All rights reserved.
%%
%%      This Standard Cell Library is licensed under the Libre Silicon
%%      public license; you can redistribute it and/or modify it under
%%      the terms of the Libre Silicon public license as published by
%%      the Libre Silicon alliance, either version 1 of the License, or
%%      (at your option) any later version.
%%
%%      This design is distributed in the hope that it will be useful,
%%      but WITHOUT ANY WARRANTY; without even the implied warranty of
%%      MERCHANTABILITY or FITNESS FOR A PARTICULAR PURPOSE.
%%      See the Libre Silicon Public License for more details.
%%
\section{AND-OR-OR-AND-OR-OR-AND-AND-OR(-Invert) Complex Gates}


%%  ************    LibreSilicon's StdCellLibrary   *******************
%%
%%  Organisation:   Chipforge
%%                  Germany / European Union
%%
%%  Profile:        Chipforge focus on fine System-on-Chip Cores in
%%                  Verilog HDL Code which are easy understandable and
%%                  adjustable. For further information see
%%                          www.chipforge.org
%%                  there are projects from small cores up to PCBs, too.
%%
%%  File:           StdCellLib/Documents/section-OOAAOAOOAI_complex.tex
%%
%%  Purpose:        Section Level File for Standard Cell Library Documentation
%%
%%  ************    LaTeX with circdia.sty package      ***************
%%
%%  ///////////////////////////////////////////////////////////////////
%%
%%  Copyright (c) 2018 - 2022 by
%%                  chipforge <stdcelllib@nospam.chipforge.org>
%%  All rights reserved.
%%
%%      This Standard Cell Library is licensed under the Libre Silicon
%%      public license; you can redistribute it and/or modify it under
%%      the terms of the Libre Silicon public license as published by
%%      the Libre Silicon alliance, either version 1 of the License, or
%%      (at your option) any later version.
%%
%%      This design is distributed in the hope that it will be useful,
%%      but WITHOUT ANY WARRANTY; without even the implied warranty of
%%      MERCHANTABILITY or FITNESS FOR A PARTICULAR PURPOSE.
%%      See the Libre Silicon Public License for more details.
%%
%%  ///////////////////////////////////////////////////////////////////
\section{OR-OR-AND-AND-OR-AND-OR-OR-AND(-Invert) Complex Gates}


%%  ************    LibreSilicon's StdCellLibrary   *******************
%%
%%  Organisation:   Chipforge
%%                  Germany / European Union
%%
%%  Profile:        Chipforge focus on fine System-on-Chip Cores in
%%                  Verilog HDL Code which are easy understandable and
%%                  adjustable. For further information see
%%                          www.chipforge.org
%%                  there are projects from small cores up to PCBs, too.
%%
%%  File:           StdCellLib/Documents/Book/section-AAOOAOAAOI_complex.tex
%%
%%  Purpose:        Section Level File for Standard Cell Library Documentation
%%
%%  ************    LaTeX with circdia.sty package      ***************
%%
%%  ///////////////////////////////////////////////////////////////////
%%
%%  Copyright (c) 2018 - 2022 by
%%                  chipforge <stdcelllib@nospam.chipforge.org>
%%  All rights reserved.
%%
%%      This Standard Cell Library is licensed under the Libre Silicon
%%      public license; you can redistribute it and/or modify it under
%%      the terms of the Libre Silicon public license as published by
%%      the Libre Silicon alliance, either version 1 of the License, or
%%      (at your option) any later version.
%%
%%      This design is distributed in the hope that it will be useful,
%%      but WITHOUT ANY WARRANTY; without even the implied warranty of
%%      MERCHANTABILITY or FITNESS FOR A PARTICULAR PURPOSE.
%%      See the Libre Silicon Public License for more details.
%%
%%  ///////////////////////////////////////////////////////////////////
\section{AND-AND-OR-OR-AND-OR-AND-AND-OR(-Invert) Complex Gates}


%%  ************    LibreSilicon's StdCellLibrary   *******************
%%
%%  Organisation:   Chipforge
%%                  Germany / European Union
%%
%%  Profile:        Chipforge focus on fine System-on-Chip Cores in
%%                  Verilog HDL Code which are easy understandable and
%%                  adjustable. For further information see
%%                          www.chipforge.org
%%                  there are projects from small cores up to PCBs, too.
%%
%%  File:           StdCellLib/Documents/Book/section-OOAAOAAOOAI_complex.tex
%%
%%  Purpose:        Section Level File for Standard Cell Library Documentation
%%
%%  ************    LaTeX with circdia.sty package      ***************
%%
%%  ///////////////////////////////////////////////////////////////////
%%
%%  Copyright (c) 2018 - 2022 by
%%                  chipforge <stdcelllib@nospam.chipforge.org>
%%  All rights reserved.
%%
%%      This Standard Cell Library is licensed under the Libre Silicon
%%      public license; you can redistribute it and/or modify it under
%%      the terms of the Libre Silicon public license as published by
%%      the Libre Silicon alliance, either version 1 of the License, or
%%      (at your option) any later version.
%%
%%      This design is distributed in the hope that it will be useful,
%%      but WITHOUT ANY WARRANTY; without even the implied warranty of
%%      MERCHANTABILITY or FITNESS FOR A PARTICULAR PURPOSE.
%%      See the Libre Silicon Public License for more details.
%%
%%  ///////////////////////////////////////////////////////////////////
\section{OR-OR-AND-AND-OR-AND-AND-OR-OR-AND(-Invert) Complex Gates}


%%  ************    LibreSilicon's StdCellLibrary   *******************
%%
%%  Organisation:   Chipforge
%%                  Germany / European Union
%%
%%  Profile:        Chipforge focus on fine System-on-Chip Cores in
%%                  Verilog HDL Code which are easy understandable and
%%                  adjustable. For further information see
%%                          www.chipforge.org
%%                  there are projects from small cores up to PCBs, too.
%%
%%  File:           StdCellLib/Documents/section-AAOOAOOAAOI_complex.tex
%%
%%  Purpose:        Section Level File for Standard Cell Library Documentation
%%
%%  ************    LaTeX with circdia.sty package      ***************
%%
%%  ///////////////////////////////////////////////////////////////////
%%
%%  Copyright (c) 2018 - 2022 by
%%                  chipforge <stdcelllib@nospam.chipforge.org>
%%  All rights reserved.
%%
%%      This Standard Cell Library is licensed under the Libre Silicon
%%      public license; you can redistribute it and/or modify it under
%%      the terms of the Libre Silicon public license as published by
%%      the Libre Silicon alliance, either version 1 of the License, or
%%      (at your option) any later version.
%%
%%      This design is distributed in the hope that it will be useful,
%%      but WITHOUT ANY WARRANTY; without even the implied warranty of
%%      MERCHANTABILITY or FITNESS FOR A PARTICULAR PURPOSE.
%%      See the Libre Silicon Public License for more details.
%%
%%  ///////////////////////////////////////////////////////////////////
\section{AND-AND-OR-OR-AND-OR-OR-AND-AND-OR(-Invert) Complex Gates}



%%  ------------    miscellaneous   -----------------------------------

\section{Miscellaneous Cells}

%%  ************    LibreSilicon's StdCellLibrary   *******************
%%
%%  Organisation:   Chipforge
%%                  Germany / European Union
%%
%%  Profile:        Chipforge focus on fine System-on-Chip Cores in
%%                  Verilog HDL Code which are easy understandable and
%%                  adjustable. For further information see
%%                          www.chipforge.org
%%                  there are projects from small cores up to PCBs, too.
%%
%%  File:           StdCellLib/Documents/LaTeX/MAJI23_manpage.tex
%%
%%  Purpose:        Auto-generated Manual Page for MAJI23
%%
%%  ************    LaTeX with circdia.sty package      ***************
%%
%%  ///////////////////////////////////////////////////////////////////
%%
%%  Copyright (c) 2021 by
%%                  chipforge <stdcelllib@nospam.chipforge.org>
%%  All rights reserved.
%%
%%      This Standard Cell Library is licensed under the Libre Silicon
%%      public license; you can redistribute it and/or modify it under
%%      the terms of the Libre Silicon public license as published by
%%      the Libre Silicon alliance, either version 1 of the License, or
%%      (at your option) any later version.
%%
%%      This design is distributed in the hope that it will be useful,
%%      but WITHOUT ANY WARRANTY; without even the implied warranty of
%%      MERCHANTABILITY or FITNESS FOR A PARTICULAR PURPOSE.
%%      See the Libre Silicon Public License for more details.
%%
%%  ///////////////////////////////////////////////////////////////////
\subsection{MAJI23 - an inverting 2-of-3 Majority gate} \label{logical:MAJI23}

\paragraph{Synopsys}
\begin{quote}
    MAJI23 (Z A2 A1 A)
\end{quote}

\paragraph{Description}
\input{MAJI23_circuit.tex}
%\input{MAJI23_schematic.tex}

\paragraph{Truth Table}
%\input{MAJI23_truthtable.tex}

\paragraph{Usage}

\paragraph{Fan-in / Fan-out}

\paragraph{Layout}

\paragraph{Files}

\clearpage



%%  -------------------------------------------------------------------
%%                  CHAPTER 3
%%  -------------------------------------------------------------------

\chapter{Storage Cells}
\clearpage

%%  ************    LibreSilicon's StdCellLibrary   *******************
%%
%%  Organisation:   Chipforge
%%                  Germany / European Union
%%
%%  Profile:        Chipforge focus on fine System-on-Chip Cores in
%%                  Verilog HDL Code which are easy understandable and
%%                  adjustable. For further information see
%%                          www.chipforge.org
%%                  there are projects from small cores up to PCBs, too.
%%
%%  File:           StdCellLib/Documents/Book/section-LAT_latches.tex
%%
%%  Purpose:        Section Level File for Standard Cell Library Documentation
%%
%%  ************    LaTeX with circdia.sty package      ***************
%%
%%  ///////////////////////////////////////////////////////////////////
%%
%%  Copyright (c) 2018 - 2022 by
%%                  chipforge <stdcelllib@nospam.chipforge.org>
%%  All rights reserved.
%%
%%      This Standard Cell Library is licensed under the Libre Silicon
%%      public license; you can redistribute it and/or modify it under
%%      the terms of the Libre Silicon public license as published by
%%      the Libre Silicon alliance, either version 1 of the License, or
%%      (at your option) any later version.
%%
%%      This design is distributed in the hope that it will be useful,
%%      but WITHOUT ANY WARRANTY; without even the implied warranty of
%%      MERCHANTABILITY or FITNESS FOR A PARTICULAR PURPOSE.
%%      See the Libre Silicon Public License for more details.
%%
%%  ///////////////////////////////////////////////////////////////////
\section{D-type Latches}

%%  ************    LibreSilicon's StdCellLibrary   *******************
%%
%%  Organisation:   Chipforge
%%                  Germany / European Union
%%
%%  Profile:        Chipforge focus on fine System-on-Chip Cores in
%%                  Verilog HDL Code which are easy understandable and
%%                  adjustable. For further information see
%%                          www.chipforge.org
%%                  there are projects from small cores up to PCBs, too.
%%
%%  File:           StdCellLib/Documents/Circuits/LATN.tex
%%
%%  Purpose:        Circuit File for LATN
%%
%%  ************    LaTeX with circdia.sty package      ***************
%%
%%  ///////////////////////////////////////////////////////////////////
%%
%%  Copyright (c) 2019 by chipforge <stdcelllib@nospam.chipforge.org>
%%  All rights reserved.
%%
%%      This Standard Cell Library is licensed under the Libre Silicon
%%      public license; you can redistribute it and/or modify it under
%%      the terms of the Libre Silicon public license as published by
%%      the Libre Silicon alliance, either version 1 of the License, or
%%      (at your option) any later version.
%%
%%      This design is distributed in the hope that it will be useful,
%%      but WITHOUT ANY WARRANTY; without even the implied warranty of
%%      MERCHANTABILITY or FITNESS FOR A PARTICULAR PURPOSE.
%%      See the Libre Silicon Public License for more details.
%%
%%  ///////////////////////////////////////////////////////////////////
\begin{center}
    Circuit
    \begin{figure}[h]
        \begin{center}
            \begin{circuitdiagram}{24}{17}
            \pin{1}{3}{L}{XN}   % pin XN
            \usgate
            \gate{not}{5}{3}{R}{}{}
            \wire{8}{3}{10}{3}
            \junct{9}{3}
            \wire{9}{3}{9}{7}
            \gate{not}{13}{3}{R}{}{}
            \wire{16}{3}{17}{3}
            \wire{17}{3}{17}{7}
            \pin{1}{9}{L}{D}    % pin D
            \wire{2}{9}{6}{9}
            \gate[\tristate{Dn}]{not}{9}{9}{R}{}{}
            \wire{12}{9}{14}{9}
            \junct{13}{9}
            \wire{13}{9}{13}{14}
            \wire{13}{14}{14}{14}
            \gate[\tristate{Dn}]{not}{17}{9}{L}{}{}
            \gate{not}{17}{14}{R}{}{}
            \wire{20}{9}{22}{9}
            \junct{21}{9}
            \wire{21}{9}{21}{14}
            \wire{20}{14}{21}{14}
            \pin{23}{9}{R}{Q}   % pin Q
            \end{circuitdiagram}
        \end{center}
    \end{figure}
\end{center}
 %%  ************    LibreSilicon's StdCellLibrary   *******************
%%
%%  Organisation:   Chipforge
%%                  Germany / European Union
%%
%%  Profile:        Chipforge focus on fine System-on-Chip Cores in
%%                  Verilog HDL Code which are easy understandable and
%%                  adjustable. For further information see
%%                          www.chipforge.org
%%                  there are projects from small cores up to PCBs, too.
%%
%%  File:           StdCellLib/Documents/Circuits/LATP.tex
%%
%%  Purpose:        Circuit File for LATP
%%
%%  ************    LaTeX with circdia.sty package      ***************
%%
%%  ///////////////////////////////////////////////////////////////////
%%
%%  Copyright (c) 2019 by chipforge <stdcelllib@nospam.chipforge.org>
%%  All rights reserved.
%%
%%      This Standard Cell Library is licensed under the Libre Silicon
%%      public license; you can redistribute it and/or modify it under
%%      the terms of the Libre Silicon public license as published by
%%      the Libre Silicon alliance, either version 1 of the License, or
%%      (at your option) any later version.
%%
%%      This design is distributed in the hope that it will be useful,
%%      but WITHOUT ANY WARRANTY; without even the implied warranty of
%%      MERCHANTABILITY or FITNESS FOR A PARTICULAR PURPOSE.
%%      See the Libre Silicon Public License for more details.
%%
%%  ///////////////////////////////////////////////////////////////////
\begin{center}
    Circuit
    \begin{figure}[h]
        \begin{center}
            \begin{circuitdiagram}{24}{17}
            \pin{1}{3}{L}{X}    % pin X
            \usgate
            \gate{not}{5}{3}{R}{}{}
            \wire{8}{3}{10}{3}
            \junct{9}{3}
            \wire{9}{3}{9}{7}
            \gate{not}{13}{3}{R}{}{}
            \wire{16}{3}{17}{3}
            \wire{17}{3}{17}{7}
            \pin{1}{9}{L}{D}    % pin D
            \wire{2}{9}{6}{9}
            \gate[\tristate{Dp}]{not}{9}{9}{R}{}{}
            \wire{12}{9}{14}{9}
            \junct{13}{9}
            \wire{13}{9}{13}{14}
            \wire{13}{14}{14}{14}
            \gate[\tristate{Dp}]{not}{17}{9}{L}{}{}
            \gate{not}{17}{14}{R}{}{}
            \wire{20}{9}{22}{9}
            \junct{21}{9}
            \wire{21}{9}{21}{14}
            \wire{20}{14}{21}{14}
            \pin{23}{9}{R}{Q}   % pin Q
            \end{circuitdiagram}
        \end{center}
    \end{figure}
\end{center}

%%  ************    LibreSilicon's StdCellLibrary   *******************
%%
%%  Organisation:   Chipforge
%%                  Germany / European Union
%%
%%  Profile:        Chipforge focus on fine System-on-Chip Cores in
%%                  Verilog HDL Code which are easy understandable and
%%                  adjustable. For further information see
%%                          www.chipforge.org
%%                  there are projects from small cores up to PCBs, too.
%%
%%  File:           StdCellLib/Documents/Circuits/LATRN.tex
%%
%%  Purpose:        Circuit File for LATRN
%%
%%  ************    LaTeX with circdia.sty package      ***************
%%
%%  ///////////////////////////////////////////////////////////////////
%%
%%  Copyright (c) 2019 by chipforge <stdcelllib@nospam.chipforge.org>
%%  All rights reserved.
%%
%%      This Standard Cell Library is licensed under the Libre Silicon
%%      public license; you can redistribute it and/or modify it under
%%      the terms of the Libre Silicon public license as published by
%%      the Libre Silicon alliance, either version 1 of the License, or
%%      (at your option) any later version.
%%
%%      This design is distributed in the hope that it will be useful,
%%      but WITHOUT ANY WARRANTY; without even the implied warranty of
%%      MERCHANTABILITY or FITNESS FOR A PARTICULAR PURPOSE.
%%      See the Libre Silicon Public License for more details.
%%
%%  ///////////////////////////////////////////////////////////////////
\begin{center}
    Circuit
    \begin{figure}[h]
        \begin{center}
            \begin{circuitdiagram}{24}{17}
            \pin{1}{3}{L}{XN}   % pin XN
            \usgate
            \gate{not}{5}{3}{R}{}{}
            \wire{8}{3}{10}{3}
            \junct{9}{3}
            \wire{9}{3}{9}{7}
            \gate{not}{13}{3}{R}{}{}
            \wire{16}{3}{17}{3}
            \wire{17}{3}{17}{7}
            \pin{1}{9}{L}{D}    % pin D
            \wire{2}{9}{6}{9}
            \gate[\tristate{Dn}]{not}{9}{9}{R}{}{}
            \wire{12}{9}{14}{9}
            \junct{13}{9}
            \wire{13}{9}{13}{12}
            \gate[\tristate{Dn}]{not}{17}{9}{L}{}{}
            \pin{12}{16}{L}{R}   % pin R
            \gate{nor}{16}{14}{R}{}{}
            \wire{20}{9}{22}{9}
            \junct{21}{9}
            \wire{21}{9}{21}{14}
            \wire{20}{14}{21}{14}
            \pin{23}{9}{R}{Q}   % pin Q
            \end{circuitdiagram}
        \end{center}
    \end{figure}
\end{center}
 %%  ************    LibreSilicon's StdCellLibrary   *******************
%%
%%  Organisation:   Chipforge
%%                  Germany / European Union
%%
%%  Profile:        Chipforge focus on fine System-on-Chip Cores in
%%                  Verilog HDL Code which are easy understandable and
%%                  adjustable. For further information see
%%                          www.chipforge.org
%%                  there are projects from small cores up to PCBs, too.
%%
%%  File:           StdCellLib/Documents/Circuits/LATRP.tex
%%
%%  Purpose:        Circuit File for LATRP
%%
%%  ************    LaTeX with circdia.sty package      ***************
%%
%%  ///////////////////////////////////////////////////////////////////
%%
%%  Copyright (c) 2019 by chipforge <stdcelllib@nospam.chipforge.org>
%%  All rights reserved.
%%
%%      This Standard Cell Library is licensed under the Libre Silicon
%%      public license; you can redistribute it and/or modify it under
%%      the terms of the Libre Silicon public license as published by
%%      the Libre Silicon alliance, either version 1 of the License, or
%%      (at your option) any later version.
%%
%%      This design is distributed in the hope that it will be useful,
%%      but WITHOUT ANY WARRANTY; without even the implied warranty of
%%      MERCHANTABILITY or FITNESS FOR A PARTICULAR PURPOSE.
%%      See the Libre Silicon Public License for more details.
%%
%%  ///////////////////////////////////////////////////////////////////
\begin{center}
    Circuit
    \begin{figure}[h]
        \begin{center}
            \begin{circuitdiagram}{24}{17}
            \pin{1}{3}{L}{X}    % pin X
            \usgate
            \gate{not}{5}{3}{R}{}{}
            \wire{8}{3}{10}{3}
            \junct{9}{3}
            \wire{9}{3}{9}{7}
            \gate{not}{13}{3}{R}{}{}
            \wire{16}{3}{17}{3}
            \wire{17}{3}{17}{7}
            \pin{1}{9}{L}{D}    % pin D
            \wire{2}{9}{6}{9}
            \gate[\tristate{Dp}]{not}{9}{9}{R}{}{}
            \wire{12}{9}{14}{9}
            \junct{13}{9}
            \wire{13}{9}{13}{12}
            \gate[\tristate{Dp}]{not}{17}{9}{L}{}{}
            \pin{12}{16}{L}{R}   % pin R
            \gate{nor}{16}{14}{R}{}{}
            \wire{20}{9}{22}{9}
            \junct{21}{9}
            \wire{21}{9}{21}{14}
            \wire{20}{14}{21}{14}
            \pin{23}{9}{R}{Q}   % pin Q
            \end{circuitdiagram}
        \end{center}
    \end{figure}
\end{center}

%%  ************    LibreSilicon's StdCellLibrary   *******************
%%
%%  Organisation:   Chipforge
%%                  Germany / European Union
%%
%%  Profile:        Chipforge focus on fine System-on-Chip Cores in
%%                  Verilog HDL Code which are easy understandable and
%%                  adjustable. For further information see
%%                          www.chipforge.org
%%                  there are projects from small cores up to PCBs, too.
%%
%%  File:           StdCellLib/Documents/Circuits/LATSN.tex
%%
%%  Purpose:        Circuit File for LATSN
%%
%%  ************    LaTeX with circdia.sty package      ***************
%%
%%  ///////////////////////////////////////////////////////////////////
%%
%%  Copyright (c) 2019 by chipforge <stdcelllib@nospam.chipforge.org>
%%  All rights reserved.
%%
%%      This Standard Cell Library is licensed under the Libre Silicon
%%      public license; you can redistribute it and/or modify it under
%%      the terms of the Libre Silicon public license as published by
%%      the Libre Silicon alliance, either version 1 of the License, or
%%      (at your option) any later version.
%%
%%      This design is distributed in the hope that it will be useful,
%%      but WITHOUT ANY WARRANTY; without even the implied warranty of
%%      MERCHANTABILITY or FITNESS FOR A PARTICULAR PURPOSE.
%%      See the Libre Silicon Public License for more details.
%%
%%  ///////////////////////////////////////////////////////////////////
\begin{center}
    Circuit
    \begin{figure}[h]
        \begin{center}
            \begin{circuitdiagram}{24}{17}
            \pin{1}{3}{L}{XN}   % pin XN
            \usgate
            \gate{not}{5}{3}{R}{}{}
            \wire{8}{3}{10}{3}
            \junct{9}{3}
            \wire{9}{3}{9}{7}
            \gate{not}{13}{3}{R}{}{}
            \wire{16}{3}{17}{3}
            \wire{17}{3}{17}{7}
            \pin{1}{9}{L}{D}    % pin D
            \wire{2}{9}{6}{9}
            \gate[\tristate{Dn}]{not}{9}{9}{R}{}{}
            \wire{12}{9}{14}{9}
            \junct{13}{9}
            \wire{13}{9}{13}{12}
            \gate[\tristate{Dn}]{not}{17}{9}{L}{}{}
            \pin{12}{16}{L}{SN}  % pin SN
            \gate{nand}{16}{14}{R}{}{}
            \wire{20}{9}{22}{9}
            \junct{21}{9}
            \wire{21}{9}{21}{14}
            \wire{20}{14}{21}{14}
            \pin{23}{9}{R}{Q}   % pin Q
            \end{circuitdiagram}
        \end{center}
    \end{figure}
\end{center}
 %%  ************    LibreSilicon's StdCellLibrary   *******************
%%
%%  Organisation:   Chipforge
%%                  Germany / European Union
%%
%%  Profile:        Chipforge focus on fine System-on-Chip Cores in
%%                  Verilog HDL Code which are easy understandable and
%%                  adjustable. For further information see
%%                          www.chipforge.org
%%                  there are projects from small cores up to PCBs, too.
%%
%%  File:           StdCellLib/Documents/Circuits/LATSP.tex
%%
%%  Purpose:        Circuit File for LATSP
%%
%%  ************    LaTeX with circdia.sty package      ***************
%%
%%  ///////////////////////////////////////////////////////////////////
%%
%%  Copyright (c) 2019 by chipforge <stdcelllib@nospam.chipforge.org>
%%  All rights reserved.
%%
%%      This Standard Cell Library is licensed under the Libre Silicon
%%      public license; you can redistribute it and/or modify it under
%%      the terms of the Libre Silicon public license as published by
%%      the Libre Silicon alliance, either version 1 of the License, or
%%      (at your option) any later version.
%%
%%      This design is distributed in the hope that it will be useful,
%%      but WITHOUT ANY WARRANTY; without even the implied warranty of
%%      MERCHANTABILITY or FITNESS FOR A PARTICULAR PURPOSE.
%%      See the Libre Silicon Public License for more details.
%%
%%  ///////////////////////////////////////////////////////////////////
\begin{center}
    Circuit
    \begin{figure}[h]
        \begin{center}
            \begin{circuitdiagram}{24}{17}
            \pin{1}{3}{L}{X}    % pin X
            \usgate
            \gate{not}{5}{3}{R}{}{}
            \wire{8}{3}{10}{3}
            \junct{9}{3}
            \wire{9}{3}{9}{7}
            \gate{not}{13}{3}{R}{}{}
            \wire{16}{3}{17}{3}
            \wire{17}{3}{17}{7}
            \pin{1}{9}{L}{D}    % pin D
            \wire{2}{9}{6}{9}
            \gate[\tristate{Dp}]{not}{9}{9}{R}{}{}
            \wire{12}{9}{14}{9}
            \junct{13}{9}
            \wire{13}{9}{13}{12}
            \gate[\tristate{Dp}]{not}{17}{9}{L}{}{}
            \pin{12}{16}{L}{SN}  % pin SN
            \gate{nand}{16}{14}{R}{}{}
            \wire{20}{9}{22}{9}
            \junct{21}{9}
            \wire{21}{9}{21}{14}
            \wire{20}{14}{21}{14}
            \pin{23}{9}{R}{Q}   % pin Q
            \end{circuitdiagram}
        \end{center}
    \end{figure}
\end{center}

%%  ************    LibreSilicon's StdCellLibrary   *******************
%%
%%  Organisation:   Chipforge
%%                  Germany / European Union
%%
%%  Profile:        Chipforge focus on fine System-on-Chip Cores in
%%                  Verilog HDL Code which are easy understandable and
%%                  adjustable. For further information see
%%                          www.chipforge.org
%%                  there are projects from small cores up to PCBs, too.
%%
%%  File:           StdCellLib/Documents/Circuits/LATEN.tex
%%
%%  Purpose:        Circuit File for LATEN
%%
%%  ************    LaTeX with circdia.sty package      ***************
%%
%%  ///////////////////////////////////////////////////////////////////
%%
%%  Copyright (c) 2019 by chipforge <stdcelllib@nospam.chipforge.org>
%%  All rights reserved.
%%
%%      This Standard Cell Library is licensed under the Libre Silicon
%%      public license; you can redistribute it and/or modify it under
%%      the terms of the Libre Silicon public license as published by
%%      the Libre Silicon alliance, either version 1 of the License, or
%%      (at your option) any later version.
%%
%%      This design is distributed in the hope that it will be useful,
%%      but WITHOUT ANY WARRANTY; without even the implied warranty of
%%      MERCHANTABILITY or FITNESS FOR A PARTICULAR PURPOSE.
%%      See the Libre Silicon Public License for more details.
%%
%%  ///////////////////////////////////////////////////////////////////
\begin{center}
    Circuit
    \begin{figure}[h]
        \begin{center}
            \begin{circuitdiagram}{25}{17}
            \pin{1}{1}{L}{XN}   % pin XN
            \pin{1}{5}{L}{EN}   % pin EN
            \usgate
            \gate{nor}{5}{3}{R}{}{}
            \wire{9}{3}{11}{3}
            \junct{10}{3}
            \wire{10}{3}{10}{7}
            \gate{not}{14}{3}{R}{}{}
            \wire{17}{3}{18}{3}
            \wire{18}{3}{18}{7}
            \pin{1}{9}{L}{D}    % pin D
            \wire{2}{9}{7}{9}
            \gate[\tristate{Dn}]{not}{10}{9}{R}{}{}
            \wire{13}{9}{15}{9}
            \junct{14}{9}
            \wire{14}{9}{14}{14}
            \wire{14}{14}{15}{14}
            \gate[\tristate{Dn}]{not}{18}{9}{L}{}{}
            \gate{not}{18}{14}{R}{}{}
            \wire{21}{9}{23}{9}
            \junct{22}{9}
            \wire{22}{9}{22}{14}
            \wire{21}{14}{22}{14}
            \pin{24}{9}{R}{Q}   % pin Q
            \end{circuitdiagram}
        \end{center}
    \end{figure}
\end{center}
 %%  ************    LibreSilicon's StdCellLibrary   *******************
%%
%%  Organisation:   Chipforge
%%                  Germany / European Union
%%
%%  Profile:        Chipforge focus on fine System-on-Chip Cores in
%%                  Verilog HDL Code which are easy understandable and
%%                  adjustable. For further information see
%%                          www.chipforge.org
%%                  there are projects from small cores up to PCBs, too.
%%
%%  File:           StdCellLib/Documents/Circuits/LATEP.tex
%%
%%  Purpose:        Circuit File for LATEP
%%
%%  ************    LaTeX with circdia.sty package      ***************
%%
%%  ///////////////////////////////////////////////////////////////////
%%
%%  Copyright (c) 2019 by chipforge <stdcelllib@nospam.chipforge.org>
%%  All rights reserved.
%%
%%      This Standard Cell Library is licensed under the Libre Silicon
%%      public license; you can redistribute it and/or modify it under
%%      the terms of the Libre Silicon public license as published by
%%      the Libre Silicon alliance, either version 1 of the License, or
%%      (at your option) any later version.
%%
%%      This design is distributed in the hope that it will be useful,
%%      but WITHOUT ANY WARRANTY; without even the implied warranty of
%%      MERCHANTABILITY or FITNESS FOR A PARTICULAR PURPOSE.
%%      See the Libre Silicon Public License for more details.
%%
%%  ///////////////////////////////////////////////////////////////////
\begin{center}
    Circuit
    \begin{figure}[h]
        \begin{center}
            \begin{circuitdiagram}{25}{17}
            \pin{1}{1}{L}{X}    % pin N
            \pin{1}{5}{L}{E}    % pin E
            \usgate
            \gate{nand}{5}{3}{R}{}{}
            \wire{9}{3}{11}{3}
            \junct{10}{3}
            \wire{10}{3}{10}{7}
            \gate{not}{14}{3}{R}{}{}
            \wire{17}{3}{18}{3}
            \wire{18}{3}{18}{7}
            \pin{1}{9}{L}{D}    % pin D
            \wire{2}{9}{7}{9}
            \gate[\tristate{Dp}]{not}{10}{9}{R}{}{}
            \wire{13}{9}{15}{9}
            \junct{14}{9}
            \wire{14}{9}{14}{14}
            \wire{14}{14}{15}{14}
            \gate[\tristate{Dp}]{not}{18}{9}{L}{}{}
            \gate{not}{18}{14}{R}{}{}
            \wire{21}{9}{23}{9}
            \junct{22}{9}
            \wire{22}{9}{22}{14}
            \wire{21}{14}{22}{14}
            \pin{24}{9}{R}{Q}   % pin Q
            \end{circuitdiagram}
        \end{center}
    \end{figure}
\end{center}

%%  ************    LibreSilicon's StdCellLibrary   *******************
%%
%%  Organisation:   Chipforge
%%                  Germany / European Union
%%
%%  Profile:        Chipforge focus on fine System-on-Chip Cores in
%%                  Verilog HDL Code which are easy understandable and
%%                  adjustable. For further information see
%%                          www.chipforge.org
%%                  there are projects from small cores up to PCBs, too.
%%
%%  File:           StdCellLib/Documents/Circuits/LATERN.tex
%%
%%  Purpose:        Circuit File for LATERN
%%
%%  ************    LaTeX with circdia.sty package      ***************
%%
%%  ///////////////////////////////////////////////////////////////////
%%
%%  Copyright (c) 2019 by chipforge <stdcelllib@nospam.chipforge.org>
%%  All rights reserved.
%%
%%      This Standard Cell Library is licensed under the Libre Silicon
%%      public license; you can redistribute it and/or modify it under
%%      the terms of the Libre Silicon public license as published by
%%      the Libre Silicon alliance, either version 1 of the License, or
%%      (at your option) any later version.
%%
%%      This design is distributed in the hope that it will be useful,
%%      but WITHOUT ANY WARRANTY; without even the implied warranty of
%%      MERCHANTABILITY or FITNESS FOR A PARTICULAR PURPOSE.
%%      See the Libre Silicon Public License for more details.
%%
%%  ///////////////////////////////////////////////////////////////////
\begin{center}
    Circuit
    \begin{figure}[h]
        \begin{center}
            \begin{circuitdiagram}{25}{17}
            \pin{1}{1}{L}{XN}   % pin XN
            \pin{1}{5}{L}{EN}   % pin EN
            \usgate
            \gate{nor}{5}{3}{R}{}{}
            \wire{9}{3}{11}{3}
            \junct{10}{3}
            \wire{10}{3}{10}{7}
            \gate{not}{14}{3}{R}{}{}
            \wire{17}{3}{18}{3}
            \wire{18}{3}{18}{7}
            \pin{1}{9}{L}{D}    % pin D
            \wire{2}{9}{7}{9}
            \gate[\tristate{Dn}]{not}{10}{9}{R}{}{}
            \wire{13}{9}{15}{9}
            \junct{14}{9}
            \wire{14}{9}{14}{12}
            \gate[\tristate{Dn}]{not}{18}{9}{L}{}{}
            \pin{13}{16}{L}{R}   % pin R
            \gate{nor}{17}{14}{R}{}{}
            \wire{21}{9}{23}{9}
            \junct{22}{9}
            \wire{22}{9}{22}{14}
            \wire{21}{14}{22}{14}
            \pin{24}{9}{R}{Q}   % pin Q
            \end{circuitdiagram}
        \end{center}
    \end{figure}
\end{center}
 %%  ************    LibreSilicon's StdCellLibrary   *******************
%%
%%  Organisation:   Chipforge
%%                  Germany / European Union
%%
%%  Profile:        Chipforge focus on fine System-on-Chip Cores in
%%                  Verilog HDL Code which are easy understandable and
%%                  adjustable. For further information see
%%                          www.chipforge.org
%%                  there are projects from small cores up to PCBs, too.
%%
%%  File:           StdCellLib/Documents/Circuits/LATERP.tex
%%
%%  Purpose:        Circuit File for LATERP
%%
%%  ************    LaTeX with circdia.sty package      ***************
%%
%%  ///////////////////////////////////////////////////////////////////
%%
%%  Copyright (c) 2019 by chipforge <stdcelllib@nospam.chipforge.org>
%%  All rights reserved.
%%
%%      This Standard Cell Library is licensed under the Libre Silicon
%%      public license; you can redistribute it and/or modify it under
%%      the terms of the Libre Silicon public license as published by
%%      the Libre Silicon alliance, either version 1 of the License, or
%%      (at your option) any later version.
%%
%%      This design is distributed in the hope that it will be useful,
%%      but WITHOUT ANY WARRANTY; without even the implied warranty of
%%      MERCHANTABILITY or FITNESS FOR A PARTICULAR PURPOSE.
%%      See the Libre Silicon Public License for more details.
%%
%%  ///////////////////////////////////////////////////////////////////
\begin{center}
    Circuit
    \begin{figure}[h]
        \begin{center}
            \begin{circuitdiagram}{25}{17}
            \pin{1}{1}{L}{X}    % pin X
            \pin{1}{5}{L}{E}    % pin E
            \usgate
            \gate{nand}{5}{3}{R}{}{}
            \wire{9}{3}{11}{3}
            \junct{10}{3}
            \wire{10}{3}{10}{7}
            \gate{not}{14}{3}{R}{}{}
            \wire{17}{3}{18}{3}
            \wire{18}{3}{18}{7}
            \pin{1}{9}{L}{D}    % pin D
            \wire{2}{9}{7}{9}
            \gate[\tristate{Dp}]{not}{10}{9}{R}{}{}
            \wire{13}{9}{15}{9}
            \junct{14}{9}
            \wire{14}{9}{14}{12}
            \gate[\tristate{Dp}]{not}{18}{9}{L}{}{}
            \pin{13}{16}{L}{R}   % pin R
            \gate{nor}{17}{14}{R}{}{}
            \wire{21}{9}{23}{9}
            \junct{22}{9}
            \wire{22}{9}{22}{14}
            \wire{21}{14}{22}{14}
            \pin{24}{9}{R}{Q}   % pin Q
            \end{circuitdiagram}
        \end{center}
    \end{figure}
\end{center}

%%  ************    LibreSilicon's StdCellLibrary   *******************
%%
%%  Organisation:   Chipforge
%%                  Germany / European Union
%%
%%  Profile:        Chipforge focus on fine System-on-Chip Cores in
%%                  Verilog HDL Code which are easy understandable and
%%                  adjustable. For further information see
%%                          www.chipforge.org
%%                  there are projects from small cores up to PCBs, too.
%%
%%  File:           StdCellLib/Documents/Circuits/LATESN.tex
%%
%%  Purpose:        Circuit File for LATESN
%%
%%  ************    LaTeX with circdia.sty package      ***************
%%
%%  ///////////////////////////////////////////////////////////////////
%%
%%  Copyright (c) 2019 by chipforge <stdcelllib@nospam.chipforge.org>
%%  All rights reserved.
%%
%%      This Standard Cell Library is licensed under the Libre Silicon
%%      public license; you can redistribute it and/or modify it under
%%      the terms of the Libre Silicon public license as published by
%%      the Libre Silicon alliance, either version 1 of the License, or
%%      (at your option) any later version.
%%
%%      This design is distributed in the hope that it will be useful,
%%      but WITHOUT ANY WARRANTY; without even the implied warranty of
%%      MERCHANTABILITY or FITNESS FOR A PARTICULAR PURPOSE.
%%      See the Libre Silicon Public License for more details.
%%
%%  ///////////////////////////////////////////////////////////////////
\begin{center}
    Circuit
    \begin{figure}[h]
        \begin{center}
            \begin{circuitdiagram}{25}{17}
            \pin{1}{1}{L}{XN}   % pin XN
            \pin{1}{5}{L}{EN}   % pin EN
            \usgate
            \gate{nor}{5}{3}{R}{}{}
            \wire{9}{3}{11}{3}
            \junct{10}{3}
            \wire{10}{3}{10}{7}
            \gate{not}{14}{3}{R}{}{}
            \wire{17}{3}{18}{3}
            \wire{18}{3}{18}{7}
            \pin{1}{9}{L}{D}    % pin D
            \wire{2}{9}{7}{9}
            \gate[\tristate{Dn}]{not}{10}{9}{R}{}{}
            \wire{13}{9}{15}{9}
            \junct{14}{9}
            \wire{14}{9}{14}{12}
            \gate[\tristate{Dn}]{not}{18}{9}{L}{}{}
            \pin{13}{16}{L}{SN}  % pin SN
            \gate{nand}{17}{14}{R}{}{}
            \wire{21}{9}{23}{9}
            \junct{22}{9}
            \wire{22}{9}{22}{14}
            \wire{21}{14}{22}{14}
            \pin{24}{9}{R}{Q}   % pin Q
            \end{circuitdiagram}
        \end{center}
    \end{figure}
\end{center}
 %%  ************    LibreSilicon's StdCellLibrary   *******************
%%
%%  Organisation:   Chipforge
%%                  Germany / European Union
%%
%%  Profile:        Chipforge focus on fine System-on-Chip Cores in
%%                  Verilog HDL Code which are easy understandable and
%%                  adjustable. For further information see
%%                          www.chipforge.org
%%                  there are projects from small cores up to PCBs, too.
%%
%%  File:           StdCellLib/Documents/Circuits/LATESP.tex
%%
%%  Purpose:        Circuit File for LATESP
%%
%%  ************    LaTeX with circdia.sty package      ***************
%%
%%  ///////////////////////////////////////////////////////////////////
%%
%%  Copyright (c) 2019 by chipforge <stdcelllib@nospam.chipforge.org>
%%  All rights reserved.
%%
%%      This Standard Cell Library is licensed under the Libre Silicon
%%      public license; you can redistribute it and/or modify it under
%%      the terms of the Libre Silicon public license as published by
%%      the Libre Silicon alliance, either version 1 of the License, or
%%      (at your option) any later version.
%%
%%      This design is distributed in the hope that it will be useful,
%%      but WITHOUT ANY WARRANTY; without even the implied warranty of
%%      MERCHANTABILITY or FITNESS FOR A PARTICULAR PURPOSE.
%%      See the Libre Silicon Public License for more details.
%%
%%  ///////////////////////////////////////////////////////////////////
\begin{center}
    Circuit
    \begin{figure}[h]
        \begin{center}
            \begin{circuitdiagram}{25}{17}
            \pin{1}{1}{L}{X}    % pin X
            \pin{1}{5}{L}{E}    % pin E
            \usgate
            \gate{nand}{5}{3}{R}{}{}
            \wire{9}{3}{11}{3}
            \junct{10}{3}
            \wire{10}{3}{10}{7}
            \gate{not}{14}{3}{R}{}{}
            \wire{17}{3}{18}{3}
            \wire{18}{3}{18}{7}
            \pin{1}{9}{L}{D}    % pin D
            \wire{2}{9}{7}{9}
            \gate[\tristate{Dp}]{not}{10}{9}{R}{}{}
            \wire{13}{9}{15}{9}
            \junct{14}{9}
            \wire{14}{9}{14}{12}
            \gate[\tristate{Dp}]{not}{18}{9}{L}{}{}
            \pin{13}{16}{L}{SN}  % pin SN
            \gate{nand}{17}{14}{R}{}{}
            \wire{21}{9}{23}{9}
            \junct{22}{9}
            \wire{22}{9}{22}{14}
            \wire{21}{14}{22}{14}
            \pin{24}{9}{R}{Q}   % pin Q
            \end{circuitdiagram}
        \end{center}
    \end{figure}
\end{center}

\include{Datasheets/LATEEN} \include{Datasheets/LATEEP}
\include{Datasheets/LATEERN} \include{Datasheets/LATEERP}
\include{Datasheets/LATEESN} \include{Datasheets/LATEESP}

%%  ************    LibreSilicon's StdCellLibrary   *******************
%%
%%  Organisation:   Chipforge
%%                  Germany / European Union
%%
%%  Profile:        Chipforge focus on fine System-on-Chip Cores in
%%                  Verilog HDL Code which are easy understandable and
%%                  adjustable. For further information see
%%                          www.chipforge.org
%%                  there are projects from small cores up to PCBs, too.
%%
%%  File:           StdCellLib/Documents/section-DFF_flipflops.tex
%%
%%  Purpose:        Section Level File for Standard Cell Library Documentation
%%
%%  ************    LaTeX with circdia.sty package      ***************
%%
%%  ///////////////////////////////////////////////////////////////////
%%
%%  Copyright (c) 2018 - 2022 by
%%                  chipforge <stdcelllib@nospam.chipforge.org>
%%  All rights reserved.
%%
%%      This Standard Cell Library is licensed under the Libre Silicon
%%      public license; you can redistribute it and/or modify it under
%%      the terms of the Libre Silicon public license as published by
%%      the Libre Silicon alliance, either version 1 of the License, or
%%      (at your option) any later version.
%%
%%      This design is distributed in the hope that it will be useful,
%%      but WITHOUT ANY WARRANTY; without even the implied warranty of
%%      MERCHANTABILITY or FITNESS FOR A PARTICULAR PURPOSE.
%%      See the Libre Silicon Public License for more details.
%%
%%  ///////////////////////////////////////////////////////////////////
\section{D-type Flip-Flops}

%%  ************    LibreSilicon's StdCellLibrary   *******************
%%
%%  Organisation:   Chipforge
%%                  Germany / European Union
%%
%%  Profile:        Chipforge focus on fine System-on-Chip Cores in
%%                  Verilog HDL Code which are easy understandable and
%%                  adjustable. For further information see
%%                          www.chipforge.org
%%                  there are projects from small cores up to PCBs, too.
%%
%%  File:           StdCellLib/Documents/Circuits/DFFN.tex
%%
%%  Purpose:        Circuit File for DFFN
%%
%%  ************    LaTeX with circdia.sty package      ***************
%%
%%  ///////////////////////////////////////////////////////////////////
%%
%%  Copyright (c) 2019 by chipforge <stdcelllib@nospam.chipforge.org>
%%  All rights reserved.
%%
%%      This Standard Cell Library is licensed under the Libre Silicon
%%      public license; you can redistribute it and/or modify it under
%%      the terms of the Libre Silicon public license as published by
%%      the Libre Silicon alliance, either version 1 of the License, or
%%      (at your option) any later version.
%%
%%      This design is distributed in the hope that it will be useful,
%%      but WITHOUT ANY WARRANTY; without even the implied warranty of
%%      MERCHANTABILITY or FITNESS FOR A PARTICULAR PURPOSE.
%%      See the Libre Silicon Public License for more details.
%%
%%  ///////////////////////////////////////////////////////////////////
\begin{center}
    Circuit
    \begin{figure}[h]
        \begin{center}
            \begin{circuitdiagram}{24}{10}
            \pin{1}{1}{L}{XN}  % pin XN
            \wire{2}{1}{13}{1}
            \junct{3}{1}
            \wire{3}{1}{3}{6}
            \wire{3}{6}{4}{6}
            \wire{13}{1}{13}{6}
            \wire{13}{6}{14}{6}
            \pin{1}{8}{L}{D}   % pin D
            \wire{2}{8}{4}{8}
            \usgate
            \flipflop[\clockin{p}]{d}{8}{6}{R}{}{}
            \wire{12}{8}{14}{8}
            \flipflop[\clockin{n}]{d}{18}{6}{R}{}{}
            \pin{23}{8}{R}{Q}  % pin Q
            \end{circuitdiagram}
        \end{center}
    \end{figure}
\end{center}
 %%  ************    LibreSilicon's StdCellLibrary   *******************
%%
%%  Organisation:   Chipforge
%%                  Germany / European Union
%%
%%  Profile:        Chipforge focus on fine System-on-Chip Cores in
%%                  Verilog HDL Code which are easy understandable and
%%                  adjustable. For further information see
%%                          www.chipforge.org
%%                  there are projects from small cores up to PCBs, too.
%%
%%  File:           StdCellLib/Documents/Circuits/DFFP.tex
%%
%%  Purpose:        Circuit File for DFFP
%%
%%  ************    LaTeX with circdia.sty package      ***************
%%
%%  ///////////////////////////////////////////////////////////////////
%%
%%  Copyright (c) 2019 by chipforge <stdcelllib@nospam.chipforge.org>
%%  All rights reserved.
%%
%%      This Standard Cell Library is licensed under the Libre Silicon
%%      public license; you can redistribute it and/or modify it under
%%      the terms of the Libre Silicon public license as published by
%%      the Libre Silicon alliance, either version 1 of the License, or
%%      (at your option) any later version.
%%
%%      This design is distributed in the hope that it will be useful,
%%      but WITHOUT ANY WARRANTY; without even the implied warranty of
%%      MERCHANTABILITY or FITNESS FOR A PARTICULAR PURPOSE.
%%      See the Libre Silicon Public License for more details.
%%
%%  ///////////////////////////////////////////////////////////////////
\begin{center}
    Circuit
    \begin{figure}[h]
        \begin{center}
            \begin{circuitdiagram}{24}{10}
            \pin{1}{1}{L}{X}   % pin X
            \wire{2}{1}{13}{1}
            \junct{3}{1}
            \wire{3}{1}{3}{6}
            \wire{3}{6}{4}{6}
            \wire{13}{1}{13}{6}
            \wire{13}{6}{14}{6}
            \pin{1}{8}{L}{D}   % pin D
            \wire{2}{8}{4}{8}
            \usgate
            \flipflop[\clockin{n}]{d}{8}{6}{R}{}{}
            \wire{12}{8}{14}{8}
            \flipflop[\clockin{p}]{d}{18}{6}{R}{}{}
            \pin{23}{8}{R}{Q}  % pin Q
            \end{circuitdiagram}
        \end{center}
    \end{figure}
\end{center}

\include{Datasheets/DFFRN} \include{Datasheets/DFFRP}
\include{Datasheets/DFFSN} \include{Datasheets/DFFSP}
%%  ************    LibreSilicon's StdCellLibrary   *******************
%%
%%  Organisation:   Chipforge
%%                  Germany / European Union
%%
%%  Profile:        Chipforge focus on fine System-on-Chip Cores in
%%                  Verilog HDL Code which are easy understandable and
%%                  adjustable. For further information see
%%                          www.chipforge.org
%%                  there are projects from small cores up to PCBs, too.
%%
%%  File:           StdCellLib/Documents/Circuits/DFFEN.tex
%%
%%  Purpose:        Circuit File for DFFEN
%%
%%  ************    LaTeX with circdia.sty package      ***************
%%
%%  ///////////////////////////////////////////////////////////////////
%%
%%  Copyright (c) 2019 by chipforge <stdcelllib@nospam.chipforge.org>
%%  All rights reserved.
%%
%%      This Standard Cell Library is licensed under the Libre Silicon
%%      public license; you can redistribute it and/or modify it under
%%      the terms of the Libre Silicon public license as published by
%%      the Libre Silicon alliance, either version 1 of the License, or
%%      (at your option) any later version.
%%
%%      This design is distributed in the hope that it will be useful,
%%      but WITHOUT ANY WARRANTY; without even the implied warranty of
%%      MERCHANTABILITY or FITNESS FOR A PARTICULAR PURPOSE.
%%      See the Libre Silicon Public License for more details.
%%
%%  ///////////////////////////////////////////////////////////////////
\begin{circuitdiagram}{31}{12}

    \usgate
    \pin{1}{10}{L}{D}   % pin D
    \wire{2}{10}{11}{10}
    \gate{nor}{5}{3}{R}{}{}
    \flipflop[\clockin{n}]{d}{15}{8}{R}{}{}
    \wire{20}{3}{20}{8}
    \wire{20}{8}{21}{8}
    \flipflop[\clockin{p}]{d}{25}{8}{R}{}{}
    \wire{19}{10}{21}{10}
    \pin{1}{5}{L}{EN}  % pin EN
    \pin{1}{1}{L}{XN}  % pin XN
    \wire{9}{3}{20}{3}
    \junct{10}{3}
    \wire{10}{3}{10}{8}
    \wire{10}{8}{11}{8}
    \pin{30}{10}{R}{Q}  % pin Q

\end{circuitdiagram}
 %%  ************    LibreSilicon's StdCellLibrary   *******************
%%
%%  Organisation:   Chipforge
%%                  Germany / European Union
%%
%%  Profile:        Chipforge focus on fine System-on-Chip Cores in
%%                  Verilog HDL Code which are easy understandable and
%%                  adjustable. For further information see
%%                          www.chipforge.org
%%                  there are projects from small cores up to PCBs, too.
%%
%%  File:           StdCellLib/Documents/Circuits/DFFEP.tex
%%
%%  Purpose:        Circuit File for DFFEP
%%
%%  ************    LaTeX with circdia.sty package      ***************
%%
%%  ///////////////////////////////////////////////////////////////////
%%
%%  Copyright (c) 2019 by chipforge <stdcelllib@nospam.chipforge.org>
%%  All rights reserved.
%%
%%      This Standard Cell Library is licensed under the Libre Silicon
%%      public license; you can redistribute it and/or modify it under
%%      the terms of the Libre Silicon public license as published by
%%      the Libre Silicon alliance, either version 1 of the License, or
%%      (at your option) any later version.
%%
%%      This design is distributed in the hope that it will be useful,
%%      but WITHOUT ANY WARRANTY; without even the implied warranty of
%%      MERCHANTABILITY or FITNESS FOR A PARTICULAR PURPOSE.
%%      See the Libre Silicon Public License for more details.
%%
%%  ///////////////////////////////////////////////////////////////////
\begin{center}
    Circuit
    \begin{figure}[h]
        \begin{center}
            \begin{circuitdiagram}{31}{12}
            \pin{1}{10}{L}{D}   % pin D
            \wire{2}{10}{11}{10}
            \usgate
            \gate{nand}{5}{3}{R}{}{}
            \flipflop[\clockin{p}]{d}{15}{8}{R}{}{}
            \wire{20}{3}{20}{8}
            \wire{20}{8}{21}{8}
            \flipflop[\clockin{n}]{d}{25}{8}{R}{}{}
            \wire{19}{10}{21}{10}
            \pin{1}{5}{L}{E}   % pin E
            \pin{1}{1}{L}{X}   % pin X
            \wire{9}{3}{20}{3}
            \junct{10}{3}
            \wire{10}{3}{10}{8}
            \wire{10}{8}{11}{8}
            \pin{30}{10}{R}{Q}  % pin Q
            \end{circuitdiagram}
        \end{center}
    \end{figure}
\end{center}

%%  ************    LibreSilicon's StdCellLibrary   *******************
%%
%%  Organisation:   Chipforge
%%                  Germany / European Union
%%
%%  Profile:        Chipforge focus on fine System-on-Chip Cores in
%%                  Verilog HDL Code which are easy understandable and
%%                  adjustable. For further information see
%%                          www.chipforge.org
%%                  there are projects from small cores up to PCBs, too.
%%
%%  File:           StdCellLib/Documents/Circuits/DFFERN.tex
%%
%%  Purpose:        Circuit File for DFFERN
%%
%%  ************    LaTeX with circdia.sty package      ***************
%%
%%  ///////////////////////////////////////////////////////////////////
%%
%%  Copyright (c) 2019 by chipforge <stdcelllib@nospam.chipforge.org>
%%  All rights reserved.
%%
%%      This Standard Cell Library is licensed under the Libre Silicon
%%      public license; you can redistribute it and/or modify it under
%%      the terms of the Libre Silicon public license as published by
%%      the Libre Silicon alliance, either version 1 of the License, or
%%      (at your option) any later version.
%%
%%      This design is distributed in the hope that it will be useful,
%%      but WITHOUT ANY WARRANTY; without even the implied warranty of
%%      MERCHANTABILITY or FITNESS FOR A PARTICULAR PURPOSE.
%%      See the Libre Silicon Public License for more details.
%%
%%  ///////////////////////////////////////////////////////////////////
\begin{center}
    Circuit
    \begin{figure}[h]
        \begin{center}
            \begin{circuitdiagram}{31}{12}
            \pin{1}{10}{L}{D}   % pin D
            \wire{2}{10}{11}{10}
            \usgate
            \gate{nor}{5}{3}{R}{}{}
            \flipflop[\clockin{n}]{d}{15}{8}{R}{}{}
            \wire{20}{3}{20}{8}
            \wire{20}{8}{21}{8}
            \flipflop[\clockin{p}\resetin{p}]{d}{25}{8}{R}{}{}
            \wire{19}{10}{21}{10}
            \pin{25}{3}{D}{R}  % pin R
            \pin{1}{5}{L}{EN}  % pin EN
            \pin{1}{1}{L}{XN}  % pin XN
            \wire{9}{3}{20}{3}
            \junct{10}{3}
            \wire{10}{3}{10}{8}
            \wire{10}{8}{11}{8}
            \pin{30}{10}{R}{Q}  % pin Q
            \end{circuitdiagram}
        \end{center}
    \end{figure}
\end{center}
 %%  ************    LibreSilicon's StdCellLibrary   *******************
%%
%%  Organisation:   Chipforge
%%                  Germany / European Union
%%
%%  Profile:        Chipforge focus on fine System-on-Chip Cores in
%%                  Verilog HDL Code which are easy understandable and
%%                  adjustable. For further information see
%%                          www.chipforge.org
%%                  there are projects from small cores up to PCBs, too.
%%
%%  File:           StdCellLib/Documents/Circuits/DFFERP.tex
%%
%%  Purpose:        Circuit File for DFFERP
%%
%%  ************    LaTeX with circdia.sty package      ***************
%%
%%  ///////////////////////////////////////////////////////////////////
%%
%%  Copyright (c) 2019 by chipforge <stdcelllib@nospam.chipforge.org>
%%  All rights reserved.
%%
%%      This Standard Cell Library is licensed under the Libre Silicon
%%      public license; you can redistribute it and/or modify it under
%%      the terms of the Libre Silicon public license as published by
%%      the Libre Silicon alliance, either version 1 of the License, or
%%      (at your option) any later version.
%%
%%      This design is distributed in the hope that it will be useful,
%%      but WITHOUT ANY WARRANTY; without even the implied warranty of
%%      MERCHANTABILITY or FITNESS FOR A PARTICULAR PURPOSE.
%%      See the Libre Silicon Public License for more details.
%%
%%  ///////////////////////////////////////////////////////////////////
\begin{center}
    Circuit
    \begin{figure}[h]
        \begin{center}
            \begin{circuitdiagram}{31}{12}
            \pin{1}{10}{L}{D}   % pin D
            \wire{2}{10}{11}{10}
            \usgate
            \gate{nand}{5}{3}{R}{}{}
            \flipflop[\clockin{p}]{d}{15}{8}{R}{}{}
            \wire{20}{3}{20}{8}
            \wire{20}{8}{21}{8}
            \flipflop[\clockin{n}\resetin{p}]{d}{25}{8}{R}{}{}
            \wire{19}{10}{21}{10}
            \pin{25}{3}{D}{R}  % pin R
            \pin{1}{5}{L}{E}   % pin E
            \pin{1}{1}{L}{X}   % pin X
            \wire{9}{3}{20}{3}
            \junct{10}{3}
            \wire{10}{3}{10}{8}
            \wire{10}{8}{11}{8}
            \pin{30}{10}{R}{Q}  % pin Q
            \end{circuitdiagram}
        \end{center}
    \end{figure}
\end{center}

%%  ************    LibreSilicon's StdCellLibrary   *******************
%%
%%  Organisation:   Chipforge
%%                  Germany / European Union
%%
%%  Profile:        Chipforge focus on fine System-on-Chip Cores in
%%                  Verilog HDL Code which are easy understandable and
%%                  adjustable. For further information see
%%                          www.chipforge.org
%%                  there are projects from small cores up to PCBs, too.
%%
%%  File:           StdCellLib/Documents/Circuits/DFFESN.tex
%%
%%  Purpose:        Circuit File for DFFESN
%%
%%  ************    LaTeX with circdia.sty package      ***************
%%
%%  ///////////////////////////////////////////////////////////////////
%%
%%  Copyright (c) 2019 by chipforge <stdcelllib@nospam.chipforge.org>
%%  All rights reserved.
%%
%%      This Standard Cell Library is licensed under the Libre Silicon
%%      public license; you can redistribute it and/or modify it under
%%      the terms of the Libre Silicon public license as published by
%%      the Libre Silicon alliance, either version 1 of the License, or
%%      (at your option) any later version.
%%
%%      This design is distributed in the hope that it will be useful,
%%      but WITHOUT ANY WARRANTY; without even the implied warranty of
%%      MERCHANTABILITY or FITNESS FOR A PARTICULAR PURPOSE.
%%      See the Libre Silicon Public License for more details.
%%
%%  ///////////////////////////////////////////////////////////////////
\begin{center}
    Circuit
    \begin{figure}[h]
        \begin{center}
            \begin{circuitdiagram}{31}{14}
            \pin{1}{10}{L}{D}   % pin D
            \wire{2}{10}{11}{10}
            \usgate
            \gate{nor}{5}{3}{R}{}{}
            \flipflop[\clockin{n}]{d}{15}{8}{R}{}{}
            \wire{20}{3}{20}{8}
            \wire{20}{8}{21}{8}
            \flipflop[\clockin{p}\setin{n}]{d}{25}{8}{R}{}{}
            \wire{19}{10}{21}{10}
            \pin{25}{13}{U}{SN}% pin SN
            \pin{1}{5}{L}{EN}  % pin EN
            \pin{1}{1}{L}{XN}  % pin XN
            \wire{9}{3}{20}{3}
            \junct{10}{3}
            \wire{10}{3}{10}{8}
            \wire{10}{8}{11}{8}
            \pin{30}{10}{R}{Q}  % pin Q
            \end{circuitdiagram}
        \end{center}
    \end{figure}
\end{center}
 %%  ************    LibreSilicon's StdCellLibrary   *******************
%%
%%  Organisation:   Chipforge
%%                  Germany / European Union
%%
%%  Profile:        Chipforge focus on fine System-on-Chip Cores in
%%                  Verilog HDL Code which are easy understandable and
%%                  adjustable. For further information see
%%                          www.chipforge.org
%%                  there are projects from small cores up to PCBs, too.
%%
%%  File:           StdCellLib/Documents/Circuits/DFFESP.tex
%%
%%  Purpose:        Circuit File for DFFESP
%%
%%  ************    LaTeX with circdia.sty package      ***************
%%
%%  ///////////////////////////////////////////////////////////////////
%%
%%  Copyright (c) 2019 by chipforge <stdcelllib@nospam.chipforge.org>
%%  All rights reserved.
%%
%%      This Standard Cell Library is licensed under the Libre Silicon
%%      public license; you can redistribute it and/or modify it under
%%      the terms of the Libre Silicon public license as published by
%%      the Libre Silicon alliance, either version 1 of the License, or
%%      (at your option) any later version.
%%
%%      This design is distributed in the hope that it will be useful,
%%      but WITHOUT ANY WARRANTY; without even the implied warranty of
%%      MERCHANTABILITY or FITNESS FOR A PARTICULAR PURPOSE.
%%      See the Libre Silicon Public License for more details.
%%
%%  ///////////////////////////////////////////////////////////////////
\begin{center}
    Circuit
    \begin{figure}[h]
        \begin{center}
            \begin{circuitdiagram}{31}{14}
            \pin{1}{10}{L}{D}   % pin D
            \wire{2}{10}{11}{10}
            \usgate
            \gate{nand}{5}{3}{R}{}{}
            \flipflop[\clockin{p}]{d}{15}{8}{R}{}{}
            \wire{20}{3}{20}{8}
            \wire{20}{8}{21}{8}
            \flipflop[\clockin{n}\setin{n}]{d}{25}{8}{R}{}{}
            \wire{19}{10}{21}{10}
            \pin{25}{13}{U}{SN}% pin SN
            \pin{1}{5}{L}{E}   % pin E
            \pin{1}{1}{L}{X}   % pin X
            \wire{9}{3}{20}{3}
            \junct{10}{3}
            \wire{10}{3}{10}{8}
            \wire{10}{8}{11}{8}
            \pin{30}{10}{R}{Q}  % pin Q
            \end{circuitdiagram}
        \end{center}
    \end{figure}
\end{center}




%%  -------------------------------------------------------------------
%%                  CHAPTER 5
%%  -------------------------------------------------------------------

\chapter{Clock Distribution Cells}
\clearpage

\section{Clock Inverter}

\include{CIN2_datasheet}
\include{CIP2_datasheet}

\section{Clock Buffers}

\include{CBN2_datasheet}
\include{CBP2_datasheet}


%%  -------------------------------------------------------------------
%%                  CHAPTER 6
%%  -------------------------------------------------------------------

\chapter{Physical Cells}
\clearpage

%%  ------------    logical also    -----------------------------------

%%  ************    LibreSilicon's StdCellLibrary   *******************
%%
%%  Organisation:   Chipforge
%%                  Germany / European Union
%%
%%  Profile:        Chipforge focus on fine System-on-Chip Cores in
%%                  Verilog HDL Code which are easy understandable and
%%                  adjustable. For further information see
%%                          www.chipforge.org
%%                  there are projects from small cores up to PCBs, too.
%%
%%  File:           StdCellLib/Documents/Book/section-TIE_cells.tex
%%
%%  Purpose:        Section Level File for Standard Cell Library Documentation
%%
%%  ************    LaTeX with circdia.sty package      ***************
%%
%%  ///////////////////////////////////////////////////////////////////
%%
%%  Copyright (c) 2018 - 2022 by
%%                  chipforge <stdcelllib@nospam.chipforge.org>
%%  All rights reserved.
%%
%%      This Standard Cell Library is licensed under the Libre Silicon
%%      public license; you can redistribute it and/or modify it under
%%      the terms of the Libre Silicon public license as published by
%%      the Libre Silicon alliance, either version 1 of the License, or
%%      (at your option) any later version.
%%
%%      This design is distributed in the hope that it will be useful,
%%      but WITHOUT ANY WARRANTY; without even the implied warranty of
%%      MERCHANTABILITY or FITNESS FOR A PARTICULAR PURPOSE.
%%      See the Libre Silicon Public License for more details.
%%
%%  ///////////////////////////////////////////////////////////////////
\section{Tie Cells}

%%  ************    LibreSilicon's StdCellLibrary   *******************
%%
%%  Organisation:   Chipforge
%%                  Germany / European Union
%%
%%  Profile:        Chipforge focus on fine System-on-Chip Cores in
%%                  Verilog HDL Code which are easy understandable and
%%                  adjustable. For further information see
%%                          www.chipforge.org
%%                  there are projects from small cores up to PCBs, too.
%%
%%  File:           StdCellLib/Documents/Circuits/TIE0.tex
%%
%%  Purpose:        Circuit File for TIE0
%%
%%  ************    LaTeX with circdia.sty package      ***************
%%
%%  ///////////////////////////////////////////////////////////////////
%%
%%  Copyright (c) 2019 by chipforge <stdcelllib@nospam.chipforge.org>
%%  All rights reserved.
%%
%%      This Standard Cell Library is licensed under the Libre Silicon
%%      public license; you can redistribute it and/or modify it under
%%      the terms of the Libre Silicon public license as published by
%%      the Libre Silicon alliance, either version 1 of the License, or
%%      (at your option) any later version.
%%
%%      This design is distributed in the hope that it will be useful,
%%      but WITHOUT ANY WARRANTY; without even the implied warranty of
%%      MERCHANTABILITY or FITNESS FOR A PARTICULAR PURPOSE.
%%      See the Libre Silicon Public License for more details.
%%
%%  ///////////////////////////////////////////////////////////////////
\begin{center}
    Circuit
    \begin{figure}[h]
        \begin{center}
            \begin{circuitdiagram}{8}{8}
            \resis{2}{4}{V}{R}{}   % pull down R
            \ground{2}{0.5}{D}
            \wire{2}{7}{6}{7}   % pin Y
            \pin{7}{7}{R}{Y}   % pin Y
            \end{circuitdiagram}
        \end{center}
    \end{figure}
\end{center}
 %%  ************    LibreSilicon's StdCellLibrary   *******************
%%
%%  Organisation:   Chipforge
%%                  Germany / European Union
%%
%%  Profile:        Chipforge focus on fine System-on-Chip Cores in
%%                  Verilog HDL Code which are easy understandable and
%%                  adjustable. For further information see
%%                          www.chipforge.org
%%                  there are projects from small cores up to PCBs, too.
%%
%%  File:           StdCellLib/Documents/Circuits/TIE1.tex
%%
%%  Purpose:        Circuit File for TIE1
%%
%%  ************    LaTeX with circdia.sty package      ***************
%%
%%  ///////////////////////////////////////////////////////////////////
%%
%%  Copyright (c) 2019 by chipforge <stdcelllib@nospam.chipforge.org>
%%  All rights reserved.
%%
%%      This Standard Cell Library is licensed under the Libre Silicon
%%      public license; you can redistribute it and/or modify it under
%%      the terms of the Libre Silicon public license as published by
%%      the Libre Silicon alliance, either version 1 of the License, or
%%      (at your option) any later version.
%%
%%      This design is distributed in the hope that it will be useful,
%%      but WITHOUT ANY WARRANTY; without even the implied warranty of
%%      MERCHANTABILITY or FITNESS FOR A PARTICULAR PURPOSE.
%%      See the Libre Silicon Public License for more details.
%%
%%  ///////////////////////////////////////////////////////////////////
\begin{center}
    Circuit
    \begin{figure}[h]
        \begin{center}
            \begin{circuitdiagram}{8}{8}
            \resis{2}{4}{V}{R}{}   % pull up R
            \power{2}{7.5}{U}{}
            \wire{2}{1}{6}{1}   % pin Y
            \pin{7}{1}{R}{Y}   % pin Y
            \end{circuitdiagram}
        \end{center}
    \end{figure}
\end{center}



%%  ------------    core layout     -----------------------------------

\section{Filler Cells}

\include{FILL1_datasheet}


%%  ------------    pad ring    ---------------------------------------

%%  ************    LibreSilicon's StdCellLibrary   *******************
%%
%%  Organisation:   Chipforge
%%                  Germany / European Union
%%
%%  Profile:        Chipforge focus on fine System-on-Chip Cores in
%%                  Verilog HDL Code which are easy understandable and
%%                  adjustable. For further information see
%%                          www.chipforge.org
%%                  there are projects from small cores up to PCBs, too.
%%
%%  File:           StdCellLib/Documents/section-PAD_ring.tex
%%
%%  Purpose:        Section Level File for Standard Cell Library Documentation
%%
%%  ************    LaTeX with circdia.sty package      ***************
%%
%%  ///////////////////////////////////////////////////////////////////
%%
%%  Copyright (c) 2018 - 2022 by
%%                  chipforge <stdcelllib@nospam.chipforge.org>
%%  All rights reserved.
%%
%%      This Standard Cell Library is licensed under the Libre Silicon
%%      public license; you can redistribute it and/or modify it under
%%      the terms of the Libre Silicon public license as published by
%%      the Libre Silicon alliance, either version 1 of the License, or
%%      (at your option) any later version.
%%
%%      This design is distributed in the hope that it will be useful,
%%      but WITHOUT ANY WARRANTY; without even the implied warranty of
%%      MERCHANTABILITY or FITNESS FOR A PARTICULAR PURPOSE.
%%      See the Libre Silicon Public License for more details.
%%
%%  ///////////////////////////////////////////////////////////////////
\section{Pad Ring Cells}

%VDDIO \\
%GND \\
%ANA



%%  -------------------------------------------------------------------
%%                  PART III
%%  -------------------------------------------------------------------

%%  ************    LibreSilicon's StdCellLibrary   *******************
%%
%%  Organisation:   Chipforge
%%                  Germany / European Union
%%
%%  Profile:        Chipforge focus on fine System-on-Chip Cores in
%%                  Verilog HDL Code which are easy understandable and
%%                  adjustable. For further information see
%%                          www.chipforge.org
%%                  there are projects from small cores up to PCBs, too.
%%
%%  File:           StdCellLib/Documents/LaTeX/part-macros.tex
%%
%%  Purpose:        Part Level File for Macro Examples
%%
%%  ************    LaTeX with circdia.sty package      ***************
%%
%%  ///////////////////////////////////////////////////////////////////
%%
%%  Copyright (c) 2021 by
%%                chipforge <stdcelllib@nospam.chipforge.org>
%%  All rights reserved.
%%
%%      This Standard Cell Library is licensed under the Libre Silicon
%%      public license; you can redistribute it and/or modify it under
%%      the terms of the Libre Silicon public license as published by
%%      the Libre Silicon alliance, either version 1 of the License, or
%%      (at your option) any later version.
%%
%%      This design is distributed in the hope that it will be useful,
%%      but WITHOUT ANY WARRANTY; without even the implied warranty of
%%      MERCHANTABILITY or FITNESS FOR A PARTICULAR PURPOSE.
%%      See the Libre Silicon Public License for more details.
%%
%%  ///////////////////////////////////////////////////////////////////
\part{Macro Examples}
\pagestyle{headings}

%%  -------------------------------------------------------------------
%%                  CHAPTER 1
%%  -------------------------------------------------------------------

\chapter{Combinatorial Macros}
\clearpage

\input{section-XOR_gates}

%%  -------------------------------------------------------------------
%%                  CHAPTER 2
%%  -------------------------------------------------------------------

\chapter{Storage Macros}
\clearpage

\section{Scan-FlipFlop Macros}

%\input{SDFFN_manpage}
%\input{SDFFP_manpage}



%%  -------------------------------------------------------------------
%%                  CHAPTER 3
%%  -------------------------------------------------------------------

\chapter{Clock Distribution Macros}
\clearpage

\section{Clock-Gating Macros}

\include{CGN2_datasheet}
\include{CGP2_datasheet}




%%  -------------------------------------------------------------------
%%                  PART IV
%%  -------------------------------------------------------------------

%%  ************    LibreSilicon's StdCellLibrary   *******************
%%
%%  Organisation:   Chipforge
%%                  Germany / European Union
%%
%%  Profile:        Chipforge focus on fine System-on-Chip Cores in
%%                  Verilog HDL Code which are easy understandable and
%%                  adjustable. For further information see
%%                          www.chipforge.org
%%                  there are projects from small cores up to PCBs, too.
%%
%%  File:           StdCellLib/Documents/Book/part-appendix.tex
%%
%%  Purpose:        Part Level File for Standard Cell Library Documentation
%%
%%  ************    LaTeX with circdia.sty package      ***************
%%
%%  ///////////////////////////////////////////////////////////////////
%%
%%  Copyright (c) 2018 - 2022 by
%%                  chipforge <stdcelllib@nospam.chipforge.org>
%%  All rights reserved.
%%
%%      This Standard Cell Library is licensed under the Libre Silicon
%%      public license; you can redistribute it and/or modify it under
%%      the terms of the Libre Silicon public license as published by
%%      the Libre Silicon alliance, either version 1 of the License, or
%%      (at your option) any later version.
%%
%%      This design is distributed in the hope that it will be useful,
%%      but WITHOUT ANY WARRANTY; without even the implied warranty of
%%      MERCHANTABILITY or FITNESS FOR A PARTICULAR PURPOSE.
%%      See the Libre Silicon Public License for more details.
%%
%%  ///////////////////////////////////////////////////////////////////
\part{Appendix}
\pagestyle{headings}

%%  -------------------------------------------------------------------
%%                  READINGS
%%  -------------------------------------------------------------------

%\chapter{Readings}
\printbibliography[title={Readings}]



\end{document}
