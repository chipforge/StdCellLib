%%  ************    LibreSilicon's StdCellLibrary   *******************
%%
%%  Organisation:   Chipforge
%%                  Germany / European Union
%%
%%  Profile:        Chipforge focus on fine System-on-Chip Cores in
%%                  Verilog HDL Code which are easy understandable and
%%                  adjustable. For further information see
%%                          www.chipforge.org
%%                  there are projects from small cores up to PCBs, too.
%%
%%  File:           StdCellLib/Documents/LaTeX/StdCellLib.tex
%%
%%  Purpose:        Top Level File for Standard Cell Library Documentation
%%
%%  ************    LaTeX with circdia.sty package      ***************
%%
%%  ///////////////////////////////////////////////////////////////////
%%
%%  Copyright (c) 2018 - 2022 by
%%                chipforge <stdcelllib@nospam.chipforge.org>
%%  All rights reserved.
%%
%%      This Standard Cell Library is licensed under the Libre Silicon
%%      public license; you can redistribute it and/or modify it under
%%      the terms of the Libre Silicon public license as published by
%%      the Libre Silicon alliance, either version 1 of the License, or
%%      (at your option) any later version.
%%
%%      This design is distributed in the hope that it will be useful,
%%      but WITHOUT ANY WARRANTY; without even the implied warranty of
%%      MERCHANTABILITY or FITNESS FOR A PARTICULAR PURPOSE.
%%      See the Libre Silicon Public License for more details.
%%
%%  ///////////////////////////////////////////////////////////////////
\documentclass[a4paper, twoside,
               openany,
               headsepline, footsepline]{scrbook}
\usepackage[utf8]{inputenc}
\usepackage[english]{babel}
\usepackage[style=authoryear]{biblatex}
\usepackage{amsmath}
\usepackage{amsfonts}
\usepackage{amssymb}
%\usepackage{gensymb}
\usepackage{booktabs}
\usepackage{caption} \captionsetup{labelformat=empty}
\usepackage{graphicx}
\usepackage[digital,srcmeas,semicon]{circdia}
% \usepackage[dvipsnames]{xcolor}
\usepackage[left=2cm,right=2cm,top=2cm,bottom=2cm]{geometry}
\usepackage{pdflscape}
\usepackage[inkscape=png]{svg}
\usepackage{nameref}
\usepackage[colorlinks=true,linkcolor=gray,citecolor=orange,filecolor=green,urlcolor=cyan]{hyperref}
\pagestyle{myheadings}
\markboth{LibreSilicon's Standard Cell Library - Reference Manual -\hfill}{\hfill - Reference Manual - LibreSilicon's Standard Cell Library}

%%  ************    LibreSilicon's StdCellLibrary   *******************
%%
%%  Organisation:   Chipforge
%%                  Germany / European Union
%%
%%  Profile:        Chipforge focus on fine System-on-Chip Cores in
%%                  Verilog HDL Code which are easy understandable and
%%                  adjustable. For further information see
%%                          www.chipforge.org
%%                  there are projects from small cores up to PCBs, too.
%%
%%  File:           StdCellLib/Documents/glossary.tex
%%
%%  Purpose:        Glossary List
%%
%%  ************    LaTeX with circdia.sty package      ***************
%%
%%  ///////////////////////////////////////////////////////////////////
%%
%%  Copyright (c) 2018 - 2022 by
%%                  chipforge <stdcelllib@nospam.chipforge.org>
%%  All rights reserved.
%%
%%      This Standard Cell Library is licensed under the Libre Silicon
%%      public license; you can redistribute it and/or modify it under
%%      the terms of the Libre Silicon public license as published by
%%      the Libre Silicon alliance, either version 1 of the License, or
%%      (at your option) any later version.
%%
%%      This design is distributed in the hope that it will be useful,
%%      but WITHOUT ANY WARRANTY; without even the implied warranty of
%%      MERCHANTABILITY or FITNESS FOR A PARTICULAR PURPOSE.
%%      See the Libre Silicon Public License for more details.
%%
%%  ///////////////////////////////////////////////////////////////////
\usepackage{glossaries}
\usepackage{glossary-longextra}
\makeglossaries

%%  -------------------------------------------------------------------
%%                                  A
%%  -------------------------------------------------------------------

\newglossaryentry{ASIC}{name={ASIC}, description={\underline{A}pplication-\underline{S}pecific \underline{I}ntegrated \underline{C}ircuit}}
\newglossaryentry{ASSP}{name={ASSP}, description={\underline{A}pplication-\underline{S}pecific \underline{S}tandard \underline{P}roduct}}

%%  -------------------------------------------------------------------
%%                                  C
%%  -------------------------------------------------------------------

\newglossaryentry{CMOS}{name={CMOS}, description={\underline{C}omplementary \underline{M}etal-\underline{O}xide-\underline{S}emiconductor}}

%%  -------------------------------------------------------------------
%%                                  D
%%  -------------------------------------------------------------------

\newglossaryentry{DfT}{name={DfT}, description={\underline{D}esign \underline{f}or \underline{T}estability}}

%%  -------------------------------------------------------------------
%%                                  L
%%  -------------------------------------------------------------------

\newglossaryentry{LSI}{name={LSI}, description={\underline{L}arge-\underline{S}cale-\underline{I}ntegration}}

%%  -------------------------------------------------------------------
%%                                  M
%%  -------------------------------------------------------------------

\newglossaryentry{MOSFET}{name={MOSFET}, description={\underline{M}etal–\underline{O}xide–\underline{S}emiconductor \underline{F}ield-\underline{E}ffect \underline{T}ransistor}}

%%  -------------------------------------------------------------------
%%                                  N
%%  -------------------------------------------------------------------

\newglossaryentry{NMOS}{name={NMOS}, description={\underline{N}-channel \underline{M}etal-\underline{O}xide-\underline{S}emiconductor}}

%%  -------------------------------------------------------------------
%%                                  P
%%  -------------------------------------------------------------------

\newglossaryentry{PMOS}{name={PMOS}, description={\underline{P}-channel \underline{M}etal-\underline{O}xide-\underline{S}emiconductor}}

%%  -------------------------------------------------------------------
%%                                  U
%%  -------------------------------------------------------------------

\newglossaryentry{ULSI}{name={ULSI}, description={\underline{U}ltra-\underline{L}arge-\underline{S}cale-\underline{I}ntegration}}

%%  -------------------------------------------------------------------
%%                                  V
%%  -------------------------------------------------------------------

\newglossaryentry{VLSI}{name={VLSI}, description={\underline{V}ery-\underline{L}arge-\underline{S}cale-\underline{I}ntegration}}




\addbibresource{readings.bib}

\title{LibreSilicon's Standard Cell Library}
\subtitle{- Reference Manual -}
\author{chipforge \texttt{<stdcelllib@nospam.chipforge.org>}}
\date{\today}

\begin{document}

\maketitle
%\setlength{\parindent}{0pt} % get rid of annoying indents

\begin{quote}
Copyright \textcopyright  2018 - 2022 CHIPFORGE. All rights reserved.

This process is licensed under the Libre Silicon public license; you can redistribute it and/or modify it under the terms of the Libre Silicon public license as published by the Libre Silicon alliance either version 2 of the License, or (at your option) any later version.

This design is distributed in the hope that it will be useful, but WITHOUT ANY WARRANTY; without even the implied warranty of MERCHANTABILITY or FITNESS FOR A PARTICULAR PURPOSE. See the Libre Silicon Public License for more details.

For further clarification consult the complete documentation of the process.
\end{quote}
\uppertitleback

This work was supported by the Stichting NLnet\footnotemark NGI Zero Program under MoU 2019-12-075.
\begin{figure}[htp]
    \begin{minipage}{.5\linewidth}
        \includesvg[width=6cm]{NLnet_banner}
    \end{minipage}
    \hfill
    \begin{minipage}{.25\linewidth}
        \includesvg[width=2cm]{NGIZero-white}
    \end{minipage}
\end{figure}
\footnotetext[1]{https://nlnet.nl}

\vfill
\input{table-revision_history}

\tableofcontents
%%  -------------------------------------------------------------------
%%                  PART I
%%  -------------------------------------------------------------------

\input{part-informal}

%%  -------------------------------------------------------------------
%%                  PART II
%%  -------------------------------------------------------------------

%%  ************    LibreSilicon's StdCellLibrary   *******************
%%
%%  Organisation:   Chipforge
%%                  Germany / European Union
%%
%%  Profile:        Chipforge focus on fine System-on-Chip Cores in
%%                  Verilog HDL Code which are easy understandable and
%%                  adjustable. For further information see
%%                          www.chipforge.org
%%                  there are projects from small cores up to PCBs, too.
%%
%%  File:           StdCellLib/Documents/LaTeX/part-catalog.tex
%%
%%  Purpose:        Part Level File for Standard Cell Library Documentation
%%
%%  ************    LaTeX with circdia.sty package      ***************
%%
%%  ///////////////////////////////////////////////////////////////////
%%
%%  Copyright (c) 2018 - 2021 by chipforge <stdcelllib@nospam.chipforge.org>
%%  All rights reserved.
%%
%%      This Standard Cell Library is licensed under the Libre Silicon
%%      public license; you can redistribute it and/or modify it under
%%      the terms of the Libre Silicon public license as published by
%%      the Libre Silicon alliance, either version 1 of the License, or
%%      (at your option) any later version.
%%
%%      This design is distributed in the hope that it will be useful,
%%      but WITHOUT ANY WARRANTY; without even the implied warranty of
%%      MERCHANTABILITY or FITNESS FOR A PARTICULAR PURPOSE.
%%      See the Libre Silicon Public License for more details.
%%
%%  ///////////////////////////////////////////////////////////////////
\part{Cell Catalog}
\pagestyle{headings}

%%  -------------------------------------------------------------------
%%                  CHAPTER 1
%%  -------------------------------------------------------------------

\chapter{Combinatorial Cells}

%%  ------------    one phase   ---------------------------------------

\section{Inverting Buffers}

\include{INV_datasheet}


\section{NAND Gates}

\input{NAND2_manpage.tex}
%\input{AND2_manpage.tex}
%%  ************    LibreSilicon's StdCellLibrary   *******************
%%
%%  Organisation:   Chipforge
%%                  Germany / European Union
%%
%%  Profile:        Chipforge focus on fine System-on-Chip Cores in
%%                  Verilog HDL Code which are easy understandable and
%%                  adjustable. For further information see
%%                          www.chipforge.org
%%                  there are projects from small cores up to PCBs, too.
%%
%%  File:           StdCellLib/Documents/LaTeX/manpage_NAND3.tex
%%
%%  Purpose:        Manual Page File for NAND3
%%
%%  ************    LaTeX with circdia.sty package      ***************
%%
%%  ///////////////////////////////////////////////////////////////////
%%
%%  Copyright (c) 2018 by chipforge <hsank@nospam.chipforge.org>
%%  All rights reserved.
%%
%%      This Standard Cell Library is licensed under the Libre Silicon
%%      public license; you can redistribute it and/or modify it under
%%      the terms of the Libre Silicon public license as published by
%%      the Libre Silicon alliance, either version 1 of the License, or
%%      (at your option) any later version.
%%
%%      This design is distributed in the hope that it will be useful,
%%      but WITHOUT ANY WARRANTY; without even the implied warranty of
%%      MERCHANTABILITY or FITNESS FOR A PARTICULAR PURPOSE.
%%      See the Libre Silicon Public License for more details.
%%
%%  ///////////////////////////////////////////////////////////////////
\label{NAND3}
\paragraph{Cell}
\begin{quote}
    \textbf{NAND3} - a 3-input Not-AND (or NAND) gate
\end{quote}

\paragraph{Synopsys}
\begin{quote}
    NAND3(Z, C, B, A)
\end{quote}

\paragraph{Description}
\input{NAND3_circuit.tex}
\input{NAND3_schematic.tex}

\paragraph{Truth Table}
\input{NAND3_truthtable.tex}

\paragraph{Usage}

\paragraph{Fan-in / Fan-out}

\paragraph{Layout}

\paragraph{Files}

\paragraph{See also}
\begin{quote}
    NAND2 - a 2-input Not-AND (or NAND) gate
\end{quote}

%\input{AND3_manpage.tex}
%\input{NAND4_manpage.tex}
%%  ************    LibreSilicon's StdCellLibrary   *******************
%%
%%  Organisation:   Chipforge
%%                  Germany / European Union
%%
%%  Profile:        Chipforge focus on fine System-on-Chip Cores in
%%                  Verilog HDL Code which are easy understandable and
%%                  adjustable. For further information see
%%                          www.chipforge.org
%%                  there are projects from small cores up to PCBs, too.
%%
%%  File:           StdCellLib/Documents/LaTeX/manpage_AND4.tex
%%
%%  Purpose:        Manual Page File for AND4
%%
%%  ************    LaTeX with circdia.sty package      ***************
%%
%%  ///////////////////////////////////////////////////////////////////
%%
%%  Copyright (c) 2019 by chipforge <stdcelllib@nospam.chipforge.org>
%%  All rights reserved.
%%
%%      This Standard Cell Library is licensed under the Libre Silicon
%%      public license; you can redistribute it and/or modify it under
%%      the terms of the Libre Silicon public license as published by
%%      the Libre Silicon alliance, either version 1 of the License, or
%%      (at your option) any later version.
%%
%%      This design is distributed in the hope that it will be useful,
%%      but WITHOUT ANY WARRANTY; without even the implied warranty of
%%      MERCHANTABILITY or FITNESS FOR A PARTICULAR PURPOSE.
%%      See the Libre Silicon Public License for more details.
%%
%%  ///////////////////////////////////////////////////////////////////
\label{AND4}
\paragraph{Cell}
\begin{quote}
    \textbf{AND4} - a 4-input AND gate
\end{quote}

\paragraph{Synopsys}
\begin{quote}
    AND4(Z, A3, A2, A1, A)
\end{quote}

\paragraph{Description}
\input{AND4_circuit.tex}
\input{AND4_schematic.tex}

\paragraph{Truth Table}
\input{AND4_truthtable.tex}

\paragraph{Usage}

\paragraph{Fan-in / Fan-out}

\paragraph{Layout}

\paragraph{Files}

\clearpage

 \include{section-AND_gates}
\section{NOR Gates}

%%  ************    LibreSilicon's StdCellLibrary   *******************
%%
%%  Organisation:   Chipforge
%%                  Germany / European Union
%%
%%  Profile:        Chipforge focus on fine System-on-Chip Cores in
%%                  Verilog HDL Code which are easy understandable and
%%                  adjustable. For further information see
%%                          www.chipforge.org
%%                  there are projects from small cores up to PCBs, too.
%%
%%  File:           StdCellLib/Documents/LaTeX/manpage_NOR2.tex
%%
%%  Purpose:        Manual Page File for NOR2
%%
%%  ************    LaTeX with circdia.sty package      ***************
%%
%%  ///////////////////////////////////////////////////////////////////
%%
%%  Copyright (c) 2018 by chipforge <hsank@nospam.chipforge.org>
%%  All rights reserved.
%%
%%      This Standard Cell Library is licensed under the Libre Silicon
%%      public license; you can redistribute it and/or modify it under
%%      the terms of the Libre Silicon public license as published by
%%      the Libre Silicon alliance, either version 1 of the License, or
%%      (at your option) any later version.
%%
%%      This design is distributed in the hope that it will be useful,
%%      but WITHOUT ANY WARRANTY; without even the implied warranty of
%%      MERCHANTABILITY or FITNESS FOR A PARTICULAR PURPOSE.
%%      See the Libre Silicon Public License for more details.
%%
%%  ///////////////////////////////////////////////////////////////////
\label{NOR2}
\paragraph{Cell}
\begin{quote}
    \textbf{NOR2} - a 2-input Not-OR (or NOR) gate
\end{quote}

\paragraph{Synopsys}
\begin{quote}
    NOR2(Z, B, A)
\end{quote}

\paragraph{Description}
\input{NOR2_circuit.tex}
\input{NOR2_schematic.tex}

\paragraph{Truth Table}
\input{NOR2_truthtable.tex}

\paragraph{Usage}

\paragraph{Fan-in / Fan-out}

\paragraph{Layout}

\paragraph{Files}

\paragraph{See also}
\begin{quote}
    NOR3 - a 3-input Not-OR (or NOR) gate
\end{quote}

%%%  ************    LibreSilicon's StdCellLibrary   *******************
%%
%%  Organisation:   Chipforge
%%                  Germany / European Union
%%
%%  Profile:        Chipforge focus on fine System-on-Chip Cores in
%%                  Verilog HDL Code which are easy understandable and
%%                  adjustable. For further information see
%%                          www.chipforge.org
%%                  there are projects from small cores up to PCBs, too.
%%
%%  File:           StdCellLib/Documents/LaTeX/OR2_manpage.tex
%%
%%  Purpose:        Auto-generated Manual Page for OR2
%%
%%  ************    LaTeX with circdia.sty package      ***************
%%
%%  ///////////////////////////////////////////////////////////////////
%%
%%  Copyright (c) 2019 by chipforge <stdcelllib@nospam.chipforge.org>
%%  All rights reserved.
%%
%%      This Standard Cell Library is licensed under the Libre Silicon
%%      public license; you can redistribute it and/or modify it under
%%      the terms of the Libre Silicon public license as published by
%%      the Libre Silicon alliance, either version 1 of the License, or
%%      (at your option) any later version.
%%
%%      This design is distributed in the hope that it will be useful,
%%      but WITHOUT ANY WARRANTY; without even the implied warranty of
%%      MERCHANTABILITY or FITNESS FOR A PARTICULAR PURPOSE.
%%      See the Libre Silicon Public License for more details.
%%
%%  ///////////////////////////////////////////////////////////////////
\subsection{OR2 - a 2-input OR gate} \label{logical:OR2}

\paragraph{Synopsys}
\begin{quote}
    OR2 (Z A1 A)
\end{quote}

\paragraph{Description}
\input{OR2_circuit.tex}
%\input{OR2_schematic.tex}

\paragraph{Truth Table}
%\input{OR2_truthtable.tex}

\paragraph{Usage}

\paragraph{Fan-in / Fan-out}

\paragraph{Layout}

\paragraph{Files}

\clearpage

%%  ************    LibreSilicon's StdCellLibrary   *******************
%%
%%  Organisation:   Chipforge
%%                  Germany / European Union
%%
%%  Profile:        Chipforge focus on fine System-on-Chip Cores in
%%                  Verilog HDL Code which are easy understandable and
%%                  adjustable. For further information see
%%                          www.chipforge.org
%%                  there are projects from small cores up to PCBs, too.
%%
%%  File:           StdCellLib/Documents/LaTeX/NOR3_manpage.tex
%%
%%  Purpose:        Auto-generated Manual Page for NOR3
%%
%%  ************    LaTeX with circdia.sty package      ***************
%%
%%  ///////////////////////////////////////////////////////////////////
%%
%%  Copyright (c) 2019 by chipforge <stdcelllib@nospam.chipforge.org>
%%  All rights reserved.
%%
%%      This Standard Cell Library is licensed under the Libre Silicon
%%      public license; you can redistribute it and/or modify it under
%%      the terms of the Libre Silicon public license as published by
%%      the Libre Silicon alliance, either version 1 of the License, or
%%      (at your option) any later version.
%%
%%      This design is distributed in the hope that it will be useful,
%%      but WITHOUT ANY WARRANTY; without even the implied warranty of
%%      MERCHANTABILITY or FITNESS FOR A PARTICULAR PURPOSE.
%%      See the Libre Silicon Public License for more details.
%%
%%  ///////////////////////////////////////////////////////////////////
\subsection{NOR3 - a 3-input Not-OR (or NOR) gate} \label{logical:NOR3}

\paragraph{Synopsys}
\begin{quote}
    NOR3 (Z A2 A1 A)
\end{quote}

\paragraph{Description}
\input{NOR3_circuit.tex}
\input{NOR3_schematic.tex}

\paragraph{Truth Table}
\input{NOR3_truthtable.tex}

\paragraph{Usage}

\paragraph{Fan-in / Fan-out}

\paragraph{Layout}

\paragraph{Files}

\clearpage

%%%  ************    LibreSilicon's StdCellLibrary   *******************
%%
%%  Organisation:   Chipforge
%%                  Germany / European Union
%%
%%  Profile:        Chipforge focus on fine System-on-Chip Cores in
%%                  Verilog HDL Code which are easy understandable and
%%                  adjustable. For further information see
%%                          www.chipforge.org
%%                  there are projects from small cores up to PCBs, too.
%%
%%  File:           StdCellLib/Documents/LaTeX/OR3_manpage.tex
%%
%%  Purpose:        Auto-generated Manual Page for OR3
%%
%%  ************    LaTeX with circdia.sty package      ***************
%%
%%  ///////////////////////////////////////////////////////////////////
%%
%%  Copyright (c) 2019 by chipforge <stdcelllib@nospam.chipforge.org>
%%  All rights reserved.
%%
%%      This Standard Cell Library is licensed under the Libre Silicon
%%      public license; you can redistribute it and/or modify it under
%%      the terms of the Libre Silicon public license as published by
%%      the Libre Silicon alliance, either version 1 of the License, or
%%      (at your option) any later version.
%%
%%      This design is distributed in the hope that it will be useful,
%%      but WITHOUT ANY WARRANTY; without even the implied warranty of
%%      MERCHANTABILITY or FITNESS FOR A PARTICULAR PURPOSE.
%%      See the Libre Silicon Public License for more details.
%%
%%  ///////////////////////////////////////////////////////////////////
\subsection{OR3 - a 3-input OR gate} \label{logical:OR3}

\paragraph{Synopsys}
\begin{quote}
    OR3 (Z A2 A1 A)
\end{quote}

\paragraph{Description}
\input{OR3_circuit.tex}
%\input{OR3_schematic.tex}

\paragraph{Truth Table}
%\input{OR3_truthtable.tex}

\paragraph{Usage}

\paragraph{Fan-in / Fan-out}

\paragraph{Layout}

\paragraph{Files}

\clearpage

%%%  ************    LibreSilicon's StdCellLibrary   *******************
%%
%%  Organisation:   Chipforge
%%                  Germany / European Union
%%
%%  Profile:        Chipforge focus on fine System-on-Chip Cores in
%%                  Verilog HDL Code which are easy understandable and
%%                  adjustable. For further information see
%%                          www.chipforge.org
%%                  there are projects from small cores up to PCBs, too.
%%
%%  File:           StdCellLib/Documents/LaTeX/NOR4_manpage.tex
%%
%%  Purpose:        Manual Page File for NOR4
%%
%%  ************    LaTeX with circdia.sty package      ***************
%%
%%  ///////////////////////////////////////////////////////////////////
%%
%%  Copyright (c) 2019 by chipforge <stdcelllib@nospam.chipforge.org>
%%  All rights reserved.
%%
%%      This Standard Cell Library is licensed under the Libre Silicon
%%      public license; you can redistribute it and/or modify it under
%%      the terms of the Libre Silicon public license as published by
%%      the Libre Silicon alliance, either version 1 of the License, or
%%      (at your option) any later version.
%%
%%      This design is distributed in the hope that it will be useful,
%%      but WITHOUT ANY WARRANTY; without even the implied warranty of
%%      MERCHANTABILITY or FITNESS FOR A PARTICULAR PURPOSE.
%%      See the Libre Silicon Public License for more details.
%%
%%  ///////////////////////////////////////////////////////////////////
\label{NOR4}
\paragraph{Cell}
\begin{quote}
    \textbf{NOR4} - a 4-input Not-OR (or NOR) gate
\end{quote}

\paragraph{Synopsys}
\begin{quote}
    NOR4(Z, A3, A2, A1, A)
\end{quote}

\paragraph{Description}
\input{NOR4_circuit.tex}
%\input{NOR4_schematic.tex}

\paragraph{Truth Table}
%\input{NOR4_truthtable.tex}

\paragraph{Usage}

\paragraph{Fan-in / Fan-out}

\paragraph{Layout}

\paragraph{Files}

\input{OR4_manpage.tex}
 \include{section-OR_gates}
\section{(Inverting) Multiplexer}

%%  ************    LibreSilicon's StdCellLibrary   *******************
%%
%%  Organisation:   Chipforge
%%                  Germany / European Union
%%
%%  Profile:        Chipforge focus on fine System-on-Chip Cores in
%%                  Verilog HDL Code which are easy understandable and
%%                  adjustable. For further information see
%%                          www.chipforge.org
%%                  there are projects from small cores up to PCBs, too.
%%
%%  File:           StdCellLib/Documents/LaTeX/manpage_MUXI2.tex
%%
%%  Purpose:        Manual Page File for MUXI2
%%
%%  ************    LaTeX with circdia.sty package      ***************
%%
%%  ///////////////////////////////////////////////////////////////////
%%
%%  Copyright (c) 2018 by chipforge <hsank@nospam.chipforge.org>
%%  All rights reserved.
%%
%%      This Standard Cell Library is licensed under the Libre Silicon
%%      public license; you can redistribute it and/or modify it under
%%      the terms of the Libre Silicon public license as published by
%%      the Libre Silicon alliance, either version 1 of the License, or
%%      (at your option) any later version.
%%
%%      This design is distributed in the hope that it will be useful,
%%      but WITHOUT ANY WARRANTY; without even the implied warranty of
%%      MERCHANTABILITY or FITNESS FOR A PARTICULAR PURPOSE.
%%      See the Libre Silicon Public License for more details.
%%
%%  ///////////////////////////////////////////////////////////////////
\label{MUXI2}
\paragraph{Cell}
\begin{quote}
    \textbf{MUXI2} - a 2-to-1 Multiplexor Invert cell
\end{quote}

\paragraph{Synopsys}
\begin{quote}
    MUXI2(Z, S, A1, A0)
\end{quote}

\paragraph{Description}
\input{MUXI2_circuit.tex}
\input{MUXI2_schematic.tex}

\paragraph{Truth Table}
\input{MUXI2_truthtable.tex}

\paragraph{Usage}

\paragraph{Fan-in / Fan-out}

\paragraph{Layout}

\paragraph{Files}
%\input{files_MUXI2.tex}

\paragraph{See also}
\begin{quote}
    MUXI3 - a 3-to-1 Multiplexor Invert cell \\
    MUXI4 - a 4-to-1 Multiplexor Invert cell
\end{quote}

%\input{MUXI4_manpage.tex}
%\input{MUX8_manpage.tex}


%%  ------------    two phases  ---------------------------------------

\section{AND-OR(-Invert) Complex Gates}

\input{AOI21_manpage.tex}
%%  ************    LibreSilicon's StdCellLibrary   *******************
%%
%%  Organisation:   Chipforge
%%                  Germany / European Union
%%
%%  Profile:        Chipforge focus on fine System-on-Chip Cores in
%%                  Verilog HDL Code which are easy understandable and
%%                  adjustable. For further information see
%%                          www.chipforge.org
%%                  there are projects from small cores up to PCBs, too.
%%
%%  File:           StdCellLib/Documents/LaTeX/AO21_manpage.tex
%%
%%  Purpose:        Auto-generated Manual Page for AO21
%%
%%  ************    LaTeX with circdia.sty package      ***************
%%
%%  ///////////////////////////////////////////////////////////////////
%%
%%  Copyright (c) 2018 by chipforge <hsank@nospam.chipforge.org>
%%  All rights reserved.
%%
%%      This Standard Cell Library is licensed under the Libre Silicon
%%      public license; you can redistribute it and/or modify it under
%%      the terms of the Libre Silicon public license as published by
%%      the Libre Silicon alliance, either version 1 of the License, or
%%      (at your option) any later version.
%%
%%      This design is distributed in the hope that it will be useful,
%%      but WITHOUT ANY WARRANTY; without even the implied warranty of
%%      MERCHANTABILITY or FITNESS FOR A PARTICULAR PURPOSE.
%%      See the Libre Silicon Public License for more details.
%%
%%  ///////////////////////////////////////////////////////////////////
\subsection{AO21 - a 2-1-input AND-OR gate} \label{logical:AO21}

\paragraph{Synopsys}
\begin{quote}
    AO21 (Z B1 B A)
\end{quote}

\paragraph{Description}
\input{AO21_circuit.tex}
%\input{AO21_schematic.tex}

\paragraph{Truth Table}
%\input{AO21_truthtable.tex}

\paragraph{Usage}

\paragraph{Fan-in / Fan-out}

\paragraph{Layout}

\paragraph{Files}

\clearpage

%%  ************    LibreSilicon's StdCellLibrary   *******************
%%
%%  Organisation:   Chipforge
%%                  Germany / European Union
%%
%%  Profile:        Chipforge focus on fine System-on-Chip Cores in
%%                  Verilog HDL Code which are easy understandable and
%%                  adjustable. For further information see
%%                          www.chipforge.org
%%                  there are projects from small cores up to PCBs, too.
%%
%%  File:           StdCellLib/Documents/LaTeX/AOI22_manpage.tex
%%
%%  Purpose:        Manual Page File for AOI22
%%
%%  ************    LaTeX with circdia.sty package      ***************
%%
%%  ///////////////////////////////////////////////////////////////////
%%
%%  Copyright (c) 2019 by chipforge <stdcelllib@nospam.chipforge.org>
%%  All rights reserved.
%%
%%      This Standard Cell Library is licensed under the Libre Silicon
%%      public license; you can redistribute it and/or modify it under
%%      the terms of the Libre Silicon public license as published by
%%      the Libre Silicon alliance, either version 1 of the License, or
%%      (at your option) any later version.
%%
%%      This design is distributed in the hope that it will be useful,
%%      but WITHOUT ANY WARRANTY; without even the implied warranty of
%%      MERCHANTABILITY or FITNESS FOR A PARTICULAR PURPOSE.
%%      See the Libre Silicon Public License for more details.
%%
%%  ///////////////////////////////////////////////////////////////////
\label{AOI22}
\paragraph{Cell}
\begin{quote}
    \textbf{AOI22} - a 2-2-input AND-OR-Invert gate
\end{quote}

\paragraph{Synopsys}
\begin{quote}
    AOI22(Z, B1, B, A1, A)
\end{quote}

\paragraph{Description}
\input{AOI22_circuit.tex}
%\input{AOI22_schematic.tex}

\paragraph{Truth Table}
%\input{AOI22_truthtable.tex}

\paragraph{Usage}

\paragraph{Fan-in / Fan-out}

\paragraph{Layout}

\paragraph{Files}

\input{AO22_manpage.tex}
%%  ************    LibreSilicon's StdCellLibrary   *******************
%%
%%  Organisation:   Chipforge
%%                  Germany / European Union
%%
%%  Profile:        Chipforge focus on fine System-on-Chip Cores in
%%                  Verilog HDL Code which are easy understandable and
%%                  adjustable. For further information see
%%                          www.chipforge.org
%%                  there are projects from small cores up to PCBs, too.
%%
%%  File:           StdCellLib/Documents/LaTeX/AOI23_manpage.tex
%%
%%  Purpose:        Auto-generated Manual Page for AOI23
%%
%%  ************    LaTeX with circdia.sty package      ***************
%%
%%  ///////////////////////////////////////////////////////////////////
%%
%%  Copyright (c) 2019 by chipforge <stdcelllib@nospam.chipforge.org>
%%  All rights reserved.
%%
%%      This Standard Cell Library is licensed under the Libre Silicon
%%      public license; you can redistribute it and/or modify it under
%%      the terms of the Libre Silicon public license as published by
%%      the Libre Silicon alliance, either version 1 of the License, or
%%      (at your option) any later version.
%%
%%      This design is distributed in the hope that it will be useful,
%%      but WITHOUT ANY WARRANTY; without even the implied warranty of
%%      MERCHANTABILITY or FITNESS FOR A PARTICULAR PURPOSE.
%%      See the Libre Silicon Public License for more details.
%%
%%  ///////////////////////////////////////////////////////////////////
\subsection{AOI23 - a 2-3-input AND-OR-Invert gate} \label{logical:AOI23}

\paragraph{Synopsys}
\begin{quote}
    AOI23 (Z B1 B A2 A1 A)
\end{quote}

\paragraph{Description}
\input{AOI23_circuit.tex}
%\input{AOI23_schematic.tex}

\paragraph{Truth Table}
%\input{AOI23_truthtable.tex}

\paragraph{Usage}

\paragraph{Fan-in / Fan-out}

\paragraph{Layout}

\paragraph{Files}

\clearpage

\input{AO23_manpage.tex}
\input{AOI31_manpage.tex}
\input{AO31_manpage.tex}
\input{AOI32_manpage.tex}
\input{AO32_manpage.tex}
\input{AOI33_manpage.tex}
%%  ************    LibreSilicon's StdCellLibrary   *******************
%%
%%  Organisation:   Chipforge
%%                  Germany / European Union
%%
%%  Profile:        Chipforge focus on fine System-on-Chip Cores in
%%                  Verilog HDL Code which are easy understandable and
%%                  adjustable. For further information see
%%                          www.chipforge.org
%%                  there are projects from small cores up to PCBs, too.
%%
%%  File:           StdCellLib/Documents/LaTeX/AO33_manpage.tex
%%
%%  Purpose:        Auto-generated Manual Page for AO33
%%
%%  ************    LaTeX with circdia.sty package      ***************
%%
%%  ///////////////////////////////////////////////////////////////////
%%
%%  Copyright (c) 2019 by chipforge <stdcelllib@nospam.chipforge.org>
%%  All rights reserved.
%%
%%      This Standard Cell Library is licensed under the Libre Silicon
%%      public license; you can redistribute it and/or modify it under
%%      the terms of the Libre Silicon public license as published by
%%      the Libre Silicon alliance, either version 1 of the License, or
%%      (at your option) any later version.
%%
%%      This design is distributed in the hope that it will be useful,
%%      but WITHOUT ANY WARRANTY; without even the implied warranty of
%%      MERCHANTABILITY or FITNESS FOR A PARTICULAR PURPOSE.
%%      See the Libre Silicon Public License for more details.
%%
%%  ///////////////////////////////////////////////////////////////////
\subsection{AO33 - a 3-3-input AND-OR-Invert gate} \label{logical:AO33}

\paragraph{Synopsys}
\begin{quote}
    AO33 (Z B2 B1 B A2 A1 A)
\end{quote}

\paragraph{Description}
\input{AO33_circuit.tex}
%\input{AO33_schematic.tex}

\paragraph{Truth Table}
%\input{AO33_truthtable.tex}

\paragraph{Usage}

\paragraph{Fan-in / Fan-out}

\paragraph{Layout}

\paragraph{Files}

\clearpage

%%  ************    LibreSilicon's StdCellLibrary   *******************
%%
%%  Organisation:   Chipforge
%%                  Germany / European Union
%%
%%  Profile:        Chipforge focus on fine System-on-Chip Cores in
%%                  Verilog HDL Code which are easy understandable and
%%                  adjustable. For further information see
%%                          www.chipforge.org
%%                  there are projects from small cores up to PCBs, too.
%%
%%  File:           StdCellLib/Documents/LaTeX/AOI41_manpage.tex
%%
%%  Purpose:        Auto-generated Manual Page for AOI41
%%
%%  ************    LaTeX with circdia.sty package      ***************
%%
%%  ///////////////////////////////////////////////////////////////////
%%
%%  Copyright (c) 2019 by chipforge <stdcelllib@nospam.chipforge.org>
%%  All rights reserved.
%%
%%      This Standard Cell Library is licensed under the Libre Silicon
%%      public license; you can redistribute it and/or modify it under
%%      the terms of the Libre Silicon public license as published by
%%      the Libre Silicon alliance, either version 1 of the License, or
%%      (at your option) any later version.
%%
%%      This design is distributed in the hope that it will be useful,
%%      but WITHOUT ANY WARRANTY; without even the implied warranty of
%%      MERCHANTABILITY or FITNESS FOR A PARTICULAR PURPOSE.
%%      See the Libre Silicon Public License for more details.
%%
%%  ///////////////////////////////////////////////////////////////////
\subsection{AOI41 - a 4-1-input AND-OR-Invert gate} \label{logical:AOI41}

\paragraph{Synopsys}
\begin{quote}
    AOI41 (Z B3 B2 B1 B A)
\end{quote}

\paragraph{Description}
\input{AOI41_circuit.tex}
%\input{AOI41_schematic.tex}

\paragraph{Truth Table}
%\input{AOI41_truthtable.tex}

\paragraph{Usage}

\paragraph{Fan-in / Fan-out}

\paragraph{Layout}

\paragraph{Files}

\clearpage

\input{AO41_manpage.tex}
%%  ************    LibreSilicon's StdCellLibrary   *******************
%%
%%  Organisation:   Chipforge
%%                  Germany / European Union
%%
%%  Profile:        Chipforge focus on fine System-on-Chip Cores in
%%                  Verilog HDL Code which are easy understandable and
%%                  adjustable. For further information see
%%                          www.chipforge.org
%%                  there are projects from small cores up to PCBs, too.
%%
%%  File:           StdCellLib/Documents/LaTeX/AOI42_manpage.tex
%%
%%  Purpose:        Manual Page File for AOI42
%%
%%  ************    LaTeX with circdia.sty package      ***************
%%
%%  ///////////////////////////////////////////////////////////////////
%%
%%  Copyright (c) 2019 by chipforge <stdcelllib@nospam.chipforge.org>
%%  All rights reserved.
%%
%%      This Standard Cell Library is licensed under the Libre Silicon
%%      public license; you can redistribute it and/or modify it under
%%      the terms of the Libre Silicon public license as published by
%%      the Libre Silicon alliance, either version 1 of the License, or
%%      (at your option) any later version.
%%
%%      This design is distributed in the hope that it will be useful,
%%      but WITHOUT ANY WARRANTY; without even the implied warranty of
%%      MERCHANTABILITY or FITNESS FOR A PARTICULAR PURPOSE.
%%      See the Libre Silicon Public License for more details.
%%
%%  ///////////////////////////////////////////////////////////////////
\label{AOI42}
\paragraph{Cell}
\begin{quote}
    \textbf{AOI42} - a 4-2-input AND-OR-Invert gate
\end{quote}

\paragraph{Synopsys}
\begin{quote}
    AOI42(Y, B3, B2, B1, B, A1, A)
\end{quote}

\paragraph{Description}
\input{AOI42_circuit.tex}
%\input{AOI42_schematic.tex}

\paragraph{Truth Table}
%\input{AOI42_truthtable.tex}

\paragraph{Usage}

\paragraph{Fan-in / Fan-out}

\paragraph{Layout}

\paragraph{Files}

\input{AO42_manpage.tex}
%%  ************    LibreSilicon's StdCellLibrary   *******************
%%
%%  Organisation:   Chipforge
%%                  Germany / European Union
%%
%%  Profile:        Chipforge focus on fine System-on-Chip Cores in
%%                  Verilog HDL Code which are easy understandable and
%%                  adjustable. For further information see
%%                          www.chipforge.org
%%                  there are projects from small cores up to PCBs, too.
%%
%%  File:           StdCellLib/Documents/LaTeX/AOI43_manpage.tex
%%
%%  Purpose:        Auto-generated Manual Page for AOI43
%%
%%  ************    LaTeX with circdia.sty package      ***************
%%
%%  ///////////////////////////////////////////////////////////////////
%%
%%  Copyright (c) 2019 by chipforge <stdcelllib@nospam.chipforge.org>
%%  All rights reserved.
%%
%%      This Standard Cell Library is licensed under the Libre Silicon
%%      public license; you can redistribute it and/or modify it under
%%      the terms of the Libre Silicon public license as published by
%%      the Libre Silicon alliance, either version 1 of the License, or
%%      (at your option) any later version.
%%
%%      This design is distributed in the hope that it will be useful,
%%      but WITHOUT ANY WARRANTY; without even the implied warranty of
%%      MERCHANTABILITY or FITNESS FOR A PARTICULAR PURPOSE.
%%      See the Libre Silicon Public License for more details.
%%
%%  ///////////////////////////////////////////////////////////////////
\subsection{AOI43 - a 4-3-input AND-OR-Invert gate} \label{logical:AOI43}

\paragraph{Synopsys}
\begin{quote}
    AOI43 (Y, B3, B2, B1, B, A2, A1, A)
\end{quote}

\paragraph{Description}
\input{AOI43_circuit.tex}
%\input{AOI43_schematic.tex}

\paragraph{Truth Table}
%\input{AOI43_truthtable.tex}

\paragraph{Usage}

\paragraph{Fan-in / Fan-out}

\paragraph{Layout}

\paragraph{Files}

\clearpage

\input{AO43_manpage.tex}
 \include{section-AO_complex}
\section{OR-AND(-Invert) Complex Gates}

\include{OAI21_datasheet} \include{OA21_datasheet}
\include{OAI22_datasheet} \include{OA22_datasheet}
\include{OAI23_datasheet} \include{OA23_datasheet}
\include{OAI31_datasheet} \include{OA31_datasheet}
\include{OAI32_datasheet} \include{OA32_datasheet}
\include{OAI33_datasheet} \include{OA33_datasheet}
\include{OAI41_datasheet} \include{OA41_datasheet}
\include{OAI42_datasheet} \include{OA42_datasheet}
\include{OAI43_datasheet} \include{OA43_datasheet}
 \include{section-OA_complex}
%%  ************    LibreSilicon's StdCellLibrary   *******************
%%
%%  Organisation:   Chipforge
%%                  Germany / European Union
%%
%%  Profile:        Chipforge focus on fine System-on-Chip Cores in
%%                  Verilog HDL Code which are easy understandable and
%%                  adjustable. For further information see
%%                          www.chipforge.org
%%                  there are projects from small cores up to PCBs, too.
%%
%%  File:           StdCellLib/Documents/LaTeX/section-AAOI_complex.tex
%%
%%  Purpose:        Section Level File for Standard Cell Library Documentation
%%
%%  ************    LaTeX with circdia.sty package      ***************
%%
%%  ///////////////////////////////////////////////////////////////////
%%
%%  Copyright (c) 2018 - 2021 by
%%                  chipforge <stdcelllib@nospam.chipforge.org>
%%  All rights reserved.
%%
%%      This Standard Cell Library is licensed under the Libre Silicon
%%      public license; you can redistribute it and/or modify it under
%%      the terms of the Libre Silicon public license as published by
%%      the Libre Silicon alliance, either version 1 of the License, or
%%      (at your option) any later version.
%%
%%      This design is distributed in the hope that it will be useful,
%%      but WITHOUT ANY WARRANTY; without even the implied warranty of
%%      MERCHANTABILITY or FITNESS FOR A PARTICULAR PURPOSE.
%%      See the Libre Silicon Public License for more details.
%%
\section{AND-AND-OR(-Invert) Complex Gates}

\include{AAOI22_datasheet} \include{AAO22_datasheet}
\include{AAOI32_datasheet} \include{AAO32_datasheet}
\include{AAOI33_datasheet} \include{AAO33_datasheet}
\include{AAOI42_datasheet} \include{AAO42_datasheet}
\include{AAOI43_datasheet} \include{AAO43_datasheet}
\include{AAOI44_datasheet} \include{AAO44_datasheet}

\include{AAOI221_datasheet} \include{AAO221_datasheet}
\include{AAOI321_datasheet} \include{AAO321_datasheet}
\include{AAOI331_datasheet} \include{AAO331_datasheet}
\include{AAOI421_datasheet} \include{AAO421_datasheet}
\include{AAOI431_datasheet} \include{AAO431_datasheet}
 \include{section-AAO_complex}
\section{OR-OR-AND(-Invert) Complex Gates}

%%  ************    LibreSilicon's StdCellLibrary   *******************
%%
%%  Organisation:   Chipforge
%%                  Germany / European Union
%%
%%  Profile:        Chipforge focus on fine System-on-Chip Cores in
%%                  Verilog HDL Code which are easy understandable and
%%                  adjustable. For further information see
%%                          www.chipforge.org
%%                  there are projects from small cores up to PCBs, too.
%%
%%  File:           StdCellLib/Documents/LaTeX/OOAI22_manpage.tex
%%
%%  Purpose:        Manual Page File for OOAI22
%%
%%  ************    LaTeX with circdia.sty package      ***************
%%
%%  ///////////////////////////////////////////////////////////////////
%%
%%  Copyright (c) 2019 by chipforge <stdcelllib@nospam.chipforge.org>
%%  All rights reserved.
%%
%%      This Standard Cell Library is licensed under the Libre Silicon
%%      public license; you can redistribute it and/or modify it under
%%      the terms of the Libre Silicon public license as published by
%%      the Libre Silicon alliance, either version 1 of the License, or
%%      (at your option) any later version.
%%
%%      This design is distributed in the hope that it will be useful,
%%      but WITHOUT ANY WARRANTY; without even the implied warranty of
%%      MERCHANTABILITY or FITNESS FOR A PARTICULAR PURPOSE.
%%      See the Libre Silicon Public License for more details.
%%
%%  ///////////////////////////////////////////////////////////////////
\label{OOAI22}
\paragraph{Cell}
\begin{quote}
    \textbf{OOAI22} - a 2-2-input OR-OR-AND-Invert gate
\end{quote}

\paragraph{Synopsys}
\begin{quote}
    OOAI22(Y, B1, B0, A1, A0)
\end{quote}

\paragraph{Description}
\input{OOAI22_circuit.tex}
\input{OOAI22_schematic.tex}

\paragraph{Truth Table}
\input{OOAI22_truthtable.tex}

\paragraph{Usage}

\paragraph{Fan-in / Fan-out}

\paragraph{Layout}

\paragraph{Files}

\clearpage

\input{OOA22_manpage.tex}
%%  ************    LibreSilicon's StdCellLibrary   *******************
%%
%%  Organisation:   Chipforge
%%                  Germany / European Union
%%
%%  Profile:        Chipforge focus on fine System-on-Chip Cores in
%%                  Verilog HDL Code which are easy understandable and
%%                  adjustable. For further information see
%%                          www.chipforge.org
%%                  there are projects from small cores up to PCBs, too.
%%
%%  File:           StdCellLib/Documents/LaTeX/OOAI32_manpage.tex
%%
%%  Purpose:        Manual Page File for OOAI32
%%
%%  ************    LaTeX with circdia.sty package      ***************
%%
%%  ///////////////////////////////////////////////////////////////////
%%
%%  Copyright (c) 2019 by chipforge <stdcelllib@nospam.chipforge.org>
%%  All rights reserved.
%%
%%      This Standard Cell Library is licensed under the Libre Silicon
%%      public license; you can redistribute it and/or modify it under
%%      the terms of the Libre Silicon public license as published by
%%      the Libre Silicon alliance, either version 1 of the License, or
%%      (at your option) any later version.
%%
%%      This design is distributed in the hope that it will be useful,
%%      but WITHOUT ANY WARRANTY; without even the implied warranty of
%%      MERCHANTABILITY or FITNESS FOR A PARTICULAR PURPOSE.
%%      See the Libre Silicon Public License for more details.
%%
%%  ///////////////////////////////////////////////////////////////////
\label{OOAI32}
\paragraph{Cell}
\begin{quote}
    \textbf{OOAI32} - a 3-2-input OR-OR-AND-Invert gate
\end{quote}

\paragraph{Synopsys}
\begin{quote}
    OOAI32(Z, B2, B1, B, A1, A)
\end{quote}

\paragraph{Description}
\input{OOAI32_circuit.tex}
\input{OOAI32_schematic.tex}

\paragraph{Truth Table}
\input{OOAI32_truthtable.tex}

\paragraph{Usage}

\paragraph{Fan-in / Fan-out}

\paragraph{Layout}

\paragraph{Files}

%%  ************    LibreSilicon's StdCellLibrary   *******************
%%
%%  Organisation:   Chipforge
%%                  Germany / European Union
%%
%%  Profile:        Chipforge focus on fine System-on-Chip Cores in
%%                  Verilog HDL Code which are easy understandable and
%%                  adjustable. For further information see
%%                          www.chipforge.org
%%                  there are projects from small cores up to PCBs, too.
%%
%%  File:           StdCellLib/Documents/LaTeX/OOA32_manpage.tex
%%
%%  Purpose:        Manual Page File for OOA32
%%
%%  ************    LaTeX with circdia.sty package      ***************
%%
%%  ///////////////////////////////////////////////////////////////////
%%
%%  Copyright (c) 2019 by chipforge <stdcelllib@nospam.chipforge.org>
%%  All rights reserved.
%%
%%      This Standard Cell Library is licensed under the Libre Silicon
%%      public license; you can redistribute it and/or modify it under
%%      the terms of the Libre Silicon public license as published by
%%      the Libre Silicon alliance, either version 1 of the License, or
%%      (at your option) any later version.
%%
%%      This design is distributed in the hope that it will be useful,
%%      but WITHOUT ANY WARRANTY; without even the implied warranty of
%%      MERCHANTABILITY or FITNESS FOR A PARTICULAR PURPOSE.
%%      See the Libre Silicon Public License for more details.
%%
%%  ///////////////////////////////////////////////////////////////////
\label{OOA32}
\paragraph{Cell}
\begin{quote}
    \textbf{OOA32} - a 3-2-input OR-OR-AND gate
\end{quote}

\paragraph{Synopsys}
\begin{quote}
    OOA32(Z, B2, B1, B, A1, A)
\end{quote}

\paragraph{Description}
\input{OOA32_circuit.tex}
%\input{OOA32_schematic.tex}

\paragraph{Truth Table}
%\input{OOA32_truthtable.tex}

\paragraph{Usage}

\paragraph{Fan-in / Fan-out}

\paragraph{Layout}

\paragraph{Files}

\input{OOAI33_manpage.tex}
\input{OOA33_manpage.tex}
\input{OOAI42_manpage.tex}
\input{OOA42_manpage.tex}
\input{OOAI43_manpage.tex}
%%  ************    LibreSilicon's StdCellLibrary   *******************
%%
%%  Organisation:   Chipforge
%%                  Germany / European Union
%%
%%  Profile:        Chipforge focus on fine System-on-Chip Cores in
%%                  Verilog HDL Code which are easy understandable and
%%                  adjustable. For further information see
%%                          www.chipforge.org
%%                  there are projects from small cores up to PCBs, too.
%%
%%  File:           StdCellLib/Documents/LaTeX/OOA43_manpage.tex
%%
%%  Purpose:        Manual Page File for OOA43
%%
%%  ************    LaTeX with circdia.sty package      ***************
%%
%%  ///////////////////////////////////////////////////////////////////
%%
%%  Copyright (c) 2019 by chipforge <stdcelllib@nospam.chipforge.org>
%%  All rights reserved.
%%
%%      This Standard Cell Library is licensed under the Libre Silicon
%%      public license; you can redistribute it and/or modify it under
%%      the terms of the Libre Silicon public license as published by
%%      the Libre Silicon alliance, either version 1 of the License, or
%%      (at your option) any later version.
%%
%%      This design is distributed in the hope that it will be useful,
%%      but WITHOUT ANY WARRANTY; without even the implied warranty of
%%      MERCHANTABILITY or FITNESS FOR A PARTICULAR PURPOSE.
%%      See the Libre Silicon Public License for more details.
%%
%%  ///////////////////////////////////////////////////////////////////
\label{OOA43}
\paragraph{Cell}
\begin{quote}
    \textbf{OOA43} - a 4-3-input OR-OR-AND gate
\end{quote}

\paragraph{Synopsys}
\begin{quote}
    OOA43(Z, B3, B2, B1, B, A2, A1, A)
\end{quote}

\paragraph{Description}
\input{OOA43_circuit.tex}
%\input{OOA43_schematic.tex}

\paragraph{Truth Table}
%\input{OOA43_truthtable.tex}

\paragraph{Usage}

\paragraph{Fan-in / Fan-out}

\paragraph{Layout}

\paragraph{Files}

%%  ************    LibreSilicon's StdCellLibrary   *******************
%%
%%  Organisation:   Chipforge
%%                  Germany / European Union
%%
%%  Profile:        Chipforge focus on fine System-on-Chip Cores in
%%                  Verilog HDL Code which are easy understandable and
%%                  adjustable. For further information see
%%                          www.chipforge.org
%%                  there are projects from small cores up to PCBs, too.
%%
%%  File:           StdCellLib/Documents/LaTeX/OOAI44_manpage.tex
%%
%%  Purpose:        Manual Page File for OOAI44
%%
%%  ************    LaTeX with circdia.sty package      ***************
%%
%%  ///////////////////////////////////////////////////////////////////
%%
%%  Copyright (c) 2019 by chipforge <stdcelllib@nospam.chipforge.org>
%%  All rights reserved.
%%
%%      This Standard Cell Library is licensed under the Libre Silicon
%%      public license; you can redistribute it and/or modify it under
%%      the terms of the Libre Silicon public license as published by
%%      the Libre Silicon alliance, either version 1 of the License, or
%%      (at your option) any later version.
%%
%%      This design is distributed in the hope that it will be useful,
%%      but WITHOUT ANY WARRANTY; without even the implied warranty of
%%      MERCHANTABILITY or FITNESS FOR A PARTICULAR PURPOSE.
%%      See the Libre Silicon Public License for more details.
%%
%%  ///////////////////////////////////////////////////////////////////
\label{OOAI44}
\paragraph{Cell}
\begin{quote}
    \textbf{OOAI44} - a 4-4-input OR-OR-AND-Invert gate
\end{quote}

\paragraph{Synopsys}
\begin{quote}
    OOAI44(Y, B3, B2, B1, B, A3, A2, A1, A)
\end{quote}

\paragraph{Description}
\input{OOAI44_circuit.tex}
%\input{OOAI44_schematic.tex}

\paragraph{Truth Table}
%\input{OOAI44_truthtable.tex}

\paragraph{Usage}

\paragraph{Fan-in / Fan-out}

\paragraph{Layout}

\paragraph{Files}

\input{OOA44_manpage.tex}

\input{OOAI221_manpage.tex}
\input{OOA221_manpage.tex}
\input{OOAI321_manpage.tex}
%%  ************    LibreSilicon's StdCellLibrary   *******************
%%
%%  Organisation:   Chipforge
%%                  Germany / European Union
%%
%%  Profile:        Chipforge focus on fine System-on-Chip Cores in
%%                  Verilog HDL Code which are easy understandable and
%%                  adjustable. For further information see
%%                          www.chipforge.org
%%                  there are projects from small cores up to PCBs, too.
%%
%%  File:           StdCellLib/Documents/LaTeX/OOA321_manpage.tex
%%
%%  Purpose:        Manual Page File for OOA321
%%
%%  ************    LaTeX with circdia.sty package      ***************
%%
%%  ///////////////////////////////////////////////////////////////////
%%
%%  Copyright (c) 2019 by chipforge <stdcelllib@nospam.chipforge.org>
%%  All rights reserved.
%%
%%      This Standard Cell Library is licensed under the Libre Silicon
%%      public license; you can redistribute it and/or modify it under
%%      the terms of the Libre Silicon public license as published by
%%      the Libre Silicon alliance, either version 1 of the License, or
%%      (at your option) any later version.
%%
%%      This design is distributed in the hope that it will be useful,
%%      but WITHOUT ANY WARRANTY; without even the implied warranty of
%%      MERCHANTABILITY or FITNESS FOR A PARTICULAR PURPOSE.
%%      See the Libre Silicon Public License for more details.
%%
%%  ///////////////////////////////////////////////////////////////////
\label{OOA321}
\paragraph{Cell}
\begin{quote}
    \textbf{OOA321} - a 3-2-1-input OR-OR-AND gate
\end{quote}

\paragraph{Synopsys}
\begin{quote}
    OOA321(Z, C2, C1, C, B1, B, A)
\end{quote}

\paragraph{Description}
\input{OOA321_circuit.tex}
%\input{OOA321_schematic.tex}

\paragraph{Truth Table}
%\input{OOA321_truthtable.tex}

\paragraph{Usage}

\paragraph{Fan-in / Fan-out}

\paragraph{Layout}

\paragraph{Files}

\input{OOAI331_manpage.tex}
\input{OOA331_manpage.tex}
%\input{OOAI421_manpage.tex}
%\input{OOA421_manpage.tex}
%\input{OOAI431_manpage.tex}
%\input{OOA431_manpage.tex}
 \include{section-OOA_complex}
%%  ************    LibreSilicon's StdCellLibrary   *******************
%%
%%  Organisation:   Chipforge
%%                  Germany / European Union
%%
%%  Profile:        Chipforge focus on fine System-on-Chip Cores in
%%                  Verilog HDL Code which are easy understandable and
%%                  adjustable. For further information see
%%                          www.chipforge.org
%%                  there are projects from small cores up to PCBs, too.
%%
%%  File:           StdCellLib/Documents/LaTeX/section-AAAOI_complex.tex
%%
%%  Purpose:        Section Level File for Standard Cell Library Documentation
%%
%%  ************    LaTeX with circdia.sty package      ***************
%%
%%  ///////////////////////////////////////////////////////////////////
%%
%%  Copyright (c) 2018 - 2021 by
%%                  chipforge <stdcelllib@nospam.chipforge.org>
%%  All rights reserved.
%%
%%      This Standard Cell Library is licensed under the Libre Silicon
%%      public license; you can redistribute it and/or modify it under
%%      the terms of the Libre Silicon public license as published by
%%      the Libre Silicon alliance, either version 1 of the License, or
%%      (at your option) any later version.
%%
%%      This design is distributed in the hope that it will be useful,
%%      but WITHOUT ANY WARRANTY; without even the implied warranty of
%%      MERCHANTABILITY or FITNESS FOR A PARTICULAR PURPOSE.
%%      See the Libre Silicon Public License for more details.
%%
\section{AND-AND-AND-OR(-Invert) Complex Gates}

\include{AAAOI222_datasheet} \include{AAAO222_datasheet}
\include{AAAOI322_datasheet} \include{AAAO322_datasheet}
\include{AAAOI332_datasheet} \include{AAAO332_datasheet}
\include{AAAOI333_datasheet} \include{AAAO333_datasheet}
\include{AAAOI432_datasheet} \include{AAAO432_datasheet}
 \include{section-AAAO_complex}
\section{OR-OR-OR-AND(-Invert) Complex Gates}

\input{OOOAI222_manpage.tex}
\input{OOOA222_manpage.tex}
%%  ************    LibreSilicon's StdCellLibrary   *******************
%%
%%  Organisation:   Chipforge
%%                  Germany / European Union
%%
%%  Profile:        Chipforge focus on fine System-on-Chip Cores in
%%                  Verilog HDL Code which are easy understandable and
%%                  adjustable. For further information see
%%                          www.chipforge.org
%%                  there are projects from small cores up to PCBs, too.
%%
%%  File:           StdCellLib/Documents/LaTeX/OOOA322_manpage.tex
%%
%%  Purpose:        Manual Page File for OOOA322
%%
%%  ************    LaTeX with circdia.sty package      ***************
%%
%%  ///////////////////////////////////////////////////////////////////
%%
%%  Copyright (c) 2019 by chipforge <stdcelllib@nospam.chipforge.org>
%%  All rights reserved.
%%
%%      This Standard Cell Library is licensed under the Libre Silicon
%%      public license; you can redistribute it and/or modify it under
%%      the terms of the Libre Silicon public license as published by
%%      the Libre Silicon alliance, either version 1 of the License, or
%%      (at your option) any later version.
%%
%%      This design is distributed in the hope that it will be useful,
%%      but WITHOUT ANY WARRANTY; without even the implied warranty of
%%      MERCHANTABILITY or FITNESS FOR A PARTICULAR PURPOSE.
%%      See the Libre Silicon Public License for more details.
%%
%%  ///////////////////////////////////////////////////////////////////
\label{OOOA322}
\paragraph{Cell}
\begin{quote}
    \textbf{OOOA322} - a 3-2-2-input OR-OR-OR-AND gate
\end{quote}

\paragraph{Synopsys}
\begin{quote}
    OOOA322(Z, C2, C1, C, B1, B, A1, A)
\end{quote}

\paragraph{Description}
\input{OOOA322_circuit.tex}
%\input{OOOA322_schematic.tex}

\paragraph{Truth Table}
%\input{OOOA322_truthtable.tex}

\paragraph{Usage}

\paragraph{Fan-in / Fan-out}

\paragraph{Layout}

\paragraph{Files}

%%  ************    LibreSilicon's StdCellLibrary   *******************
%%
%%  Organisation:   Chipforge
%%                  Germany / European Union
%%
%%  Profile:        Chipforge focus on fine System-on-Chip Cores in
%%                  Verilog HDL Code which are easy understandable and
%%                  adjustable. For further information see
%%                          www.chipforge.org
%%                  there are projects from small cores up to PCBs, too.
%%
%%  File:           StdCellLib/Documents/LaTeX/OOOAI322_manpage.tex
%%
%%  Purpose:        Manual Page File for OOOAI322
%%
%%  ************    LaTeX with circdia.sty package      ***************
%%
%%  ///////////////////////////////////////////////////////////////////
%%
%%  Copyright (c) 2019 by chipforge <stdcelllib@nospam.chipforge.org>
%%  All rights reserved.
%%
%%      This Standard Cell Library is licensed under the Libre Silicon
%%      public license; you can redistribute it and/or modify it under
%%      the terms of the Libre Silicon public license as published by
%%      the Libre Silicon alliance, either version 1 of the License, or
%%      (at your option) any later version.
%%
%%      This design is distributed in the hope that it will be useful,
%%      but WITHOUT ANY WARRANTY; without even the implied warranty of
%%      MERCHANTABILITY or FITNESS FOR A PARTICULAR PURPOSE.
%%      See the Libre Silicon Public License for more details.
%%
%%  ///////////////////////////////////////////////////////////////////
\label{OOOAI322}
\paragraph{Cell}
\begin{quote}
    \textbf{OOOAI322} - a 3-2-2-input OR-OR-OR-AND-Invert gate
\end{quote}

\paragraph{Synopsys}
\begin{quote}
    OOOAI322(Z, C2, C1, C, B1, B, A1, A)
\end{quote}

\paragraph{Description}
\input{OOOAI322_circuit.tex}
\input{OOOAI322_schematic.tex}

\paragraph{Truth Table}
\input{OOOAI322_truthtable.tex}

\paragraph{Usage}

\paragraph{Fan-in / Fan-out}

\paragraph{Layout}

\paragraph{Files}

\paragraph{See also}
\begin{quote}
    OAI222 - a 2-2-2-input OR-AND-Invert gate \\
    OAI332 - a 3-3-2-input OR-AND-Invert gate \\
    OAI333 - a 3-3-3-input OR-AND-Invert gate
\end{quote}

\input{OOOAI332_manpage.tex}
\input{OOOA332_manpage.tex}
\input{OOOAI333_manpage.tex}
%%  ************    LibreSilicon's StdCellLibrary   *******************
%%
%%  Organisation:   Chipforge
%%                  Germany / European Union
%%
%%  Profile:        Chipforge focus on fine System-on-Chip Cores in
%%                  Verilog HDL Code which are easy understandable and
%%                  adjustable. For further information see
%%                          www.chipforge.org
%%                  there are projects from small cores up to PCBs, too.
%%
%%  File:           StdCellLib/Documents/LaTeX/OOOA333_manpage.tex
%%
%%  Purpose:        Manual Page File for OOOA333
%%
%%  ************    LaTeX with circdia.sty package      ***************
%%
%%  ///////////////////////////////////////////////////////////////////
%%
%%  Copyright (c) 2019 by chipforge <stdcelllib@nospam.chipforge.org>
%%  All rights reserved.
%%
%%      This Standard Cell Library is licensed under the Libre Silicon
%%      public license; you can redistribute it and/or modify it under
%%      the terms of the Libre Silicon public license as published by
%%      the Libre Silicon alliance, either version 1 of the License, or
%%      (at your option) any later version.
%%
%%      This design is distributed in the hope that it will be useful,
%%      but WITHOUT ANY WARRANTY; without even the implied warranty of
%%      MERCHANTABILITY or FITNESS FOR A PARTICULAR PURPOSE.
%%      See the Libre Silicon Public License for more details.
%%
%%  ///////////////////////////////////////////////////////////////////
\label{OOOA333}
\paragraph{Cell}
\begin{quote}
    \textbf{OOOA333} - a 3-3-3-input OR-OR-OR-AND gate
\end{quote}

\paragraph{Synopsys}
\begin{quote}
    OOOA333(Z, C2, C1, C, B2, B1, B, A2, A1, A)
\end{quote}

\paragraph{Description}
\input{OOOA333_circuit.tex}
%\input{OOOA333_schematic.tex}

\paragraph{Truth Table}
%\input{OOOA333_truthtable.tex}

\paragraph{Usage}

\paragraph{Fan-in / Fan-out}

\paragraph{Layout}

\paragraph{Files}

%\input{OOOAI432_manpage.tex}
%\input{OOOA432_manpage.tex}
 \include{section-OOOA_complex}

%%  ------------    three phases    -----------------------------------

%%  ************    LibreSilicon's StdCellLibrary   *******************
%%
%%  Organisation:   Chipforge
%%                  Germany / European Union
%%
%%  Profile:        Chipforge focus on fine System-on-Chip Cores in
%%                  Verilog HDL Code which are easy understandable and
%%                  adjustable. For further information see
%%                          www.chipforge.org
%%                  there are projects from small cores up to PCBs, too.
%%
%%  File:           StdCellLib/Documents/LaTeX/section-AOAI_complex.tex
%%
%%  Purpose:        Section Level File for Standard Cell Library Documentation
%%
%%  ************    LaTeX with circdia.sty package      ***************
%%
%%  ///////////////////////////////////////////////////////////////////
%%
%%  Copyright (c) 2018 - 2021 by
%%                  chipforge <stdcelllib@nospam.chipforge.org>
%%  All rights reserved.
%%
%%      This Standard Cell Library is licensed under the Libre Silicon
%%      public license; you can redistribute it and/or modify it under
%%      the terms of the Libre Silicon public license as published by
%%      the Libre Silicon alliance, either version 1 of the License, or
%%      (at your option) any later version.
%%
%%      This design is distributed in the hope that it will be useful,
%%      but WITHOUT ANY WARRANTY; without even the implied warranty of
%%      MERCHANTABILITY or FITNESS FOR A PARTICULAR PURPOSE.
%%      See the Libre Silicon Public License for more details.
%%
\section{AND-OR-AND(-Invert) Complex Gates}

\include{AOAI211_datasheet} \include{AOA211_datasheet}
\include{AOAI212_datasheet} \include{AOA212_datasheet}
\include{AOAI221_datasheet} \include{AOA221_datasheet}
\include{AOAI222_datasheet} \include{AOA222_datasheet}
\include{AOAI232_datasheet} \include{AOA232_datasheet}
\include{AOAI311_datasheet} \include{AOA311_datasheet}
 \include{section-AOA_complex}
\section{OR-AND-OR(-Invert) Complex Gates}

\input{OAOI211_manpage.tex}
%%  ************    LibreSilicon's StdCellLibrary   *******************
%%
%%  Organisation:   Chipforge
%%                  Germany / European Union
%%
%%  Profile:        Chipforge focus on fine System-on-Chip Cores in
%%                  Verilog HDL Code which are easy understandable and
%%                  adjustable. For further information see
%%                          www.chipforge.org
%%                  there are projects from small cores up to PCBs, too.
%%
%%  File:           StdCellLib/Documents/LaTeX/OAO211_manpage.tex
%%
%%  Purpose:        Manual Page File for OAO211
%%
%%  ************    LaTeX with circdia.sty package      ***************
%%
%%  ///////////////////////////////////////////////////////////////////
%%
%%  Copyright (c) 2019 by chipforge <stdcelllib@nospam.chipforge.org>
%%  All rights reserved.
%%
%%      This Standard Cell Library is licensed under the Libre Silicon
%%      public license; you can redistribute it and/or modify it under
%%      the terms of the Libre Silicon public license as published by
%%      the Libre Silicon alliance, either version 1 of the License, or
%%      (at your option) any later version.
%%
%%      This design is distributed in the hope that it will be useful,
%%      but WITHOUT ANY WARRANTY; without even the implied warranty of
%%      MERCHANTABILITY or FITNESS FOR A PARTICULAR PURPOSE.
%%      See the Libre Silicon Public License for more details.
%%
%%  ///////////////////////////////////////////////////////////////////
\label{OAO211}
\paragraph{Cell}
\begin{quote}
    \textbf{OAO211} - a 2-1-1-input OR-AND-OR gate
\end{quote}

\paragraph{Synopsys}
\begin{quote}
    OAO211(Z, C1, C, B, A)
\end{quote}

\paragraph{Description}
\input{OAO211_circuit.tex}
%\input{OAO211_schematic.tex}

\paragraph{Truth Table}
%\input{OAO211_truthtable.tex}

\paragraph{Usage}

\paragraph{Fan-in / Fan-out}

\paragraph{Layout}

\paragraph{Files}

%%  ************    LibreSilicon's StdCellLibrary   *******************
%%
%%  Organisation:   Chipforge
%%                  Germany / European Union
%%
%%  Profile:        Chipforge focus on fine System-on-Chip Cores in
%%                  Verilog HDL Code which are easy understandable and
%%                  adjustable. For further information see
%%                          www.chipforge.org
%%                  there are projects from small cores up to PCBs, too.
%%
%%  File:           StdCellLib/Documents/LaTeX/OAOI212_manpage.tex
%%
%%  Purpose:        Manual Page File for OAOI212
%%
%%  ************    LaTeX with circdia.sty package      ***************
%%
%%  ///////////////////////////////////////////////////////////////////
%%
%%  Copyright (c) 2019 by chipforge <stdcelllib@nospam.chipforge.org>
%%  All rights reserved.
%%
%%      This Standard Cell Library is licensed under the Libre Silicon
%%      public license; you can redistribute it and/or modify it under
%%      the terms of the Libre Silicon public license as published by
%%      the Libre Silicon alliance, either version 1 of the License, or
%%      (at your option) any later version.
%%
%%      This design is distributed in the hope that it will be useful,
%%      but WITHOUT ANY WARRANTY; without even the implied warranty of
%%      MERCHANTABILITY or FITNESS FOR A PARTICULAR PURPOSE.
%%      See the Libre Silicon Public License for more details.
%%
%%  ///////////////////////////////////////////////////////////////////
\label{OAOI212}
\paragraph{Cell}
\begin{quote}
    \textbf{OAOI212} - a 2-1-2-input OR-AND-OR-Invert gate
\end{quote}

\paragraph{Synopsys}
\begin{quote}
    OAOI212(Z, D1, D, C, B, A)
\end{quote}

\paragraph{Description}
\input{OAOI212_circuit.tex}
%\input{OAOI212_schematic.tex}

\paragraph{Truth Table}
%\input{OAOI212_truthtable.tex}

\paragraph{Usage}

\paragraph{Fan-in / Fan-out}

\paragraph{Layout}

\paragraph{Files}

\input{OAO212_manpage.tex}
\input{OAOI221_manpage.tex}
%%  ************    LibreSilicon's StdCellLibrary   *******************
%%
%%  Organisation:   Chipforge
%%                  Germany / European Union
%%
%%  Profile:        Chipforge focus on fine System-on-Chip Cores in
%%                  Verilog HDL Code which are easy understandable and
%%                  adjustable. For further information see
%%                          www.chipforge.org
%%                  there are projects from small cores up to PCBs, too.
%%
%%  File:           StdCellLib/Documents/LaTeX/OAO221_manpage.tex
%%
%%  Purpose:        Manual Page File for OAO221
%%
%%  ************    LaTeX with circdia.sty package      ***************
%%
%%  ///////////////////////////////////////////////////////////////////
%%
%%  Copyright (c) 2019 by chipforge <stdcelllib@nospam.chipforge.org>
%%  All rights reserved.
%%
%%      This Standard Cell Library is licensed under the Libre Silicon
%%      public license; you can redistribute it and/or modify it under
%%      the terms of the Libre Silicon public license as published by
%%      the Libre Silicon alliance, either version 1 of the License, or
%%      (at your option) any later version.
%%
%%      This design is distributed in the hope that it will be useful,
%%      but WITHOUT ANY WARRANTY; without even the implied warranty of
%%      MERCHANTABILITY or FITNESS FOR A PARTICULAR PURPOSE.
%%      See the Libre Silicon Public License for more details.
%%
%%  ///////////////////////////////////////////////////////////////////
\label{OAO221}
\paragraph{Cell}
\begin{quote}
    \textbf{OAO221} - a 2-2-1-input OR-AND-OR gate
\end{quote}

\paragraph{Synopsys}
\begin{quote}
    OAO221(Z, C1, C, B1, B, A)
\end{quote}

\paragraph{Description}
\input{OAO221_circuit.tex}
%\input{OAO221_schematic.tex}

\paragraph{Truth Table}
%\input{OAO221_truthtable.tex}

\paragraph{Usage}

\paragraph{Fan-in / Fan-out}

\paragraph{Layout}

\paragraph{Files}

\input{OAOI222_manpage.tex}
\input{OAO222_manpage.tex}
%%  ************    LibreSilicon's StdCellLibrary   *******************
%%
%%  Organisation:   Chipforge
%%                  Germany / European Union
%%
%%  Profile:        Chipforge focus on fine System-on-Chip Cores in
%%                  Verilog HDL Code which are easy understandable and
%%                  adjustable. For further information see
%%                          www.chipforge.org
%%                  there are projects from small cores up to PCBs, too.
%%
%%  File:           StdCellLib/Documents/LaTeX/OAOI232_manpage.tex
%%
%%  Purpose:        Manual Page File for OAOI232
%%
%%  ************    LaTeX with circdia.sty package      ***************
%%
%%  ///////////////////////////////////////////////////////////////////
%%
%%  Copyright (c) 2019 by chipforge <stdcelllib@nospam.chipforge.org>
%%  All rights reserved.
%%
%%      This Standard Cell Library is licensed under the Libre Silicon
%%      public license; you can redistribute it and/or modify it under
%%      the terms of the Libre Silicon public license as published by
%%      the Libre Silicon alliance, either version 1 of the License, or
%%      (at your option) any later version.
%%
%%      This design is distributed in the hope that it will be useful,
%%      but WITHOUT ANY WARRANTY; without even the implied warranty of
%%      MERCHANTABILITY or FITNESS FOR A PARTICULAR PURPOSE.
%%      See the Libre Silicon Public License for more details.
%%
%%  ///////////////////////////////////////////////////////////////////
\label{OAOI232}
\paragraph{Cell}
\begin{quote}
    \textbf{OAOI232} - a 2-3-2-input OR-AND-OR-Invert gate
\end{quote}

\paragraph{Synopsys}
\begin{quote}
    OAOI232(Y, C1, C, B2, B1, B, A1, A)
\end{quote}

\paragraph{Description}
\input{OAOI232_circuit.tex}
%\input{OAOI232_schematic.tex}

\paragraph{Truth Table}
%\input{OAOI232_truthtable.tex}

\paragraph{Usage}

\paragraph{Fan-in / Fan-out}

\paragraph{Layout}

\paragraph{Files}

\input{OAO232_manpage.tex}
\input{OAOI311_manpage.tex}
%%  ************    LibreSilicon's StdCellLibrary   *******************
%%
%%  Organisation:   Chipforge
%%                  Germany / European Union
%%
%%  Profile:        Chipforge focus on fine System-on-Chip Cores in
%%                  Verilog HDL Code which are easy understandable and
%%                  adjustable. For further information see
%%                          www.chipforge.org
%%                  there are projects from small cores up to PCBs, too.
%%
%%  File:           StdCellLib/Documents/LaTeX/OAO311_manpage.tex
%%
%%  Purpose:        Manual Page File for OAO311
%%
%%  ************    LaTeX with circdia.sty package      ***************
%%
%%  ///////////////////////////////////////////////////////////////////
%%
%%  Copyright (c) 2019 by chipforge <stdcelllib@nospam.chipforge.org>
%%  All rights reserved.
%%
%%      This Standard Cell Library is licensed under the Libre Silicon
%%      public license; you can redistribute it and/or modify it under
%%      the terms of the Libre Silicon public license as published by
%%      the Libre Silicon alliance, either version 1 of the License, or
%%      (at your option) any later version.
%%
%%      This design is distributed in the hope that it will be useful,
%%      but WITHOUT ANY WARRANTY; without even the implied warranty of
%%      MERCHANTABILITY or FITNESS FOR A PARTICULAR PURPOSE.
%%      See the Libre Silicon Public License for more details.
%%
%%  ///////////////////////////////////////////////////////////////////
\label{OAO311}
\paragraph{Cell}
\begin{quote}
    \textbf{OAO311} - a 3-1-1-input OR-AND-OR gate
\end{quote}

\paragraph{Synopsys}
\begin{quote}
    OAO311(Z, C2, C1, C, B, A)
\end{quote}

\paragraph{Description}
\input{OAO311_circuit.tex}
%\input{OAO311_schematic.tex}

\paragraph{Truth Table}
%\input{OAO311_truthtable.tex}

\paragraph{Usage}

\paragraph{Fan-in / Fan-out}

\paragraph{Layout}

\paragraph{Files}

 \include{section-OAO_complex}
\section{AND-OR-OR-AND(-Invert) Complex Gates}

\include{AOOAI212_datasheet} \include{AOOA212_datasheet}
\include{AOOAI222_datasheet} \include{AOOA222_datasheet}
\include{AOOAI223_datasheet} \include{AOOA223_datasheet}
\include{AOOAI232_datasheet} \include{AOOA232_datasheet}
\include{AOOAI233_datasheet} \include{AOOA233_datasheet}
\include{AOOAI234_datasheet} \include{AOOA234_datasheet}
\include{AOOAI312_datasheet} \include{AOOA312_datasheet}

\include{AOOAI2121_datasheet} \include{AOOA2121_datasheet}
\include{AOOAI2221_datasheet} \include{AOOA2221_datasheet}
\include{AOOAI2231_datasheet} \include{AOOA2231_datasheet}
 \include{section-AOOA_complex}
%%  ************    LibreSilicon's StdCellLibrary   *******************
%%
%%  Organisation:   Chipforge
%%                  Germany / European Union
%%
%%  Profile:        Chipforge focus on fine System-on-Chip Cores in
%%                  Verilog HDL Code which are easy understandable and
%%                  adjustable. For further information see
%%                          www.chipforge.org
%%                  there are projects from small cores up to PCBs, too.
%%
%%  File:           StdCellLib/Documents/LaTeX/section-OAAOI_complex.tex
%%
%%  Purpose:        Section Level File for Standard Cell Library Documentation
%%
%%  ************    LaTeX with circdia.sty package      ***************
%%
%%  ///////////////////////////////////////////////////////////////////
%%
%%  Copyright (c) 2018 - 2021 by
%%                  chipforge <stdcelllib@nospam.chipforge.org>
%%  All rights reserved.
%%
%%      This Standard Cell Library is licensed under the Libre Silicon
%%      public license; you can redistribute it and/or modify it under
%%      the terms of the Libre Silicon public license as published by
%%      the Libre Silicon alliance, either version 1 of the License, or
%%      (at your option) any later version.
%%
%%      This design is distributed in the hope that it will be useful,
%%      but WITHOUT ANY WARRANTY; without even the implied warranty of
%%      MERCHANTABILITY or FITNESS FOR A PARTICULAR PURPOSE.
%%      See the Libre Silicon Public License for more details.
%%
\section{OR-AND-AND-OR(-Invert) Complex Gates}

\include{OAAOI212_datasheet} \include{OAAO212_datasheet}
\include{OAAOI222_datasheet} \include{OAAO222_datasheet}
\include{OAAOI223_datasheet} \include{OAAO223_datasheet}
\include{OAAOI232_datasheet} \include{OAAO232_datasheet}
\include{OAAOI233_datasheet} \include{OAAO233_datasheet}
\include{OAAOI234_datasheet} \include{OAAO234_datasheet}
\include{OAAOI312_datasheet} \include{OAAO312_datasheet}

\include{OAAOI2121_datasheet} \include{OAAO2121_datasheet}
\include{OAAOI2221_datasheet} \include{OAAO2221_datasheet}
\include{OAAOI2231_datasheet} \include{OAAO2231_datasheet}
 \include{section-OAAO_complex}
\section{AND-AND-OR-OR-AND(-Invert) Complex Gates}

\input{AAOOAI222_manpage.tex}
\input{AAOOA222_manpage.tex}
 \include{section-AAOOA_complex}
\section{OR-OR-AND-AND-OR(-Invert) Complex Gates}

\input{OOAAOI222_manpage.tex}
\input{OOAAO222_manpage.tex}
 \include{section-OOAAO_complex}
\section{AND-OR-OR-OR-AND(-Invert) Complex Gates}

 \include{section-AOOOA_complex}
\section{OR-AND-AND-AND-OR(-Invert) Complex Gates}

\include{OAAAOI2232_datasheet} \include{OAAAO2232_datasheet}
 \include{section-OAAAO_complex}
\section{AND-AND-OR-AND(-Invert) Complex Gates}

\input{AAOAI221_manpage.tex}
\input{AAOA221_manpage.tex}
\input{AAOAI321_manpage.tex}
\input{AAOA321_manpage.tex}
\input{AAOAI331_manpage.tex}
\input{AAOA331_manpage.tex}
 \include{section-AAOA_complex}
\section{OR-OR-AND-OR(-Invert) Complex Gates}

%%  ************    LibreSilicon's StdCellLibrary   *******************
%%
%%  Organisation:   Chipforge
%%                  Germany / European Union
%%
%%  Profile:        Chipforge focus on fine System-on-Chip Cores in
%%                  Verilog HDL Code which are easy understandable and
%%                  adjustable. For further information see
%%                          www.chipforge.org
%%                  there are projects from small cores up to PCBs, too.
%%
%%  File:           StdCellLib/Documents/LaTeX/OOAOI221_manpage.tex
%%
%%  Purpose:        Manual Page File for OOAOI221
%%
%%  ************    LaTeX with circdia.sty package      ***************
%%
%%  ///////////////////////////////////////////////////////////////////
%%
%%  Copyright (c) 2019 by chipforge <stdcelllib@nospam.chipforge.org>
%%  All rights reserved.
%%
%%      This Standard Cell Library is licensed under the Libre Silicon
%%      public license; you can redistribute it and/or modify it under
%%      the terms of the Libre Silicon public license as published by
%%      the Libre Silicon alliance, either version 1 of the License, or
%%      (at your option) any later version.
%%
%%      This design is distributed in the hope that it will be useful,
%%      but WITHOUT ANY WARRANTY; without even the implied warranty of
%%      MERCHANTABILITY or FITNESS FOR A PARTICULAR PURPOSE.
%%      See the Libre Silicon Public License for more details.
%%
%%  ///////////////////////////////////////////////////////////////////
\label{OOAOI221}
\paragraph{Cell}
\begin{quote}
    \textbf{OOAOI221} - a 2-2-1-input OR-OR-AND-OR-Invert gate
\end{quote}

\paragraph{Synopsys}
\begin{quote}
    OOAOI221(Y, C1, C, B1, B, A)
\end{quote}

\paragraph{Description}
\input{OOAOI221_circuit.tex}
%\input{OOAOI221_schematic.tex}

\paragraph{Truth Table}
%\input{OOAOI221_truthtable.tex}

\paragraph{Usage}

\paragraph{Fan-in / Fan-out}

\paragraph{Layout}

\paragraph{Files}

\clearpage

%%  ************    LibreSilicon's StdCellLibrary   *******************
%%
%%  Organisation:   Chipforge
%%                  Germany / European Union
%%
%%  Profile:        Chipforge focus on fine System-on-Chip Cores in
%%                  Verilog HDL Code which are easy understandable and
%%                  adjustable. For further information see
%%                          www.chipforge.org
%%                  there are projects from small cores up to PCBs, too.
%%
%%  File:           StdCellLib/Documents/LaTeX/OOAO221_manpage.tex
%%
%%  Purpose:        Manual Page File for OOAO221
%%
%%  ************    LaTeX with circdia.sty package      ***************
%%
%%  ///////////////////////////////////////////////////////////////////
%%
%%  Copyright (c) 2019 by chipforge <stdcelllib@nospam.chipforge.org>
%%  All rights reserved.
%%
%%      This Standard Cell Library is licensed under the Libre Silicon
%%      public license; you can redistribute it and/or modify it under
%%      the terms of the Libre Silicon public license as published by
%%      the Libre Silicon alliance, either version 1 of the License, or
%%      (at your option) any later version.
%%
%%      This design is distributed in the hope that it will be useful,
%%      but WITHOUT ANY WARRANTY; without even the implied warranty of
%%      MERCHANTABILITY or FITNESS FOR A PARTICULAR PURPOSE.
%%      See the Libre Silicon Public License for more details.
%%
%%  ///////////////////////////////////////////////////////////////////
\label{OOAO221}
\paragraph{Cell}
\begin{quote}
    \textbf{OOAO221} - a 2-2-1-input OR-OR-AND-OR gate
\end{quote}

\paragraph{Synopsys}
\begin{quote}
    OOAO221(Z, C1, C, B1, B, A)
\end{quote}

\paragraph{Description}
\input{OOAO221_circuit.tex}
%\input{OOAO221_schematic.tex}

\paragraph{Truth Table}
%\input{OOAO221_truthtable.tex}

\paragraph{Usage}

\paragraph{Fan-in / Fan-out}

\paragraph{Layout}

\paragraph{Files}

\clearpage

\input{OOAOI321_manpage.tex}
\input{OOAO321_manpage.tex}
\input{OOAOI331_manpage.tex}
\input{OOAO331_manpage.tex}
 \include{section-OOAO_complex}
\section{AND-AND-AND-OR-AND(-Invert) Complex Gates}

 \include{section-AAAOA_complex}
\section{OR-OR-OR-AND-OR(-Invert) Complex Gates}

 \include{section-OOOAO_complex}

%%  ------------    four phases     -----------------------------------

\section{AND-OR-AND-OR(-Invert) Complex Gates}

\include{AOAOI2111_datasheet} \include{AOAO2111_datasheet}
\include{AOAOI2211_datasheet} \include{AOAO2211_datasheet}
\include{AOAOI3211_datasheet} \include{AOAO3211_datasheet}
 \include{section-AOAO_complex}
%%  ************    LibreSilicon's StdCellLibrary   *******************
%%
%%  Organisation:   Chipforge
%%                  Germany / European Union
%%
%%  Profile:        Chipforge focus on fine System-on-Chip Cores in
%%                  Verilog HDL Code which are easy understandable and
%%                  adjustable. For further information see
%%                          www.chipforge.org
%%                  there are projects from small cores up to PCBs, too.
%%
%%  File:           StdCellLib/Documents/Book/section-OAOAI_complex.tex
%%
%%  Purpose:        Section Level File for Standard Cell Library Documentation
%%
%%  ************    LaTeX with circdia.sty package      ***************
%%
%%  ///////////////////////////////////////////////////////////////////
%%
%%  Copyright (c) 2018 - 2022 by
%%                  chipforge <stdcelllib@nospam.chipforge.org>
%%  All rights reserved.
%%
%%      This Standard Cell Library is licensed under the Libre Silicon
%%      public license; you can redistribute it and/or modify it under
%%      the terms of the Libre Silicon public license as published by
%%      the Libre Silicon alliance, either version 1 of the License, or
%%      (at your option) any later version.
%%
%%      This design is distributed in the hope that it will be useful,
%%      but WITHOUT ANY WARRANTY; without even the implied warranty of
%%      MERCHANTABILITY or FITNESS FOR A PARTICULAR PURPOSE.
%%      See the Libre Silicon Public License for more details.
%%
%%  ///////////////////////////////////////////////////////////////////
\section{OR-AND-OR-AND(-Invert) Complex Gates}

\include{Datasheets/OAOAI2111} \include{Datasheets/OAOA2111}
\include{Datasheets/OAOAI2211} \include{Datasheets/OAOA2211}
\include{Datasheets/OAOAI3211} \include{Datasheets/OAOA3211}
 \include{section-OAOA_complex}
%%  ************    LibreSilicon's StdCellLibrary   *******************
%%
%%  Organisation:   Chipforge
%%                  Germany / European Union
%%
%%  Profile:        Chipforge focus on fine System-on-Chip Cores in
%%                  Verilog HDL Code which are easy understandable and
%%                  adjustable. For further information see
%%                          www.chipforge.org
%%                  there are projects from small cores up to PCBs, too.
%%
%%  File:           StdCellLib/Documents/section-AOAAOI_complex.tex
%%
%%  Purpose:        Section Level File for Standard Cell Library Documentation
%%
%%  ************    LaTeX with circdia.sty package      ***************
%%
%%  ///////////////////////////////////////////////////////////////////
%%
%%  Copyright (c) 2018 - 2022 by
%%                  chipforge <stdcelllib@nospam.chipforge.org>
%%  All rights reserved.
%%
%%      This Standard Cell Library is licensed under the Libre Silicon
%%      public license; you can redistribute it and/or modify it under
%%      the terms of the Libre Silicon public license as published by
%%      the Libre Silicon alliance, either version 1 of the License, or
%%      (at your option) any later version.
%%
%%      This design is distributed in the hope that it will be useful,
%%      but WITHOUT ANY WARRANTY; without even the implied warranty of
%%      MERCHANTABILITY or FITNESS FOR A PARTICULAR PURPOSE.
%%      See the Libre Silicon Public License for more details.
%%
%%  ///////////////////////////////////////////////////////////////////
\section{AND-OR-AND-AND-OR(-Invert) Complex Gates}

\include{Datasheets/AOAAOI2112} \include{Datasheets/AOAAO2112}
\include{Datasheets/AOAAOI2113} \include{Datasheets/AOAAO2113}
\include{Datasheets/AOAAOI2114} \include{Datasheets/AOAAO2114}
\include{Datasheets/AOAAOI2122} \include{Datasheets/AOAAO2122}
\include{Datasheets/AOAAOI2123} \include{Datasheets/AOAAO2123}
\include{Datasheets/AOAAOI3112} \include{Datasheets/AOAAO3112}
\include{Datasheets/AOAAOI3113} \include{Datasheets/AOAAO3113}
\include{Datasheets/AOAAOI3212} \include{Datasheets/AOAAO3212}

\include{Datasheets/AOAAOI21121} \include{Datasheets/AOAAO21121}
\include{Datasheets/AOAAOI21131} \include{Datasheets/AOAAO21131}
\include{Datasheets/AOAAOI21141} \include{Datasheets/AOAAO21141}
\include{Datasheets/AOAAOI21221} \include{Datasheets/AOAAO21221}
\include{Datasheets/AOAAOI31121} \include{Datasheets/AOAAO31121}
 \include{section-AOAAO_complex}
\section{OR-AND-OR-OR-AND(-Invert) Complex Gates}

 \include{section-OAOOA_complex}
\section{AND-OR-AND-AND-AND-OR(-Invert) Complex Gates}

 \include{section-AOAAAO_complex}
\section{OR-AND-OR-OR-OR-AND(-Invert) Complex Gates}

 \include{section-OAOOOA_complex}
%%  ************    LibreSilicon's StdCellLibrary   *******************
%%
%%  Organisation:   Chipforge
%%                  Germany / European Union
%%
%%  Profile:        Chipforge focus on fine System-on-Chip Cores in
%%                  Verilog HDL Code which are easy understandable and
%%                  adjustable. For further information see
%%                          www.chipforge.org
%%                  there are projects from small cores up to PCBs, too.
%%
%%  File:           StdCellLib/Documents/section-AOOAOI_complex.tex
%%
%%  Purpose:        Section Level File for Standard Cell Library Documentation
%%
%%  ************    LaTeX with circdia.sty package      ***************
%%
%%  ///////////////////////////////////////////////////////////////////
%%
%%  Copyright (c) 2018 - 2022 by
%%                  chipforge <stdcelllib@nospam.chipforge.org>
%%  All rights reserved.
%%
%%      This Standard Cell Library is licensed under the Libre Silicon
%%      public license; you can redistribute it and/or modify it under
%%      the terms of the Libre Silicon public license as published by
%%      the Libre Silicon alliance, either version 1 of the License, or
%%      (at your option) any later version.
%%
%%      This design is distributed in the hope that it will be useful,
%%      but WITHOUT ANY WARRANTY; without even the implied warranty of
%%      MERCHANTABILITY or FITNESS FOR A PARTICULAR PURPOSE.
%%      See the Libre Silicon Public License for more details.
%%
%%  ///////////////////////////////////////////////////////////////////
\section{AND-OR-OR-AND-OR(-Invert) Complex Gates}

\include{Datasheets/AOOAOI2121} \include{Datasheets/AOOAO2121}
\include{Datasheets/AOOAOI2122} \include{Datasheets/AOOAO2122}
\include{Datasheets/AOOAOI2131} \include{Datasheets/AOOAO2131}
\include{Datasheets/AOOAOI2221} \include{Datasheets/AOOAO2221}
\include{Datasheets/AOOAOI3121} \include{Datasheets/AOOAO3121}
\include{Datasheets/AOOAOI3122} \include{Datasheets/AOOAO3122}
\include{Datasheets/AOOAOI3131} \include{Datasheets/AOOAO3131}
\include{Datasheets/AOOAOI3221} %%  ************    LibreSilicon's StdCellLibrary   *******************
%%
%%  Organisation:   Chipforge
%%                  Germany / European Union
%%
%%  Profile:        Chipforge focus on fine System-on-Chip Cores in
%%                  Verilog HDL Code which are easy understandable and
%%                  adjustable. For further information see
%%                          www.chipforge.org
%%                  there are projects from small cores up to PCBs, too.
%%
%%  File:           StdCellLib/Documents/Datasheets/Circuitry/AOOAO3221.tex
%%
%%  Purpose:        Circuit File for AOOAO3221
%%
%%  ************    LaTeX with circdia.sty package      ***************
%%
%%  ///////////////////////////////////////////////////////////////////
%%
%%  Copyright (c) 2018 - 2022 by
%%                  chipforge <stdcelllib@nospam.chipforge.org>
%%  All rights reserved.
%%
%%      This Standard Cell Library is licensed under the Libre Silicon
%%      public license; you can redistribute it and/or modify it under
%%      the terms of the Libre Silicon public license as published by
%%      the Libre Silicon alliance, either version 1 of the License, or
%%      (at your option) any later version.
%%
%%      This design is distributed in the hope that it will be useful,
%%      but WITHOUT ANY WARRANTY; without even the implied warranty of
%%      MERCHANTABILITY or FITNESS FOR A PARTICULAR PURPOSE.
%%      See the Libre Silicon Public License for more details.
%%
%%  ///////////////////////////////////////////////////////////////////
\begin{circuitdiagram}[draft]{38}{16}

    \usgate
    % ----  1st column  ----
    \pin{1}{1}{L}{A}
    \pin{1}{3}{L}{A1}
    \pin{1}{5}{L}{A2}
    \gate[\inputs{3}]{and}{5}{3}{R}{}{}

    % ----  2nd column  ----
    \wire{9}{3}{9}{5}
    \pin{8}{7}{L}{B}
    \pin{8}{9}{L}{B1}
    \gate[\inputs{3}]{or}{12}{7}{R}{}{}

    \pin{8}{15}{L}{C1}
    \pin{8}{11}{L}{C}
    \gate[\inputs{2}]{or}{12}{13}{R}{}{}

    % ----  3rd column  ----
    \wire{16}{7}{16}{9}
    \gate[\inputs{2}]{and}{19}{11}{R}{}{}

    % ----  4th column  ----
    \pin{22}{15}{L}{D}
    \gate[\inputs{2}]{nor}{26}{13}{R}{}{}

    % ----  last column ----
    \gate{not}{33}{13}{R}{}{}

    % ----  result ----
    \pin{37}{13}{R}{Z}

\end{circuitdiagram}


\include{Datasheets/AOOAOI21211} \include{Datasheets/AOOAO21211}
\include{Datasheets/AOOAOI21212} \include{Datasheets/AOOAO21212}
\include{Datasheets/AOOAOI21311} \include{Datasheets/AOOAO21311}
\include{Datasheets/AOOAOI22211} \include{Datasheets/AOOAO22211}
 \include{section-AOOAO_complex}
%%  ************    LibreSilicon's StdCellLibrary   *******************
%%
%%  Organisation:   Chipforge
%%                  Germany / European Union
%%
%%  Profile:        Chipforge focus on fine System-on-Chip Cores in
%%                  Verilog HDL Code which are easy understandable and
%%                  adjustable. For further information see
%%                          www.chipforge.org
%%                  there are projects from small cores up to PCBs, too.
%%
%%  File:           StdCellLib/Documents/section-OAAOAI_complex.tex
%%
%%  Purpose:        Section Level File for Standard Cell Library Documentation
%%
%%  ************    LaTeX with circdia.sty package      ***************
%%
%%  ///////////////////////////////////////////////////////////////////
%%
%%  Copyright (c) 2018 - 2022 by
%%                  chipforge <stdcelllib@nospam.chipforge.org>
%%  All rights reserved.
%%
%%      This Standard Cell Library is licensed under the Libre Silicon
%%      public license; you can redistribute it and/or modify it under
%%      the terms of the Libre Silicon public license as published by
%%      the Libre Silicon alliance, either version 1 of the License, or
%%      (at your option) any later version.
%%
%%      This design is distributed in the hope that it will be useful,
%%      but WITHOUT ANY WARRANTY; without even the implied warranty of
%%      MERCHANTABILITY or FITNESS FOR A PARTICULAR PURPOSE.
%%      See the Libre Silicon Public License for more details.
%%
%%  ///////////////////////////////////////////////////////////////////
\section{OR-AND-AND-OR-AND(-Invert) Complex Gates}

\include{Datasheets/OAAOAI2121} \include{Datasheets/OAAOA2121}
\include{Datasheets/OAAOAI2122} \include{Datasheets/OAAOA2122}
\include{Datasheets/OAAOAI2131} \include{Datasheets/OAAOA2131}
\include{Datasheets/OAAOAI2221} \include{Datasheets/OAAOA2221}
\include{Datasheets/OAAOAI2231} \include{Datasheets/OAAOA2231}
\include{Datasheets/OAAOAI3121} \include{Datasheets/OAAOA3121}
\include{Datasheets/OAAOAI3122} \include{Datasheets/OAAOA3122}
\include{Datasheets/OAAOAI3131} \include{Datasheets/OAAOA3131}
\include{Datasheets/OAAOAI3221} \include{Datasheets/OAAOA3221}

\include{Datasheets/OAAOAI21211} \include{Datasheets/OAAOA21211}
%%  ************    LibreSilicon's StdCellLibrary   *******************
%%
%%  Organisation:   Chipforge
%%                  Germany / European Union
%%
%%  Profile:        Chipforge focus on fine System-on-Chip Cores in
%%                  Verilog HDL Code which are easy understandable and
%%                  adjustable. For further information see
%%                          www.chipforge.org
%%                  there are projects from small cores up to PCBs, too.
%%
%%  File:           StdCellLib/Documents/Datasheets/Circuitry/OAAOAI21212.tex
%%
%%  Purpose:        Circuit File for OAAOAI21212
%%
%%  ************    LaTeX with circdia.sty package      ***************
%%
%%  ///////////////////////////////////////////////////////////////////
%%
%%  Copyright (c) 2018 - 2022 by
%%                  chipforge <stdcelllib@nospam.chipforge.org>
%%  All rights reserved.
%%
%%      This Standard Cell Library is licensed under the Libre Silicon
%%      public license; you can redistribute it and/or modify it under
%%      the terms of the Libre Silicon public license as published by
%%      the Libre Silicon alliance, either version 1 of the License, or
%%      (at your option) any later version.
%%
%%      This design is distributed in the hope that it will be useful,
%%      but WITHOUT ANY WARRANTY; without even the implied warranty of
%%      MERCHANTABILITY or FITNESS FOR A PARTICULAR PURPOSE.
%%      See the Libre Silicon Public License for more details.
%%
%%  ///////////////////////////////////////////////////////////////////
\begin{circuitdiagram}[draft]{32}{18}

    \usgate
    % ----  1st column  ----
    \pin{1}{1}{L}{A}
    \pin{1}{5}{L}{A1}
    \gate[\inputs{2}]{or}{5}{3}{R}{}{}

    % ----  2nd column  ----
    \pin{8}{7}{L}{B}
    \gate[\inputs{2}]{and}{12}{5}{R}{}{}

    \pin{8}{9}{L}{C}
    \pin{8}{13}{L}{C1}
    \gate[\inputs{2}]{and}{12}{11}{R}{}{}

    % ----  3rd column  ----
    \wire{16}{5}{16}{9}
    \pin{15}{15}{L}{D}
    \wire{16}{13}{16}{15}
    \gate[\inputs{3}]{or}{19}{11}{R}{}{}

    % ----  4th column  ----
    \wire{23}{11}{23}{13}
    \pin{22}{15}{L}{E}
    \pin{22}{17}{L}{E1}
    \gate[\inputs{3}]{nand}{26}{15}{R}{}{}

    % ----  result ----
    \pin{31}{15}{R}{Y}

\end{circuitdiagram}
 \include{Datasheets/OAAOA21212}
\include{Datasheets/OAAOAI21311} \include{Datasheets/OAAOA21311}
\include{Datasheets/OAAOAI22211} \include{Datasheets/OAAOA22211}
 \include{section-OAAOA_complex}
%%  ************    LibreSilicon's StdCellLibrary   *******************
%%
%%  Organisation:   Chipforge
%%                  Germany / European Union
%%
%%  Profile:        Chipforge focus on fine System-on-Chip Cores in
%%                  Verilog HDL Code which are easy understandable and
%%                  adjustable. For further information see
%%                          www.chipforge.org
%%                  there are projects from small cores up to PCBs, too.
%%
%%  File:           StdCellLib/Documents/section-AAOAAOI_complex.tex
%%
%%  Purpose:        Section Level File for Standard Cell Library Documentation
%%
%%  ************    LaTeX with circdia.sty package      ***************
%%
%%  ///////////////////////////////////////////////////////////////////
%%
%%  Copyright (c) 2018 - 2022 by
%%                  chipforge <stdcelllib@nospam.chipforge.org>
%%  All rights reserved.
%%
%%      This Standard Cell Library is licensed under the Libre Silicon
%%      public license; you can redistribute it and/or modify it under
%%      the terms of the Libre Silicon public license as published by
%%      the Libre Silicon alliance, either version 1 of the License, or
%%      (at your option) any later version.
%%
%%      This design is distributed in the hope that it will be useful,
%%      but WITHOUT ANY WARRANTY; without even the implied warranty of
%%      MERCHANTABILITY or FITNESS FOR A PARTICULAR PURPOSE.
%%      See the Libre Silicon Public License for more details.
%%
%%  ///////////////////////////////////////////////////////////////////
\section{AND-AND-OR-AND-AND-OR(-Invert) Complex Gates}

\include{Datasheets/AAOAAOI2212} \include{Datasheets/AAOAAO2212}
\include{Datasheets/AAOAAOI2213} \include{Datasheets/AAOAAO2213}
\include{Datasheets/AAOAAOI2214} \include{Datasheets/AAOAAO2214}
\include{Datasheets/AAOAAOI2222} \include{Datasheets/AAOAAO2222}
\include{Datasheets/AAOAAOI2223} \include{Datasheets/AAOAAO2223}
\include{Datasheets/AAOAAOI3213} \include{Datasheets/AAOAAO3213}
\include{Datasheets/AAOAAOI3312} \include{Datasheets/AAOAAO3312}

\include{Datasheets/AAOAAOI22112} \include{Datasheets/AAOAAO22112}
\include{Datasheets/AAOAAOI22113} \include{Datasheets/AAOAAO22113}
\include{Datasheets/AAOAAOI22121} \include{Datasheets/AAOAAO22121}
\include{Datasheets/AAOAAOI22122} \include{Datasheets/AAOAAO22122}
\include{Datasheets/AAOAAOI22131} \include{Datasheets/AAOAAO22131}
\include{Datasheets/AAOAAOI22141} \include{Datasheets/AAOAAO22141}
\include{Datasheets/AAOAAOI22221} \include{Datasheets/AAOAAO22221}
\include{Datasheets/AAOAAOI32112} \include{Datasheets/AAOAAO32112}
\include{Datasheets/AAOAAOI32121} \include{Datasheets/AAOAAO32121}
 \include{section-AAOAAO_complex}
%%  ************    LibreSilicon's StdCellLibrary   *******************
%%
%%  Organisation:   Chipforge
%%                  Germany / European Union
%%
%%  Profile:        Chipforge focus on fine System-on-Chip Cores in
%%                  Verilog HDL Code which are easy understandable and
%%                  adjustable. For further information see
%%                          www.chipforge.org
%%                  there are projects from small cores up to PCBs, too.
%%
%%  File:           StdCellLib/Documents/section-OOAOOAI_complex.tex
%%
%%  Purpose:        Section Level File for Standard Cell Library Documentation
%%
%%  ************    LaTeX with circdia.sty package      ***************
%%
%%  ///////////////////////////////////////////////////////////////////
%%
%%  Copyright (c) 2018 - 2022 by
%%                  chipforge <stdcelllib@nospam.chipforge.org>
%%  All rights reserved.
%%
%%      This Standard Cell Library is licensed under the Libre Silicon
%%      public license; you can redistribute it and/or modify it under
%%      the terms of the Libre Silicon public license as published by
%%      the Libre Silicon alliance, either version 1 of the License, or
%%      (at your option) any later version.
%%
%%      This design is distributed in the hope that it will be useful,
%%      but WITHOUT ANY WARRANTY; without even the implied warranty of
%%      MERCHANTABILITY or FITNESS FOR A PARTICULAR PURPOSE.
%%      See the Libre Silicon Public License for more details.
%%
%%  ///////////////////////////////////////////////////////////////////
\section{OR-OR-AND-OR-OR-AND(-Invert) Complex Gates}

\include{Datasheets/OOAOOAI2212} \include{Datasheets/OOAOOA2212}
\include{Datasheets/OOAOOAI2213} \include{Datasheets/OOAOOA2213}
\include{Datasheets/OOAOOAI2214} \include{Datasheets/OOAOOA2214}
\include{Datasheets/OOAOOAI2222} \include{Datasheets/OOAOOA2222}
\include{Datasheets/OOAOOAI3212} \include{Datasheets/OOAOOA3212}

\include{Datasheets/OOAOOAI22112} \include{Datasheets/OOAOOA22112}
\include{Datasheets/OOAOOAI22121} \include{Datasheets/OOAOOA22121}
\include{Datasheets/OOAOOAI22131} \include{Datasheets/OOAOOA22131}
 \include{section-OOAOOA_complex}
\section{AND-OR-OR-AND-AND-OR(-Invert) Complex Gates}

 \include{section-AOOAAO_complex}
\section{OR-AND-AND-OR-OR-AND(-Invert) Complex Gates}

 \include{section-OAAOOA_complex}

%%  ------------    five phases     -----------------------------------

\section{OR-AND-OR-AND-OR(-Invert) Complex Gates}

 \include{section-OAOAO_complex}
\section{AND-OR-AND-OR-AND(-Invert) Complex Gates}

 \include{section-AOAOA_complex}
%%  ************    LibreSilicon's StdCellLibrary   *******************
%%
%%  Organisation:   Chipforge
%%                  Germany / European Union
%%
%%  Profile:        Chipforge focus on fine System-on-Chip Cores in
%%                  Verilog HDL Code which are easy understandable and
%%                  adjustable. For further information see
%%                          www.chipforge.org
%%                  there are projects from small cores up to PCBs, too.
%%
%%  File:           StdCellLib/Documents/section-AOAOOAI_complex.tex
%%
%%  Purpose:        Section Level File for Standard Cell Library Documentation
%%
%%  ************    LaTeX with circdia.sty package      ***************
%%
%%  ///////////////////////////////////////////////////////////////////
%%
%%  Copyright (c) 2018 - 2022 by
%%                  chipforge <stdcelllib@nospam.chipforge.org>
%%  All rights reserved.
%%
%%      This Standard Cell Library is licensed under the Libre Silicon
%%      public license; you can redistribute it and/or modify it under
%%      the terms of the Libre Silicon public license as published by
%%      the Libre Silicon alliance, either version 1 of the License, or
%%      (at your option) any later version.
%%
%%      This design is distributed in the hope that it will be useful,
%%      but WITHOUT ANY WARRANTY; without even the implied warranty of
%%      MERCHANTABILITY or FITNESS FOR A PARTICULAR PURPOSE.
%%      See the Libre Silicon Public License for more details.
%%
%%  ///////////////////////////////////////////////////////////////////
\section{AND-OR-AND-OR-OR-AND(-Invert) Complex Gates}

\include{Datasheets/AOAOOAI21112} \include{Datasheets/AOAOOA21112}
%%  ************    LibreSilicon's StdCellLibrary   *******************
%%
%%  Organisation:   Chipforge
%%                  Germany / European Union
%%
%%  Profile:        Chipforge focus on fine System-on-Chip Cores in
%%                  Verilog HDL Code which are easy understandable and
%%                  adjustable. For further information see
%%                          www.chipforge.org
%%                  there are projects from small cores up to PCBs, too.
%%
%%  File:           StdCellLib/Documents/Datasheets/Circuitry/AOAOOAI21113.tex
%%
%%  Purpose:        Circuit File for AOAOOAI21113
%%
%%  ************    LaTeX with circdia.sty package      ***************
%%
%%  ///////////////////////////////////////////////////////////////////
%%
%%  Copyright (c) 2018 - 2022 by
%%                  chipforge <stdcelllib@nospam.chipforge.org>
%%  All rights reserved.
%%
%%      This Standard Cell Library is licensed under the Libre Silicon
%%      public license; you can redistribute it and/or modify it under
%%      the terms of the Libre Silicon public license as published by
%%      the Libre Silicon alliance, either version 1 of the License, or
%%      (at your option) any later version.
%%
%%      This design is distributed in the hope that it will be useful,
%%      but WITHOUT ANY WARRANTY; without even the implied warranty of
%%      MERCHANTABILITY or FITNESS FOR A PARTICULAR PURPOSE.
%%      See the Libre Silicon Public License for more details.
%%
%%  ///////////////////////////////////////////////////////////////////
\begin{circuitdiagram}[draft]{39}{18}

    \usgate
    % ----  1st column  ----
    \pin{1}{1}{L}{A}
    \pin{1}{5}{L}{A1}
    \gate[\inputs{2}]{and}{5}{3}{R}{}{}

    % ----  2nd column  ----
    \pin{8}{7}{L}{B}
    \gate[\inputs{2}]{or}{12}{5}{R}{}{}

    % ----  3rd column  ----
    \pin{15}{9}{L}{C}
    \gate[\inputs{2}]{and}{19}{7}{R}{}{}

    % ----  4th column  ----
    \pin{22}{11}{L}{D}
    \gate[\inputs{2}]{or}{26}{9}{R}{}{}

    \pin{22}{13}{L}{E}
    \pin{22}{15}{L}{E1}
    \pin{22}{17}{L}{E2}
    \gate[\inputs{3}]{or}{26}{15}{R}{}{}

    % ----  5th column  ----
    \wire{30}{9}{30}{11}
    \gate[\inputs{2}]{nand}{33}{13}{R}{}{}

    % ----  result ----
    \pin{38}{13}{R}{Y}

\end{circuitdiagram}
 \include{Datasheets/AOAOOA21113}
\include{Datasheets/AOAOOAI21114} \include{Datasheets/AOAOOA21114}
\include{Datasheets/AOAOOAI21122} \include{Datasheets/AOAOOA21122}
\include{Datasheets/AOAOOAI21123} \include{Datasheets/AOAOOA21123}
 \include{section-AOAOOA_complex}
\include{section-OAOAAOI_complex} \include{section-OAOAAO_complex}
\section{AND-OR-OR-AND-OR-OR-AND(-Invert) Complex Gates}

 \include{section-AOOAOOA_complex}
\section{OR-AND-AND-OR-AND-AND-OR(-Invert) Complex Gates}

 \include{section-OAAOAAO_complex}
\section{AND-OR-AND-AND-OR-OR-AND(-Invert) Complex Gates}

 \include{section-AOAAOOA_complex}
\section{OR-AND-OR-OR-AND-AND-OR(-Invert) Complex Gates}

 \include{section-OAOOAAO_complex}
\section{AND-AND-OR-AND-AND-OR-OR-AND(-Invert) Complex Gates}

 \include{section-AAOAAOOA_complex}
\include{section-OOAOOAAOI_complex} \include{section-OOAOOAAO_complex}
\section{AND-AND-OR-AND-OR-OR-AND(-Invert) Complex Gates}

 \include{section-AAOAOOA_complex}
\section{OR-OR-AND-OR-AND-AND-OR(-Invert) Complex Gates}

 \include{section-OOAOAAO_complex}

%%  ------------    six phases      -----------------------------------

\section{OR-AND-OR-AND-OR-AND(-Invert) Complex Gates}

\include{OAOAOAI211111_datasheet} \include{OAOAOA211111_datasheet}
 \include{section-OAOAOA_complex}
\section{AND-OR-AND-OR-AND-OR(-Invert) Complex Gates}

\include{AOAOAOI211111_datasheet} \include{AOAOAO211111_datasheet}
 \include{section-AOAOAO_complex}
\section{OR-AND-OR-AND-OR-OR-AND(-Invert) Complex Gates}

 \include{section-OAOAOOA_complex}
\section{AND-OR-AND-OR-AND-AND-OR(-Invert) Complex Gates}

 \include{section-AOAOAAO_complex}
\section{OR-AND-AND-OR-AND-OR-OR-AND(-Invert) Complex Gates}

 \include{section-OAAOAOOA_complex}
\include{section-AOOAOAAOI_complex} \include{section-AOOAOAAO_complex}
\include{section-OAOOAOOAI_complex} \include{section-OAOOAOOA_complex}
\include{section-AOAAOAAOI_complex} \include{section-AOAAOAAO_complex}
\include{section-OAOAAOOAI_complex} \include{section-OAOAAOOA_complex}
\section{AND-OR-AND-OR-OR-AND-AND-OR(-Invert) Complex Gates}

 \include{section-AOAOOAAO_complex}
\section{OR-AND-AND-OR-AND-AND-OR-OR-AND(-Invert) Complex Gates}

 \include{section-OAAOAAOOA_complex}
\include{section-AOOAOOAAOI_complex} \include{section-AOOAOOAAO_complex}
\include{section-OOAAOAOOAI_complex} \include{section-OOAAOAOOA_complex}
\section{AND-AND-OR-OR-AND-OR-AND-AND-OR(-Invert) Complex Gates}

 \include{section-AAOOAOAAO_complex}
\section{OR-OR-AND-AND-OR-AND-AND-OR-OR-AND(-Invert) Complex Gates}

 \include{section-OOAAOAAOOA_complex}
\section{AND-AND-OR-OR-AND-OR-OR-AND-AND-OR(-Invert) Complex Gates}

 \include{section-AAOOAOOAAO_complex}

%%  ------------    miscellaneous   -----------------------------------

\section{Miscellaneous Cells}

\input{MAJI23_manpage.tex}


%%  -------------------------------------------------------------------
%%                  CHAPTER 3
%%  -------------------------------------------------------------------

\chapter{Storage Cells}

\section{D-type Latches}

\input{LATN_manpage.tex}
\input{LATP_manpage.tex}
%%  ************    LibreSilicon's StdCellLibrary   *******************
%%
%%  Organisation:   Chipforge
%%                  Germany / European Union
%%
%%  Profile:        Chipforge focus on fine System-on-Chip Cores in
%%                  Verilog HDL Code which are easy understandable and
%%                  adjustable. For further information see
%%                          www.chipforge.org
%%                  there are projects from small cores up to PCBs, too.
%%
%%  File:           StdCellLib/Documents/LaTeX/LATRN_manpage.tex
%%
%%  Purpose:        Manual Page File for LATRN
%%
%%  ************    LaTeX with circdia.sty package      ***************
%%
%%  ///////////////////////////////////////////////////////////////////
%%
%%  Copyright (c) 2019 by chipforge <stdcelllib@nospam.chipforge.org>
%%  All rights reserved.
%%
%%      This Standard Cell Library is licensed under the Libre Silicon
%%      public license; you can redistribute it and/or modify it under
%%      the terms of the Libre Silicon public license as published by
%%      the Libre Silicon alliance, either version 1 of the License, or
%%      (at your option) any later version.
%%
%%      This design is distributed in the hope that it will be useful,
%%      but WITHOUT ANY WARRANTY; without even the implied warranty of
%%      MERCHANTABILITY or FITNESS FOR A PARTICULAR PURPOSE.
%%      See the Libre Silicon Public License for more details.
%%
%%  ///////////////////////////////////////////////////////////////////
\label{LATRN}
\paragraph{Cell}
\begin{quote}
    \textbf{LATRN} - a Low-active D-Latch with high-active asynchronous Reset
\end{quote}

\paragraph{Synopsys}
\begin{quote}
    LATRN(Q, D, R, XN)
\end{quote}

\paragraph{Description}
\input{LATRN_circuit.tex}
%\input{LATRN_schematic.tex}

\paragraph{Truth Table}
%\input{LATRN_truthtable.tex}

\paragraph{Usage}

\paragraph{Fan-in / Fan-out}

\paragraph{Layout}

\paragraph{Files}

%%  ************    LibreSilicon's StdCellLibrary   *******************
%%
%%  Organisation:   Chipforge
%%                  Germany / European Union
%%
%%  Profile:        Chipforge focus on fine System-on-Chip Cores in
%%                  Verilog HDL Code which are easy understandable and
%%                  adjustable. For further information see
%%                          www.chipforge.org
%%                  there are projects from small cores up to PCBs, too.
%%
%%  File:           StdCellLib/Documents/LaTeX/LATRP_manpage.tex
%%
%%  Purpose:        Manual Page File for LATRP
%%
%%  ************    LaTeX with circdia.sty package      ***************
%%
%%  ///////////////////////////////////////////////////////////////////
%%
%%  Copyright (c) 2019 by chipforge <stdcelllib@nospam.chipforge.org>
%%  All rights reserved.
%%
%%      This Standard Cell Library is licensed under the Libre Silicon
%%      public license; you can redistribute it and/or modify it under
%%      the terms of the Libre Silicon public license as published by
%%      the Libre Silicon alliance, either version 1 of the License, or
%%      (at your option) any later version.
%%
%%      This design is distributed in the hope that it will be useful,
%%      but WITHOUT ANY WARRANTY; without even the implied warranty of
%%      MERCHANTABILITY or FITNESS FOR A PARTICULAR PURPOSE.
%%      See the Libre Silicon Public License for more details.
%%
%%  ///////////////////////////////////////////////////////////////////
\label{LATRP}
\paragraph{Cell}
\begin{quote}
    \textbf{LATRP} - a High-active D-Latch with high-active asynchronous Reset
\end{quote}

\paragraph{Synopsys}
\begin{quote}
    LATRP(Q, D, R, X)
\end{quote}

\paragraph{Description}
\input{LATRP_circuit.tex}
%\input{LATRP_schematic.tex}

\paragraph{Truth Table}
%\input{LATRP_truthtable.tex}

\paragraph{Usage}

\paragraph{Fan-in / Fan-out}

\paragraph{Layout}

\paragraph{Files}

%%  ************    LibreSilicon's StdCellLibrary   *******************
%%
%%  Organisation:   Chipforge
%%                  Germany / European Union
%%
%%  Profile:        Chipforge focus on fine System-on-Chip Cores in
%%                  Verilog HDL Code which are easy understandable and
%%                  adjustable. For further information see
%%                          www.chipforge.org
%%                  there are projects from small cores up to PCBs, too.
%%
%%  File:           StdCellLib/Documents/LaTeX/LATSN_manpage.tex
%%
%%  Purpose:        Manual Page File for LATSN
%%
%%  ************    LaTeX with circdia.sty package      ***************
%%
%%  ///////////////////////////////////////////////////////////////////
%%
%%  Copyright (c) 2019 by chipforge <stdcelllib@nospam.chipforge.org>
%%  All rights reserved.
%%
%%      This Standard Cell Library is licensed under the Libre Silicon
%%      public license; you can redistribute it and/or modify it under
%%      the terms of the Libre Silicon public license as published by
%%      the Libre Silicon alliance, either version 1 of the License, or
%%      (at your option) any later version.
%%
%%      This design is distributed in the hope that it will be useful,
%%      but WITHOUT ANY WARRANTY; without even the implied warranty of
%%      MERCHANTABILITY or FITNESS FOR A PARTICULAR PURPOSE.
%%      See the Libre Silicon Public License for more details.
%%
%%  ///////////////////////////////////////////////////////////////////
\label{LATSN}
\paragraph{Cell}
\begin{quote}
    \textbf{LATSN} - a Low-active D-Latch with low-active asynchronous Set
\end{quote}

\paragraph{Synopsys}
\begin{quote}
    LATSN(Q, D, SN, XN)
\end{quote}

\paragraph{Description}
\input{LATSN_circuit.tex}
%\input{LATSN_schematic.tex}

\paragraph{Truth Table}
%\input{LATSN_truthtable.tex}

\paragraph{Usage}

\paragraph{Fan-in / Fan-out}

\paragraph{Layout}

\paragraph{Files}

%%  ************    LibreSilicon's StdCellLibrary   *******************
%%
%%  Organisation:   Chipforge
%%                  Germany / European Union
%%
%%  Profile:        Chipforge focus on fine System-on-Chip Cores in
%%                  Verilog HDL Code which are easy understandable and
%%                  adjustable. For further information see
%%                          www.chipforge.org
%%                  there are projects from small cores up to PCBs, too.
%%
%%  File:           StdCellLib/Documents/LaTeX/LATSP_manpage.tex
%%
%%  Purpose:        Manual Page File for LATSP
%%
%%  ************    LaTeX with circdia.sty package      ***************
%%
%%  ///////////////////////////////////////////////////////////////////
%%
%%  Copyright (c) 2019 by chipforge <stdcelllib@nospam.chipforge.org>
%%  All rights reserved.
%%
%%      This Standard Cell Library is licensed under the Libre Silicon
%%      public license; you can redistribute it and/or modify it under
%%      the terms of the Libre Silicon public license as published by
%%      the Libre Silicon alliance, either version 1 of the License, or
%%      (at your option) any later version.
%%
%%      This design is distributed in the hope that it will be useful,
%%      but WITHOUT ANY WARRANTY; without even the implied warranty of
%%      MERCHANTABILITY or FITNESS FOR A PARTICULAR PURPOSE.
%%      See the Libre Silicon Public License for more details.
%%
%%  ///////////////////////////////////////////////////////////////////
\label{LATSP}
\paragraph{Cell}
\begin{quote}
    \textbf{LATSP} - a High-active D-Latch with low-active asynchronous Set
\end{quote}

\paragraph{Synopsys}
\begin{quote}
    LATSP(Q, D, SN, X)
\end{quote}

\paragraph{Description}
\input{LATSP_circuit.tex}
%\input{LATSP_schematic.tex}

\paragraph{Truth Table}
%\input{LATSP_truthtable.tex}

\paragraph{Usage}

\paragraph{Fan-in / Fan-out}

\paragraph{Layout}

\paragraph{Files}

\input{LATEN_manpage.tex}
\input{LATEP_manpage.tex}
\input{LATERN_manpage.tex}
\input{LATERP_manpage.tex}
%%  ************    LibreSilicon's StdCellLibrary   *******************
%%
%%  Organisation:   Chipforge
%%                  Germany / European Union
%%
%%  Profile:        Chipforge focus on fine System-on-Chip Cores in
%%                  Verilog HDL Code which are easy understandable and
%%                  adjustable. For further information see
%%                          www.chipforge.org
%%                  there are projects from small cores up to PCBs, too.
%%
%%  File:           StdCellLib/Documents/LaTeX/LATESN_manpage.tex
%%
%%  Purpose:        Manual Page File for LATESN
%%
%%  ************    LaTeX with circdia.sty package      ***************
%%
%%  ///////////////////////////////////////////////////////////////////
%%
%%  Copyright (c) 2019 by chipforge <stdcelllib@nospam.chipforge.org>
%%  All rights reserved.
%%
%%      This Standard Cell Library is licensed under the Libre Silicon
%%      public license; you can redistribute it and/or modify it under
%%      the terms of the Libre Silicon public license as published by
%%      the Libre Silicon alliance, either version 1 of the License, or
%%      (at your option) any later version.
%%
%%      This design is distributed in the hope that it will be useful,
%%      but WITHOUT ANY WARRANTY; without even the implied warranty of
%%      MERCHANTABILITY or FITNESS FOR A PARTICULAR PURPOSE.
%%      See the Libre Silicon Public License for more details.
%%
%%  ///////////////////////////////////////////////////////////////////
\label{LATESN}
\paragraph{Cell}
\begin{quote}
    \textbf{LATESN} - a Low-active D-Latch with low-active Clock Enable and low-active asynchronous Set
\end{quote}

\paragraph{Synopsys}
\begin{quote}
    LATESN(Q, D, SN, EN, XN)
\end{quote}

\paragraph{Description}
\input{LATESN_circuit.tex}
%\input{LATESN_schematic.tex}

\paragraph{Truth Table}
%\input{LATESN_truthtable.tex}

\paragraph{Usage}

\paragraph{Fan-in / Fan-out}

\paragraph{Layout}

\paragraph{Files}

\input{LATESP_manpage.tex}

\section{D-type Flip-Flops}

\include{DFFN_datasheet}
\include{DFFP_datasheet}
\include{DFFRN_datasheet}
\include{DFFRP_datasheet}
\include{DFFSN_datasheet}
\include{DFFSP_datasheet}
\include{DFFEN_datasheet}
\include{DFFEP_datasheet}
\include{DFFERN_datasheet}
\include{DFFERP_datasheet}
\include{DFFESN_datasheet}
\include{DFFESP_datasheet}



%%  -------------------------------------------------------------------
%%                  CHAPTER 5
%%  -------------------------------------------------------------------

\chapter{Clock Distribution Cells}

\section{Clock Inverter}

%\input{CIN2_manpage}
%\input{CIP2_manpage}

\section{Clock Buffers}

\include{CBN2_datasheet}
\include{CBP2_datasheet}


%%  -------------------------------------------------------------------
%%                  CHAPTER 6
%%  -------------------------------------------------------------------

\chapter{Physical Cells}

%%  ------------    logical also    -----------------------------------

\section{Tie Cells}

\input{TIE0_manpage.tex}
\input{TIE1_manpage.tex}


%%  ------------    core layout     -----------------------------------

\section{Filler Cells}

\include{FILL1_datasheet}


%%  ------------    pad ring    ---------------------------------------

\section{Pad Ring Cells}

%VDDIO \\
%GND \\
%ANA



%%  -------------------------------------------------------------------
%%                  PART III
%%  -------------------------------------------------------------------

%%  ************    LibreSilicon's StdCellLibrary   *******************
%%
%%  Organisation:   Chipforge
%%                  Germany / European Union
%%
%%  Profile:        Chipforge focus on fine System-on-Chip Cores in
%%                  Verilog HDL Code which are easy understandable and
%%                  adjustable. For further information see
%%                          www.chipforge.org
%%                  there are projects from small cores up to PCBs, too.
%%
%%  File:           StdCellLib/Documents/Book/part-macros.tex
%%
%%  Purpose:        Part Level File for Macro Examples
%%
%%  ************    LaTeX with circdia.sty package      ***************
%%
%%  ///////////////////////////////////////////////////////////////////
%%
%%  Copyright (c) 2021 - 2022 by
%%                chipforge <stdcelllib@nospam.chipforge.org>
%%  All rights reserved.
%%
%%      This Standard Cell Library is licensed under the Libre Silicon
%%      public license; you can redistribute it and/or modify it under
%%      the terms of the Libre Silicon public license as published by
%%      the Libre Silicon alliance, either version 1 of the License, or
%%      (at your option) any later version.
%%
%%      This design is distributed in the hope that it will be useful,
%%      but WITHOUT ANY WARRANTY; without even the implied warranty of
%%      MERCHANTABILITY or FITNESS FOR A PARTICULAR PURPOSE.
%%      See the Libre Silicon Public License for more details.
%%
%%  ///////////////////////////////////////////////////////////////////
\part{Macrocell Examples}
\pagestyle{headings}

%%  -------------------------------------------------------------------
%%                  CHAPTER 1
%%  -------------------------------------------------------------------

\chapter{Combinatorial Macrocells}

\section{Exclusive-(N)OR Macros}

\include{EQ2_datasheet}
\include{XOR2_datasheet}


%%  -------------------------------------------------------------------
%%                  CHAPTER 2
%%  -------------------------------------------------------------------

\chapter{Clock Distribution Macrocells}

\input{section-CG_macros}

%%  -------------------------------------------------------------------
%%                  CHAPTER 3
%%  -------------------------------------------------------------------

\chapter{Other Macrocells}

\section{Design-for-Test Macros}

\include{SDFFN_datasheet}
\include{SDFFP_datasheet}




%%  -------------------------------------------------------------------
%%                  PART IV
%%  -------------------------------------------------------------------

\input{part-appendix}

\end{document}
