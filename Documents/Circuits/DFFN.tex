%%  ************    LibreSilicon's StdCellLibrary   *******************
%%
%%  Organisation:   Chipforge
%%                  Germany / European Union
%%
%%  Profile:        Chipforge focus on fine System-on-Chip Cores in
%%                  Verilog HDL Code which are easy understandable and
%%                  adjustable. For further information see
%%                          www.chipforge.org
%%                  there are projects from small cores up to PCBs, too.
%%
%%  File:           StdCellLib/Documents/Circuits/DFFN.tex
%%
%%  Purpose:        Circuit File for DFFN
%%
%%  ************    LaTeX with circdia.sty package      ***************
%%
%%  ///////////////////////////////////////////////////////////////////
%%
%%  Copyright (c) 2019 by chipforge <stdcelllib@nospam.chipforge.org>
%%  All rights reserved.
%%
%%      This Standard Cell Library is licensed under the Libre Silicon
%%      public license; you can redistribute it and/or modify it under
%%      the terms of the Libre Silicon public license as published by
%%      the Libre Silicon alliance, either version 1 of the License, or
%%      (at your option) any later version.
%%
%%      This design is distributed in the hope that it will be useful,
%%      but WITHOUT ANY WARRANTY; without even the implied warranty of
%%      MERCHANTABILITY or FITNESS FOR A PARTICULAR PURPOSE.
%%      See the Libre Silicon Public License for more details.
%%
%%  ///////////////////////////////////////////////////////////////////
\begin{center}
    Circuit
    \begin{figure}[h]
        \begin{center}
            \begin{circuitdiagram}{24}{10}
            \pin{1}{1}{L}{XN}  % pin XN
            \wire{2}{1}{13}{1}
            \junct{3}{1}
            \wire{3}{1}{3}{6}
            \wire{3}{6}{4}{6}
            \wire{13}{1}{13}{6}
            \wire{13}{6}{14}{6}
            \pin{1}{8}{L}{D}   % pin D
            \wire{2}{8}{4}{8}
            \usgate
            \flipflop[\clockin{p}]{d}{8}{6}{R}{}{}
            \wire{12}{8}{14}{8}
            \flipflop[\clockin{n}]{d}{18}{6}{R}{}{}
            \pin{23}{8}{R}{Q}  % pin Q
            \end{circuitdiagram}
        \end{center}
    \end{figure}
\end{center}
