%%  ************    LibreSilicon's StdCellLibrary   *******************
%%
%%  Organisation:   Chipforge
%%                  Germany / European Union
%%
%%  Profile:        Chipforge focus on fine System-on-Chip Cores in
%%                  Verilog HDL Code which are easy understandable and
%%                  adjustable. For further information see
%%                          www.chipforge.org
%%                  there are projects from small cores up to PCBs, too.
%%
%%  File:           StdCellLib/Documents/LaTeX/LATN_manpage.tex
%%
%%  Purpose:        Manual Page File for LATN
%%
%%  ************    LaTeX with circdia.sty package      ***************
%%
%%  ///////////////////////////////////////////////////////////////////
%%
%%  Copyright (c) 2019 by chipforge <stdcelllib@nospam.chipforge.org>
%%  All rights reserved.
%%
%%      This Standard Cell Library is licensed under the Libre Silicon
%%      public license; you can redistribute it and/or modify it under
%%      the terms of the Libre Silicon public license as published by
%%      the Libre Silicon alliance, either version 1 of the License, or
%%      (at your option) any later version.
%%
%%      This design is distributed in the hope that it will be useful,
%%      but WITHOUT ANY WARRANTY; without even the implied warranty of
%%      MERCHANTABILITY or FITNESS FOR A PARTICULAR PURPOSE.
%%      See the Libre Silicon Public License for more details.
%%
%%  ///////////////////////////////////////////////////////////////////
\label{LATN}
\paragraph{Cell}
\begin{quote}
    \textbf{LATN} - a Low-active D-Latch
\end{quote}

\paragraph{Synopsys}
\begin{quote}
    LATN(Q, D, XN)
\end{quote}

\paragraph{Description}
%%  ************    LibreSilicon's StdCellLibrary   *******************
%%
%%  Organisation:   Chipforge
%%                  Germany / European Union
%%
%%  Profile:        Chipforge focus on fine System-on-Chip Cores in
%%                  Verilog HDL Code which are easy understandable and
%%                  adjustable. For further information see
%%                          www.chipforge.org
%%                  there are projects from small cores up to PCBs, too.
%%
%%  File:           StdCellLib/Documents/LaTeX/LATN_circuit.tex
%%
%%  Purpose:        Circuit File for LATN
%%
%%  ************    LaTeX with circdia.sty package      ***************
%%
%%  ///////////////////////////////////////////////////////////////////
%%
%%  Copyright (c) 2019 by chipforge <stdcelllib@nospam.chipforge.org>
%%  All rights reserved.
%%
%%      This Standard Cell Library is licensed under the Libre Silicon
%%      public license; you can redistribute it and/or modify it under
%%      the terms of the Libre Silicon public license as published by
%%      the Libre Silicon alliance, either version 1 of the License, or
%%      (at your option) any later version.
%%
%%      This design is distributed in the hope that it will be useful,
%%      but WITHOUT ANY WARRANTY; without even the implied warranty of
%%      MERCHANTABILITY or FITNESS FOR A PARTICULAR PURPOSE.
%%      See the Libre Silicon Public License for more details.
%%
%%  ///////////////////////////////////////////////////////////////////
\begin{center}
    Circuit
    \begin{figure}[h]
        \begin{center}
            \begin{circuitdiagram}{12}{8}
            \usgate
            \flipflop[\clockin{n}]{d}{6}{4}{R}{}{}
            \pin{1}{4}{L}{XN}  % pin XN
            \pin{1}{6}{L}{D}   % pin D
            \pin{11}{6}{R}{Q}  % pin Q
            \end{circuitdiagram}
        \end{center}
    \end{figure}
\end{center}

%\input{LATN_schematic.tex}

\paragraph{Truth Table}
%\input{LATP_truthtable.tex}

\paragraph{Usage}

\paragraph{Fan-in / Fan-out}

\paragraph{Layout}

\paragraph{Files}
